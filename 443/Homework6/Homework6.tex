\documentclass[11pt]{article}
\usepackage{homework}

\usepackage{slashed}

\classname{443}
\homeworknum{6}


\begin{document}

% Environments

\newcommand{\state}[2]{\begin{statement}{#1} #2 \end{statement}}
\newcommand{\prob}[2]{\begin{problem}{#1} #2 \end{problem}}
\newcommand{\subprob}[1]{\begin{subproblem} #1 \end{subproblem}}
\newcommand{\sol}[1]{\begin{solution} #1 \end{solution}}
\newcommand{\fig}[2]{\begin{figure} \centering #2  \label{#1} \end{figure}}

\newcommand{\makebib}{
	\vfill
	\color{black}
	\bibliography{references}{}
	\bibliographystyle{lucas_unsrt}
}
	

% Implication

\newcommand{\qwhere}{\quad \text{where} \quad}
\newcommand{\qimplies}{\quad \implies \quad}
\newcommand{\impliesq}{\implies \quad}



% Brackets

\newcommand{\paren}[1]{\left( #1 \right)}
\newcommand{\brac}[1]{\left[ #1 \right]}


% Greek

\newcommand{\alp}{\alpha}
\newcommand{\bet}{\beta}
\newcommand{\gam}{\gamma}
\newcommand{\del}{\delta}
\newcommand{\eps}{\epsilon}
\newcommand{\zet}{\zeta}
\newcommand{\tht}{\theta}
\newcommand{\kap}{\kappa}
\newcommand{\lam}{\lambda}
\newcommand{\sig}{\sigma}
\newcommand{\ups}{\upsilon}
\newcommand{\omg}{\omega}

\newcommand{\Gam}{\Gamma}
\newcommand{\Del}{\Delta}
\newcommand{\Tht}{\Theta}
\newcommand{\Lam}{\Lambda}
\newcommand{\Sig}{\Sigma}
\newcommand{\Omg}{\Omega}
% Problem 1

\newcommand{\Psii}{\Psi^i}
\newcommand{\Psiix}{\Psii(x)}

\newcommand{\Pii}{\Pi^i}

\newcommand{\Phii}{\Phi^i}
\newcommand{\Phiix}{\Phii(x)}
\newcommand{\PhiN}{\Phi^N}
\newcommand{\PhiNx}{\PhiN(x)}
\newcommand{\Phiq}{\Phi^1}
\newcommand{\Phiw}{\Phi^2}

\newcommand{\ddcx}{\dd[3]{x}}

\newcommand{\delij}{\del^{i j}}
\newcommand{\delkl}{\del^{k l}}
\newcommand{\delil}{\del^{i l}}
\newcommand{\deljk}{\del^{j k}}
\newcommand{\delik}{\del^{i k}}
\newcommand{\deljl}{\del^{j l}}

\newcommand{\DF}{D_F}

\newcommand{\sigx}{\sig(x)}

\newcommand{\pii}{\pi^i}
\newcommand{\pij}{\pi^j}
\newcommand{\pik}{\pi^k}
\newcommand{\pil}{\pi^l}
\newcommand{\piix}{\pi(x)}

\newcommand{\pq}{p_1}
\newcommand{\pw}{p_2}
\newcommand{\pe}{p_3}
\newcommand{\pr}{p_4}

\newcommand{\vp}{\vb{p}}
\newcommand{\vpsi}{\vp_i}

\newcommand{\mpi}{m_\pi}


\state{Exotic contributions to $\mathbf{g - 2}$~(Peskin \& Schroeder~6.3)}{
	Any particles that couples to the electron can produce a correction to the electron-photon form factors and, in particular, a correction to $g - 2$.  Because the electron $g - 2$ agrees with QED to high accuracy, these corrections allow us to constrain the properties of hypothetical new particles.
}

\prob{
	The unified theory of weak and electromagnetic interactions contains a scalar particle $h$ called the \emph{Higgs boson}, which couples to the electron according to
	\eq{
		\Hint = \int \ddcx \frac{\lam}{\sqrt{2}} h \psib \psi.
	}
	Compute the contribution of a virtual Higgs boson to the electron $(g - 2)$, in terms of $\lam$ and the mass $\mh$ of the Higgs boson.
}

\sol{
	The Higgs field is a scalar Yukawa field, so we can use the form of the Yukawa interaction Hamiltonian of Peskin \& Schroeder~(4.112) and the appropriate Feynman rules to write~\cite[p.~118]{Peskin}
	\al{
		\text{(vertex)} &= -i \frac{\lam}{\sqrt{2}}, &
		\text{(propagator)} &= \frac{i}{q^2 - \mh^2 + i \eps}.
	}
	We are interested in the diagram (\hl{draw it})
	
	This is similar to the one on p.~189 in Peskin \& Schroeder.  We can then adapt (6.38) to write
	\aln{
		\ubpp \del\Gammppp \up &= \int \ddqkf \frac{i}{(k - p)^2 - \mh^2 + i \eps} \ubpp \paren{ -i \frac{\lam}{\sqrt{2}} } \frac{i (\ksl' + \me)}{{k'}^2 - \me^2 + i \eps} \gamm \frac{i (\ksl + \me)}{k^2 - \me^2 + i \eps} \paren{ -i \frac{\lam}{\sqrt{2}} } \up \notag \\
		&= i \frac{\lam^2}{2} \int \ddqkf \ubpp \frac{(\ksl' + \me) \gamm (\ksl + \me)}{[ (k - p)^2 - \mh^2 + i \eps ] ({k'}^2 - \me^2 + i \eps) (k^2 - \me^2 + i \eps)} \up. \label{thing1a}
	}
	To evaluate the integral, we use Peskin \& Schroeder~(6.41) with $n = 3$:
	\eq{
		\frac{1}{\Aq \Aw \cdots \An} = \intoq \ddxq \cdots \ddxn \del\paren{ \sum \xii - 1 } \frac{(n - 1)!}{(\xq \Aq + \xw \Aw + \cdots + \xn \An)^n}.
	}
	Applying this to the denominator of the integrand of Eq.~\refeq{thing1a} gives us
	\eqn{denom}{
		\frac{1}{[ (k - p)^2 - \mh^2 + i \eps ] ({k'}^2 - \me^2 + i \eps) (k^2 - \me^2 + i \eps)} = \intoq \ddx \ddy \ddz \del(x + y + z - 1) \frac{2}{D^3},
	}
	where~\cite[pp.~190--191]{Peskin}
	\al{
		D &= x (k^2 - \me^2) + y ({k'}^2 - \me^2) + z [ (k - p)^2 - \mh^2 ] + (x + y + z) i \eps \\
		&= x (k^2 - \me^2) + y (k^2 + 2 k q + q^2 - \me^2) + z (k^2 - 2 k p + p^2 - \mh^2) + i \eps \\
		&= (x + y + z) k^2 - (x + y) \me^2 + 2 k (q y - p z) + z (p^2 - \mh^2) + i \eps \\
		&= k^2 + 2 k (q y - p z) + z (p^2 - \mh^2) - (1 - z) \me^2 + i \eps.
	}
	Here we have used $x + y + z = 1$ and $k' = k + q$.  Let $\ell \equiv k + y q - z p$~\cite[p.~191]{Peskin}.  Then
	\eqn{D}{
		D = \ell^2 + x y q^2 - (1 - z)^2 \me^2 - \mh^2 z + i \eps
		\equiv \ell^2 - \Del + i \eps,
	}
	where we have defined $\Del \equiv -x y q^2 + (1 - z)^2 \me^2 + z \mh^2$~\cite[p.~191]{Peskin}.
	
	For the numerator of Eq.~\refeq{thing1a}, let $N \equiv \ubpp (\ksl' + \me) \gamm (\ksl + \me) \up$.  Then using $k' = k + q$ and $\ell \equiv k + y q - z p$~\cite[p.~191]{Peskin},
	\eqn{N1}{
		N = \ubpp (\ksl + \qsl + \me) \gamm (\ksl + \me) \up = \ubpp [ \lsl + (1 - y) \qsl + z \psl + \me ] \gamm (\lsl - y \qsl + z \psl + \me) \up.
	}
	We should be able to write this as an expression of the form given in (6.31) of Peskin \& Schroeder~\cite[p.~191]{Peskin},
	\eqn{ABC}{
		\Gamm = \gamm \cdot A + ({\pmm}' + \pmm) \cdot B + ({\pmm}' - \pmm) \cdot C
		= \gamm \cdot A + ({\pmm}' + \pmm) \cdot B + \qm \cdot C.
	}
	But from (6.45),
	\eq{
		\int \ddqlf \frac{\lm}{D^3} = 0.
	}
	This means we can discard terms of $\order{\ell}$.  We also know from (6.33) that
	\eqn{Gam}{
		\Gammppp = \gamm \Fqqs + \frac{i \sigmn \qsn}{2 \me} \Fwqs.
	}
	Since the correction to $g - 2$ is given by $\Fwqso = 0 + \del\Fwqso$ (since $\Fw = 0$ to lowest order), we can discard terms of $\order{\gamm}$ in Eq.~\refeq{N1}~\cite[pp.~186, 196]{Peskin}.  So Eq.~\refeq{N1} becomes
	\aln{
		N &= \ubpp [ \lsl + (1 - y) \qsl + z \psl + \me ] \gamm (\lsl - y \qsl + z \psl + \me) \up \notag \\[1ex]
		&= \ubpp [ \lsl \gamm \lsl - y (1 - y) \qsl \gamm \qsl + z (1 - y) \qsl \gamm \psl + \me (1 - y) \qsl \gamm - y z \psl \gamm \qsl \notag \\
		&\hspace{20em} \phantom{=\ } + z^2 \psl \gamm \psl + \me z \psl \gamm - \me y \gamm \qsl + \me z \gamm \psl ] \up. \label{N2}
	}
	To simplify these terms, we use Peskin \& Schroeder~(6.46):
	\eq{
		\int \ddqlf \frac{\lmm \lnn}{D^3} = \int \ddqlf \frac{\gmn \ell^2}{4 D^3},
	}
	as well as~\cite[pp.~191--192]{Peskin}
	\al{
		\psl \gamm &= 2 \pmm - \gamm \psl, &
		\psl \up &= \me \up, &
		\ubpp \psl' &= \ubpp \me
	}
	and~\cite{Gamma, Feynman}
	\al{
		\asl \bsl &= a \cdot b, &
		\asl \bsl + \bsl \asl = 2 a \cdot b.
	}
	We find
	\al{
		\lsl \gamm \lsl &= (2 \lm - \gamm \lsl) \lsl
		= 2 \lm \lnn \gamsn - \gamm \lsl \lsl
		\to \frac{\ell^2 \gmn \gamsn}{2} - \gamm \ell^2
		= -\frac{\ell^2 \gamm}{2} \\
		&\to 0,
		\\[1ex]
		\qsl \gamm \qsl &= (2 \qm - \gamm \qsl) \qsl
		\to - \gamm \qsl \qsl
		= -q^2 \gamm \\
		&\to 0,
		\\[1ex]
		\qsl \gamm \psl &\to \qsl \gamm \me
		= (\psl' - \psl) \gamm \me
		\to (\me - \psl) \gamm \me
		= \me^2 \gamm - 2 \me \pmm + \me \gamm \psl \\
		&\to -2 \me \pmm,
		\\[1ex]
		\qsl \gamm &= (\psl' - \psl) \gamm
		\to \me \gamm - \psl \gamm
		\to -2 \pmm + \gamm \psl
		\to -2 \pmm + \gamm \me \\
		&\to -2 \pmm,
		\\[1ex]
		\psl \gamm \qsl &= (2 \pmm - \gamm \psl) \qsl
		\to -\gamm \psl \qsl
		= -2 \gamm p \cdot q + \gamm \qsl \psl
		\to -2 \gamm p \cdot q + \gamm (\psl' - \psl) \me \\
		&\to \gamm q^2 + \me \gamm \psl' - \me^2 \gamm
		= \gamm q^2 + \me (2 {\pmm}' - \psl' \gamm) - \me^2 \gamm
		\to \gamm q^2 + 2 \me {\pmm}' - 2 \me^2 \gamm \\
		&\to 2 \me {\pmm}',
	}
		where we have used
	\eq{
		2 p \cdot q = p \cdot q + q \cdot p
		= p \cdot q + p' \cdot q - q^2
		= {p'}^2 + p' \cdot p - p' \cdot p - p^2 - q^2
		\to \me^2 - \me^2 - q^2
		= -q^2,
	}
	and
	\al{
		\psl \gamm \psl &\to \me \psl \gamm
		= \me (2 \pmm - \gamm \psl)
		\to 2 \me \pmm - \me^2 \gamm
		\to 2 \me \pmm,
		\\[1ex]
		\psl \gamm &= 2 \pmm - \gamm \psl
		= 2 \pmm - \gamm \me
		\to 2 \pmm,
		\\[1ex]
		\gamm \qsl &= \gamm (\psl' - \psl)
		\to \gamm \psl' - \gamm \me
		\to 2 {\pmm}' - \psl' \gamm
		\to 2 {\pmm}' - \me \gamm
		\to 2 {\pmm}',
		\\[1ex]
		\gamm \psl &\to 0.
	}
	Feeding these into Eq.~\refeq{N2}, we obtain
	\aln{
		N &= \ubpp [ - 2 \me z (1 - y) \pmm - 2 \me (1 - y) \pmm - 2 \me y z {\pmm}' + 2 \me z^2 \pmm + 2 \me z \pmm - 2 \me y {\pmm}' ] \up \notag \\
		&= 2 \me \ubpp \{ [ z^2 + z - z (1 - y) - (1 - y) ] \pmm - y (1 + z) {\pmm}' \} \up \notag \\
		&= 2 \me \ubpp \{ [ z^2 + y (1 + z) - 1 ] \pmm - y (1 + z) {\pmm}' \} \up \notag \\
		&= 2 \me \ubpp [ (z^2 - 1) \pmm + y (1 + z) (\pmm - {\pmm}') ] \up \notag \\
		&= \me \ubpp [ (z^2 - 1) \pmm + (z^2 - 1) \pmm + 2 y (1 + z) (\pmm - {\pmm}') + (z^2 - 1) {\pmm}' - (z^2 - 1) {\pmm}' ] \up \notag \\
		&= \me \ubpp [ (z^2 - 1) (\pmm + {\pmm}') + (z^2 - 1) (\pmm - {\pmm}') + 2 y (1 + z) (\pmm - {\pmm}') ] \up \notag \\
		&= \me \ubpp [ (z^2 - 1) (\pmm + {\pmm}') - (z^2 + 2 y (1 + z) - 1) ({\pmm}' - \pmm) ] \up, \label{N3}
	}
	which has the form of the second two terms of Eq.~\refeq{ABC}.  According to the Ward identity, the coefficient of $\qm = {\pmm} - \pmm$ vanishes~\cite[p.~192]{Peskin}.  Further, according to the Gordon identity given by Peskin \& Schroeder~(6.32),
	\eq{
		\ubpp \gamm \up = \ubpp \paren{ \frac{{\pmm}' + \pmm}{2 m} + \frac{i \sigmn \qsn}{2 m} } \up.
	}
	So Eq.~\refeq{N3} becomes
	\eqn{N4}{
		N = i \me \ubpp (1 - z^2) \sigmn \gmn \up.
	}
	Feeding Eqs.~\refeq{denom}, \refeq{D}, and \refeq{N4} into Eq.~\refeq{thing1a}, we have (ignoring the $\order{\gamm}$ term)
	\eq{
		\ubpp \del\Gammppp \up \to i \frac{\lam^2}{2} \int \ddqlf \intoq \ddx \ddy \ddz \del(x + y + z - 1) \ubpp \frac{i \me (1 - z^2) \sigmn \gmn}{(\ell^2 - \Del + i \eps)^3} \up.
	}
	From Eq.~\refeq{Gam}, we can write
	\eq{
		\del\Fwqs = i \frac{\lam^2}{2} \int \ddqlf \intoq \ddx \ddy \ddz \del(x + y + z - 1) \frac{2 \me^2 (1 - z^2)}{(\ell^2 - \Del + i \eps)^3}.
	}
	Computing the integral using Peskin \& Schroeder~(6.49),
	\eq{
		\int \ddqlf \frac{1}{(\ell^2 - \Del)^m} = \frac{i (-1)^m}{(4\pi)^2} \frac{1}{(m - 1) (m - 2)} \frac{1}{\Del^{m - 2}},
	}
	we find
	\eq{
		\del\Fwqs = \frac{\lam^2}{2 (4\pi)^2} \intoq \ddx \ddy \ddz \del(x + y + z - 1) \frac{\me^2 (1 - z^2)}{\Del}
	}
	so, using Eq.~\refeq{D},
	\eq{
		\del\Fwqso = \frac{\lam^2}{2 (4\pi)^2} \intoq \ddx \ddy \ddz \del(x + y + z - 1) \frac{\me^2 (1 - z^2)}{(1 - z)^2 \me^2 + z \mh^2}
	}
	
}



%\prob{
%	QED accounts extremely well for the electron's anomalous magnetic moment.  If $a = (g - 2) / 2$,
%	\eq{
%		\abs{\aexpt - \aQED} < \num{1e-10}.
%	}
%	What limits does this place on $\lam$ and $\mh$?  In the simplest version of the electroweak theory, $\lam = \num{3e-6}$ and $\mh > \SI{60}{\GeV}$.  Show that these values are not excluded.  The coupling of the Higgs boson to the muon is larger by a factor $(\mmu / \me)$: $\lam = 6e-4$.  Thus, although our experimental knowledge of the muon anomalous magnetic moment is not as precise,
%	\eq{
%		\abs{\aexpt - \aQED} < \num{3e-8},
%	}
%	one can still obtain a stronger limit on $\mh$.  Is it strong enough?
%}
%
%
%
%\prob{
%	Some more complex versions of this theory contain a pseudoscalar particle called the \emph{axion}, which couples to the electron according to
%	\eq{
%		\Hint = \int \ddcx \frac{i \lam}{\sqrt{2}} a \psib \gamt \psi.
%	}
%	The axion may be as light as the electron, or lighter, and may couple more strongly than the Higgs boson.  Compute the contribution of a virtual axion to the $g - 2$ of the electron, and work out the excluded values of $\lam$ and $\ma$.
%}


\makebib

\end{document}
