\documentclass[11pt]{article}
\usepackage{homework}

\classname{443}
\homeworknum{6}


\begin{document}

% Environments

\newcommand{\state}[2]{\begin{statement}{#1} #2 \end{statement}}
\newcommand{\prob}[2]{\begin{problem}{#1} #2 \end{problem}}
\newcommand{\subprob}[1]{\begin{subproblem} #1 \end{subproblem}}
\newcommand{\sol}[1]{\begin{solution} #1 \end{solution}}
\newcommand{\fig}[2]{\begin{figure} \centering #2  \label{#1} \end{figure}}

\newcommand{\makebib}{
	\vfill
	\color{black}
	\bibliography{references}{}
	\bibliographystyle{lucas_unsrt}
}
	

% Implication

\newcommand{\qwhere}{\quad \text{where} \quad}
\newcommand{\qimplies}{\quad \implies \quad}
\newcommand{\impliesq}{\implies \quad}



% Brackets

\newcommand{\paren}[1]{\left( #1 \right)}
\newcommand{\brac}[1]{\left[ #1 \right]}


% Greek

\newcommand{\alp}{\alpha}
\newcommand{\bet}{\beta}
\newcommand{\gam}{\gamma}
\newcommand{\del}{\delta}
\newcommand{\eps}{\epsilon}
\newcommand{\zet}{\zeta}
\newcommand{\tht}{\theta}
\newcommand{\kap}{\kappa}
\newcommand{\lam}{\lambda}
\newcommand{\sig}{\sigma}
\newcommand{\ups}{\upsilon}
\newcommand{\omg}{\omega}

\newcommand{\Gam}{\Gamma}
\newcommand{\Del}{\Delta}
\newcommand{\Tht}{\Theta}
\newcommand{\Lam}{\Lambda}
\newcommand{\Sig}{\Sigma}
\newcommand{\Omg}{\Omega}
% Problem 1

\newcommand{\Psii}{\Psi^i}
\newcommand{\Psiix}{\Psii(x)}

\newcommand{\Pii}{\Pi^i}

\newcommand{\Phii}{\Phi^i}
\newcommand{\Phiix}{\Phii(x)}
\newcommand{\PhiN}{\Phi^N}
\newcommand{\PhiNx}{\PhiN(x)}
\newcommand{\Phiq}{\Phi^1}
\newcommand{\Phiw}{\Phi^2}

\newcommand{\ddcx}{\dd[3]{x}}

\newcommand{\delij}{\del^{i j}}
\newcommand{\delkl}{\del^{k l}}
\newcommand{\delil}{\del^{i l}}
\newcommand{\deljk}{\del^{j k}}
\newcommand{\delik}{\del^{i k}}
\newcommand{\deljl}{\del^{j l}}

\newcommand{\DF}{D_F}

\newcommand{\sigx}{\sig(x)}

\newcommand{\pii}{\pi^i}
\newcommand{\pij}{\pi^j}
\newcommand{\pik}{\pi^k}
\newcommand{\pil}{\pi^l}
\newcommand{\piix}{\pi(x)}

\newcommand{\pq}{p_1}
\newcommand{\pw}{p_2}
\newcommand{\pe}{p_3}
\newcommand{\pr}{p_4}

\newcommand{\vp}{\vb{p}}
\newcommand{\vpsi}{\vp_i}

\newcommand{\mpi}{m_\pi}


\state{Exotic contributions to $\mathbf{g - 2}$~(Peskin \& Schroeder~6.3)}{
	Any particles that couples to the electron can produce a correction to the electron-photon form factors and, in particular, a correction to $g - 2$.  Because the electron $g - 2$ agrees with QED to high accuracy, these corrections allow us to constrain the properties of hypothetical new particles.
}

\prob{
	The unifies theory of weak and electromagnetic interactions contains a scalar particle $h$ called the \emph{Higgs boson}, which couples to the electron according to
	\eq{
		\Hint = \int \ddcx \frac{\lam}{\sqrt{2}} h \psib \psi.
	}
	Compute the contribution of a virtual Higgs boson to the electron $(g - 2)$, in terms of $\lam$ and the mass $\mh$ of the Higgs boson.
}



\prob{
	QED accounts extremely well for the electron's anomalous magnetic moment.  If $a = (g - 2) / 2$,
	\eq{
		\abs{\aexpt - \aQED} < \num{1e-10}.
	}
	What limits does this place on $\lam$ and $\mh$?  In the simplest version of the electroweak theory, $\lam = \num{3e-6}$ and $\mh > \SI{60}{\GeV}$.  Show that these values are not excluded.  The coupling of the Higgs boson to the muon is larger by a factor $(\mmu / \me)$: $\lam = 6e-4$.  Thus, although our experimental knowledge of the muon anomalous magnetic moment is not as precise,
	\eq{
		\abs{\aexpt - \aQED} < \num{3e-8},
	}
	one can still obtain a stronger limit on $\mh$.  Is it strong enough?
}



\prob{
	Some more complex versions of this theory contain a pseudoscalar particle called the \emph{axion}, which couples to the electron according to
	\eq{
		\Hint = \int \ddcx \frac{i \lam}{\sqrt{2}} a \psib \gamt \psi.
	}
	The axion may be as light as the electron, or lighter, and may couple more strongly than the Higgs boson.  Compute the contribution of a virtual axion to the $g - 2$ of the electron, and work out the excluded values of $\lam$ and $\ma$.
}


%\makebib

\end{document}
