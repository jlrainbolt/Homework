\documentclass[11pt]{article}
\usepackage{homework}
\usepackage{slashed}

\classname{443}
\homeworknum{5}



\begin{document}

% Environments

\newcommand{\state}[2]{\begin{statement}{#1} #2 \end{statement}}
\newcommand{\prob}[2]{\begin{problem}{#1} #2 \end{problem}}
\newcommand{\subprob}[1]{\begin{subproblem} #1 \end{subproblem}}
\newcommand{\sol}[1]{\begin{solution} #1 \end{solution}}
\newcommand{\fig}[2]{\begin{figure} \centering #2  \label{#1} \end{figure}}

\newcommand{\makebib}{
	\vfill
	\color{black}
	\nocite{*}
	\bibliography{references}{}
	\bibliographystyle{lucas_unsrt}
}
	

% Implication

\newcommand{\qwhere}{\quad \text{where} \quad}
\newcommand{\qimplies}{\quad \implies \quad}
\newcommand{\impliesq}{\implies \quad}



% Brackets

\newcommand{\paren}[1]{\left( #1 \right)}
\newcommand{\brac}[1]{\left[ #1 \right]}
\newcommand{\curly}[1]{\left\{ #1 \right\}}


% Greek

\newcommand{\alp}{\alpha}
\newcommand{\bet}{\beta}
\newcommand{\gam}{\gamma}
\newcommand{\del}{\delta}
\newcommand{\eps}{\epsilon}
\newcommand{\zet}{\zeta}
\newcommand{\tht}{\theta}
\newcommand{\kap}{\kappa}
\newcommand{\lam}{\lambda}
\newcommand{\sig}{\sigma}
\newcommand{\ups}{\upsilon}
\newcommand{\omg}{\omega}

\newcommand{\Gam}{\Gamma}
\newcommand{\Del}{\Delta}
\newcommand{\Tht}{\Theta}
\newcommand{\Lam}{\Lambda}
\newcommand{\Sig}{\Sigma}
\newcommand{\Omg}{\Omega}


% Text

\newcommand{\where}{\text{where }}

% Problem 1

\newcommand{\Hint}{H_\text{int}}
\newcommand{\ddcx}{\dd[3]{x}}
\newcommand{\psib}{\bar{\psi}}

\newcommand{\mh}{m_h}
\newcommand{\mmu}{m_\mu}
\newcommand{\me}{m_e}
\newcommand{\ma}{m_a}

\newcommand{\aexpt}{a_\text{expt.}}
\newcommand{\aQED}{a_\text{QED}}
\renewcommand{\GeV}{\giga\electronvolt}

\newcommand{\gamt}{\gam^5}


\state{Coulomb scattering (Peskin \& Schroeder 5.1)}{
	Repeat the computation of Problem 4.4, part~(c), this time using the full relativistic expression for the matrix element.  You should find, for the spin-averaged cross section,
	\eq{
		\dv{\sig}{\Omg} = \frac{\alp^2}{4 \abs{\vp}^2 \bet^2 \sin[4](\tht / 2)} \brac{ 1 - \bet^2 \sin[2](\frac{\tht}{2}) },
	}
	where $\vp$ is the electron's 3-momentum and $\bet$ its velocity.  This is the \emph{Mott formula} for Coulomb scattering of relativistic electrons.  Now derive it in a second way, by working out the cross section for electron-muon scattering, in the muon rest frame, retaining the electron mass but sending $\mmu \to \infty$.
}

\sol{
	From the solution to 2(b) of Homework~4, we know that
	\eqn{xsec1}{
		\dv{\sig}{\Omg} = \frac{\abs{\cM}^2}{16 \pi^2}.
	}
	We also know from 2(a) that
	\eq{
		i \cM = -i e \ubpp \gamm \up \cdot \tAsm(\vp' - \vp),
	}
	where the argument of $\tA$ includes only the spatial components of $p$ and $p'$ because $\Asm(x)$ is time independent~\cite[p.~129]{Peskin}.  Then
	\eq{
		-i \cMs = i e \ubp \gamm \upp \cdot \tAsm(\vp' - \vp),
	}
	so
	\eq{
		\abs{\cM}^2 = e^2 \ubpp \gamm \up \ubp \gamn \upp \tAsm(\vp' - \vp) \tAsn(\vp' - \vp).
	}
	Averaging over the spins, we want to compute~\cite[p.~132]{Peskin}
	\eq{
		\frac{1}{2} \sumssp \abs{\cM(s, s')}^2.
	}
	By (5.3) of Peskin \& Schroeder,
	\eq{
		\sums \usp \ubsp = \psl + m.
	}
	Then, writing in the spinor indices as in the equation above (5.4),
	\al{
		\frac{1}{2} \sumssp \abs{\cM}^2 &= \frac{e^2}{2} \sumssp \ubsasppp \gammsab \usbsp \ubscsp \gamnscd \usdsppp \tAsm(p' - p) \tAsn(p' - p) \\
		&= \frac{e^2}{2} \sumssp \usdsppp \ubsasppp \gammsab \usbsp \ubscsp \gamnscd \tAsm(p' - p) \tAsn(p' - p) \\
		&=\frac{e^2}{2} (\psl' + m)_{d a} \gammsab (\psl + m)_{b c} \gamnscd \tAsm(\vp' - p) \tAsn(\vp' - \vp) \\
		&= \frac{e^2}{2} \Tr[ (\psl' + m) \gamm (\psl + m) \gamn ] \tAsm(\vp' - \vp) \tAsn(\vp' - \vp).
	}
	Note that
	\aln{
		\Tr[ (\psl' + m) \gamm (\psl + m) \gamn ] &= \Tr(\psl' \gamm \psl \gamn) + \Tr(\psl' \gamm m \gamn) + \Tr(m \gamm \psl \gamn) + \Tr(m \gamm m \gamn) \notag \\
		&= \Tr(\psl' \gamm \psl \gamn) + \Tr(m \gamm m \gamn), \label{beg}
	}
	since the other two terms have an odd number of $\gam$s~\cite[p.~133]{Peskin}.  Then by Peskin \& Schroeder~(5.5),
	\eqn{}{
		\Tr(m \gamm m \gamn) = m^2 \Tr(\gamm \gamn) = 4 m^2 \gmn.
	}
	Applying $\psl = \gamm \psm$~\cite[p.~49]{Peskin},
	\eqn{end}{
		\Tr(\psl' \gamm \psl \gamn) = \psr' \pss \Tr(\gamr \gamm \gams \gamn)
		= 4 \psr' \pss (\grm \gsn - \grs \gmn + \grn \gms).
	}
	Then
	\al{
		\frac{1}{2} \sumssp \abs{\cM}^2 &= 2 e^2 \brac{ \psr' \pss (\grm \gsn - \grs \gmn + \grn \gms) + m^2 \gmn } \tAsm(\vp' - \vp) \tAsn(\vp' - \vp) \\
		&= 2 e^2 \brac{ \psr' \tAr \pss \tAs - \psr' \pr \tAsm \tAn + \psr' \tAn \pss \tAm + m^2 \tAsm \tAn } \\
		&= 2 e^2 \curly{ 2 (p' \cdot \tA) (p \cdot \tA) + [m^2 - (p' \cdot p)] (\tA \cdot \tA) },
	}
	where we  have omitted the arguments of $\tA$ for notational simplicity.  We know that for Coulomb scattering $\Asm$ has only a zeroth component, $\Ao  = Z e / 4\pi r$~\cite[p.~130]{Peskin}.  From the solution to 2(c) of Homework~4, we also know that $\tAso(\vq) = Z e / \abs{\vq}^2$.  Then
	\al{
		\frac{1}{2} \sumssp \abs{\cM}^2 &= 2 e^2 \brac{ 2 E E' \tAso^2 + (m^2 - E E' + \vp \cdot \vp') \tAso^2 } \\
		&= 2 e^2 \tAso^2 (2 E E' + m^2 - E' E + \vp \cdot \vp') \\
		&= 2 \frac{Z^2 e^4}{\abs{\vp' - \vp}^4} (E E' + m^2 + \vp \cdot \vp').
	}
	From 2(c) of Homework~4, we know
	\eq{
		\abs{\vp' - \vp}^2 = {\vp'}^2 - 2 \abs{\vp'} \abs{\vp} \cos\tht + \vp^2,
	}
	where $\tht$ is the angle between $\vp$ and $\vp'$.  Using this and taking the limit that $p = p'$,
	\eq{
		\frac{1}{2} \sumssp \abs{\cM}^2 = 2 Z^2 e^4 \frac{E E' + m^2 + \abs{\vp} \abs{\vp'} \cos\tht}{({\vp'}^2 - 2 \abs{\vp'} \abs{\vp} \cos\tht + \vp^2)^2}
		= 2 Z^2 e^4 \frac{E^2 + m^2 + \vp^2 \cos\tht}{4 \vp^4 (1 - \cos\tht)^2}
		= \frac{Z^2 e^4}{8} \frac{E^2 + m^2 + \vp^2 \cos\tht}{\vp^4 \sin[4](\tht / 2)}.
	}
	Noting that $E^2 = m^2 + p^2$,
	\eq{
		\frac{1}{2} \sumssp \abs{\cM}^2 = \frac{Z^2 e^4}{8} \frac{E^2 + E^2 - \vp^2 + \vp^2 \cos\tht}{\vp^4 \sin[4](\tht / 2)}
		= \frac{Z^2 e^4}{8} \frac{2 E^2 - \vp^2 (1 - \cos\tht)}{\vp^4 \sin[4](\tht / 2)}
		= \frac{Z^2 e^4}{4} \frac{E^2 - \vp^2 \sin[2](\tht / 2)}{\vp^4 \sin[4](\tht / 2)}.
	}
	Now using $\vp^2 = m^2 \bet^2$, $\bet = \abs{\vp} / E$, $E / m = 1$, and $\alp = e^2 / 4\pi$~\cite[p.~xxi]{Peskin},
	\eq{
		\frac{1}{2} \sumssp \abs{\cM}^2 = \frac{Z^2 e^4 E^2}{4} \frac{1 - \bet^2 \sin[2](\tht / 2)}{m^2 \bet^2 \vp^2 \sin[4](\tht / 2)}
		= \frac{Z^2 e^4 E^2}{4} \frac{1 - \bet^2 \sin[2](\tht / 2)}{m^2 \bet^2 \vp^2 \sin[4](\tht / 2)}
		= \frac{4 \pi^2 \alp^2 Z^2}{\bet^2 \vp^2 \sin[4](\tht / 2)} \brac{ 1 - \bet^2 \sin[2](\frac{\tht}{2}) }
	}
	Feeding this into Eq.~\refeq{xsec1} and setting $Z = 1$, we find
	\eq{
		\dv{\sig}{\Omg} = \frac{1}{16 \pi^2} \frac{1}{2} \sumssp \abs{\cM}^2
		= \ans{ \frac{\alp^2}{4 \bet^2 \vp^2 \sin[4](\tht / 2)} \brac{ 1 - \bet^2 \sin[2](\frac{\tht}{2}) } }
	}
	as we wanted to show. \qed
	
	The lowest-order Feynman diagram for electron-muon scattering is~\cite[p.~153]{Peskin} \hl{DRAW}
	\eq{
		= \frac{i e^2}{q^2} \ubpp \gamm \up \ubkp \gamsm \ubk.
	}
	The corresponding squared amplitude is given the equation above Peskin \& Schroeder (5.60):
	\eq{
		\frac{1}{4} \sumspins \abs{\cM}^2 = \frac{e^4}{4 q^4} \Tr[ (\psl' + \me) \gamm (\psl + \me) \gamn ] \Tr[ (\ksl' + \mmu) \gamsm (\ksl + \mmu) \gamsn ].
%		&= \frac{8 e^4}{q^4} \brac{ (p \cdot k') (p' \cdot k) + (p \cdot k) (p' \cdot k') - \mmu^2 (p \cdot p') }
	}
	Referring to Eqs.~\refeq{beg}--\refeq{end},
	\al{
		\frac{1}{4} \sumspins \abs{\cM}^2 &= 4 \frac{e^4}{q^4} \brac{ \psr' \pss (\grm \gsn - \grs \gmn + \grn \gms) + \me^2 \gmn } \brac{ \kpa \kb (\gsam \gsbn - \gsab \gsmn + \gsan \gsmb) + \mmu^2 \gsmn } \\
		&\approx 4 \frac{e^4}{q^4} \brac{ \psr' \pss (\grm \gsn - \grs \gmn + \grn \gms) + \me^2 \gmn } \mmu^2 \gsmn \\
		&= 4 \frac{e^4}{q^4} \paren{ \ppm \pn - \pps \pss \gmn + \ppn \pmm + \me^2 \gmn } \mmu^2 \gsmn \\
		&= 4 \frac{e^4 \mmu^2}{q^4} \paren{ p' \cdot p - p' \cdot p + p' \cdot p + \me^2 } \\
		&= 4 \frac{e^4 \mmu^2}{q^4} \paren{ p' \cdot p + \me^2 },
	}
	where we have only retained terms of $\order{\mmu^2}$.  From Peskin \& Schroeder~(5.69) and (5.72),
	\eq{
		q = t^2 = -2 \abs{\vp}^2 (1 - \cos\tht).
	}
	Then, again taking $p' = p$,
	\al{
		\frac{1}{4} \sumspins \abs{\cM}^2 &= 4 \frac{e^4 \mmu^2}{[ -2 \abs{\vp}^2 (1 - \cos\tht) ]^2} \paren{ E' E - \vp' \vdot \vp + \me^2 } \\
		&= e^2 \mmu^2 \frac{E^2 - \abs{\vp}^2 \cos\tht + \me^2}{\abs{\vp}^4 (1 - \cos\tht)^2} \\
		&= e^2 \mmu^2 \frac{2 E^2 - p^2 - 2 \abs{\vp}^2 \sin[2](\tht / 2)}{\abs{\vp}^4 (1 - \cos\tht)^2}
	}
	
	
	% We choose a reference frame such that
%	\al{
%		p &= (p, p \zh), &
%		p' &= (p, \vp), &
%		k &= (E, -p \zh), &
%		k' &= (E, -p \zh),
%	}
%	which implies~\cite[p.~154]{Peskin}
%	\al{
%		p' \cdot p &= p^2 (1 - \cos\tht), &
%		q^2 &= -2 p' \cdot p
%		= -2 p^2 (1 - \cos\tht).
%	}
%	Then, again using $E^2 = \me^2 + p^2$,
%	\al{
%		\frac{1}{4} \sumspins \abs{\cM}^2 &= 4 \frac{e^4 \mmu^2}{[ -2 p^2 (1 - \cos\tht) ]^2} \paren{ p^2 (1 - \cos\tht) + \me^2 } \\
%%		&= e^4 \mmu^2 \frac{2 p^2 \sin[2](\tht / 2) + E^2 - 2 p^2 + p^2}{p^4 (1 - \cos\tht)^2} \\
%%		&= e^4 \mmu^2 \frac{E^2 + p^2 - 2 p^2 [1 - \sin[2](\tht / 2) ]}{p^4 (1 - \cos\tht)^2}
%	}
%	
	\hl{help}	
}





\clearpage
\state{Bhabha scattering (Peskin \& Schroeder 5.2)}{
	Compute the differential cross section $\dv*{\sig}{(\cos\tht)}$ for Bhabha scattering, $\elp \elm \to \elp \elm$.  You may work in the limit $\Ecm \gg \me$, in which it is permissible to ignore the electron mass.  There are two Feynman diagrams; these must be added in the invariant matrix element before squaring.  Be sure that you have the correct relative sign between these diagrams.  The intermediate steps are complicated, but the final result is quite simple.  In particular, you may find it useful to introduce the Mandelstam variables $s$, $t$, and $u$.  Note that, if we ignore the electron mass, $s + t + u = 0$.  You should be able to cast the differential cross section into the form
	\eq{
		\dv{\sig}{(\cos\tht)} = \frac{\pi \alp^2}{s} \brac{ u^2 \paren{ \frac{1}{s} + \frac{1}{t} }^2 + \paren{ \frac{t}{s} }^2 + \paren{ \frac{s}{t} }^2 }.
	}
	Rewrite this formula in terms of $\cos\tht$ and graph it.  What feature of the diagrams causes the differential cross section to diverge as $\tht \to 0$?
}

\sol{
	The two Feynman diagrams are the $s$- and $t$-channel diagrams~\cite[p.~157]{Peskin}.  \hl{Draw them.}  The contractions for the $s$-channel diagram are
	
	For the $t$ channel we need to swap field order 3 times which gives an overall sign difference~\cite[p.~118]{Peskin}.  \hl{SHOW THIS SHIT}
	
	The matrix element for the $s$ channel is the same as Peskin \& Schroeder~(5.1),
	\eq{
		i \cM = \frac{i e^2}{s} \vbk \gamm \up \ubpp \gamsm \vkp.
	}
	For the $t$ channel, it is~\cite[p.~153]{Peskin}
	\eq{
		i \cM = \frac{i e^2}{t} \ubpp \gamm \up \vbk \gamsm \vkp
	}
	where we have used the Mandelstam variables~\hl[cite].  Their difference is
	\eq{
		i \cM = i e^2 \brac{ \frac{\vbk \gamm \up \ubpp \gamsm \vkp}{s} - \frac{\ubpp \gamm \up \vbk \gamsm \vkp}{t} },
	}
	which means~\cite[p.~132]{Peskin}
	\eq{
		-i \cMs = -i e^2 \brac{ \frac{\ubp \gamm \vk \vbkp \gamsm \upp}{s} - \frac{\ubp \gamm \upp \vbkp \gamsm \vk}{t} }.
	}
	Then
	\al{
		\abs{\cM}^2 &= e^4 \left[ \frac{\vbk \gamm \up \ubpp \gamsm \vkp \ubp \gamn \vk \vbkp \gamsn \upp}{s^2} - \frac{\vbk \gamm \up \ubpp \gamsm \vkp \ubp \gamn \upp \vbkp \gamsn \vk}{s t} \right. \\
		&\hspace{1.5em} \left. \phantom{=\ } - \frac{\ubpp \gamm \up \vbk \gamsm \vkp \ubp \gamn \vk \vbkp \gamsn \upp}{s t} + \frac{\ubpp \gamm \up \vbk \gamsm \vkp \ubp \gamn \upp \vbkp \gamsn \vk}{t^2} \right].
	}
	Note that~\cite[p.~132, 153]{Peskin}
	\al{
		\sumspins \vbk \gamm \up \ubpp \gamsm \vkp \ubp \gamn \vk \vbkp \gamsn \upp &= \Tr( \ksl \gamm \psl \gamn ) \Tr( \psl' \gamsm \ksl' \gamsn ), \\
		\sumspins \ubpp \gamm \up \vbk \gamsm \vkp \ubp \gamn \upp \vbkp \gamsn \vk &= \Tr( \psl' \gamm \psl \gamn ) \Tr( \ksl \gamsm \ksl' \gamsn ),
	}
	where we have taken $m \to 0$.  Writing the other two using components,
	\al{
		\sumspins \brac{ \vbk \gamm \up \ubpp \gamsm \vkp \ubp \gamn \upp \vbkp \gamsn \vk + \cc }
		&= \sumssp \sumrrp \brac{ \vbsar \gammsab \usbs \ubscspp \gamsmcd \vsdrp \ubses \gamnef \usfsp \vsgrp \gamsngh \vshr + \cc } \\
		&= \sumssp \sumrrp \brac{ \vshr \vbsar \gammsab \usbs \ubses \gamnef \usfsp \ubscspp \gamsmcd \vsdrp \vsgrp \gamsngh + \cc} \\
		&= \ksl_{h a} \gammsab \psl_{b e} \gamnef \psl'_{f c} \gamsmcd \ksl'_{d g} \gamsngh + \cc \\
		&= \Tr( \ksl \gamm \psl \gamn \psl' \gamsm \ksl' \gamsn ) + \cc
	}
	where we have used Peskin \& Schroeder~(5.3),
	\al{
		\sums \usp \ubsp &= \psl + m, &
		\sums \vsp \vbsp &= \psl - m,
	}
	and $m = 0$.  This gives us the matrix element
	\eq{
		\abs{\cM}^2 = e^4 \paren{ \frac{\Tr( \ksl \gamm \psl \gamn ) \Tr( \psl' \gamsm \ksl' \gamsn )}{s^2} - \frac{\Tr( \ksl \gamm \psl \gamn \psl' \gamsm \ksl' \gamsn ) + \cc}{s t} + \frac{\Tr( \psl' \gamm \psl \gamn ) \Tr( \ksl \gamsm \ksl' \gamsn )}{t^2} }.
	}
	We need to average over spins: $\sumspins \abs{\cM}^2 / 4$~\cite[p.~132]{Peskin}.  From (5.70) and (5.71),
	\al{
		\frac{1}{4} \sumspins \frac{e^4}{s^2} \Tr( \ksl \gamm \psl \gamn ) \Tr( \psl' \gamsm \ksl' \gamsn ) &= \frac{8 e^4}{s^2} \brac{ \paren{ \frac{t}{2} }^2 + \paren{ \frac{u}{2} }^2 }, \\
		\frac{1}{4} \sumspins \frac{e^4}{t^2} \Tr( \psl' \gamm \psl \gamn ) \Tr( \ksl \gamsm \ksl' \gamsn ) &= \frac{8 e^4}{t^2} \brac{ \paren{ \frac{s}{2} }^2 + \paren{ \frac{u}{2} }^2 }.
	}
	For the remaining terms, we adapt Peskin \& Schroeder~(5.69),
	\al{
		s &= (p + k)^2
		= (p' + k')^2, &
		t &= (p' - p)^2
		= (k' - k)^2, &
		u &= (k' - p)^2
		= (p' - k)^2.
	}
	For the $s$ channel in the massless limit~\cite[p.~156]{Peskin},
	\al{
		t &= -2 p \cdot p'
		= -2 k \cdot k', &
		u&= -2 p \cdot k'
		= -2 k \cdot p'.
	}
	Note that
	\al{
		\Tr( \ksl \gamm \psl \gamn \psl' \gamsm \ksl' \gamsn ) &= \ksa \psb \pspr \ksps \Tr( \gama \gamm \gamb \gamn \gamr \gamsm \gams \gamsn ) \\
		&= -2 \ksa \psb \pspr \ksps \Tr( \gama \gamr \gamn \gamb \gams \gamsn ) \\
		&= -8 \ksa \psb \pspr \ksps \Tr( \gama \gamr \gbs ) \\
		&= -8 \ksa \psb \pspr \kpb \Tr( \gama \gamr ) \\
		&= -32 \ksa \psb \pspr \kpb \gar \\
		&= -32 \ksa \ppa \psb \kpb \\
		&= -32 (k \cdot p') (p \cdot k') \\
		&= -8 u^2,
	}
	where we have used Peskin \& Schroeder~(5.9),
	\al{
		\gamm \gamn \gamr \gamsm &= 4 \gnr, &
		\gamm \gamn \gamr \gams \gamsm &= -2 \gams \gamr \gamn,
	}
	and (5.5), $\Tr(\gamm \gamn) = 4 \gmn$.  Then the matrix element is
	\aln{
		\frac{1}{4} \sumspins \abs{\cM^2} &= \frac{8 e^4}{s^2} \brac{ \paren{ \frac{t}{2} }^2 + \paren{ \frac{u}{2} }^2 } + \frac{4 e^4}{s t} u^2 + \frac{8 e^4}{t^2} \brac{ \paren{ \frac{s}{2} }^2 + \paren{ \frac{u}{2} }^2 } \notag \\
		&= 2 e^4 \brac{ \paren{ \frac{t}{s} }^2 + \frac{u^2}{s^2} + 2 \frac{u^2}{s t} + \paren{ \frac{s}{t} }^2 + \frac{u^2}{t^2} } \notag \\
		&= 2 e^4 \brac{ u^2 \paren{ \frac{1}{s^2} + 2 \frac{1}{s t} + \frac{1}{t^2} } + \paren{ \frac{t}{s} }^2 + \paren{ \frac{s}{t} }^2 } \notag \\
		&= 32 \pi^2 \alp^2 \brac{ u^2 \paren{ \frac{1}{s} + \frac{1}{t} }^2 + \paren{ \frac{t}{s} }^2 + \paren{ \frac{s}{t} }^2 }. \label{ans2a}
	}
	By Peskin \& Schroeder~(4.85), the differential cross section for four particles of the same mass is
	\eq{
		\dv{\sig}{\Omg} = \frac{\abs{\cM}^2}{64 \pi^2 \Ecm^2}.
	}
	In the center-of-mass frame, $\Ecm = (p + k) = \sqrt{s}$.  Feeding in Eq.~\refeq{ans2a},
	\eq{
		\dv{\sig}{\Omg} = \frac{1}{64 \pi^2 \Ecm^2} \frac{1}{4} \sumspins \abs{\cM^2}
		= \frac{\alp^2}{2 s} \brac{ u^2 \paren{ \frac{1}{s} + \frac{1}{t} }^2 + \paren{ \frac{t}{s} }^2 + \paren{ \frac{s}{t} }^2 }.
	}
	Noting that $\ddOmg = \ddcost \ddphi$ and $\int \ddphi = 2\pi$, we integrate over $\phi$ to find
	\eq{
		\ans{ \dv{\sig}{(\cos\tht)} = \frac{\pi \alp^2}{s} \brac{ u^2 \paren{ \frac{1}{s} + \frac{1}{t} }^2 + \paren{ \frac{t}{s} }^2 + \paren{ \frac{s}{t} }^2 } }
	}
	as desired. \qed
	
	\hl{UGH THE REST}
}






\clearpage
\state{Positronium lifetimes (Peskin \& Schroeder 5.4)}{\hfix}

\prob{
	Compute the amplitude $\cM$ for $\elp \elm$ annihilation into 2 photons in the extreme nonrelativistic limit (i.e., keep only the term proportional to zero powers of the electron and positron 3-momentum).  Use this result, together with our formalism for fermion-antifermion bound states, to compute the rate of annihilation of the $1S$ states of positronium into 2 photons.  You should find that the spin-1 states of positronium do not annihilate into 2 photons, confirming the symmetry argument of Problem~3.8.  For the spin-0 state of positronium, you should find a result proportional to the square of the $1S$ wavefunction at the origin.  Inserting the value of this wavefunction from nonrelativistic quantum mechanics, you should find
	\eq{
		\frac{1}{\tau} = \Gam
		= \frac{\alp^5 \me}{2}
		\approx \SI{8.03e9}{\per\second}.
	}
	A recent measurement gives $\Gam = 7.994 \pm \SI{0.011}{\per\nano\second}$; the 0.5\% discrepancy is accounted for by radiative corrections.
}


\makebib

\end{document}
