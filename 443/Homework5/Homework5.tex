\documentclass[11pt]{article}
\usepackage{homework}
\usepackage{slashed}
\usepackage{simpler-wick}

\classname{443}
\homeworknum{5}



\begin{document}

% Environments

\newcommand{\state}[2]{\begin{statement}{#1} #2 \end{statement}}
\newcommand{\prob}[2]{\begin{problem}{#1} #2 \end{problem}}
\newcommand{\subprob}[1]{\begin{subproblem} #1 \end{subproblem}}
\newcommand{\sol}[1]{\begin{solution} #1 \end{solution}}
\newcommand{\fig}[2]{\begin{figure} \centering #2  \label{#1} \end{figure}}

\newcommand{\makebib}{
	\vfill
	\color{black}
	\bibliography{references}{}
	\bibliographystyle{lucas_unsrt}
}
	

% Implication

\newcommand{\qwhere}{\quad \text{where} \quad}
\newcommand{\qimplies}{\quad \implies \quad}
\newcommand{\impliesq}{\implies \quad}



% Brackets

\newcommand{\paren}[1]{\left( #1 \right)}
\newcommand{\brac}[1]{\left[ #1 \right]}


% Greek

\newcommand{\alp}{\alpha}
\newcommand{\bet}{\beta}
\newcommand{\gam}{\gamma}
\newcommand{\del}{\delta}
\newcommand{\eps}{\epsilon}
\newcommand{\zet}{\zeta}
\newcommand{\tht}{\theta}
\newcommand{\kap}{\kappa}
\newcommand{\lam}{\lambda}
\newcommand{\sig}{\sigma}
\newcommand{\ups}{\upsilon}
\newcommand{\omg}{\omega}

\newcommand{\Gam}{\Gamma}
\newcommand{\Del}{\Delta}
\newcommand{\Tht}{\Theta}
\newcommand{\Lam}{\Lambda}
\newcommand{\Sig}{\Sigma}
\newcommand{\Omg}{\Omega}
% Problem 1

\newcommand{\Psii}{\Psi^i}
\newcommand{\Psiix}{\Psii(x)}

\newcommand{\Pii}{\Pi^i}

\newcommand{\Phii}{\Phi^i}
\newcommand{\Phiix}{\Phii(x)}
\newcommand{\PhiN}{\Phi^N}
\newcommand{\PhiNx}{\PhiN(x)}
\newcommand{\Phiq}{\Phi^1}
\newcommand{\Phiw}{\Phi^2}

\newcommand{\ddcx}{\dd[3]{x}}

\newcommand{\delij}{\del^{i j}}
\newcommand{\delkl}{\del^{k l}}
\newcommand{\delil}{\del^{i l}}
\newcommand{\deljk}{\del^{j k}}
\newcommand{\delik}{\del^{i k}}
\newcommand{\deljl}{\del^{j l}}

\newcommand{\DF}{D_F}

\newcommand{\sigx}{\sig(x)}

\newcommand{\pii}{\pi^i}
\newcommand{\pij}{\pi^j}
\newcommand{\pik}{\pi^k}
\newcommand{\pil}{\pi^l}
\newcommand{\piix}{\pi(x)}

\newcommand{\pq}{p_1}
\newcommand{\pw}{p_2}
\newcommand{\pe}{p_3}
\newcommand{\pr}{p_4}

\newcommand{\vp}{\vb{p}}
\newcommand{\vpsi}{\vp_i}

\newcommand{\mpi}{m_\pi}

\state{(Jackson 9.8)}{\ 
	%\emph{Hint:} The electromagnetic angular momentum density comes from more than the transverse (radiation zone) components of the fields.
}

%
%	Jackson 9.8(a)
%

\prob{}{
	Show that a classical oscillating electric dipole $\vp$ with fields given by
	\aln{ \label{fields1}
		\vH &= \frac{c k^2}{4\pi} (\nh \cross \vp) \frac{e^{i k r}}{r} \paren{ 1 - \frac{1}{i k r} }, &
		\vE &= \frac{1}{4\pi \epso} \curly{ k^2 (\nh \cross \vp) \cross \nh \frac{e^{i k r}}{r} + [ 3 \nh (\nh \vdot \vp) - \vp ] \paren{ \frac{1}{r^3} - \frac{i k}{r^2} } e^{i k r} },
	}
	radiates electromagnetic angular momentum to infinity at the rate
	\eq{
		\dv{\vL}{t} = \frac{k^3}{12 \pi \epso} \Im[ \vp^* \cross \vp ].
	}
	\vfix
}

\sol{
	According to Jackson~(9.20), the time-averaged angular momentum density is
	\eq{
		\vl = \frac{\Re[ \vx \cross (\vE \cross \vHs)}{2 c^2}.
	}
	One of the vector identities on the inside cover of Jackson is $\vaa \cross (\vbb \cross \vcc) = (\vaa \vdot \vcc) \vbb - (\vaa \vdot \vbb) \vcc$, so
	\eqn{l1}{
		\vl = \frac{(\vx \vdot \vHs) \vE - (\vx \vdot \vE) \vHs}{2 c^2}.
	}
	From Eq.~\refeq{fields1}, note that
	\eq{
		\vx \vdot \vHs \propto \vx \vdot (\nh \cross \vps)
		= \vps \vdot (\vx \cross \nh)
		= \vO,
	}
	where we have used the identity $\vaa \vdot (\vbb \cross \vcc) = \vcc \vdot (\vaa \cross \vbb)$ and the fact that $\nh$ points in the $\vx$ direction.  For $\vx \vdot \vE$, note that
	\al{
		\vx \vdot [ (\nh \cross \vp) \cross \nh ] &= -\vx \vdot [ \nh \cross (\nh \cross \vp) ]
		= -\vx \vdot [ (\nh \vdot \vp) \nh - (\nh \vdot \nh) \vp ]
		= -(\nh \vdot \vp) (\vx \vdot \nh) + \vx \vdot \vp \\
		&= -r (\nh \vdot \vp) + \vx \vdot \vp
		= \vx \vdot \vp - \vx \vdot \vp
		= 0, \\[1.5ex]
		\vx \vdot [ 3 \nh (\nh \vdot \vp) - \vp ] &= 3 (\vx \vdot \nh) (\nh \vdot \vp) - \vx \vdot \vp
		= 3r (\nh \vdot \vp) - \vx \vdot \vp
		= 3(\vx \vdot \vp) - \vx \vdot \vp
		= 2(\vx \vdot \vp),
	}
	since $\abs{\vx} = r$ and $\vx = r \,\nh$.  Then
	\eq{
		\vx \vdot \vE = \frac{1}{2\pi \epso} (\vx \vdot \vp) \paren{ \frac{1}{r^3} - \frac{i k}{r^2} } e^{i k r}
		= \frac{1}{2\pi \epso} (\nh \vdot \vp) \paren{ \frac{1}{r^2} - \frac{i k}{r} } e^{i k r}.
	}
	
	With these substitutions, Eq.~\refeq{l1} becomes
	\al{
		\vl &= -\frac{(\vx \vdot \vE) \vHs}{c^2}
		= -\frac{1}{4\pi \epso c^2} (\nh \vdot \vp) \paren{ \frac{1}{r^2} - \frac{i k}{r} } e^{i k r} \frac{c k^2}{4\pi} (\nh \cross \vps) \frac{e^{-i k r}}{r} \paren{ 1 + \frac{1}{i k r} } \\
		&= -\frac{k^2}{16\pi^2 \epso c r} (\nh \vdot \vp) (\nh \cross \vps) \paren{ \frac{1}{r^2} - \frac{i k}{r} } \paren{ 1 - \frac{i}{k r} }
		= -\frac{k^2}{16\pi^2 \epso c} (\nh \vdot \vp) (\nh \cross \vps) \paren{ \frac{1}{r^2} - \frac{i}{k r^3} - \frac{i k}{r} - \frac{1}{r^2} } \\
		&= -\frac{i k^2}{16\pi^2 \epso c r} (\nh \vdot \vp) (\nh \cross \vps) \paren{ \frac{1}{k r^3} + \frac{k}{r^2} }
		= \frac{i k^3}{16\pi^2 \epso c r^2} (\nh \vdot \vp) (\nh \cross \vps) \paren{ \frac{1}{k^2 r^2} + 1 }.
	}
	
	Let $\vL$ be the angular momentum radiated to a distance $R$.  Then
	\eq{
		\vL = \int_R \vl(r) \ddcx
		= \intopi \intotp \intoR \vl(r) \,r^2 \sin\tht \ddr \ddphi \dd\tht,
	}
	and the time derivative is
	\aln{
		\dv{\vL}{t} &= \dv{t}(\intopi \intotp \intoR \vl(r) \,r^2 \sin\tht \ddr \ddphi \dd\tht)
		= \dv{r}{t} \dv{r}(\intopi \intotp \intoR \vl(r) \,r^2 \sin\tht \ddr \ddphi \dd\tht) \notag \\
		&= c \intopi \intotp \vl(r) \,r^2 \sin\tht \ddphi \dd\tht
		= \frac{i k^3}{16\pi^2 \epso} \paren{ \frac{1}{k^2 r^2} + 1 } \intopi \intotp (\nh \vdot \vp) (\nh \cross \vps) \sin\tht \ddphi \dd\tht. \label{dLdt}
	}
	Note that
	\eq{
		[ (\nh \vdot \vp) (\nh \cross \vps) ]_i = \sumje n_j p_j (\nh \cross \vps)_i
		= \sumje \sumke \sumle \epsikl n_j p_j n_k p_l^*,
	}
	so
	\eq{
		\dv{L_i}{t} \propto \sumje \sumke \sumle \epsikl p_j p_l^* \int n_j p_k \ddOmg
		= \sumje \sumke \sumle \epsikl p_j p_l^* \frac{4\pi}{3} \del_{jk}
		= \frac{4\pi}{3} \epsikl p_k p_l^*
		= \frac{4\pi}{3} (\vp \cross \vps)_i,
	}
	where we have used Jackson~(9.47), $\int n_\bet n_\gam \ddOmg = 4\pi \del_{\bet \gam} / 3$.  Making this substitution into Eq.~\refeq{dLdt},
	\eq{
		\dv{\vL}{t} = \frac{i k^3}{6\pi \epso} \paren{ \frac{1}{k^2 r^2} + 1 } (\vp \cross \vps).
	}
	Taking the limit as $r \to \infty$, we find
	\eqn{ans1a}{
		\dv{\vL}{t} = \Re\!\brac{ \frac{i k^3}{12\pi \epso} (\vp \cross \vps) }
		= \Re\!\brac{ -\frac{i k^3}{12\pi \epso} (\vps \cross \vp) }
		= \ans{ \frac{k^3}{12\pi \epso} \Im[ \vps \cross \vp ], }
	}
	as desired. \qed
}

%
%	Jackson 9.8(b)
%

\prob{}{
	What is the ratio of angular momentum radiated to energy radiated?  Interpret.
}

\sol{
	According to Jackson~(9.24), the total power radiated by an oscillating electric dipole $\vp$ is
	\eq{
		P = \dv{E}{t}
		= \frac{c^2 \Zo k^4}{12 \pi} \abs{\vp}^2.
	}
	Then the ratio of angular momentum radiated to energy radiated is
	\eq{
		\frac{\dv*{\vL}{t}}{\dv*{E}{t}} = \frac{k^3}{12\pi \epso} \Im[ \vps \cross \vp ] \frac{12 \pi}{c^2 \Zo k^4 \abs{\vp}^2}
		= \frac{1}{\epso} \Im[ \vps \cross \vp ] \frac{1}{c^2 \Zo k \abs{\vp}^2}
		= \ans{ \frac{\Im[ \vps \cross \vp ]}{\omg \abs{\vp}^2}, }
	}
	where we have used $\Zo = \sqrt{\muo / \epso} = 1 / \sqrt{\epso^2 c^2} = 1 / \epso c$, $c^2 = 1 / (\epso \muo)$, and $\omg = k c$.
	
	In the limit of high frequency, $(\dv*{\vL}{t}) / (\dv*{E}{t}) \to 0$.  In this scenario, the energy radiated dominates over the angular momentum radiated.  Likewise, in the limit of low frequency, $(\dv*{\vL}{t}) / (\dv*{E}{t}) \to \infty$, meaning that angular momentum radiation dominates.  This is sensible because rotational kinetic energy $E \propto \omg^2$, while angular momentum $L \propto \omg$.
}

%
%	Jackson 9.8(c)
%

\prob{}{
	For a charge $e$ rotating in the $xy$ plane at radius $a$ and angular speed $\omg$, show that there is only a $z$ component of radiated angular momentum with magnitude $\dv*{\Lz}{t} = e^2 k^3 a^2 / 6 \pi \epso$.  What about a charge oscillating along the $z$ axis?
}

\sol{
	We know from Homework~5 that the position of a point charge rotating counterclockwise in the $xy$ plane is
	\eq{
		\vx(t) = a \cos(\omg t) \,\vx + a \sin(\omg t) \,\yh.
	}
	\clearpage
	Then the charge distribution is
	\eq{
		\rho(\vx, t) = e \del[ x - a \cos(\omg t) ] \,\del[ y - a \sin(\omg t) ] \,\del(z).
	}
	
	According to Jackson~(4.8), the dipole moment is defined
	\eq{
		\vp = \int \vx' \,\rho(\vx') \ddcxp.
	}
	The components of $\vp$ for the point charge are then
	\al{
		\px &= e \iiint x \,\del[ x - a \cos(\omg t) ] \,\del[ y - a \sin(\omg t) ] \,\del(z) \ddx \ddy \ddz
		= e a \cos(\omg t), \\
		\py &= e \iiint y \,\del[ x - a \cos(\omg t) ] \,\del[ y - a \sin(\omg t) ] \,\del(z) \ddx \ddy \ddz
		= e a \sin(\omg t), \\
		\pz &= e \iiint z \,\del[ x - a \cos(\omg t) ] \,\del[ y - a \sin(\omg t) ] \,\del(z) \ddx \ddy \ddz
		= 0,
	}
	so we can write $\vp = e a \,e^{-i \omg t} (\xh + i\,\yh).$  Substituting into Eq.~\refeq{ans1a},
	\al{
		\dv{\vL}{t} &= \Re\!\brac{ \frac{i k^3}{12\pi \epso} e^2 a^2 e^{-i \omg t} e^{i \omg t} [ (\xh + i\,\yh) \cross (\xh - i\,\yh) ] }
		= \Re\!\brac{ \frac{i e^2 k^3 a^2}{12\pi \epso} (-2i \,\xh \cross \yh) }
		= \Re\!\brac{ \frac{e^2 k^3 a^2}{6\pi \epso} \,\zh } \\
		&= \ans{ \frac{e^2 k^3 a^2}{6\pi \epso} \cos(\omg t) \,\zh, }
	}
	as desired. \qed
	
	A charge oscillating along the $z$ axis with amplitude $a$ has the charge density
	\eq{
		\rho(\vx, t) = e a \,\del(x) \,\del(y) \,\del[ z - \cos(\omg t) ],
	}
	which gives the dipole moment
	\al{
		\px &= e a \iiint x \,\del(x) \,\del(y) \,\del[ z - \cos(\omg t) ] \ddx \ddy \ddz
		= 0, \\
		\py &= e a \iiint y \,\del(x) \,\del(y) \,\del[ z - \cos(\omg t) ] \ddx \ddy \ddz
		= 0, \\
		\pz &= e a \iiint z \,\del(x) \,\del(y) \,\del[ z - \cos(\omg t) ] \ddx \ddy \ddz
		= e a \cos(\omg t).
	}
	In complex notation, $\vp = e a \,e^{-i\omg t} \,\zh$.  Substituting into Eq.~\refeq{ans1a}, we find
	\eq{
		\dv{\vL}{t} = \Re\!\brac{ \frac{i k^3}{12\pi \epso} e^2 a^2 e^{-i \omg t} e^{i \omg t} (\zh \cross \zh) }
		= \ans{ \vO. }
	}
	So we see that a charge undergoing linear motion does not lead to a radiated angular momentum, which is sensible.
}

%
%	Jackson 9.8(d)
%

\prob{}{
	What are the results corresponding to Probs.~{1(a)} and {1(b)} for magnetic dipole radiation?
}

\sol{
	The radiation fields for a magnetic dipole are given by Jackson~(19.35--36),
	\al{
		\vH &= \frac{1}{4\pi} \curly{ k^2 (\nh \cross \vm) \cross \nh \frac{e^{i k r}}{r} + [ 3 \nh (\nh \vdot \vm) - \vm ] \paren{ \frac{1}{r^3} - \frac{i k}{r^2} } e^{i k r} }, &
		\vE &= -\frac{\Zo}{4\pi} k^2 (\nh \cross \vm) \frac{e^{i k r}}{r} \paren{ 1 - \frac{1}{i k r} }.
	}
	\clearpage
	Comparing with Eq.~\refeq{fields1}, we see that $\vH \to -\vE / \Zo$, $\vE \to \Zo \vH$, and $\vp \to \vm / c$ as stated in the book~\cite[p.~413]{Jackson}.  Making these substitutions, the results of Probs.~{1.1(a)} and {(b)} become
	\al{
		\ans{ \dv{\vL}{t}\ }&\ans{= \frac{\muo k^3}{12\pi} \Im[ \vms \cross \vm ], } &
		\ans{ \frac{\dv*{\vL}{t}}{\dv*{E}{t}}\ }&\ans{= \frac{\Im[ \vms \cross \vm ]}{\omg \abs{\vm}^2} }
	}
	where we have used $\mu = 1 / \epso c^2$.
}
\state{Beta function of the Gross-Neveu model~(P\&S~12.2)}{
	Compute $\bet(g)$ in the two-dimensional Gross-Neveu model studied in Problem~11.3,
	\eq{
		\cL = \psibsi i \ptsl \psisi + \frac{1}{2} g^2 (\psibsi \psisi)^2,
	}
	with $i = 1, \ldots, N$.  You should find that this model is asymptotically free.  How was that fact reflected in the solution to Problem~11.3?
}

\sol{
	We saw in Problem~2 of Homework~4 that this Lagrangian can be written as
	\eq{
		\cL = \psibsi i \ptsl \psisi - \sig \psibsi \psisi - \frac{1}{2 g^2} \sig^2,
	}
	where $\sig$ is a new scalar field with no kinetic energy terms.  In the modified minimal subtraction scheme, we found the effective potential was
	\eqn{Veff}{
		\Veff = \sig^2 \curly{ \frac{1}{2 g^2} + \frac{N}{4\pi} \brac{ \ln(\frac{\sig^2}{M^2}) - 1 } }.
	}
	Since $\Gam[ \phicl ] = -(V T) \Veff(\phi)$ by P\&S~(11.50), we have
	\eqn{Gam}{
		\Gam[ \sigcl ] = -(V T)  \sig^2 \curly{ \frac{1}{2 g^2} + \frac{N}{4\pi} \brac{ \ln(\frac{\sig^2}{M^2}) - 1 } }.
	}
	Referring to p.~3 of Lecture~11, we can apply the Callan-Symanzik equation to $\Gam$.   The Callan-Symanzik equation is P\&S~(12.41),
	\eq{
		\brac{ M \pdv{M} + \bet(\lam) \pdv{\lam} + n \gam(\lam) } G^{(n)}(\{ x_i \}; M, \lam) = 0.
	}
	For our problem, $\gam$ is 0 because there are no field insertions.  That is, we have
	\eq{
		\brac{ M \pdv{M} + \bet(g) \pdv{g} } \Gam[ \phicl ] = 0.
	}
	Using Eq.~\refeq{Gam}, note that
	\al{
		\pdv{\Gam}{M} &= (V T) \frac{N \sig^2}{2 \pi M}, &
		\pdv{\Gam}{g} &= (V T) \frac{\sig^2}{g^3}.
	}
	Then
	\eq{
		0 = (V T) \paren{ \frac{N \sig^2}{2 \pi} + \bet(g) \frac{\sig^2}{g^3} }
		\qimplies
		\ans{ \betg = -\frac{N g^3}{2\pi}. }
	}
	This model is asymptotically free because the $\bet$ function is proportional to $-g^3$~\cite[pp.~424--425]{Peskin}.
	
	In 2(e) of Homework~4, we found that the vacuum expectation value of $\sig$ was
	\eq{
		\sig = \pm M e^{-\pi / N g^2} = \pm v.
	}
	We showed that the vacuum expectation value does not depend on the renormalization condition chosen.  This means that we can increase $M \to 0$ while holding $\sig$ constant, and see that $g \to 0$ logarithmically.  This is indicative of an asymptotically-free theory~\cite[p.~425]{Peskin}. \qed
}


\state{Positronium lifetimes (Peskin \& Schroeder 5.4)}{\hfix}

\prob{
	Compute the amplitude $\cM$ for $\elp \elm$ annihilation into 2 photons in the extreme nonrelativistic limit (i.e., keep only the term proportional to zero powers of the electron and positron 3-momentum).  Use this result, together with our formalism for fermion-antifermion bound states, to compute the rate of annihilation of the $1S$ states of positronium into 2 photons.  You should find that the spin-1 states of positronium do not annihilate into 2 photons, confirming the symmetry argument of Problem~3.8.  For the spin-0 state of positronium, you should find a result proportional to the square of the $1S$ wavefunction at the origin.  Inserting the value of this wavefunction from nonrelativistic quantum mechanics, you should find
	\eq{
		\frac{1}{\tau} = \Gam
		= \frac{\alp^5 \me}{2}
		\approx \SI{8.03e9}{\per\second}.
	}
	A recent measurement gives $\Gam = 7.994 \pm \SI{0.011}{\per\nano\second}$; the 0.5\% discrepancy is accounted for by radiative corrections.
}

\sol{
	The amplitude for $\elp\elm$ annihilation into two photons can be obtained from the Feynman diagrams on p.~168 of Peskin \& Schroeder:
	\al{
		\centergraphics{diag/gg_t_chan} &= \vbpw (-i e \gamm) \epssmskw \frac{i (\pslq - \kslq + m)}{(\pq - \kq)^2 - m^2} (-i e \gamn) \epssnskq \upq, \\[2ex]
		\centergraphics{diag/gg_u_chan} &= \vbpw (-i e \gamn) \epssnskq \frac{i (\pslq - \kslw + m)}{(\pq - \kw)^2 - m^2} (-i e \gamm) \epssmskw \upq.
	}
	Their sum is
	\eq{
		i \cM = -i e^2 \epssmskw \epssnskq \vbpw \paren{ \frac{\gamm (\pslq - \kslq + m) \gamn}{(\pq - \kq)^2 - m^2} + \frac{\gamn (\pslq - \kslw + m) \gamm}{(\pq - \kw)^2 - m^2} } \upq
	}
	where we have also referred to the expressions for Compton scattering, as well as the fermion and photon Feynman rules~\cite[pp.~118, 123, 158--159]{Peskin}.  
	
	In the extreme nonrelativistic limit and in the center-of-mass frame, let
	\al{
		\pq &= (m, \vo), &
		\pw &= (m, \vo), &
		\kq &= (m, m \zh), &
		\kw &= (m, -m \zh).
	}
	Here we have the two photons being emitted in opposite directions along the $z$ axis.  Note that $\pq - \kq = -m \zh$ and $\pq - \kw = m \zh$, so
	\al{
		(\pq - \kq)^2 = (\pq - \kq)_\mu (\pq - \kq)^\mu
		= -m^2
		= (\pq - \kw)^2.
	}
	We also choose the polarization vectors~\cite[p.~124]{Peskin}
	\aln{ \label{eps}
		\epskq &= (0, 1, \pm i, 0), &
		\epskw &= (0, -1, \mp i, 0).
	}
	Then the amplitude becomes
	\al{
		i \cM &\approx -i e^2 \epssmskw \epssnskq \vbpw \paren{ \frac{\gamm (m \gamo - m \gamo + m \game + m) \gamn}{-m^2 - m^2} + \frac{\gamn (m \gamo - m \gamo - m \game + m) \gamm}{-m^2 - m^2} } \upq \\
		&= \frac{i e^2}{2 m} \epssmskw \epssnskq \vbpw \brac{ \gamm (1 + \game) \gamn + \gamn (1 - \game) \gamm } \upq \\
		&= \frac{i e^2}{2 m} \epssmskw \epssnskq \vbpw \brac{ \gamm \gamn + \gamm \game \gamn + \gamn \gamm - \gamn \game \gamm } \upq \\
		&= \frac{i e^2}{2 m} \epssmskw \epssnskq \vbpw \brac{ \gamm \game \gamn - \gamn \game \gamm } \upq
	}
	where we have used $\{ \gami, \gamj \} = -2 \delij$~\cite[p.~41]{Peskin}.  Here, $\delij = 0$ since $\epskq$ and $\epskw$ both have zero time component:
	\eq{
		\epssmskw \epssnskq \{ \gamm, \gamn \} = -2 \epssmskw \epssnskq \delmn
		= \epssmskw \epssmskq
		= 0
	}
	since from Eq.~\refeq{eps}, $\epskq$ and $\epskw$ are orthogonal.  In the nonrelativistic limit, we can use Peskin \& Schroeder~(5.35),
	\al{
		u(k) &= \sqrt{m} \mqty( \xi \\ \xi ), &
		v(k') &= \sqrt{m} \mqty( \xi' \\ -\xi' ).
	}
	So using (3.25), the amplitude can be written
	\al{
		i \cM &= \frac{i e^2}{2} \epssmskw \epssnskq \mqty( \xipdag & -\xipdag ) \mqty( 0 & 1 \\ 1 & 0 ) \brac{ \mqty( 0 & \sigbm \\ \sigm & 0 ) \mqty( 0 & \sigbe \\ \sige & 0 ) \mqty( 0 & \sigbn \\ \sign & 0 ) - \mqty( 0 & \sigbn \\ \sign & 0 ) \mqty( 0 & \sigbe \\ \sige & 0 ) \mqty( 0 & \sigbm \\ \sigm & 0 ) } \mqty( \xi \\ \xi ) \\
		&= \frac{i e^2}{2} \epssmskw \epssnskq \mqty( \xipdag & -\xipdag ) \mqty( 0 & 1 \\ 1 & 0 ) \brac{ \mqty( 0 & \sigbm \sige \sigbn \\ \sigm \sigbe \sign & 0 ) - \mqty( 0 & \sigbn \sige \sigbm \\ \sign \sigbe \sigm & 0 ) } \mqty( \xi \\ \xi ) \\
		&= \frac{i e^2}{2} \epssmskw \epssnskq \mqty( \xipdag & -\xipdag ) \mqty( 0 & 1 \\ 1 & 0 ) \brac{ \mqty( 0 & \sigm \sige \sign \\ -\sigm \sige \sign & 0 ) - \mqty( 0 & \sign \sige \sigm \\ -\sign \sige \sigm & 0 ) } \mqty( \xi \\ \xi ) \\
		&= \frac{i e^2}{2} \epssmskw \epssnskq \mqty( \xipdag & -\xipdag ) \mqty( 0 & 1 \\ 1 & 0 ) \mqty( 0 & \sigm \sige \sign - \sign \sige \sigm \\ -(\sigm \sige \sign - \sign \sige \sigm) & 0 ) \mqty( \xi \\ \xi ) \\
		&= \frac{i e^2}{2} \epssmskw \epssnskq \mqty( \xipdag & -\xipdag ) \mqty( -(\sigm \sige \sign - \sign \sige \sigm) & 0 \\ 0 & \sigm \sige \sign - \sign \sige \sigm ) \mqty( \xi \\ \xi ) \\
		&= \frac{i e^2}{2} \epssmskw \epssnskq \mqty( \xipdag & -\xipdag ) \mqty( -(\sigm \sige \sign - \sign \sige \sigm) \\ \sigm \sige \sign - \sign \sige \sigm ) \xi \\
		&= -i e^2 \epssmskw \epssnskq \xipdag (\sigm \sige \sign - \sign \sige \sigm) \xi, \\
		&= \ans{ i e^2 \epssmskw \epssnskq \xipdag (\sign \sige \sigm - \sigm \sige \sign) \xi, }
	}
	since $\sigbi = -\sigi$.
	
%	Adapting Peskin \& Schroeder~(5.44),
%	\eq{
%		\cM(1S \to \gam \gam) = \sqrt{2 M} \int \ddkf \psik \frac{1}{\sqrt{2m}} \frac{1}{\sqrt{2m}} \cM(\elp \elm \to \gam \gam)
%		= \sqrt{ \frac{2}{m} } \int \ddkf \psik \cM(\elp \elm \to \gam \gam)
%	}
%	since $M \approx 2m$~\cite[p.~149]{Peskin}.  Here $\cM(\elp \elm \to \gam \gam)$ depends on the spin states and polarizations.  
	
	From Peskin \& Schroeder~(5.46),
	\eq{
		i \cM = \xidag \Gamk \xi'
		\qimplies
		\Gamk = i e^2 \epssmskw \epssnskq (\sign \sige \sigm - \sigm \sige \sign).
	}
	Adapting their (5.50),
	\eq{
		i \cM(1S \to \gam \gam) = \sqrt{ \frac{2}{m} } \int \ddkf \psik \Tr( \frac{\nhs \vdot \vsig}{\sqrt{2}} \Gamk )
	}
	which implies
	\eq{
		-i \cM^*(1S \to \gam \gam) = \sqrt{ \frac{2}{m} } \int \ddkf \psik \Tr( \frac{\nhs \vdot \vsig}{\sqrt{2}} \Gamk )
	}
	\hl{wavefunction comes from ground state hydrogen}

	A spin-0 state can be obtained by Peskin \& Schroeder~(5.49),
	\eq{
		\xi' \xidag \to \frac{\vqq}{\sqrt{2}}.
	}
	In the case $\epskq = (0, 1, i, 0) / \sqrt{2}$, $\epskw = (0, -1, i, 0) / \sqrt{2}$, $\epssmskq \sigm = (\sigq - i \sigw) / \sqrt{2}$ and $\epssmskw \sigm = -(\sigq + i \sigw) / \sqrt{2}$.  Then
	\al{
		\Tr( \frac{\vqq}{\sqrt{2}} \Gamk ) &= \Tr( -\frac{\vqq}{2 \sqrt{2}} i e^2 [ (\sigq - i \sigw) \sige (\sigq + i \sigw) - (\sigq + i \sigw) \sige (\sigq - i \sigw) ] ) \\
		&= \Tr( -\frac{\vqq}{2 \sqrt{2}} i e^2 [ \sigq \sige \sigq + i \sigq \sige \sigw - i \sigw \sige \sigq + \sigw \sige \sigw - \sigq \sige \sigq + i \sigq \sige \sigw - i \sigw \sige \sigq - \sigw \sige \sigw ] ) \\
		&= \Tr( \frac{\vqq}{\sqrt{2}} e^2 [ \sigq \sige \sigw - \sigw \sige \sigq ] ) \\
		&= \Tr( \frac{\vqq}{\sqrt{2}} e^2 [ \sigq \sige \sigw + \sigq \sige \sigw ] ) \\
		&= \Tr( \sqrt{2} e^2 \vqq \sigq \sige \sigw ) \\
		&= \sqrt{2} e^2 2i \epsqew \\
		&= -i 2 \sqrt{2} e^2,
	}
	where we have used $\Tr(\sigi \sigj \sigk) = 2i \epsijk$~\cite{Pauli}.
	
	In the case $\epskq = (0, 1, -i, 0)$, $\epskw = (0, -1, -i, 0)$, $\epssmskq \sigm = (\sigq + i \sigw) / \sqrt{2}$ and $\epssmskw \sigm = -(\sigq - i \sigw) / \sqrt{2}$.  Then
	\al{
		\Tr( \frac{\vqq}{\sqrt{2}} \Gamk ) &= \Tr( -\frac{\vqq}{2 \sqrt{2}} i e^2 [ (\sigq + i \sigw) \sige (\sigq - i \sigw) - (\sigq - i \sigw) \sige (\sigq + i \sigw) ] ) \\
		&= \Tr( -\frac{\vqq}{2 \sqrt{2}} i e^2 [ \sigq \sige \sigq - i \sigq \sige \sigw + i \sigw \sige \sigq + \sigw \sige \sigw - \sigq \sige \sigq -i \sigq \sige \sigw + i \sigw \sige \sigq - \sigw \sige \sigw ] ) \\
		&= \Tr( -\frac{\vqq}{\sqrt{2}} e^2 [ \sigq \sige \sigw - \sigw \sige \sigq ] ) \\
		&= i 2 \sqrt{2} e^2.
	}
	In the case $\epskq = (0, 1, i, 0)$, $\epskw = (0, -1, -i, 0)$, $\epssmskq \sigm = (\sigq - i \sigw) / \sqrt{2}$ and $\epssmskw \sigm = -(\sigq - i \sigw) / \sqrt{2}$.  Then
	\eq{
		\Tr( \frac{\vqq}{\sqrt{2}} \Gamk ) = \Tr( -\frac{\vqq}{2 \sqrt{2}} i e^2 [ (\sigq - i \sigw) \sige (\sigq - i \sigw) - (\sigq - i \sigw) \sige (\sigq - i \sigw) ] )
		= 0.
	}
	In the case $\epskq = (0, 1, -i, 0)$, $\epskw = (0, -1, i, 0)$, $\epssmskq \sigm = (\sigq + i \sigw) / \sqrt{2}$ and $\epssmskw \sigm = -(\sigq + i \sigw) / \sqrt{2}$.  Then
	\eq{
		\Tr( \frac{\vqq}{\sqrt{2}} \Gamk ) = \Tr( -\frac{\vqq}{2 \sqrt{2}} i e^2 [ (\sigq + i \sigw) \sige (\sigq + i \sigw) - (\sigq + i \sigw) \sige (\sigq + i \sigw) ] ) 
		= 0.
	}
	To summarize, in the singlet case we have
	\eq{
		\Tr( \frac{\vqq}{\sqrt{2}} \Gamk ) = \begin{cases}
			-i 2 \sqrt{2} e^2 &\text{if } \epskq = \dfrac{(0, 1, i, 0)}{\sqrt{2}}, \epskw = \dfrac{(0, -1, i, 0)}{\sqrt{2}}; \\[2ex]
			i 2 \sqrt{2} e^2 &\text{if } \epskq = \dfrac{(0, 1, -i, 0)}{\sqrt{2}}, \epskw = \dfrac{(0, -1, -i, 0)}{\sqrt{2}}; \\[2ex]
			0 & \text{otherwise}.
		\end{cases}
	}
		
	From Peskin \& Schroeder~(5.48), we can obtain a spin-1 state via
	\eq{
		\xi' \xidag \to \frac{\nhs \vdot \vsig}{\sqrt{2}},
	}
	where~\cite[p.~150]{Peskin}
	\eq{
		\nh \in \curly{ \frac{\xh + i \yh}{\sqrt{2}},\ \frac{\xh - i \yh}{\sqrt{2}},\ \zh }.
	}
	
	For $\nh = \zh$, $\xi' \xidag \to \sige / \sqrt{2}$.  Then for $\epskq = (0, 1, i, 0) / \sqrt{2}$, $\epskw = (0, -1, i, 0) / \sqrt{2}$,
	\eq{
		\Tr( \frac{\sige}{\sqrt{2}} \Gamk ) = \Tr( \sqrt{2} e^2 \sige \sigq \sige \sigw )
		= 0,
	}
	where we have used $\Tr(\sigi \sigj \sigk \sigl) = 2 (\del^{i j} \del^{k l} - \del^{i k} \del^{k l} + \del^{i l} \del^{j k})$~\cite{Pauli}.  The $\epskq = (0, 1, -i, 0) / \sqrt{2}$, $\epskw = (0, -1, -i, 0) / \sqrt{2}$ case is the same with an overall minus sign.  For the other two cases, the argument of the trace is identically zero.
	
	For $\nh = (\xh + i \yh) / \sqrt{2}$ and $\nh = (\xh - i \yh) / \sqrt{2}$, the argument is identical.  We get two trace terms, each being either identically zero or taking the form $\Tr(\sigi \sigj \sigk \sigl) = 2 (\del^{i j} \del^{k l} - \del^{i k} \del^{k l} + \del^{i l} \del^{j k})$ with all of the Kronecker deltas being zero.  Thus we have
	\eq{
		\Tr( \frac{\nhs \vdot \vsig}{\sqrt{2}} \Gamk ) = 0
	}
	for the spin-1 states.  This means, in the spin-1 case,
	\eq{
		\ans{ i \cM(1S \to \gam \gam) = \sqrt{ \frac{2}{m} } \int \ddkf \psik \Tr( \frac{\nhs \vdot \vsig}{\sqrt{2}} \Gamk ) = 0 }
	}
	trivially.  Thus, the spin-1 positronium states do not annihilate into two photons, as we wanted to show. \qed
	
	Returning to the spin-0 states, 
}


\makebib

\end{document}
