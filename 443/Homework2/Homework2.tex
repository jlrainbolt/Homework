\documentclass[11pt]{article}
\usepackage{homework}

\classname{443}
\homeworknum{2}



\begin{document}

% Environments

\newcommand{\state}[2]{\begin{statement}{#1} #2 \end{statement}}
\newcommand{\prob}[2]{\begin{problem}{#1} #2 \end{problem}}
\newcommand{\subprob}[1]{\begin{subproblem} #1 \end{subproblem}}
\newcommand{\sol}[1]{\begin{solution} #1 \end{solution}}
\newcommand{\fig}[2]{\begin{figure} \centering #2  \label{#1} \end{figure}}

\newcommand{\makebib}{
	\vfill
	\color{black}
	\bibliography{references}{}
	\bibliographystyle{lucas_unsrt}
}
	

% Implication

\newcommand{\qwhere}{\quad \text{where} \quad}
\newcommand{\qimplies}{\quad \implies \quad}
\newcommand{\impliesq}{\implies \quad}



% Brackets

\newcommand{\paren}[1]{\left( #1 \right)}
\newcommand{\brac}[1]{\left[ #1 \right]}


% Greek

\newcommand{\alp}{\alpha}
\newcommand{\bet}{\beta}
\newcommand{\gam}{\gamma}
\newcommand{\del}{\delta}
\newcommand{\eps}{\epsilon}
\newcommand{\zet}{\zeta}
\newcommand{\tht}{\theta}
\newcommand{\kap}{\kappa}
\newcommand{\lam}{\lambda}
\newcommand{\sig}{\sigma}
\newcommand{\ups}{\upsilon}
\newcommand{\omg}{\omega}

\newcommand{\Gam}{\Gamma}
\newcommand{\Del}{\Delta}
\newcommand{\Tht}{\Theta}
\newcommand{\Lam}{\Lambda}
\newcommand{\Sig}{\Sigma}
\newcommand{\Omg}{\Omega}
% Problem 1

\newcommand{\Psii}{\Psi^i}
\newcommand{\Psiix}{\Psii(x)}

\newcommand{\Pii}{\Pi^i}

\newcommand{\Phii}{\Phi^i}
\newcommand{\Phiix}{\Phii(x)}
\newcommand{\PhiN}{\Phi^N}
\newcommand{\PhiNx}{\PhiN(x)}
\newcommand{\Phiq}{\Phi^1}
\newcommand{\Phiw}{\Phi^2}

\newcommand{\ddcx}{\dd[3]{x}}

\newcommand{\delij}{\del^{i j}}
\newcommand{\delkl}{\del^{k l}}
\newcommand{\delil}{\del^{i l}}
\newcommand{\deljk}{\del^{j k}}
\newcommand{\delik}{\del^{i k}}
\newcommand{\deljl}{\del^{j l}}

\newcommand{\DF}{D_F}

\newcommand{\sigx}{\sig(x)}

\newcommand{\pii}{\pi^i}
\newcommand{\pij}{\pi^j}
\newcommand{\pik}{\pi^k}
\newcommand{\pil}{\pi^l}
\newcommand{\piix}{\pi(x)}

\newcommand{\pq}{p_1}
\newcommand{\pw}{p_2}
\newcommand{\pe}{p_3}
\newcommand{\pr}{p_4}

\newcommand{\vp}{\vb{p}}
\newcommand{\vpsi}{\vp_i}

\newcommand{\mpi}{m_\pi}


\state{(Jackson 9.8)}{\ 
	%\emph{Hint:} The electromagnetic angular momentum density comes from more than the transverse (radiation zone) components of the fields.
}

%
%	Jackson 9.8(a)
%

\prob{}{
	Show that a classical oscillating electric dipole $\vp$ with fields given by
	\aln{ \label{fields1}
		\vH &= \frac{c k^2}{4\pi} (\nh \cross \vp) \frac{e^{i k r}}{r} \paren{ 1 - \frac{1}{i k r} }, &
		\vE &= \frac{1}{4\pi \epso} \curly{ k^2 (\nh \cross \vp) \cross \nh \frac{e^{i k r}}{r} + [ 3 \nh (\nh \vdot \vp) - \vp ] \paren{ \frac{1}{r^3} - \frac{i k}{r^2} } e^{i k r} },
	}
	radiates electromagnetic angular momentum to infinity at the rate
	\eq{
		\dv{\vL}{t} = \frac{k^3}{12 \pi \epso} \Im[ \vp^* \cross \vp ].
	}
	\vfix
}

\sol{
	According to Jackson~(9.20), the time-averaged angular momentum density is
	\eq{
		\vl = \frac{\Re[ \vx \cross (\vE \cross \vHs)}{2 c^2}.
	}
	One of the vector identities on the inside cover of Jackson is $\vaa \cross (\vbb \cross \vcc) = (\vaa \vdot \vcc) \vbb - (\vaa \vdot \vbb) \vcc$, so
	\eqn{l1}{
		\vl = \frac{(\vx \vdot \vHs) \vE - (\vx \vdot \vE) \vHs}{2 c^2}.
	}
	From Eq.~\refeq{fields1}, note that
	\eq{
		\vx \vdot \vHs \propto \vx \vdot (\nh \cross \vps)
		= \vps \vdot (\vx \cross \nh)
		= \vO,
	}
	where we have used the identity $\vaa \vdot (\vbb \cross \vcc) = \vcc \vdot (\vaa \cross \vbb)$ and the fact that $\nh$ points in the $\vx$ direction.  For $\vx \vdot \vE$, note that
	\al{
		\vx \vdot [ (\nh \cross \vp) \cross \nh ] &= -\vx \vdot [ \nh \cross (\nh \cross \vp) ]
		= -\vx \vdot [ (\nh \vdot \vp) \nh - (\nh \vdot \nh) \vp ]
		= -(\nh \vdot \vp) (\vx \vdot \nh) + \vx \vdot \vp \\
		&= -r (\nh \vdot \vp) + \vx \vdot \vp
		= \vx \vdot \vp - \vx \vdot \vp
		= 0, \\[1.5ex]
		\vx \vdot [ 3 \nh (\nh \vdot \vp) - \vp ] &= 3 (\vx \vdot \nh) (\nh \vdot \vp) - \vx \vdot \vp
		= 3r (\nh \vdot \vp) - \vx \vdot \vp
		= 3(\vx \vdot \vp) - \vx \vdot \vp
		= 2(\vx \vdot \vp),
	}
	since $\abs{\vx} = r$ and $\vx = r \,\nh$.  Then
	\eq{
		\vx \vdot \vE = \frac{1}{2\pi \epso} (\vx \vdot \vp) \paren{ \frac{1}{r^3} - \frac{i k}{r^2} } e^{i k r}
		= \frac{1}{2\pi \epso} (\nh \vdot \vp) \paren{ \frac{1}{r^2} - \frac{i k}{r} } e^{i k r}.
	}
	
	With these substitutions, Eq.~\refeq{l1} becomes
	\al{
		\vl &= -\frac{(\vx \vdot \vE) \vHs}{c^2}
		= -\frac{1}{4\pi \epso c^2} (\nh \vdot \vp) \paren{ \frac{1}{r^2} - \frac{i k}{r} } e^{i k r} \frac{c k^2}{4\pi} (\nh \cross \vps) \frac{e^{-i k r}}{r} \paren{ 1 + \frac{1}{i k r} } \\
		&= -\frac{k^2}{16\pi^2 \epso c r} (\nh \vdot \vp) (\nh \cross \vps) \paren{ \frac{1}{r^2} - \frac{i k}{r} } \paren{ 1 - \frac{i}{k r} }
		= -\frac{k^2}{16\pi^2 \epso c} (\nh \vdot \vp) (\nh \cross \vps) \paren{ \frac{1}{r^2} - \frac{i}{k r^3} - \frac{i k}{r} - \frac{1}{r^2} } \\
		&= -\frac{i k^2}{16\pi^2 \epso c r} (\nh \vdot \vp) (\nh \cross \vps) \paren{ \frac{1}{k r^3} + \frac{k}{r^2} }
		= \frac{i k^3}{16\pi^2 \epso c r^2} (\nh \vdot \vp) (\nh \cross \vps) \paren{ \frac{1}{k^2 r^2} + 1 }.
	}
	
	Let $\vL$ be the angular momentum radiated to a distance $R$.  Then
	\eq{
		\vL = \int_R \vl(r) \ddcx
		= \intopi \intotp \intoR \vl(r) \,r^2 \sin\tht \ddr \ddphi \dd\tht,
	}
	and the time derivative is
	\aln{
		\dv{\vL}{t} &= \dv{t}(\intopi \intotp \intoR \vl(r) \,r^2 \sin\tht \ddr \ddphi \dd\tht)
		= \dv{r}{t} \dv{r}(\intopi \intotp \intoR \vl(r) \,r^2 \sin\tht \ddr \ddphi \dd\tht) \notag \\
		&= c \intopi \intotp \vl(r) \,r^2 \sin\tht \ddphi \dd\tht
		= \frac{i k^3}{16\pi^2 \epso} \paren{ \frac{1}{k^2 r^2} + 1 } \intopi \intotp (\nh \vdot \vp) (\nh \cross \vps) \sin\tht \ddphi \dd\tht. \label{dLdt}
	}
	Note that
	\eq{
		[ (\nh \vdot \vp) (\nh \cross \vps) ]_i = \sumje n_j p_j (\nh \cross \vps)_i
		= \sumje \sumke \sumle \epsikl n_j p_j n_k p_l^*,
	}
	so
	\eq{
		\dv{L_i}{t} \propto \sumje \sumke \sumle \epsikl p_j p_l^* \int n_j p_k \ddOmg
		= \sumje \sumke \sumle \epsikl p_j p_l^* \frac{4\pi}{3} \del_{jk}
		= \frac{4\pi}{3} \epsikl p_k p_l^*
		= \frac{4\pi}{3} (\vp \cross \vps)_i,
	}
	where we have used Jackson~(9.47), $\int n_\bet n_\gam \ddOmg = 4\pi \del_{\bet \gam} / 3$.  Making this substitution into Eq.~\refeq{dLdt},
	\eq{
		\dv{\vL}{t} = \frac{i k^3}{6\pi \epso} \paren{ \frac{1}{k^2 r^2} + 1 } (\vp \cross \vps).
	}
	Taking the limit as $r \to \infty$, we find
	\eqn{ans1a}{
		\dv{\vL}{t} = \Re\!\brac{ \frac{i k^3}{12\pi \epso} (\vp \cross \vps) }
		= \Re\!\brac{ -\frac{i k^3}{12\pi \epso} (\vps \cross \vp) }
		= \ans{ \frac{k^3}{12\pi \epso} \Im[ \vps \cross \vp ], }
	}
	as desired. \qed
}

%
%	Jackson 9.8(b)
%

\prob{}{
	What is the ratio of angular momentum radiated to energy radiated?  Interpret.
}

\sol{
	According to Jackson~(9.24), the total power radiated by an oscillating electric dipole $\vp$ is
	\eq{
		P = \dv{E}{t}
		= \frac{c^2 \Zo k^4}{12 \pi} \abs{\vp}^2.
	}
	Then the ratio of angular momentum radiated to energy radiated is
	\eq{
		\frac{\dv*{\vL}{t}}{\dv*{E}{t}} = \frac{k^3}{12\pi \epso} \Im[ \vps \cross \vp ] \frac{12 \pi}{c^2 \Zo k^4 \abs{\vp}^2}
		= \frac{1}{\epso} \Im[ \vps \cross \vp ] \frac{1}{c^2 \Zo k \abs{\vp}^2}
		= \ans{ \frac{\Im[ \vps \cross \vp ]}{\omg \abs{\vp}^2}, }
	}
	where we have used $\Zo = \sqrt{\muo / \epso} = 1 / \sqrt{\epso^2 c^2} = 1 / \epso c$, $c^2 = 1 / (\epso \muo)$, and $\omg = k c$.
	
	In the limit of high frequency, $(\dv*{\vL}{t}) / (\dv*{E}{t}) \to 0$.  In this scenario, the energy radiated dominates over the angular momentum radiated.  Likewise, in the limit of low frequency, $(\dv*{\vL}{t}) / (\dv*{E}{t}) \to \infty$, meaning that angular momentum radiation dominates.  This is sensible because rotational kinetic energy $E \propto \omg^2$, while angular momentum $L \propto \omg$.
}

%
%	Jackson 9.8(c)
%

\prob{}{
	For a charge $e$ rotating in the $xy$ plane at radius $a$ and angular speed $\omg$, show that there is only a $z$ component of radiated angular momentum with magnitude $\dv*{\Lz}{t} = e^2 k^3 a^2 / 6 \pi \epso$.  What about a charge oscillating along the $z$ axis?
}

\sol{
	We know from Homework~5 that the position of a point charge rotating counterclockwise in the $xy$ plane is
	\eq{
		\vx(t) = a \cos(\omg t) \,\vx + a \sin(\omg t) \,\yh.
	}
	\clearpage
	Then the charge distribution is
	\eq{
		\rho(\vx, t) = e \del[ x - a \cos(\omg t) ] \,\del[ y - a \sin(\omg t) ] \,\del(z).
	}
	
	According to Jackson~(4.8), the dipole moment is defined
	\eq{
		\vp = \int \vx' \,\rho(\vx') \ddcxp.
	}
	The components of $\vp$ for the point charge are then
	\al{
		\px &= e \iiint x \,\del[ x - a \cos(\omg t) ] \,\del[ y - a \sin(\omg t) ] \,\del(z) \ddx \ddy \ddz
		= e a \cos(\omg t), \\
		\py &= e \iiint y \,\del[ x - a \cos(\omg t) ] \,\del[ y - a \sin(\omg t) ] \,\del(z) \ddx \ddy \ddz
		= e a \sin(\omg t), \\
		\pz &= e \iiint z \,\del[ x - a \cos(\omg t) ] \,\del[ y - a \sin(\omg t) ] \,\del(z) \ddx \ddy \ddz
		= 0,
	}
	so we can write $\vp = e a \,e^{-i \omg t} (\xh + i\,\yh).$  Substituting into Eq.~\refeq{ans1a},
	\al{
		\dv{\vL}{t} &= \Re\!\brac{ \frac{i k^3}{12\pi \epso} e^2 a^2 e^{-i \omg t} e^{i \omg t} [ (\xh + i\,\yh) \cross (\xh - i\,\yh) ] }
		= \Re\!\brac{ \frac{i e^2 k^3 a^2}{12\pi \epso} (-2i \,\xh \cross \yh) }
		= \Re\!\brac{ \frac{e^2 k^3 a^2}{6\pi \epso} \,\zh } \\
		&= \ans{ \frac{e^2 k^3 a^2}{6\pi \epso} \cos(\omg t) \,\zh, }
	}
	as desired. \qed
	
	A charge oscillating along the $z$ axis with amplitude $a$ has the charge density
	\eq{
		\rho(\vx, t) = e a \,\del(x) \,\del(y) \,\del[ z - \cos(\omg t) ],
	}
	which gives the dipole moment
	\al{
		\px &= e a \iiint x \,\del(x) \,\del(y) \,\del[ z - \cos(\omg t) ] \ddx \ddy \ddz
		= 0, \\
		\py &= e a \iiint y \,\del(x) \,\del(y) \,\del[ z - \cos(\omg t) ] \ddx \ddy \ddz
		= 0, \\
		\pz &= e a \iiint z \,\del(x) \,\del(y) \,\del[ z - \cos(\omg t) ] \ddx \ddy \ddz
		= e a \cos(\omg t).
	}
	In complex notation, $\vp = e a \,e^{-i\omg t} \,\zh$.  Substituting into Eq.~\refeq{ans1a}, we find
	\eq{
		\dv{\vL}{t} = \Re\!\brac{ \frac{i k^3}{12\pi \epso} e^2 a^2 e^{-i \omg t} e^{i \omg t} (\zh \cross \zh) }
		= \ans{ \vO. }
	}
	So we see that a charge undergoing linear motion does not lead to a radiated angular momentum, which is sensible.
}

%
%	Jackson 9.8(d)
%

\prob{}{
	What are the results corresponding to Probs.~{1(a)} and {1(b)} for magnetic dipole radiation?
}

\sol{
	The radiation fields for a magnetic dipole are given by Jackson~(19.35--36),
	\al{
		\vH &= \frac{1}{4\pi} \curly{ k^2 (\nh \cross \vm) \cross \nh \frac{e^{i k r}}{r} + [ 3 \nh (\nh \vdot \vm) - \vm ] \paren{ \frac{1}{r^3} - \frac{i k}{r^2} } e^{i k r} }, &
		\vE &= -\frac{\Zo}{4\pi} k^2 (\nh \cross \vm) \frac{e^{i k r}}{r} \paren{ 1 - \frac{1}{i k r} }.
	}
	\clearpage
	Comparing with Eq.~\refeq{fields1}, we see that $\vH \to -\vE / \Zo$, $\vE \to \Zo \vH$, and $\vp \to \vm / c$ as stated in the book~\cite[p.~413]{Jackson}.  Making these substitutions, the results of Probs.~{1.1(a)} and {(b)} become
	\al{
		\ans{ \dv{\vL}{t}\ }&\ans{= \frac{\muo k^3}{12\pi} \Im[ \vms \cross \vm ], } &
		\ans{ \frac{\dv*{\vL}{t}}{\dv*{E}{t}}\ }&\ans{= \frac{\Im[ \vms \cross \vm ]}{\omg \abs{\vm}^2} }
	}
	where we have used $\mu = 1 / \epso c^2$.
}






\state{Lorentz group (Peskin \& Schroeder 3.1)}{
	Recall from Eq.~(3.17) the Lorentz commutation relations,
	\eq{
		[\Jmn, \Jrs] = i (\gnr \Jms - \gmr \Jns - \gns \Jmr + \gms \Jnr).
	}
}

\prob{}{	\label{3.1a}
	Define the generators of rotations and boosts as
	\al{
		\Li &= \frac{1}{2} \epsijk \Jjk, &
		\Ki &= \Joi,
	}
	where $i, j, k = 1, 2, 3$.  An infinitesimal Lorentz transformation can then be written
	\eqn{trans}{
		\Phi \to (1 - i \vtht \vdot \vL - i \vbet \vdot \vK) \Phi.
	}
	Write the commutation relations of these vector operators explicitly.  (For example, $[\Li, \Lj] = i \epsijk \Lk$.)  Show that the combinations
	\al{
		\vJp &= \frac{1}{2} (\vL + i \vK), &
		\vJm &= \frac{1}{2} (\vL - i \vK)
	}
	commute with one another and separately satisfy the commutation relations of angular momentum.
}

\sol{
	Firstly, using Eq.~(3.18),
	\al{
		[\Li, \Lj] &= \brac{ \frac{1}{2} \epsimn \Jmn, \frac{1}{2} \epsjrs \Jrs }
		= \frac{1}{4} \epsimn \epsjrs [\Jmn, \Jrs]
		= \frac{i}{4} \epsimn \epsjrs (\gnr \Jms - \gmr \Jns - \gns \Jmr + \gms \Jnr) \\
		&= \frac{i}{4} (\epsimn \epsjrs \gnr \Jms - \epsimn \epsjrs \gmr \Jns - \epsimn \epsjrs \gns \Jmr + \epsimn \epsjrs \gms \Jnr) \\
		&= \frac{i}{4} (\epsimn \epsjrs \gnr \Jms - \epsinm \epsjrs \gnr \Jms - \epsimn \epsjsr \gnr \Jms + \epsinm \epsjsr \gnr \Jms) \\
		&= \frac{i}{4} (\epsimn \epsjrs \gnr \Jms + \epsimn \epsjrs \gnr \Jms + \epsimn \epsjrs \gnr \Jms + \epsimn \epsjrs \gnr \Jms) \\
		&= i \epsimn \epsjrs \gnr \Jms,
	}
	where we have simply relabeled indices.  Then, using $\gij = -\delij$~\cite{Metric} and $\epsijk \epspqk = \delip \deljq - \deliq \deljp$~\cite{LeviCivita},
	\al{
		[\Li, \Lj] &= -i \epsimn \epsjrs \delnr \Jms
		= -i \epsimn \epsjns \Jms
		= i \epsimn \epsjsn \Jms
		= i (\delij \delms - \delis \delmj) \Jms
		= i (\delij \Jmm - \delis \Jjs) \\
		&= -i \Jji
		= i \Jij,
	}
	where we have used the antisymmetry of $\Jij$.  From $\Li = \frac{1}{2} \epsijk \Jjk$, we can write
	\eq{
		\epsirs \Li = \frac{1}{2} \epsijk \epsirs \Jjk
		= \frac{1}{2} (\deljr \delks - \deljs \delkr) \Jjk
		= \frac{1}{2} (\deljr \Jjs - \deljs \Jjr)
		=  \frac{1}{2} (\Jrs - \Jsr)
		= \Jrs.
	}
	Then we see that
	\eq{
		\ans{ [\Li, \Lj] = i \epsijk \Lk. }
	}
	
	Secondly,
	\eq{
		[\Ki, \Kj] = [\Joi, \Joj]
		= i (\gio \Joj - \goo \Jij - \gij \Joo + \goj \Jio)
		= -i \Jij
		= \ans{ -i \epsijk \Lk, }
	}
	and thirdly,
	\al{
		[\Ki, \Lj] &= \brac{ \Joi, \frac{1}{2} \epsjrs \Jrs }
		= \frac{1}{2} \epsjrs [\Joi, \Jrs]
		= \frac{i}{2} \epsjrs (\gir \Jos - \gor \Jis - \gis \Jor + \gos \Jir) \\
		&= \frac{i}{2} (\epsjrs \gir \Jos - \epsjrs \gis \Jor)
		= \frac{i}{2} (\epsjrs \gir \Jos - \epsjsr \gir \Jos)
		= i \epsjrs \gir \Jos
		= -i \epsjrs \delir \Jos
%		= -i \epsjis \Jos
		= i \epsijs \Jos \\
		&= \ans{ i \epsijk \Kk. }
	}
	
	Next we want to show that $[\vJp, \vJm] = 0$.  Note that
	\al{
		[\Jpi, \Jmj] &= \brac{ \frac{1}{2} (\Li + i \Ki), \frac{1}{2} (\Lj - i \Kj) }
%		= \frac{1}{4} [ \Li + i \Ki, \Lj - i \Kj ]
%		= \frac{1}{4} \paren{ [\Li, \Lj] + [\Li, -i \Kj] + [i \Ki, \Lj] + [i \Ki, -i \Kj] } \\
		= \frac{1}{4} \paren{ [\Li, \Lj] - i [\Li, \Kj] + i [\Ki, \Lj] + [\Ki, \Kj] } \\
		&= \frac{1}{4} \paren{ i \epsijk \Lk - \epsjik \Kk - \epsijk \Kk - i \epsijk \Lk } \\
		&= \ans{ 0, }
	}
	so $[\vJp, \vJm] = 0$. \qed
	
	The angular momentum commutation relations are given by Peskin \& Schroeder Eq.~(3.12): $[\Ji, \Jj] = i \epsijk \Jk$.  We have
	\al{
		[\Jpmi, \Jpmj] &= \brac{ \frac{1}{2} (\Li \pm i \Ki), \frac{1}{2} (\Lj \pm i \Kj) }
%		= \frac{1}{4} [\Li \pm i \Ki, \Lj \pm i \Kj] \\
%		&= \frac{1}{4} \paren{ [\Li, \Lj] + [\Li, \pm i \Kj] + [\pm i \Ki, \Lj] + [\pm i \Ki, \pm i \Kj] } \\
		= \frac{1}{4} \paren{ [\Li, \Lj] \pm i [\Li, \Kj] \pm i [\Ki, \Lj] - [\Ki, \Kj] } \\
		&= \frac{1}{4} \paren{ i \epsijk \Lk \pm \epsjik \Kk \mp \epsijk \Kk + i \epsijk \Lk }
		= \frac{1}{2} \paren{ i \epsijk \Lk \mp \epsijk \Kk }
		= \frac{1}{2} i \epsijk \paren{ \Lk \pm i  \Kk } \\
		&= \ans{ i \epsijk \Jpmk, }
	}
	as desired. \qed
}



\prob{}{
	The finite-dimensional representations of the rotation group correspond precisely to the allowed values for angular momentum: integers or half-integers.  The result of \ref{3.1a} implies that all finite-dimensional representations of the Lorentz group correspond to pairs of integers or half integers, $(\jpp, \jmm)$, corresponding to pairs of representations of the rotation group.  Using the fact that $\vJ = \vsig / 2$ in the spin-$1/2$ representation of angular momentum, write explicitly the transformation laws of the 2-component objects transforming according to the $(\frac{1}{2}, 0)$ and $(0, \frac{1}{2})$ representations of the Lorentz group.  Show that these correspond precisely to the transformations of $\psiL$ and $\psiR$ given in (3.37).
}

\sol{
	Equation (3.37) is
	\al{
		\psiL &\to \paren{ 1 - i \vtht \vdot \frac{\vsig}{2} - \vbet \vdot \frac{\vsig}{2} } \psiL, &
		\psiR &\to \paren{ 1 - i \vtht \vdot \frac{\vsig}{2} + \vbet \vdot \frac{\vsig}{2} } \psiR,
	}
	where $\psiL$ and $\psiR$ are the left- and right-handed Weyl spinors, respectively.
	
	We can rewrite Eq.~\refeq{trans} in terms of $\vJp$ and $\vJm$:
	\eq{
		\Phi \to [ 1 - i \vtht \vdot (\vJp + \vJm) - \vbet \vdot (\vJp - \vJm) ] \Phi
		= [ 1 - (i \vtht + \vbet) \vdot \vJp + (i \vtht - \vbet) \vdot \vJm) ] \Phi.
	}
	From the final expression, we associate $\vJp$ and $\vJm$ with $\vsig / 2$ in turn, with $\vJp = \vsig / 2$ corresponding to the $(\frac{1}{2}, 0)$ representation and $\vJm = \vsig / 2$ corresponding to the $(0, \frac{1}{2})$ representation.  The transformation laws are
	\eq{
		\Phi \to \begin{cases}
			\paren{ 1 - i \vtht \vdot \dfrac{\vsig}{2} - \vbet \vdot \dfrac{\vsig}{2} } \Phi & \text{$(\frac{1}{2}, 0)$ representation}, \\[2ex]
			\paren{ 1 - i \vtht \vdot \dfrac{\vsig}{2} + \vbet \vdot \dfrac{\vsig}{2} } \Phi & \text{$(0, \frac{1}{2})$ representation}.
		\end{cases}
	}
	Comparing to Eq.~(3.37), we see that $\Phi$ transforms as $\psiL$ under the $(\frac{1}{2}, 0)$ representation and as $\psiR$ under the $(0, \frac{1}{2})$ representation. \qed
}



\prob{}{
	The identity $\vsigT = -\sig^2 \vsig \sig^2$ allows us to rewrite the $\psiL$ transformations in the unitarily equivalent form
	\eq{
		\psi' \to \psi' \paren{ 1 + i \vtht \vdot \frac{\vsig}{2} + \vbet \vdot \frac{\vsig}{2} },
	}
	where $\psi' = \psiLT \sig^2$.  Using this law, we can represent the object that transforms as $(\frac{1}{2}, \frac{1}{2})$ as a $2 \times 2$ matrix that has the $\psiR$ transformations law on the left and, simultaneously, the transposed $\psiL$ transformation on the right.  Parametrize this matrix as
	\eq{
		\mqty(
			\Vo + \Ve & \Vq - i \Ve \\
			\Vq + i \Vw & \Vo - \Ve
		).
	}
	Show that the object $\Vm$ transforms as a 4-vector.
}

\sol{
	Firstly, note that
	\al{
		\Vsm \sigm = \mqty( \Vo & 0 \\ 0 & \Vo ) - \mqty( 0 & \Vq \\ \Vq & 0 ) - \mqty( 0 & -i \Vw \\ i \Vw & 0 ) - \mqty( \Ve & 0 \\ 0 & -\Ve )
		= \mqty(
			\Vo + \Ve & \Vq - i \Ve \\
			\Vq + i \Vw & \Vo - \Ve
		).
	}
	Then the transformation is
	\al{
		\Vsm \sigm &\to \paren{ 1 - i \vtht \vdot \frac{\vsig}{2} + \vbet \vdot \frac{\vsig}{2} } \Vsm \sigm \paren{ 1 + i \vtht \vdot \frac{\vsig}{2} + \vbet \vdot \frac{\vsig}{2} }
		= \paren{ 1 + (\vbet - i \vtht) \vdot \frac{\vsig}{2} } \Vsm \sigm \paren{ 1 + (\vbet + i \vtht) \vdot \frac{\vsig}{2} } \\
		&= \Vsm \sigm + \Vsm \sigm (\vbet + i \vtht) \vdot \frac{\vsig}{2} + (\vbet - i \vtht) \vdot \frac{\vsig}{2} \Vsm \sigm,
	}
	where we note that $\tht$ and $\bet$ are infinitesimal angles and drop terms of $\order{\tht^2} = \order{\bet^2} = \order{\tht \bet}$.  Then
	\al{
		\Vsm \sigm &\to \Vsm \sigm + \frac{1}{2} \Vsm \sigm (\betsi \sigi + i \thtsj \sigj) + \frac{1}{2} \Vsm (\betsk \sigk - i \thtsl \sigl) \sigm \\
		&= \Vsm \sigm + \frac{1}{2} \Vsm \paren{ \betsi \sigm \sigi + i \thtsj \sigm \sigj + \betsk \sigk \sigm - i \thtsl \sigl \sigm } \\
		&= \Vsm \sigm + \frac{1}{2} \Vsm \brac{ \betsi (\sigm \sigi + \sigi \sigm) + i \thtsj (\sigm \sigj - \sigj \sigm) }
		= \Vsm \sigm + \frac{1}{2} \Vsm \paren{ \betsi \{ \sigm, \sigi \} + i \thtsj [\sigm, \sigj] } \\
		&= \Vsm \sigm + \Vsm \paren{ \betsi \delmi + i \thtsj \epsmij \sigj }
		= \Vsm \sigm + \Vsm \betm + i \Vsm \thtsj \epsmij \sigj
	}
	where we have used $\{ \sigi, \sigj \} = 2 \delij$ and $[\sigi, \sigj] = 2i \epsijk \sigk$~\cite[p.~165]{Sakurai}.
	
	Peskin \& Schroeder~(3.19) shows an infinitesimal Lorentz transformation:
	\eq{
		\Va \to \brac{ \delasb - \frac{i}{2} \omgsmn \cJmnasb } \Vb,
	}
	where $V$ is a 4-vector, $\omgsmn$ is an antisymmetric tensor that gives the infinitesimal angles, and $\cJmnsasb$ is defined in Peskin \& Schroeder~(3.18): $\cJmnsasb = i (\delmsa \delnsb - \delmsb \delnsa)$.
}




%\state{Majorana fermions (Peskin \& Schroeder 3.4)}{
%	Recall from Eq.~(3.40) that one can write a relativistic equation for a massless 2-component fermion field that transforms as the upper two components of a Dirac spinor ($\psiL$).  Call such a 2-component field $\chiax$, $a = 1, 2$.
%}

%\prob{}{	\label{3.4a}
%	Show that it is possible to write an equation for $\chix$ as a massive field in the following way:
%	\eq{
%		i \sigb \cdot \partial \chi - i m \sig^2 \chis = 0.
%	}
%	That is, show, first, that this equation is relativistically invariant and, second, that it implies the Klein-Gordon equation, $(\partial^2 + m^2) \chi = 0$.  This form of the fermion mass is called a Majorana mass term.
%}



%\prob{}{	\label{3.4b}
%	Does the Majorana equation follow from a Lagrangian?  The mass term would seem to be the variation of $\sigsab \chias \chibs$; however, since $\sig^2$ is antisymmetric, this expression would vanish if $\chix$ were an ordinary c-number field.  When we go to quantum field theory, we know that $\chix$ will become an anticommuting quantum field.  Therefore, it makes sense to develop its classical theory by considering $\chix$ as a classical anticommuting field, that is, as a field that takes as values \emph{Grassmann numbers} which satisfy
%	\al{
%		\alp \bet &= -\bet \alp, &
%		\text{for any } \alp, \bet.
%	}
%	Note that this relation implies that $\alp^2 = 0$.  A Grassmann field $\xix$ can be expanded in a basis of functions as
%	\eq{
%		\xix = \sumn \alpn \phinx,
%	}
%	where the $\phinx$ are orthogonal c-number functions and the $\alpn$ are a set of independent Grassmann numbers.  Define the complex conjugate of a profuct of Grassmann numbers to reverse the order:
%	\eq{
%		(\alp \bet)^* = \bets \alps
%		= -\alps \bets.
%	}
%	This rule imitates the Hermitian conjugation of quantum fields.  Show that the classical action,
%	\eq{
%		S = \int \ddqx \brac{ \chidag i \sigb \cdot \partial \chi + \frac{i m}{2} \paren{ \chiT \sig^2 \chi - \chidag \sig^2 \chis } },
%	}
%	(where $\chidag = (\chis)^T$) is real ($\Ss = S$), and that varying this $S$ with respect to $\chi$ and $\chis$ yields the Majorana equation.
%}



%\prob{}{	\label{3.4c}
%	Let us write a 4-component Dirac field as
%	\eq{
%		\psix = \mqty( \psiL \\ \psiR ),
%	}
%	and recall that the lower components of $\psi$ transform in a way equivalent by a unitary transformation to the complex conjugate of the representation $\psiL$.  In this way, we can rewrite the 4-component Dirac field in terms of two 2-component spinors:
%	\al{
%		\psiLx &= \chiqx, &
%		\psiRx &= i \sig^2 \chiwsx.
%	}
%	Rewrite the Dirac Lagrangian in terms of $\chiq$ and $\chiw$ and note the form of the mass term.
%}



%\prob{}{
%	Show that the action of \ref{3.4c} has a global symmetry.  Compute the divergences of the currents
%	\al{
%		\Jm &= \chidag \sigbm \chi, &
%		\Jm &= \chiqdag \sigbm \chiq - \chiwdag \sigbm \chiw,
%	}
%	for the theories of \ref{3.4b} and \ref{3.4c}, respectively, and relate your results to the symmetries of those theories.  Construct a theory of $N$ free massive 2-component fermion fields with $\order{N}$ symmetry (that is, the symmetry of rotations in an $N$-dimensional space).
%}



%\prob{}{
%	Quantize the Majorana theory of \ref{3.4a} and \ref{3.4b}.  That is, promote $\chix$ to a quantum field satisfying the canonical anticommutation relation
%	\eq{
%		\{ \chiavx, \chibdagvy \} = \delsab \del^3(\vx - \vy),
%	}
%	construct a Hermitian Hamiltonian, and find a representation of the canonical anticommutation relations that diagonalizes the Hamiltonian in terms of a set of creation and annihilation operators.  (Hint: Compare $\chix$ to the top two components of the quantized Dirac field.)
%}






%\state{(Peskin \& Schroeder 3.7)}{
%	This problem concerns the discrete symmetries $P$, $C$, and $T$.
%}



%\prob{}{
%	Compute the transformation properties under $P$, $C$, and $T$ of the antisymmetric tensor fermion bilinears, $\psib \sigmn \psi$, with $\sigmn = \frac{i}{2} [\gamm, \gamn]$.  This completes the table of the transformation properties of bilinears at the end of the chapter.
%}



%\prob{}{
%	Let $\phix$ be a complex-valued Klein-Gordon field, such as we considered in Problem~2.2.  Find unitary operators $P$, $C$, and an antiunitary operator $T$ (all defined in terms of their action on the annihilation operators $\ap$ and $\bp$ for the Klein-Gordon particles and antiparticles) that give the following transformations of the Klein-Gordon field:
%	\al{
%		P \phitx P &= \phitmx, &
%		T \phitx T &= \phimtx, &
%		C \phitx C &= \phistx.
%	}
%	Find the transformation properties of the components of the current
%	\eq{
%		\Jm = i (\phis \ptm \phi - \ptm \phis \phi)
%	}
%	under $P$, $C$, and $T$.
%}



%\prob{}{
%	Show that any Hermitian Lorentz-scalar local operator built from $\psix$, $\phix$, and their conjugates has $C P T = +1$.
%}





\clearpage
\makebib

\end{document}
