\state{Lorentz group (Peskin \& Schroeder 3.1)}{
	Recall from Eq.~(3.17) the Lorentz commutation relations,
	\eq{
		[\Jmn, \Jrs] = i (\gnr \Jms - \gmr \Jns - \gns \Jmr + \gms \Jnr).
	}
}

\prob{}{	\label{3.1a}
	Define the generators of rotations and boosts as
	\al{
		\Li &= \frac{1}{2} \epsijk \Jjk, &
		\Ki &= \Joi,
	}
	where $i, j, k = 1, 2, 3$.  An infinitesimal Lorentz transformation can then be written
	\eqn{trans}{
		\Phi \to (1 - i \vtht \vdot \vL - i \vbet \vdot \vK) \Phi.
	}
	Write the commutation relations of these vector operators explicitly.  (For example, $[\Li, \Lj] = i \epsijk \Lk$.)  Show that the combinations
	\al{
		\vJp &= \frac{1}{2} (\vL + i \vK), &
		\vJm &= \frac{1}{2} (\vL - i \vK)
	}
	commute with one another and separately satisfy the commutation relations of angular momentum.
}

\sol{
	Firstly, using Eq.~(3.18),
	\al{
		[\Li, \Lj] &= \brac{ \frac{1}{2} \epsimn \Jmn, \frac{1}{2} \epsjrs \Jrs }
		= \frac{1}{4} \epsimn \epsjrs [\Jmn, \Jrs]
		= \frac{i}{4} \epsimn \epsjrs (\gnr \Jms - \gmr \Jns - \gns \Jmr + \gms \Jnr) \\
		&= \frac{i}{4} (\epsimn \epsjrs \gnr \Jms - \epsimn \epsjrs \gmr \Jns - \epsimn \epsjrs \gns \Jmr + \epsimn \epsjrs \gms \Jnr) \\
		&= \frac{i}{4} (\epsimn \epsjrs \gnr \Jms - \epsinm \epsjrs \gnr \Jms - \epsimn \epsjsr \gnr \Jms + \epsinm \epsjsr \gnr \Jms) \\
		&= \frac{i}{4} (\epsimn \epsjrs \gnr \Jms + \epsimn \epsjrs \gnr \Jms + \epsimn \epsjrs \gnr \Jms + \epsimn \epsjrs \gnr \Jms) \\
		&= i \epsimn \epsjrs \gnr \Jms,
	}
	where we have simply relabeled indices.  Then, using $\gij = -\delij$ and $\epsijk \epspqk = \delip \deljq - \deliq \deljp$~\cite{LeviCivita},
	\al{
		[\Li, \Lj] &= -i \epsimn \epsjrs \delnr \Jms
		= -i \epsimn \epsjns \Jms
		= i \epsimn \epsjsn \Jms
		= i (\delij \delms - \delis \delmj) \Jms
		= i (\delij \Jmm - \delis \Jjs) \\
		&= -i \Jji
		= i \Jij,
	}
	where we have used the antisymmetry of $\Jij$.  From $\Li = \frac{1}{2} \epsijk \Jjk$, we can write
	\eq{
		\epsirs \Li = \frac{1}{2} \epsijk \epsirs \Jjk
		= \frac{1}{2} (\deljr \delks - \deljs \delkr) \Jjk
		= \frac{1}{2} (\deljr \Jjs - \deljs \Jjr)
		=  \frac{1}{2} (\Jrs - \Jsr)
		= \Jrs.
	}
	Then we see that
	\eq{
		\ans{ [\Li, \Lj] = i \epsijk \Lk. }
	}
	
	Secondly,
	\eq{
		[\Ki, \Kj] = [\Joi, \Joj]
		= i (\gio \Joj - \goo \Jij - \gij \Joo + \goj \Jio)
		= -i \Jij
		= \ans{ -i \epsijk \Lk, }
	}
	and thirdly,
	\al{
		[\Ki, \Lj] &= \brac{ \Joi, \frac{1}{2} \epsjrs \Jrs }
		= \frac{1}{2} \epsjrs [\Joi, \Jrs]
		= \frac{i}{2} \epsjrs (\gir \Jos - \gor \Jis - \gis \Jor + \gos \Jir) \\
		&= \frac{i}{2} (\epsjrs \gir \Jos - \epsjrs \gis \Jor)
		= \frac{i}{2} (\epsjrs \gir \Jos - \epsjsr \gir \Jos)
		= i \epsjrs \gir \Jos
		= -i \epsjrs \delir \Jos
%		= -i \epsjis \Jos
		= i \epsijs \Jos \\
		&= \ans{ i \epsijk \Kk. }
	}
	
	Next we want to show that $[\vJp, \vJm] = 0$.  Note that
	\al{
		[\Jpi, \Jmj] &= \brac{ \frac{1}{2} (\Li + i \Ki), \frac{1}{2} (\Lj - i \Kj) }
%		= \frac{1}{4} [ \Li + i \Ki, \Lj - i \Kj ]
%		= \frac{1}{4} \paren{ [\Li, \Lj] + [\Li, -i \Kj] + [i \Ki, \Lj] + [i \Ki, -i \Kj] } \\
		= \frac{1}{4} \paren{ [\Li, \Lj] - i [\Li, \Kj] + i [\Ki, \Lj] + [\Ki, \Kj] } \\
		&= \frac{1}{4} \paren{ i \epsijk \Lk - \epsjik \Kk - \epsijk \Kk - i \epsijk \Lk } \\
		&= \ans{ 0, }
	}
	so $[\vJp, \vJm] = 0$. \qed
	
	The angular momentum commutation relations are given by Peskin \& Schroeder Eq.~(3.12): $[\Ji, \Jj] = i \epsijk \Jk$.  We have
	\al{
		[\Jpmi, \Jpmj] &= \brac{ \frac{1}{2} (\Li \pm i \Ki), \frac{1}{2} (\Lj \pm i \Kj) }
%		= \frac{1}{4} [\Li \pm i \Ki, \Lj \pm i \Kj] \\
%		&= \frac{1}{4} \paren{ [\Li, \Lj] + [\Li, \pm i \Kj] + [\pm i \Ki, \Lj] + [\pm i \Ki, \pm i \Kj] } \\
		= \frac{1}{4} \paren{ [\Li, \Lj] \pm i [\Li, \Kj] \pm i [\Ki, \Lj] - [\Ki, \Kj] } \\
		&= \frac{1}{4} \paren{ i \epsijk \Lk \pm \epsjik \Kk \mp \epsijk \Kk + i \epsijk \Lk }
		= \frac{1}{2} \paren{ i \epsijk \Lk \mp \epsijk \Kk }
		= \frac{1}{2} i \epsijk \paren{ \Lk \pm i  \Kk } \\
		&= \ans{ i \epsijk \Jpmk, }
	}
	as desired. \qed
}



\prob{}{
	The finite-dimensional representations of the rotation group correspond precisely to the allowed values for angular momentum: integers or half-integers.  The result of \ref{3.1a} implies that all finite-dimensional representations of the Lorentz group correspond to pairs of integers or half integers, $(\jpp, \jmm)$, corresponding to pairs of representations of the rotation group.  Using the fact that $\vJ = \vsig / 2$ in the spin-$1/2$ representation of angular momentum, write explicitly the transformation laws of the 2-component objects transforming according to the $(\frac{1}{2}, 0)$ and $(0, \frac{1}{2})$ representations of the Lorentz group.  Show that these correspond precisely to the transformations of $\psiL$ and $\psiR$ given in (3.37).
}

\sol{
	Equation (3.37) is
	\al{
		\psiL &\to \paren{ 1 - i \vtht \vdot \frac{\vsig}{2} - \vbet \vdot \frac{\vsig}{2} } \psiL, &
		\psiR &\to \paren{ 1 - i \vtht \vdot \frac{\vsig}{2} + \vbet \vdot \frac{\vsig}{2} } \psiR,
	}
	where $\psiL$ and $\psiR$ are the left- and right-handed Weyl spinors, respectively.
	
	We can rewrite Eq.~\refeq{trans} in terms of $\vJp$ and $\vJm$:
	\eq{
		\Phi \to [ 1 - i \vtht \vdot (\vJp + \vJm) - \vbet \vdot (\vJp - \vJm) ] \Phi
		= [ 1 - (i \vtht + \vbet) \vdot \vJp + (i \vtht - \vbet) \vdot \vJm) ] \Phi.
	}
	From the final expression, we associate $\vJp$ and $\vJm$ with $\vsig / 2$ in turn, with $\vJp = \vsig / 2$ corresponding to the $(\frac{1}{2}, 0)$ representation and $\vJm = \vsig / 2$ corresponding to the $(0, \frac{1}{2})$ representation.  The transformation laws are
	\eq{
		\Phi \to \begin{cases}
			\paren{ 1 - i \vtht \vdot \dfrac{\vsig}{2} - \vbet \vdot \dfrac{\vsig}{2} } \Phi & \text{$(\frac{1}{2}, 0)$ representation}, \\[2ex]
			\paren{ 1 - i \vtht \vdot \dfrac{\vsig}{2} + \vbet \vdot \dfrac{\vsig}{2} } \Phi & \text{$(0, \frac{1}{2})$ representation}.
		\end{cases}
	}
	Comparing to Eq.~(3.37), we see that $\Phi$ transforms as $\psiL$ under the $(\frac{1}{2}, 0)$ representation and as $\psiR$ under the $(0, \frac{1}{2})$ representation. \qed
}



\prob{}{
	The identity $\vsigT = -\sig^2 \vsig \sig^2$ allows us to rewrite the $\psiL$ transformations in the unitarily equivalent form
	\eq{
		\psi' \to \psi' \paren{ 1 + i \vtht \vdot \frac{\vsig}{2} + \vbet \vdot \frac{\vsig}{2} },
	}
	where $\psi' = \psiLT \sig^2$.  Using this law, we can represent the object that transforms as $(\frac{1}{2}, \frac{1}{2})$ as a $2 \times 2$ matrix that has the $\psiR$ transformations law on the left and, simultaneously, the transposed $\psiL$ transformation on the right.  Parametrize this matrix as
	\eq{
		\mqty(
			\Vo + \Ve & \Vq - i \Ve \\
			\Vq + i \Vw & \Vo - \Ve
		).
	}
	Show that the object $\Vm$ transforms as a 4-vector.
}

\sol{
	Peskin \& Schroeder~(3.19) shows an infinitesimal Lorentz transformation:
	\eq{
		\Va \to \brac{ \delasb - \frac{i}{2} \omgsmn \cJmnasb } \Vb,
	}
	where $V$ is a 4-vector, $\omgsmn$ is an antisymmetric tensor that gives the infinitesimal angles, and $\cJmnsasb = i (\delmsa \delnsb - \delmsb \delnsa)$ from Peskin \& Schroeder~(3.18).  Using this definition, the transformation is
	\aln{
		\Va &\to \brac{ \delasb + \frac{1}{2} \omgsmn (\delma \delnsb - \delmsb \delna) } \Vb
		= \brac{ \delasb + \frac{1}{2} \omgsmn \gsbg (\delma \delng - \delmg \delna) } \Vb \notag \\
		&= \brac{ \delasb + \frac{1}{2} \gsbg (\omgag - \omgga) } \Vb
		= (\delasb + \gsbg \omgag) \Vb \notag \\
		&= (\delasb + \omgasb) \Vb, \label{thing2c}
	}
	where we have used the antisymmetry of $\omgmn$.

	For the problem at hand, note that
	\al{
		\Vsm \sigm = \mqty( \Vo & 0 \\ 0 & \Vo ) - \mqty( 0 & \Vq \\ \Vq & 0 ) - \mqty( 0 & -i \Vw \\ i \Vw & 0 ) - \mqty( \Ve & 0 \\ 0 & -\Ve )
		= \mqty(
			\Vo + \Ve & \Vq - i \Ve \\
			\Vq + i \Vw & \Vo - \Ve
		).
	}
	Then the transformation is
	\al{
		\Vsm \sigm &\to \paren{ 1 - i \vtht \vdot \frac{\vsig}{2} + \vbet \vdot \frac{\vsig}{2} } \Vsm \sigm \paren{ 1 + i \vtht \vdot \frac{\vsig}{2} + \vbet \vdot \frac{\vsig}{2} }
		= \paren{ 1 + (\vbet - i \vtht) \vdot \frac{\vsig}{2} } \Vsm \sigm \paren{ 1 + (\vbet + i \vtht) \vdot \frac{\vsig}{2} } \\
		&= \Vsm \sigm + \Vsm \sigm (\vbet + i \vtht) \vdot \frac{\vsig}{2} + (\vbet - i \vtht) \vdot \frac{\vsig}{2} \Vsm \sigm,
	}
	where we note that $\tht$ and $\bet$ are infinitesimal angles and drop terms of $\order{\tht^2} = \order{\bet^2} = \order{\tht \bet}$.  Then
	\al{
		\Vsm \sigm &\to \Vsm \sigm + \frac{1}{2} \Vsm \sigm (\betsi \sigi + i \thtsj \sigj) + \frac{1}{2} \Vsm (\betsk \sigk - i \thtsl \sigl) \sigm \\
		&= \Vsm \sigm + \frac{1}{2} \Vsm \paren{ \betsi \sigm \sigi + i \thtsj \sigm \sigj + \betsk \sigk \sigm - i \thtsl \sigl \sigm } \\
		&= \Vsm \sigm + \frac{1}{2} \Vsm \brac{ \betsi (\sigm \sigi + \sigi \sigm) + i \thtsj (\sigm \sigj - \sigj \sigm) }
		= \Vsm \sigm + \frac{1}{2} \Vsm \paren{ \betsi \{ \sigm, \sigi \} + i \thtsj [\sigm, \sigj] } \\
		&= \Vsm \sigm + \frac{1}{2} \Vso \paren{ \betsi \{ \sigo, \sigi \} + i \thtsj [\sigo, \sigj] } + \frac{1}{2} \Vsk \paren{ \betsi \{ \sigk, \sigi \} + i \thtsj [\sigk, \sigj] } \\
		&= \Vsm \sigm + \betsi \Vso \sigi + \Vsk \paren{ \betsi \delik - \thtsj \epskji \sigi}
		= \Vsm \sigm + \betsi \Vso \sigi + \Vsi \betsi + \Vsk \thtsj \epsijk \sigi \\
		&= \Vsm \sigm + \Vso \beti \sigsi - \Vsi \beti - \Vsj \epsijk \thti \sigk
	}
	where we have used $\{ \sigi, \sigj \} = 2 \delij$ and $[\sigi, \sigj] = 2i \epsijk \sigk$~\cite[p.~165]{Sakurai}, as well as $\{ \sigo, \sigi \} = 2 \sigi$ and ${[\sigo, \sigi] = 0}$.
	
	Referring to Eq.~(3.19), we define
	\al{
		\omgoj &= \betj, &
		\omgij &= \epsijk \thtk.
	}
	Then we have
	\al{
		\Vsm \sigm &\to \Vsm \sigm + \Vso \omgoi \sigsi - \Vsi \omgoi - \Vsj \omgij \sigj
		= \Vsm \sigm + \Vo \omgsoi \sigi - \Vi \omgsoi \sigo + \Vj \omgsij \sigj \\
		&= \Vsm \sigm + \Vo \omgsom \sigm + \Vi \omgsio \sigo + \Vj \omgsij \sigj
		= \Vsm \sigm + \Vn \omgsnm \sigm
		= (\delnsm + \omgnsm) \Vsn \sigm
	}
	or
	\eq{
		\Va \sigsa \to (\delasb + \omgasb) \Vb \sigsa
		\qimplies
		\ans{ \Va \to (\delasb + \omgasb) \Vb, }
	}
	which is identical to Eq.~\refeq{thing2c}.  Therefore, we have shown that $\Vm$ transforms as a 4-vector. \qed
}