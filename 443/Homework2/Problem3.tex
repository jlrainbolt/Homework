\state{Majorana fermions (Peskin \& Schroeder 3.4)}{
	Recall from Eq.~(3.40) that one can write a relativistic equation for a massless 2-component fermion field that transforms as the upper two components of a Dirac spinor ($\psiL$).  Call such a 2-component field $\chiax$, $a = 1, 2$.
}

\prob{}{	\label{3.4a}
	Show that it is possible to write an equation for $\chix$ as a massive field in the following way:
	\eqn{given3a}{
		i \sigb \cdot \partial \chi - i m \sigw \chis = 0.
	}
	That is, show, first, that this equation is relativistically invariant and, second, that it implies the Klein-Gordon equation, $(\partial^2 + m^2) \chi = 0$.  This form of the fermion mass is called a Majorana mass term.
}

\sol{
	Using the result of \ref{3.1b}, let
	\al{
		\Lamho &= \exp( -i \vtht \vdot \dfrac{\vsig}{2} - \vbet \vdot \dfrac{\vsig}{2} ), &
		\Lamoh &= \exp( -i \vtht \vdot \dfrac{\vsig}{2} + \vbet \vdot \dfrac{\vsig}{2} ).
	}
	We are told that $\chi$ transforms as $\psiL$, so we know $\chi \to \Lamho \chi$.  So
	\al{
		 \sigw \chis &\to \sigw ( \Lamho \chi )^* \\
		&= \sigw \brac{ \exp( -i \vtht \vdot \dfrac{\vsig}{2} - \vbet \vdot \dfrac{\vsig}{2} ) \chi }^*
		= \sigw \brac{ \exp( i \vtht \vdot \dfrac{\vsigs}{2} - \vbet \vdot \dfrac{\vsigs}{2} ) } \chis
		= \brac{ \exp( -i \vtht \vdot \dfrac{\vsig}{2} + \vbet \vdot \dfrac{\vsig}{2} ) } \sigw \chis \\
		&= \Lamoh \sigw \chis,
	}
	where we have used Peskin \& Schroeder~(3.38), $ \sigw \vsigs = - \vsig \sigw$.
	
	We also need to know how to manipulate $\sigb$.  From Eqs.~(3.36) and (3.37), we can write a finite Lorentz transformation as
	\eq{
		\Lam \psi = \mqty(
			\Lamho & 0 \\
			0 & \Lamoh
		)	\mqty(
			\psiL \\
			\psiR
		)
		= \mqty(
			\Lamho \psiL \\
			\Lamoh \psiR
		).
	}
	Then using Peskin \& Schroeder~(3.42),
	\eq{
		\gamm = \mqty(
			0 & \sigm \\
			\sigbm & 0
		),
	}
	we can rewrite their Eq.~(3.29), $\Lammsn \gamn = \Lamhi \gamm \Lamh$, in matrix form.  The left side is
	\eq{
		\mqty(
			\Lamh & 0 \\
			0 & \Lamh
		) \mqty(
			0 & \sigm \\
			\sigbm & 0
		) = \mqty(
			0 & \Lammsn \sign \\
			\Lammsn \sigbn & 0
		),
	}
	and the right side is
	\eq{ \mqty(
			\Lamhoi & 0 \\
			0 & \Lamohi
		) \mqty(
			0 & \sigm \\
			\sigbm & 0
		) \mqty(
			\Lamho & 0 \\
			0 & \Lamoh
		) = \mqty(
			\Lamhoi & 0 \\
			0 & \Lamohi
		) \mqty(
			0 & \sigm \Lamho \\
			\sigbm \Lamoh & 0
		) = \mqty(
			0 & \Lamhoi \sigm \Lamoh \\
			\Lamohi \sigbm \Lamho & 0,
		)
	}
	So we have
	\al{
		\Lammsn \sigbn &= \Lamohi \sigbm \Lamho, &
		\sigbn \Laminsm &= \Lamoh \sigbm \Lamhoi.
	}

	Then Eq.~\refeq{given3a} transforms as
	\al{
		i \sigbm \ptsm \chix - i m \sigw \chisx &\to i \sigbm \Laminsm \ptsn \Lamho \chiLamix - i m \Lamoh \sigw \chisLamix \\
		&= i \Lamoh \sigbm \Lamhoi \Lamho \ptsn \chiLamix - i m \Lamoh \sigw \chisLamix \\
		&= \Lamoh \brac{ \sigbm \ptsn \chiLamix - i m \sigw \chisLamix } \\
		&= 0,
	}
	so this equation is relativistically invariant. \qed
	
	The Klein-Gordon equation does not include $\chis$, so we want to eliminate it from the Majorana equation.  From Eq.~\refeq{given3a},
	\al{
		i \sigb \cdot \partial \chi = i m \sigw \chis
		\qimplies
		\chis = \frac{ \sigw}{m} \sigb \cdot \partial \chi,
	}
	since $\sigi \sigi = 1$~\cite[p.~164]{Sakurai}.  We also need a $\partial^2$ term in the Klein-Gordon equation, so we can feed this expression into the complex conjugate of the Majorana equation:
	\aln{
		0 &= -i \sigbs \cdot \partial \chis + i m \sigstw \chi
		= -i \sigbstm \ptsm \paren{ \frac{ \sigw}{m} \sigb \cdot \partial \chi } + i m \sigstw \chi
		= -i \sigbstm \frac{ \sigw}{m} \sigbn \ptsm \ptsn \chi + i m \sigstw \chi \notag \\
		&= \sigbstm \sigw \sigbn \ptsm \ptsn \chi - m^2 \sigstw \chi. \label{ugh}
	}
	Since $\sigbm = ( \sigo, -\sigq, -\sigw, -\sige )$, then $\sigbstm = ( \sigo, -\sigq, \sigw, -\sige )$, and so
	\eqn{sigthing}{
		\sigbstm \sigw = (\sigw, -\sigq \sigw, \sigw \sigw, \sige \sigw)
		= (\sigw, \sigw \sigq, \sigw \sigw, -\sigw \sige)
		= \sigw \sigm,
	}
	where we have used $\{ \sigi, \sigj \} = 2 \delij$~\cite[p.~165]{Sakurai}.  Using this in Eq.~\refeq{ugh},
	\eqn{ugh2}{
		0 = \sigw \sigm \sigbn \ptsm \ptsn \chi - m^2 \sigstw \chi
		= \sigw \sigw \sigm \sigbn \ptsm \ptsn \chi - m^2 \sigw \sigstw \chi
		= \sigm \sigbn \ptsm \ptsn \chi + m^2 \chi
	}
	since $\sigstw = -\sigw$.  The anticommutation relation also implies that $\sigm \sigbn + \sign \sigbm = 2 \gmn$.  Note that
	\eq{
		\sigm \sigbn \ptsm \ptsn = \frac{1}{2} (\sigm \sigbn \ptsm \ptsn + \sigm \sigbn \ptsm \ptsn)
		= \frac{1}{2} (\sigm \sigbn \ptsm \ptsn + \sign \sigbm \ptsn \ptsm)
		= \frac{1}{2} (\sigm \sigbn + \sign \sigbm) \ptsm \ptsn
		= \gmn \ptsm \ptsn
		= \ptsm \ptm,
	}
	where we have just relabeled indices.  Then we have
	\eq{
		\ans{ 0 = \ptsm \ptm \chi + m^2 \chi, }
	}
	which is the Klein-Gordon equation. \qed
}



\prob{}{	\label{3.4b}
	Does the Majorana equation follow from a Lagrangian?  The mass term would seem to be the variation of $\sigsab \chias \chibs$; however, since $\sigw$ is antisymmetric, this expression would vanish if $\chix$ were an ordinary c-number field.  When we go to quantum field theory, we know that $\chix$ will become an anticommuting quantum field.  Therefore, it makes sense to develop its classical theory by considering $\chix$ as a classical anticommuting field, that is, as a field that takes as values \emph{Grassmann numbers} which satisfy
	\al{
		\alp \bet &= -\bet \alp, &
		\text{for any } \alp, \bet.
	}
	Note that this relation implies that $\alp^2 = 0$.  A Grassmann field $\xix$ can be expanded in a basis of functions as
	\eq{
		\xix = \sumn \alpn \phinx,
	}
	where the $\phinx$ are orthogonal c-number functions and the $\alpn$ are a set of independent Grassmann numbers.  Define the complex conjugate of a profuct of Grassmann numbers to reverse the order:
	\eq{
		(\alp \bet)^* = \bets \alps
		= -\alps \bets.
	}
	This rule imitates the Hermitian conjugation of quantum fields.  Show that the classical action,
	\eqn{action3.4b}{
		S = \int \ddqx \brac{ \chidag i \sigb \cdot \partial \chi + \frac{i m}{2} \paren{ \chiT \sigw \chi - \chidag \sigw \chis } },
	}
	(where $\chidag = (\chis)^T$) is real ($\Ss = S$), and that varying this $S$ with respect to $\chi$ and $\chis$ yields the Majorana equation.
}

\sol{
	To show that $S$ is real, we take its complex conjugate using the rules for Grassman numbers:
	\al{
		\Ss &= \int \ddqx \brac{ -i \paren{ \chidag \sigbm \ptsm \chi }^* - \frac{i m}{2} \paren{ \chiT \sigw \chi - \chidag \sigw \chis }^* }
		= \int \ddqx \brac{ -i \chiT \sigbstm \ptsm \chis + \frac{i m}{2} \paren{ \chidag \sigsw \chis - \chiT \sigsw \chi } } \\
		&= \int \ddqx \brac{ -i \ptsm \chidag \sigbdagm \chi - \frac{i m}{2} \paren{ \chidag \sigw \chis - \chiT \sigw \chi } }
		= \int \ddqx \brac{ -i \ptsm \chidag \sigbm \chi + \frac{i m}{2} \paren{ \chiT \sigw \chi - \chidag \sigw \chis } },
	}
	where we have transposed the factors in the first term.  This is allowed because it is a c-number, and the sum of the matrix elements is the same if the matrices are transposed.  We have also used the fact that $\sigm$ is Hermitian, since all $\sigi$ are~\cite[p.~165]{Sakurai}.  Then, integrating by parts, note that
	\eq{
		\int \ddqx \ptsm \chidag \sigbm \chi = \brac{ \chidag \sigbm \chi }\ii - \int \ddqx \chidag \ptsm \chi
		= \int \ddqx \chidag \ptsm \chi,
	}
	so we have
	\eq{
		\ans{ \Ss = \int \ddqx \brac{ i \chidag \sigbm \ptsm \chi + \frac{i m}{2} \paren{ \chiT \sigw \chi - \chidag \sigw \chis } }
		= S, }
	}
	as desired. \qed
	
	As usual, we may treat $\chi$ and $\chis$ independently.  Firstly, let $\del\chi$ be a variation of the field $\chi$ that vanishes at the boundaries, and that $\del S = S[\chi + \del\chi] - S[\chi] = 0$.  Then
	\al{
		\del S &= \int \ddqx \curly{ i \chidag \sigbm \ptsm (\chi + \del\chi) + \frac{i m}{2} \brac{ (\chiT + \del\chiT) \sigw (\chi + \del\chi) - \chidag \sigw \chis } - i \chidag \sigbm \ptsm \chi - \frac{i m}{2} \paren{ \chiT \sigw \chi - \chidag \sigw \chis } } \\
		&= \int \ddqx \brac{ i \chidag \sigbm \ptsm \del\chi + \frac{i m}{2} \paren{ \del\chiT \sigw \chi + \chiT \sigw \del\chi } }
		= \int \ddqx \paren{ i \chidag \sigbm \ptsm \del\chi + i m \chiT \sigw \del\chi } \\
		&= \int \ddqx \del\chi \paren{ -i \sigbm \ptsm \chidag + i m \chiT \sigw },
	}
	where we have integrated by parts and used the fact that $\del\chi$ vanishes at the boundary.  This gives us
	\eq{
		-i \sigbm \ptsm \chidag + i m \chiT \sigw = 0
		\implies
		0 = (-i \sigbm \ptsm \chidag + i m \chiT \sigw)^\dag
		= i \sigbm \ptsm \chi - i m \sigw \chis,
	}
	which is the Majorana equation.
	
	Secondly, let $\del\chis$ be a variation of the field $\chis$ that vanishes at the boundaries, and that $\del S = S[\chi + \del\chi] - S[\chi] = 0$.  Then
	\al{
		\del S &= \int \ddqx \curly{ i (\chidag + \del\chidag) \sigbm \ptsm \chi + \frac{i m}{2} \brac{ \chiT \sigw \chi - (\chidag + \del\chidag) \sigw (\chis + \del\chis) } - \chidag i \sigb \cdot \partial \chi - \frac{i m}{2} \paren{ \chiT \sigw \chi - \chidag \sigw \chis } } \\
		&= \int \ddqx \brac{ i \del\chidag \sigbm \ptsm \chi - \frac{i m}{2} \paren{ \del\chidag \sigw \chis + \chidag \sigw \del\chis } }
		= \int \ddqx \del\chidag \paren{ i \sigbm \ptsm \chi - i m \sigw \chis },
	}
	which implies
	\eq{
		\ans{ i \sigbm \ptsm \chi - i m \sigw \chis = 0, }
	}
	which is the Majorana equation. \qed
}



\prob{}{	\label{3.4c}
	Let us write a 4-component Dirac field as
	\eq{
		\psix = \mqty( \psiL \\ \psiR ),
	}
	and recall that the lower components of $\psi$ transform in a way equivalent by a unitary transformation to the complex conjugate of the representation $\psiL$.  In this way, we can rewrite the 4-component Dirac field in terms of two 2-component spinors:
	\al{
		\psiLx &= \chiqx, &
		\psiRx &= i \sigw \chiwsx.
	}
	Rewrite the Dirac Lagrangian in terms of $\chiq$ and $\chiw$ and note the form of the mass term.
}

\sol{
	Beginning from the Dirac Lagrangian in Peskin \& Schroeder~(3.34),
	\eq{
		\cLDirac = \psib (i \gamm \ptsm - m) \psi
		= \psidag \gamo (i \gamm \ptsm - m) \psi.
	}
	where we have used Eq.~(3.32), $\psib = \psidag \gamo$.  From the problem statement,
	\al{
		\psi &= \mqty( \chiq \\ i \sigw \chiws ), &
		\psidag &= \mqty( \chiqdag & i \sigsw \chiwT )
		= \mqty( \chiqdag & -i \sigw \chiwT ).
	}
	Additionally, from Peskin \& Schroeder~(3.25),
	\eq{
		\gamo = \mqty( 0 & 1 \\ 1 & 0 ),
	}
	and from~(3.42),
	\eq{
		0 = (i \gamm \ptsm - m) \psix
		= \mqty(
			-m & i \sig \cdot \partial \\
			i \sigb \cdot \partial & -m
		) \mqty( \psiL \\ \psiR ).
	}
	So we can write
	\al{
		\cLDirac &= \mqty( \chiqdag & -i \sigw \chiwT ) \mqty( 0 & 1 \\ 1 & 0 ) \mqty(
			-m & i \sig \cdot \partial \\
			i \sigb \cdot \partial & -m
		) \mqty( \chiq \\ i \sigw \chiws )
		= \mqty( \chiqdag & -i \sigw \chiwT ) \mqty( 0 & 1 \\ 1 & 0 ) \mqty(
			-\sigm \ptsm \sigw \chiws - m \chiq \\
			i \sigbm \ptsm \chiq - i m \sigw \chiws
		) \\
		&= \mqty( \chiqdag & -i \sigw \chiwT ) \mqty(
			i \sigbm \ptsm \chiq - i m \sigw \chiws \\
			-\sigm \ptsm \sigw \chiws - m \chiq
		)
		= i \chiqdag \sigbm \ptsm \chiq + i m \chiqdag \sigw \chiws + i \sigw \chiwT \sigm \ptsm \sigw \chiws + i m \sigw \chiwT \chiq \\
		&= i \chiqdag \sigbm \ptsm \chiq + i \chiwT \sigbstm \ptsm \chiws + i m (\chiwT \sigw \chiq - \chiqdag \sigw \chiws)
		= i \chiqdag \sigbm \ptsm \chiq + i \chiw \sigbdagm \ptsm \chiwdag + i m (\chiwT \sigw \chiq - \chiqdag \sigw \chiws) \\
		&= i \chiqdag \sigbm \ptsm \chiq - i \chiw \sigbm \ptsm \chiwdag + i m (\chiwT \sigw \chiq - \chiqdag \sigw \chiws),
	}
	where we have applied Eq.~\refeq{sigthing} and taken the transpose of the second term.  Since $\cLDirac$ is a Lagrangian density, we may integrate this term by parts to obtain
	\eq{
		\ans{ \cLDirac = i \chiqdag \sigbm \ptsm \chiq + i \chiwdag \sigbm \ptsm \chiw + i m (\chiwT \sigw \chiq - \chiqdag \sigw \chiws). }
	}
	
	The mass term has a form similar to the integrand in Eq.~\refeq{action3.4b}.  If $\chiq = \chiw = \chi$, then the mass term in the Dirac Lagrangian is twice the mass term in the Lagrangian of Eq.~\refeq{action3.4b}.  In fact, the entirety of the Dirac Lagrangian is twice that Lagrangian in the case $\chiq = \chiw = \chi$.
}



\stepcounter{subsection}
\prob{}{
	Quantize the Majorana theory of \ref{3.4a} and \ref{3.4b}.  That is, promote $\chix$ to a quantum field satisfying the canonical anticommutation relation
	\eq{
		\{ \chiavx, \chibdagvy \} = \delsab \del^3(\vx - \vy),
	}
	construct a Hermitian Hamiltonian, and find a representation of the canonical anticommutation relations that diagonalizes the Hamiltonian in terms of a set of creation and annihilation operators.  (Hint: Compare $\chix$ to the top two components of the quantized Dirac field.)
}

\sol{
	As was pointed out in the solution of \ref{3.4c}, the Majorana Lagrangian is half the Dirac Lagrangian if $\chiq = \chiw = \chi$.  Using the definition in the statement of \ref{3.4c}, this means $\psiL = \chi$.  From Peskin \& Schroeder~(3.99),
	\eq{
		\psix = \int \ddcpf \frac{1}{\sqrt{2 \Ep}} \sums \paren{ \avps \usp e^{-i p \cdot x} + \bvpsdag \vsp e^{i p \cdot x} },
	}
	and Eqs.~(3.50) and (3.62) give
	\al{
		\usp &= \mqty( \sqrt{p \cdot \sig} \xis \\ \sqrt{p \cdot \sigb} \xis ), &
		\vsp &= \mqty( \sqrt{p \cdot \sig} \etas \\ -\sqrt{p \cdot \sigb} \etas ), &
		s &= 1, 2.
	}
	From $\chi = \psiL$,
	\eq{
		\chi = \int \ddcpf \frac{1}{\sqrt{2 \Ep}} \sums \paren{ \avps \sqrt{p \cdot \sig} \xis e^{-i p \cdot x} + \bvpsdag \sqrt{p \cdot \sig} \etas e^{i p \cdot x} }
		= \int \ddcpf \sqrt{\frac{p \cdot \sig}{2 \Ep}} \sums \paren{ \avps \xis e^{-i p \cdot x} + \bvpsdag \etas e^{i p \cdot x} }.
	}
	
	The Dirac Hamiltonian is given by Eq.~(3.104),
	\eq{
		H = \int \ddcpf \sums \Ep \paren{ \avpsdag \avps + \bvpsdag \bvps }.
	}
	Since the Majorana Lagrangian is half the Dirac Lagrangian, the Majorana Hamiltonian is
	\eq{
		\ans{ H = \frac{1}{2} \int \ddcpf \sums \Ep \paren{ \avpsdag \avps + \bvpsdag \bvps }, }
	}
	which is Hermitian.
	
	Since we know $\chi$ obeys the Majorana equation, we can use it to search for a constraint on the creation and annihilation operators that will further distinguish the Majorana Hamiltonian from the Dirac Hamiltonian.  For the mass term of the Majorana equation,
	\al{
		i m \sigw \chis &= i m \sigw \chidag
		= i m \sigw \int \ddcpf \sqrt{\frac{p \cdot \sigs}{2 \Ep}} \sums \paren{ \avpsdag \xis e^{i p \cdot x} + \bvps \etass e^{i p \cdot x} } \\
		&= i m \int \ddcpf \sqrt{\frac{p \cdot \sigb}{2 \Ep}} \sigw \sums \paren{ \avpsdag \xis e^{i p \cdot x} + \bvps \etass e^{i p \cdot x} },
	}
	where we have used
	\eqn{sigthing3.4e}{
		\sigw \sigbm \sigw = (\sigw \sigo, -\sigw \sigq, -\sigw \sigw, -\sigw \sige) \sigw
		= (\sigo \sigw, \sigq \sigw, -\sigw \sigw, \sige \sigw) \sigw
		= \sigstm \sigw \sigw
		= \sigstm.
	}
	
	For the divergence term,
	\al{
		i \sigbm \ptsm \chi &= i \sigbm \int \ddcpf \sqrt{\frac{p \cdot \sig}{2 \Ep}} \sums \paren{ \avps \xis \ptsm e^{-i p \cdot x} + \bvpsdag \etas \ptsm e^{i p \cdot x} } \\
		&= \sigbm \int \ddcpf \sqrt{\frac{p \cdot \sig}{2 \Ep}} \sums \paren{ \avps \xis \psm e^{-i p \cdot x} - \bvpsdag \etas \psm e^{i p \cdot x} } \\
		&= \int \ddcpf \sigbm \psm \sqrt{\frac{p \cdot \sig}{2 \Ep}} \sums \paren{ \avps \xis e^{-i p \cdot x} - \bvpsdag \etas e^{i p \cdot x} } \\
		&= \int \ddcpf \sqrt{\frac{(p \cdot \sigb) (p \cdot \sigb) (p \cdot \sig)}{2 \Ep}} \sums \paren{ \avps \xis e^{-i p \cdot x} - \bvpsdag \etas e^{i p \cdot x} } \\
		&= m \int \ddcpf \sqrt{\frac{p \cdot \sigb}{2 \Ep}} \sums \paren{ \avps \xis e^{-i p \cdot x} - \bvpsdag \etas e^{i p \cdot x} },
	}
	where we have used Peskin \& Schroeder~(3.51), $(p \cdot \sig) (p \cdot \sigb) = p^2 = m^2$.
	
	A relation between $\usp$ and $\vsp$ is given by Eq.~(3.144),
	\al{
		\usp &= -i \gamw [\vsp]^*, &
		\vsp &= -i \gamw [\usp]^*.
	}
	This can be written as~\cite[p.~70]{Peskin}
	\eq{
		\mqty( \sqrt{p \cdot \sig} \xis \\ \sqrt{p \cdot \sigb} \xis ) = -i \mqty( 0 & \sigw \\ -\sigw & 0 ) \mqty( \sqrt{p \cdot \sigs} \etass \\ -\sqrt{p \cdot \sigbs} \etass )
		= i \sigw \mqty( \sqrt{p \cdot \sigbs} \\ \sqrt{p \cdot \sigs} ) \etass
		= i \mqty( \sqrt{p \cdot \sig} \\ \sqrt{p \cdot \sigb} ) \sigw \etass
	}
	where we have used Eq.~\refeq{sigthing3.4e} and
	\al{
		\sigw \sigm \sigw = (\sigw \sigo, \sigw \sigq, \sigw \sigw, \sigw \sige) \sigw
		= (\sigo \sigw, -\sigq \sigw, \sigw \sigw, -\sige \sigw) \sigw
		= \sigbstm \sigw \sigw
		= \sigbstm.
	}
	This implies
	\eq{
		\xis = i \sigw \etass.
	}
	
	Then the full Majorana equation can be written
	\eq{
		\int \ddcpf \sqrt{\frac{p \cdot \sigb}{2 \Ep}} \sums \paren{ i \sigw \bvps \etass e^{i p \cdot x} - \avpsdag \etass e^{i p \cdot x} } = \int \ddcpf \sqrt{\frac{p \cdot \sigb}{2 \Ep}} \sums \paren{ i \sigw \avps \etass e^{-i p \cdot x} - \bvpsdag \etas e^{i p \cdot x} },
	}
	which implies
	\eq{
		\avps = \bvps;
	}
	that is, the fermion is its own antiparticle.  So the Majorana Hamiltonian is
	\eq{
		\ans{ H = \int \ddcpf \sums \Ep \avpsdag \avps, }
	}
	which is diagonal.  The canonical anticommutation relation is the same as for the Dirac creation and annihilation operators, since the additional restriction on the Majorana operators does not change their commutation properties.  It is given by Peskin \& Schroeder~(3.101),
	\eq{
		\ans{ \{ \avpr, \avqsdag \} = (2 \pi)^3 \del^3(\vp - \vq) \delrs. }
	}
}