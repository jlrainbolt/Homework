\documentclass[11pt]{article}
\usepackage{homework}
\usepackage{simpler-wick}

\classname{443}
\homeworknum{3}



\begin{document}

% Environments

\newcommand{\state}[2]{\begin{statement}{#1} #2 \end{statement}}
\newcommand{\prob}[2]{\begin{problem}{#1} #2 \end{problem}}
\newcommand{\subprob}[1]{\begin{subproblem} #1 \end{subproblem}}
\newcommand{\sol}[1]{\begin{solution} #1 \end{solution}}
\newcommand{\fig}[2]{\begin{figure} \centering #2  \label{#1} \end{figure}}

\newcommand{\makebib}{
	\vfill
	\color{black}
	\nocite{*}
	\bibliography{references}{}
	\bibliographystyle{lucas_unsrt}
}
	

% Implication

\newcommand{\qwhere}{\quad \text{where} \quad}
\newcommand{\qimplies}{\quad \implies \quad}
\newcommand{\impliesq}{\implies \quad}



% Brackets

\newcommand{\paren}[1]{\left( #1 \right)}
\newcommand{\brac}[1]{\left[ #1 \right]}
\newcommand{\curly}[1]{\left\{ #1 \right\}}


% Greek

\newcommand{\alp}{\alpha}
\newcommand{\bet}{\beta}
\newcommand{\gam}{\gamma}
\newcommand{\del}{\delta}
\newcommand{\eps}{\epsilon}
\newcommand{\zet}{\zeta}
\newcommand{\tht}{\theta}
\newcommand{\kap}{\kappa}
\newcommand{\lam}{\lambda}
\newcommand{\sig}{\sigma}
\newcommand{\ups}{\upsilon}
\newcommand{\omg}{\omega}

\newcommand{\Gam}{\Gamma}
\newcommand{\Del}{\Delta}
\newcommand{\Tht}{\Theta}
\newcommand{\Lam}{\Lambda}
\newcommand{\Sig}{\Sigma}
\newcommand{\Omg}{\Omega}


% Text

\newcommand{\where}{\text{where }}

% Problem 1

\newcommand{\Hint}{H_\text{int}}
\newcommand{\ddcx}{\dd[3]{x}}
\newcommand{\psib}{\bar{\psi}}

\newcommand{\mh}{m_h}
\newcommand{\mmu}{m_\mu}
\newcommand{\me}{m_e}
\newcommand{\ma}{m_a}

\newcommand{\aexpt}{a_\text{expt.}}
\newcommand{\aQED}{a_\text{QED}}
\renewcommand{\GeV}{\giga\electronvolt}

\newcommand{\gamt}{\gam^5}

\state{Spin-wave theory~(P\&S 11.1)}{\hfix}

\prob{ \label{1a}
	Prove the following wonderful formula: Let $\phix$ be a free scalar field with propagator $\ev{T \phix \phio} = \Dx$.  Then
	\eqn{show1}{
		\ev{ T e^{i \phix} e^{-i \phio} } = e^{[ \Dx - \Do ]}.
	}
	(The  factor $\Do$ gives a formally divergent adjustment of the overall normalization.)
}

\sol{
	According to P\&S~(9.18),
	\eq{
		\ev*{T \phi(\xq) \phi(\xw)}{\Omg} = \frac{\int \DDphi \phi(\xq) \phi(\xw) \exp[ i \int \ddqx \cL ]}{\int \DDphi \exp[ i \int \ddqx \cL ]}.
	}
	We use this expression to write the left-hand side of Eq.~\refeq{show1}:
	\eqn{thing1}{
		\ev{ T e^{i \phix} e^{-i \phio} } = \frac{\int \DDphi e^{i \phix} e^{-i \phio} \exp[ i \int \ddqy \cL ]}{\int \DDphi \exp[ i \int \ddqy \cL ]}
		= \frac{\int \DDphi \exp[i \phix - i \phio + i \int \ddqy \cL ]}{\int \DDphi \exp[ i \int \ddqy \cL ]}.
	}
	For a free Klein-Gordon~(i.e., scalar) field, Eq.~(9.39) tells us that the generating functional $\ZJ$ is given by
	\eq{
		\ZJ = \Zo \exp[ -\frac{1}{2} \int \ddqx \ddqy \Jx \DF(x - y) \Jy ],
	}
	where $\Zo = Z[0]$.  Thus, we want to find some $\Jy$ such that
	\eqn{thing1b}{
		\ev{ T e^{i \phix} e^{-i \phio} } = \frac{\ZJ}{\Zo}
	}
	where in general
	\eq{
		\ZJ = \int \DDphi \exp[ i \int \ddqx [ \cL + \Jx \phi(x) ] ]
	}
	by (9.34).  Inspecting Eq.~\refeq{thing1}, we recognize the denominator as $\Zo$ and see that if
	\eq{
		\Jy = \delq(y - x) - \delq(y)
	}
	we have an expression like Eq.~\refeq{thing1b}.  Collecting these findings, we have
	\al{
		\ans{ \ev{ T e^{i \phix} e^{-i \phio} } }&= \frac{\ZJ}{\Zo} \\
		&= \exp[ -\frac{1}{2} \int \ddqy \ddqz \Jy \DF(y - z) \Jz ] \\
		&= \exp[ -\frac{1}{2} \int \ddqy \ddqz \Jy \DF(y - z) [ \delq(z - x) - \delq(z) ] ] \\
		&= \exp[ -\frac{1}{2} \int \ddqy [ \delq(y - x) - \delq(y) ] [ \DF(y - x) - \DF(y) ] ] \\
		&= \exp[ -\frac{1}{2} [ \DF(0) - \DF(x) - \DF(-x) + \DF(0) ] ] \\
		&= \exp[ \DF(x) - \DF(0) ] \\
		&\ans{\; = e^{[ \Dx - \Do ]}, }
	}
	as we wanted to show. \qed
}



\prob{ \label{1b}
	We can use this formula in Euclidean field theory to discuss correlation functions in a theory with spontaneously broken symmetry for $T < \TC$.  Let us consider only the simplest case of a broken $O(2)$ or $U(1)$ symmetry.  We can write the local spin density as a complex variable
	\eq{
		\sx = \sqx + i \swx.
	}
	The global symmetry is the transformation
	\eq{
		\sx \to e^{-i \alp} \sx.
	}
	If we assume that the physics freezes the modulus of $\sx$, we can parameterize
	\eqn{sx}{
		\sx = A e^{i \phix}
	}
	and write an effective Lagrangian for the field $\phix$.  The symmetry of the theory becomes the translation symmetry
	\eqn{symmetry}{
		\phix \to \phix - \alp.
	}
	Show that (for $d > 0$) the most general renormalizable Lagrangian consistent with this symmetry is the free field theory
	\eqn{show1b}{
		\cL = \frac{1}{2} \rho(\vgrad \phi)^2.
	}
	In statistical mechanics, the constant $\rho$ is called the \emph{spin wave modulus}.  A reasonable hypothesis for $\rho$ is that it is finite for $T < \TC$ and tends to 0 as $T \to \TC$ from below.
}

\sol{
	In accordance with the Klein-Gordon Lagrangian in P\&S~(2.6),
	\eqn{KGL}{
		\cL_\text{K-G} = \frac{1}{2} (\pt \phi)^2 - \frac{1}{2} m^2 \phi^2,
	}
	we interpret $(\vgrad \phi)^2$ as $(\pt \phi)^2$.
	
	The Lagrangian cannot have terms of $\order{\phi^n}$ for any $n \neq 0$ since $\phi(x)$ is not invariant under Eq.~\refeq{symmetry}.  Any combination of derivatives of $\phi$ is invariant, however, since $\alp$ is a constant and does not contribute to any derivative.  Thus, only terms like $\pt^n \phi^m$ (where $n$ denotes a power of $\pt$) for $n, m > 0$ and $n \geq m$ are consistent with the symmetry of Eq.~\refeq{symmetry} for $d$ an integer.
	
	Now we must determine which of these terms are renormalizable.  We know that the Lagrangian must have dimension $d$, and that $\phi$ has dimension $(d - 2) / 2$.  Taking a derivative adds a mass dimension.  The theory is renormalizable if the coupling constant $\rho$ has dimension greater than or equal to 0~\cite[p.~322]{Peskin}.  Let $p$ be the dimension of $\rho$.  The dimension of our allowed term is then
	\eq{
		[ \rho \pt^n \phi^m ] = p + n + m \frac{d - 2}{2},
	}
	which we require to be equal to $d$.  Thus we seek solutions to the system of equations
	\al{
		d &= p + n + m \frac{d - 2}{2}, &
		n &\geq m, &
		p &\geq 0.
	}
	Solving with Mathematica, we find that this system has two solutions: $n = m = 2$ and $p = 0$; and $n = m = 1$ and $p = d / 2$.  However, the term $\pt \phi$ for $n = m = 1$ does not contribute to the action because it is a total derivative and does not contribute when the integral over $\cL$ is evaluated:
	\eq{
		\int \dd[d]{x} \pt\phi = \phi \bigg|_{-\infty}^\infty
		= 0.
	}
	Thus the only possibility is $n = m = 2$.  Note that
	\eq{
		\pt^2 \phi^2 = \pt(\pt \phi^2)
		= 2 \pt( \phi \pt \phi)
		= \pt \phi \pt \phi + \phi \pt^2 \phi
		= (\pt \phi)^2,
	}
	since $\phi \pt^2 \phi$ is not invariant under Eq.~\refeq{sx}.  This means that $\rho$ must be dimensionless and that the only allowed terms in the Lagrangian are proportional to $(\pt \phi)^2$, which is consistent with Eq.~\refeq{show1b}. \qed
}



\prob{
	Compute the correlation function $\ev{ \sx \sao }$.  Adjust $A$ to give a physically sensible normalization (assuming that the system has a physical cutoff at the scale of one atomic spacing) and display the dependence of this correlation function on $x$ for $d = 1, 2, 3, 4$.  Explain the significance of your results.
}

\sol{
	Applying Eq.~\refeq{sx},
	\eq{
		\ev{ \sx \sao } = \ev*{ A e^{i \phix} \As e^{-i \phio} }
		= \ev*{ \abs{A}^2 } \ev*{ e^{i \phix} e^{-i \phio} }.
	}
	Now we can apply Eq.~\refeq{show1} to find
	\eqn{thing1c}{
		\ans{ \ev{ \sx \sao } = \abs{A}^2 \exp[ D(x) - D(0) ], }
	}
	where $D(x - y)$ is a Green's function.  Since our Lagrangian is similar to the Klein-Gordon Lagrangian Eq.~\refeq{2.6}, our Green's function is similar to that of the Klein-Gordon operator, which is given by P\&S~(2.56):
	\eq{
		(\pt^2 + m^2) D(x - y) = -i \delq(x - y).
	}
	The Feynman prescription for this Green's function is given by (2.59),
	\eqn{DF}{
		\DF(x - y) = \int \ddqpf \frac{i}{p^2 - m^2 + i \eps} e^{-i p \cdot (x - y)}.
	}
	For the Lagrangian in Eq.~\refeq{show1b}, we set $m = 0$ and insert a factor of $\rho$:
	\eq{
		\rho \pt^2 D(x - y) = -i \deld(x - y),
	}
	so adapting Eq.~\refeq{DF} for this situation yields
	\eqn{DF}{
		\DF(x - y) = \frac{1}{\rho} \int \dddpf \frac{i}{p^2 + i \eps} e^{-i p \cdot (x - y)}.
	}
	We see that $\DF(0)$ diverges, so we absorb it into the constant to make the normalization physically sensible.  We can do this because, as we showed in \ref{1b}, the theory is renormalizable.  Define $A'$ such that
	\eq{
		{A'}^2 = \abs{A}^2 e^{-D(0)}.
	}
	Then Eq.~\refeq{thing1c} can be written
	\eq{
		\ans{ \ev{ \sx \sao } =  {A'}^2 e^{D(x)}. }
	}
	
	To evaluate the divergent integral $D(x)$, we look to the Feynman parameter method we have been using to solve divergent integrals.  Apparently, the Schwinger parametrization is useful in deriving the Feynman parametrization, and it is given by~\cite{Feynman}
	\eq{
		\frac{1}{A} = \intoi \dds e^{-s A}.
	}
	Using this equation, we can write Eq.~\refeq{DF} as
	\eq{
		\DF(x) = \frac{1}{\rho} \int \dddpf \frac{i}{p^2} e^{-i p \cdot x}
		= \frac{i}{\rho} \int \dddpf \intoi \dds e^{-s p^2} e^{-i p \cdot x}.
	}
	Now we can complete the square in the exponential to get a Gaussian integral:
	\al{
		\DF(x) &= \frac{i}{\rho} \int \dddpf \intoi \dds \exp[ -s p^2 - i p \cdot x + \frac{x^2}{4 s} - \frac{x^2}{4 s} ] \\
		&= \frac{i}{\rho} \int \dddpf \intoi \dds \exp[ -s \paren{ p + \frac{i x}{2 s} }^2 - \frac{x^2}{4 s} ] \\
		&= \frac{i}{\rho (2 \pi)^d} \intoi \dds e^{-x^2 / 4 s} \int \dd[d]{u} e^{-s u^2} \\
		&= \frac{i}{\rho (2 \pi)^{d}} \intoi \dds e^{-x^2 / 4 s} \sqrt{ \frac{(2\pi)^d}{(2s)^d} } \\
		&= \frac{i}{\rho (4 \pi)^{d / 2}} \intoi \dds \frac{e^{-x^2 / 4 s}}{s^{d / 2}}
	}
	where we have used~\cite{QFT}
	\eq{
		\int \exp( -\frac{1}{2} x \cdot A \cdot x ) \dd[n]{x} = \sqrt{\frac{(2\pi)^n}{\det A}},
	}
	with $A$ a $d \times d$ diagonal matrix $2s$.  Using Mathematica to integrate with respect to $s$, we find
	\eq{
		\DF(x) = \frac{i}{\rho (4 \pi)^{d / 2}} \frac{2^{d - 2}}{x^{d - 2}} \Gam(d / 2 - 1)
		= \frac{i}{4 \pi^d \rho} \Gam(d / 2 - 1) x^{2 - d}.
	}
	The gamma function diverges as $d \to 2$, so as we have done in previous problems, we expand about $\eps = 2 - d$.  Evaluating the series expansion using Mathematica, we obtain
	\eq{
		\DF(x) = \frac{i}{4 \pi^{1 - \eps} \rho} \Gam(\eps / 2) x^\eps
		\approx \frac{i}{4 \pi \rho} \paren{ \frac{2}{\eps} - \gam + 2 \ln(\pi x) }
		\sim \frac{i}{2 \pi \rho} \ln(x)
		= i \ln(\frac{1}{x^{2 \pi \rho}}).
	}
	We Wick rotate $x \to i x$.  Then the dependence of the correlation function on $x$ for $d = 1, 2, 3, 4$ is
	\ans{\al{
		(d = 1) &\qquad \ev{ \sx \sao } \sim e^{-x / 2 \sqrt{\pi} \rho}, &
		(d = 2) &\qquad \ev{ \sx \sao } \sim x^{2 \pi \rho}, \\
		(d = 3) &\qquad \ev{ \sx \sao } \sim \frac{1}{x}, &
		(d = 4) &\qquad \ev{ \sx \sao } \sim \frac{1}{x^2}.
	}}%
	In $d > 2$ dimensions, the expectation value of the correlation function tends to 0 at large distances $x$.  For $d > 2$, it drops off more quickly as $d$ increases.  The $d \leq 2$ cases depend on $\rho$, which we assume is positive.  The $d = 1$ case drops off with increasing distance, and more quickly with smaller $\rho$.  For $d = 2$, the expectation value of the correlation function increases with increasing distance, and it blows up more quickly with larger $\rho$.
	
	These results are consistent with the Mermin--Wagner theorem, which states that a continuous symmetry cannot be broken in $d \leq 2$ dimensions~\cite{CMW}.  That is, in $d \leq 2$ dimensions, a symmetry-breaking field cannot have a nonzero vacuum expectation value~\cite[p.~460]{Peskin}.  A physical explanation is that each spin has more nearest neighbors in higher dimensions.  Since the spins are inclined to align with their neighbors, there is a higher degree of correlation in higher dimensions at the same distance.  In two dimensions, the correlations are weak enough that they are overpowered by the field fluctuations.
}






\state{(Peskin \& Schroeder 4.1)}{
	Let us return to the problem of the creation of Klein-Gordon particles by a classical source.  Recall from Chapter 2 that this process can be described by the Hamiltonian
	\eq{
		H = \Ho + \int \ddcx [ -\jtx \phix ],
	}
	where $\Ho$ is the free Klein-Gordon Hamiltonian, $\phix$ is the Klein-Gordon field, and $\jx$ is a c-number scalar function.  We found that, if the system is in the vacuum state before the source is turned on, the source will create a mean number of particles
	\eq{
		\evN = \int \ddcpf \frac{1}{2 \Ep} \abs{\jtp}^2.
	}
	In this problem we will verify that statement, and extract more detailed information, by using a perturbation expansion in the strength of the source.
}

\prob{}{	\label{4.1a}
	Show that the probability that the source creates \emph{no} particles is given by
	\eq{
		\Po = \left| \bo T \curly{ \exp( i \int \ddqx \jx \phiIx ) } \ko \right|^2.
	}
}

\sol{
	Both the initial and the final state are the vacuum state.  The probability is
	\eq{
		\Po = \abs{ \ev{\Utto}{0} }^2,
	}
	where
	\eq{
		\Utto = T \curly{ \exp( -i \inttot \ddtp \HItp ) }
	}
	from Eq.~(4.22).  A general expression for the interaction Hamiltonian in the interaction picture is given by Peskin \& Schroeder~(4.19):
	\eq{
		\HIt = e^{i \Hotto} (\Hint) e^{-i \Hotto}.
	}
	
	For the given Hamiltonian $H = \Ho + \Hint$, we have
	\eq{
		\HIt = \int \ddcx [ -\jtx \phiItx ],
	}
	where we have used (4.14),
	\eq{
		\phiItx = e^{i \Hotto} \phitox e^{-i \Hotto}.
	}
	Then we have
	\eq{
		\Utto = T \curly{ \exp( -i \inttot \ddtp \int \ddcx [ -\jtx \phiI ] ) }
		= T \curly{ \exp( i \int \ddqx \jx \phiIx ) },
	}
	so the probability of the source's creating no particles is
	\eq{
		\ans{ \Po = \abs{ \bo T \curly{ \exp( i \int \ddqx \jx \phiIx ) } \ko }^2, }
	}
	as desired. \qed
}


\clearpage
\prob{}{	\label{4.1b}
	Evaluate the term in $\Po$ of order $j^2$, and show that $\Po = 1 - \lam + \order{j^4}$, where $\lam$ equals the expression given above for $\evN$.
}

\sol{
	The first few terms of the Taylor series expansion for $e^z$ are~cite{Maclaurin}
	\eq{
		e^z \approx 1 + z + \frac{z^2}{2}.
	}
	Then
	\eq{
		\exp( i \int \ddqx \jx \phiIx ) \approx 1 + i \int \ddqx \jx \phiIx - \frac{1}{2} \iint \ddqx \ddqy \jx \phiIx \jy \phiIy.
	}
	Then the probability can be written
	\aln{
		\Po &= \abs{ 1 + i \bo \int \ddqx \jx \phiIx \ko - \frac{1}{2} \bo T \curly{ \iint \ddqx \ddqy \jx \phiIx \jy \phiIy } \ko }^2 \notag \\
		&= \abs{ 1 - \frac{1}{2} \iint \ddqx \ddqy \jx \jy \ev*{T \phiIx \phiIy}{0} }^2, \label{p0}
	}
	since $\ev{\phiI}{0} = 0$ (and if we had an $\order{j^3}$ term, it would likewise vanish since there would be an uncontracted operator remaining~\cite[p.~89]{Peskin}).  Applying Peskin \& Schroder~(4.11),
	\eq{
		\ev*{T \phiIx \phiIy}{0} = \int \ddqpf \frac{i e^{-i p \cdot (x - y)}}{p^2 - m^2 + i \eps},
	}
	we have~\cite[p.~30]{Peskin}
	\al{
		\iint \ddqx \ddqy \jx \jy \ev*{T \phiIx \phiIy}{0} &= \iint \ddqx \ddqy \int \ddqpf \jx \jy \frac{i e^{-i p \cdot (x - y)}}{p^2 - m^2 + i \eps} \\
		&= i \int \ddqpf \int \ddqx e^{-i p \cdot x} \jx \int \ddqy e^{i p \cdot y} \jy \frac{1}{p^2 - m^2 + i \eps} \\
		&= i \int \ddqpf \frac{\abs{\jtp}^2}{p^2 - m^2 + i \eps} \\
		&= \int \ddcpf \int \ddpof \frac{i \abs{\jtp}^2}{{\po}^2 - \Ep^2 + i \eps},
	}
	where we have used~\cite[p.~32]{Peskin}
	\aln{ \label{jts}
		\jtp &= \int \ddqy e^{i p \cdot y} \jy, &
		\jtsp &= \int \ddqy e^{-i p \cdot y} \jy.
	}
	Then we can perform a contour integral.  Letting $\eps = 2 \Ep \eps'$ and neglecting terms of $\order{\eps^2}$~\cite{Evans},
	\eq{
		\iint \ddqx \ddqy \jx \jy \ev*{T \phiIx \phiIy}{0} 
		= \int \ddcpf \int \ddpof \frac{i \abs{\jtp}^2}{({\po} - \Ep + i \eps') (\po + \Ep - i \eps')},
	}
	In general the poles are at $\po = \pm(\Ep - i \eps')$~\cite[p.~31]{Peskin}.  When we close the contour in the upper half plane, we enclose only the pole at $\po = -\Ep + i \eps'$.  Then, applying the residue theorem~\cite{Residue},
	\al{
		\iint \ddqx \ddqy \jx \jy \ev*{T \phiIx \phiIy}{0} &= \int \ddcpf 2\pi i \Res_{\po = -\Ep + i \eps'}\!\paren{ \frac{i \abs{\jtp}^2}{({\po} - \Ep + i \eps') (\po + \Ep - i \eps')} } \\
		&= -\int \ddcpf \frac{\abs{\jtp}^2}{-2 \Ep + 2 i \eps'} \\
		&= \ans{ \int \ddcpf \frac{\abs{\jtp}^2}{2 \Ep}, }
	}
	where in the final step we have neglected terms of $\order{\eps}$.  This is identical to the given expression for ${\lam = \evN}$.
	
	Making this substitution in Eq.~\refeq{p0},
	\eq{
		\Po =  \abs{ 1 - \frac{1}{2} \int \ddcpf \frac{\abs{\jtp}^2}{2 \Ep} }^2
		= 1 - \frac{1}{2} \int \ddcpf \frac{\abs{\jtp}^2}{2 \Ep} + \order{j^4}
		= \ans{ 1 - \lam + \order{j^4} }
	}
	as we sought to show.
}



\prob{}{	\label{4.1c}
	Represent the term computed in \ref{4.1b} as a Feynman diagram.  Now represent the whole perturbation series for $\Po$ in terms of Feynman diagrams.  Show that this series exponentiates, so that it can be summed exactly: $\Po = e^{-\lam}$.
}

\sol{
	From \ref{4.1b}, the term is
	\eq{
		-\lam = -\int \ddqpf \abs{\jtp}^2 \frac{i}{p^2 - m^2 + i \eps}.
	}
	According to the momentum space Feynman rules~\cite[p.~95]{Peskin}, this term is represented by $\centergraphics{blue/prop}$.
	
	We know that we will only have terms in $\Po$ of even powers of $\phi$.  Then for integer $n$, the term of order $j^{2 n}$ is proportional to
	\eq{
		\lam^n = \int \ddqxq \cdots \ddqxn \jxq \cdots \jxn \ev*{ T \phiq \cdots \phin }{0}
		= \int \ddqpf \abs{\jtp}^{2 n} \paren{ \frac{i}{p^2 - m^2 + i \eps} }^n.
	}
	So the whole perturbation series can be written as
	\eq{
		\ans{ \Po = \abs{ 1 + \centergraphics{blue/prop1} + \centergraphics{blue/prop2} + \centergraphics{blue/prop3} + \centergraphics{blue/prop4} + \cdots }^2, }
	}
	where each propagator represents one factor of $\lam$.  For the symmetry factor, there are $2^{2n / 2} = 2^n$ ways the $2n$ vertices can be chosen to be initial or final vertices, and a further $n!$ ways the $n$ initial vertices can be paired with the $n$ final vertices.  This gives us the symmetry factor $2^n n!$.  Then, using the power series~\cite{Exponential}
	\eq{
		e^x = \sumni \frac{x^n}{n!},
	}
	we can write
	\eq{
		\Po = \paren{ \sumni \frac{(-\lam)^n}{2^n n!} }^2
		= \brac{ \sumni \frac{1}{n!} \paren{ -\frac{\lam}{2} }^n }^2
		= ( e^{-\lam / 2} )^2
		= \ans{ e^{-\lam / 2} }
	}
	as desired. \qed
}


\clearpage
\prob{}{	\label{4.1d}
	Compute the probability that the source creates one particle of momentum $k$.  Perform this computation first to $\order{j}$ and then to all orders, using the trick of \ref{4.1c} to sum the series.
}

\sol{
	The initial state is $\ko$ and the final state is $\kk = \sqrt{2 \Ek} \akdag \ko$ from Peskin \& Schroeder~(2.35).  The probability is
	\eq{
		\Pk = \abs{ \mel*{\vk}{\Utto}{0} }^2
		= \abs{ \bk T \curly{ \exp( i \int \ddqx \jx \phiIx ) } \ko }^2
	}
	from the result of \ref{4.1a}.  To $\order{j}$, this is
	\eq{
		\Pk = \abs{ i \bk \int \ddqx \jx \phiIx \ko }^2
		= \abs{ i \sqrt{2 \Ek} \int \ddqx \ev*{\ak \jx \phiIx }{0} }^2
	}
	since $\ev{\ak}{0} = 0$.  At this point we need Peskin \& Schroeder~(2.25)~\cite[p.~83]{Peskin},
	\eq{
		\phitox = \int \ddcpf \frac{\sqrt{2 \Ek} }{\sqrt{2 \Ep}} \paren{ \ap e^{i \vp \vdot \vx} + \apdag e^{-i \vp \vdot \vx} },
	}
	and (2.26),
	\eq{
		[ \ap, \appdag ] = (2\pi)^3 \del^3(\vp - \vp').
	}
	Then
	\al{
		\Pk &= \abs{ i \int \ddqx \ev*{\ak \jx \int \ddcpf \frac{\sqrt{2 \Ek} }{\sqrt{2 \Ep}} \paren{ \ap e^{i p \cdot x} + \apdag e^{-i p \cdot x} } }{0} }^2 \\
		&= \abs{ i \int \ddqx \ev*{\jx \int \ddcpf \sqrt{\frac{\Ek}{\Ep}} \paren{ \apdag \ak + [\ak, \apdag] } e^{-i p \cdot x} }{0} }^2 \\
		&= \abs{ i \int \ddqx \jx \int \ddcp \sqrt{\frac{\Ek}{\Ep}} e^{-i p \cdot x} \del^3(\vk - \vp) }^2 \\
		&= \abs{ i \int \ddqx \jx e^{-i k \cdot x} }^2 \\
		&= \abs{ i \jtk }^2 \\
		&= \ans{ \abs{\jtk}^2, }
	}
	since $\ev{\ak \ap}{0} = \ev{\apdag \ak}{0} = 0$, and where we have used Eq.~\refeq{jts}.
	
	Then the term of $\order{j^n}$ is proportional to $\abs{\jtk}^{n + 1}$.  The symmetry factor is the same as in~\ref{4.1c}, so
	
	\hl{idek}
%	\eq{
%		P(k) = \paren{ \sumni \frac{\abs{\jtk}^{n + 1}}{2^n n!} }^n
%		= \paren{ \abs{\jtk} \sumni \frac{\abs{\jtk}^{n + 1}}{2^n n!} }^n
%	}

}



\prob{}{
	Show that the probability of producing $n$ particles is given by
	\eq{
		\Pn = \frac{\lam^n e^{-\lam}}{n!}.
	}
	This is a \emph{Poisson} distribution.
}

\sol{
	From the procedure of~\ref{4.1d}, the probability of producing $n$ particles of momentum $k$ is
	\eq{
		P(k^n) = \abs{ \bo \ak^n T \curly{ \exp( i \int \ddqx \jx \phiIx ) } \ko }^2
	}
	
	\hl{actually idgi}
}



\prob{}{
	Prove the following facts about the Poisson distribution:
	\al{
		\sumni \Pn &= 1, &
		\evN &= \sumni n \Pn = \lam.
	}
	The first identity says that the $\Pn$s are properly normalized probabilities, while the second confirms our proposal for $\evN$.  Compute the mean square fluctuation $\ev{(N - \evN)^2}$.
}

\sol{
	\hl{need answers to previous :(}
}




\clearpage

\newcommand{\tht}{\theta}
\newcommand{\thtxx}{\tht_{xx}}
\newcommand{\thtyy}{\tht_{yy}}
\newcommand{\thttt}{\tht_{tt}}
\newcommand{\thtt}{\tht_t}
\newcommand{\thtx}{\tht_x}
\newcommand{\thty}{\tht_y}
\newcommand{\sint}{\sin{\tht}}
\newcommand{\dxdydt}{\dxdy \dd{t}}

\begin{statement}{}
	The nondimensionalized, multidimensional Sine-Gordon equation,
	\beq
		\thtxx + \thtyy - \thttt = \sint
	\eeq
	for $\tht(x, y, t)$, is the Euler-Lagrange equation for the action integral
	\beq
		S[\tht] = \intR \left\{ \frac{1}{2} \left[ \thtt^2 - (\nabla\tht)^2 \right] - \sint \right\} \dxdydt
	\eeq
	with $\nabla\tht = (\pdv*{\tht}{x}, \pdv*{\tht}{y})$.  The functional $S[\tht]$ is invariant under translation of $x$, $y$, and $t$.  Find the associated energy-momentum tensor and energy-momentum vector.
\end{statement}

\begin{solution}
	Expanding out $(\nabla\tht)^2$, the Lagrangian density is
	\beqn \label{lagr3}
		\Ld = \frac{1}{2} (\thtt^2 - \thtx^2 - \thty^2) - \sint.
	\eeqn
	The energy-momentum tensor is defined by
	\beq
		T_{ij} = \pdv{\Ld}{\tht_{x_i}} \pdv{\tht}{x_j} - \Ld \, \delta_{ij},
	\eeq
	where $x_i \in \{ x_0, x_1, x_2 \} = \{ t, x, y \}$.  The diagonal elements of $T$ are then
	\begin{align*}
		T_{00} &= \pdv{\Ld}{\thtt} \pdv{\tht}{t} - \Ld = \thtt^2 - \frac{1}{2} (\thtt^2 - \thtx^2 - \thty^2) + \sint = \frac{1}{2} (\thtt^2 + \thtx^2 + \thty^2) + \sint, \\
		T_{11} &= \pdv{\Ld}{\thtx} \pdv{\tht}{x} - \Ld = -\thtx^2 - \frac{1}{2} (\thtt^2 - \thtx^2 - \thty^2) + \sint = -\frac{1}{2} (\thtt^2 + \thtx^2 - \thty^2) + \sint, \\
		T_{22} &= \pdv{\Ld}{\thty} \pdv{\tht}{y} - \Ld = -\thty^2 - \frac{1}{2} (\thtt^2 - \thtx^2 - \thty^2) + \sint = -\frac{1}{2} (\thtt^2 - \thtx^2 + \thty^2) + \sint,
	\end{align*}
	and the nondiagonal elements are
	\begin{align*}
		T_{01} &= \pdv{\Ld}{\thtt} \pdv{\tht}{x} = \thtt \thtx, &
		T_{02} &= \pdv{\Ld}{\thtt} \pdv{\tht}{y} = \thtt \thty, &
		T_{12} &= \pdv{\Ld}{\thtx} \pdv{\tht}{y} = -\thtx \thty, \\
		T_{10} &= \pdv{\Ld}{\thtx} \pdv{\tht}{t} = -\thtt \thtx, &
		T_{20} &= \pdv{\Ld}{\thty} \pdv{\tht}{t} = -\thtt \thty, &
		T_{21} &= \pdv{\Ld}{\thty} \pdv{\tht}{x} = -\thtx \thty.
	\end{align*}
	In matrix form, we have
	\beq
		T = \mqty[(\thtt^2 + \thtx^2 + \thty^2) / 2 + \sint & \thtt \thtx & \thtt \thty \\
				-\thtt \thtx & -(\thtt^2 + \thtx^2 - \thty^2) / 2 + \sint & -\thtx \thty \\
				-\thtt \thty & -\thtx \thty & -(\thtt^2 - \thtx^2 + \thty^2) / 2 + \sint ].
	\eeq
	The energy-momentum vector is defined by
	\beq
		P_j = \int T_{0j} \dd{x_1} \dd{x_2}.
	\eeq
	Its components are then
	\begin{align*}
		P_0 &= \int \left[ \frac{1}{2} (\thtt^2 + \thtx^2 + \thty^2) + \sint \right] \dxdy, &
		P_1 &= \int \thtt \thtx \dxdy, &
		P_2 &= \int \thtt \thty \dxdy.
	\end{align*}
\vfix
\end{solution}






%\state{Aspects of scaling behavior in QFT}{\hfix}

%\prob{}{
%	As discussed in class, massless free scalar field theory is invariant under scale transformations if we assign the appropriate scale dimension to the field $\phi$.  The action of dilations on coordinates is implemented by the operator $D = i \xm \ptsm$.  Find  the commutation relations of this operator with the operators that implement the Poincar\'{e} algebra of translations $\Psm = i \ptsm$ and Lorentz transformations $\Jsmn = i (\xsm \ptsn - \xsn \ptsm)$.  Find the canonical Noether current and Noether charge for this symmetry in the free field theory, and find the commutation relations of the charge with the field $\phi$ to show that it implements the infinitesimal form of the symmetry on the fields.  Check also that this current has the right commutation relations with the field theory Hamiltonian and momentum generators.
%}



%\prob{}{
%	The sound waves in a solid obey a wave equation, which at low enough wavelengths/energies is described by the action (for simplicity, consider one spatial dimension)
%	\eq{
%		\So = \int \ddtx (\ptst \phi \ptst \phi - \cs^2 \ptsx \phi \ptsx \phi)
%	}
%	where $\cs$ is the speed of sound in the material.  The field $\phi$ describes the distortion of the lattice away from equilibrium.  Model this situation by a lattice of atoms (mass $m$, lattice spacing $a$) interacting through harmonic nearest-neighbor forces (spring constant $k$).  Find the leading term in the action which corrects the above kinetic energy in a power series in $a \times (\text{derivatives})$ (this can be done for instance by expanding the exact dispersion relation for the chain of masses and springs).  Show that this correction term is an irrelevant perturbation of the action, so that all traces of the lattice structure disappear in the continuum limit $a \to 0$ (e.g.~all traces of the structure of interactions at the atomic level---such as a general potential $\Vr$ between the atoms rather than harmonic forces---disappear, being summarized in the constant $\cs$, and the effective action $\So$ is independent of the cutoff $a$).  Estimate the momentum scale at which the irrelevant corrections amount to 10\% of the total energy of a phonon.
%}


\makebib

\end{document}
