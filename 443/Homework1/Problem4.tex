\state{}{
	The classical limit of a harmonic oscillator can be described in terms of \emph{coherent states},
	\eq{
		\kalp = \exp( \alp \adag - \frac{1}{2} |\alp|^2 ) \ko.
	}
	When $\alp$ is large, the oscillator state is semiclassical.  Proceeding similarly for the Fourier modes of the quantum Klein-Gordon field,
	\al{
		\kf &= \Nf \exp( i \int \ddpf \fp \, \apdag ) \ko, &
		\Nf &= \exp( -\frac{1}{2} \int \ddpf | \fp |^2 ).
	}
}



\prob{}{
	Evaluate the expectation value of the field operator $\evphif$ and show that it satisfies the Klein-Gordon equation.
}

\sol{
	The field operator is given in terms of the creation and annihilation operators by Peskin \& Schroeder~(2.47),
	\eq{
		\phix = \int \ddpf \frac{1}{\sqrt{2 \Ep}} \paren{ \ap e^{-i p \cdot x} + \apdag e^{i p \cdot x} } \bigg|_{\po = \Ep}.
	}
	Note that
	\aln{
		\ap \kf &= \ap \Nf \exp( i \int \ddppf \fpp \, \appdag ) \ko
		= \Nf \ap \sumni \frac{i}{n!} \int \ddppf \fpp \, \appdag \ko \notag \\
		&= \Nf \ap \brac{ 1 + i \int \ddppf \fpp \, \appdag + \frac{1}{2} \paren{ i \int \ddppf \fpp \, \appdag }^2 + \cdots } \ko, \label{thing4.a}
	}
	where we have used the Maclaurin series expansion for the exponential function~\cite{Exponential}.  The first term vanishes.  For the second term,
	\al{
		i \int \ddppf \fpp \, \ap \appdag \ko
		&= i \int \ddppf \fpp \brac{ \appdag \ap + (2\pi)^3 \, \del^3(\vp - \vp')} \ko \\
		&= i \paren{ \fp + \int \ddppf \fpp \appdag \ap } \ko
		= i \fp \ko.
	}
	For the third,
	\al{
		\frac{i^2}{2} \int \ddppf \fpp \, \ap \appdag \int &\ddpppf \fppp \, \apppdag \ko
		= \frac{i^2}{2} \paren{ \fp + \int \ddppf \fpp \appdag \ap } \int \ddpppf \fppp \, \apppdag \ko \\
		&= \brac{ \frac{i^2}{2} \fp \int \ddpppf \fppp \, \apppdag + \frac{i^2}{2} \int \ddppf \fpp \appdag \paren{ \fp + \int \ddpppf \fppp \apppdag \ap } } \ko \\
		&= i^2 \fp \int \ddppf \fpp \, \appdag \ko.
	}
	Returning to Eq.~\refeq{thing4.a}, we have
	\al{
		\ap \kf &= i \fp \Nf \brac{ 1 + i \int \ddppf \fpp \, \appdag + \frac{1}{2} \paren{ i \int \ddppf \fpp \, \appdag }^2 + \cdots } \ko \\
		&= i \fp \Nf \exp( i \int \ddppf \fpp \, \appdag ) \ko
		= i \fp \kf,
	}
	where the Maclaurin series has shifted by one term.  Likewise, $\bra{f} \apdag = -i \fsp$.

	Then, using these results, we can evaluate the expectation value:
	\al{
		\evphif &= \ev*{ \int \ddpf \frac{1}{\sqrt{2 \Ep}} \paren{ \ap e^{-i p \cdot x} + \apdag e^{i p \cdot x} } }{f}
		= \int \ddpf \frac{1}{\sqrt{2 \Ep}} \paren{ \evapf e^{-i p \cdot x} + \evapdagf e^{i p \cdot x} } \\
		&= \ans{ \int \ddpf \frac{i}{\sqrt{2 \Ep}} \brac{ \fp \, e^{-i p \cdot x} - \fsp \, e^{i p \cdot x} }. }
	}
	The Klein-Gordon equation is given by Peskin \& Schroeder~(2.7), $(\pdv*[2]{t} - \laplacian + m^2) \phi = 0$.  Then
	\eqn{KG4.a}{
		(\ptm \ptsm + m^2) \evphif = 
		\int \ddpf \frac{i}{\sqrt{2 \Ep}} \paren{ \pdv[2]{t} - \laplacian + m^2 }  \brac{ \fp \, e^{-i (\Ep t - \vp \vdot \vx)} - \fsp \, e^{i (\Ep t - \vp \vdot \vx)} }.
	}
	For the time derivative,
	\al{
		\pdv[2]{t} \brac{ \fp \, e^{-i (\Ep t - \vp \vdot \vx)} - \fsp \, e^{i (\Ep t - \vp \vdot \vx)} }
		&= \pdv{t} \brac{ -i \Ep \, \fp \, e^{-i (\Ep t - \vp \vdot \vx)} - i \Ep \, \fsp \, e^{i (\Ep t - \vp \vdot \vx)} } \\
		&= -\Ep^2  \brac{ \fp \, e^{-i (\Ep t - \vp \vdot \vx)} - \fsp \, e^{i (\Ep t - \vp \vdot \vx)} },
	}
	and for the spatial derivative,
	\eq{
		\laplacian \brac{ \fp \, e^{-i (\Ep t - \vp \vdot \vx)} - \fsp \, e^{i (\Ep t - \vp \vdot \vx)} } = \vp^2 \brac{ \fp \, e^{-i (\Ep t - \vp \vdot \vx)} - \fsp \, e^{i (\Ep t - \vp \vdot \vx)} }.
	}
	Feeding these back into Eq.~\refeq{KG4.a} and using $\Ep = \sqrt{\vp^2 + m^2}$~\cite[p.~22]{Peskin},
	\al{
		(\pdv*[2]{t} - \laplacian + m^2) \evphif &= \int \ddpf \frac{i}{\sqrt{2 \Ep}} (\vp^2 + m^2 - \Ep^2) \brac{ \fp \, e^{-i (\Ep t - \vp \vdot \vx)} - \fsp \, e^{i (\Ep t - \vp \vdot \vx)} } \\
		&= \int \ddpf \frac{i}{\sqrt{2 \Ep}} (\Ep^2 - \Ep^2) \brac{ \fp \, e^{-i (\Ep t - \vp \vdot \vx)} - \fsp \, e^{i (\Ep t - \vp \vdot \vx)} } \\
		&= \ans{ 0, }
	}
	so $\evphif$ satisfies the Klein-Gordon equation. \qed
}



\prob{}{
	Evaluate the relative mean square fluctuation of the occupation number of the mode with momentum $\vp$ and the relative mean square fluctuation in the total energy:
	\al{
		&\frac{\evnhps - \evnhp^2}{\evnhp^2}, &
		&\frac{\evHs - \evH^2}{\evH^2}.
	}
	Is either of these a good measure of the degree to which the field is classical?  Justify your answer.
}

\sol{
	Note that $\nhp = \apdag \ap$~\cite[p.~90]{Sakurai}.  Then
	\al{
		\evnhp &= \ev{\apdag \ap}{f}
		= \fsp \fp = \abs{\fp}^2, \\
		\evnhps &= \ev{\apdag \ap \apdag \ap}{f}
		= \abs{\fp}^2 \ev{\ap \apdag}{f}
		= \abs{\fp}^2 \ev{\apdag \ap + [\ap, \apdag]}{f}
%		= \abs{\fp}^4,
		= \abs{\fp}^4 + \abs{\fp} \, (2\pi)^3 \, \del^3(0)
	}
%	where we have dropped the divergent term.  So
	so
	\eq{
		\frac{\evnhps - \evnhp^2}{\evnhp^2}
		= \frac{ \abs{\fp}^4 + \abs{\fp} \, (2\pi)^3 \, \del^3(0) - \abs{\fp}^4 }{\abs{\fp}^4}
		= \ans{ \frac{ (2\pi)^3 \, \del^3(0) }{\abs{\fp}^2}, }
%		= \frac{ \abs{\fp}^4 - \abs{\fp}^4 }{\abs{\fp}^4}
%		= \ans{ 0. }
	}
	which diverges.
	
	From Peskin \& Schroeder~(2.31),
	\eq{
		\int \ddpf \, \Ep \apdag \ap,
	}
	so
	\al{
		\evH = \int \ddpf \, \Ep \ev{\apdag \ap}{f}
		= \int \ddpf \, \Ep \abs{\fp}^2,
	}
	and
	\al{
		\evHs &= \int \ddppfs \Ep \Epp \ev{\apdag \ap \appdag \app}{f}
		= \int \ddppfs \, \Ep \Epp \, \fsp \, \fpp \ev{\ap \appdag}{f} \\
		&= \int \ddppfs \, \Ep \Epp \, \fsp \, \fpp \ev{\appdag \ap + [\ap, \appdag]}{f} \\
		&= \int \ddppfs \, \Ep \Epp \, \fsp \, \fpp \brac{ \fsp \fpp + (2\pi)^3 \, \del^3(\vp - \vp') } \\
%		&= \int \ddppfs \, \Ep \Epp \abs{\fp}^2 \abs{\fpp}^2 + \int \ddpf \, \Ep^2 \abs{\fp}^2 \\
		&= \paren{ \int \ddpf \, \Ep \abs{\fp}^2 }^2 + \int \ddpf \, \Ep^2 \abs{\fp}^2.
	}
	Then
	\al{
		\frac{\evHs - \evH^2}{\evH^2} &= \frac{\paren{ \dint \ddpf \, \Ep \, \abs{\fp}^2 }^2 + \dint \ddpf \, \Ep^2 \, \abs{\fp}^2 - \paren{ \dint \ddpf \, \Ep \abs{\fp}^2 }^2}{\paren{ \dint \ddpf \, \Ep \abs{\fp}^2 }^2} \\
		&= \ans{ \left. \dint \ddpf \, \Ep^2 \, \abs{\fp}^2 \middle/ \paren{ \dint \ddpf \, \Ep \, \abs{\fp}^2 \right. }^2. }
	}

	Both quantities can tell us something about how classical the field is.  Both quantities are zero in a classical system, since quantum fluctuations do not occur.  The infinite mean square fluctuation of the occupation number indicates that, in any given mode, any number of particles may be created or destroyed at any given time.  However, the changing number of particles in each state does not correspond to a large change in the total energy of the system, since the mean square fluctuation in the total energy is finite.  Both mean square fluctuations tend to zero as $\fp \to \infty$.  If $\fp$ is analogous to $\alp$ in that a large $\fp$ indicates a semiclassical system, then this is the expected behavior.  
}



\prob{}{
	Take $\Delta(x - y) = \ev{\phivx \, \phivy}{0}$ (equal times) as a measure of the fluctuations or correlations of the field amplitude.  Use your result from problem~\ref{3} to evaluate this quantity.  What is the meaning of the divergence as $\vx \to \vy$?
}

\sol{
	Since $x - y$ is described as spacelike, the solution is the same as Eq.~\refeq{sol3}:
	\eq{
		\ans{ \Delta(x - y) = \frac{m}{(2 \pi)^2 \abs{x - y}} \, \Kq(m \,\abs{x - y}). }
	}
	This quantity represents the amplitude for a particle to propagate from $y$ to $x$~\cite[p.~27]{Peskin}.  When $\xo = \yo$ and $\vx \to \vy$, the divergence means that this amplitude becomes infinite.  This seems to imply that, in this limit, a particle located at $\vx$ would remain there indefinitely.
}