\state{The complex scalar field (Peskin \& Schroeder 2.2)}{
	Consider the field theory of a complex-valued scalar field obeying the Klein-Gordon equation.  The action of this theory is
	\eqn{action2}{
		S = \int \ddqx (\ptsm \phis \ptm \phi - m^2 \phis \phi).
	}
	It is easiest to analyze this theory by considering $\phix$ and $\phisx$, rather than the real and imaginary parts of $\phix$, as the basic dynamical variables.
}



\prob{}{
	Find the conjugate momenta to $\phix$ and $\phisx$ and the canonical commutation relations.  Show that the Hamiltonian is
	\eqn{ham2.a}{
		H = \int \ddcx (\pis \pi + \grad \phis \vdot \grad \phi + m^2 \phis \phi).
	}
	Compute the Heisenberg equation of motion for $\phix$ and show that it is indeed the Klein-Gordon equation.
}

\sol{
	The momentum density conjugate to $\phix$ is defined in Peskin \& Schroeder~(2.4):
	\eq{
		\pix \equiv \pdv{\cL}{\phidx}
	}
	Here, $\cL$ is the integrand of Eq.~\refeq{action2}.  Expanding its first term yields
	\eqn{lagr2.a}{
		\cL = \phid \phisd - \grad \phi \vdot \grad \phis,
	}
	so then
	\aln{
		\ans{\pix\ }&\ans{= \phisd, } &
		\ans{\pisx\ }&\ans{= \phid, } \label{moms2.a}
	}
	where $\pisx$ is the momentum conjugate to $\phisx$.  The canonical commutation relations follow from Peskin \& Schroeder~(2.20):
	\ans{\aln{ \label{comm2.a}
		[\phivx, \pivy] &= [\phisvx, \pisvy] = i \,\del^3(\vx - \vy), \\
		[\phivx, \phivy] &= [\phisvx, \phisvy] = 0, \\
		[\pivx, \pivy] &= [\pisvx, \pisvy] = 0, \\
		[\phivx, \pisvy] &= [\phivx, \phisvy] = [\pivx, \pisvy] = 0.
	}}% 
	
	The Hamiltonian is given in general for a single field by Peskin \& Schroeder~(2.5),
	\eq{
		H = \int \ddcx \paren{ \pix \, \phidx - \cL }.
	}
	For the two fields $\phix$ and $\phisx$, this becomes
	\al{
		H &= \int \ddcx \paren{ \pix \, \phidx + \pisx \, \phisdx - \cL } \\
		&= \int \ddcx \paren{ \pi \phid + \pis \phisd - \phid \phisd + \grad \phi \vdot \grad \phis + m^2 \phis \phi } \\
		&= \int \ddcx \paren{ \pi \pis + \phid \phisd - \phid \phisd + \grad \phi \vdot \grad \phis + m^2 \phis \phi } \\
		&= \ans{ \int \ddcx \paren{ \pis \pi + \grad \phi \vdot \grad \phis + m^2 \phis \phi }, }
	}
	where we have used Eqs.~\refeq{lagr2.a} and \refeq{moms2.a} as well as the commutation relations.  So we have proven Eq.~\refeq{ham2.a}. \qed
	
	The Heisenberg equation of motion is Peskin \& Schroeder~(2.44),
	\eq{
		i \pdv{\cO}{t} = [\cO, H],
	}
	where $\cO$ is an arbitrary operator.  Then
	\al{
		i \pdv{\phix}{t} &= [\phix, H] \\
		&= \brac{ \phixt, \int \ddcxp \pisxp \, \pixp } + \brac{ \phixt, \int \ddcxp \grad' \phixp \vdot \grad' \phisxp } \\
		&\phantom{mmmmmmmmmmmmmmmmmmmmmmmmmmmmmmm} + m^2 \brac{ \phixt, \int \ddcxp \phisxp \, \phixp } \\
		&= \brac{ \phixt, \int \ddcxp \pisxp \, \pixp }
		= i \int \ddcxp \del^3(\vx - \vx') \, \pisxp
		= i \pisx, \\[2ex]
		i \pdv{\phisx}{t} &= [\phisx, H] \\
		&= \brac{ \phisxt, \int \ddcxp \pisxp \, \pixp }
		= i \int \ddcxp \del^3(\vx - \vx') \, \pixp
		= i \pix,
	}
	\al{
		i \pdv{\pix}{t} &= [\pix, H] \\
		&= \brac{ \pixt, \int \ddcxp \pisxp \, \pixp } + \brac{ \pixt, \int \ddcxp \grad' \phixp \vdot \grad' \phisxp } \\
		&\phantom{mmmmmmmmmmmmmmmmmmmmmmmmmmmmmmm} + m^2 \brac{ \pixt, \int \ddcxp \phisxp \, \phixp } \\
		&= -i \int \ddcxp \brac{ \grad' \del^3(\vx - \vx') \vdot \grad' \phisxp + m^2 \del^3(\vx - \vx') \, \phisxp } \\
		&= -i \int \ddcxp \brac{ - \del^3(\vx - \vx') \vdot {\nabla'}^2 \phisxp + m^2 \del^3(\vx - \vx') \, \phisxp }
		= -i (-\laplacian + m^2) \, \phisx, \\[2ex]
		i \pdv{\pisx}{t} &= [\pisx, H] \\
		&= -i \int \ddcxp \brac{ \grad' \phixp \vdot \grad' \del(\vx - \vx') + m^2 \del^3(\vx - \vx') \, \phixp } \\
		&= -i \int \ddcxp \brac{ - \del^3(\vx - \vx') \vdot {\nabla'}^2 \phixp + m^2 \del^3(\vx - \vx') \, \phixp }
		= -i (-\laplacian + m^2) \, \phix.
	}
	Thus we have obtained
	\al{
		\pdv{\phix}{t} &= \pisx, &
		\pdv{\phisx}{t} &= \pix, &
		\pdv{\pix}{t} &= (\laplacian - m^2) \,\phisx, &
		\pdv{\pisx}{t} &= (\laplacian - m^2) \,\phix.
	}
	Combining these results yields
	\ans{\al{
		\pdv[2]{\phi}{t} &= (\laplacian - m^2) \phi, &
		\pdv[2]{\phis}{t} &= (\laplacian - m^2) \phis,
	}}%
	which is the Klein-Gordon equation and its complex conjugate, as we sought to show. \qed
}



\prob{}{
	Diagonalize $H$ by introducing creation and annihilation operators.  Show that the theory contains two sets of particles of mass $m$.
}

\sol{
	Peskin \& Schroeder~(2.21) gives the Klein-Gordon equation in the momentum basis,
	\eq{
		\paren{ \pdv[2]{t} + \vp^2 + m^2 } \phipt = 0.
	}
	This is the same as the harmonic oscillator equation of motion.  It has solutions~\cite{SHM}
	\eq{
		\phipt = \Avp \, e^{i \omgp t} + \Bvp \, e^{-i \omgp t},
	}
	where $\omgp = \sqrt{\vp^2 + m^2}$ as in Peskin \& Schroeder Eq.~(2.22), and $\Avp$ and $\Bvp$ are arbitrary functions of $\vp$.  The complex conjugate of this solution is
	\eq{
		\phispt = \Bsvp \, e^{i \omgp t} + \Asvp \, e^{-i \omgp t}.
	}
	
	The field $\phi$ in the position basis can be expanded as~\cite[p.~20]{Peskin},
	\eq{
		\phixt = \int \ddpf e^{i \vp \vdot \vx} \, \phipt.
	}
	so we can write, as explained in class,
	\al{
		\phivx &= \int \ddpf \frac{1}{\sqrt{2 \omgp}} \paren{ \ap e^{i \vp \vdot \vx} + \bpdag e^{-i \vp \vdot \vx} }, &
		\phisvx &= \int \ddpf \frac{1}{\sqrt{2 \omgp}} \paren{ \bp e^{i \vp \vdot \vx} + \apdag e^{-i \vp \vdot \vx} },
	}
	where $\apdag, \bpdag$~($\ap, \bp$) are creation~(annihilation) operators.  By analogy to Eq.~(2.26) of Peskin \& Schroeder, we can also write
	\al{
		\pivx &= -i \int \ddpf \sqrt{\frac{\omgp}{2}} \paren{ \bp e^{i \vp \vdot \vx} - \apdag e^{-i \vp \vdot \vx} }, &
		\pisvx &= -i \int \ddpf \sqrt{\frac{\omgp}{2}} \paren{ \ap e^{i \vp \vdot \vx} - \bpdag e^{-i \vp \vdot \vx} }.
	}
	Simplifying these expressions as in their Eqs.~(2.27) and (2.28), we have
	\aln{
		\phivx &= \int \ddpf \frac{1}{\sqrt{2 \omgp}} \paren{ \ap + \bnpdag } e^{i \vp \vdot \vx}, &
		\phisvx &= \int \ddpf \frac{1}{\sqrt{2 \omgp}} \paren{ \bp + \anpdag } e^{i \vp \vdot \vx}, \label{things2.b1} \\
		\pivx &= -i \int \ddpf \sqrt{\frac{\omgp}{2}} \paren{ \bp - \anpdag } e^{i \vp \vdot \vx}, &
		\pisvx &= -i \int \ddpf \sqrt{\frac{\omgp}{2}} \paren{ \ap - \bnpdag } e^{i \vp \vdot \vx}. \label{things2.b2}
	}
	Also generalizing their Eq.~(2.24),
	\al{
		[\ap, \appdag] &= [\bp, \bppdag] = (2\pi)^3 \, \del^3(\vp - \vp'), &
		[\ap, \bppdag] &= [\bp, \appdag] = 0.
	}
	
	Feeding Eqs.~\refeq{things2.b1} and \refeq{things2.b2} into Eq.~\refeq{ham2.a} yields
	\eq{
		H = \int \ddcx \int \ddppfs \, e^{i (\vp + \vp') \vdot \vx} \brac{ -\frac{\sqrt{\omgp \omgpp}}{2} \paren{ \ap - \bnpdag } \paren{ \bpp - \anppdag } + \frac{-\vp \vdot \vp' + m^2}{2 \sqrt{\omgp \omgpp}} \paren{ \ap + \bnpdag } \paren{ \bpp + \anppdag } }.
	}
	Using the delta function identity~\cite{Dirac}
	\eq{
		\del(x - a) = \frac{1}{2\pi} \intii e^{i p (x - a)} \ddp,
	}
	this becomes
	\al{
		H &= \int \ddppfc \, \del^3(\vp + \vp') \brac{ -\frac{\sqrt{\omgp \omgpp}}{2} \paren{ \ap - \bnpdag } \paren{ \bpp - \anppdag } + \frac{-\vp \vdot \vp' + m^2}{2 \sqrt{\omgp \omgpp}} \paren{ \ap + \bnpdag } \paren{ \bpp + \anppdag } } \\
		&= \int \ddpf \brac{ -\frac{\omgp}{2} \paren{ \ap - \bnpdag } \paren{ \bnp - \apdag } + \frac{\vp^2 + m^2}{2 \omgp} \paren{ \ap + \bnpdag } \paren{ \bnp + \apdag } } \\
		&= \int \ddpf \frac{\omgp}{2} \brac{ \ap \bnp + \ap \apdag + \bnpdag \bnp + \bnpdag \apdag - \paren{ \ap \bnp - \ap \apdag - \bnpdag \bnp + \bnpdag \apdag } } \\
		&= \int \ddpf \, \omgp \paren{ \ap \apdag + \bnpdag \bnp }
		= \int \ddpf \, \omgp \paren{ \apdag \ap + \bpdag \bp + [\ap, \appdag] }.
	}
	Ignoring the infinite constant term~\cite[p.~21]{Peskin}, we have
	\eqn{diagham2.b}{
		\ans{ H = \int \ddpf \, \omgp \paren{ \apdag \ap + \bpdag \bp }. }
	}
	
	To show that the theory contains two sets of particles of mass $m$, we evaluate the commutators~\cite[p.~22]{Peskin}:
	\al{
		[H, \apdag] &= \brac{ \int \ddppf \omgpp \appdag \app, \apdag } = \omgp \apdag, &
		[H, \ap] &= \brac{ \int \ddppf \omgpp \appdag \app, \ap } = -\omgp \ap, \\
		[H, \bpdag] &= \brac{ \int \ddppf \omgpp \bppdag \bpp, \bpdag } = \omgp \bpdag, &
		[H, \bp] &= \brac{ \int \ddppf \omgpp \bppdag \bpp, \bp } = -\omgp \bp.
	}
	Then we can define the eigenstates of the Hamiltonian by
	\eq{
		(\apdag)^{\na} \,(\bpdag)^{\nb} \koo \equiv \knanb,
	}
	which have eigenvalues $(\na + \nb) \omgp$.  So the expression for the Hamiltonian in Eq.~\refeq{diagham2.b} is diagonal in the occupation number basis $\{ \knanb \}$, where $\na$ indicates the number of particles created with $\apdag$ and $\nb$ the number of antiparticles created with $\bpdag$.  The ground state is $\koo$; it has zero energy since its eigenvalue is zero.  Since each operation of $\apdag$ or $\bpdag$ imparts energy $\omgp = \sqrt{\vp^2 + m^2}$ to the system, and each operation of $\ap$ or $\bp$ removes energy $\omgp = \sqrt{\vp^2 + m^2}$ from the system, we can conclude that each of the two sets of operators corresponds to a set of particles of mass $m$. \qed
}



\prob{}{ \label{2.2(c)}
	Rewrite the conserved charge
	\eqn{Q2.c}{
		Q = \int \ddcx \frac{i}{2} (\phis \pis - \pi \phi)
	}
	in terms of creation and annihilation operators, and evaluate the charge of the particles of each type.
}

\sol{
	Applying Eqs.~\refeq{things2.b1} and \refeq{things2.b2}, we find
	\al{
		Q &= \frac{1}{4} \int \ddcx \int \ddppfs \, e^{i \vp \vdot \vx} \brac{ \paren{ \bp + \anpdag } \paren{ \app - \bnppdag } - \paren{ \bp - \anpdag } \paren{ \app + \bnppdag } } \\
		&= \frac{1}{4} \int \ddpf \brac{ \paren{ \bp + \anpdag } \paren{ \anp - \bpdag } - \paren{ \bp - \anpdag } \paren{ \anp + \bpdag } } \\
		&= \frac{1}{4} \int \ddpf \brac{ \paren{ \bp \anp - \bp \bpdag + \anpdag \anp - \anpdag \bnpdag } - \paren{ \bp \anp + \bp \bpdag - \anpdag \anp - \anpdag \bpdag} } \\
		&= \frac{1}{2} \int \ddpf \paren{ \anpdag \anp - \bp \bpdag }
		= \frac{1}{2} \int \ddpf \paren{ \ap \apdag - \bp \bpdag } \\
		&= \frac{1}{2} \int \ddpf \paren{ \ap \apdag - [\ap, \apdag] - \bpdag \bp - [\bp, \bpdag] }
		= \ans{ \frac{1}{2} \int \ddpf \paren{ \apdag \ap - \bpdag \bp }, }
	}
	where in the fifth line we have used $a = (q + i p) / \sqrt{2}$ and $\adag = (q - i p)$~\cite{Operators}.  The particles associated with $\apdag \ap$ must have positive charge, since $\apdag \ap$ represents their number and has the same sign as the conserved charge.  Similarly, the antiparticles associated with $\bpdag \bp$ must have negative charge.
}



\prob{}{
	Consider the case of two complex Klein-Gordon fields with the same mass.  Label the fields as $\phiax$, where $a = 1, 2$.  Show that there are now four conserved charges, one given by the generalization of part~\ref{2.2(c)}, and the other three given by
	\eq{
		\Qi = \int \ddcx \frac{i}{2} (\phias \sigiab \pibs - \pia \sigiab \phib),
	}
	where $\sigi$ are the Pauli sigma matrices.  Show that these three charges have the commutation relations of angular momentum ($SU(2)$).  Generalize these results to the case of $n$ identical complex scalar fields.
}

\sol{
	Generalizing Eq.~\refeq{action2}, the Lagrangian for the two Klein-Gordon fields is
	\eqn{lagr2.d}{
		\cL = \ptsm \phiqs \ptm \phiq - m^2 \phiqs \phiq + \ptsm \phiws \ptm \phiw - m^2 \phiws \phiw.
	}
	The conserved charge is given in general by Peskin \& Schroeder~(2.12) and (2.13),
	\al{
		Q &\equiv \int_\text{all space} \jo \ddcx, &
		\where \jmx &= \pdv{\cL}{(\ptsm\phi)} \Del\phi - \Jm,
	}
	where $\Jm$ is a 4-divergence that arises when transforming the Lagrangian as in Peskin \& Schroeder~(2.10):
	\eq{
		\cLx \to \cLx + \alp \ptsm \Jmx.
	}
	
	For the first conserved charge, we note that the Lagrangian in Eq.~\refeq{lagr2.d} is invariant under the transformations $\phia \to e^{i \alp} \phia$~\cite[p.~18]{Peskin}:
	\eq{
		\cL \to \suma \brac{ \ptsm(e^{-i \alp} \phias) \, \ptm(e^{i \alp} \phia) - m^2 e^{-i \alp} \phias \, e^{i \alp} \phia }
		= \cL,
	}
	so $\Jmx = 0$.  The relevant infinitesimal transformations are found by generalizing Peskin \& Schroeder~(2.15):
	\aln{ \label{infinitesimal2.d}
		\alp \, \Del\phia &= i \alp \phia, &
		\alp \, \Del\phias &= -i \alp \phias.
	}
	These transformations yield the conserved current
	\eq{
		\jm = -\frac{1}{2} \suma \paren{ \pdv{\cL}{(\ptsm\phia)} \Del\phia + \pdv{\cL}{(\ptsm\phias)} \Del\phias }
		= -\frac{i}{2} \suma \paren{ \phia \ptm\phias -\phias \ptm\phia },
	}
	where we have arbitrarily chosen the overall constant~\cite[p.~18]{Peskin}.  Then, generalizing Eq.~\refeq{moms2.a}, the corresponding conserved charge is
	\al{
		\Qo &= \int \ddcx \jo
		= -\frac{i}{2} \int \ddcx \suma \paren{ \phia \phiasd - \phias \phiad }
		= -\frac{i}{2} \int \ddcx \suma \paren{ \phia \pia - \phias \pias } \\
		&= \ans{ \int \ddcx \frac{i}{2} \paren{ \phiqs \piqs - \phiq \piq + \phiws \piws - \phiw \piw }, }
	}
	which is the generalization of Eq.~\refeq{Q2.c} for two fields.
	
	From the problem statement, we make the ansatz that $\cL$ is also invariant under rotations, $\phi \to e^{i \alpi \sigi / 2} \phi$ where $\phi = (\phiq, \phiw)$ is a two-component spinor, from Peskin \& Schroeder~(15.19) and (15.20).  To verify,
	\eq{
		\cL \to \suma \brac{ \ptsm(e^{-i \alpi \sigi / 2} \phias) \, \ptm(e^{i \alpi \sigi / 2} \phia) - m^2 e^{-i \alpi \sigi / 2} \phias \, e^{i \alpi \sigi / 2} \phia }
		= \cL.
	}
	Again, $\Jm = 0$.  By analogy with Eq.~\refeq{infinitesimal2.d}, the infinitesimal transformations are
	\al{
		\alpi \, \Del\phi &= \frac{i}{2} \phi \alpi \sigi, &
		\alpi \, \Del\phia &= -\frac{i}{2} \phia \alpi \sigi.
	}
	We have the conserved currents
	\eq{
		\jmi = -\paren{ \pdv{\cL}{(\ptsm\phi)} \Del\phi + \pdv{\cL}{(\ptsm\phis)} \Del\phis }
		= \frac{i}{2} \paren{ \phis \sigi \, \ptm\phi - \phi \sigi \, \ptm\phis },
	}
	where we have chosen a different overall constant factor.  Then the corresponding conserved charges are
	\eq{
		\Qi = \int \ddcx \joi
		= \int \ddcx \frac{i}{2} \paren{ \phis \sigi \phid - \phi \sigi \phisd }
		= \int \ddcx \frac{i}{2} \paren{ \phis \sigi \pis - \phi \sigi \pi }
		= \ans{ \int \ddcx \frac{i}{2} (\phias \sigiab \pibs - \pia \sigiab \phib) },
	}
	as desired. \qed

	If $\Qi$ have the commutation relations of angular momentum, then we want to show $[\Qi, \Qj] = i \epsijk \Qk$~\cite[p.~158]{Sakurai}.  Generalizing Eq.~\refeq{comm2.a} to two fields, the canonical commutation relations are
	\al{
		[\phiavx, \pibvy] &= [\phiasvx, \pibsvy] = i \,\delab \,\del^3(\vx - \vy), \\
		[\phiavx, \phibvy] &= [\phiasvx, \phibsvy] = 0, \\
		[\piavx, \pibvy] &= [\piasvx, \pibsvy] = 0, \\
		[\phiavx, \pibsvy] &= [\phiavx, \phibsvy] = [\piavx, \pibsvy] = 0.
	}
	The commutation relations among the Pauli matrices are $[\sigi, \sigj] = 2 i \epsijk \sigk$~\cite[p.~165]{Sakurai}.
	
	Note that
	\aln{
		[\Qi, \Qj] &= \brac{ \int \ddcx \frac{i}{2} [ \phiasx \, \sigiab \, \pibsx - \piax \, \sigiab \, \phibx ], \int \ddcxp \frac{i}{2} [ \phicsxp \, \sigjcd \, \pidsxp - \picxp \, \sigjcd \, \phidxp ] } \notag \\
		&= -\frac{1}{4} \int \ddcx \int \ddcxp \bigg\{ \brac{ \phiasx \, \sigiab \, \pibsx, \phicsxp \, \sigjcd \, \pidsxp } - \brac{ \phiasx \, \sigiab \, \pibsx, \picxp \, \sigjcd \, \phidxp } \notag \\
		&\phantom{mmmmmmm} - \brac{ \piax \, \sigiab \, \phibx, \phicsxp \, \sigjcd \, \pidsxp } + \brac{ \piax \, \sigiab \, \phibx, \picxp \, \sigjcd \, \phidxp } \bigg\}. \label{fourterms}
	}
	For the first term in Eq.~\refeq{fourterms},
	\al{
		\int \ddcx \int \ddcxp &\brac{ \phiasx \, \sigiab \, \pibsx, \phicsxp \, \sigjcd \, \pidsxp } \\
		&= \int \ddcx \int \ddcxp \left\{ \phiasx \, [ \phicsxp \, \pibsx - i \, \delbc \, \del^3(x - x') ] \, \pidsxp \, \sigiab \, \sigjcd \right. \\
		&\phantom{mmmmmmmmmmmmm} - \left. \phicsxp \, [ \phiasx \, \pidsxp - i \, \delad \, \del^3(x - x') ] \, \pibsx \, \sigjcd \, \sigiab \right\} \\
		&= \int \ddcx \int \ddcxp \left[ \phiasx \, \pibsx \, \phicsxp \, \pidsxp \, \sigiab \, \sigjcd - i \, \delbc \, \del^3(x - x') \,  \phiasx \, \pidsxp \, \sigiab \, \sigjcd \right. \\
		&\phantom{mmmmmmmmmmmmm} - \left. \phicsxp \, \pidsxp \, \phiasx \, \pibsx \, \sigjcd \, \sigiab + i \, \delad \, \phicsxp \, \del^3(x - x') \, \pibsx \, \sigjcd \, \sigiab \right] \\
		&= \int \ddcx \int \ddcxp \, i \del^3(x - x') \brac{ \phicsxp \, \pibsx \, (\sigj \sigi)_{cb} - \phiasx \, \pidsxp \, (\sigi \sigj)_{ad} } \\
		&= \int \ddcx \, i \brac{ \phiasx \, \pibsx \, (\sigj \sigi)_{ab} - \phiasx \, \pibsx \, (\sigi \sigj)_{ab} }
		= -\int \ddcx \, i \, \phiasx \, \pibsx \, [\sigi, \sigj]_{ab} \\
		&= 2 \epsijk \int \ddcx \, \phiasx \, \sigkab \, \pibsx,
	}
	where we have interchanged $a \leftrightarrow c$ and $b \leftrightarrow d$ in one of the terms that canceled, and we have also relabeled indices in the second to last line.
	
	The second and third terms of Eq.~\refeq{fourterms} both vanish since they consist entirely of fields that commute.  The fourth term may be found using similar arithmetic as for the first:
	\eq{
		\int \ddcx \int \ddcxp \brac{ \piax \, \sigiab \, \phibx, \picxp \, \sigjcd \, \phidxp } = -2 \epsijk \int \ddcx \, \phiax \, \sigkab \, \pibx.
	}
	Feeding these results into Eq.~\refeq{fourterms}, we have
	\eq{
		[\Qi, \Qj] = -\frac{\epsijk}{2} \int \ddcx \brac{ \phiasx \, \sigkab \, \pibsx - \phiax \, \sigkab \, \pibx }
		= \ans{ i \epsijk \Qk },
	}
	as we sought to show. \qed
	
%	\hl{Generalization?}
}