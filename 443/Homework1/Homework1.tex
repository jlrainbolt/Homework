\documentclass[11pt]{article}
\usepackage{homework}

\classname{443}
\homeworknum{1}


\begin{document}

% Environments

\newcommand{\state}[2]{\begin{statement}{#1} #2 \end{statement}}
\newcommand{\prob}[2]{\begin{problem}{#1} #2 \end{problem}}
\newcommand{\subprob}[1]{\begin{subproblem} #1 \end{subproblem}}
\newcommand{\sol}[1]{\begin{solution} #1 \end{solution}}
\newcommand{\fig}[2]{\begin{figure} \centering #2  \label{#1} \end{figure}}

\newcommand{\makebib}{
	\vfill
	\color{black}
	\bibliography{references}{}
	\bibliographystyle{lucas_unsrt}
}
	

% Implication

\newcommand{\qwhere}{\quad \text{where} \quad}
\newcommand{\qimplies}{\quad \implies \quad}
\newcommand{\impliesq}{\implies \quad}



% Brackets

\newcommand{\paren}[1]{\left( #1 \right)}
\newcommand{\brac}[1]{\left[ #1 \right]}


% Greek

\newcommand{\alp}{\alpha}
\newcommand{\bet}{\beta}
\newcommand{\gam}{\gamma}
\newcommand{\del}{\delta}
\newcommand{\eps}{\epsilon}
\newcommand{\zet}{\zeta}
\newcommand{\tht}{\theta}
\newcommand{\kap}{\kappa}
\newcommand{\lam}{\lambda}
\newcommand{\sig}{\sigma}
\newcommand{\ups}{\upsilon}
\newcommand{\omg}{\omega}

\newcommand{\Gam}{\Gamma}
\newcommand{\Del}{\Delta}
\newcommand{\Tht}{\Theta}
\newcommand{\Lam}{\Lambda}
\newcommand{\Sig}{\Sigma}
\newcommand{\Omg}{\Omega}
% Problem 1

\newcommand{\Psii}{\Psi^i}
\newcommand{\Psiix}{\Psii(x)}

\newcommand{\Pii}{\Pi^i}

\newcommand{\Phii}{\Phi^i}
\newcommand{\Phiix}{\Phii(x)}
\newcommand{\PhiN}{\Phi^N}
\newcommand{\PhiNx}{\PhiN(x)}
\newcommand{\Phiq}{\Phi^1}
\newcommand{\Phiw}{\Phi^2}

\newcommand{\ddcx}{\dd[3]{x}}

\newcommand{\delij}{\del^{i j}}
\newcommand{\delkl}{\del^{k l}}
\newcommand{\delil}{\del^{i l}}
\newcommand{\deljk}{\del^{j k}}
\newcommand{\delik}{\del^{i k}}
\newcommand{\deljl}{\del^{j l}}

\newcommand{\DF}{D_F}

\newcommand{\sigx}{\sig(x)}

\newcommand{\pii}{\pi^i}
\newcommand{\pij}{\pi^j}
\newcommand{\pik}{\pi^k}
\newcommand{\pil}{\pi^l}
\newcommand{\piix}{\pi(x)}

\newcommand{\pq}{p_1}
\newcommand{\pw}{p_2}
\newcommand{\pe}{p_3}
\newcommand{\pr}{p_4}

\newcommand{\vp}{\vb{p}}
\newcommand{\vpsi}{\vp_i}

\newcommand{\mpi}{m_\pi}

\state{(Jackson 9.8)}{\ 
	%\emph{Hint:} The electromagnetic angular momentum density comes from more than the transverse (radiation zone) components of the fields.
}

%
%	Jackson 9.8(a)
%

\prob{}{
	Show that a classical oscillating electric dipole $\vp$ with fields given by
	\aln{ \label{fields1}
		\vH &= \frac{c k^2}{4\pi} (\nh \cross \vp) \frac{e^{i k r}}{r} \paren{ 1 - \frac{1}{i k r} }, &
		\vE &= \frac{1}{4\pi \epso} \curly{ k^2 (\nh \cross \vp) \cross \nh \frac{e^{i k r}}{r} + [ 3 \nh (\nh \vdot \vp) - \vp ] \paren{ \frac{1}{r^3} - \frac{i k}{r^2} } e^{i k r} },
	}
	radiates electromagnetic angular momentum to infinity at the rate
	\eq{
		\dv{\vL}{t} = \frac{k^3}{12 \pi \epso} \Im[ \vp^* \cross \vp ].
	}
	\vfix
}

\sol{
	According to Jackson~(9.20), the time-averaged angular momentum density is
	\eq{
		\vl = \frac{\Re[ \vx \cross (\vE \cross \vHs)}{2 c^2}.
	}
	One of the vector identities on the inside cover of Jackson is $\vaa \cross (\vbb \cross \vcc) = (\vaa \vdot \vcc) \vbb - (\vaa \vdot \vbb) \vcc$, so
	\eqn{l1}{
		\vl = \frac{(\vx \vdot \vHs) \vE - (\vx \vdot \vE) \vHs}{2 c^2}.
	}
	From Eq.~\refeq{fields1}, note that
	\eq{
		\vx \vdot \vHs \propto \vx \vdot (\nh \cross \vps)
		= \vps \vdot (\vx \cross \nh)
		= \vO,
	}
	where we have used the identity $\vaa \vdot (\vbb \cross \vcc) = \vcc \vdot (\vaa \cross \vbb)$ and the fact that $\nh$ points in the $\vx$ direction.  For $\vx \vdot \vE$, note that
	\al{
		\vx \vdot [ (\nh \cross \vp) \cross \nh ] &= -\vx \vdot [ \nh \cross (\nh \cross \vp) ]
		= -\vx \vdot [ (\nh \vdot \vp) \nh - (\nh \vdot \nh) \vp ]
		= -(\nh \vdot \vp) (\vx \vdot \nh) + \vx \vdot \vp \\
		&= -r (\nh \vdot \vp) + \vx \vdot \vp
		= \vx \vdot \vp - \vx \vdot \vp
		= 0, \\[1.5ex]
		\vx \vdot [ 3 \nh (\nh \vdot \vp) - \vp ] &= 3 (\vx \vdot \nh) (\nh \vdot \vp) - \vx \vdot \vp
		= 3r (\nh \vdot \vp) - \vx \vdot \vp
		= 3(\vx \vdot \vp) - \vx \vdot \vp
		= 2(\vx \vdot \vp),
	}
	since $\abs{\vx} = r$ and $\vx = r \,\nh$.  Then
	\eq{
		\vx \vdot \vE = \frac{1}{2\pi \epso} (\vx \vdot \vp) \paren{ \frac{1}{r^3} - \frac{i k}{r^2} } e^{i k r}
		= \frac{1}{2\pi \epso} (\nh \vdot \vp) \paren{ \frac{1}{r^2} - \frac{i k}{r} } e^{i k r}.
	}
	
	With these substitutions, Eq.~\refeq{l1} becomes
	\al{
		\vl &= -\frac{(\vx \vdot \vE) \vHs}{c^2}
		= -\frac{1}{4\pi \epso c^2} (\nh \vdot \vp) \paren{ \frac{1}{r^2} - \frac{i k}{r} } e^{i k r} \frac{c k^2}{4\pi} (\nh \cross \vps) \frac{e^{-i k r}}{r} \paren{ 1 + \frac{1}{i k r} } \\
		&= -\frac{k^2}{16\pi^2 \epso c r} (\nh \vdot \vp) (\nh \cross \vps) \paren{ \frac{1}{r^2} - \frac{i k}{r} } \paren{ 1 - \frac{i}{k r} }
		= -\frac{k^2}{16\pi^2 \epso c} (\nh \vdot \vp) (\nh \cross \vps) \paren{ \frac{1}{r^2} - \frac{i}{k r^3} - \frac{i k}{r} - \frac{1}{r^2} } \\
		&= -\frac{i k^2}{16\pi^2 \epso c r} (\nh \vdot \vp) (\nh \cross \vps) \paren{ \frac{1}{k r^3} + \frac{k}{r^2} }
		= \frac{i k^3}{16\pi^2 \epso c r^2} (\nh \vdot \vp) (\nh \cross \vps) \paren{ \frac{1}{k^2 r^2} + 1 }.
	}
	
	Let $\vL$ be the angular momentum radiated to a distance $R$.  Then
	\eq{
		\vL = \int_R \vl(r) \ddcx
		= \intopi \intotp \intoR \vl(r) \,r^2 \sin\tht \ddr \ddphi \dd\tht,
	}
	and the time derivative is
	\aln{
		\dv{\vL}{t} &= \dv{t}(\intopi \intotp \intoR \vl(r) \,r^2 \sin\tht \ddr \ddphi \dd\tht)
		= \dv{r}{t} \dv{r}(\intopi \intotp \intoR \vl(r) \,r^2 \sin\tht \ddr \ddphi \dd\tht) \notag \\
		&= c \intopi \intotp \vl(r) \,r^2 \sin\tht \ddphi \dd\tht
		= \frac{i k^3}{16\pi^2 \epso} \paren{ \frac{1}{k^2 r^2} + 1 } \intopi \intotp (\nh \vdot \vp) (\nh \cross \vps) \sin\tht \ddphi \dd\tht. \label{dLdt}
	}
	Note that
	\eq{
		[ (\nh \vdot \vp) (\nh \cross \vps) ]_i = \sumje n_j p_j (\nh \cross \vps)_i
		= \sumje \sumke \sumle \epsikl n_j p_j n_k p_l^*,
	}
	so
	\eq{
		\dv{L_i}{t} \propto \sumje \sumke \sumle \epsikl p_j p_l^* \int n_j p_k \ddOmg
		= \sumje \sumke \sumle \epsikl p_j p_l^* \frac{4\pi}{3} \del_{jk}
		= \frac{4\pi}{3} \epsikl p_k p_l^*
		= \frac{4\pi}{3} (\vp \cross \vps)_i,
	}
	where we have used Jackson~(9.47), $\int n_\bet n_\gam \ddOmg = 4\pi \del_{\bet \gam} / 3$.  Making this substitution into Eq.~\refeq{dLdt},
	\eq{
		\dv{\vL}{t} = \frac{i k^3}{6\pi \epso} \paren{ \frac{1}{k^2 r^2} + 1 } (\vp \cross \vps).
	}
	Taking the limit as $r \to \infty$, we find
	\eqn{ans1a}{
		\dv{\vL}{t} = \Re\!\brac{ \frac{i k^3}{12\pi \epso} (\vp \cross \vps) }
		= \Re\!\brac{ -\frac{i k^3}{12\pi \epso} (\vps \cross \vp) }
		= \ans{ \frac{k^3}{12\pi \epso} \Im[ \vps \cross \vp ], }
	}
	as desired. \qed
}

%
%	Jackson 9.8(b)
%

\prob{}{
	What is the ratio of angular momentum radiated to energy radiated?  Interpret.
}

\sol{
	According to Jackson~(9.24), the total power radiated by an oscillating electric dipole $\vp$ is
	\eq{
		P = \dv{E}{t}
		= \frac{c^2 \Zo k^4}{12 \pi} \abs{\vp}^2.
	}
	Then the ratio of angular momentum radiated to energy radiated is
	\eq{
		\frac{\dv*{\vL}{t}}{\dv*{E}{t}} = \frac{k^3}{12\pi \epso} \Im[ \vps \cross \vp ] \frac{12 \pi}{c^2 \Zo k^4 \abs{\vp}^2}
		= \frac{1}{\epso} \Im[ \vps \cross \vp ] \frac{1}{c^2 \Zo k \abs{\vp}^2}
		= \ans{ \frac{\Im[ \vps \cross \vp ]}{\omg \abs{\vp}^2}, }
	}
	where we have used $\Zo = \sqrt{\muo / \epso} = 1 / \sqrt{\epso^2 c^2} = 1 / \epso c$, $c^2 = 1 / (\epso \muo)$, and $\omg = k c$.
	
	In the limit of high frequency, $(\dv*{\vL}{t}) / (\dv*{E}{t}) \to 0$.  In this scenario, the energy radiated dominates over the angular momentum radiated.  Likewise, in the limit of low frequency, $(\dv*{\vL}{t}) / (\dv*{E}{t}) \to \infty$, meaning that angular momentum radiation dominates.  This is sensible because rotational kinetic energy $E \propto \omg^2$, while angular momentum $L \propto \omg$.
}

%
%	Jackson 9.8(c)
%

\prob{}{
	For a charge $e$ rotating in the $xy$ plane at radius $a$ and angular speed $\omg$, show that there is only a $z$ component of radiated angular momentum with magnitude $\dv*{\Lz}{t} = e^2 k^3 a^2 / 6 \pi \epso$.  What about a charge oscillating along the $z$ axis?
}

\sol{
	We know from Homework~5 that the position of a point charge rotating counterclockwise in the $xy$ plane is
	\eq{
		\vx(t) = a \cos(\omg t) \,\vx + a \sin(\omg t) \,\yh.
	}
	\clearpage
	Then the charge distribution is
	\eq{
		\rho(\vx, t) = e \del[ x - a \cos(\omg t) ] \,\del[ y - a \sin(\omg t) ] \,\del(z).
	}
	
	According to Jackson~(4.8), the dipole moment is defined
	\eq{
		\vp = \int \vx' \,\rho(\vx') \ddcxp.
	}
	The components of $\vp$ for the point charge are then
	\al{
		\px &= e \iiint x \,\del[ x - a \cos(\omg t) ] \,\del[ y - a \sin(\omg t) ] \,\del(z) \ddx \ddy \ddz
		= e a \cos(\omg t), \\
		\py &= e \iiint y \,\del[ x - a \cos(\omg t) ] \,\del[ y - a \sin(\omg t) ] \,\del(z) \ddx \ddy \ddz
		= e a \sin(\omg t), \\
		\pz &= e \iiint z \,\del[ x - a \cos(\omg t) ] \,\del[ y - a \sin(\omg t) ] \,\del(z) \ddx \ddy \ddz
		= 0,
	}
	so we can write $\vp = e a \,e^{-i \omg t} (\xh + i\,\yh).$  Substituting into Eq.~\refeq{ans1a},
	\al{
		\dv{\vL}{t} &= \Re\!\brac{ \frac{i k^3}{12\pi \epso} e^2 a^2 e^{-i \omg t} e^{i \omg t} [ (\xh + i\,\yh) \cross (\xh - i\,\yh) ] }
		= \Re\!\brac{ \frac{i e^2 k^3 a^2}{12\pi \epso} (-2i \,\xh \cross \yh) }
		= \Re\!\brac{ \frac{e^2 k^3 a^2}{6\pi \epso} \,\zh } \\
		&= \ans{ \frac{e^2 k^3 a^2}{6\pi \epso} \cos(\omg t) \,\zh, }
	}
	as desired. \qed
	
	A charge oscillating along the $z$ axis with amplitude $a$ has the charge density
	\eq{
		\rho(\vx, t) = e a \,\del(x) \,\del(y) \,\del[ z - \cos(\omg t) ],
	}
	which gives the dipole moment
	\al{
		\px &= e a \iiint x \,\del(x) \,\del(y) \,\del[ z - \cos(\omg t) ] \ddx \ddy \ddz
		= 0, \\
		\py &= e a \iiint y \,\del(x) \,\del(y) \,\del[ z - \cos(\omg t) ] \ddx \ddy \ddz
		= 0, \\
		\pz &= e a \iiint z \,\del(x) \,\del(y) \,\del[ z - \cos(\omg t) ] \ddx \ddy \ddz
		= e a \cos(\omg t).
	}
	In complex notation, $\vp = e a \,e^{-i\omg t} \,\zh$.  Substituting into Eq.~\refeq{ans1a}, we find
	\eq{
		\dv{\vL}{t} = \Re\!\brac{ \frac{i k^3}{12\pi \epso} e^2 a^2 e^{-i \omg t} e^{i \omg t} (\zh \cross \zh) }
		= \ans{ \vO. }
	}
	So we see that a charge undergoing linear motion does not lead to a radiated angular momentum, which is sensible.
}

%
%	Jackson 9.8(d)
%

\prob{}{
	What are the results corresponding to Probs.~{1(a)} and {1(b)} for magnetic dipole radiation?
}

\sol{
	The radiation fields for a magnetic dipole are given by Jackson~(19.35--36),
	\al{
		\vH &= \frac{1}{4\pi} \curly{ k^2 (\nh \cross \vm) \cross \nh \frac{e^{i k r}}{r} + [ 3 \nh (\nh \vdot \vm) - \vm ] \paren{ \frac{1}{r^3} - \frac{i k}{r^2} } e^{i k r} }, &
		\vE &= -\frac{\Zo}{4\pi} k^2 (\nh \cross \vm) \frac{e^{i k r}}{r} \paren{ 1 - \frac{1}{i k r} }.
	}
	\clearpage
	Comparing with Eq.~\refeq{fields1}, we see that $\vH \to -\vE / \Zo$, $\vE \to \Zo \vH$, and $\vp \to \vm / c$ as stated in the book~\cite[p.~413]{Jackson}.  Making these substitutions, the results of Probs.~{1.1(a)} and {(b)} become
	\al{
		\ans{ \dv{\vL}{t}\ }&\ans{= \frac{\muo k^3}{12\pi} \Im[ \vms \cross \vm ], } &
		\ans{ \frac{\dv*{\vL}{t}}{\dv*{E}{t}}\ }&\ans{= \frac{\Im[ \vms \cross \vm ]}{\omg \abs{\vm}^2} }
	}
	where we have used $\mu = 1 / \epso c^2$.
}



\state{The complex scalar field (Peskin \& Schroeder 2.2)}{
	Consider the field theory of a complex-valued scalar field obeying the Klein-Gordon equation.  The action of this theory is
	\eqn{action2}{
		S = \int \ddqx (\ptsm \phis \ptm \phi - m^2 \phis \phi).
	}
	It is easiest to analyze this theory by considering $\phix$ and $\phisx$, rather than the real and imaginary parts of $\phix$, as the basic dynamical variables.
}



\prob{}{
	Find the conjugate momenta to $\phix$ and $\phisx$ and the canonical commutation relations.  Show that the Hamiltonian is
	\eqn{ham2.a}{
		H = \int \ddcx (\pis \pi + \grad \phis \vdot \grad \phi + m^2 \phis \phi).
	}
	Compute the Heisenberg equation of motion for $\phix$ and show that it is indeed the Klein-Gordon equation.
}

\sol{
	The momentum density conjugate to $\phix$ is defined in Peskin \& Schroeder~(2.4):
	\eq{
		\pix \equiv \pdv{\cL}{\phidx}
	}
	Here, $\cL$ is the integrand of Eq.~\refeq{action2}.  Expanding its first term yields
	\eqn{lagr2.a}{
		\cL = \phid \phisd - \grad \phi \vdot \grad \phis,
	}
	so then
	\aln{
		\ans{\pix\ }&\ans{= \phisd, } &
		\ans{\pisx\ }&\ans{= \phid, } \label{moms2.a}
	}
	where $\pisx$ is the momentum conjugate to $\phisx$.  The canonical commutation relations follow from Peskin \& Schroeder~(2.20):
	\ans{\al{
		[\phivx, \pivy] &= [\phisvx, \pisvy] = i \,\del^3(\vx - \vy), \\
		[\phivx, \phivy] &= [\phisvx, \phisvy] = 0, \\
		[\pivx, \pivy] &= [\pisvx, \pisvy] = 0, \\
		[\phivx, \pisvy] &= [\phivx, \phisvy] = [\pivx, \pisvy] = 0.
	}}% 
	
	The Hamiltonian is given in general for a single field by Peskin \& Schroeder~(2.5),
	\eq{
		H = \int \ddcx \paren{ \pix \, \phidx - \cL }.
	}
	For the two fields $\phix$ and $\phisx$, this becomes
	\al{
		H &= \int \ddcx \paren{ \pix \, \phidx + \pisx \, \phisdx - \cL } \\
		&= \int \ddcx \paren{ \pi \phid + \pis \phisd - \phid \phisd + \grad \phi \vdot \grad \phis + m^2 \phis \phi } \\
		&= \int \ddcx \paren{ \pi \pis + \phid \phisd - \phid \phisd + \grad \phi \vdot \grad \phis + m^2 \phis \phi } \\
		&= \ans{ \int \ddcx \paren{ \pis \pi + \grad \phi \vdot \grad \phis + m^2 \phis \phi }, }
	}
	where we have used Eqs.~\refeq{lagr2.a} and \refeq{moms2.a} as well as the commutation relations.  So we have proven Eq.~\refeq{ham2.a}. \qed
	
	The Heisenberg equation of motion is Peskin \& Schroeder~(2.44),
	\eq{
		i \pdv{\cO}{t} = [\cO, H],
	}
	where $\cO$ is an arbitrary operator.  Then
	\al{
		i \pdv{\phix}{t} &= [\phix, H] \\
		&= \brac{ \phixt, \int \ddcxp \pisxp \, \pixp } + \brac{ \phixt, \int \ddcxp \grad' \phixp \vdot \grad' \phisxp } \\
		&\phantom{mmmmmmmmmmmmmmmmmmmmmmmmmmmmmmm} + m^2 \brac{ \phixt, \int \ddcxp \phisxp \, \phixp } \\
		&= \brac{ \phixt, \int \ddcxp \pisxp \, \pixp }
		= i \int \ddcxp \del^3(\vx - \vx') \, \pisxp
		= i \pisx, \\[2ex]
		i \pdv{\phisx}{t} &= [\phisx, H] \\
		&= \brac{ \phisxt, \int \ddcxp \pisxp \, \pixp }
		= i \int \ddcxp \del^3(\vx - \vx') \, \pixp
		= i \pix,
	}
	\al{
		i \pdv{\pix}{t} &= [\pix, H] \\
		&= \brac{ \pixt, \int \ddcxp \pisxp \, \pixp } + \brac{ \pixt, \int \ddcxp \grad' \phixp \vdot \grad' \phisxp } \\
		&\phantom{mmmmmmmmmmmmmmmmmmmmmmmmmmmmmmm} + m^2 \brac{ \pixt, \int \ddcxp \phisxp \, \phixp } \\
		&= -i \int \ddcxp \brac{ \grad' \del^3(\vx - \vx') \vdot \grad' \phisxp + m^2 \del^3(\vx - \vx') \, \phisxp } \\
		&= -i \int \ddcxp \brac{ - \del^3(\vx - \vx') \vdot {\nabla'}^2 \phisxp + m^2 \del^3(\vx - \vx') \, \phisxp }
		= -i (-\laplacian + m^2) \, \phisx, \\[2ex]
		i \pdv{\pisx}{t} &= [\pisx, H] \\
		&= -i \int \ddcxp \brac{ \grad' \phixp \vdot \grad' \del(\vx - \vx') + m^2 \del^3(\vx - \vx') \, \phixp } \\
		&= -i \int \ddcxp \brac{ - \del^3(\vx - \vx') \vdot {\nabla'}^2 \phixp + m^2 \del^3(\vx - \vx') \, \phixp }
		= -i (-\laplacian + m^2) \, \phix.
	}
	Thus we have obtained
	\al{
		\pdv{\phix}{t} &= \pisx, &
		\pdv{\phisx}{t} &= \pix, &
		\pdv{\pix}{t} &= (\laplacian - m^2) \,\phisx, &
		\pdv{\pisx}{t} &= (\laplacian - m^2) \,\phix.
	}
	Combining these results yields
	\ans{\al{
		\pdv[2]{\phi}{t} &= (\laplacian - m^2) \phi, &
		\pdv[2]{\phis}{t} &= (\laplacian - m^2) \phis,
	}}%
	which is the Klein-Gordon equation and its complex conjugate, as we sought to show. \qed
}



\prob{}{
	Diagonalize $H$ by introducing creation and annihilation operators.  Show that the theory contains two sets of particles of mass $m$.
}

\sol{
	Peskin \& Schroeder~(2.21) gives the Klein-Gordon equation in the momentum basis,
	\eq{
		\paren{ \pdv[2]{t} + \vp^2 + m^2 } \phipt = 0.
	}
	This is the same as the harmonic oscillator equation of motion.  It has solutions~\cite{SHM}
	\eq{
		\phipt = \Avp \, e^{i \omgp t} + \Bvp \, e^{-i \omgp t},
	}
	where $\omgp = \sqrt{\vp^2 + m^2}$ as in Peskin \& Schroeder Eq.~(2.22), and $\Avp$ and $\Bvp$ are arbitrary functions of $\vp$.  The complex conjugate of this solution is
	\eq{
		\phispt = \Bsvp \, e^{i \omgp t} + \Asvp \, e^{-i \omgp t}.
	}
	
	The field $\phi$ in the position basis can be expanded as~\cite[p.~20]{Peskin},
	\eq{
		\phixt = \int \ddpf e^{i \vp \vdot \vx} \, \phipt.
	}
	so we can write~\cite[p.~33]{Tong}
	\al{
		\phivx &= \int \ddpf \frac{1}{\sqrt{2 \omgp}} \paren{ \ap e^{i \vp \vdot \vx} + \aspdag e^{-i \vp \vdot \vx} }, &
		\phisvx &= \int \ddpf \frac{1}{\sqrt{2 \omgp}} \paren{ \asp e^{i \vp \vdot \vx} + \apdag e^{-i \vp \vdot \vx} },
	}
	where $\ap$ and $\asp$ are creation and annihilation operators.  By analogy to Eq.~(2.26) of Peskin \& Schroeder, we can also write
	\al{
		\pivx &= -i \int \ddpf \sqrt{\frac{\omgp}{2}} \paren{ \asp e^{i \vp \vdot \vx} - \apdag e^{-i \vp \vdot \vx} }, &
		\pisvx &= -i \int \ddpf \sqrt{\frac{\omgp}{2}} \paren{ \ap e^{i \vp \vdot \vx} - \aspdag e^{-i \vp \vdot \vx} }.
	}
	Simplifying these expressions as in their Eqs.~(2.27) and (2.28), we have
	\aln{
		\phivx &= \int \ddpf \frac{1}{\sqrt{2 \omgp}} \paren{ \ap + \asnpdag } e^{i \vp \vdot \vx}, &
		\phisvx &= \int \ddpf \frac{1}{\sqrt{2 \omgp}} \paren{ \asp + \anpdag } e^{i \vp \vdot \vx}, \label{things2.b1} \\
		\pivx &= -i \int \ddpf \sqrt{\frac{\omgp}{2}} \paren{ \asp - \anpdag } e^{i \vp \vdot \vx}, &
		\pisvx &= -i \int \ddpf \sqrt{\frac{\omgp}{2}} \paren{ \ap - \asnpdag } e^{i \vp \vdot \vx}. \label{things2.b2}
	}
	Also generalizing their Eq.~(2.24),
	\al{
		[\ap, \appdag] &= [\asp, \asppdag] = (2\pi)^3 \, \del^3(\vp - \vp'), &
		[\ap, \asppdag] &= [\asp, \appdag] = 0.
	}
	
	Feeding Eqs.~\refeq{things2.b1} and \refeq{things2.b2} into Eq.~\refeq{ham2.a},
	\eq{
		H = \int \ddcx \int \ddppf e^{i (\vp + \vp') \vdot \vx} \brac{ -\frac{\sqrt{\omgp \omgpp}}{2} \paren{ \ap - \asnpdag } \paren{ \aspp - \anppdag } + \frac{-\vp \vdot \vp' + m^2}{2 \sqrt{\omgp \omgpp}} \paren{ \ap + \asnpdag } \paren{ \aspp + \anppdag } }
	}
}



%\prob{}{
%	\label{2.2(c)}
%	
%	Rewrite the conserved charge
%	\eq{
%		Q = \int \ddcx \frac{i}{2} (\phis \pis - \pi \phi)
%	}
%	in terms of creation and annihilation operators, and evaluate the charge of the particles of each type.
%}



%\prob{}{
%	Consider the case of two complex Klein-Gordon fields with the same mass.  Label the fields as $\phiax$, where $a = 1, 2$.  Show that there are now four conserved charges, one given by the generalization of part~\ref{2.2(c)}, and the other three given by
%	\eq{
%		\Qi = \int \ddcx \frac{i}{2} (\phias \sigiab \pibs - \pia \sigiab \phib),
%	}
%	where $\sigi$ are the Pauli sigma matrices.  Show that these three charges have the commutation relations of angular momentum ($SU(2)$).  Generalize these results to the case of $n$ identical complex scalar fields.
%}






%\state{(Peskin \& Schroeder 2.3)}{
%	\label{3}
%	
%	Evaluate the function
%	\eq{
%		\ev{\phix \, \phiy}{0} = D(x - y)
%	= \int \ddpf \frac{1}{2 \Ep} e^{i p \, (x - y)},
%	}
%	for $(x - y)$ spacelike so that $(x - y)^2 = -r^2$, explicitly in terms of Bessel functions.
%}






%\state{}{
%	The classical limit of a harmonic oscillator can be described in terms of \emph{coherent states},
%	\eq{
%		\kalp = \exp( \alp \adag - \frac{1}{2} |\alp|^2 ) \ko.
%	}
%	When $\alp$ is large, the oscillator state is semiclassical.  Proceeding similarly for the Fourier modes of the quantum Klein-Gordon field,
%	\al{
%		\kf &= \Nf \exp( i \int \ddpf \fp \, \apdag ) \ko, &
%		\Nf &= \exp( -\frac{1}{2} \int \ddpf | \fp |^2 ).
%	}
%}



%\prob{}{
%	Evaluate the expectation value of the field operator $\ev{\phix}{f}$ and show that it satisfies the Klein-Gordon equation.
%}



%\prob{}{
%	Evaluate the relative mean square fluctuation of the occupation number of the mode with momentum $\vp$ and the relative mean square fluctuation in the total energy:
%	\al{
%		&\frac{\evnhps - \evnhp^2}{\evnhp^2}, &
%		&\frac{\evHs - \evH^2}{\evH^2}.
%	}
%	Is either of these a good measure of the degree to which the field is classical?  Justify your answer.
%}



%\prob{}{
%	Take $\Delta(x - y) = \ev{\phivx \, \phivy}{0}$ (equal times) as a measure of the fluctuations or correlations of the field amplitude.  Use your result from problem~\ref{3} to evaluate this quantity.  What is the meaning of the divergence as $\vx \to \vy$?
%}


%\makebib

\end{document}
