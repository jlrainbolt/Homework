\documentclass[11pt]{article}
\usepackage{homework}

\classname{443}
\homeworknum{1}


\begin{document}

% Environments

\newcommand{\state}[2]{\begin{statement}{#1} #2 \end{statement}}
\newcommand{\prob}[2]{\begin{problem}{#1} #2 \end{problem}}
\newcommand{\subprob}[1]{\begin{subproblem} #1 \end{subproblem}}
\newcommand{\sol}[1]{\begin{solution} #1 \end{solution}}
\newcommand{\fig}[2]{\begin{figure} \centering #2  \label{#1} \end{figure}}

\newcommand{\makebib}{
	\vfill
	\color{black}
	\nocite{*}
	\bibliography{references}{}
	\bibliographystyle{lucas_unsrt}
}
	

% Implication

\newcommand{\qwhere}{\quad \text{where} \quad}
\newcommand{\qimplies}{\quad \implies \quad}
\newcommand{\impliesq}{\implies \quad}



% Brackets

\newcommand{\paren}[1]{\left( #1 \right)}
\newcommand{\brac}[1]{\left[ #1 \right]}
\newcommand{\curly}[1]{\left\{ #1 \right\}}


% Greek

\newcommand{\alp}{\alpha}
\newcommand{\bet}{\beta}
\newcommand{\gam}{\gamma}
\newcommand{\del}{\delta}
\newcommand{\eps}{\epsilon}
\newcommand{\zet}{\zeta}
\newcommand{\tht}{\theta}
\newcommand{\kap}{\kappa}
\newcommand{\lam}{\lambda}
\newcommand{\sig}{\sigma}
\newcommand{\ups}{\upsilon}
\newcommand{\omg}{\omega}

\newcommand{\Gam}{\Gamma}
\newcommand{\Del}{\Delta}
\newcommand{\Tht}{\Theta}
\newcommand{\Lam}{\Lambda}
\newcommand{\Sig}{\Sigma}
\newcommand{\Omg}{\Omega}


% Text

\newcommand{\where}{\text{where }}

% Problem 1

\newcommand{\Hint}{H_\text{int}}
\newcommand{\ddcx}{\dd[3]{x}}
\newcommand{\psib}{\bar{\psi}}

\newcommand{\mh}{m_h}
\newcommand{\mmu}{m_\mu}
\newcommand{\me}{m_e}
\newcommand{\ma}{m_a}

\newcommand{\aexpt}{a_\text{expt.}}
\newcommand{\aQED}{a_\text{QED}}
\renewcommand{\GeV}{\giga\electronvolt}

\newcommand{\gamt}{\gam^5}


\state{(Peskin \& Schroeder 2.1)}{
	Classical electromagnetism (with no sources) follows from the action
	\al{
		S &= \int \ddqx \paren{ -\frac{1}{4} \Fsmn \Fmn }, &
		\where \Fsmn = \ptsm \Asn - \ptsn \Asm.
	}
}



\prob{}{
	Derive Maxwell's equations as the Euler-Lagrange equations of this action, treating the components $\Asm(x)$ as the dynamical variables.  Write the equations in standard form by identifying $\Ei = -\Foi$ and ${\epsijk \Bk = -\Fij}$.
}



\prob{}{
	Construct the energy-momentum tensor for this theory.  Note that the usual procedure does not result in a symmetric tensor.  To remedy that, we can add to $\Tmn$ a term of the form $\ptsl \Klmn$, where $\Klmn$ is antisymmetric in its first two indices.  Such an object is automatically divergenceless, so
	\eq{
		\Thmn = \Tmn + \ptsl \Klmn
	}
	is an equally good energy-momentum tensor with the same globally conserved energy and momentum.  Show that this construction, with
	\eq{
		\Klmn = \Fmn \An,
	}
	leads to an energy-momentum tensor $\Th$ that is symmetric and yields the standard formulae for the electromagnetic energy and momentum densities:
	\al{
		\cE &= \frac{E^2 + B^2}{2}; &
		\vS &= \vE \cross \vB.
	}
}



\state{The complex scalar field (Peskin \& Schroeder 2.2)}{
	Consider the field theory of a complex-valued scalar field obeying the Klein-Gordon equation.  The action of this theory is
	\eq{
		S = \int \ddqx (\ptsm \phis \ptm \phi - m^2 \phis \phi).
	}
	It is easiest to analyze this theory by considering $\phix$ and $\phisx$, rather than the real and imaginary parts of $\phix$, as the basic dynamical variables.
}



\prob{}{
	Find the conjugate momenta to $\phix$ and $\phisx$ and the canonical commutation relations.  Show that the Hamiltonian is
	\eq{
		H = \int \ddcx (\pis \pi + \grad \phis \vdot  \grad \phi + m^2 \phis \phi).
	}
	Compute the Heisenberg equation of motion for $\phix$ and show that it is indeed the Klein-Gordon equation.
}



\prob{}{
	Diagonalize $H$ by introducing creation and annihilation operators.  Show that the theory contains two sets of particles of mass $m$.
}



\prob{}{
	\label{2.3(c)}
	
	Rewrite the conserved charge
	\eq{
		Q = \int \ddcx \frac{i}{2} (\phis \pis - \pi \phi)
	}
	in terms of creation and annihilation operators, and evaluate the charge of the particles of each type.
}



\prob{}{
	Consider the case of two complex Klein-Gordon fields with the same mass.  Label the fields as $\phiax$, where $a = 1, 2$.  Show that there are now four conserved charges, one given by the generalization of part~\ref{2.3(c)}, and the other three given by
	\eq{
		\Qi = \int \ddcx \frac{i}{2} (\phias \sigiab \pibs - \pia \sigiab \phib),
	}
	where $\sigi$ are the Pauli sigma matrices.  Show that these three charges have the commutation relations of angular momentum ($SU(2)$).  Generalize these results to the case of $n$ identical complex scalar fields.
}




\state{(Peskin \& Schroeder 2.3)}{
	\label{3}
	
	Evaluate the function
	\eq{
		\ev{\phix \, \phiy}{0} = D(x - y)
	= \int \ddpf \frac{1}{2 \Ep} e^{i p \, (x - y)},
	}
	for $(x - y)$ spacelike so that $(x - y)^2 = -r^2$, explicitly in terms of Bessel functions.
}





\state{}{
	The classical limit of a harmonic oscillator can be described in terms of \emph{coherent states},
	\eq{
		\kalp = \exp( \alp \adag - \frac{1}{2} |\alp|^2 ) \ko.
	}
	When $\alp$ is large, the oscillator state is semiclassical.  Proceeding similarly for the Fourier modes of the quantum Klein-Gordon field,
	\al{
		\kf &= \Nf \exp( i \int \ddpf \fp \, \apdag ) \ko, &
		\Nf &= \exp( -\frac{1}{2} \int \ddpf | \fp |^2 ).
	}
}



\prob{}{
	Evaluate the expectation value of the field operator $\ev{\phix}{f}$ and show that it satisfies the Klein-Gordon equation.
}



\prob{}{
	Evaluate the relative mean square fluctuation of the occupation number of the mode with momentum $\vp$ and the relative mean square fluctuation in the total energy:
	\al{
		&\frac{\evnhps - \evnhp^2}{\evnhp^2}, &
		&\frac{\evHs - \evH^2}{\evH^2}.
	}
	Is either of these a good measure of the degree to which the field is classical?  Justify your answer.
}



\prob{}{
	Take $\Delta(x - y) = \ev{\phivx \, \phivy}{0}$ (equal times) as a measure of the fluctuations or correlations of the field amplitude.  Use your result from problem~\ref{3} to evaluate this quantity.  What is the meaning of the divergence as $\vx \to \vy$?
}




%\newcommand{\phiq}{\phi_1}
\newcommand{\phiw}{\phi_2}
\newcommand{\rhoq}{\rho_1}
\newcommand{\rhow}{\rho_2}
\newcommand{\intS}{\int_S}
\newcommand{\dS}{\dd{S}}
\newcommand{\vv}{\vec{v}}
\newcommand{\phixp}{\phi(\vx')}

\begin{statement}{}
	The ``mean value theorem'' is stated as follows: For any solution $\phi$ to $\lap \phi = 0$, the value of $\phi$ at $\vx$ is equal to the average value of $\phi$ on a sphere of radius $R$ (for any $R$) centered at $\vx$.
\end{statement}

\begin{problem} \label{5a}
	Prove the mean value theorem.  Hint: Apply Green's theorem to $\phi$ and $1/\abs{\vx - \vx'}$ for a suitable choice of region and a suitable choice of $\vx'$.
\end{problem}

\begin{solution}
	Green's theorem is given by Eq.~(2.96),
	\beq
		\intS \nh \cdot (\phiq \nabla\phiw - \phiw \nabla\phiq) \dS = -4\pi \intV (\phiq \rhow - \phiw \rhoq) \dcx.
	\eeq
	We will choose our volume as a sphere centered at $\vx$ with radius $r$, so $\nh = \rh$.  Suppose $\phiq = \phix$ is a solution to Laplace's equation as stated.  Let $\vx'$ point radially from $\vx$, located at the center of the sphere, to a point a distance $r$ away; that is, $\vx' = \vx + r \, \rh$.  Then
	\beq
		\phiw = \frac{1}{\abs{\vx - \vx'}} = \frac{1}{\abs{-r \, \rh}} = \frac{1}{r}.
	\eeq
	From Poisson's equation, $\lap\phi = -4\pi\rho$ in general.  This means $\rhoq = 0$.  For the Green's function, $\rhow = \delta(\vx - \vx') = \delta(r)$.
	
	Applying Green's theorem,
	\beqn \label{gt}
		\intS \rh \cdot \left( \phix \nabla\frac{1}{r} - \frac{1}{r} \nabla\phix \right) \dS = -4\pi \intV \phi \, \delta(r) \dcx
	\eeqn
	For the first term on the left side, note that
	\beq
		\rh \cdot \nabla\frac{1}{r} = \pdv{}{r} \frac{1}{r} \rh \cdot \rh = -\frac{1}{r^2}.
	\eeq
	Gauss's theorem is given by Eq.~(2.6),
	\beq
		\intV \nabla \cdot \vv \dcx = \intS \vv \cdot \nh \dS.
	\eeq
	Applying this to the second term on the left side of \refeq{gt},
	\beq
		-\intS \nh \cdot \frac{1}{r} \nabla\phix \dS = -\intV \nabla \cdot \frac{1}{r} \nabla\phix \dcx
		= -\frac{1}{r} \lap\phix
		= 0.
	\eeq
	For the right side of \refeq{gt},
	\beq
		-4\pi \intV \phix \, \delta(r) \dcx = -4\pi \phi(0).
	\eeq
	
	Putting this together, \refeq{gt} becomes
	\beq
		-\intS \frac{\phix}{r^2} \dS = -4\pi \phi(0) \dS.
	\eeq
	We can choose $\vx = 0$ without loss of generality and switch $\vx$ with $\vx'$, which gives us
	\beqn \label{mvt}
		\phix = \frac{1}{4\pi r^2} \intS \phixp \dd{S'}.
	\eeqn
	This equation demonstrates that the value of $\phi$ at $\vx$ is equal to its average value on a sphere of arbitrary radius $r$.  Thus, we have proven the mean value theorem. \qed
\end{solution}


\begin{problem}
	Use this result to show that a point charge can never be in stable equilibrium if placed in an electric field $\vE$ that is source free in a neighborhood of the charge.
\end{problem}

\begin{solution}
	Let $\cV$ denote the neighborhood of the point charge, which can be described as a sphere of radius $r$ centerd at the location of the point charge.  We will choose this point as the origin.  Let $S$ denote the boundary of $\cV$.
	
	Suppose, contrary to the problem statement, that the point charge is in stable equilibrium.  This means that the electrostatic potential $\phi$ has a local minimum at the origin, and so $\phi(0) < \phix$ for all other $\vx \neq 0$ within $\cV$.  In particular, $\phi(0) < \phi|_S$ at all points on the boundary, and so
	\beqn \label{fake}
		\phi(0) < \frac{1}{4\pi r^2} \intS \phix \dS.
	\eeqn
	However, $\phi$ must satisfy $\lap\phi = 0$, since $\cV$ is source free.  As proven in \ref{5a}, $\phi$ therefore obeys \refeq{mvt}, which contradicts \refeq{fake} and therefore our assumption that the point charge is in stable equilibrium.  So we have shown that stable equilibrium is impossible in this situation. \qed
\end{solution}


%\makebib

\end{document}
