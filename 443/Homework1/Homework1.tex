\documentclass[11pt]{article}
\usepackage{homework}

\classname{443}
\homeworknum{1}


\begin{document}

% Environments

\newcommand{\state}[2]{\begin{statement}{#1} #2 \end{statement}}
\newcommand{\prob}[2]{\begin{problem}{#1} #2 \end{problem}}
\newcommand{\subprob}[1]{\begin{subproblem} #1 \end{subproblem}}
\newcommand{\sol}[1]{\begin{solution} #1 \end{solution}}
\newcommand{\fig}[2]{\begin{figure} \centering #2  \label{#1} \end{figure}}

\newcommand{\makebib}{
	\vfill
	\color{black}
	\nocite{*}
	\bibliography{references}{}
	\bibliographystyle{lucas_unsrt}
}
	

% Implication

\newcommand{\qwhere}{\quad \text{where} \quad}
\newcommand{\qimplies}{\quad \implies \quad}
\newcommand{\impliesq}{\implies \quad}



% Brackets

\newcommand{\paren}[1]{\left( #1 \right)}
\newcommand{\brac}[1]{\left[ #1 \right]}
\newcommand{\curly}[1]{\left\{ #1 \right\}}


% Greek

\newcommand{\alp}{\alpha}
\newcommand{\bet}{\beta}
\newcommand{\gam}{\gamma}
\newcommand{\del}{\delta}
\newcommand{\eps}{\epsilon}
\newcommand{\zet}{\zeta}
\newcommand{\tht}{\theta}
\newcommand{\kap}{\kappa}
\newcommand{\lam}{\lambda}
\newcommand{\sig}{\sigma}
\newcommand{\ups}{\upsilon}
\newcommand{\omg}{\omega}

\newcommand{\Gam}{\Gamma}
\newcommand{\Del}{\Delta}
\newcommand{\Tht}{\Theta}
\newcommand{\Lam}{\Lambda}
\newcommand{\Sig}{\Sigma}
\newcommand{\Omg}{\Omega}


% Text

\newcommand{\where}{\text{where }}

% Problem 1

\newcommand{\Hint}{H_\text{int}}
\newcommand{\ddcx}{\dd[3]{x}}
\newcommand{\psib}{\bar{\psi}}

\newcommand{\mh}{m_h}
\newcommand{\mmu}{m_\mu}
\newcommand{\me}{m_e}
\newcommand{\ma}{m_a}

\newcommand{\aexpt}{a_\text{expt.}}
\newcommand{\aQED}{a_\text{QED}}
\renewcommand{\GeV}{\giga\electronvolt}

\newcommand{\gamt}{\gam^5}

\state{Spin-wave theory~(P\&S 11.1)}{\hfix}

\prob{ \label{1a}
	Prove the following wonderful formula: Let $\phix$ be a free scalar field with propagator $\ev{T \phix \phio} = \Dx$.  Then
	\eqn{show1}{
		\ev{ T e^{i \phix} e^{-i \phio} } = e^{[ \Dx - \Do ]}.
	}
	(The  factor $\Do$ gives a formally divergent adjustment of the overall normalization.)
}

\sol{
	According to P\&S~(9.18),
	\eq{
		\ev*{T \phi(\xq) \phi(\xw)}{\Omg} = \frac{\int \DDphi \phi(\xq) \phi(\xw) \exp[ i \int \ddqx \cL ]}{\int \DDphi \exp[ i \int \ddqx \cL ]}.
	}
	We use this expression to write the left-hand side of Eq.~\refeq{show1}:
	\eqn{thing1}{
		\ev{ T e^{i \phix} e^{-i \phio} } = \frac{\int \DDphi e^{i \phix} e^{-i \phio} \exp[ i \int \ddqy \cL ]}{\int \DDphi \exp[ i \int \ddqy \cL ]}
		= \frac{\int \DDphi \exp[i \phix - i \phio + i \int \ddqy \cL ]}{\int \DDphi \exp[ i \int \ddqy \cL ]}.
	}
	For a free Klein-Gordon~(i.e., scalar) field, Eq.~(9.39) tells us that the generating functional $\ZJ$ is given by
	\eq{
		\ZJ = \Zo \exp[ -\frac{1}{2} \int \ddqx \ddqy \Jx \DF(x - y) \Jy ],
	}
	where $\Zo = Z[0]$.  Thus, we want to find some $\Jy$ such that
	\eqn{thing1b}{
		\ev{ T e^{i \phix} e^{-i \phio} } = \frac{\ZJ}{\Zo}
	}
	where in general
	\eq{
		\ZJ = \int \DDphi \exp[ i \int \ddqx [ \cL + \Jx \phi(x) ] ]
	}
	by (9.34).  Inspecting Eq.~\refeq{thing1}, we recognize the denominator as $\Zo$ and see that if
	\eq{
		\Jy = \delq(y - x) - \delq(y)
	}
	we have an expression like Eq.~\refeq{thing1b}.  Collecting these findings, we have
	\al{
		\ans{ \ev{ T e^{i \phix} e^{-i \phio} } }&= \frac{\ZJ}{\Zo} \\
		&= \exp[ -\frac{1}{2} \int \ddqy \ddqz \Jy \DF(y - z) \Jz ] \\
		&= \exp[ -\frac{1}{2} \int \ddqy \ddqz \Jy \DF(y - z) [ \delq(z - x) - \delq(z) ] ] \\
		&= \exp[ -\frac{1}{2} \int \ddqy [ \delq(y - x) - \delq(y) ] [ \DF(y - x) - \DF(y) ] ] \\
		&= \exp[ -\frac{1}{2} [ \DF(0) - \DF(x) - \DF(-x) + \DF(0) ] ] \\
		&= \exp[ \DF(x) - \DF(0) ] \\
		&\ans{\; = e^{[ \Dx - \Do ]}, }
	}
	as we wanted to show. \qed
}



\prob{ \label{1b}
	We can use this formula in Euclidean field theory to discuss correlation functions in a theory with spontaneously broken symmetry for $T < \TC$.  Let us consider only the simplest case of a broken $O(2)$ or $U(1)$ symmetry.  We can write the local spin density as a complex variable
	\eq{
		\sx = \sqx + i \swx.
	}
	The global symmetry is the transformation
	\eq{
		\sx \to e^{-i \alp} \sx.
	}
	If we assume that the physics freezes the modulus of $\sx$, we can parameterize
	\eqn{sx}{
		\sx = A e^{i \phix}
	}
	and write an effective Lagrangian for the field $\phix$.  The symmetry of the theory becomes the translation symmetry
	\eqn{symmetry}{
		\phix \to \phix - \alp.
	}
	Show that (for $d > 0$) the most general renormalizable Lagrangian consistent with this symmetry is the free field theory
	\eqn{show1b}{
		\cL = \frac{1}{2} \rho(\vgrad \phi)^2.
	}
	In statistical mechanics, the constant $\rho$ is called the \emph{spin wave modulus}.  A reasonable hypothesis for $\rho$ is that it is finite for $T < \TC$ and tends to 0 as $T \to \TC$ from below.
}

\sol{
	In accordance with the Klein-Gordon Lagrangian in P\&S~(2.6),
	\eqn{KGL}{
		\cL_\text{K-G} = \frac{1}{2} (\pt \phi)^2 - \frac{1}{2} m^2 \phi^2,
	}
	we interpret $(\vgrad \phi)^2$ as $(\pt \phi)^2$.
	
	The Lagrangian cannot have terms of $\order{\phi^n}$ for any $n \neq 0$ since $\phi(x)$ is not invariant under Eq.~\refeq{symmetry}.  Any combination of derivatives of $\phi$ is invariant, however, since $\alp$ is a constant and does not contribute to any derivative.  Thus, only terms like $\pt^n \phi^m$ (where $n$ denotes a power of $\pt$) for $n, m > 0$ and $n \geq m$ are consistent with the symmetry of Eq.~\refeq{symmetry} for $d$ an integer.
	
	Now we must determine which of these terms are renormalizable.  We know that the Lagrangian must have dimension $d$, and that $\phi$ has dimension $(d - 2) / 2$.  Taking a derivative adds a mass dimension.  The theory is renormalizable if the coupling constant $\rho$ has dimension greater than or equal to 0~\cite[p.~322]{Peskin}.  Let $p$ be the dimension of $\rho$.  The dimension of our allowed term is then
	\eq{
		[ \rho \pt^n \phi^m ] = p + n + m \frac{d - 2}{2},
	}
	which we require to be equal to $d$.  Thus we seek solutions to the system of equations
	\al{
		d &= p + n + m \frac{d - 2}{2}, &
		n &\geq m, &
		p &\geq 0.
	}
	Solving with Mathematica, we find that this system has two solutions: $n = m = 2$ and $p = 0$; and $n = m = 1$ and $p = d / 2$.  However, the term $\pt \phi$ for $n = m = 1$ does not contribute to the action because it is a total derivative and does not contribute when the integral over $\cL$ is evaluated:
	\eq{
		\int \dd[d]{x} \pt\phi = \phi \bigg|_{-\infty}^\infty
		= 0.
	}
	Thus the only possibility is $n = m = 2$.  Note that
	\eq{
		\pt^2 \phi^2 = \pt(\pt \phi^2)
		= 2 \pt( \phi \pt \phi)
		= \pt \phi \pt \phi + \phi \pt^2 \phi
		= (\pt \phi)^2,
	}
	since $\phi \pt^2 \phi$ is not invariant under Eq.~\refeq{sx}.  This means that $\rho$ must be dimensionless and that the only allowed terms in the Lagrangian are proportional to $(\pt \phi)^2$, which is consistent with Eq.~\refeq{show1b}. \qed
}



\prob{
	Compute the correlation function $\ev{ \sx \sao }$.  Adjust $A$ to give a physically sensible normalization (assuming that the system has a physical cutoff at the scale of one atomic spacing) and display the dependence of this correlation function on $x$ for $d = 1, 2, 3, 4$.  Explain the significance of your results.
}

\sol{
	Applying Eq.~\refeq{sx},
	\eq{
		\ev{ \sx \sao } = \ev*{ A e^{i \phix} \As e^{-i \phio} }
		= \ev*{ \abs{A}^2 } \ev*{ e^{i \phix} e^{-i \phio} }.
	}
	Now we can apply Eq.~\refeq{show1} to find
	\eqn{thing1c}{
		\ans{ \ev{ \sx \sao } = \abs{A}^2 \exp[ D(x) - D(0) ], }
	}
	where $D(x - y)$ is a Green's function.  Since our Lagrangian is similar to the Klein-Gordon Lagrangian Eq.~\refeq{2.6}, our Green's function is similar to that of the Klein-Gordon operator, which is given by P\&S~(2.56):
	\eq{
		(\pt^2 + m^2) D(x - y) = -i \delq(x - y).
	}
	The Feynman prescription for this Green's function is given by (2.59),
	\eqn{DF}{
		\DF(x - y) = \int \ddqpf \frac{i}{p^2 - m^2 + i \eps} e^{-i p \cdot (x - y)}.
	}
	For the Lagrangian in Eq.~\refeq{show1b}, we set $m = 0$ and insert a factor of $\rho$:
	\eq{
		\rho \pt^2 D(x - y) = -i \deld(x - y),
	}
	so adapting Eq.~\refeq{DF} for this situation yields
	\eqn{DF}{
		\DF(x - y) = \frac{1}{\rho} \int \dddpf \frac{i}{p^2 + i \eps} e^{-i p \cdot (x - y)}.
	}
	We see that $\DF(0)$ diverges, so we absorb it into the constant to make the normalization physically sensible.  We can do this because, as we showed in \ref{1b}, the theory is renormalizable.  Define $A'$ such that
	\eq{
		{A'}^2 = \abs{A}^2 e^{-D(0)}.
	}
	Then Eq.~\refeq{thing1c} can be written
	\eq{
		\ans{ \ev{ \sx \sao } =  {A'}^2 e^{D(x)}. }
	}
	
	To evaluate the divergent integral $D(x)$, we look to the Feynman parameter method we have been using to solve divergent integrals.  Apparently, the Schwinger parametrization is useful in deriving the Feynman parametrization, and it is given by~\cite{Feynman}
	\eq{
		\frac{1}{A} = \intoi \dds e^{-s A}.
	}
	Using this equation, we can write Eq.~\refeq{DF} as
	\eq{
		\DF(x) = \frac{1}{\rho} \int \dddpf \frac{i}{p^2} e^{-i p \cdot x}
		= \frac{i}{\rho} \int \dddpf \intoi \dds e^{-s p^2} e^{-i p \cdot x}.
	}
	Now we can complete the square in the exponential to get a Gaussian integral:
	\al{
		\DF(x) &= \frac{i}{\rho} \int \dddpf \intoi \dds \exp[ -s p^2 - i p \cdot x + \frac{x^2}{4 s} - \frac{x^2}{4 s} ] \\
		&= \frac{i}{\rho} \int \dddpf \intoi \dds \exp[ -s \paren{ p + \frac{i x}{2 s} }^2 - \frac{x^2}{4 s} ] \\
		&= \frac{i}{\rho (2 \pi)^d} \intoi \dds e^{-x^2 / 4 s} \int \dd[d]{u} e^{-s u^2} \\
		&= \frac{i}{\rho (2 \pi)^{d}} \intoi \dds e^{-x^2 / 4 s} \sqrt{ \frac{(2\pi)^d}{(2s)^d} } \\
		&= \frac{i}{\rho (4 \pi)^{d / 2}} \intoi \dds \frac{e^{-x^2 / 4 s}}{s^{d / 2}}
	}
	where we have used~\cite{QFT}
	\eq{
		\int \exp( -\frac{1}{2} x \cdot A \cdot x ) \dd[n]{x} = \sqrt{\frac{(2\pi)^n}{\det A}},
	}
	with $A$ a $d \times d$ diagonal matrix $2s$.  Using Mathematica to integrate with respect to $s$, we find
	\eq{
		\DF(x) = \frac{i}{\rho (4 \pi)^{d / 2}} \frac{2^{d - 2}}{x^{d - 2}} \Gam(d / 2 - 1)
		= \frac{i}{4 \pi^d \rho} \Gam(d / 2 - 1) x^{2 - d}.
	}
	The gamma function diverges as $d \to 2$, so as we have done in previous problems, we expand about $\eps = 2 - d$.  Evaluating the series expansion using Mathematica, we obtain
	\eq{
		\DF(x) = \frac{i}{4 \pi^{1 - \eps} \rho} \Gam(\eps / 2) x^\eps
		\approx \frac{i}{4 \pi \rho} \paren{ \frac{2}{\eps} - \gam + 2 \ln(\pi x) }
		\sim \frac{i}{2 \pi \rho} \ln(x)
		= i \ln(\frac{1}{x^{2 \pi \rho}}).
	}
	We Wick rotate $x \to i x$.  Then the dependence of the correlation function on $x$ for $d = 1, 2, 3, 4$ is
	\ans{\al{
		(d = 1) &\qquad \ev{ \sx \sao } \sim e^{-x / 2 \sqrt{\pi} \rho}, &
		(d = 2) &\qquad \ev{ \sx \sao } \sim x^{2 \pi \rho}, \\
		(d = 3) &\qquad \ev{ \sx \sao } \sim \frac{1}{x}, &
		(d = 4) &\qquad \ev{ \sx \sao } \sim \frac{1}{x^2}.
	}}%
	In $d > 2$ dimensions, the expectation value of the correlation function tends to 0 at large distances $x$.  For $d > 2$, it drops off more quickly as $d$ increases.  The $d \leq 2$ cases depend on $\rho$, which we assume is positive.  The $d = 1$ case drops off with increasing distance, and more quickly with smaller $\rho$.  For $d = 2$, the expectation value of the correlation function increases with increasing distance, and it blows up more quickly with larger $\rho$.
	
	These results are consistent with the Mermin--Wagner theorem, which states that a continuous symmetry cannot be broken in $d \leq 2$ dimensions~\cite{CMW}.  That is, in $d \leq 2$ dimensions, a symmetry-breaking field cannot have a nonzero vacuum expectation value~\cite[p.~460]{Peskin}.  A physical explanation is that each spin has more nearest neighbors in higher dimensions.  Since the spins are inclined to align with their neighbors, there is a higher degree of correlation in higher dimensions at the same distance.  In two dimensions, the correlations are weak enough that they are overpowered by the field fluctuations.
}



\state{The complex scalar field (Peskin \& Schroeder 2.2)}{
	Consider the field theory of a complex-valued scalar field obeying the Klein-Gordon equation.  The action of this theory is
	\eqn{action2}{
		S = \int \ddqx (\ptsm \phis \ptm \phi - m^2 \phis \phi).
	}
	It is easiest to analyze this theory by considering $\phix$ and $\phisx$, rather than the real and imaginary parts of $\phix$, as the basic dynamical variables.
}



\prob{}{
	Find the conjugate momenta to $\phix$ and $\phisx$ and the canonical commutation relations.  Show that the Hamiltonian is
	\eqn{ham2.a}{
		H = \int \ddcx (\pis \pi + \grad \phis \vdot \grad \phi + m^2 \phis \phi).
	}
	Compute the Heisenberg equation of motion for $\phix$ and show that it is indeed the Klein-Gordon equation.
}

\sol{
	The momentum density conjugate to $\phix$ is defined in Peskin \& Schroeder~(2.4):
	\eq{
		\pix \equiv \pdv{\cL}{\phidx}
	}
	Here, $\cL$ is the integrand of Eq.~\refeq{action2}.  Expanding its first term yields
	\eqn{lagr2.a}{
		\cL = \phid \phisd - \grad \phi \vdot \grad \phis,
	}
	so then
	\aln{
		\ans{\pix\ }&\ans{= \phisd, } &
		\ans{\pisx\ }&\ans{= \phid, } \label{moms2.a}
	}
	where $\pisx$ is the momentum conjugate to $\phisx$.  The canonical commutation relations follow from Peskin \& Schroeder~(2.20):
	\ans{\aln{ \label{comm2.a}
		[\phivx, \pivy] &= [\phisvx, \pisvy] = i \,\del^3(\vx - \vy), \\
		[\phivx, \phivy] &= [\phisvx, \phisvy] = 0, \\
		[\pivx, \pivy] &= [\pisvx, \pisvy] = 0, \\
		[\phivx, \pisvy] &= [\phivx, \phisvy] = [\pivx, \pisvy] = 0.
	}}% 
	
	The Hamiltonian is given in general for a single field by Peskin \& Schroeder~(2.5),
	\eq{
		H = \int \ddcx \paren{ \pix \, \phidx - \cL }.
	}
	For the two fields $\phix$ and $\phisx$, this becomes
	\al{
		H &= \int \ddcx \paren{ \pix \, \phidx + \pisx \, \phisdx - \cL } \\
		&= \int \ddcx \paren{ \pi \phid + \pis \phisd - \phid \phisd + \grad \phi \vdot \grad \phis + m^2 \phis \phi } \\
		&= \int \ddcx \paren{ \pi \pis + \phid \phisd - \phid \phisd + \grad \phi \vdot \grad \phis + m^2 \phis \phi } \\
		&= \ans{ \int \ddcx \paren{ \pis \pi + \grad \phi \vdot \grad \phis + m^2 \phis \phi }, }
	}
	where we have used Eqs.~\refeq{lagr2.a} and \refeq{moms2.a} as well as the commutation relations.  So we have proven Eq.~\refeq{ham2.a}. \qed
	
	The Heisenberg equation of motion is Peskin \& Schroeder~(2.44),
	\eq{
		i \pdv{\cO}{t} = [\cO, H],
	}
	where $\cO$ is an arbitrary operator.  Then
	\al{
		i \pdv{\phix}{t} &= [\phix, H] \\
		&= \brac{ \phixt, \int \ddcxp \pisxp \, \pixp } + \brac{ \phixt, \int \ddcxp \grad' \phixp \vdot \grad' \phisxp } \\
		&\phantom{mmmmmmmmmmmmmmmmmmmmmmmmmmmmmmm} + m^2 \brac{ \phixt, \int \ddcxp \phisxp \, \phixp } \\
		&= \brac{ \phixt, \int \ddcxp \pisxp \, \pixp }
		= i \int \ddcxp \del^3(\vx - \vx') \, \pisxp
		= i \pisx, \\[2ex]
		i \pdv{\phisx}{t} &= [\phisx, H] \\
		&= \brac{ \phisxt, \int \ddcxp \pisxp \, \pixp }
		= i \int \ddcxp \del^3(\vx - \vx') \, \pixp
		= i \pix,
	}
	\al{
		i \pdv{\pix}{t} &= [\pix, H] \\
		&= \brac{ \pixt, \int \ddcxp \pisxp \, \pixp } + \brac{ \pixt, \int \ddcxp \grad' \phixp \vdot \grad' \phisxp } \\
		&\phantom{mmmmmmmmmmmmmmmmmmmmmmmmmmmmmmm} + m^2 \brac{ \pixt, \int \ddcxp \phisxp \, \phixp } \\
		&= -i \int \ddcxp \brac{ \grad' \del^3(\vx - \vx') \vdot \grad' \phisxp + m^2 \del^3(\vx - \vx') \, \phisxp } \\
		&= -i \int \ddcxp \brac{ - \del^3(\vx - \vx') \vdot {\nabla'}^2 \phisxp + m^2 \del^3(\vx - \vx') \, \phisxp }
		= -i (-\laplacian + m^2) \, \phisx, \\[2ex]
		i \pdv{\pisx}{t} &= [\pisx, H] \\
		&= -i \int \ddcxp \brac{ \grad' \phixp \vdot \grad' \del(\vx - \vx') + m^2 \del^3(\vx - \vx') \, \phixp } \\
		&= -i \int \ddcxp \brac{ - \del^3(\vx - \vx') \vdot {\nabla'}^2 \phixp + m^2 \del^3(\vx - \vx') \, \phixp }
		= -i (-\laplacian + m^2) \, \phix.
	}
	Thus we have obtained
	\al{
		\pdv{\phix}{t} &= \pisx, &
		\pdv{\phisx}{t} &= \pix, &
		\pdv{\pix}{t} &= (\laplacian - m^2) \,\phisx, &
		\pdv{\pisx}{t} &= (\laplacian - m^2) \,\phix.
	}
	Combining these results yields
	\ans{\al{
		\pdv[2]{\phi}{t} &= (\laplacian - m^2) \phi, &
		\pdv[2]{\phis}{t} &= (\laplacian - m^2) \phis,
	}}%
	which is the Klein-Gordon equation and its complex conjugate, as we sought to show. \qed
}



\prob{}{
	Diagonalize $H$ by introducing creation and annihilation operators.  Show that the theory contains two sets of particles of mass $m$.
}

\sol{
	Peskin \& Schroeder~(2.21) gives the Klein-Gordon equation in the momentum basis,
	\eq{
		\paren{ \pdv[2]{t} + \vp^2 + m^2 } \phipt = 0.
	}
	This is the same as the harmonic oscillator equation of motion.  It has solutions~\cite{SHM}
	\eq{
		\phipt = \Avp \, e^{i \omgp t} + \Bvp \, e^{-i \omgp t},
	}
	where $\omgp = \sqrt{\vp^2 + m^2}$ as in Peskin \& Schroeder Eq.~(2.22), and $\Avp$ and $\Bvp$ are arbitrary functions of $\vp$.  The complex conjugate of this solution is
	\eq{
		\phispt = \Bsvp \, e^{i \omgp t} + \Asvp \, e^{-i \omgp t}.
	}
	
	The field $\phi$ in the position basis can be expanded as~\cite[p.~20]{Peskin},
	\eq{
		\phixt = \int \ddpf e^{i \vp \vdot \vx} \, \phipt.
	}
	so we can write, as explained in class,
	\al{
		\phivx &= \int \ddpf \frac{1}{\sqrt{2 \omgp}} \paren{ \ap e^{i \vp \vdot \vx} + \bpdag e^{-i \vp \vdot \vx} }, &
		\phisvx &= \int \ddpf \frac{1}{\sqrt{2 \omgp}} \paren{ \bp e^{i \vp \vdot \vx} + \apdag e^{-i \vp \vdot \vx} },
	}
	where $\apdag, \bpdag$~($\ap, \bp$) are creation~(annihilation) operators.  By analogy to Eq.~(2.26) of Peskin \& Schroeder, we can also write
	\al{
		\pivx &= -i \int \ddpf \sqrt{\frac{\omgp}{2}} \paren{ \bp e^{i \vp \vdot \vx} - \apdag e^{-i \vp \vdot \vx} }, &
		\pisvx &= -i \int \ddpf \sqrt{\frac{\omgp}{2}} \paren{ \ap e^{i \vp \vdot \vx} - \bpdag e^{-i \vp \vdot \vx} }.
	}
	Simplifying these expressions as in their Eqs.~(2.27) and (2.28), we have
	\aln{
		\phivx &= \int \ddpf \frac{1}{\sqrt{2 \omgp}} \paren{ \ap + \bnpdag } e^{i \vp \vdot \vx}, &
		\phisvx &= \int \ddpf \frac{1}{\sqrt{2 \omgp}} \paren{ \bp + \anpdag } e^{i \vp \vdot \vx}, \label{things2.b1} \\
		\pivx &= -i \int \ddpf \sqrt{\frac{\omgp}{2}} \paren{ \bp - \anpdag } e^{i \vp \vdot \vx}, &
		\pisvx &= -i \int \ddpf \sqrt{\frac{\omgp}{2}} \paren{ \ap - \bnpdag } e^{i \vp \vdot \vx}. \label{things2.b2}
	}
	Also generalizing their Eq.~(2.24),
	\al{
		[\ap, \appdag] &= [\bp, \bppdag] = (2\pi)^3 \, \del^3(\vp - \vp'), &
		[\ap, \bppdag] &= [\bp, \appdag] = 0.
	}
	
	Feeding Eqs.~\refeq{things2.b1} and \refeq{things2.b2} into Eq.~\refeq{ham2.a} yields
	\eq{
		H = \int \ddcx \int \ddppfs \, e^{i (\vp + \vp') \vdot \vx} \brac{ -\frac{\sqrt{\omgp \omgpp}}{2} \paren{ \ap - \bnpdag } \paren{ \bpp - \anppdag } + \frac{-\vp \vdot \vp' + m^2}{2 \sqrt{\omgp \omgpp}} \paren{ \ap + \bnpdag } \paren{ \bpp + \anppdag } }.
	}
	Using the delta function identity~\cite{Dirac}
	\eq{
		\del(x - a) = \frac{1}{2\pi} \intii e^{i p (x - a)} \ddp,
	}
	this becomes
	\al{
		H &= \int \ddppfc \, \del^3(\vp + \vp') \brac{ -\frac{\sqrt{\omgp \omgpp}}{2} \paren{ \ap - \bnpdag } \paren{ \bpp - \anppdag } + \frac{-\vp \vdot \vp' + m^2}{2 \sqrt{\omgp \omgpp}} \paren{ \ap + \bnpdag } \paren{ \bpp + \anppdag } } \\
		&= \int \ddpf \brac{ -\frac{\omgp}{2} \paren{ \ap - \bnpdag } \paren{ \bnp - \apdag } + \frac{\vp^2 + m^2}{2 \omgp} \paren{ \ap + \bnpdag } \paren{ \bnp + \apdag } } \\
		&= \int \ddpf \frac{\omgp}{2} \brac{ \ap \bnp + \ap \apdag + \bnpdag \bnp + \bnpdag \apdag - \paren{ \ap \bnp - \ap \apdag - \bnpdag \bnp + \bnpdag \apdag } } \\
		&= \int \ddpf \, \omgp \paren{ \ap \apdag + \bnpdag \bnp }
		= \int \ddpf \, \omgp \paren{ \apdag \ap + \bpdag \bp + [\ap, \appdag] }.
	}
	Ignoring the infinite constant term~\cite[p.~21]{Peskin}, we have
	\eqn{diagham2.b}{
		\ans{ H = \int \ddpf \, \omgp \paren{ \apdag \ap + \bpdag \bp }. }
	}
	
	To show that the theory contains two sets of particles of mass $m$, we evaluate the commutators~\cite[p.~22]{Peskin}:
	\al{
		[H, \apdag] &= \brac{ \int \ddppf \omgpp \appdag \app, \apdag } = \omgp \apdag, &
		[H, \ap] &= \brac{ \int \ddppf \omgpp \appdag \app, \ap } = -\omgp \ap, \\
		[H, \bpdag] &= \brac{ \int \ddppf \omgpp \bppdag \bpp, \bpdag } = \omgp \bpdag, &
		[H, \bp] &= \brac{ \int \ddppf \omgpp \bppdag \bpp, \bp } = -\omgp \bp.
	}
	Then we can define the eigenstates of the Hamiltonian by
	\eq{
		(\apdag)^{\na} \,(\bpdag)^{\nb} \koo \equiv \knanb,
	}
	which have eigenvalues $(\na + \nb) \omgp$.  So the expression for the Hamiltonian in Eq.~\refeq{diagham2.b} is diagonal in the occupation number basis $\{ \knanb \}$, where $\na$ indicates the number of particles created with $\apdag$ and $\nb$ the number of antiparticles created with $\bpdag$.  The ground state is $\koo$; it has zero energy since its eigenvalue is zero.  Since each operation of $\apdag$ or $\bpdag$ imparts energy $\omgp = \sqrt{\vp^2 + m^2}$ to the system, and each operation of $\ap$ or $\bp$ removes energy $\omgp = \sqrt{\vp^2 + m^2}$ from the system, we can conclude that each of the two sets of operators corresponds to a set of particles of mass $m$. \qed
}



\prob{}{ \label{2.2(c)}
	Rewrite the conserved charge
	\eqn{Q2.c}{
		Q = \int \ddcx \frac{i}{2} (\phis \pis - \pi \phi)
	}
	in terms of creation and annihilation operators, and evaluate the charge of the particles of each type.
}

\sol{
	Applying Eqs.~\refeq{things2.b1} and \refeq{things2.b2}, we find
	\al{
		Q &= \frac{1}{4} \int \ddcx \int \ddppfs \, e^{i \vp \vdot \vx} \brac{ \paren{ \bp + \anpdag } \paren{ \app - \bnppdag } - \paren{ \bp - \anpdag } \paren{ \app + \bnppdag } } \\
		&= \frac{1}{4} \int \ddpf \brac{ \paren{ \bp + \anpdag } \paren{ \anp - \bpdag } - \paren{ \bp - \anpdag } \paren{ \anp + \bpdag } } \\
		&= \frac{1}{4} \int \ddpf \brac{ \paren{ \bp \anp - \bp \bpdag + \anpdag \anp - \anpdag \bnpdag } - \paren{ \bp \anp + \bp \bpdag - \anpdag \anp - \anpdag \bpdag} } \\
		&= \frac{1}{2} \int \ddpf \paren{ \anpdag \anp - \bp \bpdag }
		= \frac{1}{2} \int \ddpf \paren{ \ap \apdag - \bp \bpdag } \\
		&= \frac{1}{2} \int \ddpf \paren{ \ap \apdag - [\ap, \apdag] - \bpdag \bp - [\bp, \bpdag] }
		= \ans{ \frac{1}{2} \int \ddpf \paren{ \apdag \ap - \bpdag \bp }, }
	}
	where in the fifth line we have used $a = (q + i p) / \sqrt{2}$ and $\adag = (q - i p)$~\cite{Operators}.  The particles associated with $\apdag \ap$ must have positive charge, since $\apdag \ap$ represents their number and has the same sign as the conserved charge.  Similarly, the antiparticles associated with $\bpdag \bp$ must have negative charge.
}



\prob{}{
	Consider the case of two complex Klein-Gordon fields with the same mass.  Label the fields as $\phiax$, where $a = 1, 2$.  Show that there are now four conserved charges, one given by the generalization of part~\ref{2.2(c)}, and the other three given by
	\eq{
		\Qi = \int \ddcx \frac{i}{2} (\phias \sigiab \pibs - \pia \sigiab \phib),
	}
	where $\sigi$ are the Pauli sigma matrices.  Show that these three charges have the commutation relations of angular momentum ($SU(2)$).  Generalize these results to the case of $n$ identical complex scalar fields.
}

\sol{
	Generalizing Eq.~\refeq{action2}, the Lagrangian for the two Klein-Gordon fields is
	\eqn{lagr2.d}{
		\cL = \ptsm \phiqs \ptm \phiq - m^2 \phiqs \phiq + \ptsm \phiws \ptm \phiw - m^2 \phiws \phiw.
	}
	The conserved charge is given in general by Peskin \& Schroeder~(2.12) and (2.13),
	\al{
		Q &\equiv \int_\text{all space} \jo \ddcx, &
		\where \jmx &= \pdv{\cL}{(\ptsm\phi)} \Del\phi - \Jm,
	}
	where $\Jm$ is a 4-divergence that arises when transforming the Lagrangian as in Peskin \& Schroeder~(2.10):
	\eq{
		\cLx \to \cLx + \alp \ptsm \Jmx.
	}
	
	For the first conserved charge, we note that the Lagrangian in Eq.~\refeq{lagr2.d} is invariant under the transformations $\phia \to e^{i \alp} \phia$~\cite[p.~18]{Peskin}:
	\eq{
		\cL \to \suma \brac{ \ptsm(e^{-i \alp} \phias) \, \ptm(e^{i \alp} \phia) - m^2 e^{-i \alp} \phias \, e^{i \alp} \phia }
		= \cL,
	}
	so $\Jmx = 0$.  The relevant infinitesimal transformations are found by generalizing Peskin \& Schroeder~(2.15):
	\aln{ \label{infinitesimal2.d}
		\alp \, \Del\phia &= i \alp \phia, &
		\alp \, \Del\phias &= -i \alp \phias.
	}
	These transformations yield the conserved current
	\eq{
		\jm = -\frac{1}{2} \suma \paren{ \pdv{\cL}{(\ptsm\phia)} \Del\phia + \pdv{\cL}{(\ptsm\phias)} \Del\phias }
		= -\frac{i}{2} \suma \paren{ \phia \ptm\phias -\phias \ptm\phia },
	}
	where we have arbitrarily chosen the overall constant~\cite[p.~18]{Peskin}.  Then, generalizing Eq.~\refeq{moms2.a}, the corresponding conserved charge is
	\al{
		\Qo &= \int \ddcx \jo
		= -\frac{i}{2} \int \ddcx \suma \paren{ \phia \phiasd - \phias \phiad }
		= -\frac{i}{2} \int \ddcx \suma \paren{ \phia \pia - \phias \pias } \\
		&= \ans{ \int \ddcx \frac{i}{2} \paren{ \phiqs \piqs - \phiq \piq + \phiws \piws - \phiw \piw }, }
	}
	which is the generalization of Eq.~\refeq{Q2.c} for two fields.
	
	From the problem statement, we make the ansatz that $\cL$ is also invariant under rotations, $\phi \to e^{i \alpi \sigi / 2} \phi$ where $\phi = (\phiq, \phiw)$ is a two-component spinor, from Peskin \& Schroeder~(15.19) and (15.20).  To verify,
	\eq{
		\cL \to \suma \brac{ \ptsm(e^{-i \alpi \sigi / 2} \phias) \, \ptm(e^{i \alpi \sigi / 2} \phia) - m^2 e^{-i \alpi \sigi / 2} \phias \, e^{i \alpi \sigi / 2} \phia }
		= \cL.
	}
	Again, $\Jm = 0$.  By analogy with Eq.~\refeq{infinitesimal2.d}, the infinitesimal transformations are
	\al{
		\alpi \, \Del\phi &= \frac{i}{2} \phi \alpi \sigi, &
		\alpi \, \Del\phia &= -\frac{i}{2} \phia \alpi \sigi.
	}
	We have the conserved currents
	\eq{
		\jmi = -\paren{ \pdv{\cL}{(\ptsm\phi)} \Del\phi + \pdv{\cL}{(\ptsm\phis)} \Del\phis }
		= \frac{i}{2} \paren{ \phis \sigi \, \ptm\phi - \phi \sigi \, \ptm\phis },
	}
	where we have chosen a different overall constant factor.  Then the corresponding conserved charges are
	\eq{
		\Qi = \int \ddcx \joi
		= \int \ddcx \frac{i}{2} \paren{ \phis \sigi \phid - \phi \sigi \phisd }
		= \int \ddcx \frac{i}{2} \paren{ \phis \sigi \pis - \phi \sigi \pi }
		= \ans{ \int \ddcx \frac{i}{2} (\phias \sigiab \pibs - \pia \sigiab \phib) },
	}
	as desired. \qed

	If $\Qi$ have the commutation relations of angular momentum, then we want to show $[\Qi, \Qj] = i \epsijk \Qk$~\cite[p.~158]{Sakurai}.  Generalizing Eq.~\refeq{comm2.a} to two fields, the canonical commutation relations are
	\al{
		[\phiavx, \pibvy] &= [\phiasvx, \pibsvy] = i \,\delab \,\del^3(\vx - \vy), \\
		[\phiavx, \phibvy] &= [\phiasvx, \phibsvy] = 0, \\
		[\piavx, \pibvy] &= [\piasvx, \pibsvy] = 0, \\
		[\phiavx, \pibsvy] &= [\phiavx, \phibsvy] = [\piavx, \pibsvy] = 0.
	}
	The commutation relations among the Pauli matrices are $[\sigi, \sigj] = 2 i \epsijk \sigk$~\cite[p.~165]{Sakurai}.
	
	Note that
	\aln{
		[\Qi, \Qj] &= \brac{ \int \ddcx \frac{i}{2} [ \phiasx \, \sigiab \, \pibsx - \piax \, \sigiab \, \phibx ], \int \ddcxp \frac{i}{2} [ \phicsxp \, \sigjcd \, \pidsxp - \picxp \, \sigjcd \, \phidxp ] } \notag \\
		&= -\frac{1}{4} \int \ddcx \int \ddcxp \bigg\{ \brac{ \phiasx \, \sigiab \, \pibsx, \phicsxp \, \sigjcd \, \pidsxp } - \brac{ \phiasx \, \sigiab \, \pibsx, \picxp \, \sigjcd \, \phidxp } \notag \\
		&\phantom{mmmmmmm} - \brac{ \piax \, \sigiab \, \phibx, \phicsxp \, \sigjcd \, \pidsxp } + \brac{ \piax \, \sigiab \, \phibx, \picxp \, \sigjcd \, \phidxp } \bigg\}. \label{fourterms}
	}
	For the first term in Eq.~\refeq{fourterms},
	\al{
		\int \ddcx \int \ddcxp &\brac{ \phiasx \, \sigiab \, \pibsx, \phicsxp \, \sigjcd \, \pidsxp } \\
		&= \int \ddcx \int \ddcxp \left\{ \phiasx \, [ \phicsxp \, \pibsx - i \, \delbc \, \del^3(x - x') ] \, \pidsxp \, \sigiab \, \sigjcd \right. \\
		&\phantom{mmmmmmmmmmmmm} - \left. \phicsxp \, [ \phiasx \, \pidsxp - i \, \delad \, \del^3(x - x') ] \, \pibsx \, \sigjcd \, \sigiab \right\} \\
		&= \int \ddcx \int \ddcxp \left[ \phiasx \, \pibsx \, \phicsxp \, \pidsxp \, \sigiab \, \sigjcd - i \, \delbc \, \del^3(x - x') \,  \phiasx \, \pidsxp \, \sigiab \, \sigjcd \right. \\
		&\phantom{mmmmmmmmmmmmm} - \left. \phicsxp \, \pidsxp \, \phiasx \, \pibsx \, \sigjcd \, \sigiab + i \, \delad \, \phicsxp \, \del^3(x - x') \, \pibsx \, \sigjcd \, \sigiab \right] \\
		&= \int \ddcx \int \ddcxp \, i \del^3(x - x') \brac{ \phicsxp \, \pibsx \, (\sigj \sigi)_{cb} - \phiasx \, \pidsxp \, (\sigi \sigj)_{ad} } \\
		&= \int \ddcx \, i \brac{ \phiasx \, \pibsx \, (\sigj \sigi)_{ab} - \phiasx \, \pibsx \, (\sigi \sigj)_{ab} }
		= -\int \ddcx \, i \, \phiasx \, \pibsx \, [\sigi, \sigj]_{ab} \\
		&= 2 \epsijk \int \ddcx \, \phiasx \, \pibsx \, \sigkab,
	}
	where we have interchanged $a \leftrightarrow c$ and $b \leftrightarrow d$ in one of the terms that canceled, and we have also relabeled indices in the second to last line.
	
	The second and third terms of Eq.~\refeq{fourterms} both vanish since they consist entirely of fields that commute.  The fourth term may be found using similar arithmetic as for the first:
	\al{
		\int \ddcx \int \ddcxp &\brac{ \piax \, \sigiab \, \phibx, \picxp \, \sigjcd \, \phidxp } \\
		&= 
	}
	
%	where
%	\al{
%		\phicsxp \, \pidsxp \, \phiasx \, \pibsx \, \sigjcd \, \sigiab &= \phicsxp \, [ \phiasx \, \pidsxp - i \, \delad \, \del^3(x - x') ] \, \pibsx \, \sigjcd \, \sigiab \\
%		&= \phiasx \, \phicsxp \, \pibsx \, \pidsxp \, \sigjcd \, \sigiab - i \, \delad \, \del^3(x - x') \, \phicsxp \, \pibsx \, \sigjcd \, \sigiab 
%	}
%	and
%	\al{
%		\phiasx \, \phicsxp \, \pibsx \, \pidsxp \, \sigjcd \, \sigiab &= \phiasx \, [ \pibsx \, \phicsxp + i \, \delbc \, \del^3(x - x') ] \, \pidsxp \, \sigjcd \, \sigiab \\
%		&= \phiasx \, \pibsx \, \phicsxp \, \pidsxp \, \sigjcd \, \sigiab + i \, \delbc \, \del^3(x - x') \, \phiasx \, \pidsxp \, \sigjcd \, \sigiab,
%	}
%%	and
%%	\eq{
%%		\phiasx \, \pibsx \, \phicsxp \, \pidsxp \, \sigjcd \, \sigiab = % \phiasx \, \pibsx \, \phicsxp \, \pidsxp \, ( \sigjab \, \sigicd 
%%	}
%	
%	so
%	\al{
%		\int \ddcx \int \ddcxp &\brac{ \phiasx \, \sigiab \, \pibsx, \phicsxp \, \sigjcd \, \pidsxp } \\
%		&= \int \ddcx \int \ddcxp i \brac{ \del^3(x - x') \, \phicsxp \, \pibsx \, (\sigi \sigj)_{bc} - \del^3(x - x') \, \phiasx \, \pidsxp \, (\sigi \sigj)_{ad} }
%	}
}






\state{(Peskin \& Schroeder 2.3)}{
	\label{3}
	
	Evaluate the function
	\eq{
		\ev{\phix \, \phiy}{0} = D(x - y)
		= \int \ddpf \frac{1}{2 \Ep} e^{i p \, (x - y)},
	}
	for $(x - y)$ spacelike so that $(x - y)^2 = -r^2$, explicitly in terms of Bessel functions.
}






\state{}{
	The classical limit of a harmonic oscillator can be described in terms of \emph{coherent states},
	\eq{
		\kalp = \exp( \alp \adag - \frac{1}{2} |\alp|^2 ) \ko.
	}
	When $\alp$ is large, the oscillator state is semiclassical.  Proceeding similarly for the Fourier modes of the quantum Klein-Gordon field,
	\al{
		\kf &= \Nf \exp( i \int \ddpf \fp \, \apdag ) \ko, &
		\Nf &= \exp( -\frac{1}{2} \int \ddpf | \fp |^2 ).
	}
}



\prob{}{
	Evaluate the expectation value of the field operator $\ev{\phix}{f}$ and show that it satisfies the Klein-Gordon equation.
}



\prob{}{
	Evaluate the relative mean square fluctuation of the occupation number of the mode with momentum $\vp$ and the relative mean square fluctuation in the total energy:
	\al{
		&\frac{\evnhps - \evnhp^2}{\evnhp^2}, &
		&\frac{\evHs - \evH^2}{\evH^2}.
	}
	Is either of these a good measure of the degree to which the field is classical?  Justify your answer.
}



\prob{}{
	Take $\Delta(x - y) = \ev{\phivx \, \phivy}{0}$ (equal times) as a measure of the fluctuations or correlations of the field amplitude.  Use your result from problem~\ref{3} to evaluate this quantity.  What is the meaning of the divergence as $\vx \to \vy$?
}


\clearpage

\makebib

\end{document}
