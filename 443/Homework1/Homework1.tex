\documentclass[11pt]{article}
\usepackage{homework}

\classname{443}
\homeworknum{1}


\begin{document}

% Environments

\newcommand{\state}[2]{\begin{statement}{#1} #2 \end{statement}}
\newcommand{\prob}[2]{\begin{problem}{#1} #2 \end{problem}}
\newcommand{\subprob}[1]{\begin{subproblem} #1 \end{subproblem}}
\newcommand{\sol}[1]{\begin{solution} #1 \end{solution}}
\newcommand{\fig}[2]{\begin{figure} \centering #2  \label{#1} \end{figure}}

\newcommand{\makebib}{
	\vfill
	\color{black}
	\bibliography{references}{}
	\bibliographystyle{lucas_unsrt}
}
	

% Implication

\newcommand{\qwhere}{\quad \text{where} \quad}
\newcommand{\qimplies}{\quad \implies \quad}
\newcommand{\impliesq}{\implies \quad}



% Brackets

\newcommand{\paren}[1]{\left( #1 \right)}
\newcommand{\brac}[1]{\left[ #1 \right]}


% Greek

\newcommand{\alp}{\alpha}
\newcommand{\bet}{\beta}
\newcommand{\gam}{\gamma}
\newcommand{\del}{\delta}
\newcommand{\eps}{\epsilon}
\newcommand{\zet}{\zeta}
\newcommand{\tht}{\theta}
\newcommand{\kap}{\kappa}
\newcommand{\lam}{\lambda}
\newcommand{\sig}{\sigma}
\newcommand{\ups}{\upsilon}
\newcommand{\omg}{\omega}

\newcommand{\Gam}{\Gamma}
\newcommand{\Del}{\Delta}
\newcommand{\Tht}{\Theta}
\newcommand{\Lam}{\Lambda}
\newcommand{\Sig}{\Sigma}
\newcommand{\Omg}{\Omega}
% Problem 1

\newcommand{\Psii}{\Psi^i}
\newcommand{\Psiix}{\Psii(x)}

\newcommand{\Pii}{\Pi^i}

\newcommand{\Phii}{\Phi^i}
\newcommand{\Phiix}{\Phii(x)}
\newcommand{\PhiN}{\Phi^N}
\newcommand{\PhiNx}{\PhiN(x)}
\newcommand{\Phiq}{\Phi^1}
\newcommand{\Phiw}{\Phi^2}

\newcommand{\ddcx}{\dd[3]{x}}

\newcommand{\delij}{\del^{i j}}
\newcommand{\delkl}{\del^{k l}}
\newcommand{\delil}{\del^{i l}}
\newcommand{\deljk}{\del^{j k}}
\newcommand{\delik}{\del^{i k}}
\newcommand{\deljl}{\del^{j l}}

\newcommand{\DF}{D_F}

\newcommand{\sigx}{\sig(x)}

\newcommand{\pii}{\pi^i}
\newcommand{\pij}{\pi^j}
\newcommand{\pik}{\pi^k}
\newcommand{\pil}{\pi^l}
\newcommand{\piix}{\pi(x)}

\newcommand{\pq}{p_1}
\newcommand{\pw}{p_2}
\newcommand{\pe}{p_3}
\newcommand{\pr}{p_4}

\newcommand{\vp}{\vb{p}}
\newcommand{\vpsi}{\vp_i}

\newcommand{\mpi}{m_\pi}


\state{(Peskin \& Schroeder 2.1)}{
	Classical electromagnetism (with no sources) follows from the action
	\al{
		S &= \int \ddqx \paren{ -\frac{1}{4} \Fsmn \Fmn }, &
		\where \Fsmn = \ptsm \Asn - \ptsn \Asm.
	}
}



\prob{}{
	Derive Maxwell's equations as the Euler-Lagrange equations of this action, treating the components $\Asm(x)$ as the dynamical variables.  Write the equations in standard form by identifying $\Ei = -\Foi$ and ${\epsijk \Bk = -\Fij}$.
}



\prob{}{
	Construct the energy-momentum tensor for this theory.  Note that the usual procedure does not result in a symmetric tensor.  To remedy that, we can add to $\Tmn$ a term of the form $\ptsl \Klmn$, where $\Klmn$ is antisymmetric in its first two indices.  Such an object is automatically divergenceless, so
	\eq{
		\Thmn = \Tmn + \ptsl \Klmn
	}
	is an equally good energy-momentum tensor with the same globally conserved energy and momentum.  Show that this construction, with
	\eq{
		\Klmn = \Fmn \An,
	}
	leads to an energy-momentum tensor $\Th$ that is symmetric and yields the standard formulae for the electromagnetic energy and momentum densities:
	\al{
		\cE &= \frac{E^2 + B^2}{2}; &
		\vS &= \vE \cross \vB.
	}
}



\state{The complex scalar field (Peskin \& Schroeder 2.2)}{
	Consider the field theory of a complex-valued scalar field obeying the Klein-Gordon equation.  The action of this theory is
	\eq{
		S = \int \ddqx (\ptsm \phis \ptm \phi - m^2 \phis \phi).
	}
	It is easiest to analyze this theory by considering $\phix$ and $\phisx$, rather than the real and imaginary parts of $\phix$, as the basic dynamical variables.
}



\prob{}{
	Find the conjugate momenta to $\phix$ and $\phisx$ and the canonical commutation relations.  Show that the Hamiltonian is
	\eq{
		H = \int \ddcx (\pis \pi + \grad \phis \vdot  \grad \phi + m^2 \phis \phi).
	}
	Compute the Heisenberg equation of motion for $\phix$ and show that it is indeed the Klein-Gordon equation.
}



\prob{}{
	Diagonalize $H$ by introducing creation and annihilation operators.  Show that the theory contains two sets of particles of mass $m$.
}



\prob{}{
	\label{2.3(c)}
	
	Rewrite the conserved charge
	\eq{
		Q = \int \ddcx \frac{i}{2} (\phis \pis - \pi \phi)
	}
	in terms of creation and annihilation operators, and evaluate the charge of the particles of each type.
}



\prob{}{
	Consider the case of two complex Klein-Gordon fields with the same mass.  Label the fields as $\phiax$, where $a = 1, 2$.  Show that there are now four conserved charges, one given by the generalization of part~\ref{2.3(c)}, and the other three given by
	\eq{
		\Qi = \int \ddcx \frac{i}{2} (\phias \sigiab \pibs - \pia \sigiab \phib),
	}
	where $\sigi$ are the Pauli sigma matrices.  Show that these three charges have the commutation relations of angular momentum ($SU(2)$).  Generalize these results to the case of $n$ identical complex scalar fields.
}




\state{(Peskin \& Schroeder 2.3)}{
	\label{3}
	
	Evaluate the function
	\eq{
		\ev{\phix \, \phiy}{0} = D(x - y)
	= \int \ddpf \frac{1}{2 \Ep} e^{i p \, (x - y)},
	}
	for $(x - y)$ spacelike so that $(x - y)^2 = -r^2$, explicitly in terms of Bessel functions.
}





\state{}{
	The classical limit of a harmonic oscillator can be described in terms of \emph{coherent states},
	\eq{
		\kalp = \exp( \alp \adag - \frac{1}{2} |\alp|^2 ) \ko.
	}
	When $\alp$ is large, the oscillator state is semiclassical.  Proceeding similarly for the Fourier modes of the quantum Klein-Gordon field,
	\al{
		\kf &= \Nf \exp( i \int \ddpf \fp \, \apdag ) \ko, &
		\Nf &= \exp( -\frac{1}{2} \int \ddpf | \fp |^2 ).
	}
}



\prob{}{
	Evaluate the expectation value of the field operator $\ev{\phix}{f}$ and show that it satisfies the Klein-Gordon equation.
}



\prob{}{
	Evaluate the relative mean square fluctuation of the occupation number of the mode with momentum $\vp$ and the relative mean square fluctuation in the total energy:
	\al{
		&\frac{\evnhps - \evnhp^2}{\evnhp^2}, &
		&\frac{\evHs - \evH^2}{\evH^2}.
	}
	Is either of these a good measure of the degree to which the field is classical?  Justify your answer.
}



\prob{}{
	Take $\Delta(x - y) = \ev{\phivx \, \phivy}{0}$ (equal times) as a measure of the fluctuations or correlations of the field amplitude.  Use your result from problem~\ref{3} to evaluate this quantity.  What is the meaning of the divergence as $\vx \to \vy$?
}




%\state{Acoustic and optic phonons in the diatomic chain}{
	In the diatomic chain, we take the unit cell to be of length $a$, and take $\xA$ and $\xB$ to be the coordinates of the A and B atoms within the unit cell.  Hence, in the $n$th cell,
	\al{
		\rnA &= n a + \xA; &
		\rnB &= n a + \xB
	}
	\vfix
}

\prob{}{
	In the equations of motion Eq.~(2.30), look for solutions of the form
	\eqn{5a}{
		\unalp = \ealpq \exp( i [ q \rnalp - \omgq t ] ) + \ealpsq \exp( i [-q \rnalp + \omgq t] )
	}
	where $\alp = A$ or $B$, and $\ealp$ are complex numbers that give the amplitude and phase of the oscillation of the two atoms.
	
	Separating out the terms that have the same time dependence, show that (for equal masses, ${\mA = \mB = m}$)
	\al{
		m \omgsq \eAq &= \DAAq \eAq + \DABq \eBq, \\
		m \omgsq \eBq &= \DBAq \eAq + \DBBq \eBq,
	}
	where
	\al{
		\DAAq &= \DBBq = K + K', \\
		-\DABq &= K \exp( i q [ \rnB - \rnA ] ) + K' \exp( i q [ \rnmqB - \rnA ] ), \\
		-\DBAq &= K \exp( i q [ \rnA - \rnB ] ) + K' \exp( i q [ \rnpqA - \rnB ] ).
	}
	Check that $\DAB = \DsBA$.
}

\sol{
	Equation~(2.30) is
	\aln{ \label{thing5a}
		\mA \pdv[2]{\unA}{t} &= K (\unB - \unA) + K' (\unmqB - \unA), &
		\mB \pdv[2]{\unB}{t} &= K' (\unpqA - \unB) + K (\unA - \unB).
	}
	Note that
	\al{
		\pdv[2]{\unalp}{t} &= \pdv{t}\{ -i \omgq \ealpq \exp( i [ q \rnalp - \omgq t ] ) + i \omgq \ealpsq \exp( i [-q \rnalp + \omgq t] ) \} \\
		&= -\omgsq \{ \ealpq \exp( i [ q \rnalp - \omgq t ] ) + \ealpsq \exp( i [-q \rnalp + \omgq t] ) \} \\
		&= -\omgsq \unalp,
	}
	so the first of Eq.~\refeq{5a} can be written
	\al{
		[ K + K' - \mA \omgsq ] \unA &= K \unB + K' \unmqB, \\[1ex]
		&= K \eBq e^{i q \rnB} e^{-i \omgq t} + K \eBsq e^{-i q \rnB} e^{i \omgq t} + K' \eBq e^{i q \rnmqB} e^{-i \omgq t} \\
		&\hspace{5em} \phantom{=\ } + K' \eBsq e^{-i q \rnmqB} e^{i \omgq t} \\[1ex]
		&= K \eBq e^{i q [ na + \xB ]} e^{-i \omgq t} + K' \eBq e^{i q [ (n - 1) a + \xB ]} e^{-i \omgq t} + K \eBsq e^{-i q [ na + \xB ]} e^{i \omgq t} \\
		&\hspace{5em} \phantom{=\ } + K' \eBsq e^{-i q [ (n - 1) a + \xB ]} e^{i \omgq t} \\[1ex]
		&= (K + e^{-i q a} K') \eBq e^{i q (na + \xB)} e^{-i \omgq t} + (K + e^{-i q a} K') \eBsq e^{-i q (na + \xB)} e^{i \omgq t} \\[1ex]
		&= (K + e^{-i q a} K') \unB.
	}
	Generalizing this, we have
	\al{
		[ K + K' - \mA \omgsq ] \unA &= (K + e^{-i q a} K') \unB, &
		[ K + K' - \mB \omgsq ] \unB &= (K + e^{i q a} K') \unA.
	}
	
	Collecting terms of like time dependence yields
	\aln{
		[ K + K' - m \omgsq ] \eAq e^{i q \rnA} &= (K + e^{-i q a} K') \eBq e^{i q \rnB}, \label{thing5.a1} \\
		[ K + K' - m \omgsq ] \eBq e^{i q \rnB} &= (K + e^{i q a} K') \eAq e^{i q \rnA}, \label{thing5.a2}
	}
	for $e^{-i \omg t}$, and
	\al{
		[ K + K' - m \omgsq ] \eAsq e^{-i q \rnA} &= (K + e^{-i q a} K') \eBsq e^{-i q \rnB}, \\
		[ K + K' - m \omgsq ] \eBsq e^{-i q \rnB} &= (K + e^{-i q a} K') \eAsq e^{-i q \rnA}.
	}
	for $e^{i \omg t}$.
	
	Rearranging Eqs.~\refeq{thing5.a1} and \refeq{thing5.a2}, we have
	\al{
		m \omgsq \eAq &= (K + K') \eAq - (K + e^{-i q a} K') e^{i q (\rnB - \rnA)} \eBq \\
		&= (K + K') \eAq - (e^{i q (\rnB - \rnA)} K + e^{i q (\rnmqB - \rnA)} K') \eBq, \\[1ex]
		m \omgsq \eBq &= -(K + e^{i q a} K') e^{i q (\rnA - \rnB)} \eAq - (K + K') \eBq \\
		&= -(e^{i q (\rnA - \rnB)} K + e^{i q (\rnpqA - \rnB)} K') \eAq - (K + K') \eBq,
	}
	which gives us
	\ans{ \al{
		\DAAq &= \DBBq = K + K', \\
		\DABq &= -e^{i q (\rnB - \rnA)} K - e^{i q (\rnmqB - \rnA)} K', \\
		\DBAq &= -e^{i q (\rnA - \rnB)} K - e^{i q (\rnpqA - \rnB)} K',
	}}%
	as we wanted to show. \qed
	
	Finally, note that
	\al{
		\DsBA &= [ -e^{i q (\rnA - \rnB)} K - e^{i q (\rnpqA - \rnB)} K' ]^*
		= -e^{i q (\rnB - \rnA)} K - e^{i q (\rnB - \rnpqA)} K' \\
		&= -e^{i q (\rnB - \rnA)} K - e^{i q (\rnB - \rnA)} e^{-i q a} K'
		= -e^{i q (\rnB - \rnA)} K - e^{i q (\rnmqB - \rnA)} K' \\
		&= \ans{ \DAB }
	}
	as desired. \qed
}



\prob{}{
	The $2 \times 2$ matrix equation can have a nontrivial solution if the determinant vanishes:
	\eq{
		\mqty| 	\DAAq - m \omgsq & \DABq \\
				\DBAq & \DBBq - m \omgsq |
		= 0.
	}
	Hence show that the frequencies of the modes are given by
	\eq{
		m \omgsq = K + K' \pm \sqrt{ (K + K')^2 - 4 K K' \sin[2]( \frac{q a}{2} ) }.
	}
	\vfix
}

\sol{
	The determinant is
	\eq{
		0 = [ \DAAq - m \omgsq ] [ \DBBq - m \omgsq ] - \DABq \DBAq
		= [ \DAAq - m \omgsq ]^2 - \DABq \DBAq,
	}
	which implies
	{\allowdisplaybreaks
	\aln{
		m \omgsq &= \DAAq \pm \sqrt{\DABq \DBAq} \notag \\
		&= K + K' \pm \sqrt{(e^{i q (\rnB - \rnA)} K + e^{i q (\rnmqB - \rnA)} K') (e^{i q (\rnA - \rnB)} K + e^{i q (\rnpqA - \rnB)} K')} \notag \\
		&= K + K' \pm \sqrt{(K + e^{-i q a} K') e^{i q (\rnB - \rnA)} (K + e^{i q a} K') e^{i q (\rnA - \rnB)}} \notag \\
		&= K + K' \pm \sqrt{(K + e^{-i q a} K') (K + e^{i q a} K')}
		= K + K' \pm \sqrt{K^2 + (e^{i q a} + e^{-i q a}) K K' + {K'}^2} \notag \\
		&= K + K' \pm \sqrt{K^2 + 2\cos(q a) K K' + {K'}^2} \label{5bo} \\
		&= K + K' \pm \sqrt{K^2 + \brac{ 2 - 4 \sin[2](\frac{q a}{2}) } K K' + {K'}^2} \notag \\
		&= K + K' \pm \sqrt{K^2 + 2 K K' + {K'}^2 - 4 \sin[2](\frac{q a}{2}) K K'} \notag \\
		&= \ans{ K + K' \pm \sqrt{ (K + K')^2 - 4 K K' \sin[2]( \frac{q a}{2} ) }, } \label{5b}
	}}
	where we have used the double-angle formula $\cos(2x) = 1 - 2 \sin[2](x)$~\cite{DoubleAngle}. \qed
}



\prob{}{
	Sketch the dispersion relations when $K / K' =  2$.
}

\sol{
	There are two dispersion curves since there are two solutions in Eq.~\refeq{5b}.  The expressions for the branches are
	\eqn{5ceq}{
		\omgq = \frac{1}{\sqrt{m}} \sqrt{ K + K' \pm \sqrt{ (K + K')^2 - 4 K K' \sin[2]( \frac{q a}{2} ) } }
		\begin{cases}
			\text{optical}, \\
			\text{acoustic},
		\end{cases}
	}
	where the acoustic~(optical) branch corresponds to the upper~(lower) sign.  Both branches are shown in Fig.~\ref{5c}, with the $K / K' =  2$ case on the left and the $K = K'$ case on the right.
	
	\fig{5c}{
		\includegraphics[width=0.5\textwidth,trim=1.5cm 0 0 0,clip]{5c}
		\caption{Dispersion curves for $K / K' =  2$.  The optical branch~(blue) corresponds to the upper sign in Eq.~\refeq{5ceq}, and the acoustic branch~(gold) to the lower sign.}
	}
}



\prob{}{
	  Discuss what happens if $K = K'$.
}

\sol{
	If $K = K'$, then not only are the masses of the two atoms identical, but so are their restorative forces.  Thus, the system is essentially reduced to a monatomic chain~\cite[p.~437]{Ashcroft}.  Picking up from Eq.~\refeq{5bo},
	\eq{
		m \omgsq = 2 K \pm \sqrt{2 K^2 + 2 \cos(q a) K^2}
		= 2 K \pm K \sqrt{4 \cos[2](\frac{q a}{2})}
		= 2 K \brac{ 1 - \cos(\frac{q a}{2}) }
		= 4 K \sin[2](\frac{q a}{4}),
	}
	where we have used the double-angle formula $\cos(2 x) = 2 \cos[2](x) - 1$~\cite{DoubleAngle}.  This is Eq.~\refeq{2.25} with $q a \to q a / 2$.  So in this limit, the diatomic chain is reduced to a monatomic chain with lattice constant $a / 2$~\cite[p.~437]{Ashcroft}.
}


%\makebib

\end{document}
