\documentclass[11pt]{article}
\usepackage{homework}

\classname{443}
\homeworknum{1}


\begin{document}

% Environments

\newcommand{\state}[2]{\begin{statement}{#1} #2 \end{statement}}
\newcommand{\prob}[2]{\begin{problem}{#1} #2 \end{problem}}
\newcommand{\subprob}[1]{\begin{subproblem} #1 \end{subproblem}}
\newcommand{\sol}[1]{\begin{solution} #1 \end{solution}}
\newcommand{\fig}[2]{\begin{figure} \centering #2  \label{#1} \end{figure}}

\newcommand{\makebib}{
	\vfill
	\color{black}
	\bibliography{references}{}
	\bibliographystyle{lucas_unsrt}
}
	

% Implication

\newcommand{\qwhere}{\quad \text{where} \quad}
\newcommand{\qimplies}{\quad \implies \quad}
\newcommand{\impliesq}{\implies \quad}



% Brackets

\newcommand{\paren}[1]{\left( #1 \right)}
\newcommand{\brac}[1]{\left[ #1 \right]}


% Greek

\newcommand{\alp}{\alpha}
\newcommand{\bet}{\beta}
\newcommand{\gam}{\gamma}
\newcommand{\del}{\delta}
\newcommand{\eps}{\epsilon}
\newcommand{\zet}{\zeta}
\newcommand{\tht}{\theta}
\newcommand{\kap}{\kappa}
\newcommand{\lam}{\lambda}
\newcommand{\sig}{\sigma}
\newcommand{\ups}{\upsilon}
\newcommand{\omg}{\omega}

\newcommand{\Gam}{\Gamma}
\newcommand{\Del}{\Delta}
\newcommand{\Tht}{\Theta}
\newcommand{\Lam}{\Lambda}
\newcommand{\Sig}{\Sigma}
\newcommand{\Omg}{\Omega}
% Problem 1

\newcommand{\Psii}{\Psi^i}
\newcommand{\Psiix}{\Psii(x)}

\newcommand{\Pii}{\Pi^i}

\newcommand{\Phii}{\Phi^i}
\newcommand{\Phiix}{\Phii(x)}
\newcommand{\PhiN}{\Phi^N}
\newcommand{\PhiNx}{\PhiN(x)}
\newcommand{\Phiq}{\Phi^1}
\newcommand{\Phiw}{\Phi^2}

\newcommand{\ddcx}{\dd[3]{x}}

\newcommand{\delij}{\del^{i j}}
\newcommand{\delkl}{\del^{k l}}
\newcommand{\delil}{\del^{i l}}
\newcommand{\deljk}{\del^{j k}}
\newcommand{\delik}{\del^{i k}}
\newcommand{\deljl}{\del^{j l}}

\newcommand{\DF}{D_F}

\newcommand{\sigx}{\sig(x)}

\newcommand{\pii}{\pi^i}
\newcommand{\pij}{\pi^j}
\newcommand{\pik}{\pi^k}
\newcommand{\pil}{\pi^l}
\newcommand{\piix}{\pi(x)}

\newcommand{\pq}{p_1}
\newcommand{\pw}{p_2}
\newcommand{\pe}{p_3}
\newcommand{\pr}{p_4}

\newcommand{\vp}{\vb{p}}
\newcommand{\vpsi}{\vp_i}

\newcommand{\mpi}{m_\pi}


\state{(Peskin \& Schroeder 2.1)}{
	Classical electromagnetism (with no sources) follows from the action
	\aln{ \label{action1.a}
		S &= \int \ddqx \paren{ -\frac{1}{4} \Fsmn \Fmn }, &
		\where \Fsmn = \ptsm \Asn - \ptsn \Asm.
	}
\vfix
}

\prob{}{
	Derive Maxwell's equations as the Euler-Lagrange equations of this action, treating the components $\Asm(x)$ as the dynamical variables.  Write the equations in standard form by identifying
	\aln{ \label{identities1.a}
		\Ei &= -\Foi; &
		\epsijk \Bk &= -\Fij.
	}
\vfix
}

\sol{
	We want to extremize the action,
	\eq{
		S[\Asm] = \int \ddqx \cL(\Asm, \ptsm \Asm)
	}
	Let $\del\Asm$ denote some arbitrary variation that vanishes at the boundaries of spacetime.  The action for $\Asm + \del\Asm$ is
	\eq{
		S[\Asm + \del\Asm] = \int \ddqx \cL(\Asm + \del\Asm, \ptsn\Asm + \ptsn\del\Asm).
	}
	Then, to first order in $\del\Asm$, the variation of the action is
	\eq{
		\del S = S[\Asm + \del\Asm] - S[\Asm],
	}
	which we want to vanish for all $\del\Asm$.  Let $\del\Fsmn = \ptsm \, \del\Asn - \ptsn \, \del\Asm$.  Then, applying the definition of $\Fsmn$ given in Eq.~\refeq{action1.a},
	\aln{
		\del S &= \int  \ddqx \paren{ -\frac{1}{4} (\Fsmn + \del\Fsmn) (\Fmn + \del\Fmn) + \frac{1}{4} \Fsmn \Fmn } \notag \\
		&\approx \int \ddqx \paren{ -\frac{1}{4} (\Fsmn \Fmn + \Fsmn \,\del\Fmn + \del\Fsmn \,\Fmn) + \frac{1}{4} \Fsmn \Fmn } \notag \\
		&= \int \ddqx \paren{ -\frac{1}{4} (\Fsmn \,\del\Fmn + \del\Fsmn \,\Fmn) } \notag \\
		&= \int \ddqx \paren{ -\frac{1}{2} \del\Fsmn \,\Fmn }, \label{delS1.a}
	}
	where we have discarded terms of $\order{(\del\Am)^2}$ and swapped covariant and contravariant indices in going to the final equality.
	
	Note that
	\aln{
		\del\Fsmn \,\Fmn &= (\ptsm \del\Asn - \ptsn \del\Asm) (\ptm \An - \ptn \Am) \notag \\
		&= \ptsm \del\Asn \,\ptm \An - \ptsm \del\Asn \,\ptn \Am - \ptsn \del\Asm \,\ptm \An + \ptsn \del\Asm \,\ptn \Am.  \label{exp1.a}
	}
	Integrating the first term of Eq.~\refeq{exp1.a} by parts, we have
	\al{
		\int \ddqx \pdv{\,\del\Asn}{\xm} \pdv{\An}{\xsm} = \bigg[ \del\Asn \pdv{\An}{\xsm} \bigg]_{-\infty}^\infty - \int \ddqx \del\Asn \pdv{\An}{\xm}{\xsm}
		= -\int \ddqx \del\Asn \,\ptsm \ptm \An,
	}
	because $\del\An$ vanishes at $\pm\infty$.  The other terms follow similarly.  Then we find
	\al{
		\int \ddqx \del\Fsmn \,\Fmn &= -\int \ddqx (\del\Asn \,\ptsm \ptm \An - \del\Asn \,\ptsm \ptn \Am - \del\Asm \,\ptsn \ptm \An + \del\Asm \,\ptsn \ptn \Am) \\
		&= -\int \ddqx (\del\Asn \,\ptsm \Fmn + \del\Asm \,\ptsn \Fnm)
		= -\int \ddqx (\del\Asn \,\ptsm \Fmn + \del\Asn \,\ptsm \Fmn) \\
		&= -2 \int \ddqx \del\Asn \,\ptsm \Fmn,
	}
	where in going to the second-to-last equality we have simply swapped the indices.
	
	Making this substitution in Eq.~\refeq{delS1.a}, we obtain
	\eq{
		\del S = \del\Asn \int \ddqx \ptsm \Fmn.
	}
	In order for the action to be at a local extremum, we need $\del S = 0$ for any $\del\Asn$.  This implies that the integrand is 0.  Thus, we obtain
	\eqn{maxwell1.a}{
		\ans{ \ptsm \Fmn = 0, }
	}
	which is the covariant form of the inhomogeneous Maxwell equations in a source-free region~\cite[p.~557]{Jackson}, as we sought to derive. \qed
	
	From Eq.~\refeq{identities1.a} and the knowledge that $\Fmn$ is antisymmetric~\cite[p.~556]{Jackson}, we can write
	\eq{
		\Fmn = \mqty[
			0 & -\Ex & -\Ey & -\Ez \\
			\Ex & 0 & -\Bz & \By \\
			\Ey & \Bz & 0 & -\Bx \\
			\Ez & -\By & \Bx & 0
		].
	}
	The first equation of Eq.~\refeq{identities1.a} is equivalent to $\Ei = \Fio$.  Then the zeroth component of Eq.~\refeq{maxwell1.a} can be written
	\eq{
		\ptsm \Fmo = \pdv{\Ex}{x} + \pdv{\Ey}{y} + \pdv{\Ez}{z}
		= \ans{ \grad \vdot \vE
		= 0, }
	}
	which is the differential form of Gauss's law.
	
	For the remaining components of Eq.~\refeq{maxwell1.a}, we apply the second equation of Eq.~\refeq{maxwell1.a} to find
	\eq{
		\ptsm \Fmi = -\pdv{\Ei}{t} + \epsijk \pdv{\Bk}{\xj} = 0.
	}
	In vector form, this is
	\eq{
		\ans{ \grad \cross \vB - \pdv{\vE}{t} = \vo, }
	}
	the differential form of Amp\`{e}re's law.
}



%\prob{}{
%	Construct the energy-momentum tensor for this theory.  Note that the usual procedure does not result in a symmetric tensor.  To remedy that, we can add to $\Tmn$ a term of the form $\ptsl \Klmn$, where $\Klmn$ is antisymmetric in its first two indices.  Such an object is automatically divergenceless, so
%	\eq{
%		\Thmn = \Tmn + \ptsl \Klmn
%	}
%	is an equally good energy-momentum tensor with the same globally conserved energy and momentum.  Show that this construction, with
%	\eq{
%		\Klmn = \Fmn \An,
%	}
%	leads to an energy-momentum tensor $\Th$ that is symmetric and yields the standard formulae for the electromagnetic energy and momentum densities:
%	\al{
%		\cE &= \frac{E^2 + B^2}{2}; &
%		\vS &= \vE \cross \vB.
%	}
%}






%\state{The complex scalar field (Peskin \& Schroeder 2.2)}{
%	Consider the field theory of a complex-valued scalar field obeying the Klein-Gordon equation.  The action of this theory is
%	\eq{
%		S = \int \ddqx (\ptsm \phis \ptm \phi - m^2 \phis \phi).
%	}
%	It is easiest to analyze this theory by considering $\phix$ and $\phisx$, rather than the real and imaginary parts of $\phix$, as the basic dynamical variables.
%}



%\prob{}{
%	Find the conjugate momenta to $\phix$ and $\phisx$ and the canonical commutation relations.  Show that the Hamiltonian is
%	\eq{
%		H = \int \ddcx (\pis \pi + \grad \phis \vdot  \grad \phi + m^2 \phis \phi).
%	}
%	Compute the Heisenberg equation of motion for $\phix$ and show that it is indeed the Klein-Gordon equation.
%}



%\prob{}{
%	Diagonalize $H$ by introducing creation and annihilation operators.  Show that the theory contains two sets of particles of mass $m$.
%}



%\prob{}{
%	\label{2.3(c)}
%	
%	Rewrite the conserved charge
%	\eq{
%		Q = \int \ddcx \frac{i}{2} (\phis \pis - \pi \phi)
%	}
%	in terms of creation and annihilation operators, and evaluate the charge of the particles of each type.
%}



%\prob{}{
%	Consider the case of two complex Klein-Gordon fields with the same mass.  Label the fields as $\phiax$, where $a = 1, 2$.  Show that there are now four conserved charges, one given by the generalization of part~\ref{2.3(c)}, and the other three given by
%	\eq{
%		\Qi = \int \ddcx \frac{i}{2} (\phias \sigiab \pibs - \pia \sigiab \phib),
%	}
%	where $\sigi$ are the Pauli sigma matrices.  Show that these three charges have the commutation relations of angular momentum ($SU(2)$).  Generalize these results to the case of $n$ identical complex scalar fields.
%}






%\state{(Peskin \& Schroeder 2.3)}{
%	\label{3}
%	
%	Evaluate the function
%	\eq{
%		\ev{\phix \, \phiy}{0} = D(x - y)
%	= \int \ddpf \frac{1}{2 \Ep} e^{i p \, (x - y)},
%	}
%	for $(x - y)$ spacelike so that $(x - y)^2 = -r^2$, explicitly in terms of Bessel functions.
%}






%\state{}{
%	The classical limit of a harmonic oscillator can be described in terms of \emph{coherent states},
%	\eq{
%		\kalp = \exp( \alp \adag - \frac{1}{2} |\alp|^2 ) \ko.
%	}
%	When $\alp$ is large, the oscillator state is semiclassical.  Proceeding similarly for the Fourier modes of the quantum Klein-Gordon field,
%	\al{
%		\kf &= \Nf \exp( i \int \ddpf \fp \, \apdag ) \ko, &
%		\Nf &= \exp( -\frac{1}{2} \int \ddpf | \fp |^2 ).
%	}
%}



%\prob{}{
%	Evaluate the expectation value of the field operator $\ev{\phix}{f}$ and show that it satisfies the Klein-Gordon equation.
%}



%\prob{}{
%	Evaluate the relative mean square fluctuation of the occupation number of the mode with momentum $\vp$ and the relative mean square fluctuation in the total energy:
%	\al{
%		&\frac{\evnhps - \evnhp^2}{\evnhp^2}, &
%		&\frac{\evHs - \evH^2}{\evH^2}.
%	}
%	Is either of these a good measure of the degree to which the field is classical?  Justify your answer.
%}



%\prob{}{
%	Take $\Delta(x - y) = \ev{\phivx \, \phivy}{0}$ (equal times) as a measure of the fluctuations or correlations of the field amplitude.  Use your result from problem~\ref{3} to evaluate this quantity.  What is the meaning of the divergence as $\vx \to \vy$?
%}


%\makebib

\end{document}
