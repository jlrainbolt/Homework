\documentclass[11pt]{article}
\usepackage{homework}

\classname{443}
\homeworknum{4}



\begin{document}

% Environments

\newcommand{\state}[2]{\begin{statement}{#1} #2 \end{statement}}
\newcommand{\prob}[2]{\begin{problem}{#1} #2 \end{problem}}
\newcommand{\subprob}[1]{\begin{subproblem} #1 \end{subproblem}}
\newcommand{\sol}[1]{\begin{solution} #1 \end{solution}}
\newcommand{\fig}[2]{\begin{figure} \centering #2  \label{#1} \end{figure}}

\newcommand{\makebib}{
	\vfill
	\color{black}
	\nocite{*}
	\bibliography{references}{}
	\bibliographystyle{lucas_unsrt}
}
	

% Implication

\newcommand{\qwhere}{\quad \text{where} \quad}
\newcommand{\qimplies}{\quad \implies \quad}
\newcommand{\impliesq}{\implies \quad}



% Brackets

\newcommand{\paren}[1]{\left( #1 \right)}
\newcommand{\brac}[1]{\left[ #1 \right]}
\newcommand{\curly}[1]{\left\{ #1 \right\}}


% Greek

\newcommand{\alp}{\alpha}
\newcommand{\bet}{\beta}
\newcommand{\gam}{\gamma}
\newcommand{\del}{\delta}
\newcommand{\eps}{\epsilon}
\newcommand{\zet}{\zeta}
\newcommand{\tht}{\theta}
\newcommand{\kap}{\kappa}
\newcommand{\lam}{\lambda}
\newcommand{\sig}{\sigma}
\newcommand{\ups}{\upsilon}
\newcommand{\omg}{\omega}

\newcommand{\Gam}{\Gamma}
\newcommand{\Del}{\Delta}
\newcommand{\Tht}{\Theta}
\newcommand{\Lam}{\Lambda}
\newcommand{\Sig}{\Sigma}
\newcommand{\Omg}{\Omega}


% Text

\newcommand{\where}{\text{where }}

% Problem 1

\newcommand{\Hint}{H_\text{int}}
\newcommand{\ddcx}{\dd[3]{x}}
\newcommand{\psib}{\bar{\psi}}

\newcommand{\mh}{m_h}
\newcommand{\mmu}{m_\mu}
\newcommand{\me}{m_e}
\newcommand{\ma}{m_a}

\newcommand{\aexpt}{a_\text{expt.}}
\newcommand{\aQED}{a_\text{QED}}
\renewcommand{\GeV}{\giga\electronvolt}

\newcommand{\gamt}{\gam^5}



\state{Linear sigma model (Peskin \& Schroeder 4.3)}{
	The interactions of pions at low energy can be described by a phenomenological model called the \emph{linear sigma model}.  Essentially, this model consists of $N$ real scalar fields coupled by a $\phi^4$ interaction that is symmetric under rotations of the $N$ fields.  More specifically, let $\Psiix$, $i = 1, \ldots, N$ be a set of $N$ fields, governed by the Hamiltonian
	\eq{
		H = \int \ddcx \paren{ \frac{1}{2} (\Pii)^2 + \frac{1}{2} (\grad \Phii)^2 + V(\Phi^2) },
	}
	where $(\Phii)^2 = \vPhi \vdot \vPhi$, and
	\eqn{V1}{
		V(\Phi^2) = \frac{1}{2} m^2 (\Phii)^2 + \frac{\lam}{4} \paren{ (\Phii)^2 }^2
	}
	is a function symmetric under rotations of $\vPhi$.  For (classical) field configurations of $\Phiix$ that are constant in space and time, this term gives the only contribution to $H$; hence, $V$ is the field potential energy.
}

\prob{
	Analyze the linear sigma model for $m^2 > 0$ by noticing that, for $\lam = 0$, the Hamiltonian given above is exactly $N$ copies of the Klein-Gordon Hamiltonian.  We can then calculate scattering amplitudes as perturbation series in the parameter $\lam$.  Show that the propagator is
	\eq{
		\wick{\c{\Phi}^i(x) \c{\Phi}^j(y)} = \delij \DF(x - y),
	}
	where $\DF$ is the standard Klein-Gordon propagator for mass $m$, and that there is one type of vertex given by
	\eqn{diag1a}{
		\centergraphics{diag/ijkl_vertex} = -2 i \lam (\delij \delkl + \delil \deljk + \delik \deljl).
	}
	Compute, to leading order in $\lam$, the differential cross sections $\dv*{\sig}{\Omg}$, in the center-of-mass frame, for the scattering processes
	\al{
		\Phiq \Phiw &\to \Phiq \Phiw, &
		\Phiq \Phiq &\to \Phiw \Phiw, &
		\Phiq \Phiq &\to \Phiq \Phiq
	}
	as functions of the center-of-mass energy.
}

\sol{
	The Klein-Gordon Hamiltonian is given by Peskin \& Schroeder~(2.8),
	\eqn{KG}{
		H = \int \ddcx \paren{ \frac{1}{2} \pi^2 + \frac{1}{2} (\grad \phi)^2 + \frac{1}{2} m^2 \phi^2 }
	}
	For $\lam = 0$, the linear sigma model has the Hamiltonian
	\eq{
		H = \int \ddcx \paren{ \frac{1}{2} (\Pii)^2 + \frac{1}{2} (\grad \Phii)^2 + \frac{1}{2} m^2 (\Phii)^2 }
	}
	which is clearly $N$ copies of the Klein-Gordon Hamiltonian, one for each $i$.
	
	From (4.36) we know that the Feynman propagator is the contraction of two fields:
	\eq{
		\wick{\c\phi(x) \c\phi(y)} = \DF(x - y).
	}
	No terms $\Phii \Phij$ for $i \neq j$ appear in the Hamiltonian, so fields with $i \neq j$ cannot be contracted.  Moreover, each field is governed by its own independent Klein-Gordon Hamiltonian to zeroth order.  So the propagator must be
	\eq{
		\ans{ \wick{\c{\Phi}^i(x) \c{\Phi}^j(y)} = \delij \DF(x - y) }
	}
	where $\DF(x - y)$ is the Klein-Gordon propagator. \qed
	
	In order to determine the Feynman rules, we use Peskin \& Schroeder~(4.90),
	\eq{
		\mel*{\vpsq \cdots \vpsn}{i T}{\vpsA \vpsB} = \lim_{T \to \infty (1 - i \eps)} \sOi \mel*{\vpsq \cdots \vpsn}{T \curly{ \exp(-i \intmTT \ddt \HIt)}}{\vpsA \vpsB} \sOf.
	}
	Our interaction Hamiltonian is
	\eq{
		\HI = \int \ddcx \frac{\lam}{4} \paren{ (\Phii)^2 }^2
		= \int \ddcx \frac{\lam}{4} (\vPhi \vdot \vPhi)^2
		= \frac{\lam}{4} \int \ddcx \paren{ \sumi (\Phii)^4 + 2 \suminj (\Phii)^2 (\Phij)^2 },
	}
	W have two final momenta, $\vpsq$ and $\vpsw$.  Now we have
	\eq{
		\mel*{\vpsq \vpsw}{i T}{\vpsA \vpsB} = \lim_{T \to \infty (1 - i \eps)} \sOi \mel*{\vpsq \vpsw}{T \curly{ \exp[ -i \int \ddqx \frac{\lam}{4} \paren{  \sumi (\Phii)^4 + 2 \suminj (\Phii)^2 (\Phij)^2 } ] }}{\vpsA \vpsB} \sOf.
	}
	The first term that contributes to leading order is, by analogy to (4.92),
	\al{
		&\sOi \mel*{\vpsq \vpsw}{T \curly{ -i \int \ddqx \frac{\lam}{4} \paren{ \sumi (\Phii)^4 + 2 \suminj (\Phii)^2 (\Phij)^2 } }}{\vpsA \vpsB} \sOf \\
		&\hspace{5em} = \sOi \mel*{\vpsq \vpsw}{N \curly{ -i \int \ddqx \frac{\lam}{4} \paren{ \sumi (\Phii)^4 + 2 \suminj (\Phii)^2 (\Phij)^2 } + \contractions }}{\vpsA \vpsB} \sOf,
	}
	but only the terms in which none of the fields are contracted with each other will contribute~\cite[p.~111]{Peskin}.
	
	The first term represents the process $\Phii \Phii \to \Phii \Phii$:
	\eq{
		\sOi \mel*{\vpsq \vpsw}{N \curly{ -i \int \ddqx \frac{\lam}{4} \sumi \Phii \Phii \Phii \Phii + \contractions }}{\vpsA \vpsB} \sOf.
	}
	The fields are all the same, so there are $4!$ ways of contracting the fields with the momenta, and we will obtain a diagram similar to (4.98).  Adapting that expression, we find
	\eq{
		-4! i \int \frac{\lam}{4} \ddqx e^{-i (\psA + \psB - \psq - \psw) \cdot x}
		= -6 i \lam (4 \pi)^4 \del^4(\psA + \psB - \psq - \psw).
	}
	The diagram in Eq.~\refeq{diag1a} is $\Phii \Phij \to \Phik \Phil$.  Since $i = j = k = l$ for this term, we have
	\eq{
		\centergraphics{diag/ijkl_vertex} = -6 i \lam
		= -2 i \lam (1 + 1 + 1)
		= \ans{ -2 i \lam (\delij \delkl + \delil \deljk + \delik \deljl). }
	}
	The second term can represent the processes $\Phii \Phii \to \Phij \Phij$ or $\Phii \Phij \to \Phii \Phij$ (where the indices and the order of the fields on either side is interchangeable):
	\eq{
		\sOi \mel*{\vpsq \vpsw}{N \curly{ -i \int \ddqx \frac{\lam}{4} 2 \suminj \Phii \Phii \Phij \Phij + \contractions }}{\vpsA \vpsB} \sOf.
	}
	Now there are only $2! \times 2! = 4$ ways to contract the fields with the momenta.  We have
	\eq{
		-4 i \int \frac{\lam}{2} \ddqx e^{-i (\psA + \psB - \psq - \psw) \cdot x}
		= -2 i \lam (4 \pi)^4 \del^4(\psA + \psB - \psq - \psw).
	}
	Here, either $i = j$ and $k = l$, $i = l$ and $j = k$, or $i = k$ and $j = l$.  We have
	\eq{
		\centergraphics{diag/ijkl_vertex} = -2 i \lam
		= -2 i \lam (1 + 0 + 0)
		= \ans{ -2 i \lam (\delij \delkl + \delil \deljk + \delik \deljl). }
	}
	Both of the terms can therefore be represented by Eq.~\refeq{diag1a} as we wanted to show. \qed
	
	When all four of the particles in the interaction have the same mass, the differential cross section in the center-of-mass frame is given by Peskin \& Schroeder~(4.85)
	\eq{
		\paren{ \dv{\sig}{\Omg} }_\CM = \frac{\abs{\cM}^2}{64 \pi^2 \Ecm^2},
	}
	where $\Ecm$ is the center-of-mass energy and $\cM$ is the invariant matrix element.  We know that the diagrams we calculated before have the form $i \cM (2\pi)^4 \del^4(\psA + \psB - \psq - \psw)$~\cite[p.~112]{Peskin}.  Then
	\al{
		\cM = -6 \lam &\qq{for} \Phiq \Phiq \to \Phiq \Phiq, &
		\cM = -2 \lam &\qq{for} \Phiq \Phiw \to \Phiq \Phiw \text{ and } \Phiq \Phiq \to \Phiw \Phiw.
	}
	So the differential cross sections are, to leading order in $\lam$,
	\al{
		(\Phiq \Phiw \to \Phiq \Phiw) &\quad
		\paren{ \dv{\sig}{\Omg} }_\CM = \frac{\abs{-2 \lam}^2}{64 \pi^2 \Ecm^2}
		= \ans{ \frac{\lam^2}{16 \pi^2 \Ecm^2}, } \\
		(\Phiq \Phiw \to \Phiq \Phiw) &\quad
		\paren{ \dv{\sig}{\Omg} }_\CM = \frac{\abs{-6 \lam}^2}{64 \pi^2 \Ecm^2}
		= \ans{ \frac{9 \lam^2}{16 \pi^2 \Ecm^2}, } \\
		(\Phiq \Phiq \to \Phiw \Phiw) &\quad
		\paren{ \dv{\sig}{\Omg} }_\CM = \frac{\abs{-6 \lam}^2}{64 \pi^2 \Ecm^2}
		= \ans{ \frac{9 \lam^2}{16 \pi^2 \Ecm^2}. }
	}
	\vfix
}



\prob{
	Now consider the case $m^2 < 0$: $m^2 = -\mu^2$.  In this case, $V$ has a local maximum, rather than a minimum, at $\Phii = 0$.  Since $V$ is a potential energy, this implies that the ground state of the theory is not near $\Phii = 0$ but rather is obtained by shifting $\Phii$ toward the minimum of $V$.  By rotational invariance, we can consider this shift to be in the $N$th direction.  Write, then,
	\al{
		\Phiix &= \piix, \qquad i = 1, \ldots, N - 1, &
		\PhiNx &= v + \sigx
	}
	where $v$ is a constant chosen to minimize $V$.  (The notation $\pii$ suggests a pion field and should not be confused with a canonical momentum.)  Show that, in these new coordinates (and substituting for $v$ its expression in terms of $\lam$ and $\mu$), we have a theory of a massive $\sig$ field and $N - 1$ \emph{massless} pion fields, interacting through cubic and quartic potential energy terms which all become small as $\lam \to 0$.  Construct the Feynman rules by assigning values to the propagators and vertices:
	\al{
		\wick{\c{\sig} \c{\sig}} &= \centergraphics{diag/dbl_arrow} &\qquad
		& \qquad \qquad \centergraphics{diag/ij_triangle} \qquad \centergraphics{diag/dbl_triangle} \\
		\wick{\c{\pi}^i \c{\pi}^j} &= \centergraphics{diag/ij_arrow} &\qquad
		& \centergraphics{diag/ijkl_vertex} \qquad \centergraphics{diag/ij_vertex} \qquad \centergraphics{diag/dbl_vertex}
	}
	\vfix
}

\sol{
	With the negative mass, Eq.~\refeq{V1} becomes
	\eq{
		V(\Phi^2) = -\frac{1}{2} \mu^2 (\Phii)^2 + \frac{\lam}{4} \paren{ (\Phii)^2 }^2.
	}
	To find the minimum of $V$, we differentiate with respect to $\PhiN$.  We stipulate
	\eq{
		0 = \pdv{V}{\PhiN} = -\mu^2 \PhiN + \lam (\vPhi \vdot \vPhi) \PhiN
		= -\mu^2 \PhiN + \lam (\PhiN)^3,
	}
	where we have used the chain rule to evaluate the second term, and the fact that $V$ is minimal for all $\Phii = 0$ with $i \neq N$.  This implies $\PhiN = 0$ or
	\eq{
		(\PhiN)^2 = \frac{\mu^2}{\lam}
	}
	when $V$ is minimal.  Thus
	\eq{
		v = \frac{\mu}{\sqrt{\lam}}.
	}
	In order to determine the form of the theory, we need to rewrite $V(\Phi^2)$ in the new coordinates.  Note that $\vPhi = (\vpi, v + \sig)$.  Then
	\al{
		V(\Phi^2) &= -\frac{1}{2} \mu^2 \brac{ \vpi^2 + (v + \sig)^2 } + \frac{\lam}{4} \brac{ \vpi^2 + (v + \sig)^2 }^2 \\[1ex]
		%
		&= -\frac{1}{2} \mu^2 \paren{ \vpi^2 + \frac{\mu^2}{\lam} + 2 \frac{\mu \sig}{\sqrt{\lam}} + \sig^2 } + \frac{\lam}{4} \paren{ \vpi^2 + \frac{\mu^2}{\lam} + 2 \frac{\mu \sig}{\sqrt{\lam}} + \sig^2 }^2 \\[1ex]
		%
		&= -\frac{1}{2} \mu^2 \paren{ \vpi^2 + \frac{\mu^2}{\lam} + 2 \frac{\mu \sig}{\sqrt{\lam}} + \sig^2 } \\
		&\hspace{5em} \phantom{=\ } + \frac{\lam}{4} \paren{ (\vpi^2)^2 + 2 \frac{\vpi^2 \mu^2}{\lam} + \frac{\mu^4}{\lam^2} + 4 \frac{\vpi^2 \mu \sig}{\sqrt{\lam}} + 4 \frac{\mu^3 \sig}{\lam^{3/2}} + 2 \vpi^2 \sig^2 + 6 \frac{\mu^2 \sig^2}{\lam} + 4 \frac{\mu \sig^3}{\sqrt{\lam}} + \sig^4 } \\[1ex]
		%
		&= -\frac{\vpi^2 \mu^2}{2} - \frac{\mu^4}{2 \lam} - \frac{\mu^3 \sig}{\sqrt{\lam}} - \frac{\mu^2 \sig^2}{2} + \frac{(\vpi^2)^2 \lam}{4} + \frac{\vpi^2 \mu^2}{2} + \frac{\mu^4}{4 \lam} \\
		&\hspace{5em} \phantom{=\ } + \vpi^2 \mu \sig \sqrt{\lam} + \frac{\mu^3 \sig}{\sqrt{\lam}} + \frac{\vpi^2 \sig^2 \lam}{2} + \frac{3 \mu^2 \sig^2}{2} + \mu \sig^3 \sqrt{\lam} + \frac{\sig^4 \lam}{4} \\[1ex]
		%
		&= \ans{ -\frac{\mu^4}{4 \lam} + \mu^2 \sig^2 + \frac{(\vpi^2)^2 \lam}{4} + \vpi^2 \mu \sig \sqrt{\lam} + \frac{\vpi^2 \sig^2 \lam}{2} + \mu \sig^3 \sqrt{\lam} + \frac{\sig^4 \lam}{4}. }
	}
	This expression includes a $\mu^2 \sig^2$ term, which indicates a massive sigma field.  Comparing with Eq.~\refeq{KG}, the pion mass is $\sqrt{2} \mu$.  However, there is no $\mu^2 \vpi^2$ term, which indicates that the pion field is massless.  The terms of $\order{\sqrt{\lam}}$ and $\order{\lam}$ have factors of $\vpi^4$, $\vpi^2 \sig$, $\vpi^2 \sig^2$, $\sig^3$, and $\sig^4$; these are all cubic and quartic factors.  Since they are of $\order{\sqrt{\lam}}$ and $\order{\lam}$, they become small as $\lam \to 0$.  This is what we wanted to show. \qed
	
	For the propagators, we can use (4.46) of Peskin \& Schroeder:
	\eq{
		\DF(x - y) = \int \ddqpf \frac{i e^{-i p \cdot (x - y)}}{p^2 - m^2 + i \eps}.
	}
	Then we can write
	\aln{ \label{props}
		\centergraphics{diag/dbl_arrow} &= \ans{ \int \ddqpf \frac{i e^{-i p \cdot (x - y)}}{p^2 - 2 \mu^2 + i \eps}, } &
		\centergraphics{diag/ij_arrow} &= \ans{ \delij \int \ddqpf \frac{i e^{-i p \cdot (x - y)}}{p^2 - 2 \mu^2 + i \eps}. }
	}
	We can associate each of the vertices with a term in $V(\Psi^2)$.  The symmetry factors for each of the terms are
	\al{
		\vpi^2 \mu \sig \sqrt{\lam} &: 2! = 2, &
		\mu \sig^3 \sqrt{\lam} &: 3! = 6, &
		\frac{(\vpi^2)^2 \lam}{4} &: 4! = 24, &
		\frac{\vpi^2 \sig^2 \lam}{2} &: 2! 2! = 4, &
		\frac{\sig^4 \lam}{4} &: 4! = 24.
	}
	Then the vertices are
	\aln{
		\centergraphics{diag/ij_triangle} &= \ans{ -2i \mu \sqrt{\lam} \delij, } &
		\centergraphics{diag/dbl_triangle} &= \ans{ -6i \mu \sqrt{\lam} }, \label{verts} \\
		\centergraphics{diag/ijkl_vertex} &= \ans{ -2 i \lam (\delij \delkl + \delil \deljk + \delik \deljl), } &
		\centergraphics{diag/ij_vertex} &= \ans{-2 i \lam \delij, } &
		\centergraphics{diag/dbl_vertex} &= \ans{ -6i \lam. } \notag
	}
	\vfix
}



\prob{
	Compute the scattering amplitude for the process
	\eq{
		\pii(\pq) \pij(\pw) \to \pik(\pe) \pil(\pr)
	}
	to leading order in $\lam$.  There are now four Feynman diagrams that contribute:
	\eq{
		\centergraphics{diag/s_chan} \qq{+} \centergraphics{diag/t_chan} \qq{+} \centergraphics{diag/u_chan} \qq{+} \centergraphics{diag/vertex}
	}
	Show that, at threshold ($\vpsi = 0$), these diagrams sum to \emph{zero}.  Show that, in the special case $N = 2$ (1 species of pion), the term $\order{p^2}$ also cancels.
}

\sol{
	Using the propagators and vertices of Eqs.~\refeq{props} and \refeq{verts}, the contributions from each diagram are
	\eq{
		\centergraphics{diag/ijkl_s_chan} = (-2i \mu \sqrt{\lam} \delij) \frac{i}{(\pq + \pw)^2 - 2 \mu^2} (-2i \mu \sqrt{\lam} \delkl)
		= -\frac{4 i \mu^2 \lam}{(\pq + \pw)^2 - 2 \mu^2} \delij \delkl,
	}
	\al{
		\centergraphics{diag/ijkl_t_chan} &= (-2i \mu \sqrt{\lam} \delik) \frac{i}{(\pq - \pe)^2 - 2 \mu^2} (-2i \mu \sqrt{\lam} \deljl)
		= -\frac{4 i \mu^2 \lam}{(\pq - \pe)^2 - 2 \mu^2} \delik \deljl, \\
		%
		\centergraphics{diag/ijkl_u_chan} &= (-2i \mu \sqrt{\lam} \delil) \frac{i}{(\pq - \pr)^2 - 2 \mu^2} (-2i \mu \sqrt{\lam} \deljk)
		= -\frac{4 i \mu^2 \lam}{(\pq - \pr)^2 - 2 \mu^2} \delil \deljk, \\
		%
		\centergraphics{diag/ijkl_vertex} &= -2 i \lam (\delij \delkl + \delil \deljk + \delik \deljl),
	}
	so the total amplitude is
	\al{
		i \cM &= -\frac{4 i \mu^2 \lam}{(\pq + \pw)^2 - 2 \mu^2} \delij \delkl - \frac{4 i \mu^2 \lam}{(\pq - \pe)^2 - 2 \mu^2} \delik \deljl - \frac{4 i \mu^2 \lam}{(\pq - \pr)^2 - 2 \mu^2} \delil \deljk - 2 i \lam (\delij \delkl + \delil \deljk + \delik \deljl) \\
		&= \ans{ -4 i \mu^2 \lam \paren{ \frac{\delij \delkl}{(\pq + \pw)^2 - 2 \mu^2} + \frac{\delik \deljl}{(\pq - \pe)^2 - 2 \mu^2} + \frac{\delil \deljk}{(\pq - \pr)^2 - 2 \mu^2} } - 2 i \lam (\delij \delkl + \delil \deljk + \delik \deljl), }
	}
	where we have referred to (4.119) of Peskin \& Schroeder.
	
	When $\vpsi = 0$, $p_i = 0$.  This is because $p_i = (\Ei, \vpsi)$ and $\mpi = 0$, so $\Ei = 0$ at zero momentum.  Then the amplitude is
	\al{
		i \cM &= 4 i \mu^2 \lam \paren{ \frac{\delij \delkl}{2 \mu^2} + \frac{\delik \deljl}{2 \mu^2} + \frac{\delil \deljk}{2 \mu^2} } - 2 i \lam (\delij \delkl + \delil \deljk + \delik \deljl) \\
		&= 2 i \lam (\delij \delkl + \delil \deljk + \delik \deljl) - 2 i \lam (\delij \delkl + \delil \deljk + \delik \deljl) \\
		&= \ans{ 0 }
	}
	as we wanted to show. \qed
	
	When there is only one species of pion, $i = j = k = l$.  So the amplitude is
	\eq{
		i \cM = -4 i \mu^2 \lam \paren{ \frac{1}{(\pq + \pw)^2 - 2 \mu^2} + \frac{1}{(\pq - \pe)^2 - 2 \mu^2} + \frac{1}{(\pq - \pr)^2 - 2 \mu^2} } - 6 i \lam.
	}
	Assuming that the pions have similar momenta, $(p_i \pm p-j)^2 \ll \mu^2$.  Let $x = (p_i \pm p-j)^2 / \mu^2$.  Then the first three terms of $i \cM$ each look like
	\eq{
		\frac{1}{2 \mu^2} \frac{1}{x - 1} = -\frac{1 + x}{2 \mu^2} + \order{x^2},
	}
	where we have performed a Taylor series expansion about 0 to first order in $x$~\cite{Maclaurin}.  Then the amplitude is
	\aln{
		i \cM &= -2 i \lam \paren{ -1 - \frac{(\pq + \pw)^2}{2 \mu^2} - 1 - \frac{(\pq - \pe)^2}{2 \mu^2} - 1 - \frac{(\pq - \pr)^2}{2 \mu^2} } - 6 i \lam + \order{p^4} \notag \\
		&= \frac{i \lam}{\mu^2} \brac{ (\pq + \pw)^2 + (\pq - \pe)^2 + (\pq - \pr)^2 } + \order{p^4}. \label{thing1c}
	}
	From Peskin \& Schroeder (5.69), we can write the Mandelstam variables
	\al{
		s &= (\pq + \pw)^2 = (\pe + \pr)^2, &
		t &= (\pe - \pq)^2 = (\pr - \pw)^2, &
		u &= (\pr - \pq)^2 = (\pe - \pw)^2.
	}
	For four particles of the same mass $m$~\cite[p.~159]{Peskin},
	\eq{
		s + t + u = 4 m^2.
	}
	The pion has mass $\mpi = 0$, so $s + t + u = 0$ for the pions.  Substituting the Mandelstam variables into Eq.~\refeq{thing1c}, we have
	\eq{
		i \cM = \frac{i \lam}{\mu^2} \brac{ s + t + u } + \order{p^4}
		= \ans{ \order{p^4}. }
	}
	So we have shown that the $\order{p^2}$ terms cancel, as desired. \qed
}


%\prob{
%	Add to $V$ a symmetry-breaking term,
%	\eq{
%		\Delta V = -a \Phi^N,
%	}
%	where $a$ is a (small) constant.  Find the new value of $v$ that minimizes $V$, and work out the content of the theory about that point.  Show that the pion acquires a mass such that $\mpi^2 \sim a$, and show that the pion scattering amplitude at threshold is now nonvanishing and also proportional to $a$.
%}







%\state{Rutherford scattering (Peskin \& Schroeder 4.4)}{
%	The cross section for scattering of an electron by the Coulomb field of a nucleus can be computed, to lowest order, without quantizing the electromagnetic field.  Instead, treat the field as a given, classical potential $\Asmx$.  The interaction Hamiltonian is
%	\eq{
%		\HI = \int \ddcx e \psib \gamm \psi \Asm,
%	}
%	where $\psix$ is the usual quantized Dirac field.
%}

%\prob{
%	Show that the $T$-matrix element for electron scattering off a localized classical potential is, to lowest order,
%	\eq{
%		\melppiTp = -i e \ubpp \gamm \up \cdot \tAsm(p' - p),
%	}
%	where $\tAsmq$ is the four-dimensional Fourier transform of $\Asmx$.
%}



%\prob{
%	If $\Asmx$ is time independent, its Fourier transform contains a delta function of energy.  It is then natural to define
%	\eq{
%		\melppiTp \equiv i \cM \cdot (2\pi) \del(\Ef - \Ei),
%	}
%	where $\Ei$ and $\Ef$ are the initial and final energies of the particle, and to adopt a new Feynman rule for computing $\cM$:
%	\eq{
%		\centergraphics{diag/boson} = -i e \gamm \tAsmvq,
%	}
%	where $\tAsmvq$ is the three-dimensional Fourier transform of $\Asmx$.  Given this definition of $\cM$, show that the cross section for scattering off a time-independent, localized potential is
%	\eq{
%		\ddsig = \frac{1}{\vi} \frac{1}{2 \Ei} \ddcpff \frac{1}{2 \Ef} \abs{\cM(\pin \to \pf)}^2 (2\pi) \del(\Ef - \Ei),
%	}
%	where $\vi$ is the particle's initial velocity.  This formula is a natural modification of (4.79).  Integrate over $\abs{\pf}$ to find a simple expression for $\dv*{\sig}{\Omg}$.
%}



%\prob{
%	Specialize to the case of electron scattering from a Coulomb potential ($\Ao = Z e / 4\pi r$).  Working in the nonrelativistic limit, derive the Rutherford formula,
%	\eq{
%		\dv{\sig}{\Omg} = \frac{\alp^2 Z^2}{4 m^2 v^4 \sin[4](\tht / 2)}.
%	}
%}


\makebib

\end{document}
