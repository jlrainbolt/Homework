\documentclass[11pt]{article}
\usepackage{homework}

\classname{443}
\homeworknum{4}



\begin{document}

% Environments

\newcommand{\state}[2]{\begin{statement}{#1} #2 \end{statement}}
\newcommand{\prob}[2]{\begin{problem}{#1} #2 \end{problem}}
\newcommand{\subprob}[1]{\begin{subproblem} #1 \end{subproblem}}
\newcommand{\sol}[1]{\begin{solution} #1 \end{solution}}
\newcommand{\fig}[2]{\begin{figure} \centering #2  \label{#1} \end{figure}}

\newcommand{\makebib}{
	\vfill
	\color{black}
	\nocite{*}
	\bibliography{references}{}
	\bibliographystyle{lucas_unsrt}
}
	

% Implication

\newcommand{\qwhere}{\quad \text{where} \quad}
\newcommand{\qimplies}{\quad \implies \quad}
\newcommand{\impliesq}{\implies \quad}



% Brackets

\newcommand{\paren}[1]{\left( #1 \right)}
\newcommand{\brac}[1]{\left[ #1 \right]}
\newcommand{\curly}[1]{\left\{ #1 \right\}}


% Greek

\newcommand{\alp}{\alpha}
\newcommand{\bet}{\beta}
\newcommand{\gam}{\gamma}
\newcommand{\del}{\delta}
\newcommand{\eps}{\epsilon}
\newcommand{\zet}{\zeta}
\newcommand{\tht}{\theta}
\newcommand{\kap}{\kappa}
\newcommand{\lam}{\lambda}
\newcommand{\sig}{\sigma}
\newcommand{\ups}{\upsilon}
\newcommand{\omg}{\omega}

\newcommand{\Gam}{\Gamma}
\newcommand{\Del}{\Delta}
\newcommand{\Tht}{\Theta}
\newcommand{\Lam}{\Lambda}
\newcommand{\Sig}{\Sigma}
\newcommand{\Omg}{\Omega}


% Text

\newcommand{\where}{\text{where }}

% Problem 1

\newcommand{\Hint}{H_\text{int}}
\newcommand{\ddcx}{\dd[3]{x}}
\newcommand{\psib}{\bar{\psi}}

\newcommand{\mh}{m_h}
\newcommand{\mmu}{m_\mu}
\newcommand{\me}{m_e}
\newcommand{\ma}{m_a}

\newcommand{\aexpt}{a_\text{expt.}}
\newcommand{\aQED}{a_\text{QED}}
\renewcommand{\GeV}{\giga\electronvolt}

\newcommand{\gamt}{\gam^5}



\state{Linear sigma model (Peskin \& Schroeder 4.3)}{
	The interactions of pions at low energy can be described by a phenomenological model called the \emph{linear sigma model}.  Essentially, this model consists of $N$ real scalar fields coupled by a $\phi^4$ interaction that is symmetric under rotations of the $N$ fields.  More specifically, let $\Psiix$, $i = 1, \ldots, N$ be a set of $N$ fields, governed by the Hamiltonian
	\eq{
		H = \int \ddcx \paren{ \frac{1}{2} (\Pii)^2 + \frac{1}{2} (\grad \Phii)^2 + V(\Phi^2) },
	}
	where $(\Phii)^2 = \Phi \cdot \Phi$, and
	\eq{
		V(\Phi^2) = \frac{1}{2} m^2 (\Phii)^2 + \frac{\lam}{4} \paren{ (\Phii)^2 }^2
	}
	is a function symmetric under rotations of $\Phi$.  For (classical) field configurations of $\Phiix$ that are constant in space and time, this term gives the only contribution to $H$; hence, $V$ is the field potential energy.
}

\prob{
	Analyze the linear sigma model for $m^2 > 0$ by noticing that, for $\lam = 0$, the Hamiltonian given above is exactly $N$ copies of the Klein-Gordon Hamiltonian.  We can then calculate scattering amplitudes as perturbation series in the parameter $\lam$.  Show that the propagator is
	\eq{
		\wick{\c{\Phi}^i(x) \c{\Phi}^j(y)} = \delij \DF(x - y),
	}
	where $\DF$ is the standard Klein-Gordon propagator for mass $m$, and that there is one type of vertex given by
	\eq{
		\centergraphics{diag/vertex} = -2 i \lam(\delij \delkl + \delil \deljk + \delik \deljl).
	}
	Compute, to leading order in $\lam$, the differential cross sections $\dv*{\sig}{\Omg}$, in the center-of-mass frame, for the scattering processes
	\al{
		\Phiq \Phiw &\to \Phiq \Phiw, &
		\Phiq \Phiq &\to \Phiw \Phiw, &
		\Phiq \Phiq &\to \Phiq \Phiq
	}
	as functions of the center-of-mass energy.
}



\prob{
	Now consider the case $m^2 < 0$: $m^2 = -\mu^2$.  In this case, $V$ has a local maximum, rather than a minimum, at $\Phii = 0$.  Since $V$ is a potential energy, this implies that the ground state of the theory is not near $\Phii = 0$ but rather is obtained by shifting $\Phii$ toward the minimum of $V$.  By rotational invariance, we can consider this shift to be in the $N$th direction.  Write, then,
	\al{
		\Phiix &= \piix, \qquad i = 1, \ldots, N - 1, &
		\PhiNx &= v + \sigx
	}
	where $v$ is a constant chosen to minimize $V$.  (The notation $\pii$ suggests a pion field and should not be confused with a canonical momentum.)  Show that, in these new coordinates (and substituting for $v$ its expression in terms of $\lam$ and $\mu$), we have a theory of a massive $\sig$ field and $N - 1$ \emph{massless} pion fields, interacting through cubic and quartic potential energy terms which all become small as $\lam \to 0$.  Construct the Feynman rules by assigning values to the propagators and vertices:
	\al{
		\wick{\c{\sig} \c{\sig}} &= \centergraphics{diag/dbl_arrow}, &\qquad
		& \qquad \qquad \centergraphics{diag/ij_triangle} \qquad \centergraphics{diag/dbl_triangle} \\
		\wick{\c{\pi}^i \c{\pi}^j} &= \centergraphics{diag/ij_arrow}, &\qquad
		& \centergraphics{diag/ijkl_vertex} \qquad \centergraphics{diag/ij_vertex} \qquad \centergraphics{diag/dbl_vertex}
	}
	\vfix
}


\prob{
	Compute the scattering amplitude for the process
	\eq{
		\pii(\pq) \pij(\pw) \to \pik(\pe) \pil(\pr)
	}
	to leading order in $\lam$.  There are now four Feynman diagrams that contribute:
	\eq{
		\centergraphics{diag/t_chan} \qq{+} \centergraphics{diag/s_chan} \qq{+} \centergraphics{diag/cross_chan} \qq{+} \centergraphics{diag/vertex}
	}
	Show that, at threshold ($\vpsi = 0$), these diagrams sum to \emph{zero}.  Show that, in the special case $N = 2$ (1 species of pion), the term $\order{p^2}$ also cancels.
}


\prob{
	Add to $V$ a symmetry-breaking term,
	\eq{
		\Delta V = -a \Phi^N,
	}
	where $a$ is a (small) constant.  Find the new value of $v$ that minimizes $V$, and work out the content of the theory about that point.  Show that the pion acquires a mass such that $\mpi^2 \sim a$, and show that the pion scattering amplitude at threshold is now nonvanishing and also proportional to $a$.
}







\state{Rutherford scattering (Peskin \& Schroeder 4.4)}{
	The cross section for scattering of an electron by the Coulomb field of a nucleus can be computed, to lowest order, without quantizing the electromagnetic field.  Instead, treat the field as a given, classical potential $\Asmx$.  The interaction Hamiltonian is
	\eq{
		\HI = \int \ddcx e \psib \gamm \psi \Asm,
	}
	where $\psix$ is the usual quantized Dirac field.
}

\prob{
	Show that the $T$-matrix element for electron scattering off a localized classical potential is, to lowest order,
	\eq{
		\melppiTp = -i e \ubpp \gamm \up \cdot \tAsm(p' - p),
	}
	where $\tAsmq$ is the four-dimensional Fourier transform of $\Asmx$.
}



\prob{
	If $\Asmx$ is time independent, its Fourier transform contains a delta function of energy.  It is then natural to define
	\eq{
		\melppiTp \equiv i \cM \cdot (2\pi) \del(\Ef - \Ei),
	}
	where $\Ei$ and $\Ef$ are the initial and final energies of the particle, and to adopt a new Feynman rule for computing $\cM$:
	\eq{
		\centergraphics{diag/boson} = -i e \gamm \tAsmvq,
	}
	where $\tAsmvq$ is the three-dimensional Fourier transform of $\Asmx$.  Given this definition of $\cM$, show that the cross section for scattering off a time-independent, localized potential is
	\eq{
		\ddsig = \frac{1}{\vi} \frac{1}{2 \Ei} \ddcpff \frac{1}{2 \Ef} \abs{\cM(\pin \to \pf)}^2 (2\pi) \del(\Ef - \Ei),
	}
	where $\vi$ is the particle's initial velocity.  This formula is a natural modification of (4.79).  Integrate over $\abs{\pf}$ to find a simple expression for $\dv*{\sig}{\Omg}$.
}



\prob{
	Specialize to the case of electron scattering from a Coulomb potential ($\Ao = Z e / 4\pi r$).  Working in the nonrelativistic limit, derive the Rutherford formula,
	\eq{
		\dv{\sig}{\Omg} = \frac{\alp^2 Z^2}{4 m^2 v^4 \sin[4](\tht / 2)}.
	}
}


%\makebib

\end{document}
