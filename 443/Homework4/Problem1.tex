\state{Linear sigma model (Peskin \& Schroeder 4.3)}{
	The interactions of pions at low energy can be described by a phenomenological model called the \emph{linear sigma model}.  Essentially, this model consists of $N$ real scalar fields coupled by a $\phi^4$ interaction that is symmetric under rotations of the $N$ fields.  More specifically, let $\Psiix$, $i = 1, \ldots, N$ be a set of $N$ fields, governed by the Hamiltonian
	\eq{
		H = \int \ddcx \paren{ \frac{1}{2} (\Pii)^2 + \frac{1}{2} (\grad \Phii)^2 + V(\Phi^2) },
	}
	where $(\Phii)^2 = \vPhi \vdot \vPhi$, and
	\eqn{V1}{
		V(\Phi^2) = \frac{1}{2} m^2 (\Phii)^2 + \frac{\lam}{4} \paren{ (\Phii)^2 }^2
	}
	is a function symmetric under rotations of $\vPhi$.  For (classical) field configurations of $\Phiix$ that are constant in space and time, this term gives the only contribution to $H$; hence, $V$ is the field potential energy.
}

\prob{
	Analyze the linear sigma model for $m^2 > 0$ by noticing that, for $\lam = 0$, the Hamiltonian given above is exactly $N$ copies of the Klein-Gordon Hamiltonian.  We can then calculate scattering amplitudes as perturbation series in the parameter $\lam$.  Show that the propagator is
	\eq{
		\wick{\c{\Phi}^i(x) \c{\Phi}^j(y)} = \delij \DF(x - y),
	}
	where $\DF$ is the standard Klein-Gordon propagator for mass $m$, and that there is one type of vertex given by
	\eqn{diag1a}{
		\centergraphics{diag/ijkl_vertex} = -2 i \lam (\delij \delkl + \delil \deljk + \delik \deljl).
	}
	Compute, to leading order in $\lam$, the differential cross sections $\dv*{\sig}{\Omg}$, in the center-of-mass frame, for the scattering processes
	\al{
		\Phiq \Phiw &\to \Phiq \Phiw, &
		\Phiq \Phiq &\to \Phiw \Phiw, &
		\Phiq \Phiq &\to \Phiq \Phiq
	}
	as functions of the center-of-mass energy.
}

\sol{
	The Klein-Gordon Hamiltonian is given by Peskin \& Schroeder~(2.8),
	\eqn{KG}{
		H = \int \ddcx \paren{ \frac{1}{2} \pi^2 + \frac{1}{2} (\grad \phi)^2 + \frac{1}{2} m^2 \phi^2 }
	}
	For $\lam = 0$, the linear sigma model has the Hamiltonian
	\eq{
		H = \int \ddcx \paren{ \frac{1}{2} (\Pii)^2 + \frac{1}{2} (\grad \Phii)^2 + \frac{1}{2} m^2 (\Phii)^2 }
	}
	which is clearly $N$ copies of the Klein-Gordon Hamiltonian, one for each $i$.
	
	From (4.36) we know that the Feynman propagator is the contraction of two fields:
	\eq{
		\wick{\c\phi(x) \c\phi(y)} = \DF(x - y).
	}
	No terms $\Phii \Phij$ for $i \neq j$ appear in the Hamiltonian, so fields with $i \neq j$ cannot be contracted.  Moreover, each field is governed by its own independent Klein-Gordon Hamiltonian to zeroth order.  So the propagator must be
	\eq{
		\ans{ \wick{\c{\Phi}^i(x) \c{\Phi}^j(y)} = \delij \DF(x - y) }
	}
	where $\DF(x - y)$ is the Klein-Gordon propagator. \qed
	
	In order to determine the Feynman rules, we use Peskin \& Schroeder~(4.90),
	\eq{
		\mel*{\vpsq \cdots \vpsn}{i T}{\vpsA \vpsB} = \lim_{T \to \infty (1 - i \eps)} \sOi \mel*{\vpsq \cdots \vpsn}{T \curly{ \exp(-i \intmTT \ddt \HIt)}}{\vpsA \vpsB} \sOf.
	}
	Our interaction Hamiltonian is
	\eq{
		\HI = \int \ddcx \frac{\lam}{4} \paren{ (\Phii)^2 }^2
		= \int \ddcx \frac{\lam}{4} (\vPhi \vdot \vPhi)^2
		= \frac{\lam}{4} \int \ddcx \paren{ \sumi (\Phii)^4 + 2 \suminj (\Phii)^2 (\Phij)^2 },
	}
	W have two final momenta, $\vpsq$ and $\vpsw$.  Now we have
	\eq{
		\mel*{\vpsq \vpsw}{i T}{\vpsA \vpsB} = \lim_{T \to \infty (1 - i \eps)} \sOi \mel*{\vpsq \vpsw}{T \curly{ \exp[ -i \int \ddqx \frac{\lam}{4} \paren{  \sumi (\Phii)^4 + 2 \suminj (\Phii)^2 (\Phij)^2 } ] }}{\vpsA \vpsB} \sOf.
	}
	The first term that contributes to leading order is, by analogy to (4.92),
	\al{
		&\sOi \mel*{\vpsq \vpsw}{T \curly{ -i \int \ddqx \frac{\lam}{4} \paren{ \sumi (\Phii)^4 + 2 \suminj (\Phii)^2 (\Phij)^2 } }}{\vpsA \vpsB} \sOf \\
		&\hspace{5em} = \sOi \mel*{\vpsq \vpsw}{N \curly{ -i \int \ddqx \frac{\lam}{4} \paren{ \sumi (\Phii)^4 + 2 \suminj (\Phii)^2 (\Phij)^2 } + \contractions }}{\vpsA \vpsB} \sOf,
	}
	but only the terms in which none of the fields are contracted with each other will contribute~\cite[p.~111]{Peskin}.
	
	The first term represents the process $\Phii \Phii \to \Phii \Phii$:
	\eq{
		\sOi \mel*{\vpsq \vpsw}{N \curly{ -i \int \ddqx \frac{\lam}{4} \sumi \Phii \Phii \Phii \Phii + \contractions }}{\vpsA \vpsB} \sOf.
	}
	The fields are all the same, so there are $4!$ ways of contracting the fields with the momenta, and we will obtain a diagram similar to (4.98).  Adapting that expression, we find
	\eq{
		-4! i \int \frac{\lam}{4} \ddqx e^{-i (\psA + \psB - \psq - \psw) \cdot x}
		= -6 i \lam (4 \pi)^4 \del^4(\psA + \psB - \psq - \psw).
	}
	The diagram in Eq.~\refeq{diag1a} is $\Phii \Phij \to \Phik \Phil$.  Since $i = j = k = l$ for this term, we have
	\eq{
		\centergraphics{diag/ijkl_vertex} = -6 i \lam
		= -2 i \lam (1 + 1 + 1)
		= \ans{ -2 i \lam (\delij \delkl + \delil \deljk + \delik \deljl). }
	}
	The second term can represent the processes $\Phii \Phii \to \Phij \Phij$ or $\Phii \Phij \to \Phii \Phij$ (where the indices and the order of the fields on either side is interchangeable):
	\eq{
		\sOi \mel*{\vpsq \vpsw}{N \curly{ -i \int \ddqx \frac{\lam}{4} 2 \suminj \Phii \Phii \Phij \Phij + \contractions }}{\vpsA \vpsB} \sOf.
	}
	Now there are only $2! \times 2! = 4$ ways to contract the fields with the momenta.  We have
	\eq{
		-4 i \int \frac{\lam}{2} \ddqx e^{-i (\psA + \psB - \psq - \psw) \cdot x}
		= -2 i \lam (4 \pi)^4 \del^4(\psA + \psB - \psq - \psw).
	}
	Here, either $i = j$ and $k = l$, $i = l$ and $j = k$, or $i = k$ and $j = l$.  We have
	\eq{
		\centergraphics{diag/ijkl_vertex} = -2 i \lam
		= -2 i \lam (1 + 0 + 0)
		= \ans{ -2 i \lam (\delij \delkl + \delil \deljk + \delik \deljl). }
	}
	Both of the terms can therefore be represented by Eq.~\refeq{diag1a} as we wanted to show. \qed
	
	When all four of the particles in the interaction have the same mass, the differential cross section in the center-of-mass frame is given by Peskin \& Schroeder~(4.85)
	\eq{
		\paren{ \dv{\sig}{\Omg} }_\CM = \frac{\abs{\cM}^2}{64 \pi^2 \Ecm^2},
	}
	where $\Ecm$ is the center-of-mass energy and $\cM$ is the invariant matrix element.  We know that the diagrams we calculated before have the form $i \cM (2\pi)^4 \del^4(\psA + \psB - \psq - \psw)$~\cite[p.~112]{Peskin}.  Then
	\al{
		\cM = -6 \lam &\qq{for} \Phiq \Phiq \to \Phiq \Phiq, &
		\cM = -2 \lam &\qq{for} \Phiq \Phiw \to \Phiq \Phiw \text{ and } \Phiq \Phiq \to \Phiw \Phiw.
	}
	So the differential cross sections are, to leading order in $\lam$,
	\al{
		(\Phiq \Phiw \to \Phiq \Phiw) &\quad
		\paren{ \dv{\sig}{\Omg} }_\CM = \frac{\abs{-2 \lam}^2}{64 \pi^2 \Ecm^2}
		= \ans{ \frac{\lam^2}{16 \pi^2 \Ecm^2}, } \\
		(\Phiq \Phiw \to \Phiq \Phiw) &\quad
		\paren{ \dv{\sig}{\Omg} }_\CM = \frac{\abs{-6 \lam}^2}{64 \pi^2 \Ecm^2}
		= \ans{ \frac{9 \lam^2}{16 \pi^2 \Ecm^2}, } \\
		(\Phiq \Phiq \to \Phiw \Phiw) &\quad
		\paren{ \dv{\sig}{\Omg} }_\CM = \frac{\abs{-6 \lam}^2}{64 \pi^2 \Ecm^2}
		= \ans{ \frac{9 \lam^2}{16 \pi^2 \Ecm^2}. }
	}
	\vfix
}



\prob{ \label{1b}
	Now consider the case $m^2 < 0$: $m^2 = -\mu^2$.  In this case, $V$ has a local maximum, rather than a minimum, at $\Phii = 0$.  Since $V$ is a potential energy, this implies that the ground state of the theory is not near $\Phii = 0$ but rather is obtained by shifting $\Phii$ toward the minimum of $V$.  By rotational invariance, we can consider this shift to be in the $N$th direction.  Write, then,
	\al{
		\Phiix &= \piix, \qquad i = 1, \ldots, N - 1, &
		\PhiNx &= v + \sigx
	}
	where $v$ is a constant chosen to minimize $V$.  (The notation $\pii$ suggests a pion field and should not be confused with a canonical momentum.)  Show that, in these new coordinates (and substituting for $v$ its expression in terms of $\lam$ and $\mu$), we have a theory of a massive $\sig$ field and $N - 1$ \emph{massless} pion fields, interacting through cubic and quartic potential energy terms which all become small as $\lam \to 0$.  Construct the Feynman rules by assigning values to the propagators and vertices:
	\al{
		\wick{\c{\sig} \c{\sig}} &= \centergraphics{diag/dbl_arrow} &\qquad
		& \qquad \qquad \centergraphics{diag/ij_triangle} \qquad \centergraphics{diag/dbl_triangle} \\
		\wick{\c{\pi}^i \c{\pi}^j} &= \centergraphics{diag/ij_arrow} &\qquad
		& \centergraphics{diag/ijkl_vertex} \qquad \centergraphics{diag/ij_vertex} \qquad \centergraphics{diag/dbl_vertex}
	}
	\vfix
}

\sol{
	With the negative mass, Eq.~\refeq{V1} becomes
	\eqn{V1b}{
		V(\Phi^2) = -\frac{1}{2} \mu^2 (\Phii)^2 + \frac{\lam}{4} \paren{ (\Phii)^2 }^2.
	}
	To find the minimum of $V$, we differentiate with respect to $\PhiN$.  We stipulate
	\eq{
		0 = \pdv{V}{\PhiN} = -\mu^2 \PhiN + \lam (\vPhi \vdot \vPhi) \PhiN
		= -\mu^2 \PhiN + \lam (\PhiN)^3,
	}
	where we have used the chain rule to evaluate the second term, and the fact that $V$ is minimal for all $\Phii = 0$ with $i \neq N$.  This implies $\PhiN = 0$ or
	\eqn{Phib}{
		(\PhiN)^2 = \frac{\mu^2}{\lam}
	}
	when $V$ is minimal.  Thus
	\eq{
		v = \frac{\mu}{\sqrt{\lam}}.
	}
	In order to determine the form of the theory, we need to rewrite $V(\Phi^2)$ in the new coordinates.  Note that $\vPhi = (\vpi, v + \sig)$.  Then
	\al{
		V(\Phi^2) &= -\frac{1}{2} \mu^2 \brac{ \vpi^2 + (v + \sig)^2 } + \frac{\lam}{4} \brac{ \vpi^2 + (v + \sig)^2 }^2 \\[1ex]
		%
		&= -\frac{1}{2} \mu^2 \paren{ \vpi^2 + \frac{\mu^2}{\lam} + 2 \frac{\mu \sig}{\sqrt{\lam}} + \sig^2 } + \frac{\lam}{4} \paren{ \vpi^2 + \frac{\mu^2}{\lam} + 2 \frac{\mu \sig}{\sqrt{\lam}} + \sig^2 }^2 \\[1ex]
		%
		&= -\frac{1}{2} \mu^2 \paren{ \vpi^2 + \frac{\mu^2}{\lam} + 2 \frac{\mu \sig}{\sqrt{\lam}} + \sig^2 } \\
		&\hspace{5em} \phantom{=\ } + \frac{\lam}{4} \paren{ (\vpi^2)^2 + 2 \frac{\vpi^2 \mu^2}{\lam} + \frac{\mu^4}{\lam^2} + 4 \frac{\vpi^2 \mu \sig}{\sqrt{\lam}} + 4 \frac{\mu^3 \sig}{\lam^{3/2}} + 2 \vpi^2 \sig^2 + 6 \frac{\mu^2 \sig^2}{\lam} + 4 \frac{\mu \sig^3}{\sqrt{\lam}} + \sig^4 } \\[1ex]
		%
		&= -\frac{\vpi^2 \mu^2}{2} - \frac{\mu^4}{2 \lam} - \frac{\mu^3 \sig}{\sqrt{\lam}} - \frac{\mu^2 \sig^2}{2} + \frac{(\vpi^2)^2 \lam}{4} + \frac{\vpi^2 \mu^2}{2} + \frac{\mu^4}{4 \lam} \\
		&\hspace{5em} \phantom{=\ } + \vpi^2 \mu \sig \sqrt{\lam} + \frac{\mu^3 \sig}{\sqrt{\lam}} + \frac{\vpi^2 \sig^2 \lam}{2} + \frac{3 \mu^2 \sig^2}{2} + \mu \sig^3 \sqrt{\lam} + \frac{\sig^4 \lam}{4} \\[1ex]
		%
		&= \ans{ \frac{(\vpi^2)^2 \lam}{4} + \vpi^2 \mu \sig \sqrt{\lam} + \frac{\vpi^2 \sig^2 \lam}{2} + \mu^2 \sig^2 + \mu \sig^3 \sqrt{\lam} + \frac{\sig^4 \lam}{4}, }
	}
	where we have dropped the constant term.  This expression includes a $\mu^2 \sig^2$ term, which indicates a massive sigma field.  Comparing with Eq.~\refeq{KG}, the pion mass is $\sqrt{2} \mu$.  However, there is no $\mu^2 \vpi^2$ term, which indicates that the pion field is massless.  The terms of $\order{\sqrt{\lam}}$ and $\order{\lam}$ have factors of $\vpi^4$, $\vpi^2 \sig$, $\vpi^2 \sig^2$, $\sig^3$, and $\sig^4$; these are all cubic and quartic factors.  Since they are of $\order{\sqrt{\lam}}$ and $\order{\lam}$, they become small as $\lam \to 0$.  This is what we wanted to show. \qed
	
	For the propagators, we can use (4.46) of Peskin \& Schroeder:
	\eq{
		\DF(x - y) = \int \ddqpf \frac{i e^{-i p \cdot (x - y)}}{p^2 - m^2 + i \eps}.
	}
	Then we can write
	\aln{ \label{props}
		\centergraphics{diag/dbl_arrow} &= \ans{ \int \ddqpf \frac{i e^{-i p \cdot (x - y)}}{p^2 - 2 \mu^2 + i \eps}, } &
		\centergraphics{diag/ij_arrow} &= \ans{ \delij \int \ddqpf \frac{i e^{-i p \cdot (x - y)}}{p^2 + i \eps}. }
	}
	We can associate each of the vertices with a term in $V(\Psi^2)$.  The symmetry factors for each of the terms are
	\al{
		\vpi^2 \mu \sig \sqrt{\lam} &: 2! = 2, &
		\mu \sig^3 \sqrt{\lam} &: 3! = 6, &
		\frac{(\vpi^2)^2 \lam}{4} &: 4! = 24, &
		\frac{\vpi^2 \sig^2 \lam}{2} &: 2! 2! = 4, &
		\frac{\sig^4 \lam}{4} &: 4! = 24.
	}
	Then the vertices are
	\aln{
		\centergraphics{diag/ij_triangle} &= \ans{ -2i \mu \sqrt{\lam} \delij, } &
		\centergraphics{diag/dbl_triangle} &= \ans{ -6i \mu \sqrt{\lam} }, \label{verts} \\
		\centergraphics{diag/ijkl_vertex} &= \ans{ -2 i \lam (\delij \delkl + \delil \deljk + \delik \deljl), } &
		\centergraphics{diag/ij_vertex} &= \ans{-2 i \lam \delij, } &
		\centergraphics{diag/dbl_vertex} &= \ans{ -6i \lam. } \notag
	}
	\vfix
}



\prob{
	Compute the scattering amplitude for the process
	\eq{
		\pii(\pq) \pij(\pw) \to \pik(\pe) \pil(\pr)
	}
	to leading order in $\lam$.  There are now four Feynman diagrams that contribute:
	\eq{
		\centergraphics{diag/s_chan} \qq{+} \centergraphics{diag/t_chan} \qq{+} \centergraphics{diag/u_chan} \qq{+} \centergraphics{diag/vertex}
	}
	Show that, at threshold ($\vpsi = 0$), these diagrams sum to \emph{zero}.  Show that, in the special case $N = 2$ (1 species of pion), the term $\order{p^2}$ also cancels.
}

\sol{
	Using the propagators and vertices of Eqs.~\refeq{props} and \refeq{verts}, the contributions from each diagram are
	\eq{
		\centergraphics{diag/ijkl_s_chan} = (-2i \mu \sqrt{\lam} \delij) \frac{i}{(\pq + \pw)^2 - 2 \mu^2} (-2i \mu \sqrt{\lam} \delkl)
		= -\frac{4 i \mu^2 \lam}{(\pq + \pw)^2 - 2 \mu^2} \delij \delkl,
	}
	\al{
		\centergraphics{diag/ijkl_t_chan} &= (-2i \mu \sqrt{\lam} \delik) \frac{i}{(\pq - \pe)^2 - 2 \mu^2} (-2i \mu \sqrt{\lam} \deljl)
		= -\frac{4 i \mu^2 \lam}{(\pq - \pe)^2 - 2 \mu^2} \delik \deljl, \\
		%
		\centergraphics{diag/ijkl_u_chan} &= (-2i \mu \sqrt{\lam} \delil) \frac{i}{(\pq - \pr)^2 - 2 \mu^2} (-2i \mu \sqrt{\lam} \deljk)
		= -\frac{4 i \mu^2 \lam}{(\pq - \pr)^2 - 2 \mu^2} \delil \deljk, \\
		%
		\centergraphics{diag/ijkl_vertex} &= -2 i \lam (\delij \delkl + \delil \deljk + \delik \deljl),
	}
	so the total amplitude is
	\al{
		i \cM &= -\frac{4 i \mu^2 \lam}{(\pq + \pw)^2 - 2 \mu^2} \delij \delkl - \frac{4 i \mu^2 \lam}{(\pq - \pe)^2 - 2 \mu^2} \delik \deljl - \frac{4 i \mu^2 \lam}{(\pq - \pr)^2 - 2 \mu^2} \delil \deljk - 2 i \lam (\delij \delkl + \delil \deljk + \delik \deljl) \\
		&= \ans{ -4 i \mu^2 \lam \paren{ \frac{\delij \delkl}{(\pq + \pw)^2 - 2 \mu^2} + \frac{\delik \deljl}{(\pq - \pe)^2 - 2 \mu^2} + \frac{\delil \deljk}{(\pq - \pr)^2 - 2 \mu^2} } - 2 i \lam (\delij \delkl + \delil \deljk + \delik \deljl), }
	}
	where we have referred to (4.119) of Peskin \& Schroeder.
	
	When $\vpsi = 0$, $p_i = 0$.  This is because $p_i = (\Ei, \vpsi)$ and $\mpi = 0$, so $\Ei = 0$ at zero momentum.  Then the amplitude is
	\al{
		i \cM &= 4 i \mu^2 \lam \paren{ \frac{\delij \delkl}{2 \mu^2} + \frac{\delik \deljl}{2 \mu^2} + \frac{\delil \deljk}{2 \mu^2} } - 2 i \lam (\delij \delkl + \delil \deljk + \delik \deljl) \\
		&= 2 i \lam (\delij \delkl + \delil \deljk + \delik \deljl) - 2 i \lam (\delij \delkl + \delil \deljk + \delik \deljl) \\
		&= \ans{ 0 }
	}
	as we wanted to show. \qed
	
	When there is only one species of pion, $i = j = k = l$.  So the amplitude is
	\eq{
		i \cM = -4 i \mu^2 \lam \paren{ \frac{1}{(\pq + \pw)^2 - 2 \mu^2} + \frac{1}{(\pq - \pe)^2 - 2 \mu^2} + \frac{1}{(\pq - \pr)^2 - 2 \mu^2} } - 6 i \lam.
	}
	Assuming that the pions have similar momenta, $(p_i \pm p-j)^2 \ll \mu^2$.  Let $x = (p_i \pm p-j)^2 / \mu^2$.  Then the first three terms of $i \cM$ each look like
	\eq{
		\frac{1}{2 \mu^2} \frac{1}{x - 1} = -\frac{1 + x}{2 \mu^2} + \order{x^2},
	}
	where we have performed a Taylor series expansion about 0 to first order in $x$~\cite{Maclaurin}.  Then the amplitude is
	\aln{
		i \cM &= -2 i \lam \paren{ -1 - \frac{(\pq + \pw)^2}{2 \mu^2} - 1 - \frac{(\pq - \pe)^2}{2 \mu^2} - 1 - \frac{(\pq - \pr)^2}{2 \mu^2} } - 6 i \lam + \order{p^4} \notag \\
		&= \frac{i \lam}{\mu^2} \brac{ (\pq + \pw)^2 + (\pq - \pe)^2 + (\pq - \pr)^2 } + \order{p^4}. \label{thing1c}
	}
	From Peskin \& Schroeder (5.69), we can write the Mandelstam variables
	\al{
		s &= (\pq + \pw)^2 = (\pe + \pr)^2, &
		t &= (\pe - \pq)^2 = (\pr - \pw)^2, &
		u &= (\pr - \pq)^2 = (\pe - \pw)^2.
	}
	For four particles of the same mass $m$~\cite[p.~159]{Peskin},
	\eq{
		s + t + u = 4 m^2.
	}
	The pion has mass $\mpi = 0$, so $s + t + u = 0$ for the pions.  Substituting the Mandelstam variables into Eq.~\refeq{thing1c}, we have
	\eq{
		i \cM = \frac{i \lam}{\mu^2} \brac{ s + t + u } + \order{p^4}
		= \ans{ \order{p^4}. }
	}
	So we have shown that the $\order{p^2}$ terms cancel, as desired. \qed
}


\prob{
	Add to $V$ a symmetry-breaking term,
	\eq{
		\Delta V = -a \Phi^N,
	}
	where $a$ is a (small) constant.  Find the new value of $v$ that minimizes $V$, and work out the content of the theory about that point.  Show that the pion acquires a mass such that $\mpi^2 \sim a$, and show that the pion scattering amplitude at threshold is now nonvanishing and also proportional to $a$.
}

\sol{
	With this term, Eq.~\refeq{V1b} becomes
	\eq{
		V(\Phi^2) = -\frac{1}{2} \mu^2 (\Phii)^2 + \frac{\lam}{4} \paren{ (\Phii)^2 }^2 - a \Phi^N.
	}
	Differentiating with respect to $\PhiN$ as in \ref{1b} to find the minimum, we stipulate that
	\eqn{thing1d}{
		0 = \pdv{V}{\PhiN} = -\mu^2 \PhiN + \lam (\vPhi \vdot \vPhi) \PhiN - a
		= -\mu^2 \PhiN + \lam (\PhiN)^3 - a.
	}
	We know $\PhiN$ is a function of $a$.  Let $\PhiN = F(a)$.  Since $a$ is small, we can Taylor expand to first order about $a = 0$:
	\eq{
		\PhiN \approx F(0) + a \left. \pdv{F(a)}{a} \right|_{a=0}
		= \sqrt{\frac{\mu^2}{\lam}} + a F'(0),
	}
	where we have used Eq.~\refeq{Phib} as $F(0)$.  Then \refeq{thing1d} becomes
	\al{
		0 &= -\mu^2 \paren{ \sqrt{\frac{\mu^2}{\lam}} + a F'(0) } + \lam \paren{ \sqrt{\frac{\mu^2}{\lam}} + a F'(0) }^3 - a \\
		&= -\mu^2 \sqrt{\frac{\mu^2}{\lam}} -\mu^2 a F'(0) + \lam a^3 [F'(0)]^2 + 3 a \mu^2 F'(0) + 2 a^2 [F'(0)]^2 \sqrt{\frac{\mu^2}{\lam}} + \mu^2 \sqrt{\frac{\mu^2}{\lam}} - a \\
		&\approx 2 a \mu^2 F'(0) - a,
	}
	where we have dropped terms of $\order{a^2}$.  This implies $F'(0) = 0$ or
	\eq{
		F'(0) = \frac{1}{2 \mu^2},
	}
	when $V$ is mininal, and thus
	\eq{
		\ans{ v = \sqrt{\frac{\mu^2}{\lam}} + \frac{a}{2 \mu^2}. }
	}
	In the new coordinate, $V(\Phi^2)$ can be written % \hl{(SIMPLIFY RADICALS)}
	\al{
		V(\Phi^2) &= -\frac{1}{2} \mu^2 \brac{ \vpi^2 + (v + \sig)^2 } + \frac{\lam}{4} \brac{ \vpi^2 + (v + \sig)^2 }^2 - a (v + \sig) \\[1ex]
		&= -\frac{1}{2} \mu^2 \brac{ \vpi^2 +\paren{ \sqrt{\frac{\mu^2}{\lam}} + \frac{a}{2 \mu^2} + \sig }^2 } + \frac{\lam}{4} \brac{ \vpi^2 + \paren{ \sqrt{\frac{\mu^2}{\lam}} + \frac{a}{2 \mu^2} + \sig }^2 }^2 - a \paren{ \sqrt{\frac{\mu^2}{\lam}} + \frac{a}{2 \mu^2} + \sig } \\[1ex]
		&= \frac{3 a^2 \lam \sig^2}{8 \mu^4} + \frac{a^3 \lam \sig}{8 \mu^6} + \frac{3 a^2 \lam \sig}{4 \mu^3} \sqrt{\frac{\mu^2}{\lam}} + \frac{a^4 \lam}{64 \mu^8} + \frac{a^3 \lam}{8 \mu^6} \sqrt{\frac{\mu^2}{\lam}} + \frac{a^2 \lam \vpi^2}{8 \mu^4} - \frac{a^2}{4 \mu^2} + \frac{a \lam \sig^3}{2 \mu^2} + \frac{3 a \lam \sig^2}{2 \mu^2} \sqrt{\frac{\mu^2}{\lam}} + \frac{a \lam \vpi^2 \sig}{2 \mu^2} \\
		&\hspace{5em} \phantom{=\ } + \frac{a \lam \vpi^2}{2 \mu^2} \sqrt{\frac{\mu^2}{\lam}} - a \sqrt{\frac{\mu^2}{\lam}} + \lam \sig^3 \sqrt{\frac{\mu^2}{\lam}} + \lam \vpi^2 \sig \sqrt{\frac{\mu^2}{\lam}} - \frac{\mu^4}{4 \lam} + \frac{\lam \sig^4}{4} + \frac{\lam \vpi^2 \sig^2}{2} + \frac{\lam (\vpi^2)^2}{4} + \mu^2 \sig^2 \\[1ex]
		&\approx \frac{a \lam \sig^3}{2 \mu^2} + \frac{3 a \lam \sig^2}{2 \mu^2} \sqrt{\frac{\mu^2}{\lam}} + \frac{a \lam \vpi^2 \sig}{2 \mu^2} + \frac{a \lam \vpi^2}{2 \mu^2} \sqrt{\frac{\mu^2}{\lam}} + \lam \sig^3 \sqrt{\frac{\mu^2}{\lam}} \\
		&\hspace{5em} \phantom{=\ } + \lam \vpi^2 \sig \sqrt{\frac{\mu^2}{\lam}} + \frac{\lam \sig^4}{4} + \frac{\lam \vpi^2 \sig^2}{2} + \frac{\lam (\vpi^2)^2}{4} + \mu^2 \sig^2 \\[1ex]
		&= \ans{ \frac{\lam}{4} (\vpi^2)^2 + \frac{a}{2} \sqrt{\frac{\lam}{\mu^2}} \vpi^2 + \lam \vpi^2 \sig \paren{ \frac{a}{2 \mu^2} + \sqrt{\frac{\mu^2}{\lam}} } + \frac{\lam \vpi^2 \sig^2}{2} } \\
		&\hspace{5em} \phantom{=\ } \ans{ + \sig^2 \paren{ \mu^2 + \frac{3 a}{2} \sqrt{\frac{\lam}{\mu^2}} } + \lam \sig^3 \paren{ \frac{a}{2 \mu^2} + \sqrt{\frac{\mu^2}{\lam}} } + \frac{\lam \sig^4}{4}, }
	}
	where we have dropped constant terms and terms of $\order{a^2}$.  Since the term
	\eqn{mpi}{
		\frac{a}{2} \sqrt{\frac{\lam}{\mu^2}} \vpi^2 = \frac{\mpi^2}{2} \vpi^2
	}
	appears, we can say that the pion acquires a mass such that $\mpi^2 \sim a$. \qed
	
	Note also that the sigma mass has changed:
	\eqn{msig}{
		\mu^2 + \frac{3 a}{2} \sqrt{\frac{\lam}{\mu^2}} = \frac{\msig^2}{2}.
	}
	The Feynman rules are
	\al{
		\centergraphics{diag/dbl_arrow} &= \ans{ \int \ddqpf \frac{i e^{-i p \cdot (x - y)}}{p^2 - \msig^2 + i \eps}, } &
		\centergraphics{diag/ij_arrow} &= \ans{ \delij \int \ddqpf \frac{i e^{-i p \cdot (x - y)}}{p^2 - \mpi^2 + i \eps}, } \\
		\centergraphics{diag/ij_triangle} &= \ans{ -2i \lam \paren{ \frac{a}{2 \mu^2} + \sqrt{\frac{\mu^2}{\lam}} } \delij, } &
		\centergraphics{diag/dbl_triangle} &= \ans{ -6i \lam \paren{ \frac{a}{2 \mu^2} + \sqrt{\frac{\mu^2}{\lam}} }, }
	}
	\vfix\vfix
	\al{
		\centergraphics{diag/ijkl_vertex} &= \ans{ -2 i \lam (\delij \delkl + \delil \deljk + \delik \deljl), } &
		\centergraphics{diag/ij_vertex} &= \ans{-2 i \lam \delij, } &
		\centergraphics{diag/dbl_vertex} &= \ans{ -6i \lam. }
	}
	
	Using the new Feynman rules, the contributions from each diagram are
	\al{
		\centergraphics{diag/ijkl_s_chan} &= \brac{ -2i \lam^2 \paren{ \frac{a}{2 \mu^2} + \sqrt{\frac{\mu^2}{\lam}} } \delij } \frac{i}{(\pq + \pw)^2 - \msig^2} \brac{ -2i \lam \paren{ \frac{a}{2 \mu^2} + \sqrt{\frac{\mu^2}{\lam}} } \delkl } \\
		&= -\frac{4 i \lam^2}{(\pq + \pw)^2 - \msig^2} \paren{ \frac{a}{2 \mu^2} + \sqrt{\frac{\mu^2}{\lam}} }^2 \delij \delkl, \\[2ex]
		%
		\centergraphics{diag/ijkl_t_chan} &= \brac{ -2i \lam^2 \paren{ \frac{a}{2 \mu^2} + \sqrt{\frac{\mu^2}{\lam}} } \delik } \frac{i}{(\pq - \pe)^2 - \msig^2} \brac{ -2i \lam \paren{ \frac{a}{2 \mu^2} + \sqrt{\frac{\mu^2}{\lam}} } \deljl } \\
		&= -\frac{4 i \lam^2}{(\pq - \pe)^2 - \msig^2} \paren{ \frac{a}{2 \mu^2} + \sqrt{\frac{\mu^2}{\lam}} }^2 \delik \deljl, \\[2ex]
		%
		\centergraphics{diag/ijkl_u_chan} &= \brac{ -2i \lam^2 \paren{ \frac{a}{2 \mu^2} + \sqrt{\frac{\mu^2}{\lam}} } \delil } \frac{i}{(\pq - \pr)^2 - \msig^2} \brac{ -2i \lam \paren{ \frac{a}{2 \mu^2} + \sqrt{\frac{\mu^2}{\lam}} } \deljk } \\
		&= -\frac{4 i \lam^2}{(\pq - \pr)^2 - \msig^2} \paren{ \frac{a}{2 \mu^2} + \sqrt{\frac{\mu^2}{\lam}} }^2 \delil \deljk, \\[2ex]
		%
		\centergraphics{diag/ijkl_vertex} &= -2 i \lam (\delij \delkl + \delil \deljk + \delik \deljl),
	}
	and the total amplitude for their sum is
	\al{
		i \cM &= -4 i \lam^2 \paren{ \frac{a}{2 \mu^2} + \sqrt{\frac{\mu^2}{\lam}} }^2 \paren{ \frac{\delij \delkl}{(\pq + \pw)^2 - \msig^2} + \frac{\delik \deljl}{(\pq - \pe)^2 - \msig^2} + \frac{\delil \deljk}{(\pq - \pr)^2 - \msig^2} } \\
		&\hspace{5em} \phantom{=\ } - 2 i \lam (\delij \delkl + \delil \deljk + \delik \deljl).
	}
	When $\vpsi = 0$ at the threshold, $p_i = (\mpi, 0)$.  With this substitution,
	\al{
		i \cM &= -4 i \lam^2 \paren{ \frac{a}{2 \mu^2} + \sqrt{\frac{\mu^2}{\lam}} }^2 \paren{ \frac{\delij \delkl}{(\mpi + \mpi)^2 - \msig^2} + \frac{\delik \deljl}{(\mpi - \mpi)^2 - \msig^2} + \frac{\delil \deljk}{(\mpi - \mpi)^2 - \msig^2} } \\
		&\hspace{5em} \phantom{=\ } - 2 i \lam (\delij \delkl + \delil \deljk + \delik \deljl) \\[1ex]
		&= -4 i \lam^2 \paren{ \frac{a}{2 \mu^2} + \sqrt{\frac{\mu^2}{\lam}} }^2 \paren{ \frac{\delij \delkl}{4\mpi^2 - \msig^2} - \frac{\delik \deljl}{\msig^2} - \frac{\delil \deljk}{\msig^2} } - 2 i \lam (\delij \delkl + \delil \deljk + \delik \deljl).
	}
	We can write $\mpi$ and $\msig$ in terms of $a$ using Eqs.~\refeq{mpi} and \refeq{msig}, and drop terms of $\order{a^2}$:
	\al{
		i \cM &= -4 i \lam^2 \paren{ \frac{a}{2 \mu^2} + \sqrt{\frac{\mu^2}{\lam}} }^2 \brac{ \delij \delkl \paren{ 4 a \sqrt{\frac{\lam}{\mu^2}} - 2 \mu^2 - 3 a \sqrt{\frac{\lam}{\mu^2}} }^{-1} - (\delik \deljl + \delil \deljk) \paren{ 2 \mu^2 + 3 a \sqrt{\frac{\lam}{\mu^2}} }^{-1} } \\
		&\hspace{5em} \phantom{=\ } - 2 i \lam (\delij \delkl + \delil \deljk + \delik \deljl) \\[1ex]
		&\approx -4 i \lam^2 \paren{ \frac{\mu^2}{\lam} + \frac{a}{\mu \sqrt{\lam}} + \frac{a^2}{4 \mu^4} } \paren{ \frac{\delij \delkl}{a \sqrt{\lam} / \mu - 2 \mu^2} - \frac{\delik \deljl + \delil \deljk}{3 a \sqrt{\lam} / \mu + 2 \mu^2 } } - 2 i \lam (\delij \delkl + \delil \deljk + \delik \deljl) \\[1ex]
		&\approx 4 i \lam^2 \paren{ \frac{\mu^2}{\lam} + \frac{a}{\mu \sqrt{\lam}} } \paren{ \frac{\delij \delkl + \delik \deljl + \delil \deljk}{2 \mu^2 } } - 2 i \lam (\delij \delkl + \delil \deljk + \delik \deljl) \\[1ex]
		&= \ans{ \frac{2 i a \lam^{3/2}}{\mu^3} (\delij \delkl + \delik \deljl + \delil \deljk), }
	}
	which is nonzero and proportional to $a$ as we wanted to show. \qed
}