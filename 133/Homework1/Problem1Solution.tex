\documentclass[11pt]{article}
\usepackage{geometry, titlesec}
\usepackage[parfill]{parskip}
\usepackage[italicdiff]{physics}
\usepackage{amsfonts, amsthm}
\usepackage[cm]{fullpage}
\usepackage{fancyhdr}
\usepackage{xcolor}
\usepackage{siunitx}
\usepackage{enumitem}
%\allowdisplaybreaks
 
\renewcommand{\footrulewidth}{.2pt}
%\setlist[enumerate]{leftmargin=*}
\pagestyle{fancy}
\fancyhf{}
\lhead{Homework 1}
\rhead{Physics 133-B}
\setlength{\headheight}{11pt}
\setlength{\headsep}{11pt}
\setlength{\footskip}{24pt}
\lfoot{\today}
\rfoot{\thepage}

\titleformat{\section}[runin]{\normalfont\large\bfseries}{Problem \thesection.}{1em}{}
\titleformat{\subsection}[runin]{\normalfont\large\bfseries}{\thesubsection}{1em}{}
\titleformat{\subparagraph}[leftmargin]{\normalfont\large\bfseries}{}{0pt}{}
\newcommand{\refeq}[1]{(\ref{#1})}

\setenumerate[1]{label={(\alph*)}}

\newcommand{\beq}{\begin{equation*}}
\newcommand{\eeq}{\end{equation*}}

\newcommand{\beqn}{\begin{equation}}
\newcommand{\eeqn}{\end{equation}}

\newcommand{\qimplies}{\quad \implies \quad}


\newenvironment{statement}[1]
{
	\section{#1}
	\ignorespaces
}

\newenvironment{problem}
{
    \color{darkgray}
    \ignorespaces
}

\newenvironment{solution}
{
    \paragraph{Solution.}
    \ignorespaces
}

\begin{document}

\newcommand{\yxt}{y(x, t)}
\newcommand{\yo}{y_0}
\newcommand{\omg}{\omega}
\newcommand{\lam}{\lambda}
\newcommand{\evP}{\ev{P}}

\newcommand{\amp}{\SI{2.30}{\mm}}
\newcommand{\waven}{\SI{6.98}{\radian\per\meter}}
\newcommand{\freq}{\SI{742}{\radian\per\second}}
\newcommand{\len}{\SI{1.35}{\meter}}
\newcommand{\mass}{\SI{0.00338}{\kg}}

\begin{statement}{}
	% Y&F 15.26
	A fellow student with a mathematical bent tells you that the wave function of a traveling wave on a thin rope is
	\beqn \label{given}
		\yxt = (\amp) \cos[(\waven) x + (\freq) t ].
	\eeqn
	Being more practical, you measure the rope to have a length of {\len} and a mass of {\mass}.  You are then asked to determine the following:
	\begin{enumerate}
		\item amplitude;
		\item frequency;
		\item wavelength;
		\item wave speed;
		\item direction the wave is traveling;
		\item tension in the rope;
		\item average power transmitted by the wave.
	\end{enumerate}
\end{statement}



\begin{solution}
	\begin{enumerate}
		\item A standing wave has the general form
		\beq
			\yxt = \yo \sin(k x - \omg t),
		\eeq
		where $\yo$ is the amplitude of the wave, $k$ its wavenumber, and $\omg$ its angular frequency.  However, since sine and cosine differ only by a phase, we might as well write
		\beqn \label{form}
			\yxt = \yo \cos(k x - \omg t),
		\eeqn
		which is the form given in the problem, Eq.~\refeq{given}.  Then we can easily read off the amplitude:
		\beq
			\yo = {\color{blue} \amp}.
		\eeq
		
		\item Once again referring to Eq.~\refeq{form}, we can read off the \emph{angular} frequency $\omg = \freq$ from Eq.~\refeq{given}.  Then we can easily solve for the frequency $f$:
		\beq
			f = \frac{\omg}{2\pi}
			= \frac{\freq}{2\pi \,\si{\radian}}
			= {\color{blue} \SI{118}{\Hz}}.
		\eeq
		
		\item Reading off the wave number from Eq.~\refeq{given}, we find $k = \waven$.  Solving for the wavelength $\lam$, we find
		\beq
			\lam = \frac{2\pi}{k}
			= \frac{2\pi \,\si{\radian}}{\waven}
			= {\color{blue} \SI{0.90}{\meter}}
			= {\color{blue} \SI{90}{\cm}}.
		\eeq
		
\newcommand{\wspd}{\SI{106}{\meter\per\second}}
		
		\item The wave speed is defined as $v = \omg / k$.  Plugging in the values of $\omg$ and $k$ that we found in (b) and (c),
		\beq
			v = \frac{\omg}{k}
			= \frac{\freq}{\waven}
			= {\color{blue} \wspd}.
		\eeq
		
		\item Equation~\refeq{form} gives the general expression for a wave traveling in the $+x$ direction.  Here, the argument of the cosine function is $k x + \omg t$.  However, in the given expression of Eq.~\refeq{given}, the argument has the form $k x - \omg t$.  This means that the wave is traveling in the {\color{blue} $-x$ direction}.
		
		\item Another expression for the wave speed is
		\beq	
			v = \sqrt{\frac{T}{\mu}},
		\eeq
		where $\mu$ is the mass density of the rope, and $T$ is the tension in the rope.  Solving this definition for $T$ and substituting in $\mu = m / L$ gives us
		\beq
			T = \mu v^2
			= \frac{m v^2}{L}.
		\eeq
		Plugging in the given values of $m$ and $L$, and our result for $v$ from (d), we find
		\beq
			T = \frac{(\mass) (\wspd)^2}{\len}
			= {\color{blue} \SI{28.3}{\newton}}.
		\eeq
		
		\item The average power $\evP$ transmitted by the wave is given by
		\beq
			\evP = \frac{1}{2} \mu \omg^2 \yo^2 v
			= \frac{m \omg^2 \yo^2 v}{2 L},
		\eeq
		so plugging in known quantities and previous results gives us
		\beq
			\evP = \frac{(\mass) (\freq)^2 (\SI{0.0023}{\meter})^2 (\wspd)}{2 (\len)}
			= {\color{blue} \SI{0.39}{\watt}}.
		\eeq
		
	\end{enumerate}
\end{solution}



\end{document}