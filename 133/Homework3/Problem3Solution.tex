\documentclass[11pt]{article}
\usepackage{geometry, titlesec}
\usepackage[parfill]{parskip}
\usepackage[italicdiff]{physics}
\usepackage{amsfonts, amsthm}
\usepackage[cm]{fullpage}
\usepackage{fancyhdr}
\usepackage{xcolor}
\usepackage{siunitx}
%\allowdisplaybreaks
 
\renewcommand{\footrulewidth}{.2pt}
\pagestyle{fancy}
\fancyhf{}
\lhead{Homework 3}
\rhead{Physics 133-B}
\setlength{\headheight}{11pt}
\setlength{\headsep}{11pt}
\setlength{\footskip}{24pt}
\cfoot{\today}
\rfoot{\thepage}

\titleformat{\section}[runin]{\normalfont\large\bfseries}{Problem \thesection.}{1em}{}
\titleformat{\subsection}[runin]{\normalfont\large\bfseries}{\thesubsection}{1em}{}
\titleformat{\subparagraph}[leftmargin]{\normalfont\large\bfseries}{}{0pt}{}
\newcommand{\refeq}[1]{(\ref{#1})}

\newcommand{\beq}{\begin{equation*}}
\newcommand{\eeq}{\end{equation*}}

\newcommand{\beqn}{\begin{equation}}
\newcommand{\eeqn}{\end{equation}}

\newcommand{\qimplies}{\quad \implies \quad}


\newenvironment{statement}[1]
{
	\section{#1}
	\ignorespaces
}

\newenvironment{solution}
{
    \paragraph{Solution.}
    \ignorespaces
}

\begin{document}

\newcommand{\lam}{\lambda}
\newcommand{\omg}{\omega}

\newcommand{\vE}{\vb{E}}
\newcommand{\vB}{\vb{B}}
\newcommand{\Eo}{E_0}
\newcommand{\Bo}{B_0}

\newcommand{\Eamp}{\SI{3.20e-3}{\volt\per\meter}}
\newcommand{\wavel}{\SI{475}{\nano\meter}}
\newcommand{\wavelSI}{\SI{475e-9}{\meter}}
\newcommand{\lspeed}{\SI{3.00e8}{\meter\per\second}}

\newcommand{\xh}{\vb{\hat{x}}}
\newcommand{\yh}{\vb{\hat{y}}}
\newcommand{\zh}{\vb{\hat{z}}}


\setcounter{section}{2}

\begin{statement}{}
	% Y&F 32.8
	Consider a sinusoidal electromagnetic wave propagating in the $+x$ direction, whose electric field is parallel to the $y$ axis.  The wave has wavelength {\wavel}, and the electric field has amplitude {\Eamp}.  What is the frequency of the wave?  What is the amplitude of the magnetic field?  What are the vector equations for $\vE(x, t)$ and $\vB(x, t)$?
\end{statement}

\begin{solution}
	Frequency is related to wavelength by $v = \lam f$, where the wave speed $v = c$ for an electromagnetic wave in vacuum.  So
	\beq
		f = \frac{c}{\lam}
		= \frac{\lspeed}{\wavelSI}
		= \frac{\num{3.00e8}}{\num{4.75e-7}} \,\si{\Hz}
		= {\color{blue} \SI{6.32e14}{\Hz}}.
%		= {\color{blue} \SI{632}{\tera\Hz}}.
	\eeq
	
	The amplitudes of the fields are related by $\Eo = c \Bo$, so
	\beq
		\Bo = \frac{\Eo}{c}
		= \frac{\Eamp}{\lspeed}
		= \frac{3.20}{3.00} \times \SI{e-11}{\tesla}
		= {\color{blue} \SI{1.07e-11}{\tesla}}.
%		= {\color{blue} \SI{10.7}{\pico\tesla}},
	\eeq
	where we have used $\SI{1}{\tesla} = \SI{1}{\volt\second\per\square\meter}$.
	
	The direction of propagation of the wave is $\vE \cross \vB$.  We know from the problem statement that the wave is propagating in the $\xh$ direction, and that the electric field points in the $\yh$ direction.  Since $\yh \cross \zh = \xh$, the magnetic field must point in the $\zh$ direction.  Then the vector equations are
	\begin{align*}
		\vE(x, t) &= \Eo \cos(k x - \omg t), &
		\vB(x, t) &= \Bo \cos(k x - \omg t).
	\end{align*}	
	We can find the wave number $k$ and angular frequency $\omg$ as follows:
	\begin{align*}
		k &= \frac{2\pi}{\lam}
		= \frac{\SI{2\pi}{\radian}}{\SI{4.75e-7}{\meter}}
		= \SI{1.32e7}{\radian\per\meter}, \\
		\omg &= 2\pi f
		= (\SI{2\pi}{\radian}) (\SI{6.32e14}{\Hz)}
		= \SI{3.97e15}{\radian\per\second}.
	\end{align*}
	Then we have
		\begin{align*}
		\vE(x, t) &= (\Eamp) \cos[ (\SI{1.32e7}{\radian\per\meter}) x - (\SI{3.97e15}{\radian\per\second}) t], \\
		\vB(x, t) &= (\SI{1.07e-11}{\tesla}) \cos[ (\SI{1.32e7}{\radian\per\meter}) x - (\SI{3.97e15}{\radian\per\second}) t].
	\end{align*}
\end{solution}

\end{document}