\documentclass[11pt]{article}
\usepackage{solutions}

\classname{133}
\homeworknum{6}


\DeclareSIUnit{\foot}{ft}
\DeclareSIUnit{\mile}{mi}

\begin{document}

% Environments

\newcommand{\state}[2]{\begin{statement}{#1} #2 \end{statement}}
\newcommand{\prob}[2]{\begin{problem}{#1} #2 \end{problem}}
\newcommand{\subprob}[1]{\begin{subproblem} #1 \end{subproblem}}
\newcommand{\sol}[1]{\begin{solution} #1 \end{solution}}
\newcommand{\fig}[2]{\begin{figure} \centering #2  \label{#1} \end{figure}}

\newcommand{\makebib}{
	\vfill
	\color{black}
	\nocite{*}
	\bibliography{references}{}
	\bibliographystyle{lucas_unsrt}
}
	

% Implication

\newcommand{\qwhere}{\quad \text{where} \quad}
\newcommand{\qimplies}{\quad \implies \quad}
\newcommand{\impliesq}{\implies \quad}



% Brackets

\newcommand{\paren}[1]{\left( #1 \right)}
\newcommand{\brac}[1]{\left[ #1 \right]}
\newcommand{\curly}[1]{\left\{ #1 \right\}}


% Greek

\newcommand{\alp}{\alpha}
\newcommand{\bet}{\beta}
\newcommand{\gam}{\gamma}
\newcommand{\del}{\delta}
\newcommand{\eps}{\epsilon}
\newcommand{\zet}{\zeta}
\newcommand{\tht}{\theta}
\newcommand{\kap}{\kappa}
\newcommand{\lam}{\lambda}
\newcommand{\sig}{\sigma}
\newcommand{\ups}{\upsilon}
\newcommand{\omg}{\omega}

\newcommand{\Gam}{\Gamma}
\newcommand{\Del}{\Delta}
\newcommand{\Tht}{\Theta}
\newcommand{\Lam}{\Lambda}
\newcommand{\Sig}{\Sigma}
\newcommand{\Omg}{\Omega}


% Text

\newcommand{\where}{\text{where }}


%
%	Prob. 3
%

\newcommand{\angleq}{\SI{17.0}{\degree}}
\newcommand{\anglew}{\SI{32.0}{\degree}}
\newcommand{\inten}{\SI{53.0}{\watt\per\square\cm}}

\newcommand{\Io}{I_0}
\newcommand{\Imax}{I_\text{max}}

\newcommand{\Iq}{I_1}
\newcommand{\Iw}{I_2}
\newcommand{\Ie}{I_3}
\newcommand{\phiqw}{\phi_{1 2}}
\newcommand{\phiwe}{\phi_{2 3}}
\newcommand{\phiqe}{\phi_{1 3}}

\newcommand{\angleqw}{\SI{49.0}{\degree}}
\newcommand{\angleqe}{\SI{58.0}{\degree}}
\newcommand{\inteno}{\SI{269}{\watt\per\square\cm}}

\setcounter{section}{2}

% Y&F 33.34
\state{}{
	You have three polarizing filters, which you stack on top of one another to see how much light will pass through.  You first arrange the polarizers such that the middle polarizer's axis is {\angleq} clockwise to that of the bottom polarizer, and the top polarizer's axis is {\anglew} counterclockwise to that of the bottom polarizer.  You shine unpolarized light on the stack, and measure the intensity of the light that passes through as {\inten}.  Now you remove the middle polarizer and shine the same light on the remaining two polarizers.  What intensity will you measure?
}

\sol{
	When \emph{unpolarized} light of intensity $\Io$ is incident on a polarizer, the intensity of the outgoing light is
	\eq{
		I = \frac{\Io}{2}.
	}
	The outgoing light is polarized in the direction of the filter's polarizing axis.
	
	When \emph{polarized} light of intensity $\Imax$ is incident on an analyzer, the intensity of the outgoing light is given by Malus's law:
	\eq{
		I = \Imax \cos^2\phi,
	}
	where $\phi$ is the angle between the incident light's polarization direction and the analyzer's polarizing axis.
	
	We will use these two expressions to find $\Io$ using the data for three polarizers, and then find $I$ for two polarizers.
	
	Let $\Iq$ be the intensity of the light outgoing from the top polarizer, which is
	\eq{
		\Iq = \frac{\Io}{2}.
	}
	The light incident on the middle filter is then polarized in the direction of the top filter's polarization axis.
	
	Let $\Iw$ be the intensity of the light outgoing from the middle polarizer.  The angle between the polarization axes of the top and middle filters is $\phiqw = \angleq + \anglew = \angleqw$, and
	\eq{
		\Iw = \Iq \cos^2\phiqw.
	}
	The light incident on the bottom filter is then polarized in the direction of the middle filter's polarization axis.
	
	Let $\Ie$ be the intensity of the light outgoing from the bottom polarizer, which we know to be {\inten}.    The angle between the polarization axes of the middle and bottom filters is $\phiwe = \angleq$, and
	\eq{
		\Ie = \Iw \cos^2\phiwe.
	}
	
	We can combine these results and plug in known quantities to find
	\eq{
		\Io = 2 \Iq
		= \frac{2 \Iw}{\cos^2\phiqw}
		= \frac{2 \Ie}{\cos^2\phiqw \cos^2\phiwe}
		= \frac{2 (\inten)}{\cos[2](\angleqw) \cos[2](\angleq)}
		\approx \frac{106}{(0.656)^2 (0.956)^2} \,\si{\watt\per\square\cm}
		\approx \ans{ \inteno. }
	}
	
	When the middle polarizer is removed, the intensity of the outgoing light from the top polarizer is still $\Iq$.  Let $\Ie'$ be the intensity of the outgoing light from the bottom polarizer in this case.  The angle between the polarization axes of the top and bottom filters is $\phiqe = \SI{90.0}{\degree} - \anglew = \angleqe$, and
	\eq{
		\Ie' = \Iq \cos^2\phiqe
		= \frac{\Io \cos^2\phiqe}{2}
		\approx \frac{(\inteno) \cos[2](\angleqe)}{2}
		\approx (134.5) (0.530)^2  \,\si{\watt\per\square\cm}
		\approx \ans{ \SI{37.8}{\watt\per\square\cm}, }
	}
	which is less than when all three polarizers were present.
}



%
%	Prob. 4
%

\newcommand{\objf}{\SI{100.0}{\cm}}
\newcommand{\ocuf}{\SI{18.0}{\cm}}
\newcommand{\bheight}{\SI{200.7}{\foot}}
\newcommand{\bdist}{\SI{2.00}{\mile}}

\newcommand{\fobj}{f_\text{obj}}
\newcommand{\focu}{f_\text{ocu}}

\newcommand{\hi}{h_i}
\newcommand{\hs}{h_s}
\newcommand{\thti}{\tht_i}
\newcommand{\thts}{\tht_s}

\newcommand{\Mag}{\num{-5.56}}
\newcommand{\bdistSI}{\SI{3220}{\meter}}
\newcommand{\bheightSI}{\SI{61.2}{\meter}}
\newcommand{\objfSI}{\SI{1.00}{\meter}}
\newcommand{\iheight}{\SI{1.90}{\cm}}
\newcommand{\angsize}{\SI{0.106}{\radian}}

% Y&F 34.61
\state{}{
	You have a refracting telescope with an objective lens of focal length {\objf} and an an ocular lens of focal length {\ocuf}.  What is your telescope's magnification?  You use your telescope to look at the Rockefeller Chapel, which is {\bheight} tall, from {\bdist} away.  What is the height of the image formed by the objective lens?  What is the angular size of the image that you view through the ocular lens?
}

\sol{
	Angular magnification is defined
	\eq{
		M = \frac{\thti}{\thts},
	}
	where $\thti$ is the angular size of the image and $\thts$ that of the object. The angular magnification of a telescope is given by
	\eq{
		M = -\frac{\fobj}{\focu},
	}
	where $\fobj$ is the focal length of the objective lens and $\focu$ that of the ocular lens.  Plugging in numbers, we find
	\eq{
		M = -\frac{\objf}{\ocuf}
		\approx \ans{ \Mag. }
	}
	
	To find the height of the image of Rockefeller chapel, we use the approximation that the image distance $i \approx \fobj$, which works when the object distance $s$ is very large.  The lateral magnification for a thin lens, such as the objective, is
	\eq{
		m = -\frac{i}{s}
		\approx -\frac{\fobj}{s}.
	}
	The lateral magnification also relates object and image heights by
	\eq{
		m = \frac{\hi}{\hs},
	}
	where $\hi$ is the height of the image and $\hs$ that of the object.  Combining the two expressions and solving for $\hi$, we have
	\eq{
		-\frac{\fobj}{s} = \frac{\hi}{\hs}
		\qimplies
		\hi = -\frac{\fobj \hs}{s}.
	}
	Note that
	\al{
		s &= (\bdist) \frac{\SI{1609}{\meter}}{\SI{1}{\mile}}
		\approx \bdistSI, &
		\hs &= (\bheight) \frac{\SI{30.48}{\cm}}{\SI{1}{\foot}}
		\approx \bheightSI.
	}
	Then, plugging in numbers,
	\eq{
		\hi = -\frac{(\objfSI) (\bheightSI)}{\bdistSI}
		= -\frac{61.2}{3220} \,\si{\meter}
		= -\SI{0.0190}{\meter}
		= -\iheight,
	}
	so the height of the image is \ans{\iheight. }
	
	The angular size of the image formed by the objective lens, which is the object for the ocular lens, is
	\eq{
		\thts = \frac{\abs{\hi}}{s}
		\approx \frac{\abs{\hi}}{\fobj}.
	}
	Then the angular size of the image formed by the ocular lens can be found using the magnification:
	\eq{
		\thti = M \thts
		= -\frac{\fobj}{\focu} \frac{\abs{\hi}}{\fobj}
		= -\frac{\abs{\hi}}{\focu}
		\approx -\frac{\iheight}{\ocuf}
		\approx -\angsize,
	}
	so the angular size is \ans{\angsize.}  The minus sign tells us that the image is inverted.
}



\end{document}