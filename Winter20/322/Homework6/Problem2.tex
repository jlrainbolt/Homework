\begin{statement}{}
	A point charge of charge $q$ and mass $m$ is placed on the end of a spring of spring constant $k$.  The charge is displaced in the $z$ direction by an amount $\alp$ away from its equilibrium position and is then released into oscillation.  Assume that the resulting motion is nonrelativistic, $v \ll c$.
\end{statement}

\begin{problem}
	Assume that the charge oscillates harmonically with amplitude $\alp$.  To order $1 / r$ in distance from the charge and to leading order in $v / c$, what are the resulting electromagnetic potentials $\phi, \vA$?
\end{problem}

\begin{solution}
	The multipole expansions of $\phi$ and $\vA$ are given to order $1 / r$ in distance and to leading order in $v / c$ in Eqs.~(5.66) and (5.69),
	\begin{align*}
		\phi(t, \vx) &= \frac{q}{\absx} + \frac{1}{c \absx} \xh \vdot \dv{\vp}{t} \bigg|_\ret, &
		\vA(t, \vx) &= \frac{1}{c \absx} \dv{\vp}{t} \bigg|_\ret,
	\end{align*}
	where $\xh = \vx / \absx$, and the retarded time $t' = t - \absx / c$ to this order.

	The charge's position is
	\beq
		\vx'(t) = \alp \cos \omg t \,\zh,
	\eeq
	where $\omg = \sqrt{k / m}$, and the charge density is
	\beq
		\rho(t, \vx) = q \,\del(\vx - \vx').
	\eeq
	Then from \refeq{dipole}, the dipole moment is
	\beq
		\vp = q \int \vx \,\del(\vx - \vx') \dcx
		= q \vx'
		= q \alp \cos \omg t \,\zh,
	\eeq
	which has the time derivative
	\beqn \label{pder}
		\dv{\vp}{t} = - q \alp \omg \sin \omg t \,\zh.
	\eeqn
	The potentials are then
	\begin{align*}
		\phi(t, \vx) &= \frac{q}{r} - \frac{q \alp \omg }{c r} \sin(\omg t - \frac{\omg}{c} r) (\rh \vdot \zh)
		= \frac{q}{r} \left[ 1 - \frac{\alp \omg \cos\tht}{c} \sin(\omg t - \frac{\omg}{c} r) \right], \\
		\vA(t, \vx) &= -\frac{q}{r} \frac{\alp \omg }{c} \sin(\omg t - \frac{\omg}{c} r) \,\zh,
	\end{align*}
	where we have used $\zh = \cos\tht \,\rh - \sin\tht \,\thh$.
\end{solution}



\begin{problem}
	What is the radiated power?
\end{problem}

\begin{solution}
	To this order, we may use \refeq{power} with the retarded time $t' = t - \absx / c$.  Differentiating \refeq{pder}, we have
	\beq
		\dv[2]{\vp}{t} = -q \alp \omg^2 \cos \omg t \,\zh,
	\eeq
	so the radiated power is
	\beqn \label{radpow}
		P = \frac{2}{3 c^3} \abs{-q \alp \omg^2 \cos \omg t \,\zh}^2_\ret
		= \frac{2 q^2 \alp^2 \omg^4}{3 c^3} \abs{\cos[2](\omg t - \frac{\omg}{c} r)}
		= \frac{q^2 \alp^2 \omg^4}{3 c^3},
	\eeqn
	where we have used the time average $\abs{\cos^2 t} = 1/2$.
\end{solution}



\begin{problem}
	As a result of the radiation of electromagnetic energy, the maximum amplitude of oscillation, $\alp$, will slowly decay with time.  Find $\alp(t)$.
\end{problem}

\begin{solution}
	The quality factor $Q$ characterizes energy loss, and is defined in Jackson~(8.86) as
	\beq
		Q = \omg_0 \frac{\text{Stored energy}}{\text{Power loss}},
	\eeq
	where $\omg_0$ is a resonant frequency.  For the oscillating charge,
	\beq
		Q = \omg \frac{\sEosc}{P},
	\eeq
	where $\sEosc$ is the energy stored in the oscillator, and $P$ is the radiated power given by \refeq{radpow}.  Adapting Eq.~(5.75) in the course notes, we can write
	\beq
		P = -\dv{\sEosc}{t}.
	\eeq
	Applying the definition of $Q$ gives us a differential equation:
	\beq
		\dv{\sE}{t} = -\frac{\omg}{Q} \sEosc
		\qimplies
		\int \frac{\dd{\sEosc}}{\sEosc} = -\frac{\omg}{Q} \int \dd{t}
		\qimplies
		\ln\frac{\sEosc}{\sEosc(0)} = -\frac{\omg}{Q} t,
	\eeq
	and the final solution is
	\beqn \label{ensol}
		\sEosc(t) = \sEosc(0) \, e^{-\omg t / Q},
	\eeqn
	where $\sEosc(0)$ is the initial energy stored in the oscillator.
	
	The energy stored in an oscillator at any given time is equivalent to its maximal potential or kinetic energy,
	\beq
		\sEosc = \frac{m \omg^2 \alp^2}{2} = \frac{k \alp^2}{2},
	\eeq
	and the average power radiated over one oscillation comes directly from \refeq{radpow}.  We can use these to write $Q$ with no $\alp$ dependence:
	\beq
		Q = \omg \frac{m \omg^2 \alp^2}{2} \frac{3 c^3}{q^2 \alp^2 \omg^4}
		= \frac{3 m c^3}{2 q^2 \omg}.
	\eeq
	Substituting these into \refeq{ensol}, we obtain
	\beq
		\frac{k}{2} \alp^2(t) = \frac{k}{2} \alp^2(0) \,\exp(-\omg t \frac{2 q^2 \omg}{3 m c^3})
		\qimplies
		\alp^2(t) = \alpo^2 \,\exp(-\frac{2 q^2 \omg^2}{3 m c^3} t)
	\eeq
	which implies
	\beq
		\alp(t) = \alpo \,\exp(-\frac{q^2 \omg^2}{3 m c^3} t),
	\eeq
	where $\alpo$ is the initial displacement from equilibrium.
\end{solution}