\newcommand{\vp}{\vec{p}}
\newcommand{\vpq}{\vp_1}
\newcommand{\vpw}{\vp_2}
\newcommand{\vxq}{\vx_1}
\newcommand{\vxw}{\vx_2}
\newcommand{\Qq}{Q_1}
\newcommand{\Qw}{Q_2}
\newcommand{\phiq}{\phi_1}
\newcommand{\phiw}{\phi_2}
\newcommand{\vdq}{\vec{d}_1}
\newcommand{\vdw}{\vec{d}_2}
\newcommand{\vF}{\vec{F}}
\newcommand{\vEo}{\vE_0}

\newcommand{\rhow}{\rho_2}
\newcommand{\vEq}{\vE_1}
\newcommand{\pq}{p_1}
\newcommand{\pw}{p_2}


\begin{statement}{}
	The potential of an electrostatic dipole of dipole moment $\vp$ located at $\vx'$ is given by
	\beqn \label{pot1}
		\phi(\vx) = \frac{\vp \cdot (\vx - \vx')}{|\vx - \vx'|^3}.
	\eeqn
	Suppose a dipole $\vpq$ is located at $\vxq$ and another dipole $\vpw$ is located at $\vxw$.
\end{statement}

\begin{problem} \label{3a}
	What is the electrostatic force on the second dipole?
\end{problem}

\begin{solution}
	The total force on a charged body in electrostatics is given by Eq.~(2.42) in the text,
	\beq
		\vF = \int \rho(\vx) \vEo(\vx) \dcx,
	\eeq
	where $\rho$ is the charge density of the body and $\vEo$ is the external field.  For the force on $\vpw$, this becomes
	\beq
		\vF = \int \rhow(\vx) \vEo(\vx) \dcx.
	\eeq
	Since $\vpw$ consists of two point charges,
	\beqn \label{rho1}
		\rhow(\vx) = \Qw [\delta(\vx - \vxw) - \delta(\vx - \vxw + \vdw)],
	\eeqn
	where we have defined the displacement vector of $\vpw$ as $\vdw$.  Let $\phiq$ be the potential of $\vpq$; that is,
	\beqn \label{phi1}
		\phiq(\vx) = \frac{\vpq \cdot (\vx - \vxq)}{|\vx - \vxq|^3}.
	\eeqn
	Let $\vr = \vxw - \vxq$, and assume $d_2 \ll r$.  The force on $\vpw$ is given by
	\begin{align*}
		\vF &= -\int \Qw [\delta(\vx - \vxw) - \delta(\vx - \vxw + \vdw)] \nabla \phiq \dcx \\
		&= \Qw \left[ \nabla \left( \frac{\vpq \cdot (\vxw - \vdw - \vxq)}{|\vxw - \vdw - \vxq|^3} \right) - \nabla \left( \frac{\vpq \cdot (\vxw - \vxq)}{|\vxw - \vxq|^3} \right) \right]
		= \Qw \nabla \left( \frac{\vpq \cdot (\vr - \vdw)}{|\vr - \vdw|^3} - \frac{\vpq \cdot \vr}{r^3} \right) \\
		&\approx \Qw \nabla \left[ \vpq \cdot (\vr - \vdw) \left( \frac{1}{r^3} + \frac{3 \vdw \cdot \vr}{r^5} \right) - \frac{\vpq \cdot \vr}{r^3} \right]
		= \Qw \nabla \left( \frac{\vpq \cdot (\vr - \vdw)}{r^3} + \frac{3 \vpq \cdot (\vr - \vdw) (\vdw \cdot \vr)}{r^5} - \frac{\vpq \cdot \vr}{r^3} \right) \\
		&\approx \Qw \nabla \left( \frac{3(\vpq \cdot \vr)(\vdw \cdot \vr)}{r^5} - \frac{\vpq \cdot \vdw}{r^3} \right)
		= \nabla \left( \frac{3(\vpq \cdot \vr)(\vpw \cdot \vr)}{r^5} - \frac{\vpq \cdot \vpw}{r^3} \right) \\
		&= \nabla \left( \frac{\vpq \cdot \vpw}{|\vxw - \vxq|^3} - \frac{3[\vpq \cdot (\vxw - \vxq)][\vpw \cdot (\vxw - \vxq)]}{|\vxw - \vxq|^5} \right),
	\end{align*}
	where we have used the Taylor series expansion
	\beq
		\frac{1}{|\vr - \vdw|^3} = \frac{1}{r^3} + \frac{3 \vdw \cdot \vr}{r^5} + \cdots
	\eeq
	and neglected terms of $\order{d^2}$.  Evaluating the gradient,
	\begin{align*}
		\vF &= \frac{3 \nabla(\vpq \cdot \vr) (\vpw \cdot \vr)}{r^5} + 3 \frac{(\vpq \cdot \vr) \nabla(\vpw \cdot \vr)}{r^5} + \nabla(r^{-5}) (\vpq \cdot \vr)(\vpw \cdot \vr) - \nabla(r^{-3}) (\vpq \cdot \vpw) \\
		&= \frac{3 \pq (\vpw \cdot \vr)}{r^5} + \frac{3 \pw (\vpq \cdot \vr)}{r^5} - \frac{15 \vr (\vpq \cdot \vr) (\vpw \cdot \vr)}{r^7} + \frac{3 (\vpq \cdot \vpw)}{r^5} \\
		&= \frac{3}{r^4} [ \pq (\vpw \cdot \rh) + \pw (\vpq \cdot \rh) + \vpq \cdot \vpw - 5 (\vpq \cdot \rh) (\vpw \cdot \rh) ].
	\end{align*}
\end{solution}
\vfix

\begin{problem}
	What is the electrostatic interaction energy of the two dipoles?
\end{problem}

\begin{solution}
	Equation~(2.30) in the course notes gives the interaction energy of two charge distributions:
	\beqn \label{Eint}
		\sEint = \int \rhow \phiq \dcx,
	\eeqn
	where $\rhow$ is the charge density of $\vpw$ and $\phiq$ is given by \refeq{phi1}.  Applying \refeq{rho1}, \refeq{Eint} becomes
	\begin{align*}
		\sEint &= \Qw \int [\delta(\vx - \vxw) - \delta(\vx - \vxw + \vdw)] \phiq \dcx
		= \Qw \left( \frac{\vpq \cdot \vr}{r^3} - \frac{\vpq \cdot (\vr - \vdw)}{|\vr - \vdw|^3} \right) \\
		&= \Qw \left( \frac{3[\vpq \cdot (\vxw - \vxq)][\vpw \cdot (\vxw - \vxq)]}{|\vxw - \vxq|^5} - \frac{\vpq \cdot \vpw}{|\vxw - \vxq|^3} \right),
	\end{align*}
	where we have repeated the calculations of \ref{3a}.  So we see that the force on the second dipole may be found by $F = -\nabla \sEint$.
\end{solution}