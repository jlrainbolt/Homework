\documentclass[12pt]{article}
\usepackage{geometry, titlesec}
\usepackage[parfill]{parskip}
\usepackage{amsfonts, amsthm}
%\usepackage[cm]{fullpage}
\usepackage{fancyhdr}
\usepackage{enumitem}
%\allowdisplaybreaks

%\renewcommand{\footrulewidth}{.2pt}
%\setlist[enumerate]{leftmargin=*}
\pagestyle{fancy}
\fancyhf{}
\lhead{\textbf{Lab Report Guidelines}}
\rhead{Physics 132-B}
\setlength{\headheight}{48pt}
\setlength{\headsep}{24pt}
%\setlength{\footskip}{24pt}
%\cfoot{\today}
%\rfoot{\thepage}


\begin{document}

When writing your lab report, aim to provide enough information that a reader unfamiliar with the experiment would be able to perform it themselves, given only your report and the appropriate tools.  However, this does not mean that the report has to be extremely long!  Just include enough detail to provide the reader with a complete picture of the experiment.

\bigskip

A complete report should include the following, for a total of \textbf{6} possible points:

\medskip
\begin{description}[style=nextline]
	\item[(1.5) Experimental setup and procedure]
	
	State what type of experiment you are performing and what you are measuring.  Name the instruments you used and describe how they were set up.  (Pictures and diagrams are a great aid, but not substitutes for verbal descriptions.)  Describe how you used the instruments to collect your data.

\medskip
\item[(0.5) Experimental uncertainties]

	List any \emph{major} sources of systematic uncertainty.  State how much of an impact, if any, you expect them to have on your results.  If you are asked to include a systematic uncertainty as part of your result, explain how you will do that.

\medskip
\item[(1.5) Data and data analysis]

	Include all of the data you recorded in a table or plot with appropriate labels and units.  Describe how the data were analyzed to obtain the result(s), and include intermediate calculations or plots as necessary.  If the data analysis involves software, like a Jupyter notebook, say so and state what was calculated with it.  Include any numbers and analysis relevant to uncertainties, if you are asked to calculate them.

\medskip
\item[(1.5) Conclusion]

	State your result, and uncertainty if applicable.  Does it make sense?  If not, what might have gone wrong?  If your result can be compared to a model, state whether it agrees with the prediction.
	
\medskip
\item[(1.0) Questions]

	Answer any questions asked in the lab manual or in the lab report template.
	
\end{description}
\bigskip

Not every lab will fit neatly into these sections.  If you ever have any questions about how to write the report or what should be included, just ask!

\end{document}