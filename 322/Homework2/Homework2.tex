\documentclass[11pt]{article}
\usepackage{geometry, titlesec}
\usepackage[parfill]{parskip}
\usepackage[italicdiff]{physics}
\usepackage{amsfonts, amsthm, mathrsfs}
\usepackage[cm]{fullpage}
\usepackage{fancyhdr}
\usepackage{enumitem}
\usepackage{xcolor, soul}
\usepackage{siunitx}
%\allowdisplaybreaks

\renewcommand{\thesubsection}{\thesection.\alph{subsection}}
\renewcommand{\vec}[1]{\mathbf{#1}}
\newcommand{\vfix}{\vspace{-\baselineskip}}

\makeatletter
\renewcommand*\env@cases[1][1.2]{%
  \let\@ifnextchar\new@ifnextchar
  \left\lbrace
  \def\arraystretch{#1}%
  \array{@{}l@{\quad}l@{}}%
}
\makeatother
 
\renewcommand{\footrulewidth}{.2pt}
%\setlist[enumerate]{leftmargin=*}
\pagestyle{fancy}
\fancyhf{}
\lhead{\textbf{Physics 322 Homework 2}}
\rhead{Lacey Rainbolt}
\setlength{\headheight}{11pt}
\setlength{\headsep}{11pt}
\setlength{\footskip}{24pt}
\lfoot{\today}
\rfoot{\thepage}

\titleformat{\section}[runin]{\normalfont\large\bfseries}{Problem \thesection.}{1em}{}
\titleformat{\subsection}[runin]{\normalfont\large\bfseries}{\thesubsection}{1em}{}
\titleformat{\subparagraph}[leftmargin]{\normalfont\normalsize\bfseries}{}{0pt}{}

\newcommand{\refeq}[1]{(\ref{#1})}

\newcommand{\beq}{\begin{equation*}}
\newcommand{\eeq}{\end{equation*}}

\newcommand{\beqn}{\begin{equation}}
\newcommand{\eeqn}{\end{equation}}

\newcommand{\blg}{\begin{align*}}
\newcommand{\elg}{\end{align*}}


\newenvironment{statement}[1]
{
	\section{#1}
	\color{darkgray}
	\ignorespaces
}
{
%    \smallskip
}

\newenvironment{problem}
{
	\subsection{}
	\color{darkgray}
%	\paragraph{Problem.}
    \ignorespaces
}
{

}

\newenvironment{solution}
{
    \paragraph{Solution.}
    \ignorespaces
}
{
    \bigskip
}


\DeclareSIUnit\esu{esu}
\DeclareSIUnit\erg{erg}

\begin{document}


\state{Spin-wave theory~(P\&S 11.1)}{\hfix}

\prob{ \label{1a}
	Prove the following wonderful formula: Let $\phix$ be a free scalar field with propagator $\ev{T \phix \phio} = \Dx$.  Then
	\eqn{show1}{
		\ev{ T e^{i \phix} e^{-i \phio} } = e^{[ \Dx - \Do ]}.
	}
	(The  factor $\Do$ gives a formally divergent adjustment of the overall normalization.)
}

\sol{
	According to P\&S~(9.18),
	\eq{
		\ev*{T \phi(\xq) \phi(\xw)}{\Omg} = \frac{\int \DDphi \phi(\xq) \phi(\xw) \exp[ i \int \ddqx \cL ]}{\int \DDphi \exp[ i \int \ddqx \cL ]}.
	}
	We use this expression to write the left-hand side of Eq.~\refeq{show1}:
	\eqn{thing1}{
		\ev{ T e^{i \phix} e^{-i \phio} } = \frac{\int \DDphi e^{i \phix} e^{-i \phio} \exp[ i \int \ddqy \cL ]}{\int \DDphi \exp[ i \int \ddqy \cL ]}
		= \frac{\int \DDphi \exp[i \phix - i \phio + i \int \ddqy \cL ]}{\int \DDphi \exp[ i \int \ddqy \cL ]}.
	}
	For a free Klein-Gordon~(i.e., scalar) field, Eq.~(9.39) tells us that the generating functional $\ZJ$ is given by
	\eq{
		\ZJ = \Zo \exp[ -\frac{1}{2} \int \ddqx \ddqy \Jx \DF(x - y) \Jy ],
	}
	where $\Zo = Z[0]$.  Thus, we want to find some $\Jy$ such that
	\eqn{thing1b}{
		\ev{ T e^{i \phix} e^{-i \phio} } = \frac{\ZJ}{\Zo}
	}
	where in general
	\eq{
		\ZJ = \int \DDphi \exp[ i \int \ddqx [ \cL + \Jx \phi(x) ] ]
	}
	by (9.34).  Inspecting Eq.~\refeq{thing1}, we recognize the denominator as $\Zo$ and see that if
	\eq{
		\Jy = \delq(y - x) - \delq(y)
	}
	we have an expression like Eq.~\refeq{thing1b}.  Collecting these findings, we have
	\al{
		\ans{ \ev{ T e^{i \phix} e^{-i \phio} } }&= \frac{\ZJ}{\Zo} \\
		&= \exp[ -\frac{1}{2} \int \ddqy \ddqz \Jy \DF(y - z) \Jz ] \\
		&= \exp[ -\frac{1}{2} \int \ddqy \ddqz \Jy \DF(y - z) [ \delq(z - x) - \delq(z) ] ] \\
		&= \exp[ -\frac{1}{2} \int \ddqy [ \delq(y - x) - \delq(y) ] [ \DF(y - x) - \DF(y) ] ] \\
		&= \exp[ -\frac{1}{2} [ \DF(0) - \DF(x) - \DF(-x) + \DF(0) ] ] \\
		&= \exp[ \DF(x) - \DF(0) ] \\
		&\ans{\; = e^{[ \Dx - \Do ]}, }
	}
	as we wanted to show. \qed
}



\prob{ \label{1b}
	We can use this formula in Euclidean field theory to discuss correlation functions in a theory with spontaneously broken symmetry for $T < \TC$.  Let us consider only the simplest case of a broken $O(2)$ or $U(1)$ symmetry.  We can write the local spin density as a complex variable
	\eq{
		\sx = \sqx + i \swx.
	}
	The global symmetry is the transformation
	\eq{
		\sx \to e^{-i \alp} \sx.
	}
	If we assume that the physics freezes the modulus of $\sx$, we can parameterize
	\eqn{sx}{
		\sx = A e^{i \phix}
	}
	and write an effective Lagrangian for the field $\phix$.  The symmetry of the theory becomes the translation symmetry
	\eqn{symmetry}{
		\phix \to \phix - \alp.
	}
	Show that (for $d > 0$) the most general renormalizable Lagrangian consistent with this symmetry is the free field theory
	\eqn{show1b}{
		\cL = \frac{1}{2} \rho(\vgrad \phi)^2.
	}
	In statistical mechanics, the constant $\rho$ is called the \emph{spin wave modulus}.  A reasonable hypothesis for $\rho$ is that it is finite for $T < \TC$ and tends to 0 as $T \to \TC$ from below.
}

\sol{
	In accordance with the Klein-Gordon Lagrangian in P\&S~(2.6),
	\eqn{KGL}{
		\cL_\text{K-G} = \frac{1}{2} (\pt \phi)^2 - \frac{1}{2} m^2 \phi^2,
	}
	we interpret $(\vgrad \phi)^2$ as $(\pt \phi)^2$.
	
	The Lagrangian cannot have terms of $\order{\phi^n}$ for any $n \neq 0$ since $\phi(x)$ is not invariant under Eq.~\refeq{symmetry}.  Any combination of derivatives of $\phi$ is invariant, however, since $\alp$ is a constant and does not contribute to any derivative.  Thus, only terms like $\pt^n \phi^m$ (where $n$ denotes a power of $\pt$) for $n, m > 0$ and $n \geq m$ are consistent with the symmetry of Eq.~\refeq{symmetry} for $d$ an integer.
	
	Now we must determine which of these terms are renormalizable.  We know that the Lagrangian must have dimension $d$, and that $\phi$ has dimension $(d - 2) / 2$.  Taking a derivative adds a mass dimension.  The theory is renormalizable if the coupling constant $\rho$ has dimension greater than or equal to 0~\cite[p.~322]{Peskin}.  Let $p$ be the dimension of $\rho$.  The dimension of our allowed term is then
	\eq{
		[ \rho \pt^n \phi^m ] = p + n + m \frac{d - 2}{2},
	}
	which we require to be equal to $d$.  Thus we seek solutions to the system of equations
	\al{
		d &= p + n + m \frac{d - 2}{2}, &
		n &\geq m, &
		p &\geq 0.
	}
	Solving with Mathematica, we find that this system has two solutions: $n = m = 2$ and $p = 0$; and $n = m = 1$ and $p = d / 2$.  However, the term $\pt \phi$ for $n = m = 1$ does not contribute to the action because it is a total derivative and does not contribute when the integral over $\cL$ is evaluated:
	\eq{
		\int \dd[d]{x} \pt\phi = \phi \bigg|_{-\infty}^\infty
		= 0.
	}
	Thus the only possibility is $n = m = 2$.  Note that
	\eq{
		\pt^2 \phi^2 = \pt(\pt \phi^2)
		= 2 \pt( \phi \pt \phi)
		= \pt \phi \pt \phi + \phi \pt^2 \phi
		= (\pt \phi)^2,
	}
	since $\phi \pt^2 \phi$ is not invariant under Eq.~\refeq{sx}.  This means that $\rho$ must be dimensionless and that the only allowed terms in the Lagrangian are proportional to $(\pt \phi)^2$, which is consistent with Eq.~\refeq{show1b}. \qed
}



\prob{
	Compute the correlation function $\ev{ \sx \sao }$.  Adjust $A$ to give a physically sensible normalization (assuming that the system has a physical cutoff at the scale of one atomic spacing) and display the dependence of this correlation function on $x$ for $d = 1, 2, 3, 4$.  Explain the significance of your results.
}

\sol{
	Applying Eq.~\refeq{sx},
	\eq{
		\ev{ \sx \sao } = \ev*{ A e^{i \phix} \As e^{-i \phio} }
		= \ev*{ \abs{A}^2 } \ev*{ e^{i \phix} e^{-i \phio} }.
	}
	Now we can apply Eq.~\refeq{show1} to find
	\eqn{thing1c}{
		\ans{ \ev{ \sx \sao } = \abs{A}^2 \exp[ D(x) - D(0) ], }
	}
	where $D(x - y)$ is a Green's function.  Since our Lagrangian is similar to the Klein-Gordon Lagrangian Eq.~\refeq{2.6}, our Green's function is similar to that of the Klein-Gordon operator, which is given by P\&S~(2.56):
	\eq{
		(\pt^2 + m^2) D(x - y) = -i \delq(x - y).
	}
	The Feynman prescription for this Green's function is given by (2.59),
	\eqn{DF}{
		\DF(x - y) = \int \ddqpf \frac{i}{p^2 - m^2 + i \eps} e^{-i p \cdot (x - y)}.
	}
	For the Lagrangian in Eq.~\refeq{show1b}, we set $m = 0$ and insert a factor of $\rho$:
	\eq{
		\rho \pt^2 D(x - y) = -i \deld(x - y),
	}
	so adapting Eq.~\refeq{DF} for this situation yields
	\eqn{DF}{
		\DF(x - y) = \frac{1}{\rho} \int \dddpf \frac{i}{p^2 + i \eps} e^{-i p \cdot (x - y)}.
	}
	We see that $\DF(0)$ diverges, so we absorb it into the constant to make the normalization physically sensible.  We can do this because, as we showed in \ref{1b}, the theory is renormalizable.  Define $A'$ such that
	\eq{
		{A'}^2 = \abs{A}^2 e^{-D(0)}.
	}
	Then Eq.~\refeq{thing1c} can be written
	\eq{
		\ans{ \ev{ \sx \sao } =  {A'}^2 e^{D(x)}. }
	}
	
	To evaluate the divergent integral $D(x)$, we look to the Feynman parameter method we have been using to solve divergent integrals.  Apparently, the Schwinger parametrization is useful in deriving the Feynman parametrization, and it is given by~\cite{Feynman}
	\eq{
		\frac{1}{A} = \intoi \dds e^{-s A}.
	}
	Using this equation, we can write Eq.~\refeq{DF} as
	\eq{
		\DF(x) = \frac{1}{\rho} \int \dddpf \frac{i}{p^2} e^{-i p \cdot x}
		= \frac{i}{\rho} \int \dddpf \intoi \dds e^{-s p^2} e^{-i p \cdot x}.
	}
	Now we can complete the square in the exponential to get a Gaussian integral:
	\al{
		\DF(x) &= \frac{i}{\rho} \int \dddpf \intoi \dds \exp[ -s p^2 - i p \cdot x + \frac{x^2}{4 s} - \frac{x^2}{4 s} ] \\
		&= \frac{i}{\rho} \int \dddpf \intoi \dds \exp[ -s \paren{ p + \frac{i x}{2 s} }^2 - \frac{x^2}{4 s} ] \\
		&= \frac{i}{\rho (2 \pi)^d} \intoi \dds e^{-x^2 / 4 s} \int \dd[d]{u} e^{-s u^2} \\
		&= \frac{i}{\rho (2 \pi)^{d}} \intoi \dds e^{-x^2 / 4 s} \sqrt{ \frac{(2\pi)^d}{(2s)^d} } \\
		&= \frac{i}{\rho (4 \pi)^{d / 2}} \intoi \dds \frac{e^{-x^2 / 4 s}}{s^{d / 2}}
	}
	where we have used~\cite{QFT}
	\eq{
		\int \exp( -\frac{1}{2} x \cdot A \cdot x ) \dd[n]{x} = \sqrt{\frac{(2\pi)^n}{\det A}},
	}
	with $A$ a $d \times d$ diagonal matrix $2s$.  Using Mathematica to integrate with respect to $s$, we find
	\eq{
		\DF(x) = \frac{i}{\rho (4 \pi)^{d / 2}} \frac{2^{d - 2}}{x^{d - 2}} \Gam(d / 2 - 1)
		= \frac{i}{4 \pi^d \rho} \Gam(d / 2 - 1) x^{2 - d}.
	}
	The gamma function diverges as $d \to 2$, so as we have done in previous problems, we expand about $\eps = 2 - d$.  Evaluating the series expansion using Mathematica, we obtain
	\eq{
		\DF(x) = \frac{i}{4 \pi^{1 - \eps} \rho} \Gam(\eps / 2) x^\eps
		\approx \frac{i}{4 \pi \rho} \paren{ \frac{2}{\eps} - \gam + 2 \ln(\pi x) }
		\sim \frac{i}{2 \pi \rho} \ln(x)
		= i \ln(\frac{1}{x^{2 \pi \rho}}).
	}
	We Wick rotate $x \to i x$.  Then the dependence of the correlation function on $x$ for $d = 1, 2, 3, 4$ is
	\ans{\al{
		(d = 1) &\qquad \ev{ \sx \sao } \sim e^{-x / 2 \sqrt{\pi} \rho}, &
		(d = 2) &\qquad \ev{ \sx \sao } \sim x^{2 \pi \rho}, \\
		(d = 3) &\qquad \ev{ \sx \sao } \sim \frac{1}{x}, &
		(d = 4) &\qquad \ev{ \sx \sao } \sim \frac{1}{x^2}.
	}}%
	In $d > 2$ dimensions, the expectation value of the correlation function tends to 0 at large distances $x$.  For $d > 2$, it drops off more quickly as $d$ increases.  The $d \leq 2$ cases depend on $\rho$, which we assume is positive.  The $d = 1$ case drops off with increasing distance, and more quickly with smaller $\rho$.  For $d = 2$, the expectation value of the correlation function increases with increasing distance, and it blows up more quickly with larger $\rho$.
	
	These results are consistent with the Mermin--Wagner theorem, which states that a continuous symmetry cannot be broken in $d \leq 2$ dimensions~\cite{CMW}.  That is, in $d \leq 2$ dimensions, a symmetry-breaking field cannot have a nonzero vacuum expectation value~\cite[p.~460]{Peskin}.  A physical explanation is that each spin has more nearest neighbors in higher dimensions.  Since the spins are inclined to align with their neighbors, there is a higher degree of correlation in higher dimensions at the same distance.  In two dimensions, the correlations are weak enough that they are overpowered by the field fluctuations.
}


\newcommand{\cV}{\mathcal{V}}
\newcommand{\phio}{\phi_0}
\newcommand{\phiox}{\phio(\vx)}
\newcommand{\const}{\text{const.}}
\newcommand{\sig}{\sigma}
\newcommand{\alp}{\alpha}
\newcommand{\sigtv}{\sig(\tht, \vph)}

\newcommand{\sE}{\mathscr{E}}
\newcommand{\dcx}{\dd[3]{x}}
\newcommand{\rhoo}{\rho_0}
\newcommand{\dS}{\dd{S}}
\newcommand{\evS}{|_S}
\newcommand{\evV}{|_\cV}
\newcommand{\intS}{\int_S}
\newcommand{\intV}{\int_\cV}

\begin{statement}{}
	Let $\cV$ be an arbitrary bounded region of space and suppose that a total charge $Q$ is to be distributed in $\cV$ in an arbitrary way, with $\rho = 0$ outside of $\cV$.  Show that the total energy is minimized if the charge is distributed the way that it would be if $\cV$ were a conductor, so that $\phi = \const$ within $\cV$ (and thus, in particular, all of the charge lies on the boundary of $\cV$).
	
	Hint: Let $\phiox$ be the potential one would obtain if $\cV$ were filled by a conducting body.  Consider the energy of $\phio + \phi'$, where the source $\rho'$ of $\phi'$ vanishes outside of $\cV$ and has no net charge within $\cV$.
\end{statement}

\begin{solution}
	Let $S = \partial \cV$ denote the boundary of $\cV$.  We separate space into three mutually exclusive regions: $\cV$, $S$, and the region outside (in which we are not interested).  By the superposition principle, we may write
	\begin{align*}
		\rho &= \rhoo + \rho', &
		\phi &= \phio + \phi',
	\end{align*}
	where $\rhoo$ is the charge of a conducting body filling $\cV$, $\phio$ is the electrostatic potential due to $\rhoo$, $\rho'$ is the charge distribution within $\cV$, and $\phi'$ is the electrostatic potential due to $\rho'$.  In order to eliminate ambiguity on the boundary, we require
	\begin{align} \label{rhoconds}
		\rhoo\evV &= 0, &
		\rho'\evS &= 0.
	\end{align}
	That is, $\rhoo = 0$ inside the conductor by definition, and $\rho'$ vanishes on the boundary where $\rhoo$ is nonzero.  For the entire body to have charge $Q$, we need
	\beq
		\int \rhoo \dcx = \intV \rho' \dcx + \intS \rhoo \dS = Q.
	\eeq
	From \refeq{rhoconds}, it follows that
	\beq
		\phio\evV = \phio\evS = \const
	\eeq

	The total energy is given by Eq.~(2.25) in the course notes,
	\beq
		\sE = \frac{1}{2} \int \phi \rho \dcx.
	\eeq
	So
	\begin{align*}
		\sE &= \frac{1}{2} \int (\phio + \phi') (\rhoo + \rho') \dcx
		= \frac{1}{2} \left( \int \phio (\rhoo + \rho') \dcx + \int \phi' (\rhoo + \rho') \dcx \right) \\
		&= \frac{1}{2} \left( \phio Q + \intV \phi' \rho' \dcx + \intS \phi' \rhoo \dS \right).
	\end{align*}
	\hl{help}
\end{solution}



\newcommand{\lap}{\nabla^2}
\newcommand{\Gd}{G_D}
\newcommand{\Gxxp}{G(\vx, \vx')}
\newcommand{\Gdxxp}{\Gd(\vx, \vx')}
\newcommand{\Gdxpx}{\Gd(\vx', \vx)}
\newcommand{\psixp}{\psi(\vx')}
\newcommand{\nh}{\vec{\hat{n}}}
\newcommand{\rh}{\vec{\hat{r}}}
\newcommand{\nabxp}{\nabla_{\!x'}}
\newcommand{\dSxp}{\dd{S_{x'}}}
\newcommand{\absxp}{\abs{\vx'}}

\begin{statement}{}
	Charge is distributed on a (nonconducting) sphere of radius $R$, i.e., the charge density throughout space is of the form $\rhox = \sig\tv \, \delta(r - R)$.  The surface charge distribution $\sig$ on the sphere is chosen in such a way that the electrostatic potential on the sphere is $\phi(r=R, \tht, \vph) = \alp \cost$, where $\alp$ is a constant.
\end{statement}

\begin{problem}
	Find the electrostatic potential $\phix$ at all $r \leq R$.
\end{problem}

\begin{solution}
	This is a Dirichlet boundary value problem.  We are seeking the solution to Poisson's equation $\lap\phi = -4\pi\rho$ subject to $\phi\evS = \psi = \alp \cost$.  Equation~(2.100) in the lecture notes gives the general solution,
	\beqn \label{sol3}
		\phix = \intV \Gdxxp \, \rhoxp \dcxp - \frac{1}{4\pi} \intS \psixp \, \nh' \cdot \nabxp \Gdxpx \dSxp.
	\eeqn
	The Green's function for a sphere is given by Eq.~(2.91),
	\beq
		\Gdxxp = \frac{1}{\abs{\vx - \vx'}} + \frac{\beta}{\abs{\vx - \vx''}} \qq{where} \vx'' = \vx' \frac{R^2}{\absxp^2} \qand \beta = - \frac{R}{\absxp}.
	\eeq
	The Green's function for unbounded space, $\Gxxp$, can be expanded in spherical harmonics according to Eq.~(2.78):
	\beq
		\Gxxp = \frac{1}{\abs{\vx - \vx'}}
		= \begin{cases} \sum_{l,m} \dfrac{4\pi}{2l + 1} \dfrac{r^l}{{r'}^{l + 1}} \Ylm^*\tvp \, \Ylm\tv & \text{if } r < r', \\
		\sum_{l,m} \dfrac{4\pi}{2l + 1} \dfrac{{r'}^l}{r^{l + 1}} \Ylm^*\tvp \, \Ylm\tv & \text{if } r > r'. \end{cases}
	\eeq
	Expanding $\Gdxxp$ in spherical harmonics in the region $r < r'$ gives us
	\begin{align*}
		\Gdxxp &= \sum_{l,m} \left( \frac{4\pi}{2l + 1} \frac{r^l}{{r'}^{l + 1}} \Ylm^*\tvp \, \Ylm\tv + \beta \frac{4\pi}{2l + 1} \frac{r^l}{{r''}^{l + 1}} \Ylm^*(\tht'', \vph'') \, \Ylm\tv \right) \\
		&= \sum_{l,m} \frac{4\pi}{2l + 1} r^l \, \Ylm\tv \left( \frac{1}{{r'}^{l + 1}} \Ylm^*\tvp + \frac{\beta}{{r''}^{l + 1}} \Ylm^*(\tht'', \vph'') \right) \\
		&= \sum_{l,m} \frac{4\pi}{2l + 1} \frac{r^l}{R} \, \Ylm\tv \left[ \frac{R}{{r'}^{l + 1}} \Ylm^*\tvp - \frac{{r'}^l}{R^{2l}} \Ylm^*\!\left( \frac{R^2 \tht'}{{r'}^2}, \frac{R^2 \vph'}{{r'}^2} \right) \right].
	\end{align*}
	We can also write $\psi$ in terms spherical harmonics:
	\beq
		\psi(r=R, \tht, \vph) = \alp \sqfr{4\pi}{3} Y_{1 0}\tv = \alp \sqfr{4\pi}{3} Y_{1 0}^*\tv.
	\eeq
	
	Since charge is distributed only on the surface of the sphere, the first integral in \refeq{sol3} is 0.  For the second integral, note that $\nh = \rh$ for a sphere.  Making the substitutions from the problem statement, we have
	\beqn \label{thing}
		\phix = -\frac{\alp}{4\pi} \intS \costp \, \rh' \cdot \nabxp \Gdxpx \dSxp
		= -\frac{\alp}{4\pi} \intotp \intono \intoR \costp \pdv{\Gdxpx}{r'} \delta(r' - R) \drp \dctp \dvp.
	\eeqn
	Note that
	\begin{align*}
		\pdv{\Gdxpx}{r'} = \sum_{l,m} 4\pi \frac{l}{2l + 1} \frac{{r'}^{l-1}}{R} \, \Ylm\tvp \left[ \frac{R}{r^{l + 1}} \Ylm^*\tv - \frac{r^l}{R^{2l}} \Ylm^*\!\left( \frac{R^2 \tht}{r^2}, \frac{R^2 \vph}{r^2} \right) \right],
	\end{align*}
	and that \refeq{thing} is only nonzero for the $l = 1$, $m = 0$ term of $\pdv*{\Gdxpx}{r'}$, which is
	\beq
		\frac{4\pi}{3} \frac{1}{R} \, Y_{1 0}\tvp \left[ \frac{R}{r^2} Y_{1 0}^*\tv - \frac{r}{R^2} Y_{1 0}^*\!\left( \frac{R^2 \tht}{r^2}, \frac{R^2 \vph}{r^2} \right) \right]
		= \frac{4\pi}{3} Y_{1 0}\tvp \left[ \frac{1}{r^2} Y_{1 0}^*\tv - \frac{r}{R^3} Y_{1 0}^*\!\left( \frac{R^2 \tht}{r^2}, \frac{R^2 \vph}{r^2} \right) \right].
	\eeq
	So \refeq{thing} becomes
	\beq
		\phix = -\frac{\alp}{3} \sqfr{4\pi}{3} \left[ \frac{1}{r^2} Y_{1 0}^*\tv - \frac{r}{R^3} Y_{1 0}^*\!\left( \frac{R^2 \tht}{r^2}, \frac{R^2 \vph}{r^2} \right) \right] \\
		= \frac{\alp}{3} \left[ \frac{r}{R^3} \cos(\frac{R^2 \tht}{r^2}) - \frac{1}{r^2} \cost \right].
	\eeq
\end{solution}
\vfix


\begin{problem}
	Find the electrostatic potential $\phix$ at all $r \geq R$.
\end{problem}

\begin{solution}
	Expanding $\Gdxxp$ in spherical harmonics in the region $r > r'$ gives us
	\beq
		\Gdxxp = \sum_{l,m} \frac{4\pi}{2l + 1} \frac{1}{r^{l+1} r'} \, \Ylm\tv \left[ {r'}^{l+1} \Ylm^*\tvp - \frac{R^{2l+1}}{{r'}^l} \Ylm^*\!\left( \frac{R^2 \tht'}{{r'}^2}, \frac{R^2 \vph'}{{r'}^2} \right) \right].
	\eeq
	In this region,
	\beq
		\pdv{\Gdxpx}{r'} = \sum_{l,m} -4\pi \frac{l + 1}{2l + 1} \frac{1}{{r'}^{l+2} r} \, \Ylm\tv \left[ r^{l+1} \Ylm^*\tv - \frac{R^{2l+1}}{r^l} \Ylm^*\!\left( \frac{R^2 \tht}{r^2}, \frac{R^2 \vph}{r^2} \right) \right].
	\eeq
	We may once again use \refeq{thing} with the $l = 1$, $m = 0$ term of $\pdv*{\Gdxpx}{r'}$, which is
	\beq
		-\frac{8\pi}{3} \frac{1}{{r'}^3 r} Y_{1 0}\tvp \left[ r^2 Y_{1 0}^*\tv - \frac{R^3}{r} Y_{1 0}^*\!\left( \frac{R^2 \tht}{r^2}, \frac{R^2 \vph}{r^2} \right) \right]
		= -\frac{8\pi}{3} \frac{1}{{r'}^3} Y_{1 0}\tvp \left[ r Y_{1 0}^*\tv - \frac{R^3}{r^2} Y_{1 0}^*\!\left( \frac{R^2 \tht}{r^2}, \frac{R^2 \vph}{r^2} \right) \right].
	\eeq
	Now \refeq{thing} becomes
	\beq
		\phix = -\frac{2\alp}{3} \sqfr{4\pi}{3} \frac{1}{r^3} \left[ r Y_{1 0}^*\tv - \frac{R^3}{r^2} Y_{1 0}^*\!\left( \frac{R^2 \tht}{r^2}, \frac{R^2 \vph}{r^2} \right) \right] \\
		= \frac{2\alp}{3} \left[ \frac{R^3}{r^5} \cos(\frac{R^2 \tht}{r^2}) - \frac{1}{r^2} \cost \right].
	\eeq
\end{solution}
\vfix

\newcommand{\vE}{\vec{E}}
\newcommand{\Er}{E_r}
\newcommand{\thh}{\boldsymbol{\hat{\tht}}}
\newcommand{\phh}{\boldsymbol{\hat{\vph}}}


\begin{problem}
	Find the surface charge density $\sigtv$ that was required in order to produce this potential $\phi$.
\end{problem}

\begin{solution}
	The surface charge density can be found from Eq.~(2.86),
	\beqn \label{scdeq}
		\vE \cdot \nh = 4\pi \sig,
	\eeqn
	where $\vE = -\nabla \phi$ in electrostatics.  The gradient in spherical coordinates is given by
	\beq
		\nabla = \pdv{}{r} \,\rh + \frac{1}{r} \pdv{}{\tht} \,\thh + \frac{1}{r \sint} \pdv{}{\vph} \, \phh.
	\eeq
	Since $\nh = -\rh$ for the inner surface of a sphere, we are interested in only the $r$ component of the field.  We will find this using the solution to (a):
	\beq
		\Er = \frac{\alp}{3} \left[ \frac{1}{R^3} \cos(\frac{R^2 \tht}{r^2}) - \frac{r}{R^3} \frac{2 R^2 \tht}{r^3} \sin(\frac{R^2 \tht}{r^2}) + \frac{2}{r^3} \cost \right]
		= \frac{\alp}{3} \left[ \frac{1}{R^3} \cos(\frac{R^2 \tht}{r^2}) - \frac{2 \tht}{R r^2} \sin(\frac{R^2 \tht}{r^2}) + \frac{2}{r^3} \cost \right].
	\eeq
	  On the surface of the sphere, the field is $\Er(r=R) \,\rh$.  So we have

\end{solution}


\begin{problem}
	Find the total electrostatic energy.
\end{problem}



\state{Degenerate Bose gas}{\hfix}

%
%	4.1
%

\prob{}{
	The chemical potential of the degenerate Bose gas vanishes below $\Ts$ (the critical temperature of the BEC).  Find its temperature dependence at temperatures slightly above $\Ts$.
}

\sol{
	In three dimensions, the energy distribution of a Bose gas is~\cite[p.~149]{Landau}
	\eqn{ddNeps}{
		\ddNeps = \frac{g V}{\pi^2 \hbar^3} \sqrt{\frac{m^3}{2}} \frac{\sqrt{\eps}}{e^{(\eps - \mu) / T} - 1} \ddeps.
	}
	Integrating over all energies, we find the total number of molecules~\cite[p.~149]{Landau}.  This gives an expression relating the chemical potential $\mu$ and the density $\nb$~\cite[p.~159]{Landau}:
	\eqn{nb}{
		\nb = \frac{g}{\pi^2 \hbar^3} \sqrt{\frac{m^3}{2}} \intoi \frac{\sqrt{\eps}}{e^{(\eps - \mu) / T} - 1} \ddeps.
	}
	The critical temperature $\Ts$ satisfies this relation for $\mu = 0$, and can be found by making the substitution $z = \eps / T$:
	\eq{
		\nb = \frac{g}{\pi^2 \hbar^3} \sqrt{\frac{m^3}{2}} \intoi \frac{\sqrt{\eps}}{e^{\eps / T} - 1} \ddeps
		= \frac{g}{\pi^2 \hbar^3} \sqrt{\frac{m^3 T^3}{2}} \intoi \frac{\sqrt{z}}{e^z - 1} \ddeps.
	}
	The integral may be evaluated using the formula~\cite[p.~156]{Landau}
	\eqn{formula}{
		\intoi \frac{z^{x - 1}}{e^z - 1} \ddz = \Gam(x) \zeta(x),
	}
	with $x > 1$.  The relevant values are $\Gam(3/2) = \sqrt{\pi} / 2$, and $\zeta(3/2) = 2.612$~\cite[p.~156]{Landau}.  Thus,
	\eq{
		\nb = \frac{g}{\pi^2 \hbar^3} \sqrt{\frac{m^3 T^3}{2}} (2.612)\frac{\sqrt{\pi}}{2}
		= \frac{g}{\pi^2 \hbar^3} \sqrt{\frac{m^3 T^3}{2}} (2.612)\frac{\sqrt{\pi}}{2}
		= \frac{0.9235 \,g}{\hbar^3} \paren{ \frac{m T}{\pi} }^{3/2},
	}
	and
	\eq{
		\paren{ \frac{m \Ts}{\pi} }^{3/2} = \frac{\nb \hbar^3}{0.9235 \,g}
		\qimplies
		\Ts = \frac{\pi}{m} \paren{ \frac{\nb \hbar^3}{0.9235 \,g} }^{2/3}
		= \frac{1.054\, \pi}{m \hbar^2} \paren{ \frac{\nb}{g} }^{2/3}.
	}
	
	Define the function
	\eq{
		\nbs(T) = \frac{g}{\pi^2 \hbar^3} \sqrt{\frac{m^3}{2}} \intoi \frac{\sqrt{\eps}}{e^{\eps / T} - 1} \ddeps
		= \frac{0.9235 \,g}{\hbar^3} \paren{ \frac{m T}{\pi} }^{3/2},
	}
	and note that $\nbs(\Ts) = \nb$.  Then we can rewrite Eq.~\refeq{nb} as
	\eq{
		\nb = \nbs(T) + \frac{g}{\pi^2 \hbar^3} \sqrt{\frac{m^3}{2}} \intoi \frac{\sqrt{\eps}}{e^{(\eps - \mu) / T} - 1} \ddeps - \nbs(T)
		= \nbs(T) + \frac{g}{\pi^2 \hbar^3} \sqrt{\frac{m^3}{2}} \intoi \paren{ \frac{\sqrt{\eps}}{e^{(\eps - \mu) / T} - 1} - \frac{\sqrt{\eps}}{e^{\eps / T} - 1} } \ddeps.
	}
	Expanding the integrand for small exponential powers using $e^x \approx 1 + x$, which is vaild for $T - \Ts \ll 1$, we find
	\eq{
		\frac{\sqrt{\eps}}{e^{(\eps - \mu) / T} - 1} - \frac{\sqrt{\eps}}{e^{\eps / T} - 1}
		\approx \frac{\sqrt{\eps}}{1 + (\eps - \mu) / T - 1} - \frac{\sqrt{\eps}}{1 + \eps / T - 1}
		= \frac{T \sqrt{\eps}}{\eps - \mu} - \frac{T}{\sqrt{\eps}}
		= \frac{T \eps - T (\eps - \mu)}{\sqrt{\eps} (\eps - \mu)}
		= \frac{T \mu}{\sqrt{\eps} (\eps - \mu)}.
	}
	Then the integral is
	\eq{
		T \mu \intoi \frac{\ddeps}{\sqrt{\eps} (\eps - \mu)} = T \mu \frac{\pi}{\sqrt{-\mu}}
		= \pi T \sqrt{-\mu},
	}
	so long as $\mu < 0$, which is true for the Bose distribution~\cite[p.~145]{Landau}.  Making this substitution and solving for $\mu$, we find
	\eqn{density}{
		\nb = \nbs(T) - \frac{g T}{\pi \hbar^3} \sqrt{\frac{-\mu m^3}{2}}
		\qimplies
		\mu = -\frac{2}{m^3} \paren{ \frac{\pi \hbar^3 [ \nbs(T) - \nb ]}{g T} }^2
		= -\frac{2 \pi^2 \hbar^6 [ \nbs(T) - \nb ]^2}{m^3 g^2 T^2}.
	}
	Note that
	\eq{
		\nbs(T) - \nb = \nb \paren{ \frac{\nbs(T)}{\nb} - 1 }
		= \nb \paren{ \frac{\nbs(T)}{\nbs(\Ts)} - 1 }
		= \nb \paren{ \frac{T^{3/2}}{{\Ts}^{3/2}} - 1 },
	}
	since $\nbs(\Ts) = \nb$.  Then the relationship between chemical potential and temperature is
	\eqn{mu}{
		\mu = -\frac{2 \pi^2 \hbar^6 \nb^2}{m^3 g^2 T^2} \paren{ \frac{T^{3/2}}{{\Ts}^{3/2}} - 1 }^2
		= \ans{ -\frac{2 \pi^2 \hbar^6 \nb^2}{m^3 g^2} \paren{ \frac{T^{1/2}}{{\Ts}^{3/2}} - \frac{1}{T} }^2. }
	}
	Since $T / \Ts \approx 1$, the leading behavior is \ans{$\mu \sim -1 / T^2$.}
}

%
%	4.2
%

\prob{}{
	Find the discontinuities in the derivatives of thermodynamic quantities (energy, particle density, entropy, thermodynamic potential, and specific heat) at the BEC transition.  Which order is this phase transition?
}

\sol{
	Using Eq.~\refeq{ddNeps}, the energy of the Bose gas is
	\eq{
		E = \intoi \eps \ddNeps
		= \frac{g V}{\pi^2 \hbar^3} \sqrt{\frac{m^3}{2}} \intoi \frac{\eps^{3/2}}{e^{(\eps - \mu) / T} - 1} \ddeps.
	}
	\clearpage
	The thermodynamic potential for a Bose gas is~\cite[p.~146]{Landau}
	\eq{
		\Omg = T \sumk \ln(1 - e^{(\mu - \epsk) / T}).
	}
	Transforming the sum to an integral as in Prob.~{3.2}, we have~\cite[p.~149]{Landau}
	\al{
		\Omg &= \frac{g V T}{\pi^2 \hbar^3} \sqrt{\frac{m^3}{2}} \intoi \sqrt{\eps} \ln(1 - e^{(\mu - \eps) / T}) \ddeps \\
		&= \frac{g V T}{\pi^2 \hbar^3} \sqrt{\frac{m^3}{2}} \paren{ \brac{ \frac{2}{3} \eps^{3/2} \ln(1 - e^{(\mu - \epsk) / T}) }\oi - \frac{2}{3 T} \intoi \frac{\eps^{3/2}}{e^{(\eps - \mu) / T} - 1} \ddeps } \\
	&= -\frac{3 g V T}{\pi^2 \hbar^3} \paren{ \frac{m}{2} }^{3/2} \intoi \frac{\eps^{3/2}}{e^{(\eps - \mu) / T} - 1} \ddeps
	= -\frac{2}{3} E.
	}
	
	Note that $N = -(\pdv*{\Omg}{\mu})_{T, V}$~\cite[p.~24]{Landau}.  Then~\cite[p.~161]{Landau}
	\eq{
		\nb = -\frac{1}{V} \pdv{\Omg}{\mu}
		= \frac{2}{3 V} \pdv{E}{\mu}
		\approx \nbs,
	}
	since the contribution to $\nb$ is small for $\mu \ll 1$.  This gives us
	\al{
		\Omg &= \Omgs - \nbs V \mu, &
		E &= \Es + \frac{3}{2} \nbs V \mu,
	}
	where $\Omgs$ and $\Es$ are the thermodynamic potential and the energy at $\mu = 0$.  Using Eq.~\refeq{formula},
	\al{
		\Es &= \frac{g V}{\pi^2 \hbar^3} \sqrt{\frac{m^3}{2}} \intoi \frac{\eps^{3/2}}{e^{\eps / T} - 1} \ddeps
		= \frac{g V}{\pi^2 \hbar^3} \sqrt{\frac{m^3 T^5}{2}} \intoi \frac{z^{3/2}}{e^z - 1} \ddz
		= \frac{g V}{\pi^2 \hbar^3} \sqrt{\frac{m^3 T^5}{2}} \Gam(5/2) \zeta(5/2) \\
		&= \frac{0.711 \,g V}{\hbar^3} \sqrt{\frac{m^3 T^5}{\pi^3}}, \\[2ex]
		\Omgs &= -\frac{0.474 \,g V}{\hbar^3} \sqrt{\frac{m^3 T^5}{\pi^3}},
	}
	both of which are continuously differentiable in $T$.  So the discontinuities in the $T$ derivatives of $\Omg$ and $E$ stem from $\mu$, given by Eq.~\refeq{mu}.  Since
	\eq{
		\pdv{\mu}{T} \sim -\pdv{T}(\frac{1}{T^2})
		\propto -\frac{1}{T^3},
	}
	where $T$ is, by definition, slightly larger than $\Ts$, we conclude that
	\ans{
	\al{
		\pdv{\Omg}{T} &\sim \frac{1}{(T - \Ts)^3}, &
		\pdv{E}{T} &\sim -\frac{1}{(T - \Ts)^3},
	}
	which both have infinite discontinuities at $T = \Ts$.
	}
	
	The particle density is given in Eq.~\refeq{density}.  Differentiating with respect to chemical potential, we see that
	\eq{
		\pdv{\nb}{\mu} = \pdv{\mu}(\nbs(T) - \frac{g T}{\pi \hbar^3} \sqrt{\frac{-\mu m^3}{2}})
		\propto \ans{ \frac{1}{\sqrt{-\mu}}, }
	}
	\ans{which diverges as $\mu \to 0$ from the left} (and is negative for real $\mu$).
	
	Entropy can be found by $S = -(\pdv*{\Omg}{T})_{V, \mu}$~\cite[p.~150]{Landau}, and the specific heat by $\Cv = (\pdv*{E}{T})_V$~\cite[p.~165]{Landau}.  Since
	\al{
		S &\sim -\frac{1}{T^3}, &
		\Cv &\sim -\frac{1}{T^3},
	}
	again for small $T - \Ts$,
	\ans{
	\al{
		\pdv{S}{T} &\sim -\pdv{T}(\frac{1}{T^3})
		\sim -\frac{1}{(T - \Ts)^4}, &
		\pdv{\Cv}{T} &\sim -\frac{1}{(T - \Ts)^4},
	}
	which both have infinite discontinuities at $T = \Ts$.
	}
	
	The order of a phase transition is determined by whether the first or the second derivative of the free energy with respect to some thermodynamic quantity is discontinuous~\cite{Wikipedia}.  The free energy can be found by $F - \mu N = \Omg$~\cite[p.~69]{Landau}.  Since $\pdv*{\mu}{T}$ and $\pdv*{\Omg}{T}$ are discontinuous, so is $\pdv*{F}{T}$.  Thus, this is a \ans{ first-order phase transition. }
}

%
%	4.3
%

\prob{}{
	Can the ideal Bose gas condense in spatial dimensions 1 and 2?  Discuss what happens in these cases. 
}

\sol{
	The ideal Bose gas can condense if the equivalent of Eq.~\refeq{nb} can be solved with $\mu = 0$ to obtain an expression for $\Ts$.  The number of quantum states in the interval $\ddp$ is the same as for a Fermi gas, and so is given by Eq.~\refeq{Nstates}~\cite[p.~148]{Landau}.  Transforming this to the number of states in the interval $\ddeps$ by Eq.~\refeq{transform}, we obtain
	\aln{ \label{thing4}
		\frac{g L}{2\pi \hbar} \sqrt{\frac{m}{2}} \frac{1}{\sqrt{\eps}} \ddeps
		&\quad (d = 1), &
		\frac{m g A}{2\pi \hbar^2} \ddeps
		&\quad (d = 2), &
		\frac{g V}{\pi^2 \hbar^3} \sqrt{\frac{m^3}{2}} \eps^{3/2} \ddeps
		&\quad (d = 3).
	}
	Applying the expression for the total number of particles in a Bose gas~\cite[p.~146]{Landau},
	\eq{
		N = \sumk \frac{1}{e^{(\epsk - \mu) / T} - 1},
	}
	replacing the sum by an integral over $p \in (0, \infty)$, and transforming coordinates to $z = \eps / \Ts$ as in Prob.~{4.1}, we obtain
	\al{
		(d = 1) \quad
		\nb &= \frac{g L}{2\pi \hbar} \sqrt{\frac{m}{2}} \intoi \frac{\ddeps}{\sqrt{\eps} (e^{\eps / \Ts} - 1)}
		= \frac{g L}{2\pi \hbar} \sqrt{\frac{m \Ts}{2}} \intoi \frac{\ddz}{\sqrt{z} (e^z - 1)}
		\to \infty, \\[2ex]
		(d = 2) \quad
		\nb &= \frac{m g A}{2\pi \hbar^2} \intoi \frac{\ddeps}{e^{\eps / \Ts} - 1}
		= \frac{m g A \Ts}{2\pi \hbar^2} \intoi \frac{\ddz}{e^z - 1}
		\to \infty.
	}
	Both integrals diverge, making it impossible to solve for $\Ts$ in either case.
	
	However, these integrals will converge in the limit that $z \to \infty$, which is equivalent to $T \to 0$.  In this limit,
		\al{
		(d = 1) \quad
		\limTo \nb &= \frac{g L}{2\pi \hbar} \sqrt{\frac{m \Ts}{2}} \intoi \frac{\ddz}{e^z \sqrt{z}}
		= \frac{g L}{2 \hbar} \sqrt{\frac{m \Ts}{2 \pi}}, \\[2ex]
		(d = 2) \quad
		\limTo \nb &= \frac{m g A \Ts}{2 \pi \hbar^3} \intoi \frac{\ddz}{e^z}
		= \frac{m g A \Ts}{2 \pi \hbar^3}.
	}
	Thus, we conclude that, \ans{it is not possible for the 1D and 2D ideal Bose gases to condense above $T = 0$.}
	
	Referring back to Eq.~\refeq{thing4}, for $d = 1$ the number of states in the interval $\ddeps$ diverges as $\eps \to 0$.  For $d = 2$, the number of states is independent of $\eps$.  For $d = 3$, the number of states approaches 0 as $\eps \to 0$.  It would seem that, in 1D and in 2D, there are many states with very low energy that may be occupied instead of $\eps = 0$, while this is not the case in 3D.  Since the particles are therefore not ``forced'' into the ground state at nonzero temperature, the gas will not condense.
}



\begin{statement}{}
	The ``mean value theorem'' is stated as follows: For any solution $\phi$ to $\lap \phi = 0$, the value of $\phi$ at $\vx$ is equal to the average value of $\phi$ on a sphere of radius $R$ (for any $R$) centered at $\vx$.
\end{statement}

\begin{problem}
	Prove the mean value theorem.  Hint: Apply Green's theorem to $\phi$ and $1/\abs{\vx - \vx'}$ for a suitable choice of region and a suitable choice of $\vx'$.
\end{problem}

\begin{problem}
	Use this result to show that a point charge can never be in stable equilibrium if placed in an electric field $\vE$ that is source free in a neighborhood of the charge---and, indeed, it can be in neutral equilibrium only if $\vE = 0$ in this neighborhood.
\end{problem}

%\vfill
%In addition to the course lecture notes, I consulted Griffiths's \emph{Introduction to Electrodynamics} and David Tong's lecture notes on electrodynamics while writing up these solutions.
\end{document}