\documentclass[11pt]{article}
\usepackage{geometry, titlesec}
\usepackage[parfill]{parskip}
\usepackage[italicdiff]{physics}
\usepackage{amsfonts, amsthm, mathrsfs}
\usepackage[cm]{fullpage}
\usepackage{fancyhdr}
\usepackage{enumitem}
\usepackage{xcolor, soul}
\usepackage{siunitx}
\allowdisplaybreaks

\renewcommand{\thesubsection}{\thesection.\alph{subsection}}
\renewcommand{\vec}[1]{\mathbf{#1}}
\newcommand{\vfix}{\vspace{-\baselineskip}}

\makeatletter
\renewcommand*\env@cases[1][1.2]{%
  \let\@ifnextchar\new@ifnextchar
  \left\lbrace
  \def\arraystretch{#1}%
  \array{@{}l@{\quad}l@{}}%
}
\makeatother
 
\renewcommand{\footrulewidth}{.2pt}
%\setlist[enumerate]{leftmargin=*}
\pagestyle{fancy}
\fancyhf{}
\lhead{\textbf{Physics 322 Homework 2}}
\rhead{Lacey Rainbolt}
\setlength{\headheight}{11pt}
\setlength{\headsep}{11pt}
\setlength{\footskip}{24pt}
\lfoot{\today}
\rfoot{\thepage}

\titleformat{\section}[runin]{\normalfont\large\bfseries}{Problem \thesection.}{1em}{}
\titleformat{\subsection}[runin]{\normalfont\large\bfseries}{\thesubsection}{1em}{}
\titleformat{\subparagraph}[leftmargin]{\normalfont\normalsize\bfseries}{}{0pt}{}

\newcommand{\refeq}[1]{(\ref{#1})}

\newcommand{\beq}{\begin{equation*}}
\newcommand{\eeq}{\end{equation*}}

\newcommand{\beqn}{\begin{equation}}
\newcommand{\eeqn}{\end{equation}}

\newcommand{\blg}{\begin{align*}}
\newcommand{\elg}{\end{align*}}


\newenvironment{statement}[1]
{
	\section{#1}
	\color{darkgray}
	\ignorespaces
}
{
%    \smallskip
}

\newenvironment{problem}
{
	\subsection{}
	\color{darkgray}
%	\paragraph{Problem.}
    \ignorespaces
}
{

}

\newenvironment{solution}
{
    \paragraph{Solution.}
    \ignorespaces
}
{
    \bigskip
}


\DeclareSIUnit\esu{esu}
\DeclareSIUnit\erg{erg}

\begin{document}


\state{(Jackson 9.8)}{\ 
	%\emph{Hint:} The electromagnetic angular momentum density comes from more than the transverse (radiation zone) components of the fields.
}

%
%	Jackson 9.8(a)
%

\prob{}{
	Show that a classical oscillating electric dipole $\vp$ with fields given by
	\aln{ \label{fields1}
		\vH &= \frac{c k^2}{4\pi} (\nh \cross \vp) \frac{e^{i k r}}{r} \paren{ 1 - \frac{1}{i k r} }, &
		\vE &= \frac{1}{4\pi \epso} \curly{ k^2 (\nh \cross \vp) \cross \nh \frac{e^{i k r}}{r} + [ 3 \nh (\nh \vdot \vp) - \vp ] \paren{ \frac{1}{r^3} - \frac{i k}{r^2} } e^{i k r} },
	}
	radiates electromagnetic angular momentum to infinity at the rate
	\eq{
		\dv{\vL}{t} = \frac{k^3}{12 \pi \epso} \Im[ \vp^* \cross \vp ].
	}
	\vfix
}

\sol{
	According to Jackson~(9.20), the time-averaged angular momentum density is
	\eq{
		\vl = \frac{\Re[ \vx \cross (\vE \cross \vHs)}{2 c^2}.
	}
	One of the vector identities on the inside cover of Jackson is $\vaa \cross (\vbb \cross \vcc) = (\vaa \vdot \vcc) \vbb - (\vaa \vdot \vbb) \vcc$, so
	\eqn{l1}{
		\vl = \frac{(\vx \vdot \vHs) \vE - (\vx \vdot \vE) \vHs}{2 c^2}.
	}
	From Eq.~\refeq{fields1}, note that
	\eq{
		\vx \vdot \vHs \propto \vx \vdot (\nh \cross \vps)
		= \vps \vdot (\vx \cross \nh)
		= \vO,
	}
	where we have used the identity $\vaa \vdot (\vbb \cross \vcc) = \vcc \vdot (\vaa \cross \vbb)$ and the fact that $\nh$ points in the $\vx$ direction.  For $\vx \vdot \vE$, note that
	\al{
		\vx \vdot [ (\nh \cross \vp) \cross \nh ] &= -\vx \vdot [ \nh \cross (\nh \cross \vp) ]
		= -\vx \vdot [ (\nh \vdot \vp) \nh - (\nh \vdot \nh) \vp ]
		= -(\nh \vdot \vp) (\vx \vdot \nh) + \vx \vdot \vp \\
		&= -r (\nh \vdot \vp) + \vx \vdot \vp
		= \vx \vdot \vp - \vx \vdot \vp
		= 0, \\[1.5ex]
		\vx \vdot [ 3 \nh (\nh \vdot \vp) - \vp ] &= 3 (\vx \vdot \nh) (\nh \vdot \vp) - \vx \vdot \vp
		= 3r (\nh \vdot \vp) - \vx \vdot \vp
		= 3(\vx \vdot \vp) - \vx \vdot \vp
		= 2(\vx \vdot \vp),
	}
	since $\abs{\vx} = r$ and $\vx = r \,\nh$.  Then
	\eq{
		\vx \vdot \vE = \frac{1}{2\pi \epso} (\vx \vdot \vp) \paren{ \frac{1}{r^3} - \frac{i k}{r^2} } e^{i k r}
		= \frac{1}{2\pi \epso} (\nh \vdot \vp) \paren{ \frac{1}{r^2} - \frac{i k}{r} } e^{i k r}.
	}
	
	With these substitutions, Eq.~\refeq{l1} becomes
	\al{
		\vl &= -\frac{(\vx \vdot \vE) \vHs}{c^2}
		= -\frac{1}{4\pi \epso c^2} (\nh \vdot \vp) \paren{ \frac{1}{r^2} - \frac{i k}{r} } e^{i k r} \frac{c k^2}{4\pi} (\nh \cross \vps) \frac{e^{-i k r}}{r} \paren{ 1 + \frac{1}{i k r} } \\
		&= -\frac{k^2}{16\pi^2 \epso c r} (\nh \vdot \vp) (\nh \cross \vps) \paren{ \frac{1}{r^2} - \frac{i k}{r} } \paren{ 1 - \frac{i}{k r} }
		= -\frac{k^2}{16\pi^2 \epso c} (\nh \vdot \vp) (\nh \cross \vps) \paren{ \frac{1}{r^2} - \frac{i}{k r^3} - \frac{i k}{r} - \frac{1}{r^2} } \\
		&= -\frac{i k^2}{16\pi^2 \epso c r} (\nh \vdot \vp) (\nh \cross \vps) \paren{ \frac{1}{k r^3} + \frac{k}{r^2} }
		= \frac{i k^3}{16\pi^2 \epso c r^2} (\nh \vdot \vp) (\nh \cross \vps) \paren{ \frac{1}{k^2 r^2} + 1 }.
	}
	
	Let $\vL$ be the angular momentum radiated to a distance $R$.  Then
	\eq{
		\vL = \int_R \vl(r) \ddcx
		= \intopi \intotp \intoR \vl(r) \,r^2 \sin\tht \ddr \ddphi \dd\tht,
	}
	and the time derivative is
	\aln{
		\dv{\vL}{t} &= \dv{t}(\intopi \intotp \intoR \vl(r) \,r^2 \sin\tht \ddr \ddphi \dd\tht)
		= \dv{r}{t} \dv{r}(\intopi \intotp \intoR \vl(r) \,r^2 \sin\tht \ddr \ddphi \dd\tht) \notag \\
		&= c \intopi \intotp \vl(r) \,r^2 \sin\tht \ddphi \dd\tht
		= \frac{i k^3}{16\pi^2 \epso} \paren{ \frac{1}{k^2 r^2} + 1 } \intopi \intotp (\nh \vdot \vp) (\nh \cross \vps) \sin\tht \ddphi \dd\tht. \label{dLdt}
	}
	Note that
	\eq{
		[ (\nh \vdot \vp) (\nh \cross \vps) ]_i = \sumje n_j p_j (\nh \cross \vps)_i
		= \sumje \sumke \sumle \epsikl n_j p_j n_k p_l^*,
	}
	so
	\eq{
		\dv{L_i}{t} \propto \sumje \sumke \sumle \epsikl p_j p_l^* \int n_j p_k \ddOmg
		= \sumje \sumke \sumle \epsikl p_j p_l^* \frac{4\pi}{3} \del_{jk}
		= \frac{4\pi}{3} \epsikl p_k p_l^*
		= \frac{4\pi}{3} (\vp \cross \vps)_i,
	}
	where we have used Jackson~(9.47), $\int n_\bet n_\gam \ddOmg = 4\pi \del_{\bet \gam} / 3$.  Making this substitution into Eq.~\refeq{dLdt},
	\eq{
		\dv{\vL}{t} = \frac{i k^3}{6\pi \epso} \paren{ \frac{1}{k^2 r^2} + 1 } (\vp \cross \vps).
	}
	Taking the limit as $r \to \infty$, we find
	\eqn{ans1a}{
		\dv{\vL}{t} = \Re\!\brac{ \frac{i k^3}{12\pi \epso} (\vp \cross \vps) }
		= \Re\!\brac{ -\frac{i k^3}{12\pi \epso} (\vps \cross \vp) }
		= \ans{ \frac{k^3}{12\pi \epso} \Im[ \vps \cross \vp ], }
	}
	as desired. \qed
}

%
%	Jackson 9.8(b)
%

\prob{}{
	What is the ratio of angular momentum radiated to energy radiated?  Interpret.
}

\sol{
	According to Jackson~(9.24), the total power radiated by an oscillating electric dipole $\vp$ is
	\eq{
		P = \dv{E}{t}
		= \frac{c^2 \Zo k^4}{12 \pi} \abs{\vp}^2.
	}
	Then the ratio of angular momentum radiated to energy radiated is
	\eq{
		\frac{\dv*{\vL}{t}}{\dv*{E}{t}} = \frac{k^3}{12\pi \epso} \Im[ \vps \cross \vp ] \frac{12 \pi}{c^2 \Zo k^4 \abs{\vp}^2}
		= \frac{1}{\epso} \Im[ \vps \cross \vp ] \frac{1}{c^2 \Zo k \abs{\vp}^2}
		= \ans{ \frac{\Im[ \vps \cross \vp ]}{\omg \abs{\vp}^2}, }
	}
	where we have used $\Zo = \sqrt{\muo / \epso} = 1 / \sqrt{\epso^2 c^2} = 1 / \epso c$, $c^2 = 1 / (\epso \muo)$, and $\omg = k c$.
	
	In the limit of high frequency, $(\dv*{\vL}{t}) / (\dv*{E}{t}) \to 0$.  In this scenario, the energy radiated dominates over the angular momentum radiated.  Likewise, in the limit of low frequency, $(\dv*{\vL}{t}) / (\dv*{E}{t}) \to \infty$, meaning that angular momentum radiation dominates.  This is sensible because rotational kinetic energy $E \propto \omg^2$, while angular momentum $L \propto \omg$.
}

%
%	Jackson 9.8(c)
%

\prob{}{
	For a charge $e$ rotating in the $xy$ plane at radius $a$ and angular speed $\omg$, show that there is only a $z$ component of radiated angular momentum with magnitude $\dv*{\Lz}{t} = e^2 k^3 a^2 / 6 \pi \epso$.  What about a charge oscillating along the $z$ axis?
}

\sol{
	We know from Homework~5 that the position of a point charge rotating counterclockwise in the $xy$ plane is
	\eq{
		\vx(t) = a \cos(\omg t) \,\vx + a \sin(\omg t) \,\yh.
	}
	\clearpage
	Then the charge distribution is
	\eq{
		\rho(\vx, t) = e \del[ x - a \cos(\omg t) ] \,\del[ y - a \sin(\omg t) ] \,\del(z).
	}
	
	According to Jackson~(4.8), the dipole moment is defined
	\eq{
		\vp = \int \vx' \,\rho(\vx') \ddcxp.
	}
	The components of $\vp$ for the point charge are then
	\al{
		\px &= e \iiint x \,\del[ x - a \cos(\omg t) ] \,\del[ y - a \sin(\omg t) ] \,\del(z) \ddx \ddy \ddz
		= e a \cos(\omg t), \\
		\py &= e \iiint y \,\del[ x - a \cos(\omg t) ] \,\del[ y - a \sin(\omg t) ] \,\del(z) \ddx \ddy \ddz
		= e a \sin(\omg t), \\
		\pz &= e \iiint z \,\del[ x - a \cos(\omg t) ] \,\del[ y - a \sin(\omg t) ] \,\del(z) \ddx \ddy \ddz
		= 0,
	}
	so we can write $\vp = e a \,e^{-i \omg t} (\xh + i\,\yh).$  Substituting into Eq.~\refeq{ans1a},
	\al{
		\dv{\vL}{t} &= \Re\!\brac{ \frac{i k^3}{12\pi \epso} e^2 a^2 e^{-i \omg t} e^{i \omg t} [ (\xh + i\,\yh) \cross (\xh - i\,\yh) ] }
		= \Re\!\brac{ \frac{i e^2 k^3 a^2}{12\pi \epso} (-2i \,\xh \cross \yh) }
		= \Re\!\brac{ \frac{e^2 k^3 a^2}{6\pi \epso} \,\zh } \\
		&= \ans{ \frac{e^2 k^3 a^2}{6\pi \epso} \cos(\omg t) \,\zh, }
	}
	as desired. \qed
	
	A charge oscillating along the $z$ axis with amplitude $a$ has the charge density
	\eq{
		\rho(\vx, t) = e a \,\del(x) \,\del(y) \,\del[ z - \cos(\omg t) ],
	}
	which gives the dipole moment
	\al{
		\px &= e a \iiint x \,\del(x) \,\del(y) \,\del[ z - \cos(\omg t) ] \ddx \ddy \ddz
		= 0, \\
		\py &= e a \iiint y \,\del(x) \,\del(y) \,\del[ z - \cos(\omg t) ] \ddx \ddy \ddz
		= 0, \\
		\pz &= e a \iiint z \,\del(x) \,\del(y) \,\del[ z - \cos(\omg t) ] \ddx \ddy \ddz
		= e a \cos(\omg t).
	}
	In complex notation, $\vp = e a \,e^{-i\omg t} \,\zh$.  Substituting into Eq.~\refeq{ans1a}, we find
	\eq{
		\dv{\vL}{t} = \Re\!\brac{ \frac{i k^3}{12\pi \epso} e^2 a^2 e^{-i \omg t} e^{i \omg t} (\zh \cross \zh) }
		= \ans{ \vO. }
	}
	So we see that a charge undergoing linear motion does not lead to a radiated angular momentum, which is sensible.
}

%
%	Jackson 9.8(d)
%

\prob{}{
	What are the results corresponding to Probs.~{1(a)} and {1(b)} for magnetic dipole radiation?
}

\sol{
	The radiation fields for a magnetic dipole are given by Jackson~(19.35--36),
	\al{
		\vH &= \frac{1}{4\pi} \curly{ k^2 (\nh \cross \vm) \cross \nh \frac{e^{i k r}}{r} + [ 3 \nh (\nh \vdot \vm) - \vm ] \paren{ \frac{1}{r^3} - \frac{i k}{r^2} } e^{i k r} }, &
		\vE &= -\frac{\Zo}{4\pi} k^2 (\nh \cross \vm) \frac{e^{i k r}}{r} \paren{ 1 - \frac{1}{i k r} }.
	}
	\clearpage
	Comparing with Eq.~\refeq{fields1}, we see that $\vH \to -\vE / \Zo$, $\vE \to \Zo \vH$, and $\vp \to \vm / c$ as stated in the book~\cite[p.~413]{Jackson}.  Making these substitutions, the results of Probs.~{1.1(a)} and {(b)} become
	\al{
		\ans{ \dv{\vL}{t}\ }&\ans{= \frac{\muo k^3}{12\pi} \Im[ \vms \cross \vm ], } &
		\ans{ \frac{\dv*{\vL}{t}}{\dv*{E}{t}}\ }&\ans{= \frac{\Im[ \vms \cross \vm ]}{\omg \abs{\vm}^2} }
	}
	where we have used $\mu = 1 / \epso c^2$.
}


\newcommand{\cV}{\mathcal{V}}
\newcommand{\phio}{\phi_0}
\newcommand{\phiox}{\phio(\vx)}
\newcommand{\const}{\text{const.}}
\newcommand{\sig}{\sigma}
\newcommand{\alp}{\alpha}
\newcommand{\sigtv}{\sig(\tht, \vph)}

\newcommand{\sE}{\mathscr{E}}
\newcommand{\dcx}{\dd[3]{x}}
\newcommand{\rhoo}{\rho_0}
\newcommand{\dS}{\dd{S}}
\newcommand{\evS}{|_S}
\newcommand{\evV}{|_\cV}
\newcommand{\intS}{\int_S}
\newcommand{\intV}{\int_\cV}

\begin{statement}{}
	Let $\cV$ be an arbitrary bounded region of space and suppose that a total charge $Q$ is to be distributed in $\cV$ in an arbitrary way, with $\rho = 0$ outside of $\cV$.  Show that the total energy is minimized if the charge is distributed the way that it would be if $\cV$ were a conductor, so that $\phi = \const$ within $\cV$ (and thus, in particular, all of the charge lies on the boundary of $\cV$).
	
	Hint: Let $\phiox$ be the potential one would obtain if $\cV$ were filled by a conducting body.  Consider the energy of $\phio + \phi'$, where the source $\rho'$ of $\phi'$ vanishes outside of $\cV$ and has no net charge within $\cV$.
\end{statement}

\begin{solution}
	Let $S = \partial \cV$ denote the boundary of $\cV$.  We separate space into three mutually exclusive regions: $\cV$, $S$, and the region outside (in which we are not interested).  By the superposition principle, we may write
	\begin{align*}
		\rho &= \rhoo + \rho', &
		\phi &= \phio + \phi',
	\end{align*}
	where $\rhoo$ is the charge of a conducting body filling $\cV$, $\phio$ is the electrostatic potential due to $\rhoo$, $\rho'$ is the charge distribution within $\cV$, and $\phi'$ is the electrostatic potential due to $\rho'$.  In order to eliminate ambiguity on the boundary, we require
	\begin{align} \label{rhoconds}
		\rhoo\evV &= 0, &
		\rho'\evS &= 0.
	\end{align}
	That is, $\rhoo = 0$ inside the conductor by definition, and $\rho'$ vanishes on the boundary where $\rhoo$ is nonzero.  For the entire body to have charge $Q$, we need
	\beq
		\int \rhoo \dcx = \intV \rho' \dcx + \intS \rhoo \dS = Q.
	\eeq
	From \refeq{rhoconds}, it follows that
	\beq
		\phio\evV = \phio\evS = \const
	\eeq

	The total energy is given by Eq.~(2.25) in the course notes,
	\beq
		\sE = \frac{1}{2} \int \phi \rho \dcx.
	\eeq
	So
	\begin{align*}
		\sE &= \frac{1}{2} \int (\phio + \phi') (\rhoo + \rho') \dcx
		= \frac{1}{2} \left( \int \phio (\rhoo + \rho') \dcx + \int \phi' (\rhoo + \rho') \dcx \right) \\
		&= \frac{1}{2} \left( \phio Q + \intV \phi' \rho' \dcx + \intS \phi' \rhoo \dS \right).
	\end{align*}
	\hl{help}
\end{solution}



\newcommand{\lap}{\nabla^2}
\newcommand{\Gd}{G_D}
\newcommand{\Gdxxp}{\Gd(\vx, \vx')}
\newcommand{\Gdxpx}{\Gd(\vx', \vx)}
\newcommand{\psixp}{\psi(\vx')}
\newcommand{\nh}{\vec{\hat{n}}}
\newcommand{\nabxp}{\nabla_{x'}}
\newcommand{\dSxp}{\dd{S_{x'}}}
\newcommand{\absxp}{\abs{\vx'}}

\begin{statement}{}
	Charge is distributed on a (nonconducting) sphere of radius $R$, i.e., the charge density throughout space is of the form $\rhox = \sig(\tht, \vph) \, \delta(r - R)$.  The surface charge distribution $\sig$ on the sphere is chosen in such a way that the electrostatic potential on the sphere is $\phi(r=R, \tht, \vph) = \alp \cost$, where $\alp$ is a constant.
\end{statement}

\begin{problem}
	Find the electrostatic potential $\phix$ at all $r \leq R$.
\end{problem}

\begin{solution}
	This is a Dirichlet boundary value problem.  We are seeking the solution to Poisson's equation $\lap\phi = -4\pi\rho$ subject to $\phi\evS = \psi = \alp \cost$.  Equation~(2.100) in the lecture notes gives the general solution,
	\beq
		\phix = \intV \Gdxxp \, \rhoxp \dcxp - \frac{1}{4\pi} \intS \psixp \, \nh' \cdot \nabxp \Gdxpx \dSxp.
	\eeq
	The Green's function for a sphere is given by Eq.~(2.91),
	\beq
		\Gdxxp = \frac{1}{\abs{\vx - \vx'}} + \frac{\beta}{\abs{\vx - \vx''}} \qq{where} \vx'' = \vx' \frac{R^2}{\absxp^2} \qand \beta = - \frac{R}{\absxp}.
	\eeq
	\hl{Spherical harmonic expansion (2.78)?}
\end{solution}


\begin{problem}
	Find the electrostatic potential $\phix$ at all $r \geq R$.
\end{problem}

\begin{problem}
	Find the surface charge density $\sigtv$ that was required in order to produce this potential $\phi$.
\end{problem}

\begin{problem}
	Find the total electrostatic energy.
\end{problem}



\newcommand{\vF}{\vec{F}}
\newcommand{\vE}{\vec{E}}
\newcommand{\rh}{\vec{\hat{r}}}
\newcommand{\thh}{\boldsymbol{\hat{\tht}}}
\newcommand{\phh}{\boldsymbol{\hat{\vph}}}
\newcommand{\nabx}{\nabla_{\!x}}
\newcommand{\Er}{E_r}

\begin{statement}{}
	A point charge of charge $q$ is placed at point $\vx'$ inside a conducting spherical shell of radius $R$.  There is no net charge on the conductor.  The potential inside the sphere is thus given by $q \, \Gdxxp$, where the explicit formula for $\Gdxxp$ for a spherical cavity is given in the lecture notes.
\end{statement}

\begin{problem}
	Find the surface charge density $\sigtv$ on the conducting shell.
\end{problem}

\begin{solution}
	The Green's function for a spherical cavity is given by Eq.~(2.91),
	\beq
		\Gdxxp = \frac{1}{\abs{\vx - \vx'}} + \frac{\alp}{\abs{\vx - \vx''}} \qq{where} \vx'' = \vx' \frac{R^2}{\absxp^2} \qand \alp = - \frac{R}{\absxp}.
	\eeq
	The surface charge density can be found from Eq.~(2.86),
	\beqn \label{scdeq}
		\vE \cdot \nh = 4\pi \sig,
	\eeqn
	where $\vE = -\nabla \phi$ in electrostatics.
	
	We will begin by finding $\vE$.  We will orient our coordinate system such that $\vx'$ (and consequently $\vx''$) points along the $z$ axis.  Note that
	\beq
		\Gdxxp = \frac{1}{\abs{\vx - \vx'}} - \frac{R}{\absxp \abs{\vx - \dfrac{R^2}{\absxp^2} \vx'}}
		= \frac{1}{\sqrt{\vx^2 - 2 \vx \cdot \vx' + {\vx'}^2}} - \frac{R}{\absxp \sqrt{\vx^2 - 2 \dfrac{R^2}{{\vx'}^2} \vx \cdot \vx' + \dfrac{R^4}{{\vx'}^4} {\vx'}^2}}.
	\eeq
	In spherical coordinates, we have
	\beq
		\Gdxxp = \frac{1}{\sqrt{r^2 - 2 r r' \cost + {r'}^2}} - \frac{R}{r'} \frac{1}{\sqrt{r^2 - 2 R^2 r \cost / r' + R^4 / {r'}^2}},
	\eeq
	where we note that $\tht$ is the angle between $\vx$ and the $z$ axis.  The gradient in spherical coordinates is given by
	\beq
		\nabla = \pdv{}{r} \,\rh + \frac{1}{r} \pdv{}{\tht} \,\thh + \frac{1}{r \sint} \pdv{}{\vph} \, \phh.
	\eeq
	The $r$ component of the electric field inside the conductor is then
	\beq
		\Er(\vx) = -q \pdv{\Gdxxp}{r}
		= q \left( \frac{r - r' \cost}{(r^2 - 2 r r' \cost + {r'}^2)^{3/2}} - \frac{R}{r'} \frac{r - R^2 \cost / r'}{(r^2 - 2 R^2 r \cost / r' + R^4 / {r'}^2)^{3/2}} \right).
	\eeq
	Since $\nh = -\rh$ for the inner surface of a sphere, we are interested in only the $r$ component of the field.  On the surface of the sphere, the field is $\Er(r=R) \,\rh$.  So we have
	\begin{align*}
		\Er(r=R) &= q \left( \frac{R - r' \cost}{(R^2 - 2 R r' \cost + {r'}^2)^{3/2}} - \frac{R}{r'} \frac{R - R^2 \cost / r'}{(R^2 - 2 R^3 \cost / r' + R^4 / {r'}^2)^{3/2}} \right) \\
		&= q \left( \frac{R - r' \cost}{{r'}^3 (R^2 / {r'}^2 - 2 R \cost / r' + 1)^{3/2}} - \frac{R}{r'} \frac{R - R^2 \cost / r'}{R^3 (1 - 2 R \cost / r' + R^2 / {r'}^2)^{3/2}} \right) \\
		&= \frac{q}{r'} \frac{R^3 - R^2 r' \cost - R {r'}^2 + R^2 r' \cost}{R^2 {r'}^2 (R^2 / {r'}^2 - 2 R \cost / r' + 1)^{3/2}}
		= \frac{q}{R {r'}^3} \frac{R^2 - {r'}^2}{(R^2 / {r'}^2 - 2 R \cost / r' + 1)^{3/2}}.
	\end{align*}
	Finally, feeding this into \refeq{scdeq},
	\beq
		\sig = -\frac{\vE \cdot \rh}{4\pi}
		= \frac{q}{4\pi R {r'}^3} \frac{{r'}^2 - R^2}{(R^2 / {r'}^2 - 2 R \cost / r' + 1)^{3/2}}
		= \frac{q}{4\pi R \absxp^3} \frac{\absxp^2 - R^2}{(R^2 / \absxp^2 - 2 R \cost / \absxp + 1)^{3/2}}.
	\eeq
\end{solution}
\vfix


\newcommand{\vEo}{\vE_0}
\newcommand{\del}{\delta}
\newcommand{\Etht}{E_\tht}
\newcommand{\Fr}{F_r}

\begin{problem}
	Find the force $\vF$ that must be exerted on the point charge in order to hold it in place.
\end{problem}

\begin{solution}
	The total force on a charge distribution arises only from the external electric field $\vEo$, and is given by Eq.~(2.42) in the lecture notes:
	\beq
		\vF = \int \rhox \, \vEo(\vx) \dcx.
	\eeq
	We now need the $\tht$ component of the field inside the conductor, which is
	\beq
		\Etht(\vx) = -\frac{q}{r} \pdv{\Gdxxp}{\tht}
		= -q \left( \frac{r' \sint}{(r^2 - 2 r r' \cost + {r'}^2)^{3/2}} - \frac{R^3 \sint}{{r'}^2 (r^2 - 2 R^2 r \cost / r' + R^4 / {r'}^2)^{3/2}} \right).
	\eeq
	The charge density for a point charge located at $\vx'$ is given by $\rhox = q \, \delta(\vx - \vx')$.  Evaluating the integral, we have
	\beq
		\vF = \int q \, \delta(\vx - \vx') \, \vE(\vx) \dcx = q \vE(\vx').
	\eeq
	Recall that we chose $\vx'$ to point along the $z$ axis, so $\tht' = 0$.  The $\tht$ component of $\vF$ is then $0$, and the $r$ component is
	\begin{align*}
		\Fr &= q^2 \left( \frac{r' - r'}{({r'}^2 - 2 {r'}^2 + {r'}^2)^{3/2}} - \frac{R}{r'} \frac{r' - R^2 / r'}{({r'}^2 - 2 R^2 + R^4 / {r'}^2)^{3/2}} \right)
		= q^2 R {r'}^2 \frac{R^2 / r' - r'}{({r'}^4 - 2 R^2 {r'}^2 + R^4)^{3/2}} \\
		&= q^2 R {r'}^2 \frac{(R^2 - {r'}^2) / r'}{(R^2 - {r'}^2)^3}
		= q^2 \frac{R r'}{(R^2 - {r'}^2)^2}.
	\end{align*}
	Since only the $r$ component of $\vF$ is nonzero, it points in the $z$ direction, which we chose to be equivalent to the unit vector $\vx' / \absxp$.  Therefore,
	\beq
		\vF = q^2 \frac{R \absxp}{(R^2 - \absxp^2)^2} \frac{\vx'}{\absxp}
		= q^2 \frac{R}{(R^2 - \absxp^2)^2} \vx'.
	\eeq
\end{solution}




\begin{statement}{}
	The ``mean value theorem'' is stated as follows: For any solution $\phi$ to $\lap \phi = 0$, the value of $\phi$ at $\vx$ is equal to the average value of $\phi$ on a sphere of radius $R$ (for any $R$) centered at $\vx$.
\end{statement}

\begin{problem}
	Prove the mean value theorem.  Hint: Apply Green's theorem to $\phi$ and $1/\abs{\vx - \vx'}$ for a suitable choice of region and a suitable choice of $\vx'$.
\end{problem}

\begin{problem}
	Use this result to show that a point charge can never be in stable equilibrium if placed in an electric field $\vE$ that is source free in a neighborhood of the charge---and, indeed, it can be in neutral equilibrium only if $\vE = 0$ in this neighborhood.
\end{problem}

%\vfill
%In addition to the course lecture notes, I consulted Griffiths's \emph{Introduction to Electrodynamics} and David Tong's lecture notes on electrodynamics while writing up these solutions.
\end{document}