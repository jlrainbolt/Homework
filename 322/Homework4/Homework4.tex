\documentclass[11pt]{article}
\usepackage{geometry, titlesec}
\usepackage[parfill]{parskip}
\usepackage[italicdiff]{physics}
\usepackage{amsfonts, amsthm, mathrsfs}
\usepackage[cm]{fullpage}
\usepackage{fancyhdr}
\usepackage{enumitem}
\usepackage{xcolor, soul}
\usepackage{siunitx}
\allowdisplaybreaks

\renewcommand{\thesubsection}{\thesection.\alph{subsection}}
%\renewcommand{\vb}[1]{\mathbf{#1}}
\newcommand{\vfix}{\vspace{-\baselineskip}}

\makeatletter
\renewcommand*\env@cases[1][1.2]{%
  \let\@ifnextchar\new@ifnextchar
  \left\lbrace
  \def\arraystretch{#1}%
  \array{@{}l@{\quad}l@{}}%
}
\makeatother
 
\renewcommand{\footrulewidth}{.2pt}
%\setlist[enumerate]{leftmargin=*}
\pagestyle{fancy}
\fancyhf{}
\lhead{\textbf{Physics 322 Homework 4}}
\rhead{Lacey Rainbolt}
\setlength{\headheight}{11pt}
\setlength{\headsep}{11pt}
\setlength{\footskip}{24pt}
\lfoot{\today}
\rfoot{\thepage}

\titleformat{\section}[runin]{\normalfont\large\bfseries}{Problem \thesection.}{1em}{}
\titleformat{\subsection}[runin]{\normalfont\large\bfseries}{\thesubsection}{1em}{}
\titleformat{\subparagraph}[leftmargin]{\normalfont\normalsize\bfseries}{}{0pt}{}

\newcommand{\refeq}[1]{(\ref{#1})}

\newcommand{\beq}{\begin{equation*}}
\newcommand{\eeq}{\end{equation*}}

\newcommand{\beqn}{\begin{equation}}
\newcommand{\eeqn}{\end{equation}}

\newcommand{\blg}{\begin{align*}}
\newcommand{\elg}{\end{align*}}


\newenvironment{statement}[1]
{
	\section{#1}
	\color{darkgray}
	\ignorespaces
}
{
%    \smallskip
}

\newenvironment{problem}
{
	\subsection{}
	\color{darkgray}
%	\paragraph{Problem.}
    \ignorespaces
}
{

}

\newenvironment{solution}
{
    \paragraph{Solution.}
    \ignorespaces
}
{
    \bigskip
}



\begin{document}

\newcommand{\alp}{\alpha}
\newcommand{\tht}{\theta}
\newcommand{\vph}{\varphi}
\newcommand{\omg}{\omega}

\newcommand{\vA}{\vb{A}}
\newcommand{\vB}{\vb{B}}
\newcommand{\vE}{\vb{E}}
\newcommand{\vJ}{\vb{J}}
\newcommand{\vP}{\vb{P}}
\newcommand{\vv}{\vb{v}}
\newcommand{\vx}{\vb{x}}
\newcommand{\vomg}{\boldsymbol{\omg}}
\newcommand{\rh}{\vb{\hat{r}}}
\newcommand{\phh}{\boldsymbol{\hat{\vph}}}
\newcommand{\thh}{\boldsymbol{\hat{\tht}}}

\newcommand{\vAx}{\vA(\vx)}
\newcommand{\vBx}{\vB(\vx)}
\newcommand{\vJx}{\vJ(\vx)}
\newcommand{\vvx}{\vv(\vx)}

\newcommand{\rhox}{\rho(\vx)}
\newcommand{\absx}{\abs{\vx}}
\newcommand{\dcxp}{\dd[3]{\vx'}}

\newcommand{\xh}{\vb{\hat{x}}}
\newcommand{\yh}{\vb{\hat{y}}}
\newcommand{\zh}{\vb{\hat{z}}}
\newcommand{\omgh}{\boldsymbol{\hat{\omg}}}

\newcommand{\intotp}{\int_0^{2\pi}}
\newcommand{\intono}{\int_{-1}^1}
\newcommand{\intoi}{\int_0^\infty}
\newcommand{\drp}{\dd{r'}}
\newcommand{\dctp}{\dd{(\cos\tht')}}
\newcommand{\dvp}{\dd{\vph}}

\state{(Jackson 9.8)}{\ 
	%\emph{Hint:} The electromagnetic angular momentum density comes from more than the transverse (radiation zone) components of the fields.
}

%
%	Jackson 9.8(a)
%

\prob{}{
	Show that a classical oscillating electric dipole $\vp$ with fields given by
	\aln{ \label{fields1}
		\vH &= \frac{c k^2}{4\pi} (\nh \cross \vp) \frac{e^{i k r}}{r} \paren{ 1 - \frac{1}{i k r} }, &
		\vE &= \frac{1}{4\pi \epso} \curly{ k^2 (\nh \cross \vp) \cross \nh \frac{e^{i k r}}{r} + [ 3 \nh (\nh \vdot \vp) - \vp ] \paren{ \frac{1}{r^3} - \frac{i k}{r^2} } e^{i k r} },
	}
	radiates electromagnetic angular momentum to infinity at the rate
	\eq{
		\dv{\vL}{t} = \frac{k^3}{12 \pi \epso} \Im[ \vp^* \cross \vp ].
	}
	\vfix
}

\sol{
	According to Jackson~(9.20), the time-averaged angular momentum density is
	\eq{
		\vl = \frac{\Re[ \vx \cross (\vE \cross \vHs)}{2 c^2}.
	}
	One of the vector identities on the inside cover of Jackson is $\vaa \cross (\vbb \cross \vcc) = (\vaa \vdot \vcc) \vbb - (\vaa \vdot \vbb) \vcc$, so
	\eqn{l1}{
		\vl = \frac{(\vx \vdot \vHs) \vE - (\vx \vdot \vE) \vHs}{2 c^2}.
	}
	From Eq.~\refeq{fields1}, note that
	\eq{
		\vx \vdot \vHs \propto \vx \vdot (\nh \cross \vps)
		= \vps \vdot (\vx \cross \nh)
		= \vO,
	}
	where we have used the identity $\vaa \vdot (\vbb \cross \vcc) = \vcc \vdot (\vaa \cross \vbb)$ and the fact that $\nh$ points in the $\vx$ direction.  For $\vx \vdot \vE$, note that
	\al{
		\vx \vdot [ (\nh \cross \vp) \cross \nh ] &= -\vx \vdot [ \nh \cross (\nh \cross \vp) ]
		= -\vx \vdot [ (\nh \vdot \vp) \nh - (\nh \vdot \nh) \vp ]
		= -(\nh \vdot \vp) (\vx \vdot \nh) + \vx \vdot \vp \\
		&= -r (\nh \vdot \vp) + \vx \vdot \vp
		= \vx \vdot \vp - \vx \vdot \vp
		= 0, \\[1.5ex]
		\vx \vdot [ 3 \nh (\nh \vdot \vp) - \vp ] &= 3 (\vx \vdot \nh) (\nh \vdot \vp) - \vx \vdot \vp
		= 3r (\nh \vdot \vp) - \vx \vdot \vp
		= 3(\vx \vdot \vp) - \vx \vdot \vp
		= 2(\vx \vdot \vp),
	}
	since $\abs{\vx} = r$ and $\vx = r \,\nh$.  Then
	\eq{
		\vx \vdot \vE = \frac{1}{2\pi \epso} (\vx \vdot \vp) \paren{ \frac{1}{r^3} - \frac{i k}{r^2} } e^{i k r}
		= \frac{1}{2\pi \epso} (\nh \vdot \vp) \paren{ \frac{1}{r^2} - \frac{i k}{r} } e^{i k r}.
	}
	
	With these substitutions, Eq.~\refeq{l1} becomes
	\al{
		\vl &= -\frac{(\vx \vdot \vE) \vHs}{c^2}
		= -\frac{1}{4\pi \epso c^2} (\nh \vdot \vp) \paren{ \frac{1}{r^2} - \frac{i k}{r} } e^{i k r} \frac{c k^2}{4\pi} (\nh \cross \vps) \frac{e^{-i k r}}{r} \paren{ 1 + \frac{1}{i k r} } \\
		&= -\frac{k^2}{16\pi^2 \epso c r} (\nh \vdot \vp) (\nh \cross \vps) \paren{ \frac{1}{r^2} - \frac{i k}{r} } \paren{ 1 - \frac{i}{k r} }
		= -\frac{k^2}{16\pi^2 \epso c} (\nh \vdot \vp) (\nh \cross \vps) \paren{ \frac{1}{r^2} - \frac{i}{k r^3} - \frac{i k}{r} - \frac{1}{r^2} } \\
		&= -\frac{i k^2}{16\pi^2 \epso c r} (\nh \vdot \vp) (\nh \cross \vps) \paren{ \frac{1}{k r^3} + \frac{k}{r^2} }
		= \frac{i k^3}{16\pi^2 \epso c r^2} (\nh \vdot \vp) (\nh \cross \vps) \paren{ \frac{1}{k^2 r^2} + 1 }.
	}
	
	Let $\vL$ be the angular momentum radiated to a distance $R$.  Then
	\eq{
		\vL = \int_R \vl(r) \ddcx
		= \intopi \intotp \intoR \vl(r) \,r^2 \sin\tht \ddr \ddphi \dd\tht,
	}
	and the time derivative is
	\aln{
		\dv{\vL}{t} &= \dv{t}(\intopi \intotp \intoR \vl(r) \,r^2 \sin\tht \ddr \ddphi \dd\tht)
		= \dv{r}{t} \dv{r}(\intopi \intotp \intoR \vl(r) \,r^2 \sin\tht \ddr \ddphi \dd\tht) \notag \\
		&= c \intopi \intotp \vl(r) \,r^2 \sin\tht \ddphi \dd\tht
		= \frac{i k^3}{16\pi^2 \epso} \paren{ \frac{1}{k^2 r^2} + 1 } \intopi \intotp (\nh \vdot \vp) (\nh \cross \vps) \sin\tht \ddphi \dd\tht. \label{dLdt}
	}
	Note that
	\eq{
		[ (\nh \vdot \vp) (\nh \cross \vps) ]_i = \sumje n_j p_j (\nh \cross \vps)_i
		= \sumje \sumke \sumle \epsikl n_j p_j n_k p_l^*,
	}
	so
	\eq{
		\dv{L_i}{t} \propto \sumje \sumke \sumle \epsikl p_j p_l^* \int n_j p_k \ddOmg
		= \sumje \sumke \sumle \epsikl p_j p_l^* \frac{4\pi}{3} \del_{jk}
		= \frac{4\pi}{3} \epsikl p_k p_l^*
		= \frac{4\pi}{3} (\vp \cross \vps)_i,
	}
	where we have used Jackson~(9.47), $\int n_\bet n_\gam \ddOmg = 4\pi \del_{\bet \gam} / 3$.  Making this substitution into Eq.~\refeq{dLdt},
	\eq{
		\dv{\vL}{t} = \frac{i k^3}{6\pi \epso} \paren{ \frac{1}{k^2 r^2} + 1 } (\vp \cross \vps).
	}
	Taking the limit as $r \to \infty$, we find
	\eqn{ans1a}{
		\dv{\vL}{t} = \Re\!\brac{ \frac{i k^3}{12\pi \epso} (\vp \cross \vps) }
		= \Re\!\brac{ -\frac{i k^3}{12\pi \epso} (\vps \cross \vp) }
		= \ans{ \frac{k^3}{12\pi \epso} \Im[ \vps \cross \vp ], }
	}
	as desired. \qed
}

%
%	Jackson 9.8(b)
%

\prob{}{
	What is the ratio of angular momentum radiated to energy radiated?  Interpret.
}

\sol{
	According to Jackson~(9.24), the total power radiated by an oscillating electric dipole $\vp$ is
	\eq{
		P = \dv{E}{t}
		= \frac{c^2 \Zo k^4}{12 \pi} \abs{\vp}^2.
	}
	Then the ratio of angular momentum radiated to energy radiated is
	\eq{
		\frac{\dv*{\vL}{t}}{\dv*{E}{t}} = \frac{k^3}{12\pi \epso} \Im[ \vps \cross \vp ] \frac{12 \pi}{c^2 \Zo k^4 \abs{\vp}^2}
		= \frac{1}{\epso} \Im[ \vps \cross \vp ] \frac{1}{c^2 \Zo k \abs{\vp}^2}
		= \ans{ \frac{\Im[ \vps \cross \vp ]}{\omg \abs{\vp}^2}, }
	}
	where we have used $\Zo = \sqrt{\muo / \epso} = 1 / \sqrt{\epso^2 c^2} = 1 / \epso c$, $c^2 = 1 / (\epso \muo)$, and $\omg = k c$.
	
	In the limit of high frequency, $(\dv*{\vL}{t}) / (\dv*{E}{t}) \to 0$.  In this scenario, the energy radiated dominates over the angular momentum radiated.  Likewise, in the limit of low frequency, $(\dv*{\vL}{t}) / (\dv*{E}{t}) \to \infty$, meaning that angular momentum radiation dominates.  This is sensible because rotational kinetic energy $E \propto \omg^2$, while angular momentum $L \propto \omg$.
}

%
%	Jackson 9.8(c)
%

\prob{}{
	For a charge $e$ rotating in the $xy$ plane at radius $a$ and angular speed $\omg$, show that there is only a $z$ component of radiated angular momentum with magnitude $\dv*{\Lz}{t} = e^2 k^3 a^2 / 6 \pi \epso$.  What about a charge oscillating along the $z$ axis?
}

\sol{
	We know from Homework~5 that the position of a point charge rotating counterclockwise in the $xy$ plane is
	\eq{
		\vx(t) = a \cos(\omg t) \,\vx + a \sin(\omg t) \,\yh.
	}
	\clearpage
	Then the charge distribution is
	\eq{
		\rho(\vx, t) = e \del[ x - a \cos(\omg t) ] \,\del[ y - a \sin(\omg t) ] \,\del(z).
	}
	
	According to Jackson~(4.8), the dipole moment is defined
	\eq{
		\vp = \int \vx' \,\rho(\vx') \ddcxp.
	}
	The components of $\vp$ for the point charge are then
	\al{
		\px &= e \iiint x \,\del[ x - a \cos(\omg t) ] \,\del[ y - a \sin(\omg t) ] \,\del(z) \ddx \ddy \ddz
		= e a \cos(\omg t), \\
		\py &= e \iiint y \,\del[ x - a \cos(\omg t) ] \,\del[ y - a \sin(\omg t) ] \,\del(z) \ddx \ddy \ddz
		= e a \sin(\omg t), \\
		\pz &= e \iiint z \,\del[ x - a \cos(\omg t) ] \,\del[ y - a \sin(\omg t) ] \,\del(z) \ddx \ddy \ddz
		= 0,
	}
	so we can write $\vp = e a \,e^{-i \omg t} (\xh + i\,\yh).$  Substituting into Eq.~\refeq{ans1a},
	\al{
		\dv{\vL}{t} &= \Re\!\brac{ \frac{i k^3}{12\pi \epso} e^2 a^2 e^{-i \omg t} e^{i \omg t} [ (\xh + i\,\yh) \cross (\xh - i\,\yh) ] }
		= \Re\!\brac{ \frac{i e^2 k^3 a^2}{12\pi \epso} (-2i \,\xh \cross \yh) }
		= \Re\!\brac{ \frac{e^2 k^3 a^2}{6\pi \epso} \,\zh } \\
		&= \ans{ \frac{e^2 k^3 a^2}{6\pi \epso} \cos(\omg t) \,\zh, }
	}
	as desired. \qed
	
	A charge oscillating along the $z$ axis with amplitude $a$ has the charge density
	\eq{
		\rho(\vx, t) = e a \,\del(x) \,\del(y) \,\del[ z - \cos(\omg t) ],
	}
	which gives the dipole moment
	\al{
		\px &= e a \iiint x \,\del(x) \,\del(y) \,\del[ z - \cos(\omg t) ] \ddx \ddy \ddz
		= 0, \\
		\py &= e a \iiint y \,\del(x) \,\del(y) \,\del[ z - \cos(\omg t) ] \ddx \ddy \ddz
		= 0, \\
		\pz &= e a \iiint z \,\del(x) \,\del(y) \,\del[ z - \cos(\omg t) ] \ddx \ddy \ddz
		= e a \cos(\omg t).
	}
	In complex notation, $\vp = e a \,e^{-i\omg t} \,\zh$.  Substituting into Eq.~\refeq{ans1a}, we find
	\eq{
		\dv{\vL}{t} = \Re\!\brac{ \frac{i k^3}{12\pi \epso} e^2 a^2 e^{-i \omg t} e^{i \omg t} (\zh \cross \zh) }
		= \ans{ \vO. }
	}
	So we see that a charge undergoing linear motion does not lead to a radiated angular momentum, which is sensible.
}

%
%	Jackson 9.8(d)
%

\prob{}{
	What are the results corresponding to Probs.~{1(a)} and {1(b)} for magnetic dipole radiation?
}

\sol{
	The radiation fields for a magnetic dipole are given by Jackson~(19.35--36),
	\al{
		\vH &= \frac{1}{4\pi} \curly{ k^2 (\nh \cross \vm) \cross \nh \frac{e^{i k r}}{r} + [ 3 \nh (\nh \vdot \vm) - \vm ] \paren{ \frac{1}{r^3} - \frac{i k}{r^2} } e^{i k r} }, &
		\vE &= -\frac{\Zo}{4\pi} k^2 (\nh \cross \vm) \frac{e^{i k r}}{r} \paren{ 1 - \frac{1}{i k r} }.
	}
	\clearpage
	Comparing with Eq.~\refeq{fields1}, we see that $\vH \to -\vE / \Zo$, $\vE \to \Zo \vH$, and $\vp \to \vm / c$ as stated in the book~\cite[p.~413]{Jackson}.  Making these substitutions, the results of Probs.~{1.1(a)} and {(b)} become
	\al{
		\ans{ \dv{\vL}{t}\ }&\ans{= \frac{\muo k^3}{12\pi} \Im[ \vms \cross \vm ], } &
		\ans{ \frac{\dv*{\vL}{t}}{\dv*{E}{t}}\ }&\ans{= \frac{\Im[ \vms \cross \vm ]}{\omg \abs{\vm}^2} }
	}
	where we have used $\mu = 1 / \epso c^2$.
}



\newcommand{\cP}{\mathcal{P}}
\newcommand{\vcP}{\vb{\cP}}
\newcommand{\vcPx}{\vcP(\vx)}
\newcommand{\dcx}{\dd[3]{x}}
\newcommand{\phix}{\phi(\vx)}

\newcommand{\vaa}{\vb{a}}
\newcommand{\vbb}{\vb{b}}
\newcommand{\vc}{\vb{c}}
\newcommand{\lap}{\laplacian}

\newcommand{\Ai}{A_i}
\newcommand{\Aj}{A_j}
\newcommand{\Ei}{E_i}
\newcommand{\Ej}{E_j}
\newcommand{\Ek}{E_k}
\renewcommand{\xi}{x_i}
\newcommand{\xj}{x_j}
\newcommand{\xk}{x_k}
\newcommand{\dxi}{\dd{\xi}}
\newcommand{\dxj}{\dd{\xj}}
\newcommand{\intLL}{\int_{-L}^L}
\newcommand{\limLi}{\lim_{L \to \infty}}
\newcommand{\bigLL}{\bigg]_{-L}^L}

\newcommand{\cV}{\mathcal{V}}
\newcommand{\intcV}{\int_\cV}
\newcommand{\intS}{\int_S}
\newcommand{\dS}{\dd{S}}
\newcommand{\nh}{\vb{\hat{n}}}
\newcommand{\phiq}{\phi_1}
\newcommand{\phiw}{\phi_2}
\newcommand{\rhoq}{\rho_1}
\newcommand{\rhow}{\rho_2}

\newcommand{\sh}{\vb{\hat{s}}}

\clearpage
\state{Beta function of the Gross-Neveu model~(P\&S~12.2)}{
	Compute $\bet(g)$ in the two-dimensional Gross-Neveu model studied in Problem~11.3,
	\eq{
		\cL = \psibsi i \ptsl \psisi + \frac{1}{2} g^2 (\psibsi \psisi)^2,
	}
	with $i = 1, \ldots, N$.  You should find that this model is asymptotically free.  How was that fact reflected in the solution to Problem~11.3?
}

\sol{
	We saw in Problem~2 of Homework~4 that this Lagrangian can be written as
	\eq{
		\cL = \psibsi i \ptsl \psisi - \sig \psibsi \psisi - \frac{1}{2 g^2} \sig^2,
	}
	where $\sig$ is a new scalar field with no kinetic energy terms.  In the modified minimal subtraction scheme, we found the effective potential was
	\eqn{Veff}{
		\Veff = \sig^2 \curly{ \frac{1}{2 g^2} + \frac{N}{4\pi} \brac{ \ln(\frac{\sig^2}{M^2}) - 1 } }.
	}
	Since $\Gam[ \phicl ] = -(V T) \Veff(\phi)$ by P\&S~(11.50), we have
	\eqn{Gam}{
		\Gam[ \sigcl ] = -(V T)  \sig^2 \curly{ \frac{1}{2 g^2} + \frac{N}{4\pi} \brac{ \ln(\frac{\sig^2}{M^2}) - 1 } }.
	}
	Referring to p.~3 of Lecture~11, we can apply the Callan-Symanzik equation to $\Gam$.   The Callan-Symanzik equation is P\&S~(12.41),
	\eq{
		\brac{ M \pdv{M} + \bet(\lam) \pdv{\lam} + n \gam(\lam) } G^{(n)}(\{ x_i \}; M, \lam) = 0.
	}
	For our problem, $\gam$ is 0 because there are no field insertions.  That is, we have
	\eq{
		\brac{ M \pdv{M} + \bet(g) \pdv{g} } \Gam[ \phicl ] = 0.
	}
	Using Eq.~\refeq{Gam}, note that
	\al{
		\pdv{\Gam}{M} &= (V T) \frac{N \sig^2}{2 \pi M}, &
		\pdv{\Gam}{g} &= (V T) \frac{\sig^2}{g^3}.
	}
	Then
	\eq{
		0 = (V T) \paren{ \frac{N \sig^2}{2 \pi} + \bet(g) \frac{\sig^2}{g^3} }
		\qimplies
		\ans{ \betg = -\frac{N g^3}{2\pi}. }
	}
	This model is asymptotically free because the $\bet$ function is proportional to $-g^3$~\cite[pp.~424--425]{Peskin}.
	
	In 2(e) of Homework~4, we found that the vacuum expectation value of $\sig$ was
	\eq{
		\sig = \pm M e^{-\pi / N g^2} = \pm v.
	}
	We showed that the vacuum expectation value does not depend on the renormalization condition chosen.  This means that we can increase $M \to 0$ while holding $\sig$ constant, and see that $g \to 0$ logarithmically.  This is indicative of an asymptotically-free theory~\cite[p.~425]{Peskin}. \qed
}



\newcommand{\vl}{\vb{l}}
\newcommand{\vL}{\vb{L}}
\newcommand{\li}{l_i}
\newcommand{\Bi}{B_i}
\newcommand{\Bj}{B_j}
\newcommand{\Bk}{B_k}
\newcommand{\vi}{v_i}
\newcommand{\vj}{v_j}

\begin{statement}{}
	The angular momentum density of the electromagnetic field is given by
	\beq
		\vl = \vx \times \vcP
		= \frac{c}{4\pi} \vx \times (\vE \times \vB).
	\eeq
	Consider a source free ($\rho = 0$, $\vJ = 0$) solution to Maxwell's equations in electrodynamics with $\vE$ and $\vB$ vanishing rapidly as $\absx \to \infty$, so the total angular momentum
	\beq
		\vL = \int \vl \dcx
	\eeq
	is well defined.  Show that $\vL$ is conserved, i.e., independent of time.
\end{statement}

\begin{solution}
	We want to show that
	\beq
		\dv{\vL}{t} = 0.
	\eeq
	From Eq.~(4.12) in the lecture notes,
	\beqn \label{baccab}
		\vaa \times (\vbb \times \vc) = (\vaa \cdot \vc) \vbb - (\vaa \cdot \vbb) \vc.
	\eeqn
	Then
	\beq
		\vl = (\vx \cdot \vB) \vE - (\vx \cdot \vE) \vB.
	\eeq
	Differentiating with respect to time and applying the product rule, we find
	\begin{align*}
		\pdv{\vl}{t} &= \vE \pdv{}{t} (\vx \cdot \vB) + (\vx \cdot \vB) \pdv{\vE}{t} - \vB \pdv{}{t} (\vx \cdot \vE) + (\vx \cdot \vE) \pdv{\vB}{t} \\
		&= \vE \left(\pdv{\vx}{t} \cdot \vB + \vx \cdot \pdv{\vB}{t} \right) + (\vx \cdot \vB) \pdv{\vE}{t} - \vB \left (\pdv{\vx}{t} \cdot \vE + \vx \cdot \pdv{\vE}{t} \right) + (\vx \cdot \vE) \pdv{\vB}{t}
	\end{align*}
	For a source-free solution, Maxwell's equations~(1.1)--(1.4) in the lecture notes become
	\begin{align*}
		\grad \cdot \vE &= 0, &
		\grad \times \vB - \frac{1}{c} \pdv{\vE}{t} &= 0, &
		\grad \cdot \vB &= 0, &
		\grad \times \vE + \frac{1}{c} \pdv{\vB}{t} &= 0.
	\end{align*}
	This allows us to write
	\begin{align*}
		\pdv{\vl}{t} &= \vE [\vv \cdot \vB - c \vx \cdot (\grad \times \vE)] + c (\vx \cdot \vB) (\grad \times \vB) - \vB [\vv \cdot \vE + c \vx \cdot (\grad \times \vB)] - c (\vx \cdot \vE) (\grad \times \vE) \\
		&= \vE (\vv \cdot \vB) - \vB (\vv \cdot \vE) - c \vE [ \vx \cdot (\grad \times \vE) ] - c \vB [ \vx \cdot (\grad \times \vB) ] + c (\vx \cdot \vB) (\grad \times \vB) - c (\vx \cdot \vE) (\grad \times \vE)
	\end{align*}
	where $\vv = \pdv*{\vx}{t}$.  
	Vector calculus identity (6) in Griffiths is
	\beq
		\grad \cdot (\vaa \times \vbb) = \vbb \cdot (\grad \times \vaa) - \vaa \cdot (\grad \times \vbb).
	\eeq
	Then
	\beq
		\vx \cdot (\grad \times \vE) = \grad \cdot (\vE \times \vx) + \vE \cdot (\grad \times \vx)
		= \grad \cdot (\vE \times \vx),
	\eeq
	and similarly for $\vE \to \vB$.  So
	\beq
		\pdv{\vl}{t} = \vE (\vv \cdot \vB) - \vB (\vv \cdot \vE) - c \vE [ \grad \cdot (\vx \times \vE) ] - c \vB [ \grad \cdot (\vx \times \vB) ] + c (\vx \cdot \vB) (\grad \times \vB) - c (\vx \cdot \vE) (\grad \times \vE).
	\eeq
	
%	In component notation,
%	\begin{align*}
%		\pdv{\li}{t} &= \Ei [ \vj \Bj - c \xj (\grad \times \vE)_j ] + c \xj \Bj (\grad \times \vB)_i - \Bi [ \vj \Ej + c \xj (\grad \times \vB)_j ] - c \xj \Ej (\grad \times \vE)_i \\
%%		&= \Ei \left[ \vj \Bj - c \xj \left( \pdv{\Ei}{\xk} - \pdv{\Ek}{\xi} \right) \right] + c \xj \Bj (\grad \times \vB)_i - \Bi [ \vj \Ej + c \xj (\grad \times \vB)_j ] - c \xj \Ej \left( \pdv{\Ek}{\xj} - \pdv{\Ej}{\xk} \right) \\
%		&= \Ei \vj \Bj - c \Ei \xj (\grad \times \vE)_j + c \xj \Bj (\grad \times \vB)_i - \Bi \vj \Ej + c \Bi \xj (\grad \times \vB)_j - c \xj \Ej (\grad \times \vE)_i
%	\end{align*}
\end{solution}


\vfill
In addition to the course lecture notes, I consulted Griffiths's \emph{Introduction to Electrodynamics}, Jackson's \emph{Classical Electrodynamics}, and Kirk McDonald's electromagnetism notes while writing up these solutions.
\end{document}