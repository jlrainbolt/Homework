\documentclass[11pt]{article}
\usepackage{geometry, titlesec}
\usepackage[parfill]{parskip}
\usepackage[italicdiff]{physics}
\usepackage{amsfonts, amsthm, mathrsfs}
\usepackage[cm]{fullpage}
\usepackage{fancyhdr}
\usepackage{enumitem}
\usepackage{xcolor, soul}
\usepackage{siunitx}
%\allowdisplaybreaks

\renewcommand{\thesubsection}{\thesection.\alph{subsection}}
\renewcommand{\vec}[1]{\mathbf{#1}}
\newcommand{\vfix}{\vspace{-\baselineskip}}

\makeatletter
\renewcommand*\env@cases[1][1.2]{%
  \let\@ifnextchar\new@ifnextchar
  \left\lbrace
  \def\arraystretch{#1}%
  \array{@{}l@{\quad}l@{}}%
}
\makeatother
 
\renewcommand{\footrulewidth}{.2pt}
%\setlist[enumerate]{leftmargin=*}
\pagestyle{fancy}
\fancyhf{}
\lhead{\textbf{Physics 322 Homework 1}}
\rhead{Lacey Rainbolt}
\setlength{\headheight}{11pt}
\setlength{\headsep}{11pt}
\setlength{\footskip}{24pt}
\lfoot{\today}
\rfoot{\thepage}

\titleformat{\section}[runin]{\normalfont\large\bfseries}{Problem \thesection.}{1em}{}
\titleformat{\subsection}[runin]{\normalfont\large\bfseries}{\thesubsection}{1em}{}
\titleformat{\subparagraph}[leftmargin]{\normalfont\normalsize\bfseries}{}{0pt}{}

\newcommand{\refeq}[1]{(\ref{#1})}

\newcommand{\beq}{\begin{equation*}}
\newcommand{\eeq}{\end{equation*}}

\newcommand{\beqn}{\begin{equation}}
\newcommand{\eeqn}{\end{equation}}

\newcommand{\blg}{\begin{align*}}
\newcommand{\elg}{\end{align*}}


\newenvironment{statement}[1]
{
	\section{#1}
	\color{darkgray}
	\ignorespaces
}
{
%    \smallskip
}

\newenvironment{problem}
{
	\subsection{}
	\color{darkgray}
%	\paragraph{Problem.}
    \ignorespaces
}
{

}

\newenvironment{solution}
{
    \paragraph{Solution.}
    \ignorespaces
}
{
    \bigskip
}


\DeclareSIUnit\esu{esu}
\DeclareSIUnit\erg{erg}

\begin{document}

\newcommand{\cV}{\mathcal{V}}
\newcommand{\lap}{\nabla^2}
\newcommand{\vx}{\vec{x}}
\newcommand{\dV}{\partial \cV}
\newcommand{\nh}{\vec{\hat{n}}}
\newcommand{\dcx}{\dd[3]{x}}
\newcommand{\vE}{\vec{E}}
\newcommand{\En}{E_n}

\newcommand{\dS}{\dd{S}}

\begin{statement}{}
	Let $\cV$ be a bounded region of space and let $\phi$ be an electrostatic potential that is source free in this region, so that $\lap\phi = 0$ throughout $\cV$.  Suppose that for all $\vx$ lying on the boundary $S = \dV$, we have
	\beqn \label{given}
		\phi(\vx) = -f(\vx) \nh \cdot \nabla \phi(\vx),
	\eeqn
	where $f$ is a positive function ($f(\vx) \geq 0$) and $\nh$ is the outward pointing normal.  Show that $\phi = 0$ throughout $\cV$.
\end{statement}

\begin{solution}
%	Integrating both sides of \refeq{given} over the surface $S$ gives us
%	\beq
%		\int_S \phi(\vx) \dS = -\int_S f(\vx) \nabla \phi(\vx) \cdot \nh \dS
%		= -\int_\cV \nabla \cdot f(\vx) \nabla\phi(\vx) \dcx,
%	\eeq
%	where we have applied Gauss's theorem, Eq.~(2.6) in the course notes, to obtain the second equality.  For the third integrand, note that
%	\beq
%		\nabla \cdot f(\vx) \nabla\phi(\vx) = f(\vx) \lap \phi(\vx) + \nabla f(\vx) \cdot \nabla\phi(\vx) = \nabla f(\vx) \cdot \nabla\phi(\vx),
%	\eeq
%	where in going to the third equality we have applied $\lap\phi = 0$.  Then
%	\beq
%		\int_S f(\vx) \nabla \phi(\vx) \cdot \nh \dS = \int_\cV \nabla f(\vx) \cdot \nabla\phi(\vx) \dcx
%	\eeq

	Maxwell's equations for electrostatics are
	\begin{align}
		\nabla \cdot \vE &= 4\pi\rho, \label{me1} \\
		\vE &= -\nabla \phi. \label{me2}
	\end{align}
	Substituting \refeq{me2} into \refeq{given}, we have
	\beq
		\phi(\vx) = f(\vx) \nh \cdot \vE(\vx).
	\eeq
	Then, taking the gradient of both sides,
	\begin{align*}
		\nabla \phi(\vx) &= (\nabla f) (\nh \cdot \vE) + f \nabla(\nh \cdot \vE) \\
		&= (\nabla f) (\nh \cdot \vE)+ f [(\nh \cdot \nabla) \vE + (\vE \cdot \nabla) \nh + \nh \times (\nabla \times \vE) + \vE \times (\nabla \times \nh)] \\
		&= (\nabla f) (\nh \cdot \vE)+ f (\nh \cdot \nabla) \vE + f (\vE \cdot \nabla) \nh,
	\end{align*}
	where we have used the fact that the curl of $\vE$ is zero \hl{(ME)} and that the curl of the outward-pointing normal vector must also be zero for a closed boundary.  Then taking the divergence of both sides,
	\begin{align*}
		\nabla^2 \phi &= \nabla \cdot [(\nabla f) (\nh \cdot \vE)+ f (\nh \cdot \nabla) \vE + f (\vE \cdot \nabla) \nh] \\
		&= 0
	\end{align*}
%Once again applying \refeq{me2}, we obtain
%	\begin{align*}
%		\vE &= -\nabla \big( f(\vx) \En(\vx) \big)
%		= -\En(\vx) \nabla f(\vx) - f(\vx) \nabla \En(\vx)
%		= -(\nh \cdot \vE) \nabla f(\vx) - f(\vx) \nabla(\nh \cdot \vE)
%	\end{align*}
%	Plugging back in...
%	\begin{align*}
%		\phi(\vx) &= -f(\vx) \nh \cdot [(\nh \cdot \vE) \nabla f(\vx) + f(\vx) \nabla(\nh \cdot \vE)] \\
%		&= -f(\vx) \En \nh \cdot \nabla f(\vx) - f^2(\vx) \nh \cdot \nabla \En
%	\end{align*}
%	
%	
%	Integrating both sides of this expression, we have
%	\beq
%		\int_S \phi(\vx) \dS = \int_S f(\vx) \vE \cdot \nh \dS = \int_\cV \nabla \cdot f(\vx) \vE \dcx,
%	\eeq
%	where we have used Gauss's theorem, Eq.~(2.6), to obtain the second equality.
%	
%	\beq 
%		\int_S \phi(\vx) \dS = \int_S f(\vx) \vE \cdot \nh \dS = \int_\cV \nabla \cdot f(\vx) \vE \dcx,
%	\eeq
\end{solution}


\state{Beta function of the Gross-Neveu model~(P\&S~12.2)}{
	Compute $\bet(g)$ in the two-dimensional Gross-Neveu model studied in Problem~11.3,
	\eq{
		\cL = \psibsi i \ptsl \psisi + \frac{1}{2} g^2 (\psibsi \psisi)^2,
	}
	with $i = 1, \ldots, N$.  You should find that this model is asymptotically free.  How was that fact reflected in the solution to Problem~11.3?
}

\sol{
	We saw in Problem~2 of Homework~4 that this Lagrangian can be written as
	\eq{
		\cL = \psibsi i \ptsl \psisi - \sig \psibsi \psisi - \frac{1}{2 g^2} \sig^2,
	}
	where $\sig$ is a new scalar field with no kinetic energy terms.  In the modified minimal subtraction scheme, we found the effective potential was
	\eqn{Veff}{
		\Veff = \sig^2 \curly{ \frac{1}{2 g^2} + \frac{N}{4\pi} \brac{ \ln(\frac{\sig^2}{M^2}) - 1 } }.
	}
	Since $\Gam[ \phicl ] = -(V T) \Veff(\phi)$ by P\&S~(11.50), we have
	\eqn{Gam}{
		\Gam[ \sigcl ] = -(V T)  \sig^2 \curly{ \frac{1}{2 g^2} + \frac{N}{4\pi} \brac{ \ln(\frac{\sig^2}{M^2}) - 1 } }.
	}
	Referring to p.~3 of Lecture~11, we can apply the Callan-Symanzik equation to $\Gam$.   The Callan-Symanzik equation is P\&S~(12.41),
	\eq{
		\brac{ M \pdv{M} + \bet(\lam) \pdv{\lam} + n \gam(\lam) } G^{(n)}(\{ x_i \}; M, \lam) = 0.
	}
	For our problem, $\gam$ is 0 because there are no field insertions.  That is, we have
	\eq{
		\brac{ M \pdv{M} + \bet(g) \pdv{g} } \Gam[ \phicl ] = 0.
	}
	Using Eq.~\refeq{Gam}, note that
	\al{
		\pdv{\Gam}{M} &= (V T) \frac{N \sig^2}{2 \pi M}, &
		\pdv{\Gam}{g} &= (V T) \frac{\sig^2}{g^3}.
	}
	Then
	\eq{
		0 = (V T) \paren{ \frac{N \sig^2}{2 \pi} + \bet(g) \frac{\sig^2}{g^3} }
		\qimplies
		\ans{ \betg = -\frac{N g^3}{2\pi}. }
	}
	This model is asymptotically free because the $\bet$ function is proportional to $-g^3$~\cite[pp.~424--425]{Peskin}.
	
	In 2(e) of Homework~4, we found that the vacuum expectation value of $\sig$ was
	\eq{
		\sig = \pm M e^{-\pi / N g^2} = \pm v.
	}
	We showed that the vacuum expectation value does not depend on the renormalization condition chosen.  This means that we can increase $M \to 0$ while holding $\sig$ constant, and see that $g \to 0$ logarithmically.  This is indicative of an asymptotically-free theory~\cite[p.~425]{Peskin}. \qed
}

\newcommand{\kq}{\ket{1}}
\newcommand{\kw}{\ket{2}}
\newcommand{\ke}{\ket{3}}

\newcommand{\vq}{v_1}
\newcommand{\vw}{v_2}
\newcommand{\ve}{v_3}

\newcommand{\vqs}{\vq^*}
\newcommand{\vws}{\vw^*}

\newcommand{\Heff}{H_\text{eff}}
\newcommand{\Eo}{E\suo}
\newcommand{\Eod}{\Eo_D}

\newcommand{\Pq}{P_1}

%\clearpage
\begin{statement}{}
	Consider the Hamiltonian $\Ho$ acting on a three-dimensional Hilbert space spanned by the orthonormal basis $\{\kq, \kw, \ke\}$.  $\Ho = \sum_{i = 3}^3 E_i \ketbra{i}$, with energy eigenvalues $\Eoq, \Eow, \Eoe$.  Assume $\Eoq = \Eow = \Eod$.  To $\Ho$, we add a perturbation
	\beq
		V = \vq \ketbra{1}{3} + \vqs \ketbra{3}{1} + \vw \ketbra{2}{3} + \vws \ketbra{3}{2}.
	\eeq
	Here, $\vq$ and $\vw$ are complex constants and small compared to $\Ee$.
\end{statement}

\begin{problem}
	To second order in $V$, write down the explicit form of the effective Hamiltonian acting on the subspace spanned by $\{\kq, \kw\}$.
\end{problem}

\begin{solution}
	We have
	\begin{align*}
		\Ho &= \mqty[ \Eod & 0 & 0 \\ 0 & \Eod & 0 \\ 0 & 0 & \Eoe ], &
		V &= \mqty[ 0 & 0 & \vq \\ 0 & 0 & \vw \\ \vqs & \vws & 0 ], &
		H &= \Ho + \lam V = \mqty[ \Eod & 0 & \lam \vq \\ 0 & \Eod & \lam \vw \\ \vqs & \vws & \Eoe].
	\end{align*}
	From the lecture notes and (5.2.12) in Sakurai, the effective Hamiltonian is given to second order in $\lam$ by
	\beq
		\Heff = \Eod + \lam \Po V \Po + \lam^2 \Po V \Pq (\Eod - \Ho)^{-1} \Pq V \Po,
	\eeq
	where $\Po$ is the projection onto the degenerate subspace, $\Pq$ is the projection onto the nondegenerate subspace, and $\Eod$ is the degenerate energy.  Here, $\Po$ projects onto the subspace spanned by $\{ \kq, \kw \}$ and $\Pq$ onto that spanned by $\{ \ke \}$.
	
	Note that
	\beq
		Po V \Po = \mqty[ 1 & 0 & 0 \\ 0 & 1 & 0 \\ 0 & 0 & 0 ] \mqty[ 0 & 0 & \vq \\ 0 & 0 & \vw \\ \vqs & \vws & 0 ] \mqty[ 1 & 0 & 0 \\ 0 & 1 & 0 \\ 0 & 0 & 0 ]
		= \mqty[ 0 & 0 & 0 \\ 0 & 0 & 0 \\ 0 & 0 & 0 ],
	\eeq
	and
	\begin{align*}
		\Po V \Pq (\Eod - \Ho)^{-1} &\Pq V \Po = \frac{1}{\Eod - \Eoe} \Po \mqty[ 0 & 0 & \vq \\ 0 & 0 & \vw \\ \vqs & \vws & 0 ] \mqty[ 0 & 0 & 0 \\ 0 & 0 & 0 \\ 0 & 0 & 1] \mqty[ 0 & 0 & \vq \\ 0 & 0 & \vw \\ \vqs & \vws & 0 ] \Po \\
		&= \frac{1}{\Eod - \Eoe} \mqty[ 1 & 0 & 0 \\ 0 & 1 & 0 \\ 0 & 0 & 0 ] \mqty[|\vq|^2 & \vq \vws & 0 \\ \vqs \vw & |\vq|^2 & 0 \\ 0 & 0 & 0] \mqty[ 1 & 0 & 0 \\ 0 & 1 & 0 \\ 0 & 0 & 0 ]
		= \frac{1}{\Eod - \Eoe} \mqty[|\vq|^2 & \vq \vws & 0 \\ \vqs \vw & |\vq|^2 & 0 \\ 0 & 0 & 0].
	\end{align*}
	
	In the degenerate subspace, we have
	\begin{align*}
		\Heff = \mqty[ \Eod + |\vq|^2/(\Eod - \Eoe) & \vq \vws / (\Eod - \Eoe) \\ \vqs \vw / (\Eod - \Eoe) & \Eod + |\vw|^2/(\Eod - \Eoe)].
	\end{align*}
\end{solution}


%\clearpage
\newcommand{\uq}{u_1}
\newcommand{\uw}{u_2}
\newcommand{\wq}{w_1}
\newcommand{\ww}{w_2}

\begin{problem}
	By solving the effective Hamiltonian, construct the approximate solution for the eigenvalues and eigenfunctions of $\Ho + V$.  (The eigenkets only need to be constructed within the degenerate subspace.)
\end{problem}

\begin{solution}
	Let $E$ be the eigenvalues of $\Heff$.  We need to solve the characteristic equation
	\begin{align*}
		0 &= \det(\Heff - E I)
		= \vmqty{ \Eod + |\vq|^2/(\Eod - \Eoe) - E & \vq \vws / (\Eod - \Eoe) \\ \vqs \vw / (\Eod - \Eoe) & \Eod + |\vw|^2/(\Eod - \Eoe) - E } \\
		&= \left( \Eod + \frac{|\vq|^2}{\Eod - \Eoe} - E \right) \left( \Eod + \frac{|\vw|^2}{\Eod - \Eoe} - E \right) - \frac{|\vq|^2 |\vw|^2}{(\Eod - \Eoe)^2} \\
		&= {\Eod}^2 + \Eod \frac{|\vw|^2}{\Eod - \Eoe} - \Eod E + \Eod \frac{|\vq|^2}{\Eod - \Eoe} - E \frac{|\vq|^2}{\Eod - \Eoe} - \Eod E - E \frac{|\vw|^2}{\Eod - \Eoe} + E^2 \\
		&= E^2 - \Eod E - E \frac{|\vq|^2 + |\vw|^2}{\Eod - \Eoe} - \Eod E + {\Eod}^2 + \Eod \frac{|\vq|^2 + |\vw|^2}{\Eod - \Eoe} \\
		&= (E - \Eod) \left( E - \Eod - \frac{|\vq|^2 + |\vw|^2}{(\Eod - \Eoe)^2} \right),
	\end{align*}
	so the eigenvalues are
	\begin{align*}
		\Eq &= \Eod, &
		\Ew &= \Eod + \frac{|\vq|^2 + |\vw|^2}{(\Eod - \Eoe)^2}.
	\end{align*}
	
	The eigenvector corresponding to $\Eq$ can be found by
	\beq
		\mqty[ \Eod + |\vq|^2 / (\Eod - \Eoe) & \vq \vws / (\Eod - \Eoe) \\ \vqs \vw / (\Eod - \Eoe) & \Eod + |\vw|^2 / (\Eod - \Eoe) ] \mqty[ \uq \\ \uw ] = \Eod \mqty[ \uq \\ \uw ]
	\eeq
	which is equivalent to the system of equations
	\begin{align*}
		\left( \Eod + \frac{|\vq|^2}{\Eod - \Eoe} \right) \uq + \frac{\vq \vws}{\Eod - \Eoe} \uw &= \Eod \uq, &
		\frac{\vqs \vw}{\Eod - \Eoe} \uq + \left( \Eod + \frac{|\vw|^2}{\Eod - \Eoe} \right) \uw &= \Eod \uw.
	\end{align*}
	By inspection, these are satisfied when $\uq = -\vws$ and $\uw = \vqs$.
	
	For the eigenvector corresponding to $\Ew$, we have
	\beq
		\mqty[ \Eod + |\vq|^2 / (\Eod - \Eoe) & \vq \vws / (\Eod - \Eoe) \\ \vqs \vw / (\Eod - \Eoe) & \Eod + |\vw|^2 / (\Eod - \Eoe) ] \mqty[ \wq \\ \ww ] = \left( \Eod + \frac{|\vq|^2 + |\vw|^2}{\Eod - \Eoe} \right)\mqty[ \wq \\ \ww ]
	\eeq
	which is equivalent to the system of equations
	\begin{align*}
		\left( \Eod + \frac{|\vq|^2}{\Eod - \Eoe} \right) \wq + \frac{\vq \vws}{\Eod - \Eoe} \ww &= \left( \Eod + \frac{|\vq|^2 + |\vw|^2}{\Eod - \Eoe} \right) \wq, \\
		\frac{\vqs \vw}{\Eod - \Eoe} \wq + \left( \Eod + \frac{|\vw|^2}{\Eod - \Eoe} \right) \ww &= \left( \Eod + \frac{|\vq|^2 + |\vw|^2}{\Eod - \Eoe} \right) \ww.
	\end{align*}
	By inspection, these are satisfied when $\wq = \vq$ and $\ww = \ww$.  So we have the eigenvectors
	\begin{align*}
		\ket*{\Eq} &= \mqty[ -\vws \\ \vqs ], &
		\ket*{\Ew} &= \mqty[ \vq \\ \vw ].
	\end{align*}

\vfill
In writing up these solutions, I consulted Griffiths's \emph{Introduction to Electrodynamics} and David Tong's lecture notes on electrodynamics.
\end{document}