\newcommand{\cV}{\mathcal{V}}
\newcommand{\lap}{\nabla^2}
\newcommand{\vx}{\vec{x}}
\newcommand{\dV}{\partial \cV}
\newcommand{\nh}{\vec{\hat{n}}}
\newcommand{\dcx}{\dd[3]{x}}
\newcommand{\vE}{\vec{E}}
\newcommand{\En}{E_n}
\newcommand{\dS}{\dd{S}}

\begin{statement}{}
	Let $\cV$ be a bounded region of space and let $\phi$ be an electrostatic potential that is source free in this region, so that $\lap\phi = 0$ throughout $\cV$.  Suppose that for all $\vx$ lying on the boundary $S = \dV$, we have
	\beqn \label{given}
		\phi(\vx) = -f(\vx) \nh \cdot \nabla \psi(\vx),
	\eeqn
	where $f$ is a positive function ($f(\vx) \geq 0$) and $\nh$ is the outward pointing normal.  Show that $\phi = 0$ throughout $\cV$.
\end{statement}

\begin{solution}
	Clearly $\phi = 0$ trivially satisfies the Laplace equation $\lap\phi = 0$ and the boundary condition \refeq{given}.  We will show that this solution is unique.
	
	Suppose to the contrary that there exists another solution $\psi$.  This means $\lap\psi = 0$ snd that $\psi$ satisfies the boundary condition
	\beqn \label{psibound}
		\psi(\vx) = -f(\vx) \nh \cdot \nabla \psi(\vx).
	\eeqn
	Multiplying both sides of $\lap\psi = 0$ by $\psi$ and integrating over $\cV$,
	\begin{align}
		0 &= \int_\cV \psi \lap \psi \dcx
		= \int_\cV \left( \nabla \cdot (\psi \nabla \psi) - |\nabla \psi|^2 \right) \dcx \notag \\
		&= \int_\cV \nabla \cdot (\psi \nabla \psi) \dcx - \int_\cV |\nabla \psi|^2 \dcx, \label{thing}
	\end{align}
	where we have factored out the divergence.  Applying Gauss's theorem, Eq.~(2.6) in the course notes, to the first integral on the right-hand side of \refeq{thing}, we have
	\beq
		\int_\cV \nabla \cdot (\psi \nabla \psi) \dcx = \int_S \psi \nh \cdot \nabla\psi \dS
		= - \int_S f(\vx) (\nh \cdot \nabla \psi)^2 \dS,
	\eeq
	where we have made the substitution \refeq{psibound}.  Substituting this back into \refeq{thing} yields
	\beq
		0 = \int_S f(\vx) (\nh \cdot \nabla \psi)^2 \dS + \int_\cV |\nabla \psi|^2 \dcx.
	\eeq
	Each of the integrands is positive, because $f(\vx)$ is known to be positive and the other quantities are squared.  Therefore, in order to satisfy the equality, we must have $\nabla \psi = 0$ and $\nh \cdot \nabla \psi = 0$.  Feeding the latter into \refeq{psibound}, we conclude that $\psi = 0$.  Therefore, we have shown that $\phi = 0$ is indeed the only solution to the Laplace equation subject to the boundary condition \refeq{given}. \qed
\end{solution}