\documentclass[11pt]{article}
\usepackage{geometry, titlesec}
\usepackage[parfill]{parskip}
\usepackage[italicdiff]{physics}
\usepackage{amsfonts, amsthm, mathrsfs}
\usepackage[cm]{fullpage}
\usepackage{fancyhdr}
\usepackage{enumitem}
\usepackage{xcolor, soul}
\usepackage{siunitx}
\allowdisplaybreaks

\renewcommand{\thesubsection}{\thesection.\alph{subsection}}
%\renewcommand{\vb}[1]{\mathbf{#1}}
\newcommand{\vfix}{\vspace{-\baselineskip}}

\makeatletter
\renewcommand*\env@cases[1][1.2]{%
  \let\@ifnextchar\new@ifnextchar
  \left\lbrace
  \def\arraystretch{#1}%
  \array{@{}l@{\quad}l@{}}%
}
\makeatother
 
\renewcommand{\footrulewidth}{.2pt}
%\setlist[enumerate]{leftmargin=*}
\pagestyle{fancy}
\fancyhf{}
\lhead{\textbf{Physics 322 Homework 3}}
\rhead{Lacey Rainbolt}
\setlength{\headheight}{11pt}
\setlength{\headsep}{11pt}
\setlength{\footskip}{24pt}
\lfoot{\today}
\rfoot{\thepage}

\titleformat{\section}[runin]{\normalfont\large\bfseries}{Problem \thesection.}{1em}{}
\titleformat{\subsection}[runin]{\normalfont\large\bfseries}{\thesubsection}{1em}{}
\titleformat{\subparagraph}[leftmargin]{\normalfont\normalsize\bfseries}{}{0pt}{}

\newcommand{\refeq}[1]{(\ref{#1})}

\newcommand{\beq}{\begin{equation*}}
\newcommand{\eeq}{\end{equation*}}

\newcommand{\beqn}{\begin{equation}}
\newcommand{\eeqn}{\end{equation}}

\newcommand{\blg}{\begin{align*}}
\newcommand{\elg}{\end{align*}}


\newenvironment{statement}[1]
{
	\section{#1}
	\color{darkgray}
	\ignorespaces
}
{
%    \smallskip
}

\newenvironment{problem}
{
	\subsection{}
	\color{darkgray}
%	\paragraph{Problem.}
    \ignorespaces
}
{

}

\newenvironment{solution}
{
    \paragraph{Solution.}
    \ignorespaces
}
{
    \bigskip
}



\begin{document}

\newcommand{\eps}{\epsilon}
\newcommand{\vx}{\vb{x}}
\newcommand{\phix}{\phi(\vx)}
\newcommand{\dcx}{\dd[3]{x}}
\newcommand{\dcxp}{\dd[3]{x'}}
\newcommand{\rhox}{\rho(\vx)}
\newcommand{\rhoxp}{\rho(\vx')}
\newcommand{\rhopxp}{\rho'(\vx')}
\newcommand{\xh}{\vec{\hat{x}}}
\newcommand{\absx}{\abs{\vx}}
\newcommand{\absxp}{\abs{\vx'}}

\newcommand{\Ylm}{Y_{l m}}
\newcommand{\qlm}{q_{l m}}
\newcommand{\Plm}{P_l^m}
\newcommand{\tht}{\theta}
\newcommand{\cost}{\cos\tht}
\newcommand{\vph}{\varphi}
\newcommand{\tv}{(\tht, \vph)}
\newcommand{\tvp}{(\tht', \vph')}
\newcommand{\Gxxp}{G(\vx, \vx')}
\newcommand{\Gpxxp}{G'(\vx, \vx')}
\newcommand{\Gdxxp}{G_D(\vx, \vx')}
\newcommand{\qplm}{q'_{l m}}

\newcommand{\lap}{\nabla^2}
\newcommand{\evphi}{\ev{\phi}}
\newcommand{\evphix}{\evphi\!(\vx)}
\newcommand{\rhof}{\rho_f}
\newcommand{\evrhof}{\ev{\rhof}}
\newcommand{\fe}{\frac{1}{\eps}}
\newcommand{\tif}{\text{if }}
\newcommand{\Al}{A_l}
\newcommand{\Bl}{B_l}
\newcommand{\Cl}{C_l}

\newcommand{\intoi}{\int_0^\infty}
\newcommand{\intono}{\int_{-1}^{1}}
\newcommand{\intotp}{\int_0^{2\pi}}
\newcommand{\drp}{\dd{r'}}
\newcommand{\dctp}{\dd{(\cost')}}
\newcommand{\dvp}{\dd{\vph'}}

\newcommand{\Ylotv}{Y_{l 0}\tv}
\newcommand{\dr}{\dd{r}}
\newcommand{\dct}{\dd{(\cost)}}
\newcommand{\ddv}{\dd{\vph}}

\newcommand{\Alm}{A_{l m}}
\newcommand{\Blm}{B_{l m}}
\newcommand{\Ploct}{P_l^0(\cost)}
\newcommand{\Ploctp}{P_l^0(\cost')}
\newcommand{\alp}{\alpha}
\newcommand{\rtp}{(r, \tht, \phi)}
\newcommand{\phirtp}{\phi\rtp}
\newcommand{\qlo}{q_{l 0}}
\newcommand{\qplo}{q'_{l 0}}
\newcommand{\qpplm}{q''_{l m}}
\newcommand{\qpplo}{q''_{l 0}}

\newcommand{\vD}{\vb{D}}
\newcommand{\evD}{\ev{\vD}}
\newcommand{\nh}{\vb{\hat{n}}}
\newcommand{\vE}{\vb{E}}
\newcommand{\evE}{\ev{\vE}}
\newcommand{\Er}{E_r}
\newcommand{\evEr}{\ev{\Er}}
\newcommand{\rh}{\vb{\hat{r}}}

\newcommand{\dint}{\displaystyle \int}
\newcommand{\dsum}{\displaystyle \sum}

\begin{statement}{} \label{1}
	Consider a dielectric ball of radius $R$ with dielectric constant $\eps$.  Obtain a multipole expansion for the field, $\phix$, of a point charge $q$ placed at a point $\vx'$ with $\abs{\vx'} = d > R$ (so the charge is outside of the dielectric ball).
	
	Hint: Follow the procedure we used in class to find the multipole expansion of a point charge without the dielectric, but now consider the three regions $r \leq R$, $R \leq r \leq d$, and $r \geq d$.  Obtain the form of the solution in these regions and match suitably.
\end{statement}

\begin{solution}
	The spherical harmonic expansion of the Green's function $\Gxxp$ is given by Eq.~(2.78):
	\beqn \label{greenexp}
		\Gxxp = \frac{1}{\abs{\vx - \vx'}}
		= \begin{cases}
			\dsum_{l,m} \dfrac{4\pi}{2l + 1} \dfrac{r^l}{{r'}^{l + 1}} \Ylm^*\tvp \, \Ylm\tv & \tif r < r', \\[2ex]
			\dsum_{l,m} \dfrac{4\pi}{2l + 1} \dfrac{{r'}^l}{r^{l + 1}} \Ylm^*\tvp \, \Ylm\tv & \tif r > r'.
		\end{cases}
	\eeqn
	The spherical harmonics $\Ylm$ are given by Eq.~(2.58),
	\beq
		\Ylm\tv = \sqrt{\frac{2l + 1}{4\pi}} \sqrt{\frac{(l - m)!}{(l + m)!}} \Plm(\cost) e^{i m \vph},
	\eeq
	and the associated Legendre polynomials $\Plm$ are given by Eq.~(2.59),
	\beq
		\Plm(x) = \frac{(-1)^m}{2^l l!} (1 - x^2)^{m/2} \dv[l + m]{}{x} (x^2 - 1)^l.
	\eeq	
	
	We assume the dielectric is linear, homogeneous, and isotropic.  Poisson's equation inside such a dielectric is given by Eq.~(3.22) in the course notes,
	\beq
		\lap\evphi = -\frac{4\pi}{\eps} \evrhof\!,
	\eeq
	where $\rhof$ is the free charge density.  Here, $\evrhof = 0$ since there are no free charges within the dielectric, so this reduces to Laplace's equation.  The general solution to Laplace's equation is given by Eq.~(3.61) in Jackson,
	\beqn \label{lapsol}
		\evphi\!(r, \tht, \vph) = \sum_{l,m} \left( \Alm r^l + \frac{\Blm}{r^{l+1}} \right) \, \Ylm\tv,
	\eeqn
	where $\Alm$ and $\Blm$ are constant coefficients.
	
	We will begin inside the dielectric, where $r \leq R$.  Here we must have $\Blm = 0$ because $1/r^{l+1}$ is undefined at the origin.  Without loss of generality, we may choose the location of the point charge to be on the $z$ axis at $z = d$, so $\vx' = (r', 0, 0)$.  Clearly, the system is azimuthally symmetric, so $m = 0$.  This gives us the macroscopically averaged potential
	\beqn \label{inside}
		\evphi\!(r, \tht, \vph) = \sum_{l} \Al r^l \, Y_{l 0}\tv
		= \sum_{l} \sqrt{\frac{2l + 1}{4\pi}} \Al r^l \Ploct \quad \tif r \leq R.
	\eeqn
	
	In the region $R \leq d \leq r$, we are in free space so $\evphi = \phi$.  The point charge is at greater $r$, so we account for its contribution using the first case of \refeq{greenexp}.  Additionally, there are multipole contributions from the dielectric at lesser $r$, so we must account for these using the second case of \refeq{greenexp}.  We can use the method of images to keep track of the dielectric contribution in this regime.  We find the Green's function for the image charge as the second term in the Dirichlet Green's function for a spherical cavity, which is Eq.~(2.91) in the lecture notes:
	\beq
		\Gdxxp = \frac{1}{\abs{\vx - \vx'}} + \frac{\alp}{\abs{\vx - \vx''}}
		\qq{where} \vx'' = \vx' \frac{R^2}{\abs{\vx'}^2}
		\qand \alp = -\frac{R}{\abs{\vx'}}.
	\eeq
	Adapting the second case of \refeq{greenexp} to this case, we obtain
	\beq
		\Gpxxp = \frac{\alp}{\abs{\vx - \vx''}}
		= -\sum_{l,m} \frac{4\pi}{2l + 1} \frac{R^{2l+1}}{{r'}^{l+1} r^{l + 1}} \Ylm^*\tvp \, \Ylm\tv \quad \tif r \leq R.
	\eeq
	Putting these together, and again taking advantage of the azimuthal symmetry, we have
	\begin{align}
		\phirtp &= q \sum_{l,m} \frac{4\pi}{2l + 1} \frac{r^l}{{r'}^{l + 1}} \Ylm^*\tvp \, \Ylm\tv - q \sum_{l,m} \Blm \frac{4\pi}{2l + 1} \frac{R^{2l+1}}{{r'}^{l+1} r^{l + 1}} \Ylm^*\tvp \, \Ylm\tv \notag \\
		&= q \sum_{l,m} \frac{4\pi}{2l + 1} \Ylm^*\tvp \, \Ylm\tv \frac{1}{{r'}^{l+1}} \left( r^l - \Blm \frac{R^{2l+1}}{r^{l+1}} \right) \notag \\
		&= q \sum_{l} \Ploctp \, \Ploct \frac{1}{{r'}^{l+1}} \left( r^l - \Bl \frac{R^{2l+1}}{r^{l+1}} \right) \quad \tif R \leq r leq d. \label{middle}
	\end{align}
	Here the second term has coefficient 1 because it applies to the point charge.
	
	In the region $r \geq d$, we account for both the point charge and the image charge using the second case of \refeq{greenexp}.  With the azimuthal symmetry, this gives us
	\begin{align} \label{outside}
		\phirtp = q \sum_{l,m} \dfrac{4\pi}{2l + 1} \dfrac{{r'}^l}{r^{l + 1}} \Ylm^*\tvp \, \Ylm\tv - \sum_{l,m} \frac{4\pi}{2l + 1} \frac{R^{2l+1}}{{r'}^{l+1} r^{l + 1}} \Ylm^*\tvp \, \Ylm\tv \notag \\
		& \quad \tif r \geq d.
	\end{align}
	
	Now we must match $\evphi$ at the boundaries of each region.  We will begin with \refeq{middle} and \refeq{outside}.  Evaluating at $r = d$, we have
	\beq
		\phi(d, \tht, \phi) = \sum_{l} \sqrt{\frac{4\pi}{2l+1}} \Ploct \begin{cases}
			\Bl R^{2l+1} \dfrac{\qpplo}{d^{l+1}} + d^l \qplo & \tif R \leq r \leq d \\[2ex]
			\Cl R^{2l+1} \dfrac{\qpplo}{d^{l+1}} + \dfrac{\qlo}{d^{l+1}} & \tif r \geq d.
		\end{cases}
	\eeq
	Equating these two cases gives us $\Bl = \Cl$.
	
	For \refeq{inside} and \refeq{middle}, we must match at $r = R$:
	\beq
		\evphi\!(R, \tht, \phi) = \sum_{l} \Ploct \begin{cases}
			\sqrt{\dfrac{2l + 1}{4\pi}} \Al R^l & \tif r \leq R, \\[2ex]
			\sqrt{\dfrac{4\pi}{2l+1}} R^l (\Bl \qpplo + \qplo) & \tif R \leq r \leq d,
		\end{cases}
	\eeq
	which gives us
	\beqn \label{A1}
		\Al = \frac{4\pi}{2l+1} (\Bl \qpplo + \qplo).
	\eeqn
	Here we must also match $\nh \cdot \evD$ at the boundary, where
	\beqn \label{D}
		\evD = \eps \!\evE
	\eeqn
	inside the dielectric, from Eq.~(3.20) in the course notes.  (In vacuum, $\vD = \vE$.)  Here $\nh = \rh$, so we are only concerned with the $r$ component of $\evE$.  Applying $\evE = -\grad\!\evphi$ to \refeq{inside} and \refeq{outside} gives us
	\beq
		\evEr\!(R, \tht, \phi) = -\sum_l R^{l-1} \Ploct \begin{cases}
			\sqrt{\dfrac{2l+1}{4\pi}} \Al l & \tif r \leq R, \\[2ex]
			\sqrt{\dfrac{4\pi}{2l+1}} [-(l + 1) \Bl \qpplo + l \qplo] & \tif R \leq r \leq d.
		\end{cases}
	\eeq
	Then we stipulate that
	\beq
		\rh \cdot \evD\!(R, \tht, \phi) = -\eps \sqrt{\frac{2l+1}{4\pi}} \Al l = \sqrt{\frac{4\pi}{2l+1}} [(l + 1) \Bl \qpplo - l \qplo],
	\eeq
	which implies
	\beqn \label{A2}
		\Al = \frac{1}{\eps} \frac{4\pi}{2l+1} \left( \qplo - \frac{l + 1}{l} \Bl \qpplo \right).
	\eeqn
	By equating \refeq{A1} and \refeq{A2}, we can solve for $\Bl$:
	\beq
		\Bl \qpplo + \qplo = \frac{1}{\eps} \left( \qplo - \frac{l + 1}{l} \Bl \qpplo \right)
		\implies
		\left( 1 + \frac{1}{\eps} \frac{l + 1}{l} \right) \Bl \qpplo = \left( \frac{1}{\eps} - 1 \right) \qplo
		\implies
		\Bl = \frac{1 - \eps}{1 + \eps + l^{-1}} \frac{\qplo}{\qpplo}.
	\eeq
	Feeding this back into \refeq{A1},
	\beq
		\Al = \frac{4\pi}{2l+1} \left( \frac{1 - \eps}{1 + \eps + l^{-1}} \frac{\qplo}{\qpplo} \qpplo + \qplo \right)
		= \frac{4\pi}{2l+1} \frac{1 - \eps + 1 + \eps + l^{-1}}{1 + \eps + l^{-1}} \qplo
		= \frac{4\pi}{2l+1} \frac{2 + l^{-1}}{1 + \eps + l^{-1}} \qplo
		= \frac{4\pi}{l (1 + \eps) + 1} \qplo.
	\eeq
	
	Substituting in all of the coefficients, \refeq{inside}, \refeq{middle}, and \refeq{outside} can be written as
	\beqn \label{sol1}
		\evphi\!\rtp = \sum_{l} \sqrt{\frac{4\pi}{2l+1}} \Ploct \begin{cases}
			\dfrac{2l + 1}{l (1 + \eps) + 1} \qplo r^l & \tif r \leq R, \\[2ex]
			\dfrac{1 - \eps}{1 + \eps + l^{-1}} R^{2l+1} \dfrac{\qplo}{r^{l+1}} + r^l \qplo & \tif R \leq r \leq d, \\[2ex]
			\dfrac{1 - \eps}{1 + \eps + l^{-1}} R^{2l+1} \dfrac{\qplo}{r^{l+1}} + \dfrac{\qlo}{r^{l+1}} & \tif r \geq d,
		\end{cases}
	\eeqn
	where $\qlm$ and $\qplm$ are given by \refeq{qlm}, and $\rhox = q \,\delta(r - d\cost)$.

	For this problem, $\rhox = q \,\delta(r - d\cost)$.  From \refeq{qlm}, we have
	\begin{align*}
		\qplo &= q \int \frac{\delta(r' - d\cost')}{{r'}^{l+1}} Y_{l 0}^*\tvp \dcxp
		= \sqrt{\frac{2l + 1}{4\pi}} q \intotp \intono \intoi \frac{\delta(r' - d\cost')}{{r'}^{l+1}} \Ploctp {r'}^2 \drp \dctp \dvp \\
		&= \sqrt{\frac{2l + 1}{4\pi}} q \intotp \dvp \intono \intoi \Ploctp \frac{\delta(r' - d\cost')}{{r'}^{l-1}} \dr \dctp
		= \sqrt{\pi (2l + 1)} \frac{q}{d^{l-1}} \intono \frac{\Ploctp}{\cos^{l-1}\tht'} \dctp
	\end{align*}
	and
	\begin{align*}
		\qlo &= q \int \delta(r' - d\cost') {r'}^l \, \Ylm^*\tvp \dcxp
		= \sqrt{\frac{2l + 1}{4\pi}} q \intotp \intono \intoi \delta(r' - d\cost') {r'}^{l+2} \Ploctp \drp \dctp \dvp \\
		&= \sqrt{\pi (2l + 1)} q d^{l+2} \intono \cos^{l+2}\tht' \Ploctp \dctp.
	\end{align*}
	\hl{which seems wrong}
	Finally, \hl{pretending the integrals evaluate to 1}, \refeq{sol1} becomes
	\beq
		\evphi\!\rtp = \sum_{l} 2\pi \Ploct \begin{cases}
			\dfrac{2l + 1}{l (1 + \eps) + 1} \dfrac{r^l}{d^{l-1}} & \tif r \leq R, \\[2ex]
			\dfrac{1 - \eps}{1 + \eps + l^{-1}} \dfrac{R^{2l+1}}{d^{l-1}} \dfrac{1}{r^{l+1}} + \dfrac{r^l}{d^{l-1}} & \tif R \leq r \leq d, \\[2ex]
			\dfrac{1 - \eps}{1 + \eps + l^{-1}} \dfrac{R^{2l+1}}{d^{l-1}} \dfrac{1}{r^{l+1}} + \dfrac{d^{l+2}}{r^{l+1}} & \tif r \geq d.
		\end{cases}
	\eeq
\end{solution}


\clearpage
\state{Beta function of the Gross-Neveu model~(P\&S~12.2)}{
	Compute $\bet(g)$ in the two-dimensional Gross-Neveu model studied in Problem~11.3,
	\eq{
		\cL = \psibsi i \ptsl \psisi + \frac{1}{2} g^2 (\psibsi \psisi)^2,
	}
	with $i = 1, \ldots, N$.  You should find that this model is asymptotically free.  How was that fact reflected in the solution to Problem~11.3?
}

\sol{
	We saw in Problem~2 of Homework~4 that this Lagrangian can be written as
	\eq{
		\cL = \psibsi i \ptsl \psisi - \sig \psibsi \psisi - \frac{1}{2 g^2} \sig^2,
	}
	where $\sig$ is a new scalar field with no kinetic energy terms.  In the modified minimal subtraction scheme, we found the effective potential was
	\eqn{Veff}{
		\Veff = \sig^2 \curly{ \frac{1}{2 g^2} + \frac{N}{4\pi} \brac{ \ln(\frac{\sig^2}{M^2}) - 1 } }.
	}
	Since $\Gam[ \phicl ] = -(V T) \Veff(\phi)$ by P\&S~(11.50), we have
	\eqn{Gam}{
		\Gam[ \sigcl ] = -(V T)  \sig^2 \curly{ \frac{1}{2 g^2} + \frac{N}{4\pi} \brac{ \ln(\frac{\sig^2}{M^2}) - 1 } }.
	}
	Referring to p.~3 of Lecture~11, we can apply the Callan-Symanzik equation to $\Gam$.   The Callan-Symanzik equation is P\&S~(12.41),
	\eq{
		\brac{ M \pdv{M} + \bet(\lam) \pdv{\lam} + n \gam(\lam) } G^{(n)}(\{ x_i \}; M, \lam) = 0.
	}
	For our problem, $\gam$ is 0 because there are no field insertions.  That is, we have
	\eq{
		\brac{ M \pdv{M} + \bet(g) \pdv{g} } \Gam[ \phicl ] = 0.
	}
	Using Eq.~\refeq{Gam}, note that
	\al{
		\pdv{\Gam}{M} &= (V T) \frac{N \sig^2}{2 \pi M}, &
		\pdv{\Gam}{g} &= (V T) \frac{\sig^2}{g^3}.
	}
	Then
	\eq{
		0 = (V T) \paren{ \frac{N \sig^2}{2 \pi} + \bet(g) \frac{\sig^2}{g^3} }
		\qimplies
		\ans{ \betg = -\frac{N g^3}{2\pi}. }
	}
	This model is asymptotically free because the $\bet$ function is proportional to $-g^3$~\cite[pp.~424--425]{Peskin}.
	
	In 2(e) of Homework~4, we found that the vacuum expectation value of $\sig$ was
	\eq{
		\sig = \pm M e^{-\pi / N g^2} = \pm v.
	}
	We showed that the vacuum expectation value does not depend on the renormalization condition chosen.  This means that we can increase $M \to 0$ while holding $\sig$ constant, and see that $g \to 0$ logarithmically.  This is indicative of an asymptotically-free theory~\cite[p.~425]{Peskin}. \qed
}



\vfill
In addition to the course lecture notes, I consulted Jackson's \emph{Classical Electrodynamics} while writing up these solutions.
\end{document}