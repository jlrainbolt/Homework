\newcommand{\phio}{\phi_0}
\newcommand{\cE}{\mathcal{E}}
\newcommand{\cEint}{\cE_\text{int}}
\newcommand{\vP}{\vb{P}}
\newcommand{\evP}{\ev{\vP}}
\newcommand{\divP}{\div{\evP}}

\newcommand{\tfor}{\quad \text{for }}
\newcommand{\Ytotv}{Y_{2 0}\tv}

\begin{statement}{}
	A dielectric ball of radius $R$ and dielectric constant $\eps$ is placed in the external electrostatic potential $\phio = \alp (2z^2 - x^2 - y^2)$ where $\alp$ is a constant, with the center of the ball at $\vx = 0$.
\end{statement}

\begin{problem}
	Find the total electrostatic potential $\phi$ everywhere.
	
	Hint: It is useful to note that the external potential is proportional to $r^2 \, \Ytotv$.  This should allow you to determine/guess the form of the total potential inside and outside the dielectric up to unknown constants, which can then be determined by matching.
\end{problem}

\begin{solution}
	Firstly, note that
	\beq
		\Ytotv = \frac{1}{4} \sqrt{\frac{5}{\pi}} (3 \cos^2\tht - 1),
	\eeq
	and so
	\beq
		\phio = \alp r^2 (3 \cos^2\tht - 1) = 4 \alp r^2 \sqrt{\frac{\pi}{5}} \Ytotv \equiv \beta r^2 \, \Ytotv,
	\eeq
	where we have defined $\beta \equiv 4 \alp \sqrt{\pi/5}$.

	As in problem~\ref{1}, $\evrhof = 0$ so we need to solve Lapace's equation, which has general solutions given by \refeq{lapsol}.  In the region $r < R$, we must have $\Blm = 0$ because $1/r^{l+1}$ is undefined at the origin.  For the region $r > R$, we expect the external potential to dominate as $r \to \infty$, so may invoke the boundary condition at infinity:
	\beq
		\lim_{r \to \infty} \phi(r, \tht, \vph) = \phio = \beta r^2 \, \Ytotv.
	\eeq
	This implies that the only nonzero $\Alm$ here is $A_{2 0} = \beta$.  Thus we have
	\beq
		\evphi\!(r, \tht, \vph) = \begin{cases}
			\dsum_{l,m} \Alm r^l \, \Ylm\tv & \tif r \leq R, \\[2ex]
			\beta r^2 \, \Ytotv + \dsum_{l,m} \dfrac{\Blm}{r^{l+1}} \, \Ylm\tv & \tif r \geq R.
		\end{cases}
	\eeq
	
	To solve for the remaining coefficients, we invoke the boundary conditions at $r = R$.  Firstly, $\evphi$ must be continuous at the boundary.  This gives us
	\beq
		\evphi\!(R, \tht, \vph) = \sum_{l,m} \Alm R^l \, \Ylm\tv = \beta R^2 \, \Ytotv + \sum_{l,m} \dfrac{\Blm}{R^{l+1}} \, \Ylm\tv,
	\eeq
	so
	\begin{align} \label{c1}
		A_{2 0} &= \beta + \dfrac{B_{2 0}}{R^{5}}, &
		\Alm &= \frac{\Blm}{R^{l+3}} \tfor (l,m) \neq (2,0).
	\end{align}
	
	Secondly, we require that $\nh \cdot \evD$ is also continuous at the boundary, where $\evD$ is defined in \refeq{D}.  Here $\nh = \rh$, so we are only concerned with the $r$ component of $\evE$.  Applying $\evE = -\grad\!\evphi$, we have
	\beq
		\evEr\!(r, \tht, \phi) = \begin{cases}
			\dsum_{l,m} \Alm l r^{l-1} \, \Ylm\tv & \tif r \leq R, \\[2ex]
			2 \beta r \, \Ytotv - \dsum_{l,m} (l + 1) \dfrac{\Blm}{r^{l+2}} \, \Ylm\tv & \tif r \geq R.
		\end{cases}
	\eeq
	Then we need to satisfy
	\beq
		\rh \cdot \evD\!(R, \tht, \vph) = \eps \sum_{l,m} \Alm l R^{l-1} \, \Ylm\tv = 2 \beta R \, \Ytotv - \sum_{l,m} (l + 1) \dfrac{\Blm}{R^{l+2}} \, \Ylm\tv,
	\eeq
	which stipulates
	\begin{align} \label{c2}
		A_{2 0} &= \frac{1}{\eps} \left( \beta - \frac{3}{2} \frac{B_{2 0}}{R^{5}} \right), &
		\Alm &= -\frac{1}{\eps} \frac{(l + 1)}{l} \frac{\Blm}{R^{2l+1}} \tfor (l,m) \neq (2,0).
	\end{align}
	Eliminating $\Blm$ from \refeq{c1} and \refeq{c2}, we obtain
	\begin{align*}
		A_{2 0} &= \frac{5 \beta}{2\eps + 3}, &
		\Alm &= 0 \tfor (l,m) \neq (2,0),
	\end{align*}
	and substituting back into \refeq{c1} yields
	\begin{align*}
		B_{2 0} &= 2 \beta R^5 \frac{1 - \eps}{2\eps + 3}, &
		\Blm &= 0 \tfor (l,m) \neq (2,0).
	\end{align*}
	
	Finally, the total electrostatic potential everywhere is
	\beqn \label{phisph}
		\evphi\!(r, \tht, \vph) = \alp r^2 (3 \cos^2\tht - 1) \times \begin{cases}
			\dfrac{5}{2\eps + 3} & \tif r \leq R, \\[2ex]
			1 + 2 \dfrac{1 - \eps}{2\eps + 3} \dfrac{R^5}{r^5} & \tif r \geq R,
		\end{cases}
	\eeqn
	or, in Cartesian coordinates,
	\beq
		\evphi\!(x, y, z) = \alp (2z^2 - x^2 - y^2) \times \begin{cases}
			\dfrac{5}{2\eps + 3} & \tif r \leq R, \\[2ex]
			1 + 2 \dfrac{1 - \eps}{2\eps + 3} \dfrac{R^5}{\sqrt{x^2 + y^2 + z^2}} & \tif r \geq R.
		\end{cases}
	\eeq
\end{solution}
\vfix



\newcommand{\sE}{\mathscr{E}}
\newcommand{\sEint}{\sE_\text{int}}
\newcommand{\vEo}{\vE_0}
\newcommand{\thh}{\boldsymbol{\hat{\tht}}}
\newcommand{\phh}{\boldsymbol{\hat{\vph}}}
\newcommand{\sint}{\sin\tht}
\newcommand{\Eth}{E_\tht}
\newcommand{\Eph}{E_\vph}
\newcommand{\evEth}{\ev{\Eth}}
\newcommand{\evEph}{\ev{\Eph}}
\newcommand{\Eo}{E_0}
\newcommand{\Eor}{{\Eo}_r}
\newcommand{\Eoth}{{\Eo}_\tht}
\newcommand{\Eoph}{{\Eo}_\vph}

\newcommand{\evPr}{\ev{P_r}}
\newcommand{\evPth}{\ev{P_\tht}}
\newcommand{\evPph}{\ev{P_\vph}}
\newcommand{\dph}{\dd{\vph}}
\newcommand{\intRi}{\int_R^\infty}
\newcommand{\intoR}{\int_0^R}

\begin{problem}
	Calculate the interaction energy between the field produced by the dielectric and the external field.  Assume that the potential arises from ``distant charges'' so that the formula for $\cEint$ given in class and the notes can be used.
\end{problem}

\begin{solution}
	Equation~(3.34) in the lectures notes gives the interaction energy:
	\beq
		\sEint = \int (\evrhof \!\phio - \evP \cdot \vEo) \dcx,
	\eeq
	where $\vEo$ is the electric field due to the external potential $\phio$.  Again, $\evrhof = 0$.  For our assumption of a linear, homogeneous, and isotropic dielectric,
	\beqn \label{P}
		\evP = \chi \!\evE
	\eeqn
	by Eq.~(3.19), where
	\beq
		\eps = 1 + 4\pi \chi
	\eeq
	from Eq.~(3.21).
	
	The gradient in spherical coordinates is
	\beqn \label{grad}
		\grad = \pdv{}{r} \,\rh + \frac{1}{r} \pdv{}{\tht} \,\thh + \frac{1}{r \sint} \pdv{}{\vph} \, \phh.
	\eeqn
	Differentiating \refeq{phisph} for $r \leq R$,
	\begin{align} \label{Ein}
		\evEr &= 2 \alp \frac{5}{2\eps + 3} r (3 \cos^2\tht - 1), &
		\evEth &= -6 \alp \frac{5}{2\eps + 3} r \cost \sint, &
		\evEph &= 0.
	\end{align}
	For the external field,
	\begin{align} \label{E0}
		\Eor &= 2 \alp r (3 \cos^2\tht - 1), &
		\Eoth &= -6 \alp r \cost \sint, &
		\Eoph &= 0.
	\end{align}
	Note that $\evP = (\eps - 1) \!\evE\! / 4 \pi$, so
	\beq
		\evP \cdot \vEo = 4 \alp^2 \frac{\eps - 1}{4\pi} \frac{5}{2\eps + 3} r^2 \left[ (3 \cos^2\tht - 1)^2 + 9 \cos^2\tht \sin^2\tht \right].
	\eeq
	Then
	\begin{align*}
		\sEint &= -\int \evP \cdot \vEo \dcx
		= 4 \alp^2 \frac{1 - \eps}{4\pi} \frac{5}{2\eps + 3} \intotp \dph \intono (3 \cos^2\tht + 1) \dct \intoR r^4 \dr \\
		&= 4 \alp^2 \frac{1 - \eps}{4\pi} \frac{5}{2\eps + 3} \bigg[ \vph \bigg]_0^{2\pi} \bigg[ \cos^3\tht + \cost \bigg]_{-1}^1 \left[ \frac{r^5}{5} \right]_0^R
		= 4 \alp^2 \frac{1 - \eps}{4\pi} \frac{5}{2\eps + 3} (2\pi) (4) \frac{R^5}{5} \\
		&= 8 \alp^2 \frac{1 - \eps}{2\eps + 3} R^5.
	\end{align*}
\end{solution}
\vfix



\newcommand{\vF}{\vb{F}}

\begin{problem}
	Calculate the total force needed to hold the dielectric ball in place.
\end{problem}

\begin{solution}
	Equation~(3.26) in the lecture notes gives the total force on a dielectric:
	\beq
		\vF = \int [\evrhof\! \vEo + (\evP \cdot \grad) \vEo] \dcx,
	\eeq
	where we note that there is no contribution from the dielectric's self field in electrostatics.  Recall that $\evrhof = 0$.  Substituting in \refeq{P} and \refeq{D},
	\beq
		\vF = \int (\chi\!\evE \cdot \grad) \vEo \dcx
		= \int \frac{\chi}{\eps} (\div{\evD}) \vEo \dcx
		= 0,
	\eeq
	because
	\beq
		\div{\evD} = -4\pi \!\evrhof
	\eeq
	according to Eq.~(3.17).
\end{solution}