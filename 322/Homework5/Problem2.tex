\begin{statement}{}
	A particle of charge $\qq$ moves with velocity $v$ in a circular orbit of radius $R$ about the origin in the $xy$ plane, such that its $\vph$ coordinate varies as $\vph = \omg t$, with $\omg = v / R$.  Assume that $v \ll c$.  Another particle of charge $\qw$ is at rest at point $\vx$, where $\absx \gg R$.  To order $1 / \absx$, find the force $\vF$ on the particle of charge $\qw$ at time $t$.
\end{statement}

\begin{solution}
	The Lorentz force equation, Eq.~(1.25), is written
	\beqn \label{lorentz}
		\vF = q \left( \vE + \frac{\vv}{c} \cross \vB \right),
	\eeqn
	where $\vv$ is the velocity of the charge $q$ on which the force is exerted, and $\vE$ and $\vB$ are the total electric and magnetic fields.  For this problem, we are interested in the force acting on a stationary point charge $\qw$, so $\vv_2 = 0$.  Additionally, we do not have to consider the self-field contribution to $\vE$, since static charge distributions do not experience any self force.  Thus we need only find the electric field due to $\qq$, $\vEq$.  The multipole expansion of the electric field in electrodynamics is given by Eq.~(5.70),
	\beqn \label{Efield}
		\vE\tx = \frac{1}{c^2 \absx} \left[ \left(\xh \vdot \dv[2]{\vp}{t} \right) \xh - \dv[2]{\vp}{t} \right]_\ret + \order{\frac{1}{\absx^2}},
	\eeqn
	where $\xh = \vx / \absx$ is the unit vector in the direction of the point at which we are evaluating the field, and $\vp$ is the dipole moment defined by \refeq{dipole}.  In addition, \refeq{Efield} relies upon the assumption that the velocity of $\qq$, $v$, satisfies $v \ll c$.
	
	The position of $\qq$ at time $t$ can be expressed as
	\beq
		\vxqt = R \cos(\omg t) \,\xh + R \sin(\omg t) \,\yh,
	\eeq
	so the charge density for $\qq$ everywhere is
	\beq
		\rhoq\tx = \qq \,\del\big( \vx - \vxqt \big).
	\eeq
	Then the dipole moment $\vpq\tx$ is
	\beq
		\vpq\tx = \int \vx \,\rhoq\tx \dcx
		= \qq \int \vx \,\del\big(\vx - \vxqt \big) \dcx
		= \qq \vxqt
		= \qq R \cos(\omg t) \,\xh + \qq R \sin(\omg t) \,\yh,
	\eeq
	and so its second time derivative is
	\beq
		\dv[2]{\vpqt}{t} = \dv{}{t} \left( \dv{\vpq}{t} \right)
		= \dv{}{t} \bigg( \!-\!\qq R \omg \sin(\omg t) \,\xh + \qq R \omg \cos(\omg t) \,\yh \bigg)
		= -\qq R \omg^2 \cos(\omg t) \,\xh - \qq R \omg^2 \sin(\omg t) \,\yh.
	\eeq
	
	To this order, the retarded time $t'$ is defined
	\beqn \label{rettime}
		t' = t - \frac{\absx}{c}.
	\eeqn
	In \refeq{Efield}, let $\vx \to \vr$.  Then $\xh \to \rh$, which is the radial unit vector, and $\absx \to r$.  To first order in $1/\absx$, we get
	\begin{align*}
		\vE\tx &= \frac{1}{c^2 r} \left[ -\frac{\qq R \omg^2}{r} \left[ \cos(\omg t') (\rh \vdot \xh) + \sin(\omg t') (\rh \vdot \yh) \right] \vr + \qq R \omg^2 \big[ \cos(\omg t') \,\xh + \sin(\omg t') \,\yh \big] \right]_\ret \\
		&= \qq \frac{R \omg^2}{c^2 r} \bigg[ \cos(\omg t') \big[ \xh - (\rh \vdot \xh) \rh \big] + \sin(\omg t') \big[ \yh - (\rh \vdot \yh) \rh \big] \bigg]_\ret \\
		&= \qq \frac{R \omg^2}{c^2 r} \left\{ \cos(\omg t - \frac{\omg \absx}{c}) \big[ \xh - (\rh \vdot \xh) \rh \big] + \sin(\omg t - \frac{\omg \absx}{c}) \big[ \yh - (\rh \vdot \yh) \rh \big] \right\}.
	\end{align*}
	Note that
	\begin{align*}
		\xh &= \sin\tht \cos\vph \,\rh + \cos\tht \cos\vph \,\thh - \sin\vph \,\phh, &
		\yh &= \sin\tht \sin\vph \,\rh + \cos\tht \sin\vph \,\thh + \cos\vph \,\phh,
	\end{align*}
	so
	\begin{align*}
		\xh - (\rh \vdot \xh) \rh &= \xh - \sin\tht \cos\vph \,\rh
		= \cos\tht \cos\vph \,\thh - \sin\vph \,\phh, \\
		\yh - (\rh \vdot \yh) \rh &= \yh - \sin\tht \sin\vph \,\rh
		= \cos\tht \sin\vph \,\thh + \cos\vph \,\phh,
	\end{align*}
	and then
	\beq
		\vE\tx = \qq \frac{R \omg^2}{c^2 r} \left[ \cos(\omg t - \frac{\omg r}{c}) (\cos\tht \cos\vph \,\thh - \sin\vph \,\phh) + \sin(\omg t - \frac{\omg r}{c}) (\cos\tht \sin\vph \,\thh + \cos\vph \,\phh) \right].
	\eeq
	Applying \refeq{lorentz} with $\vv = 0$, we have
	\beq
		\vF\tx = \qq \qw \frac{R \omg^2}{c^2 r} \left[ \cos(\omg t - \frac{\omg r}{c}) (\cos\tht \cos\vph \,\thh - \sin\vph \,\phh) + \sin(\omg t - \frac{\omg r}{c}) (\cos\tht \sin\vph \,\thh + \cos\vph \,\phh) \right].
	\eeq
	\vfix
\end{solution}