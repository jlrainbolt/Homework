\documentclass[11pt]{article}
\usepackage{geometry, titlesec}
\usepackage[parfill]{parskip}
\usepackage[italicdiff]{physics}
\usepackage{amsfonts, amsthm, mathrsfs}
\usepackage[cm]{fullpage}
\usepackage{fancyhdr}
\usepackage{enumitem}
\usepackage{xcolor, soul}
\usepackage{siunitx}
\allowdisplaybreaks

\renewcommand{\thesubsection}{\thesection.\alph{subsection}}
%\renewcommand{\vb}[1]{\mathbf{#1}}
\newcommand{\vfix}{\vspace{-\baselineskip}}

\makeatletter
\renewcommand*\env@cases[1][1.2]{%
  \let\@ifnextchar\new@ifnextchar
  \left\lbrace
  \def\arraystretch{#1}%
  \array{@{}l@{\quad}l@{}}%
}
\makeatother
 
\renewcommand{\footrulewidth}{.2pt}
%\setlist[enumerate]{leftmargin=*}
\pagestyle{fancy}
\fancyhf{}
\lhead{\textbf{Physics 322 Homework 5}}
\rhead{Lacey Rainbolt}
\setlength{\headheight}{11pt}
\setlength{\headsep}{11pt}
\setlength{\footskip}{24pt}
\lfoot{\today}
\rfoot{\thepage}

\titleformat{\section}[runin]{\normalfont\large\bfseries}{Problem \thesection.}{1em}{}
\titleformat{\subsection}[runin]{\normalfont\large\bfseries}{\thesubsection}{1em}{}
\titleformat{\subparagraph}[leftmargin]{\normalfont\normalsize\bfseries}{}{0pt}{}

\newcommand{\refeq}[1]{(\ref{#1})}

\newcommand{\beq}{\begin{equation*}}
\newcommand{\eeq}{\end{equation*}}

\newcommand{\beqn}{\begin{equation}}
\newcommand{\eeqn}{\end{equation}}

\newcommand{\blg}{\begin{align*}}
\newcommand{\elg}{\end{align*}}

\newcommand{\qimplies}{\quad \implies \quad}


\newenvironment{statement}[1]
{
	\section{#1}
	\color{darkgray}
	\ignorespaces
}
{
%    \smallskip
}

\newenvironment{problem}
{
	\subsection{}
	\color{darkgray}
%	\paragraph{Problem.}
    \ignorespaces
}
{

}

\newenvironment{solution}
{
    \paragraph{Solution.}
    \ignorespaces
}
{
    \bigskip
}



\begin{document}

\newcommand{\vaa}{\vb{a}}
\newcommand{\vp}{\vb{p}}
\newcommand{\vv}{\vb{v}}
\newcommand{\vx}{\vb{x}}
\newcommand{\vB}{\vb{B}}
\newcommand{\vE}{\vb{E}}
\newcommand{\vF}{\vb{F}}
\newcommand{\vJ}{\vb{J}}
\newcommand{\vomg}{\boldsymbol{\omega}}

\newcommand{\del}{\delta}
\newcommand{\vph}{\varphi}
\newcommand{\tht}{\theta}
\newcommand{\omg}{\omega}

\newcommand{\rhox}{\rho(x)}
\newcommand{\dcx}{\dd[3]{x}}

\newcommand{\cV}{\mathcal{V}}
\newcommand{\intcV}{\int_\cV}
\newcommand{\intS}{\int_S}
\newcommand{\dS}{\dd{S}}

\renewcommand{\xi}{x_i}
\newcommand{\limRi}{\lim_{R \to \infty}}

\newcommand{\nh}{\vb{\hat{n}}}
\newcommand{\rh}{\vb{\hat{r}}}
\newcommand{\xh}{\vb{\hat{x}}}
\newcommand{\zh}{\vb{\hat{z}}}
\newcommand{\phh}{\boldsymbol{\hat{\vph}}}

\state{(Jackson 9.8)}{\ 
	%\emph{Hint:} The electromagnetic angular momentum density comes from more than the transverse (radiation zone) components of the fields.
}

%
%	Jackson 9.8(a)
%

\prob{}{
	Show that a classical oscillating electric dipole $\vp$ with fields given by
	\aln{ \label{fields1}
		\vH &= \frac{c k^2}{4\pi} (\nh \cross \vp) \frac{e^{i k r}}{r} \paren{ 1 - \frac{1}{i k r} }, &
		\vE &= \frac{1}{4\pi \epso} \curly{ k^2 (\nh \cross \vp) \cross \nh \frac{e^{i k r}}{r} + [ 3 \nh (\nh \vdot \vp) - \vp ] \paren{ \frac{1}{r^3} - \frac{i k}{r^2} } e^{i k r} },
	}
	radiates electromagnetic angular momentum to infinity at the rate
	\eq{
		\dv{\vL}{t} = \frac{k^3}{12 \pi \epso} \Im[ \vp^* \cross \vp ].
	}
	\vfix
}

\sol{
	According to Jackson~(9.20), the time-averaged angular momentum density is
	\eq{
		\vl = \frac{\Re[ \vx \cross (\vE \cross \vHs)}{2 c^2}.
	}
	One of the vector identities on the inside cover of Jackson is $\vaa \cross (\vbb \cross \vcc) = (\vaa \vdot \vcc) \vbb - (\vaa \vdot \vbb) \vcc$, so
	\eqn{l1}{
		\vl = \frac{(\vx \vdot \vHs) \vE - (\vx \vdot \vE) \vHs}{2 c^2}.
	}
	From Eq.~\refeq{fields1}, note that
	\eq{
		\vx \vdot \vHs \propto \vx \vdot (\nh \cross \vps)
		= \vps \vdot (\vx \cross \nh)
		= \vO,
	}
	where we have used the identity $\vaa \vdot (\vbb \cross \vcc) = \vcc \vdot (\vaa \cross \vbb)$ and the fact that $\nh$ points in the $\vx$ direction.  For $\vx \vdot \vE$, note that
	\al{
		\vx \vdot [ (\nh \cross \vp) \cross \nh ] &= -\vx \vdot [ \nh \cross (\nh \cross \vp) ]
		= -\vx \vdot [ (\nh \vdot \vp) \nh - (\nh \vdot \nh) \vp ]
		= -(\nh \vdot \vp) (\vx \vdot \nh) + \vx \vdot \vp \\
		&= -r (\nh \vdot \vp) + \vx \vdot \vp
		= \vx \vdot \vp - \vx \vdot \vp
		= 0, \\[1.5ex]
		\vx \vdot [ 3 \nh (\nh \vdot \vp) - \vp ] &= 3 (\vx \vdot \nh) (\nh \vdot \vp) - \vx \vdot \vp
		= 3r (\nh \vdot \vp) - \vx \vdot \vp
		= 3(\vx \vdot \vp) - \vx \vdot \vp
		= 2(\vx \vdot \vp),
	}
	since $\abs{\vx} = r$ and $\vx = r \,\nh$.  Then
	\eq{
		\vx \vdot \vE = \frac{1}{2\pi \epso} (\vx \vdot \vp) \paren{ \frac{1}{r^3} - \frac{i k}{r^2} } e^{i k r}
		= \frac{1}{2\pi \epso} (\nh \vdot \vp) \paren{ \frac{1}{r^2} - \frac{i k}{r} } e^{i k r}.
	}
	
	With these substitutions, Eq.~\refeq{l1} becomes
	\al{
		\vl &= -\frac{(\vx \vdot \vE) \vHs}{c^2}
		= -\frac{1}{4\pi \epso c^2} (\nh \vdot \vp) \paren{ \frac{1}{r^2} - \frac{i k}{r} } e^{i k r} \frac{c k^2}{4\pi} (\nh \cross \vps) \frac{e^{-i k r}}{r} \paren{ 1 + \frac{1}{i k r} } \\
		&= -\frac{k^2}{16\pi^2 \epso c r} (\nh \vdot \vp) (\nh \cross \vps) \paren{ \frac{1}{r^2} - \frac{i k}{r} } \paren{ 1 - \frac{i}{k r} }
		= -\frac{k^2}{16\pi^2 \epso c} (\nh \vdot \vp) (\nh \cross \vps) \paren{ \frac{1}{r^2} - \frac{i}{k r^3} - \frac{i k}{r} - \frac{1}{r^2} } \\
		&= -\frac{i k^2}{16\pi^2 \epso c r} (\nh \vdot \vp) (\nh \cross \vps) \paren{ \frac{1}{k r^3} + \frac{k}{r^2} }
		= \frac{i k^3}{16\pi^2 \epso c r^2} (\nh \vdot \vp) (\nh \cross \vps) \paren{ \frac{1}{k^2 r^2} + 1 }.
	}
	
	Let $\vL$ be the angular momentum radiated to a distance $R$.  Then
	\eq{
		\vL = \int_R \vl(r) \ddcx
		= \intopi \intotp \intoR \vl(r) \,r^2 \sin\tht \ddr \ddphi \dd\tht,
	}
	and the time derivative is
	\aln{
		\dv{\vL}{t} &= \dv{t}(\intopi \intotp \intoR \vl(r) \,r^2 \sin\tht \ddr \ddphi \dd\tht)
		= \dv{r}{t} \dv{r}(\intopi \intotp \intoR \vl(r) \,r^2 \sin\tht \ddr \ddphi \dd\tht) \notag \\
		&= c \intopi \intotp \vl(r) \,r^2 \sin\tht \ddphi \dd\tht
		= \frac{i k^3}{16\pi^2 \epso} \paren{ \frac{1}{k^2 r^2} + 1 } \intopi \intotp (\nh \vdot \vp) (\nh \cross \vps) \sin\tht \ddphi \dd\tht. \label{dLdt}
	}
	Note that
	\eq{
		[ (\nh \vdot \vp) (\nh \cross \vps) ]_i = \sumje n_j p_j (\nh \cross \vps)_i
		= \sumje \sumke \sumle \epsikl n_j p_j n_k p_l^*,
	}
	so
	\eq{
		\dv{L_i}{t} \propto \sumje \sumke \sumle \epsikl p_j p_l^* \int n_j p_k \ddOmg
		= \sumje \sumke \sumle \epsikl p_j p_l^* \frac{4\pi}{3} \del_{jk}
		= \frac{4\pi}{3} \epsikl p_k p_l^*
		= \frac{4\pi}{3} (\vp \cross \vps)_i,
	}
	where we have used Jackson~(9.47), $\int n_\bet n_\gam \ddOmg = 4\pi \del_{\bet \gam} / 3$.  Making this substitution into Eq.~\refeq{dLdt},
	\eq{
		\dv{\vL}{t} = \frac{i k^3}{6\pi \epso} \paren{ \frac{1}{k^2 r^2} + 1 } (\vp \cross \vps).
	}
	Taking the limit as $r \to \infty$, we find
	\eqn{ans1a}{
		\dv{\vL}{t} = \Re\!\brac{ \frac{i k^3}{12\pi \epso} (\vp \cross \vps) }
		= \Re\!\brac{ -\frac{i k^3}{12\pi \epso} (\vps \cross \vp) }
		= \ans{ \frac{k^3}{12\pi \epso} \Im[ \vps \cross \vp ], }
	}
	as desired. \qed
}

%
%	Jackson 9.8(b)
%

\prob{}{
	What is the ratio of angular momentum radiated to energy radiated?  Interpret.
}

\sol{
	According to Jackson~(9.24), the total power radiated by an oscillating electric dipole $\vp$ is
	\eq{
		P = \dv{E}{t}
		= \frac{c^2 \Zo k^4}{12 \pi} \abs{\vp}^2.
	}
	Then the ratio of angular momentum radiated to energy radiated is
	\eq{
		\frac{\dv*{\vL}{t}}{\dv*{E}{t}} = \frac{k^3}{12\pi \epso} \Im[ \vps \cross \vp ] \frac{12 \pi}{c^2 \Zo k^4 \abs{\vp}^2}
		= \frac{1}{\epso} \Im[ \vps \cross \vp ] \frac{1}{c^2 \Zo k \abs{\vp}^2}
		= \ans{ \frac{\Im[ \vps \cross \vp ]}{\omg \abs{\vp}^2}, }
	}
	where we have used $\Zo = \sqrt{\muo / \epso} = 1 / \sqrt{\epso^2 c^2} = 1 / \epso c$, $c^2 = 1 / (\epso \muo)$, and $\omg = k c$.
	
	In the limit of high frequency, $(\dv*{\vL}{t}) / (\dv*{E}{t}) \to 0$.  In this scenario, the energy radiated dominates over the angular momentum radiated.  Likewise, in the limit of low frequency, $(\dv*{\vL}{t}) / (\dv*{E}{t}) \to \infty$, meaning that angular momentum radiation dominates.  This is sensible because rotational kinetic energy $E \propto \omg^2$, while angular momentum $L \propto \omg$.
}

%
%	Jackson 9.8(c)
%

\prob{}{
	For a charge $e$ rotating in the $xy$ plane at radius $a$ and angular speed $\omg$, show that there is only a $z$ component of radiated angular momentum with magnitude $\dv*{\Lz}{t} = e^2 k^3 a^2 / 6 \pi \epso$.  What about a charge oscillating along the $z$ axis?
}

\sol{
	We know from Homework~5 that the position of a point charge rotating counterclockwise in the $xy$ plane is
	\eq{
		\vx(t) = a \cos(\omg t) \,\vx + a \sin(\omg t) \,\yh.
	}
	\clearpage
	Then the charge distribution is
	\eq{
		\rho(\vx, t) = e \del[ x - a \cos(\omg t) ] \,\del[ y - a \sin(\omg t) ] \,\del(z).
	}
	
	According to Jackson~(4.8), the dipole moment is defined
	\eq{
		\vp = \int \vx' \,\rho(\vx') \ddcxp.
	}
	The components of $\vp$ for the point charge are then
	\al{
		\px &= e \iiint x \,\del[ x - a \cos(\omg t) ] \,\del[ y - a \sin(\omg t) ] \,\del(z) \ddx \ddy \ddz
		= e a \cos(\omg t), \\
		\py &= e \iiint y \,\del[ x - a \cos(\omg t) ] \,\del[ y - a \sin(\omg t) ] \,\del(z) \ddx \ddy \ddz
		= e a \sin(\omg t), \\
		\pz &= e \iiint z \,\del[ x - a \cos(\omg t) ] \,\del[ y - a \sin(\omg t) ] \,\del(z) \ddx \ddy \ddz
		= 0,
	}
	so we can write $\vp = e a \,e^{-i \omg t} (\xh + i\,\yh).$  Substituting into Eq.~\refeq{ans1a},
	\al{
		\dv{\vL}{t} &= \Re\!\brac{ \frac{i k^3}{12\pi \epso} e^2 a^2 e^{-i \omg t} e^{i \omg t} [ (\xh + i\,\yh) \cross (\xh - i\,\yh) ] }
		= \Re\!\brac{ \frac{i e^2 k^3 a^2}{12\pi \epso} (-2i \,\xh \cross \yh) }
		= \Re\!\brac{ \frac{e^2 k^3 a^2}{6\pi \epso} \,\zh } \\
		&= \ans{ \frac{e^2 k^3 a^2}{6\pi \epso} \cos(\omg t) \,\zh, }
	}
	as desired. \qed
	
	A charge oscillating along the $z$ axis with amplitude $a$ has the charge density
	\eq{
		\rho(\vx, t) = e a \,\del(x) \,\del(y) \,\del[ z - \cos(\omg t) ],
	}
	which gives the dipole moment
	\al{
		\px &= e a \iiint x \,\del(x) \,\del(y) \,\del[ z - \cos(\omg t) ] \ddx \ddy \ddz
		= 0, \\
		\py &= e a \iiint y \,\del(x) \,\del(y) \,\del[ z - \cos(\omg t) ] \ddx \ddy \ddz
		= 0, \\
		\pz &= e a \iiint z \,\del(x) \,\del(y) \,\del[ z - \cos(\omg t) ] \ddx \ddy \ddz
		= e a \cos(\omg t).
	}
	In complex notation, $\vp = e a \,e^{-i\omg t} \,\zh$.  Substituting into Eq.~\refeq{ans1a}, we find
	\eq{
		\dv{\vL}{t} = \Re\!\brac{ \frac{i k^3}{12\pi \epso} e^2 a^2 e^{-i \omg t} e^{i \omg t} (\zh \cross \zh) }
		= \ans{ \vO. }
	}
	So we see that a charge undergoing linear motion does not lead to a radiated angular momentum, which is sensible.
}

%
%	Jackson 9.8(d)
%

\prob{}{
	What are the results corresponding to Probs.~{1(a)} and {1(b)} for magnetic dipole radiation?
}

\sol{
	The radiation fields for a magnetic dipole are given by Jackson~(19.35--36),
	\al{
		\vH &= \frac{1}{4\pi} \curly{ k^2 (\nh \cross \vm) \cross \nh \frac{e^{i k r}}{r} + [ 3 \nh (\nh \vdot \vm) - \vm ] \paren{ \frac{1}{r^3} - \frac{i k}{r^2} } e^{i k r} }, &
		\vE &= -\frac{\Zo}{4\pi} k^2 (\nh \cross \vm) \frac{e^{i k r}}{r} \paren{ 1 - \frac{1}{i k r} }.
	}
	\clearpage
	Comparing with Eq.~\refeq{fields1}, we see that $\vH \to -\vE / \Zo$, $\vE \to \Zo \vH$, and $\vp \to \vm / c$ as stated in the book~\cite[p.~413]{Jackson}.  Making these substitutions, the results of Probs.~{1.1(a)} and {(b)} become
	\al{
		\ans{ \dv{\vL}{t}\ }&\ans{= \frac{\muo k^3}{12\pi} \Im[ \vms \cross \vm ], } &
		\ans{ \frac{\dv*{\vL}{t}}{\dv*{E}{t}}\ }&\ans{= \frac{\Im[ \vms \cross \vm ]}{\omg \abs{\vm}^2} }
	}
	where we have used $\mu = 1 / \epso c^2$.
}



\renewcommand{\qq}{q_1}
\newcommand{\qw}{q_2}
\newcommand{\absx}{\abs{\vx}}
\newcommand{\tx}{(t, \vx)}

\newcommand{\rhoq}{\rho_1}
\newcommand{\vEq}{\vE_1}
\newcommand{\vpq}{\vp_1}
\newcommand{\vpqt}{\vpq(t)}
\newcommand{\vxq}{\vx_1}
\newcommand{\vxqt}{\vxq(t)}

\newcommand{\ret}{\text{ret}}

\begin{statement}{}
	A particle of charge $\qq$ moves with velocity $v$ in a circular orbit of radius $R$ about the origin in the $xy$ plane, such that its $\vph$ coordinate varies as $\vph = \omg t$, with $\omg = v / R$.  Assume that $v \ll c$.  Another particle of charge $\qw$ is at rest at point $\vx$, where $\absx \gg R$.  To order $1 / \absx$, find the force $\vF$ on the particle of charge $\qw$ at time $t$.
\end{statement}

\begin{solution}
	The Lorentz force equation, Eq.~(1.25), is written
	\beq
		\vF = q \left( \vE + \frac{\vv}{c} \cross \vB \right),
	\eeq
	where $\vv$ is the velocity of the charge $q$ on which the force is exerted, and $\vE$ and $\vB$ are the total electric and magnetic fields.  For this problem, we are interested in the force acting on a stationary point charge $\qw$, so $\vv_2 = 0$.  Additionally, we do not have to consider the self-field contribution to $\vE$, since static charge distributions do not experience any self force.  Thus we need only find the electric field due to $\qq$, $\vEq$.  The multipole expansion of the electric field in electrodynamics is given by Eq.~(5.70),
	\beqn \label{Efield}
		\vE\tx = \frac{1}{c^2 \absx} \left[ \left(\xh \vdot \dv[2]{\vp}{t} \right) \xh - \dv[2]{\vp}{t} \right]_\ret + \order{\frac{1}{\absx^2}},
	\eeqn
	where $\xh = \vx / \absx$ is the unit vector in the direction of the point at which we are evaluating the field, and $\vp$ is the dipole moment defined by \refeq{dipole}.  In addition, \refeq{Efield} relies upon the assumption that the velocity of $\qq$, $v$, satisfies $v \ll c$.
	
	The position of $\qq$ at time $t$ can be expressed as
	\beq
		\vxqt = R \cos(\omg t) \,\phh,
	\eeq
	so the charge density for $\qq$ everywhere is
	\beq
		\rhoq\tx = \qq \,\del\big( \vx - \vxqt \big).
	\eeq
	Then the dipole moment $\vpq\tx$ is
	\beq
		\vpq\tx = \int \vx \,\rhoq\tx \dcx
		= \qq \int \vx \,\del\big(\vx - \vxqt \big) \dcx
		= \qq \vxqt
		= \qq R \cos(\omg t) \,\phh,
	\eeq
	and so its second time derivative is
	\beq
		\dv[2]{\vpqt}{t} = \dv{}{t} \left( \dv{\vpq}{t} \right)
		= \dv{}{t} \bigg( \!-\!\qq R \omg \sin(\omg t) \,\phh \bigg)
		= -\qq R \omg^2 \cos(\omg t) \,\phh.
	\eeq
	
	The retarded time $t'$ is defined
	\beq
		t' = t - \frac{\abs{\vx - \vx'}}{c},
	\eeq
	so here
	\beq
		t' = t - \frac{\abs{\vx - \vxqt}}{c}
		= t - \frac{\abs{\vx - R \cos(\omg t) \,\phh}}{c}.
	\eeq
	Since $R \ll \absx$, we can Taylor expand the second term about $R = 0$ as in Eq.~(5.57),
	\beq
		\abs{\vx - \vx'} = \absx - \xh \vdot \vx' + \order{\frac{1}{\absx}}.
	\eeq
	This gives us
	\beq
		\abs{\vx - R \cos(\omg t) \,\phh} \approx \absx - R \cos(\omg t) (\xh \vdot \phh),
	\eeq
	so
	\beq
		t' \approx t - \frac{\absx}{c} - \frac{R \cos(\omg t)}{c} (\xh \vdot \phh),
	\eeq
	\hl{which is really not helpful.}
\end{solution}



\newcommand{\Io}{I_0}
\newcommand{\tz}{(t, z)}

\newcommand{\dt}{\dd{t}}

\begin{statement}{}
	An ``antenna'' is a segment of conducting wire in which a current flows (driven by an external power supply).  Suppose an antenna of length $L$ is placed on the $z$ axis between $z = 0$ and $z = L$, and suppose that the current in the antenna is
	\beqn \label{J3}
		\vJ\tz = \Io \sin(\frac{\pi z}{L}) \cos(\omg t) \,\del(x) \,\del(y) \,\zh.
	\eeqn
	\vfix
\end{statement}

\begin{problem}
	Find the charge density $\rho\tz$ in the antenna.
\end{problem}

\begin{solution}
	From the charge-current conservation law \refeq{continuity}, we have
	\beq
		\rho\tz = -\int \div{\vJ} \dt.
	\eeq
	For $\vJ$ given by \refeq{J3},
	\beq
		\div{\vJ} = \pdv{J_z}{z}
		= \frac{\pi}{L} \Io \cos(\frac{\pi z}{L}) \cos(\omg t) \,\del(x) \,\del(y),
	\eeq
	and so
	\beq
		\rho\tz = -\frac{\pi}{L} \Io \cos(\frac{\pi z}{L}) \,\del(x) \,\del(y) \int \cos(\omg t) \dt
		= -\frac{\pi}{L} \frac{\Io}{\omg} \cos(\frac{\pi z}{L}) \sin(\omg t) \,\del(x) \,\del(y),
	\eeq
	where we have set the integration constant to 0.
\end{solution}



\begin{problem}
	Assume that $\omg L \ll c$.  Find the electric and magnetic fields, $\vE\tz$ and $\vB\tz$, at large distances from the antenna (valid to order $1 / \absx$).
\end{problem}



\vfill
{\small In addition to the course lecture notes, I consulted Griffiths's \emph{Introduction to Electrodynamics}, Jackson's \emph{Classical Electrodynamics}, and K.~T.~McDonald's and D.~K.~Ghosh's notes on electromagnetism while writing up these solutions.}
\end{document}