\documentclass[11pt]{article}
\usepackage{geometry, titlesec}
\usepackage[parfill]{parskip}
\usepackage[italicdiff]{physics}
\usepackage{amsfonts, amsthm, mathrsfs}
\usepackage[cm]{fullpage}
\usepackage{fancyhdr}
\usepackage{enumitem}
\usepackage{xcolor, soul}
\usepackage{siunitx}
\allowdisplaybreaks

\renewcommand{\thesubsection}{\thesection.\alph{subsection}}
%\renewcommand{\vb}[1]{\mathbf{#1}}
\newcommand{\vfix}{\vspace{-\baselineskip}}

\makeatletter
\renewcommand*\env@cases[1][1.2]{%
  \let\@ifnextchar\new@ifnextchar
  \left\lbrace
  \def\arraystretch{#1}%
  \array{@{}l@{\quad}l@{}}%
}
\makeatother
 
\renewcommand{\footrulewidth}{.2pt}
%\setlist[enumerate]{leftmargin=*}
\pagestyle{fancy}
\fancyhf{}
\lhead{\textbf{Physics 322 Homework 5}}
\rhead{Lacey Rainbolt}
\setlength{\headheight}{11pt}
\setlength{\headsep}{11pt}
\setlength{\footskip}{24pt}
\lfoot{\today}
\rfoot{\thepage}

\titleformat{\section}[runin]{\normalfont\large\bfseries}{Problem \thesection.}{1em}{}
\titleformat{\subsection}[runin]{\normalfont\large\bfseries}{\thesubsection}{1em}{}
\titleformat{\subparagraph}[leftmargin]{\normalfont\normalsize\bfseries}{}{0pt}{}

\newcommand{\refeq}[1]{(\ref{#1})}

\newcommand{\beq}{\begin{equation*}}
\newcommand{\eeq}{\end{equation*}}

\newcommand{\beqn}{\begin{equation}}
\newcommand{\eeqn}{\end{equation}}

\newcommand{\blg}{\begin{align*}}
\newcommand{\elg}{\end{align*}}

\newcommand{\qimplies}{\quad \implies \quad}


\newenvironment{statement}[1]
{
	\section{#1}
	\color{darkgray}
	\ignorespaces
}
{
%    \smallskip
}

\newenvironment{problem}
{
	\subsection{}
	\color{darkgray}
%	\paragraph{Problem.}
    \ignorespaces
}
{

}

\newenvironment{solution}
{
    \paragraph{Solution.}
    \ignorespaces
}
{
    \bigskip
}



\begin{document}

\newcommand{\vaa}{\vb{a}}
\newcommand{\vp}{\vb{p}}
\newcommand{\vv}{\vb{v}}
\newcommand{\vx}{\vb{x}}
\newcommand{\vB}{\vb{B}}
\newcommand{\vE}{\vb{E}}
\newcommand{\vF}{\vb{F}}
\newcommand{\vJ}{\vb{J}}
\newcommand{\vomg}{\boldsymbol{\omega}}

\newcommand{\del}{\delta}
\newcommand{\vph}{\varphi}
\newcommand{\tht}{\theta}
\newcommand{\omg}{\omega}

\newcommand{\rhox}{\rho(x)}
\newcommand{\dcx}{\dd[3]{x}}

\newcommand{\cV}{\mathcal{V}}
\newcommand{\intcV}{\int_\cV}
\newcommand{\intS}{\int_S}
\newcommand{\dS}{\dd{S}}

\renewcommand{\xi}{x_i}
\newcommand{\limRi}{\lim_{R \to \infty}}

\newcommand{\nh}{\vb{\hat{n}}}
\newcommand{\rh}{\vb{\hat{r}}}
\newcommand{\xh}{\vb{\hat{x}}}
\newcommand{\zh}{\vb{\hat{z}}}
\newcommand{\phh}{\boldsymbol{\hat{\vph}}}

\state{Spin-wave theory~(P\&S 11.1)}{\hfix}

\prob{ \label{1a}
	Prove the following wonderful formula: Let $\phix$ be a free scalar field with propagator $\ev{T \phix \phio} = \Dx$.  Then
	\eqn{show1}{
		\ev{ T e^{i \phix} e^{-i \phio} } = e^{[ \Dx - \Do ]}.
	}
	(The  factor $\Do$ gives a formally divergent adjustment of the overall normalization.)
}

\sol{
	According to P\&S~(9.18),
	\eq{
		\ev*{T \phi(\xq) \phi(\xw)}{\Omg} = \frac{\int \DDphi \phi(\xq) \phi(\xw) \exp[ i \int \ddqx \cL ]}{\int \DDphi \exp[ i \int \ddqx \cL ]}.
	}
	We use this expression to write the left-hand side of Eq.~\refeq{show1}:
	\eqn{thing1}{
		\ev{ T e^{i \phix} e^{-i \phio} } = \frac{\int \DDphi e^{i \phix} e^{-i \phio} \exp[ i \int \ddqy \cL ]}{\int \DDphi \exp[ i \int \ddqy \cL ]}
		= \frac{\int \DDphi \exp[i \phix - i \phio + i \int \ddqy \cL ]}{\int \DDphi \exp[ i \int \ddqy \cL ]}.
	}
	For a free Klein-Gordon~(i.e., scalar) field, Eq.~(9.39) tells us that the generating functional $\ZJ$ is given by
	\eq{
		\ZJ = \Zo \exp[ -\frac{1}{2} \int \ddqx \ddqy \Jx \DF(x - y) \Jy ],
	}
	where $\Zo = Z[0]$.  Thus, we want to find some $\Jy$ such that
	\eqn{thing1b}{
		\ev{ T e^{i \phix} e^{-i \phio} } = \frac{\ZJ}{\Zo}
	}
	where in general
	\eq{
		\ZJ = \int \DDphi \exp[ i \int \ddqx [ \cL + \Jx \phi(x) ] ]
	}
	by (9.34).  Inspecting Eq.~\refeq{thing1}, we recognize the denominator as $\Zo$ and see that if
	\eq{
		\Jy = \delq(y - x) - \delq(y)
	}
	we have an expression like Eq.~\refeq{thing1b}.  Collecting these findings, we have
	\al{
		\ans{ \ev{ T e^{i \phix} e^{-i \phio} } }&= \frac{\ZJ}{\Zo} \\
		&= \exp[ -\frac{1}{2} \int \ddqy \ddqz \Jy \DF(y - z) \Jz ] \\
		&= \exp[ -\frac{1}{2} \int \ddqy \ddqz \Jy \DF(y - z) [ \delq(z - x) - \delq(z) ] ] \\
		&= \exp[ -\frac{1}{2} \int \ddqy [ \delq(y - x) - \delq(y) ] [ \DF(y - x) - \DF(y) ] ] \\
		&= \exp[ -\frac{1}{2} [ \DF(0) - \DF(x) - \DF(-x) + \DF(0) ] ] \\
		&= \exp[ \DF(x) - \DF(0) ] \\
		&\ans{\; = e^{[ \Dx - \Do ]}, }
	}
	as we wanted to show. \qed
}



\prob{ \label{1b}
	We can use this formula in Euclidean field theory to discuss correlation functions in a theory with spontaneously broken symmetry for $T < \TC$.  Let us consider only the simplest case of a broken $O(2)$ or $U(1)$ symmetry.  We can write the local spin density as a complex variable
	\eq{
		\sx = \sqx + i \swx.
	}
	The global symmetry is the transformation
	\eq{
		\sx \to e^{-i \alp} \sx.
	}
	If we assume that the physics freezes the modulus of $\sx$, we can parameterize
	\eqn{sx}{
		\sx = A e^{i \phix}
	}
	and write an effective Lagrangian for the field $\phix$.  The symmetry of the theory becomes the translation symmetry
	\eqn{symmetry}{
		\phix \to \phix - \alp.
	}
	Show that (for $d > 0$) the most general renormalizable Lagrangian consistent with this symmetry is the free field theory
	\eqn{show1b}{
		\cL = \frac{1}{2} \rho(\vgrad \phi)^2.
	}
	In statistical mechanics, the constant $\rho$ is called the \emph{spin wave modulus}.  A reasonable hypothesis for $\rho$ is that it is finite for $T < \TC$ and tends to 0 as $T \to \TC$ from below.
}

\sol{
	In accordance with the Klein-Gordon Lagrangian in P\&S~(2.6),
	\eqn{KGL}{
		\cL_\text{K-G} = \frac{1}{2} (\pt \phi)^2 - \frac{1}{2} m^2 \phi^2,
	}
	we interpret $(\vgrad \phi)^2$ as $(\pt \phi)^2$.
	
	The Lagrangian cannot have terms of $\order{\phi^n}$ for any $n \neq 0$ since $\phi(x)$ is not invariant under Eq.~\refeq{symmetry}.  Any combination of derivatives of $\phi$ is invariant, however, since $\alp$ is a constant and does not contribute to any derivative.  Thus, only terms like $\pt^n \phi^m$ (where $n$ denotes a power of $\pt$) for $n, m > 0$ and $n \geq m$ are consistent with the symmetry of Eq.~\refeq{symmetry} for $d$ an integer.
	
	Now we must determine which of these terms are renormalizable.  We know that the Lagrangian must have dimension $d$, and that $\phi$ has dimension $(d - 2) / 2$.  Taking a derivative adds a mass dimension.  The theory is renormalizable if the coupling constant $\rho$ has dimension greater than or equal to 0~\cite[p.~322]{Peskin}.  Let $p$ be the dimension of $\rho$.  The dimension of our allowed term is then
	\eq{
		[ \rho \pt^n \phi^m ] = p + n + m \frac{d - 2}{2},
	}
	which we require to be equal to $d$.  Thus we seek solutions to the system of equations
	\al{
		d &= p + n + m \frac{d - 2}{2}, &
		n &\geq m, &
		p &\geq 0.
	}
	Solving with Mathematica, we find that this system has two solutions: $n = m = 2$ and $p = 0$; and $n = m = 1$ and $p = d / 2$.  However, the term $\pt \phi$ for $n = m = 1$ does not contribute to the action because it is a total derivative and does not contribute when the integral over $\cL$ is evaluated:
	\eq{
		\int \dd[d]{x} \pt\phi = \phi \bigg|_{-\infty}^\infty
		= 0.
	}
	Thus the only possibility is $n = m = 2$.  Note that
	\eq{
		\pt^2 \phi^2 = \pt(\pt \phi^2)
		= 2 \pt( \phi \pt \phi)
		= \pt \phi \pt \phi + \phi \pt^2 \phi
		= (\pt \phi)^2,
	}
	since $\phi \pt^2 \phi$ is not invariant under Eq.~\refeq{sx}.  This means that $\rho$ must be dimensionless and that the only allowed terms in the Lagrangian are proportional to $(\pt \phi)^2$, which is consistent with Eq.~\refeq{show1b}. \qed
}



\prob{
	Compute the correlation function $\ev{ \sx \sao }$.  Adjust $A$ to give a physically sensible normalization (assuming that the system has a physical cutoff at the scale of one atomic spacing) and display the dependence of this correlation function on $x$ for $d = 1, 2, 3, 4$.  Explain the significance of your results.
}

\sol{
	Applying Eq.~\refeq{sx},
	\eq{
		\ev{ \sx \sao } = \ev*{ A e^{i \phix} \As e^{-i \phio} }
		= \ev*{ \abs{A}^2 } \ev*{ e^{i \phix} e^{-i \phio} }.
	}
	Now we can apply Eq.~\refeq{show1} to find
	\eqn{thing1c}{
		\ans{ \ev{ \sx \sao } = \abs{A}^2 \exp[ D(x) - D(0) ], }
	}
	where $D(x - y)$ is a Green's function.  Since our Lagrangian is similar to the Klein-Gordon Lagrangian Eq.~\refeq{2.6}, our Green's function is similar to that of the Klein-Gordon operator, which is given by P\&S~(2.56):
	\eq{
		(\pt^2 + m^2) D(x - y) = -i \delq(x - y).
	}
	The Feynman prescription for this Green's function is given by (2.59),
	\eqn{DF}{
		\DF(x - y) = \int \ddqpf \frac{i}{p^2 - m^2 + i \eps} e^{-i p \cdot (x - y)}.
	}
	For the Lagrangian in Eq.~\refeq{show1b}, we set $m = 0$ and insert a factor of $\rho$:
	\eq{
		\rho \pt^2 D(x - y) = -i \deld(x - y),
	}
	so adapting Eq.~\refeq{DF} for this situation yields
	\eqn{DF}{
		\DF(x - y) = \frac{1}{\rho} \int \dddpf \frac{i}{p^2 + i \eps} e^{-i p \cdot (x - y)}.
	}
	We see that $\DF(0)$ diverges, so we absorb it into the constant to make the normalization physically sensible.  We can do this because, as we showed in \ref{1b}, the theory is renormalizable.  Define $A'$ such that
	\eq{
		{A'}^2 = \abs{A}^2 e^{-D(0)}.
	}
	Then Eq.~\refeq{thing1c} can be written
	\eq{
		\ans{ \ev{ \sx \sao } =  {A'}^2 e^{D(x)}. }
	}
	
	To evaluate the divergent integral $D(x)$, we look to the Feynman parameter method we have been using to solve divergent integrals.  Apparently, the Schwinger parametrization is useful in deriving the Feynman parametrization, and it is given by~\cite{Feynman}
	\eq{
		\frac{1}{A} = \intoi \dds e^{-s A}.
	}
	Using this equation, we can write Eq.~\refeq{DF} as
	\eq{
		\DF(x) = \frac{1}{\rho} \int \dddpf \frac{i}{p^2} e^{-i p \cdot x}
		= \frac{i}{\rho} \int \dddpf \intoi \dds e^{-s p^2} e^{-i p \cdot x}.
	}
	Now we can complete the square in the exponential to get a Gaussian integral:
	\al{
		\DF(x) &= \frac{i}{\rho} \int \dddpf \intoi \dds \exp[ -s p^2 - i p \cdot x + \frac{x^2}{4 s} - \frac{x^2}{4 s} ] \\
		&= \frac{i}{\rho} \int \dddpf \intoi \dds \exp[ -s \paren{ p + \frac{i x}{2 s} }^2 - \frac{x^2}{4 s} ] \\
		&= \frac{i}{\rho (2 \pi)^d} \intoi \dds e^{-x^2 / 4 s} \int \dd[d]{u} e^{-s u^2} \\
		&= \frac{i}{\rho (2 \pi)^{d}} \intoi \dds e^{-x^2 / 4 s} \sqrt{ \frac{(2\pi)^d}{(2s)^d} } \\
		&= \frac{i}{\rho (4 \pi)^{d / 2}} \intoi \dds \frac{e^{-x^2 / 4 s}}{s^{d / 2}}
	}
	where we have used~\cite{QFT}
	\eq{
		\int \exp( -\frac{1}{2} x \cdot A \cdot x ) \dd[n]{x} = \sqrt{\frac{(2\pi)^n}{\det A}},
	}
	with $A$ a $d \times d$ diagonal matrix $2s$.  Using Mathematica to integrate with respect to $s$, we find
	\eq{
		\DF(x) = \frac{i}{\rho (4 \pi)^{d / 2}} \frac{2^{d - 2}}{x^{d - 2}} \Gam(d / 2 - 1)
		= \frac{i}{4 \pi^d \rho} \Gam(d / 2 - 1) x^{2 - d}.
	}
	The gamma function diverges as $d \to 2$, so as we have done in previous problems, we expand about $\eps = 2 - d$.  Evaluating the series expansion using Mathematica, we obtain
	\eq{
		\DF(x) = \frac{i}{4 \pi^{1 - \eps} \rho} \Gam(\eps / 2) x^\eps
		\approx \frac{i}{4 \pi \rho} \paren{ \frac{2}{\eps} - \gam + 2 \ln(\pi x) }
		\sim \frac{i}{2 \pi \rho} \ln(x)
		= i \ln(\frac{1}{x^{2 \pi \rho}}).
	}
	We Wick rotate $x \to i x$.  Then the dependence of the correlation function on $x$ for $d = 1, 2, 3, 4$ is
	\ans{\al{
		(d = 1) &\qquad \ev{ \sx \sao } \sim e^{-x / 2 \sqrt{\pi} \rho}, &
		(d = 2) &\qquad \ev{ \sx \sao } \sim x^{2 \pi \rho}, \\
		(d = 3) &\qquad \ev{ \sx \sao } \sim \frac{1}{x}, &
		(d = 4) &\qquad \ev{ \sx \sao } \sim \frac{1}{x^2}.
	}}%
	In $d > 2$ dimensions, the expectation value of the correlation function tends to 0 at large distances $x$.  For $d > 2$, it drops off more quickly as $d$ increases.  The $d \leq 2$ cases depend on $\rho$, which we assume is positive.  The $d = 1$ case drops off with increasing distance, and more quickly with smaller $\rho$.  For $d = 2$, the expectation value of the correlation function increases with increasing distance, and it blows up more quickly with larger $\rho$.
	
	These results are consistent with the Mermin--Wagner theorem, which states that a continuous symmetry cannot be broken in $d \leq 2$ dimensions~\cite{CMW}.  That is, in $d \leq 2$ dimensions, a symmetry-breaking field cannot have a nonzero vacuum expectation value~\cite[p.~460]{Peskin}.  A physical explanation is that each spin has more nearest neighbors in higher dimensions.  Since the spins are inclined to align with their neighbors, there is a higher degree of correlation in higher dimensions at the same distance.  In two dimensions, the correlations are weak enough that they are overpowered by the field fluctuations.
}



\renewcommand{\qq}{q_1}
\newcommand{\qw}{q_2}
\newcommand{\absx}{\abs{\vx}}
\newcommand{\tx}{(t, \vx)}

\newcommand{\rhoq}{\rho_1}
\newcommand{\vEq}{\vE_1}
\newcommand{\vpq}{\vp_1}
\newcommand{\vpqt}{\vpq(t)}
\newcommand{\vxq}{\vx_1}
\newcommand{\vxqt}{\vxq(t)}

\newcommand{\ret}{\text{ret}}

\begin{statement}{}
	A particle of charge $\qq$ moves with velocity $v$ in a circular orbit of radius $R$ about the origin in the $xy$ plane, such that its $\vph$ coordinate varies as $\vph = \omg t$, with $\omg = v / R$.  Assume that $v \ll c$.  Another particle of charge $\qw$ is at rest at point $\vx$, where $\absx \gg R$.  To order $1 / \absx$, find the force $\vF$ on the particle of charge $\qw$ at time $t$.
\end{statement}

\begin{solution}
	The Lorentz force equation, Eq.~(1.25), is written
	\beq
		\vF = q \left( \vE + \frac{\vv}{c} \cross \vB \right),
	\eeq
	where $\vv$ is the velocity of the charge $q$ on which the force is exerted, and $\vE$ and $\vB$ are the total electric and magnetic fields.  For this problem, we are interested in the force acting on a stationary point charge $\qw$, so $\vv_2 = 0$.  Additionally, we do not have to consider the self-field contribution to $\vE$, since static charge distributions do not experience any self force.  Thus we need only find the electric field due to $\qq$, $\vEq$.  The multipole expansion of the electric field in electrodynamics is given by Eq.~(5.70),
	\beqn \label{Efield}
		\vE\tx = \frac{1}{c^2 \absx} \left[ \left(\xh \vdot \dv[2]{\vp}{t} \right) \xh - \dv[2]{\vp}{t} \right]_\ret + \order{\frac{1}{\absx^2}},
	\eeqn
	where $\xh = \vx / \absx$ is the unit vector in the direction of the point at which we are evaluating the field, and $\vp$ is the dipole moment defined by \refeq{dipole}.  In addition, \refeq{Efield} relies upon the assumption that the velocity of $\qq$, $v$, satisfies $v \ll c$.
	
	The position of $\qq$ at time $t$ can be expressed as
	\beq
		\vxqt = R \cos(\omg t) \,\phh,
	\eeq
	so the charge density for $\qq$ everywhere is
	\beq
		\rhoq\tx = \qq \,\del\big( \vx - \vxqt \big).
	\eeq
	Then the dipole moment $\vpq\tx$ is
	\beq
		\vpq\tx = \int \vx \,\rhoq\tx \dcx
		= \qq \int \vx \,\del\big(\vx - \vxqt \big) \dcx
		= \qq \vxqt
		= \qq R \cos(\omg t) \,\phh,
	\eeq
	and so its second time derivative is
	\beq
		\dv[2]{\vpqt}{t} = \dv{}{t} \left( \dv{\vpq}{t} \right)
		= \dv{}{t} \bigg( \!-\!\qq R \omg \sin(\omg t) \,\phh \bigg)
		= -\qq R \omg^2 \cos(\omg t) \,\phh.
	\eeq
	
	The retarded time $t'$ is defined
	\beq
		t' = t - \frac{\abs{\vx - \vx'}}{c},
	\eeq
	so here
	\beq
		t' = t - \frac{\abs{\vx - \vxqt}}{c}
		= t - \frac{\abs{\vx - R \cos(\omg t) \,\phh}}{c}.
	\eeq
	Since $R \ll \absx$, we can Taylor expand the second term about $R = 0$ as in Eq.~(5.57),
	\beq
		\abs{\vx - \vx'} = \absx - \xh \vdot \vx' + \order{\frac{1}{\absx}}.
	\eeq
	This gives us
	\beq
		\abs{\vx - R \cos(\omg t) \,\phh} \approx \absx - R \cos(\omg t) (\xh \vdot \phh),
	\eeq
	so
	\beq
		t' \approx t - \frac{\absx}{c} - \frac{R \cos(\omg t)}{c} (\xh \vdot \phh),
	\eeq
	\hl{which is really not helpful.}
\end{solution}



\newcommand{\Io}{I_0}
\newcommand{\tz}{(t, z)}

\newcommand{\dt}{\dd{t}}

\begin{statement}{}
	An ``antenna'' is a segment of conducting wire in which a current flows (driven by an external power supply).  Suppose an antenna of length $L$ is placed on the $z$ axis between $z = 0$ and $z = L$, and suppose that the current in the antenna is
	\beqn \label{J3}
		\vJ\tz = \Io \sin(\frac{\pi z}{L}) \cos(\omg t) \,\del(x) \,\del(y) \,\zh.
	\eeqn
	\vfix
\end{statement}

\begin{problem}
	Find the charge density $\rho\tz$ in the antenna.
\end{problem}

\begin{solution}
	From the charge-current conservation law \refeq{continuity}, we have
	\beq
		\rho\tz = -\int \div{\vJ} \dt.
	\eeq
	For $\vJ$ given by \refeq{J3},
	\beq
		\div{\vJ} = \pdv{J_z}{z}
		= \frac{\pi}{L} \Io \cos(\frac{\pi z}{L}) \cos(\omg t) \,\del(x) \,\del(y),
	\eeq
	and so
	\beq
		\rho\tz = -\frac{\pi}{L} \Io \cos(\frac{\pi z}{L}) \,\del(x) \,\del(y) \int \cos(\omg t) \dt
		= -\frac{\pi}{L} \frac{\Io}{\omg} \cos(\frac{\pi z}{L}) \sin(\omg t) \,\del(x) \,\del(y),
	\eeq
	where we have set the integration constant to 0.
\end{solution}



\begin{problem}
	Assume that $\omg L \ll c$.  Find the electric and magnetic fields, $\vE\tz$ and $\vB\tz$, at large distances from the antenna (valid to order $1 / \absx$).
\end{problem}



\vfill
{\small In addition to the course lecture notes, I consulted Griffiths's \emph{Introduction to Electrodynamics}, Jackson's \emph{Classical Electrodynamics}, and K.~T.~McDonald's and D.~K.~Ghosh's notes on electromagnetism while writing up these solutions.}
\end{document}