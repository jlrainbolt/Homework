\documentclass[11pt]{article}
\usepackage{homework}

\classname{361}
\homeworknum{1}



\begin{document}

% Environments

\newcommand{\state}[2]{\begin{statement}{#1} #2 \end{statement}}
\newcommand{\prob}[2]{\begin{problem}{#1} #2 \end{problem}}
\newcommand{\subprob}[1]{\begin{subproblem} #1 \end{subproblem}}
\newcommand{\sol}[1]{\begin{solution} #1 \end{solution}}
\newcommand{\fig}[2]{\begin{figure} \centering #2  \label{#1} \end{figure}}

\newcommand{\makebib}{
	\vfill
	\color{black}
	\bibliography{references}{}
	\bibliographystyle{lucas_unsrt}
}
	

% Implication

\newcommand{\qwhere}{\quad \text{where} \quad}
\newcommand{\qimplies}{\quad \implies \quad}
\newcommand{\impliesq}{\implies \quad}



% Brackets

\newcommand{\paren}[1]{\left( #1 \right)}
\newcommand{\brac}[1]{\left[ #1 \right]}


% Greek

\newcommand{\alp}{\alpha}
\newcommand{\bet}{\beta}
\newcommand{\gam}{\gamma}
\newcommand{\del}{\delta}
\newcommand{\eps}{\epsilon}
\newcommand{\zet}{\zeta}
\newcommand{\tht}{\theta}
\newcommand{\kap}{\kappa}
\newcommand{\lam}{\lambda}
\newcommand{\sig}{\sigma}
\newcommand{\ups}{\upsilon}
\newcommand{\omg}{\omega}

\newcommand{\Gam}{\Gamma}
\newcommand{\Del}{\Delta}
\newcommand{\Tht}{\Theta}
\newcommand{\Lam}{\Lambda}
\newcommand{\Sig}{\Sigma}
\newcommand{\Omg}{\Omega}
% Problem 1

\newcommand{\Psii}{\Psi^i}
\newcommand{\Psiix}{\Psii(x)}

\newcommand{\Pii}{\Pi^i}

\newcommand{\Phii}{\Phi^i}
\newcommand{\Phiix}{\Phii(x)}
\newcommand{\PhiN}{\Phi^N}
\newcommand{\PhiNx}{\PhiN(x)}
\newcommand{\Phiq}{\Phi^1}
\newcommand{\Phiw}{\Phi^2}

\newcommand{\ddcx}{\dd[3]{x}}

\newcommand{\delij}{\del^{i j}}
\newcommand{\delkl}{\del^{k l}}
\newcommand{\delil}{\del^{i l}}
\newcommand{\deljk}{\del^{j k}}
\newcommand{\delik}{\del^{i k}}
\newcommand{\deljl}{\del^{j l}}

\newcommand{\DF}{D_F}

\newcommand{\sigx}{\sig(x)}

\newcommand{\pii}{\pi^i}
\newcommand{\pij}{\pi^j}
\newcommand{\pik}{\pi^k}
\newcommand{\pil}{\pi^l}
\newcommand{\piix}{\pi(x)}

\newcommand{\pq}{p_1}
\newcommand{\pw}{p_2}
\newcommand{\pe}{p_3}
\newcommand{\pr}{p_4}

\newcommand{\vp}{\vb{p}}
\newcommand{\vpsi}{\vp_i}

\newcommand{\mpi}{m_\pi}

\state{(Jackson 9.8)}{\ 
	%\emph{Hint:} The electromagnetic angular momentum density comes from more than the transverse (radiation zone) components of the fields.
}

%
%	Jackson 9.8(a)
%

\prob{}{
	Show that a classical oscillating electric dipole $\vp$ with fields given by
	\aln{ \label{fields1}
		\vH &= \frac{c k^2}{4\pi} (\nh \cross \vp) \frac{e^{i k r}}{r} \paren{ 1 - \frac{1}{i k r} }, &
		\vE &= \frac{1}{4\pi \epso} \curly{ k^2 (\nh \cross \vp) \cross \nh \frac{e^{i k r}}{r} + [ 3 \nh (\nh \vdot \vp) - \vp ] \paren{ \frac{1}{r^3} - \frac{i k}{r^2} } e^{i k r} },
	}
	radiates electromagnetic angular momentum to infinity at the rate
	\eq{
		\dv{\vL}{t} = \frac{k^3}{12 \pi \epso} \Im[ \vp^* \cross \vp ].
	}
	\vfix
}

\sol{
	According to Jackson~(9.20), the time-averaged angular momentum density is
	\eq{
		\vl = \frac{\Re[ \vx \cross (\vE \cross \vHs)}{2 c^2}.
	}
	One of the vector identities on the inside cover of Jackson is $\vaa \cross (\vbb \cross \vcc) = (\vaa \vdot \vcc) \vbb - (\vaa \vdot \vbb) \vcc$, so
	\eqn{l1}{
		\vl = \frac{(\vx \vdot \vHs) \vE - (\vx \vdot \vE) \vHs}{2 c^2}.
	}
	From Eq.~\refeq{fields1}, note that
	\eq{
		\vx \vdot \vHs \propto \vx \vdot (\nh \cross \vps)
		= \vps \vdot (\vx \cross \nh)
		= \vO,
	}
	where we have used the identity $\vaa \vdot (\vbb \cross \vcc) = \vcc \vdot (\vaa \cross \vbb)$ and the fact that $\nh$ points in the $\vx$ direction.  For $\vx \vdot \vE$, note that
	\al{
		\vx \vdot [ (\nh \cross \vp) \cross \nh ] &= -\vx \vdot [ \nh \cross (\nh \cross \vp) ]
		= -\vx \vdot [ (\nh \vdot \vp) \nh - (\nh \vdot \nh) \vp ]
		= -(\nh \vdot \vp) (\vx \vdot \nh) + \vx \vdot \vp \\
		&= -r (\nh \vdot \vp) + \vx \vdot \vp
		= \vx \vdot \vp - \vx \vdot \vp
		= 0, \\[1.5ex]
		\vx \vdot [ 3 \nh (\nh \vdot \vp) - \vp ] &= 3 (\vx \vdot \nh) (\nh \vdot \vp) - \vx \vdot \vp
		= 3r (\nh \vdot \vp) - \vx \vdot \vp
		= 3(\vx \vdot \vp) - \vx \vdot \vp
		= 2(\vx \vdot \vp),
	}
	since $\abs{\vx} = r$ and $\vx = r \,\nh$.  Then
	\eq{
		\vx \vdot \vE = \frac{1}{2\pi \epso} (\vx \vdot \vp) \paren{ \frac{1}{r^3} - \frac{i k}{r^2} } e^{i k r}
		= \frac{1}{2\pi \epso} (\nh \vdot \vp) \paren{ \frac{1}{r^2} - \frac{i k}{r} } e^{i k r}.
	}
	
	With these substitutions, Eq.~\refeq{l1} becomes
	\al{
		\vl &= -\frac{(\vx \vdot \vE) \vHs}{c^2}
		= -\frac{1}{4\pi \epso c^2} (\nh \vdot \vp) \paren{ \frac{1}{r^2} - \frac{i k}{r} } e^{i k r} \frac{c k^2}{4\pi} (\nh \cross \vps) \frac{e^{-i k r}}{r} \paren{ 1 + \frac{1}{i k r} } \\
		&= -\frac{k^2}{16\pi^2 \epso c r} (\nh \vdot \vp) (\nh \cross \vps) \paren{ \frac{1}{r^2} - \frac{i k}{r} } \paren{ 1 - \frac{i}{k r} }
		= -\frac{k^2}{16\pi^2 \epso c} (\nh \vdot \vp) (\nh \cross \vps) \paren{ \frac{1}{r^2} - \frac{i}{k r^3} - \frac{i k}{r} - \frac{1}{r^2} } \\
		&= -\frac{i k^2}{16\pi^2 \epso c r} (\nh \vdot \vp) (\nh \cross \vps) \paren{ \frac{1}{k r^3} + \frac{k}{r^2} }
		= \frac{i k^3}{16\pi^2 \epso c r^2} (\nh \vdot \vp) (\nh \cross \vps) \paren{ \frac{1}{k^2 r^2} + 1 }.
	}
	
	Let $\vL$ be the angular momentum radiated to a distance $R$.  Then
	\eq{
		\vL = \int_R \vl(r) \ddcx
		= \intopi \intotp \intoR \vl(r) \,r^2 \sin\tht \ddr \ddphi \dd\tht,
	}
	and the time derivative is
	\aln{
		\dv{\vL}{t} &= \dv{t}(\intopi \intotp \intoR \vl(r) \,r^2 \sin\tht \ddr \ddphi \dd\tht)
		= \dv{r}{t} \dv{r}(\intopi \intotp \intoR \vl(r) \,r^2 \sin\tht \ddr \ddphi \dd\tht) \notag \\
		&= c \intopi \intotp \vl(r) \,r^2 \sin\tht \ddphi \dd\tht
		= \frac{i k^3}{16\pi^2 \epso} \paren{ \frac{1}{k^2 r^2} + 1 } \intopi \intotp (\nh \vdot \vp) (\nh \cross \vps) \sin\tht \ddphi \dd\tht. \label{dLdt}
	}
	Note that
	\eq{
		[ (\nh \vdot \vp) (\nh \cross \vps) ]_i = \sumje n_j p_j (\nh \cross \vps)_i
		= \sumje \sumke \sumle \epsikl n_j p_j n_k p_l^*,
	}
	so
	\eq{
		\dv{L_i}{t} \propto \sumje \sumke \sumle \epsikl p_j p_l^* \int n_j p_k \ddOmg
		= \sumje \sumke \sumle \epsikl p_j p_l^* \frac{4\pi}{3} \del_{jk}
		= \frac{4\pi}{3} \epsikl p_k p_l^*
		= \frac{4\pi}{3} (\vp \cross \vps)_i,
	}
	where we have used Jackson~(9.47), $\int n_\bet n_\gam \ddOmg = 4\pi \del_{\bet \gam} / 3$.  Making this substitution into Eq.~\refeq{dLdt},
	\eq{
		\dv{\vL}{t} = \frac{i k^3}{6\pi \epso} \paren{ \frac{1}{k^2 r^2} + 1 } (\vp \cross \vps).
	}
	Taking the limit as $r \to \infty$, we find
	\eqn{ans1a}{
		\dv{\vL}{t} = \Re\!\brac{ \frac{i k^3}{12\pi \epso} (\vp \cross \vps) }
		= \Re\!\brac{ -\frac{i k^3}{12\pi \epso} (\vps \cross \vp) }
		= \ans{ \frac{k^3}{12\pi \epso} \Im[ \vps \cross \vp ], }
	}
	as desired. \qed
}

%
%	Jackson 9.8(b)
%

\prob{}{
	What is the ratio of angular momentum radiated to energy radiated?  Interpret.
}

\sol{
	According to Jackson~(9.24), the total power radiated by an oscillating electric dipole $\vp$ is
	\eq{
		P = \dv{E}{t}
		= \frac{c^2 \Zo k^4}{12 \pi} \abs{\vp}^2.
	}
	Then the ratio of angular momentum radiated to energy radiated is
	\eq{
		\frac{\dv*{\vL}{t}}{\dv*{E}{t}} = \frac{k^3}{12\pi \epso} \Im[ \vps \cross \vp ] \frac{12 \pi}{c^2 \Zo k^4 \abs{\vp}^2}
		= \frac{1}{\epso} \Im[ \vps \cross \vp ] \frac{1}{c^2 \Zo k \abs{\vp}^2}
		= \ans{ \frac{\Im[ \vps \cross \vp ]}{\omg \abs{\vp}^2}, }
	}
	where we have used $\Zo = \sqrt{\muo / \epso} = 1 / \sqrt{\epso^2 c^2} = 1 / \epso c$, $c^2 = 1 / (\epso \muo)$, and $\omg = k c$.
	
	In the limit of high frequency, $(\dv*{\vL}{t}) / (\dv*{E}{t}) \to 0$.  In this scenario, the energy radiated dominates over the angular momentum radiated.  Likewise, in the limit of low frequency, $(\dv*{\vL}{t}) / (\dv*{E}{t}) \to \infty$, meaning that angular momentum radiation dominates.  This is sensible because rotational kinetic energy $E \propto \omg^2$, while angular momentum $L \propto \omg$.
}

%
%	Jackson 9.8(c)
%

\prob{}{
	For a charge $e$ rotating in the $xy$ plane at radius $a$ and angular speed $\omg$, show that there is only a $z$ component of radiated angular momentum with magnitude $\dv*{\Lz}{t} = e^2 k^3 a^2 / 6 \pi \epso$.  What about a charge oscillating along the $z$ axis?
}

\sol{
	We know from Homework~5 that the position of a point charge rotating counterclockwise in the $xy$ plane is
	\eq{
		\vx(t) = a \cos(\omg t) \,\vx + a \sin(\omg t) \,\yh.
	}
	\clearpage
	Then the charge distribution is
	\eq{
		\rho(\vx, t) = e \del[ x - a \cos(\omg t) ] \,\del[ y - a \sin(\omg t) ] \,\del(z).
	}
	
	According to Jackson~(4.8), the dipole moment is defined
	\eq{
		\vp = \int \vx' \,\rho(\vx') \ddcxp.
	}
	The components of $\vp$ for the point charge are then
	\al{
		\px &= e \iiint x \,\del[ x - a \cos(\omg t) ] \,\del[ y - a \sin(\omg t) ] \,\del(z) \ddx \ddy \ddz
		= e a \cos(\omg t), \\
		\py &= e \iiint y \,\del[ x - a \cos(\omg t) ] \,\del[ y - a \sin(\omg t) ] \,\del(z) \ddx \ddy \ddz
		= e a \sin(\omg t), \\
		\pz &= e \iiint z \,\del[ x - a \cos(\omg t) ] \,\del[ y - a \sin(\omg t) ] \,\del(z) \ddx \ddy \ddz
		= 0,
	}
	so we can write $\vp = e a \,e^{-i \omg t} (\xh + i\,\yh).$  Substituting into Eq.~\refeq{ans1a},
	\al{
		\dv{\vL}{t} &= \Re\!\brac{ \frac{i k^3}{12\pi \epso} e^2 a^2 e^{-i \omg t} e^{i \omg t} [ (\xh + i\,\yh) \cross (\xh - i\,\yh) ] }
		= \Re\!\brac{ \frac{i e^2 k^3 a^2}{12\pi \epso} (-2i \,\xh \cross \yh) }
		= \Re\!\brac{ \frac{e^2 k^3 a^2}{6\pi \epso} \,\zh } \\
		&= \ans{ \frac{e^2 k^3 a^2}{6\pi \epso} \cos(\omg t) \,\zh, }
	}
	as desired. \qed
	
	A charge oscillating along the $z$ axis with amplitude $a$ has the charge density
	\eq{
		\rho(\vx, t) = e a \,\del(x) \,\del(y) \,\del[ z - \cos(\omg t) ],
	}
	which gives the dipole moment
	\al{
		\px &= e a \iiint x \,\del(x) \,\del(y) \,\del[ z - \cos(\omg t) ] \ddx \ddy \ddz
		= 0, \\
		\py &= e a \iiint y \,\del(x) \,\del(y) \,\del[ z - \cos(\omg t) ] \ddx \ddy \ddz
		= 0, \\
		\pz &= e a \iiint z \,\del(x) \,\del(y) \,\del[ z - \cos(\omg t) ] \ddx \ddy \ddz
		= e a \cos(\omg t).
	}
	In complex notation, $\vp = e a \,e^{-i\omg t} \,\zh$.  Substituting into Eq.~\refeq{ans1a}, we find
	\eq{
		\dv{\vL}{t} = \Re\!\brac{ \frac{i k^3}{12\pi \epso} e^2 a^2 e^{-i \omg t} e^{i \omg t} (\zh \cross \zh) }
		= \ans{ \vO. }
	}
	So we see that a charge undergoing linear motion does not lead to a radiated angular momentum, which is sensible.
}

%
%	Jackson 9.8(d)
%

\prob{}{
	What are the results corresponding to Probs.~{1(a)} and {1(b)} for magnetic dipole radiation?
}

\sol{
	The radiation fields for a magnetic dipole are given by Jackson~(19.35--36),
	\al{
		\vH &= \frac{1}{4\pi} \curly{ k^2 (\nh \cross \vm) \cross \nh \frac{e^{i k r}}{r} + [ 3 \nh (\nh \vdot \vm) - \vm ] \paren{ \frac{1}{r^3} - \frac{i k}{r^2} } e^{i k r} }, &
		\vE &= -\frac{\Zo}{4\pi} k^2 (\nh \cross \vm) \frac{e^{i k r}}{r} \paren{ 1 - \frac{1}{i k r} }.
	}
	\clearpage
	Comparing with Eq.~\refeq{fields1}, we see that $\vH \to -\vE / \Zo$, $\vE \to \Zo \vH$, and $\vp \to \vm / c$ as stated in the book~\cite[p.~413]{Jackson}.  Making these substitutions, the results of Probs.~{1.1(a)} and {(b)} become
	\al{
		\ans{ \dv{\vL}{t}\ }&\ans{= \frac{\muo k^3}{12\pi} \Im[ \vms \cross \vm ], } &
		\ans{ \frac{\dv*{\vL}{t}}{\dv*{E}{t}}\ }&\ans{= \frac{\Im[ \vms \cross \vm ]}{\omg \abs{\vm}^2} }
	}
	where we have used $\mu = 1 / \epso c^2$.
}

\state{Beta function of the Gross-Neveu model~(P\&S~12.2)}{
	Compute $\bet(g)$ in the two-dimensional Gross-Neveu model studied in Problem~11.3,
	\eq{
		\cL = \psibsi i \ptsl \psisi + \frac{1}{2} g^2 (\psibsi \psisi)^2,
	}
	with $i = 1, \ldots, N$.  You should find that this model is asymptotically free.  How was that fact reflected in the solution to Problem~11.3?
}

\sol{
	We saw in Problem~2 of Homework~4 that this Lagrangian can be written as
	\eq{
		\cL = \psibsi i \ptsl \psisi - \sig \psibsi \psisi - \frac{1}{2 g^2} \sig^2,
	}
	where $\sig$ is a new scalar field with no kinetic energy terms.  In the modified minimal subtraction scheme, we found the effective potential was
	\eqn{Veff}{
		\Veff = \sig^2 \curly{ \frac{1}{2 g^2} + \frac{N}{4\pi} \brac{ \ln(\frac{\sig^2}{M^2}) - 1 } }.
	}
	Since $\Gam[ \phicl ] = -(V T) \Veff(\phi)$ by P\&S~(11.50), we have
	\eqn{Gam}{
		\Gam[ \sigcl ] = -(V T)  \sig^2 \curly{ \frac{1}{2 g^2} + \frac{N}{4\pi} \brac{ \ln(\frac{\sig^2}{M^2}) - 1 } }.
	}
	Referring to p.~3 of Lecture~11, we can apply the Callan-Symanzik equation to $\Gam$.   The Callan-Symanzik equation is P\&S~(12.41),
	\eq{
		\brac{ M \pdv{M} + \bet(\lam) \pdv{\lam} + n \gam(\lam) } G^{(n)}(\{ x_i \}; M, \lam) = 0.
	}
	For our problem, $\gam$ is 0 because there are no field insertions.  That is, we have
	\eq{
		\brac{ M \pdv{M} + \bet(g) \pdv{g} } \Gam[ \phicl ] = 0.
	}
	Using Eq.~\refeq{Gam}, note that
	\al{
		\pdv{\Gam}{M} &= (V T) \frac{N \sig^2}{2 \pi M}, &
		\pdv{\Gam}{g} &= (V T) \frac{\sig^2}{g^3}.
	}
	Then
	\eq{
		0 = (V T) \paren{ \frac{N \sig^2}{2 \pi} + \bet(g) \frac{\sig^2}{g^3} }
		\qimplies
		\ans{ \betg = -\frac{N g^3}{2\pi}. }
	}
	This model is asymptotically free because the $\bet$ function is proportional to $-g^3$~\cite[pp.~424--425]{Peskin}.
	
	In 2(e) of Homework~4, we found that the vacuum expectation value of $\sig$ was
	\eq{
		\sig = \pm M e^{-\pi / N g^2} = \pm v.
	}
	We showed that the vacuum expectation value does not depend on the renormalization condition chosen.  This means that we can increase $M \to 0$ while holding $\sig$ constant, and see that $g \to 0$ logarithmically.  This is indicative of an asymptotically-free theory~\cite[p.~425]{Peskin}. \qed
}






\state{Thermodynamic properties of a free electron metal}{\hfix}

%\prob{}{
%	Derive the free electron formula for the fermi energy $\EF$, the fermi wavevector $\kF$, and the density of states at the fermi level $\gEF$.
%}

%\sol{
%	
%}



%\prob{}{
%	Within the free electron model at zero temperature, show that the total energy for $N$ electrons is $\Eb = 3 N \EF / 5$.
%}



%\prob{}{
%	Within the free electron model at zero temperature, calculate the pressure, $p$, using $p = -\dv*{\Eb}{\Omg}$, where $\Omg$ is the volume.
%}



%\prob{}{	\label{bulk}
%	Within the free electron model at zero temperature, calculate the bulk modulus $B = -\Omg \dv*{p}{\Omg}$.
%}



%\prob{}{
%	Potassium is monovalent and has an atomic concentration of \SI{1.402e28}{\per\cubic\meter}.  Compare the bulk modulus calculated in \ref{bulk} with the experimental value of \SI{3.7e9}{\pascal}.
%}



%\prob{}{
%	Estimate $\gEF$ for magnesium, which has a valence of 2 and an atomic concentration of \SI{4.3e28}{\per\cubic\meter}.  Use this value to estimate the asymptotic low temperature specific heat, compared to the experimental value of $\cv /T = \SI{1.3}{\milli\joule\per\mole\per\square\kelvin}$.
%}






\newcommand{\lap}{\nabla^2}
\newcommand{\vF}{\vec{F}}
\newcommand{\nabx}{\nabla_{\!x}}
\newcommand{\absxp}{\abs{\vx'}}
\newcommand{\nh}{\vec{\hat{n}}}
\newcommand{\rh}{\vec{\hat{r}}}
\newcommand{\Gd}{G_D}
\newcommand{\Gdxxp}{\Gd(\vx,\vx')}

\begin{statement}{}
	A point charge of charge $q$ is placed at point $\vx'$ inside a conducting spherical shell of radius $R$.  There is no net charge on the conductor.  The potential inside the sphere is thus given by $q \, \Gdxxp$, where the explicit formula for $\Gdxxp$ for a spherical cavity is given in the lecture notes.
\end{statement}

\begin{problem}
	Find the surface charge density $\sigtv$ on the conducting shell.
\end{problem}

\begin{solution}
	The Green's function for a spherical cavity is given by Eq.~(2.91),
	\beq
		\Gdxxp = \frac{1}{\abs{\vx - \vx'}} + \frac{\alp}{\abs{\vx - \vx''}} \qq{where} \vx'' = \vx' \frac{R^2}{\absxp^2} \qand \alp = - \frac{R}{\absxp}.
	\eeq
	The surface charge density can be found from Eq.~(2.86),
	\beqn \label{scdeq}
		\vE \cdot \nh = 4\pi \sig,
	\eeqn
	where $\vE = -\nabla \phi$ in electrostatics.
	
	We will begin by finding $\vE$.  We will orient our coordinate system such that $\vx'$ (and consequently $\vx''$) points along the $z$ axis.  Note that
	\beq
		\Gdxxp = \frac{1}{\abs{\vx - \vx'}} - \frac{R}{\absxp \abs{\vx - \dfrac{R^2}{\absxp^2} \vx'}}
		= \frac{1}{\sqrt{\vx^2 - 2 \vx \cdot \vx' + {\vx'}^2}} - \frac{R}{\absxp \sqrt{\vx^2 - 2 \dfrac{R^2}{{\vx'}^2} \vx \cdot \vx' + \dfrac{R^4}{{\vx'}^4} {\vx'}^2}}.
	\eeq
	In spherical coordinates, we have
	\beq
		\Gdxxp = \frac{1}{\sqrt{r^2 - 2 r r' \cost + {r'}^2}} - \frac{R}{r'} \frac{1}{\sqrt{r^2 - 2 R^2 r \cost / r' + R^4 / {r'}^2}},
	\eeq
	where we note that $\tht$ is the angle between $\vx$ and the $z$ axis.  The gradient in spherical coordinates is given by
	\beq
		\nabla = \pdv{}{r} \,\rh + \frac{1}{r} \pdv{}{\tht} \,\thh + \frac{1}{r \sint} \pdv{}{\vph} \, \phh.
	\eeq
	The $r$ component of the electric field inside the conductor is then
	\beq
		\Er(\vx) = -q \pdv{\Gdxxp}{r}
		= q \left( \frac{r - r' \cost}{(r^2 - 2 r r' \cost + {r'}^2)^{3/2}} - \frac{R}{r'} \frac{r - R^2 \cost / r'}{(r^2 - 2 R^2 r \cost / r' + R^4 / {r'}^2)^{3/2}} \right).
	\eeq
	Since $\nh = -\rh$ for the inner surface of a sphere, we are interested in only the $r$ component of the field.  On the surface of the sphere, the field is $\Er(r=R) \,\rh$.  So we have
	\begin{align*}
		\Er(r=R) &= q \left( \frac{R - r' \cost}{(R^2 - 2 R r' \cost + {r'}^2)^{3/2}} - \frac{R}{r'} \frac{R - R^2 \cost / r'}{(R^2 - 2 R^3 \cost / r' + R^4 / {r'}^2)^{3/2}} \right) \\
		&= q \left( \frac{R - r' \cost}{{r'}^3 (R^2 / {r'}^2 - 2 R \cost / r' + 1)^{3/2}} - \frac{R}{r'} \frac{R - R^2 \cost / r'}{R^3 (1 - 2 R \cost / r' + R^2 / {r'}^2)^{3/2}} \right) \\
		&= \frac{q}{r'} \frac{R^3 - R^2 r' \cost - R {r'}^2 + R^2 r' \cost}{R^2 {r'}^2 (R^2 / {r'}^2 - 2 R \cost / r' + 1)^{3/2}}
		= \frac{q}{R {r'}^3} \frac{R^2 - {r'}^2}{(R^2 / {r'}^2 - 2 R \cost / r' + 1)^{3/2}}.
	\end{align*}
	Finally, feeding this into \refeq{scdeq},
	\beq
		\sig = -\frac{\vE \cdot \rh}{4\pi}
		= \frac{q}{4\pi R {r'}^3} \frac{{r'}^2 - R^2}{(R^2 / {r'}^2 - 2 R \cost / r' + 1)^{3/2}}
		= \frac{q}{4\pi R \absxp^3} \frac{\absxp^2 - R^2}{(R^2 / \absxp^2 - 2 R \cost / \absxp + 1)^{3/2}}.
	\eeq
\end{solution}
\vfix


\newcommand{\vEo}{\vE_0}
\newcommand{\del}{\delta}
\newcommand{\Etht}{E_\tht}
\newcommand{\Fr}{F_r}

\begin{problem}
	Find the force $\vF$ that must be exerted on the point charge in order to hold it in place.
\end{problem}

\begin{solution}
	The total force on a charge distribution arises only from the external electric field $\vEo$, and is given by Eq.~(2.42) in the lecture notes:
	\beq
		\vF = \int \rhox \, \vEo(\vx) \dcx.
	\eeq
	The force required to keep the point charge in place is equal and opposite to this force, so we need to insert a minus sign.  We also need the $\tht$ component of the field inside the conductor, which is
	\beq
		\Etht(\vx) = -\frac{q}{r} \pdv{\Gdxxp}{\tht}
		= -q \left( \frac{r' \sint}{(r^2 - 2 r r' \cost + {r'}^2)^{3/2}} - \frac{R^3 \sint}{{r'}^2 (r^2 - 2 R^2 r \cost / r' + R^4 / {r'}^2)^{3/2}} \right).
	\eeq
	The charge density for a point charge located at $\vx'$ is given by $\rhox = q \, \delta(\vx - \vx')$.  Evaluating the integral, we have
	\beq
		\vF = -\int q \, \delta(\vx - \vx') \, \vE(\vx) \dcx
		= -q \vE(\vx').
	\eeq
	Recall that we chose $\vx'$ to point along the $z$ axis, so $\tht' = 0$.  The $\tht$ component of $\vF$ is then $0$, and the $r$ component is
	\begin{align*}
		\Fr &= -q^2 \left( \frac{r' - r'}{({r'}^2 - 2 {r'}^2 + {r'}^2)^{3/2}} - \frac{R}{r'} \frac{r' - R^2 / r'}{({r'}^2 - 2 R^2 + R^4 / {r'}^2)^{3/2}} \right)
		= q^2 R {r'}^2 \frac{r' - R^2 / r'}{({r'}^4 - 2 R^2 {r'}^2 + R^4)^{3/2}} \\
		&= -q^2 R {r'}^2 \frac{({r'}^2 - R^2) / r'}{({r'}^2 - R^2)^3}
		= -q^2 \frac{R r'}{({r'}^2 - R^2)^2}.
	\end{align*}
	Since only the $r$ component of $\vF$ is nonzero, it points in the $z$ direction, which we chose to be equivalent to the unit vector $\vx' / \absxp$.  Therefore,
	\beq
		\vF = -q^2 \frac{R \absxp}{(R^2 - \absxp^2)^2} \frac{\vx'}{\absxp}
		= -q^2 \frac{R}{(R^2 - \absxp^2)^2} \vx'.
	\eeq
\end{solution}







\state{Acoustic and optic phonons in the diatomic chain}{
	In the diatomic chain, we take the unit cell to be of length $a$, and take $\xA$ and $\xB$ to be the coordinates of the A and B atoms within the unit cell.  Hence, in the $n$th cell,
	\al{
		\rnA &= n a + \xA; &
		\rnB &= n a + \xB
	}
	\vfix
}

%\prob{}{
%	In the equations of motion Eq.~(2.30), look for solutions of the form
%	\eq{
%		\unalp = \ealpq \exp( i [ q \rnalp - \omgq t ] ) + \ealpsq \exp( i [-q \rnalp + \omgq t] )
%	}
%	where $\alp = A$ or $B$, and $\ealp$ are complex numbers that give the amplitude and phase of the oscillation of the two atoms.
%}



%\prob{}{
%	Separating out the terms that have the same time dependence, show that (for equal masses, $\mA = \mB = m$)
%	\al{
%		m \omgsq \eAq &= \DAAq \eAq + \DABq \eBq, \\
%		m \omgsq \eBq &= \DBAq \eAq + \DBBq \eBq,
%	}
%	where
%	\al{
%		\DAAq &= \DBBq = K + K', \\
%		-\DABq &= K \exp( i q [ \rnB - \rnA ] ) + K' \exp( i q [ \rnmqB - \rnA ] ), \\
%		-\DBAq &= K \exp( i q [ \rnA - \rnB ] ) + K' \exp( i q [ \rnpqA - \rnB ] )
%	}
%}



%\prob{}{
%	Check that $\DAB = \DsBA$.
%}



%\prob{}{
%	The $2 \times 2$ matrix equation can have a nontrivial solution if the determinant vanishes:
%	\eq{
%		\mqty| 	\DAAq - m \omgsq & \DABq \\
%				\DBAq & \DBBq - m \omgsq |
%		= 0.
%	}
%	Hence show that the frequencies of the modes are given by
%	\eq{
%		m \omgsq = K + K' \pm \sqrt{ (K + K')^2 - 4 K K' \sin[2]( \frac{q a}{2} ) }.
%	}
%	\vfix
%}



%\prob{}{
%	Sketch the dispersion relations when $K / K' = 2$.
%}



%\prob{}{
%	Discuss what happens if $K = K'$.
%}






\state{Partition function as a generating functional}{
	Consider the Gibbs distribution of the system described in Problem~5.  For simplicity neglect the kinetic energy. Let $n(x) = \sumi \del(x - \xii)$ be the density, and $\evnx$ its expectation value. Let $C(x, y) = \ev{\del n(x) \,\del n(y)}$, where $\del n(x) = n(x) - \evn$, be the two-point correlation function.
}

%
%	6.1
%

\prob{}{
	Show that $\evnx = -T \,\deldvs{\ln Z}{U(x)}$, where $Z[U(x)]$ is the partition function of the Gibbs distribution treated as a functional of the potential $U$.
}

\sol{
	The expectation value of $n(x)$ is
	\eqn{evn}{
		\evnx = \frac{1}{Z} \int n(x) \,e^{-\bet H} \prodjN \ddxjj
		= \frac{1}{Z} \int \sumiN \del(x - \xii) \,e^{-\bet H} \prodjN \ddxjj,
	}
	where $Z$ is the partition function.
	\clearpage
	Adapting the Hamiltonian in Eq.~\refeq{Ham5}, we have
	\eq{
		H = \sumiN U(\xii) + \sumiN \sumji V(\xii - \xjj).
	}
	The partition function of the Gibbs distribution for this system is then
	\al{
		Z &= \int e^{-\bet H} \prodjN \ddxjj
		= \int \exp( \bet \sumiN U(\xii) + \bet \sumiN \sumji V(\xii - \xjj) ) \prodkN \ddxkk.
	}
	The basic definition of the functional derivative in one dimension is~\cite[p.~289]{Peskin}
	\al{
		\deldv{J(y)}{J(x)} &= \del(x - y), &
		\deldv{}{J(x)} \int J(y) \,\phi(y) \dd{y} &= \int \del(x - y) \,\phi(y) \dd{y}
		= \phi(x).
	}
	Note that
	\eqn{thing5}{
		\deldv{\ln Z}{U(x)}
		= \pdv{\ln Z}{Z} \deldv{Z}{U(x)}
		= \frac{1}{Z} \deldv{Z}{U(x)},
	}
	and that
	\aln{
		\deldv{Z}{U(x)} &= \deldv{}{U(x)} \int \exp( \bet \sumiN U(\xii) + \bet \sumiN \sumji V(\xii - \xjj) ) \prodkN \ddxkk
%		&= \int -\bet \sumiN \deldv{U(\xii)}{U(x)} \exp( \bet \sumiN U(\xii) + \bet \sumiN \sumji V(\xii - \xjj) ) \prodkN \ddxkk
		= \int -\bet \sumiN \deldv{U(\xii)}{U(x)} \, e^{-\bet H} \prodjN \ddxjj \notag \\
		&= -\frac{1}{T} \int \sumiN \del(x - \xii) \,e^{-\bet H} \prodjN \ddxjj
		= -\frac{Z}{T} \evnx, \label{ans6.1}
	}
	where we have used Eq.~\refeq{evn}.  Then, from Eq.~\refeq{thing5}, we have
	\eq{
		-T \deldv{\ln Z}{U(x)} = \frac{T}{Z} \deldv{Z}{U(x)}
		= \ans{ \evnx, }
	}
	as desired. \qed
}

%
%	6.2
%

\prob{}{
	Show that
	\eqn{show6.2}{
		C(x, y) = T^2 \deldvm{\ln Z}{U(x)}{U(y)}
		= -T \deldv{\evnx}{U(y)}
		= -T \deldv{\evny}{U(x)}.
	}
}

\sol{
	Firstly,
	\eq{
		C(x, y) = \ev{n(x) \,n(y)} - \evn^2
		= \frac{1}{Z} \int n(x) \,n(y) \,e^{-\bet H} \prodjN \ddxjj - \evn^2.
	}
	The final two equalities of Eq.~\refeq{show6.2} follow directly from Eq.~\refeq{ans6.1} and the fact that the order of the derivatives is interchangeable:
	\al{
		T^2 \deldvm{\ln Z}{U(x)}{U(y)} &= T \deldv{}{U(y)} \paren{ T \deldv{\ln Z}{U(x)} }
		= \ans{ -T \deldv{\evnx}{U(y)}, } \\[2ex]
		T^2 \deldvm{\ln Z}{U(x)}{U(y)} &= T \deldv{}{U(x)} \paren{ T \deldv{\ln Z}{U(y)} }
		= \ans{ -T \deldv{\evny}{U(x)}. }
	}
	To prove the first equality, we will show that $C(x, y) = -T \,\del\evnx / \del U(y)$.  Note that
	\aln{
		\deldv{\evnx}{U(y)} &= \deldv{}{U(y)} \paren{ \frac{1}{Z} \int n(x) \,e^{-\bet H} \prodjN \ddxjj } \notag \\
		&= \deldv{}{U(y)} \paren{ \frac{1}{Z} } \int n(x) \,e^{-\bet H} \prodjN \ddxjj + \frac{1}{Z} \deldv{}{U(y)} \paren{ \int n(x) \,e^{-\bet H} \prodjN \ddxjj }. \label{thing6.2}
	}
	For the first term,
	\eq{
		\deldv{}{U(y)} \paren{ \frac{1}{Z} } = \pdv{(1/Z)}{Z} \deldv{Z}{U(y)}
		= -\frac{1}{Z^2} \deldv{Z}{U(y)}
		= \frac{\evny}{Z T},
	}
	where we have used Eq.~\refeq{ans6.1}.  For the second term,
	\al{
		\deldv{}{U(y)} \int n(x) \,e^{-\bet H} \prodjN \ddxjj &= \deldv{}{U(y)} \int n(x) \exp( \bet \sumiN U(\xii) + \bet \sumiN \sumji V(\xii - \xjj) ) \prodkN \ddxkk \\
		&= \int -\bet n(x) \sumiN \deldv{U(\xii)}{U(y)} e^{-\bet H} \prodjN \ddxjj
		= -\frac{1}{T} \int n(x) \,n(y) e^{-\bet H} \prodjN \ddxjj.
	}
	Substituting back into Eq.~\refeq{thing6.2},
	\eq{
		\deldv{\evnx}{U(y)} = \frac{\evny}{Z T} \int n(x) \,e^{-\bet H} \prodjN \ddxjj - \frac{1}{Z T} \int n(x) \,n(y) e^{-\bet H} \prodjN \ddxjj
		= \frac{\evny \evnx}{T} - \frac{\ev{n(x) \,n(y)}}{T}.
	}
	Then
	\eq{
		-T \deldv{\evnx}{U(y)} = \ev{n(x) \,n(y)} - \evn^2
		= \ans{ C(x, y), }
	}
	as desired.  So we have proven Eq.~\refeq{show6.2} in its entirety. \qed
}




\makebib

\end{document}
