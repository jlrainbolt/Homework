\documentclass[11pt]{article}
\usepackage{homework}

\classname{361}
\homeworknum{5}



\begin{document}

% Environments

\newcommand{\state}[2]{\begin{statement}{#1} #2 \end{statement}}
\newcommand{\prob}[2]{\begin{problem}{#1} #2 \end{problem}}
\newcommand{\subprob}[1]{\begin{subproblem} #1 \end{subproblem}}
\newcommand{\sol}[1]{\begin{solution} #1 \end{solution}}
\newcommand{\fig}[2]{\begin{figure} \centering #2  \label{#1} \end{figure}}

\newcommand{\makebib}{
	\vfill
	\color{black}
	\bibliography{references}{}
	\bibliographystyle{lucas_unsrt}
}
	

% Implication

\newcommand{\qwhere}{\quad \text{where} \quad}
\newcommand{\qimplies}{\quad \implies \quad}
\newcommand{\impliesq}{\implies \quad}



% Brackets

\newcommand{\paren}[1]{\left( #1 \right)}
\newcommand{\brac}[1]{\left[ #1 \right]}


% Greek

\newcommand{\alp}{\alpha}
\newcommand{\bet}{\beta}
\newcommand{\gam}{\gamma}
\newcommand{\del}{\delta}
\newcommand{\eps}{\epsilon}
\newcommand{\zet}{\zeta}
\newcommand{\tht}{\theta}
\newcommand{\kap}{\kappa}
\newcommand{\lam}{\lambda}
\newcommand{\sig}{\sigma}
\newcommand{\ups}{\upsilon}
\newcommand{\omg}{\omega}

\newcommand{\Gam}{\Gamma}
\newcommand{\Del}{\Delta}
\newcommand{\Tht}{\Theta}
\newcommand{\Lam}{\Lambda}
\newcommand{\Sig}{\Sigma}
\newcommand{\Omg}{\Omega}
% Problem 1

\newcommand{\Psii}{\Psi^i}
\newcommand{\Psiix}{\Psii(x)}

\newcommand{\Pii}{\Pi^i}

\newcommand{\Phii}{\Phi^i}
\newcommand{\Phiix}{\Phii(x)}
\newcommand{\PhiN}{\Phi^N}
\newcommand{\PhiNx}{\PhiN(x)}
\newcommand{\Phiq}{\Phi^1}
\newcommand{\Phiw}{\Phi^2}

\newcommand{\ddcx}{\dd[3]{x}}

\newcommand{\delij}{\del^{i j}}
\newcommand{\delkl}{\del^{k l}}
\newcommand{\delil}{\del^{i l}}
\newcommand{\deljk}{\del^{j k}}
\newcommand{\delik}{\del^{i k}}
\newcommand{\deljl}{\del^{j l}}

\newcommand{\DF}{D_F}

\newcommand{\sigx}{\sig(x)}

\newcommand{\pii}{\pi^i}
\newcommand{\pij}{\pi^j}
\newcommand{\pik}{\pi^k}
\newcommand{\pil}{\pi^l}
\newcommand{\piix}{\pi(x)}

\newcommand{\pq}{p_1}
\newcommand{\pw}{p_2}
\newcommand{\pe}{p_3}
\newcommand{\pr}{p_4}

\newcommand{\vp}{\vb{p}}
\newcommand{\vpsi}{\vp_i}

\newcommand{\mpi}{m_\pi}



\state{Exchange}{
	Consider single-particle wavefunctions on two neighboring identical atoms $\psiA, \psiB$, which may be assumed real.  These are to be used as the basis for a two-electron state.  Show that the charge density in a singlet (triplet) state made out of the two orbitals is given by
	\eq{
		\rhor = \abs{\psiAr}^2 + \abs{\psiBr}^2 \pm 2 \bkpsiAB \psiAr \psiBr.
	}
	Explain why the singlet state will usually be lower in energy.
}






\state{One-dimensional spin waves}{
	Assume a one-dimensional chain of spins, precessing according to Eq.~(6.30).  By considering two neighbors of the $n$th spin, as in Fig.~6.8, each at relative angles $\tht$, show that the rate of precession according to Eq.~(6.30) is
	\eq{
		\omg = \frac{4 J S}{\hbar} [ 1 - \cos(\tht) ].
	}
	Hence show that for a spin wave of wavevector $q$, the dispersion is
	\eq{
		\hbar \omg = 4 J S [ 1 - \cos(q a) ].
	}
	\vfix
}






\state{Colossal magnetoresistance}{
	In a material like that shown in Fig.~6.10 the magnetism arises from a mechanism called double exchange, which is a version of itinerant exchange but involving two types of d-bands.  The prototype compound is $\LaSrMnO$, where the valence of La is $3+$ and Sr is $2+$.  This is a cubic (perovskite) crystal structure where the $\Mn$ ions are nominally equidistant from six oxygen neighbors in three Cartesian directions.
}

\prob{
	Explain why the valence of $\Mn$ in the compound $\LaSrMnO$ is expected to be between $3+$ and $4+$ and that the occupancy of the d-levels is expected to be $4 - x$ electrons per $\Mn$ ion.
}



\prob{
	The degeneracy of the 5 d-levels in the free ion is split by the cubic environment into a low energy three-fold degenerate subset (whose notation is $t_{2g}$) and a higher energy doubly degenerate orbital set ($\eg$).  Explain why the spin configurations of these levels for the $\Mn^{3+}$ and $\Mn^{4+}$ ions are expected to be as shown in Fig.~6.12.
}



\prob{
	The lowest three electron states can be regarded as forming a classical spin $S = 3/2$ which has negligible hopping from site to site, whereas the highest state is potentially itinerant.  Now consider two neighboring sites $i,j$ in the solid, each having the same ``core'' spin $S$, and sharing a single itinerant $\eg$ electron, that has a tight-binding matrix element
	\eq{
		t = \mel*{\phiegrRi}{H}{\phiegrRj}
	}
	for hopping from site to site.
	
	Explain the origin of the terms
	\eq{
		\Hint = J \sumi \shi \vdot \vSi + \Jx \sumij \vSi \vdot \vSj
	}
	in the total Hamiltonian ($\shi$ is the spin of the $\eg$ electron) and suggest relative magnitudes of $U$, $J$, and $\Jx$.
}


\prob{
	Consider two neighboring core spins $\vSi, \vSj$ that are at a relative angle $\thtij$.  By considering that the spin wavefunction of the itinerant electron must, for $J \gg t$, be always aligned with the local core spin $\vS$, explain why the {\Schrodinger} equation for the itinerant electron can be simplified to one in which the tight-binding hopping matrix element from site $i$ to site $j$ is replaced by
	\eq{
		\teff = r \cos(\frac{\thtij}{2}).
	}
	To do this, you may wish to note that under a rotation by an angle $\tht$, the spin wavefunction transforms as
	\eq{
		\mqty( \kupp \\ \kdnp ) = \mqty( \cos(\tht / 2) & \sin(\tht / 2) \\ -\sin(\tht / 2) & \cos(\tht / 2) ) \mqty( \kup \\ \kdn ).
	}
	\vfix
}



\prob{
	Sketch the density of states of the itinerant electrons for different alignments of the core spins $\vS$:
	\begin{enumerate}
		\item \emph{ferro}magnetic~(all core spins aligned),
		\item \emph{antiferro}magnetic~(all neighboring core spins anti-aligned),
		\item \emph{para}magnetic~(core spins randomly aligned).
	\end{enumerate}
	Discuss how the total free energies of these states differ, and suggest what is the magnetic ground state when $x = 0$, and when $t x > \Jx$; give rough estimates of the transition temperatures of the ordered magnetic states toward high temperature paramagnetism.
}



\prob{
	Figure~6.13 shows the resistivity as a function of temperature of several samples of $\LaSrMnO$ with different concentrations $x$, as well as the magnetic field dependence of the resistivity (which gives rise to the label ``colossal'' magnetoresistance).  Discuss this data in light of the results above.
}


%\makebib

\end{document}
