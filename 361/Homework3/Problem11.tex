\state{Ferromagnetism in the HF approximation}{
	Previously, we considered the unpolarized spin state, which is a paramagnet.  Now consider a fully spin polarized state at the same density: the Hartree--Fock Slater determinant corresponds to singly occupying each state in the  Fermi sphere.  In analogy to Eq.~(4.93), compute the total energy of the spin polarized state, and show that this is lower in energy than the unpolarized state if $\rs > 5.45$ in the Hartree--Fock approximation.
}

\sol{
	Equation~(4.93) is
	\eqn{thing11}{
		E = N \paren{ \frac{3}{5} \frac{\hbar^2 \kF^2}{2m} - \frac{3}{4} \frac{e^2 \kF}{\pi} }
		= N \paren{ \frac{3}{5} (\kF \aB)^2 - \frac{3}{2\pi} \kF \aB } \Ry,
	}
	where we have used the definition of the Rydberg in (4.87):
	\eq{
		\SI{1}{\rydberg} = \frac{\hbar^2}{2 m \aB^2}
		= \frac{e^2}{2 \aB},
	}
	where $\aB$ is the Bohr radius~\cite[p.~682]{Ashcroft}.
	
	The exchange term of the Hartree-Fock equations (4.84) in the lecture notes is
	\eq{
		\sumj \int \ddvrp \frac{e^2}{\absrmrp} \psijsrp \psiirp \psijr \delsigij,
	}
	which is nonzero only for electrons of the same spin.  This means we can write an expression like Eq.~\refeq{thing11} for electrons of each spin state~\cite[p.~683]{Ashcroft}:
	\aln{ \label{energy11}
		\Eup &= \Nup \paren{ \frac{3}{5} (\kup \aB)^2 - \frac{3}{2\pi} \kup \aB } \Ry, &
		\Edn &= \Ndn \paren{ \frac{3}{5} (\kdn \aB)^2 - \frac{3}{2\pi} \kdn \aB } \Ry.
	}
	Here, $\kup$ and $\kdn$ are the wave vectors corresponding to the highest energy states occupied by spin up and by spin down electrons, respectively.
	
	The total energy is the sum of the energies for each spin state: $E = \Eup + \Edn$.  The total electron density is the sum of the densities of spin up and of spin down electrons~\cite[p.~683]{Ashcroft}:
	\eqn{n11}{
		n = \frac{N}{V}
		= \frac{\Nup}{V} + \frac{\Ndn}{V}
		= \frac{4 \kup^3 / 3}{(2\pi)^3} + \frac{4 \kdn^3 / 3}{(2\pi)^3}
		= \frac{\kup^3}{6 \pi^2} + \frac{\kdn^3}{6 \pi^2}
		= \frac{\kF}{3 \pi^2}.
	}
	For a fully spin polarized state, either $\Nup$ or $\Ndn$ is 0 and the other $N$; that is, all electrons are in the same spin state.  Say $\Ndn = N$ and $\Nup = 0$.  Then $E = \Edn$, and~\cite[p.~683]{Ashcroft}
	\eq{
		\frac{\kdn^3}{6 \pi^2} = \frac{\kF}{3 \pi^2}
		\qimplies
		\kdn = 2^{1/3} \kF.
	}
	Making these substitutions in Eq.~\refeq{energy11}, we have~\cite[p.~683]{Ashcroft}
	\aln{
		E &= N \paren{ \frac{3}{5} (2^{1/3} \kF \aB)^2 - \frac{3}{2\pi} 2^{1/3} \kF \aB } \Ry \notag \\
		&= N \paren{ \frac{2^{2/3} 3}{5} (\kF \aB)^2 - \frac{2^{1/3} 3}{2 \pi} \kF \aB } \Ry \notag \\
		&= \ans{ N \paren{ \frac{2^{2/3} 3}{5} \frac{\hbar^2 \kF^2}{2m} - \frac{2^{1/3} 3}{4} \frac{e^2 \kF}{\pi} }. } \label{ans11}
	}
	
	Call the energy for the spin unpolarized state $\Epar$, and the energy for the fully polarized state $\Efer$.  Using Ashcroft \& Mermin~(2.22),
	\eq{
		\kF = \frac{(9\pi / 4)^3}{\rs} \approx \frac{1.92}{\rs},
	}
	we have
	\al{
		\Epar &= N \brac{ \frac{3}{5} \paren{ 1.92 \frac{\aB}{\rs} }^2 - \frac{3}{2\pi} 1.92 \frac{\aB}{\rs} } \Ry, &
		\Efer &= N \brac{ \frac{2^{2/3} 3}{5} \paren{ 1.92 \frac{\aB}{\rs} }^2 - \frac{2^{1/3} 3}{2 \pi} 1.92 \frac{\aB}{\rs} } \Ry
	}
	Then $\Efer < \Epar$ can be written as
	\eq{
		\frac{2^{2/3} 3}{5} \paren{ 1.92 \frac{\aB}{\rs} }^2 - \frac{2^{1/3} 3}{2 \pi} 1.92 \frac{\aB}{\rs} < \frac{3}{5} \paren{ 1.92 \frac{\aB}{\rs} }^2 - \frac{3}{2\pi} 1.92 \frac{\aB}{\rs},
	}
	which is equivalent to
	\al{
		0 > (2^{2/3} - 1) \frac{3}{5} \paren{ 1.92 \frac{\aB}{\rs} }^2 - (2^{1}{3} - 1) \frac{3}{2\pi} 1.92 \frac{\aB}{\rs} \\
		\approx 1.299 \paren{ \frac{\aB}{\rs} }^2 - 0.238 \frac{\aB}{\rs}.
	}
	The solutions of the quadratic equation are 0 and
	\eq{
		\frac{\aB}{\rs} < \frac{0.238 + 0.238}{2 \times 1.299}
		\approx 0.183,
	}
	which is equivalent to
	\eq{
		\frac{\rs}{\aB} > 5.45.
	}
	Thus, we have shown that \ans{$\rs / \aB > 5.45 \implies \Efer < \Epar$}. \qed
}