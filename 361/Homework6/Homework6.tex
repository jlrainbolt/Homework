\documentclass[11pt]{article}
\usepackage{homework}

\classname{361}
\homeworknum{6}



\begin{document}

% Environments

\newcommand{\state}[2]{\begin{statement}{#1} #2 \end{statement}}
\newcommand{\prob}[2]{\begin{problem}{#1} #2 \end{problem}}
\newcommand{\subprob}[1]{\begin{subproblem} #1 \end{subproblem}}
\newcommand{\sol}[1]{\begin{solution} #1 \end{solution}}
\newcommand{\fig}[2]{\begin{figure} \centering #2  \label{#1} \end{figure}}

\newcommand{\makebib}{
	\vfill
	\color{black}
	\bibliography{references}{}
	\bibliographystyle{lucas_unsrt}
}
	

% Implication

\newcommand{\qwhere}{\quad \text{where} \quad}
\newcommand{\qimplies}{\quad \implies \quad}
\newcommand{\impliesq}{\implies \quad}



% Brackets

\newcommand{\paren}[1]{\left( #1 \right)}
\newcommand{\brac}[1]{\left[ #1 \right]}


% Greek

\newcommand{\alp}{\alpha}
\newcommand{\bet}{\beta}
\newcommand{\gam}{\gamma}
\newcommand{\del}{\delta}
\newcommand{\eps}{\epsilon}
\newcommand{\zet}{\zeta}
\newcommand{\tht}{\theta}
\newcommand{\kap}{\kappa}
\newcommand{\lam}{\lambda}
\newcommand{\sig}{\sigma}
\newcommand{\ups}{\upsilon}
\newcommand{\omg}{\omega}

\newcommand{\Gam}{\Gamma}
\newcommand{\Del}{\Delta}
\newcommand{\Tht}{\Theta}
\newcommand{\Lam}{\Lambda}
\newcommand{\Sig}{\Sigma}
\newcommand{\Omg}{\Omega}
% Problem 1

\newcommand{\Psii}{\Psi^i}
\newcommand{\Psiix}{\Psii(x)}

\newcommand{\Pii}{\Pi^i}

\newcommand{\Phii}{\Phi^i}
\newcommand{\Phiix}{\Phii(x)}
\newcommand{\PhiN}{\Phi^N}
\newcommand{\PhiNx}{\PhiN(x)}
\newcommand{\Phiq}{\Phi^1}
\newcommand{\Phiw}{\Phi^2}

\newcommand{\ddcx}{\dd[3]{x}}

\newcommand{\delij}{\del^{i j}}
\newcommand{\delkl}{\del^{k l}}
\newcommand{\delil}{\del^{i l}}
\newcommand{\deljk}{\del^{j k}}
\newcommand{\delik}{\del^{i k}}
\newcommand{\deljl}{\del^{j l}}

\newcommand{\DF}{D_F}

\newcommand{\sigx}{\sig(x)}

\newcommand{\pii}{\pi^i}
\newcommand{\pij}{\pi^j}
\newcommand{\pik}{\pi^k}
\newcommand{\pil}{\pi^l}
\newcommand{\piix}{\pi(x)}

\newcommand{\pq}{p_1}
\newcommand{\pw}{p_2}
\newcommand{\pe}{p_3}
\newcommand{\pr}{p_4}

\newcommand{\vp}{\vb{p}}
\newcommand{\vpsi}{\vp_i}

\newcommand{\mpi}{m_\pi}

\state{Electron-phonon interaction}{
	Write short notes explaining the physical effects that may be produced by the electron-phonon interaction in metals.
}

\sol{
	Phonons cause the crystal lattice to distort on a local scale, which moves the ions from their equilibrium positions.  Since the ions carry charge, this disturbance creates an electric potential that is screened by nearby conduction electrons.  The potential scatters electrons from state $\vk$ to state $\vk'$, which alters the density distribution of the electron gas.  The disturbance in the electron density caused by the scattering may in turn create a new phonon or lattice distortion, the degree of which is determined by the phonon susceptibility of the crystal~[lecture notes, p.~129--130]\cite[pp.~671--672]{Kittel}\cite[p.~512]{Ashcroft}.

	The lattice distortion created by an electron density fluctuation lasts longer than the fluctuation itself, and creates more local electron density fluctuations over its lifetime.  This creates an effective ``attraction'' between conduction electrons in the metal, which can lead to superconductivity and the creation of Cooper pairs.  In addition, the interaction between phonons and electrons causes electrons to effectively carry polarized lattice distortions with them as they move.  This decreases their effective velocity and increases their effective mass~[lecture notes, p.~130--134]\cite[pp.~672]{Kittel}.
}




\state{Electronic mass enhancement}{\hfix}

\prob{The integral in Eq.~(7.10) can be approximated by neglecting the momentum dependence of the coupling constant $g$, and replacing the phonon frequency by the characteristic scale $\omgD$.  Show that in this case the integral becomes
	\eq{
		g^2 \intnimu \ddepsp \frac{\Nepsp}{(\eps' - \epsk)^2 - \omgD^2}
	}
	where $\Neps$ is the density of states in energy.
}

\sol{
	Equation~(7.10) is
	\eqn{7.10}{
		\epsk - \mu = \epsko - \mu - \int \frac{\ddkp}{(2\pi)^3} \frac{\abs{\gkmkp}^2 \nkp}{(\epsk - \epskp)^2 - \omg (\vk - \vk')^2}.
	}
	Applying $\omg \to \omgD$ and neglecting the momentum dependence of $g$,
	\eq{
		\int \frac{\ddkp}{(2\pi)^3} \frac{\abs{\gkmkp}^2 \nkp}{(\epsk - \epskp)^2 - \omg (\vk - \vk')^2}
		\approx g^2 \int \frac{\ddkp}{(2\pi)^3} \frac{\nkp}{(\epsk - \epskp)^2 - \omgD (\vk - \vk')^2}.
	}
	Since the mass enhancement only exists for states whose energies are the same within $\hbar \omgD$~(lecture notes p.~132), $(\vk - \vk')^2 \approx \omgD$.  Thus
	\eq{
		g^2 \int \frac{\ddkp}{(2\pi)^3} \frac{\nkp}{(\epsk - \epskp)^2 - \omgD (\vk - \vk')^2}
		\approx g^2 \int \frac{\ddkp}{(2\pi)^3} \frac{\nkp}{(\epsk - \epskp)^2 - \omgD^2}.
	}
	
	For an electron gas, (2.10) in the lecture notes gives
	\eq{
		\Neps \ddeps = 2 \frac{4 \pi k^2 \ddk}{(2\pi)^3 / V}
		= \frac{V k^2}{\pi^2} \ddk.
	}
	Making this substitution,
	\eq{
		g^2 \int \frac{\ddepsp}{(2\pi)^3} \frac{\Nepsp \pi^2}{V \vk^2} \frac{\nkp}{(\epsk - \epskp)^2 - \omgD^2}
	}
	
	\hl{????} 

	
%	Since $\epsk = \hbar^2 \vk^2 / 2 m$ by (2.3) in the lecture notes, $\ddeps = \hbar^2 k \ddk / m$:
%	\eq{
%		g^2 \int \frac{\ddkp}{(2\pi)^3} \frac{\nkp}{(\epsk - \epskp)^2 - \omgD (\vk - \vk')^2}.
%		= g^2 \int \frac{\ddepsp}{(2\pi)^3} \frac{m}{\hbar^2 k'} \frac{\nkp}{(\epsk - \epskp)^2 - \omgD (\vk - \vk')^2}.
%	}
	
%	From (2.38) in the lecture notes, $\omgD^3 = 6 \pi^2 v^3 N / V$.  This is the same as replacing the cutoff in momentum space by a sphere of radius $\kD = \omgD / v$ by (2.39).
}

\clearpage
\prob{
	Since the dominant part of the integral comes from energies near the Fermi energy, we can usually replace $\Neps$ by $\Nmu$.  Making this approximation, show that for energies $\abs{\epsk - \mu} \ll \omgD$
	\eq{
		\epsk - \mu = \frac{\epsko - \mu}{1 + \lam}
	}
	where
	\eq{
		\lam = \frac{g^2 \Nmu}{\omgD^2}.
	}
}

\sol{
	Applying Eq.~\refeq{ans2a} and this approximation, Eq.~\refeq{7.10} becomes
	\eqn{thing2b}{
		\epsk - \mu = \epsko - \mu - g^2 \Nmu \intnimu \frac{\ddepsp}{(\eps' - \epsk)^2 - \omgD^2}.
	}
	Since we assume that only energies very close to the Fermi energy $\epsF \approx \mu$ contribute, we approximate the integral by the indefinite integral at $\eps' = \mu$.  Thus (using Mathematica)
	\eq{
		\intnimu \frac{\ddepsp}{(\eps' - \epsk)^2 - \omgD^2} \approx \int \frac{\ddmu}{(\mu - \epsk)^2 - \omgD^2}
		= -\frac{1}{\omgD} \tanh[-1](\frac{\mu - \epsk}{\omgD}).
	}
	Since $\abs{\epsk - \mu} \ll \omgD$, we can perform a Taylor expansion (again using Mathematica):
	\eq{
		\tanh[-1](\frac{\mu - \epsk}{\omgD}) \approx \frac{\mu - \epsk}{\omgD}.
	}
	So we have for Eq.~\refeq{thing2b}
	\eq{
		\epsk - \mu = \epsko - \mu + g^2 \Nmu \frac{\mu - \epsk}{\omgD^2}
		= \epsko - \mu - \lam (\epsk - \mu).
	}
	We assume that replacing $\epsk$ by $\epsko$ on the right side, thereby  ignoring the ionic correction to the screening in that term~\cite[p.~520]{Ashcroft}, is a valid approximation in this regime.  Then we have
	\eq{
		\epsk - \mu = (\epsko - \mu) (1 - \lam)
		\approx \frac{\epsko - \mu}{1 + \lam},
	}
	where we have simply used the Taylor expansion $1 / (1 + x) \approx 1 - x$ for small $x$.  In doing so we have assumed $\lam$, and therefore the mass enhancement, is small.  Nevertheless, we have achieved the desired result. \qed
}



\prob{
	Making the approximation $\Neps \approx \Nmu$, show that for energies $\abs{\epsk - \mu}$ several times $\omgD$ the correction to $\epsk$ is of order
	\eq{
		\lam \frac{\omgD^2}{(\epsk - \mu)^2} (\epsk - \mu).
	}
}

\sol{
	In this limit, we again approximate the integral in Eq.~\refeq{thing2b} as an antiderivative at $\eps' = \mu$.  In this regime we also approximate the denominator of the integrand by $(\eps' - \epsk)^2$, since $(\eps' - \epsk)^2 \gg \omgD^2$.  Then
	\eq{
		\intnimu \frac{\ddepsp}{(\eps' - \epsk)^2 - \omgD^2} \approx \int \frac{\ddmu}{(\mu - \epsk)^2}
		= -\frac{1}{\mu - \epsk},
	}
	so the correction term in Eq.~\refeq{thing2b} becomes
	\eq{
		\frac{g^2 \Nmu}{\mu - \epsk} = -\frac{\lam \omgD^2}{\epsk - \mu}
		\propto \lam \frac{\omgD^2}{(\epsk - \mu)^2} (\epsk - \mu)
	}
	as we wanted to show. \qed
}






\state{Cooper's problem}{
	The wavefunction of a Cooper pair of electrons added to the Fermi sea is
	\eq{
		\kpsiC = \sumkkF \gvk \ckupd \cmkdnd \kFS,
	}
	where only terms in the sum for $k > \kF$ are allowed.  We can now test out the pair wavefunction with the Hamiltonian
	\eqn{given3}{
		H = \sump \epsp \cpd \cpo + \frac{1}{2} \sumpppq \Vq \cpd \cppd \cppmq \cppq
	}
	applied to the two electrons in question, but leaving the fermi sea inert.  $\Vq$ is here taken to be an \emph{attractive} interaction.
}

\prob{
	Show that the first term in Eq.~\refeq{given3} operating on $\kpsiC$ is
	\eq{
		\Ho \kpsiC = \sumpksig \epsp \gk \cpsd \cps \ckupd \cmkdnd \kFS
		= \sumk 2 \epsk \gk \ckupd \cmkdnd \kFS.
	}
	(Hint: the trick in all of these operator manipulations is to move the annihilation operator to the RHS, so that it can destroy the vacuum state.  Along the way, it has to anticommute with the creation operators initially on its right and these anticommutators always generate an extra delta function.  The two terms in the last equation come because we must have either $p = k$, $\sig = {\up}$, or $p = -k$, $\sig = \dn$ and $\epsmp = \epsp$.  Remember that for this toy problem alone, we don't apply the Hamiltonian to the Fermi sea.)
}



\prob{
	Similarly, show that the operation of the second term in Eq.~\refeq{given3} gives
	\eq{
		\Hint = \sumkpppqsigsigp \Vq \gk \cpsd \cppspd \delppqk \delsup \delppmqk \delspdn \kFS
		= \sumkkpkF \Vkmkp \gkp \ckupd \cmkdnd \kFS.
	}
	Getting to the final equation involves a little crafty relabeling of the momenta in the sum.  This gets us to the two-particle {\Schrodinger} equation Eq.~(7.19).
}


\makebib

\end{document}
