\documentclass[11pt]{article}
\usepackage{homework}

\classname{361}
\homeworknum{0}



\begin{document}

% Environments

\newcommand{\state}[2]{\begin{statement}{#1} #2 \end{statement}}
\newcommand{\prob}[2]{\begin{problem}{#1} #2 \end{problem}}
\newcommand{\subprob}[1]{\begin{subproblem} #1 \end{subproblem}}
\newcommand{\sol}[1]{\begin{solution} #1 \end{solution}}
\newcommand{\fig}[2]{\begin{figure} \centering #2  \label{#1} \end{figure}}

\newcommand{\makebib}{
	\vfill
	\color{black}
	\bibliography{references}{}
	\bibliographystyle{lucas_unsrt}
}
	

% Implication

\newcommand{\qwhere}{\quad \text{where} \quad}
\newcommand{\qimplies}{\quad \implies \quad}
\newcommand{\impliesq}{\implies \quad}



% Brackets

\newcommand{\paren}[1]{\left( #1 \right)}
\newcommand{\brac}[1]{\left[ #1 \right]}


% Greek

\newcommand{\alp}{\alpha}
\newcommand{\bet}{\beta}
\newcommand{\gam}{\gamma}
\newcommand{\del}{\delta}
\newcommand{\eps}{\epsilon}
\newcommand{\zet}{\zeta}
\newcommand{\tht}{\theta}
\newcommand{\kap}{\kappa}
\newcommand{\lam}{\lambda}
\newcommand{\sig}{\sigma}
\newcommand{\ups}{\upsilon}
\newcommand{\omg}{\omega}

\newcommand{\Gam}{\Gamma}
\newcommand{\Del}{\Delta}
\newcommand{\Tht}{\Theta}
\newcommand{\Lam}{\Lambda}
\newcommand{\Sig}{\Sigma}
\newcommand{\Omg}{\Omega}
% Problem 1

\newcommand{\Psii}{\Psi^i}
\newcommand{\Psiix}{\Psii(x)}

\newcommand{\Pii}{\Pi^i}

\newcommand{\Phii}{\Phi^i}
\newcommand{\Phiix}{\Phii(x)}
\newcommand{\PhiN}{\Phi^N}
\newcommand{\PhiNx}{\PhiN(x)}
\newcommand{\Phiq}{\Phi^1}
\newcommand{\Phiw}{\Phi^2}

\newcommand{\ddcx}{\dd[3]{x}}

\newcommand{\delij}{\del^{i j}}
\newcommand{\delkl}{\del^{k l}}
\newcommand{\delil}{\del^{i l}}
\newcommand{\deljk}{\del^{j k}}
\newcommand{\delik}{\del^{i k}}
\newcommand{\deljl}{\del^{j l}}

\newcommand{\DF}{D_F}

\newcommand{\sigx}{\sig(x)}

\newcommand{\pii}{\pi^i}
\newcommand{\pij}{\pi^j}
\newcommand{\pik}{\pi^k}
\newcommand{\pil}{\pi^l}
\newcommand{\piix}{\pi(x)}

\newcommand{\pq}{p_1}
\newcommand{\pw}{p_2}
\newcommand{\pe}{p_3}
\newcommand{\pr}{p_4}

\newcommand{\vp}{\vb{p}}
\newcommand{\vpsi}{\vp_i}

\newcommand{\mpi}{m_\pi}

\state{}{
	The material $\LaSrCuO$ has a layered crystal structure that consists of two-dimensional square lattices of $\CuO$ planes (shown in Fig.~\ref{f1}) separated by layers of $\LaSrO$.  You may assume that $\La$ has valence $3+$, $\Sr$ valence $2+$, and $\Ox$ valence $2-$; the electrons from these cations are donated uniformly to the widely separated $\CuO$ layers, which thus have a two-dimensional electronic structure.  Neutral atomic $\Cu$ has the configuration $[\Ar] 4s^2 3d^9$.  In this compound, four of the $\Cu$ $d$ levels are completely filled, and there is a partially filled band formed from $d_{x^2 - y^2}$ orbitals.  You may assume the $\Cu$ $4s$ levels are unoccupied, and the $\Ox$ $2p$ levels are fully occupied.  Electronic dispersion perpendicular to the planes may be neglected.
	
	The band structure in the independent particle approximation is well described by a tight-binding model incorporating a single orbital (per unit cell) of $d_{x^2 - y^2}$ symmetry centered on the $\Cu$ atom, with nonzero Hamiltonian matrix elements $t$ between nearest neighbor orbitals in the $x$ and $y$ directions, and matrix elements $t'$ between second neighbors across the diagonals.
}

\prob{
	Show that in this approximation the energy dispersion of an electron is
	\eq{
		\Ek = 2 t [ \cos(\kx a) + \cos(\ky a) ] + 4 t' \cos(\kx a) \cos(\ky a).
	}
}



\prob{ \label{1b}
	What do you expect to be the signs of $t$ and $t'$?  Explain your reasoning.
}



\prob{
	For the case that $\abs{t' / t} = 0$, sketch the Fermi surface for $\Sr$ concentrations of $x = 0$, $x \approx 0.2$, and $x \approx 0.5$.
}



\prob{ \label{1d}
	How do these contours change qualitatively if $\abs{t' / t} \sim 0.1$?  (Choose the signs of $t$ and $t'$ that you proposed in \ref{1b}.)
}



\prob{
	Assuming the dispersion of \ref{1d}, sketch  the electronic density of states in energy, paying particular attention to the behavior near the edges of the band and at a saddle point in the middle of the band.
}



\prob{
	What would the independent electron model predict for the temperature dependence of the low-temperature electronic specific heat when the chemical potential is exactly at the saddle point near the middle of the band?
}



\prob{
	$\LaCuO$ is an antiferromagnetic insulator.  Suggest, and discuss, reason(s) why the ground state differs from that predicted by the band structure in the independent particle approximation.  Your answer should include a qualitative explanation of both the magnetic and the insulating behavior.
}






\state{}{ %\hfix 
(Mathematical results may be stated without proof, and marks will be given for demonstrating a rounded understanding of the topic, for appropriate use of examples, and for good use of figures.  One or two pages of text is expected, in bullet rather than essay form, for each part.)
}

\prob{
	Write short notes on the properties of a fermi liquid.
}



\prob{
	Write short notes on effects due to the electron-phonon interaction in metals.
}



\prob{
	Write short notes on the use of density functional theory to perform practical calculations of ground state properties of solids.
}



\prob{
	Write short notes on photoemission spectroscopy.
}






\state{}{
	A long-wavelength optical phonon in an insulating ionic solid is described by the following equation of motion:
	\eq{
		M \vudd + K \vuu = Q \vE,
	}
	where $\vuu$ is the atomic displacement, $M$ the mass, $K$ a local restoring force, and $\vE$ is the electric field.  A displacement of the ions by $\vuu$ results in a polarization
	\eq{
		\vP = n Q \vuu,
	}
	where $n$ is the density of the ions, and $Q$ their effective charge.  We will consider only longitudinal modes, where $\vD$, $\vE$, $\vuu$ are all parallel.
}

\prob{
	Neglecting any further electronic response of the solid, calculate the response of the phonon mode to a uniform external electric displacement field $\vD = \epso \vE + \vP$, oscillating with a frequency $\omg$.  Hence show that the phonon contribution to the frequency-dependent dielectric function, defined by $D = \epsion \epso E$ may be written
	\eq{
		\epsionomg = 1 - \frac{\OmgP^2}{\omg^2 - \Omgo^2},
	}
	giving formulae for $\OmgP$ and $\Omgo$ in terms of the constants $M$, $K$, $n$, and $Q$.
}



\prob{
	Figure~\ref{f3} shows measurements of the frequency squared of an optical phonon in an insulating oxide (solid line) and measurements of the reciprocal of the static dielectric constant $1 / \eps = 1 / \epsionomgo$ (dotted-dashed line) in the same material.  Using the model above, estimate the ion plasma frequency $\OmgP$ and discuss whether the magnitude of your result seems appropriate for a typical ionic solid.
}



\prob{
	Explain how the extrapolated vanishing of the optical phonon frequency at low temperatures, marked by the arrow in the figure, is indicative of a phase transition.
}



\prob{
	Construct a phenomenological Landau theory for the free energy of the solid as a power series expansion in the polarization, consistent with cubic symmetry.
}



\prob{
	Use your model to predict how the phonon frequency $\Omgo$ and static dielectric constant $\epsion$ would vary with temperature below the critical temperature.
}

%\makebib

\end{document}
