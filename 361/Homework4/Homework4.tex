\documentclass[11pt]{article}
\usepackage{homework}

\classname{361}
\homeworknum{4}



\begin{document}

% Environments

\newcommand{\state}[2]{\begin{statement}{#1} #2 \end{statement}}
\newcommand{\prob}[2]{\begin{problem}{#1} #2 \end{problem}}
\newcommand{\subprob}[1]{\begin{subproblem} #1 \end{subproblem}}
\newcommand{\sol}[1]{\begin{solution} #1 \end{solution}}
\newcommand{\fig}[2]{\begin{figure} \centering #2  \label{#1} \end{figure}}

\newcommand{\makebib}{
	\vfill
	\color{black}
	\bibliography{references}{}
	\bibliographystyle{lucas_unsrt}
}
	

% Implication

\newcommand{\qwhere}{\quad \text{where} \quad}
\newcommand{\qimplies}{\quad \implies \quad}
\newcommand{\impliesq}{\implies \quad}



% Brackets

\newcommand{\paren}[1]{\left( #1 \right)}
\newcommand{\brac}[1]{\left[ #1 \right]}


% Greek

\newcommand{\alp}{\alpha}
\newcommand{\bet}{\beta}
\newcommand{\gam}{\gamma}
\newcommand{\del}{\delta}
\newcommand{\eps}{\epsilon}
\newcommand{\zet}{\zeta}
\newcommand{\tht}{\theta}
\newcommand{\kap}{\kappa}
\newcommand{\lam}{\lambda}
\newcommand{\sig}{\sigma}
\newcommand{\ups}{\upsilon}
\newcommand{\omg}{\omega}

\newcommand{\Gam}{\Gamma}
\newcommand{\Del}{\Delta}
\newcommand{\Tht}{\Theta}
\newcommand{\Lam}{\Lambda}
\newcommand{\Sig}{\Sigma}
\newcommand{\Omg}{\Omega}
% Problem 1

\newcommand{\Psii}{\Psi^i}
\newcommand{\Psiix}{\Psii(x)}

\newcommand{\Pii}{\Pi^i}

\newcommand{\Phii}{\Phi^i}
\newcommand{\Phiix}{\Phii(x)}
\newcommand{\PhiN}{\Phi^N}
\newcommand{\PhiNx}{\PhiN(x)}
\newcommand{\Phiq}{\Phi^1}
\newcommand{\Phiw}{\Phi^2}

\newcommand{\ddcx}{\dd[3]{x}}

\newcommand{\delij}{\del^{i j}}
\newcommand{\delkl}{\del^{k l}}
\newcommand{\delil}{\del^{i l}}
\newcommand{\deljk}{\del^{j k}}
\newcommand{\delik}{\del^{i k}}
\newcommand{\deljl}{\del^{j l}}

\newcommand{\DF}{D_F}

\newcommand{\sigx}{\sig(x)}

\newcommand{\pii}{\pi^i}
\newcommand{\pij}{\pi^j}
\newcommand{\pik}{\pi^k}
\newcommand{\pil}{\pi^l}
\newcommand{\piix}{\pi(x)}

\newcommand{\pq}{p_1}
\newcommand{\pw}{p_2}
\newcommand{\pe}{p_3}
\newcommand{\pr}{p_4}

\newcommand{\vp}{\vb{p}}
\newcommand{\vpsi}{\vp_i}

\newcommand{\mpi}{m_\pi}

\state{(Jackson 9.8)}{\ 
	%\emph{Hint:} The electromagnetic angular momentum density comes from more than the transverse (radiation zone) components of the fields.
}

%
%	Jackson 9.8(a)
%

\prob{}{
	Show that a classical oscillating electric dipole $\vp$ with fields given by
	\aln{ \label{fields1}
		\vH &= \frac{c k^2}{4\pi} (\nh \cross \vp) \frac{e^{i k r}}{r} \paren{ 1 - \frac{1}{i k r} }, &
		\vE &= \frac{1}{4\pi \epso} \curly{ k^2 (\nh \cross \vp) \cross \nh \frac{e^{i k r}}{r} + [ 3 \nh (\nh \vdot \vp) - \vp ] \paren{ \frac{1}{r^3} - \frac{i k}{r^2} } e^{i k r} },
	}
	radiates electromagnetic angular momentum to infinity at the rate
	\eq{
		\dv{\vL}{t} = \frac{k^3}{12 \pi \epso} \Im[ \vp^* \cross \vp ].
	}
	\vfix
}

\sol{
	According to Jackson~(9.20), the time-averaged angular momentum density is
	\eq{
		\vl = \frac{\Re[ \vx \cross (\vE \cross \vHs)}{2 c^2}.
	}
	One of the vector identities on the inside cover of Jackson is $\vaa \cross (\vbb \cross \vcc) = (\vaa \vdot \vcc) \vbb - (\vaa \vdot \vbb) \vcc$, so
	\eqn{l1}{
		\vl = \frac{(\vx \vdot \vHs) \vE - (\vx \vdot \vE) \vHs}{2 c^2}.
	}
	From Eq.~\refeq{fields1}, note that
	\eq{
		\vx \vdot \vHs \propto \vx \vdot (\nh \cross \vps)
		= \vps \vdot (\vx \cross \nh)
		= \vO,
	}
	where we have used the identity $\vaa \vdot (\vbb \cross \vcc) = \vcc \vdot (\vaa \cross \vbb)$ and the fact that $\nh$ points in the $\vx$ direction.  For $\vx \vdot \vE$, note that
	\al{
		\vx \vdot [ (\nh \cross \vp) \cross \nh ] &= -\vx \vdot [ \nh \cross (\nh \cross \vp) ]
		= -\vx \vdot [ (\nh \vdot \vp) \nh - (\nh \vdot \nh) \vp ]
		= -(\nh \vdot \vp) (\vx \vdot \nh) + \vx \vdot \vp \\
		&= -r (\nh \vdot \vp) + \vx \vdot \vp
		= \vx \vdot \vp - \vx \vdot \vp
		= 0, \\[1.5ex]
		\vx \vdot [ 3 \nh (\nh \vdot \vp) - \vp ] &= 3 (\vx \vdot \nh) (\nh \vdot \vp) - \vx \vdot \vp
		= 3r (\nh \vdot \vp) - \vx \vdot \vp
		= 3(\vx \vdot \vp) - \vx \vdot \vp
		= 2(\vx \vdot \vp),
	}
	since $\abs{\vx} = r$ and $\vx = r \,\nh$.  Then
	\eq{
		\vx \vdot \vE = \frac{1}{2\pi \epso} (\vx \vdot \vp) \paren{ \frac{1}{r^3} - \frac{i k}{r^2} } e^{i k r}
		= \frac{1}{2\pi \epso} (\nh \vdot \vp) \paren{ \frac{1}{r^2} - \frac{i k}{r} } e^{i k r}.
	}
	
	With these substitutions, Eq.~\refeq{l1} becomes
	\al{
		\vl &= -\frac{(\vx \vdot \vE) \vHs}{c^2}
		= -\frac{1}{4\pi \epso c^2} (\nh \vdot \vp) \paren{ \frac{1}{r^2} - \frac{i k}{r} } e^{i k r} \frac{c k^2}{4\pi} (\nh \cross \vps) \frac{e^{-i k r}}{r} \paren{ 1 + \frac{1}{i k r} } \\
		&= -\frac{k^2}{16\pi^2 \epso c r} (\nh \vdot \vp) (\nh \cross \vps) \paren{ \frac{1}{r^2} - \frac{i k}{r} } \paren{ 1 - \frac{i}{k r} }
		= -\frac{k^2}{16\pi^2 \epso c} (\nh \vdot \vp) (\nh \cross \vps) \paren{ \frac{1}{r^2} - \frac{i}{k r^3} - \frac{i k}{r} - \frac{1}{r^2} } \\
		&= -\frac{i k^2}{16\pi^2 \epso c r} (\nh \vdot \vp) (\nh \cross \vps) \paren{ \frac{1}{k r^3} + \frac{k}{r^2} }
		= \frac{i k^3}{16\pi^2 \epso c r^2} (\nh \vdot \vp) (\nh \cross \vps) \paren{ \frac{1}{k^2 r^2} + 1 }.
	}
	
	Let $\vL$ be the angular momentum radiated to a distance $R$.  Then
	\eq{
		\vL = \int_R \vl(r) \ddcx
		= \intopi \intotp \intoR \vl(r) \,r^2 \sin\tht \ddr \ddphi \dd\tht,
	}
	and the time derivative is
	\aln{
		\dv{\vL}{t} &= \dv{t}(\intopi \intotp \intoR \vl(r) \,r^2 \sin\tht \ddr \ddphi \dd\tht)
		= \dv{r}{t} \dv{r}(\intopi \intotp \intoR \vl(r) \,r^2 \sin\tht \ddr \ddphi \dd\tht) \notag \\
		&= c \intopi \intotp \vl(r) \,r^2 \sin\tht \ddphi \dd\tht
		= \frac{i k^3}{16\pi^2 \epso} \paren{ \frac{1}{k^2 r^2} + 1 } \intopi \intotp (\nh \vdot \vp) (\nh \cross \vps) \sin\tht \ddphi \dd\tht. \label{dLdt}
	}
	Note that
	\eq{
		[ (\nh \vdot \vp) (\nh \cross \vps) ]_i = \sumje n_j p_j (\nh \cross \vps)_i
		= \sumje \sumke \sumle \epsikl n_j p_j n_k p_l^*,
	}
	so
	\eq{
		\dv{L_i}{t} \propto \sumje \sumke \sumle \epsikl p_j p_l^* \int n_j p_k \ddOmg
		= \sumje \sumke \sumle \epsikl p_j p_l^* \frac{4\pi}{3} \del_{jk}
		= \frac{4\pi}{3} \epsikl p_k p_l^*
		= \frac{4\pi}{3} (\vp \cross \vps)_i,
	}
	where we have used Jackson~(9.47), $\int n_\bet n_\gam \ddOmg = 4\pi \del_{\bet \gam} / 3$.  Making this substitution into Eq.~\refeq{dLdt},
	\eq{
		\dv{\vL}{t} = \frac{i k^3}{6\pi \epso} \paren{ \frac{1}{k^2 r^2} + 1 } (\vp \cross \vps).
	}
	Taking the limit as $r \to \infty$, we find
	\eqn{ans1a}{
		\dv{\vL}{t} = \Re\!\brac{ \frac{i k^3}{12\pi \epso} (\vp \cross \vps) }
		= \Re\!\brac{ -\frac{i k^3}{12\pi \epso} (\vps \cross \vp) }
		= \ans{ \frac{k^3}{12\pi \epso} \Im[ \vps \cross \vp ], }
	}
	as desired. \qed
}

%
%	Jackson 9.8(b)
%

\prob{}{
	What is the ratio of angular momentum radiated to energy radiated?  Interpret.
}

\sol{
	According to Jackson~(9.24), the total power radiated by an oscillating electric dipole $\vp$ is
	\eq{
		P = \dv{E}{t}
		= \frac{c^2 \Zo k^4}{12 \pi} \abs{\vp}^2.
	}
	Then the ratio of angular momentum radiated to energy radiated is
	\eq{
		\frac{\dv*{\vL}{t}}{\dv*{E}{t}} = \frac{k^3}{12\pi \epso} \Im[ \vps \cross \vp ] \frac{12 \pi}{c^2 \Zo k^4 \abs{\vp}^2}
		= \frac{1}{\epso} \Im[ \vps \cross \vp ] \frac{1}{c^2 \Zo k \abs{\vp}^2}
		= \ans{ \frac{\Im[ \vps \cross \vp ]}{\omg \abs{\vp}^2}, }
	}
	where we have used $\Zo = \sqrt{\muo / \epso} = 1 / \sqrt{\epso^2 c^2} = 1 / \epso c$, $c^2 = 1 / (\epso \muo)$, and $\omg = k c$.
	
	In the limit of high frequency, $(\dv*{\vL}{t}) / (\dv*{E}{t}) \to 0$.  In this scenario, the energy radiated dominates over the angular momentum radiated.  Likewise, in the limit of low frequency, $(\dv*{\vL}{t}) / (\dv*{E}{t}) \to \infty$, meaning that angular momentum radiation dominates.  This is sensible because rotational kinetic energy $E \propto \omg^2$, while angular momentum $L \propto \omg$.
}

%
%	Jackson 9.8(c)
%

\prob{}{
	For a charge $e$ rotating in the $xy$ plane at radius $a$ and angular speed $\omg$, show that there is only a $z$ component of radiated angular momentum with magnitude $\dv*{\Lz}{t} = e^2 k^3 a^2 / 6 \pi \epso$.  What about a charge oscillating along the $z$ axis?
}

\sol{
	We know from Homework~5 that the position of a point charge rotating counterclockwise in the $xy$ plane is
	\eq{
		\vx(t) = a \cos(\omg t) \,\vx + a \sin(\omg t) \,\yh.
	}
	\clearpage
	Then the charge distribution is
	\eq{
		\rho(\vx, t) = e \del[ x - a \cos(\omg t) ] \,\del[ y - a \sin(\omg t) ] \,\del(z).
	}
	
	According to Jackson~(4.8), the dipole moment is defined
	\eq{
		\vp = \int \vx' \,\rho(\vx') \ddcxp.
	}
	The components of $\vp$ for the point charge are then
	\al{
		\px &= e \iiint x \,\del[ x - a \cos(\omg t) ] \,\del[ y - a \sin(\omg t) ] \,\del(z) \ddx \ddy \ddz
		= e a \cos(\omg t), \\
		\py &= e \iiint y \,\del[ x - a \cos(\omg t) ] \,\del[ y - a \sin(\omg t) ] \,\del(z) \ddx \ddy \ddz
		= e a \sin(\omg t), \\
		\pz &= e \iiint z \,\del[ x - a \cos(\omg t) ] \,\del[ y - a \sin(\omg t) ] \,\del(z) \ddx \ddy \ddz
		= 0,
	}
	so we can write $\vp = e a \,e^{-i \omg t} (\xh + i\,\yh).$  Substituting into Eq.~\refeq{ans1a},
	\al{
		\dv{\vL}{t} &= \Re\!\brac{ \frac{i k^3}{12\pi \epso} e^2 a^2 e^{-i \omg t} e^{i \omg t} [ (\xh + i\,\yh) \cross (\xh - i\,\yh) ] }
		= \Re\!\brac{ \frac{i e^2 k^3 a^2}{12\pi \epso} (-2i \,\xh \cross \yh) }
		= \Re\!\brac{ \frac{e^2 k^3 a^2}{6\pi \epso} \,\zh } \\
		&= \ans{ \frac{e^2 k^3 a^2}{6\pi \epso} \cos(\omg t) \,\zh, }
	}
	as desired. \qed
	
	A charge oscillating along the $z$ axis with amplitude $a$ has the charge density
	\eq{
		\rho(\vx, t) = e a \,\del(x) \,\del(y) \,\del[ z - \cos(\omg t) ],
	}
	which gives the dipole moment
	\al{
		\px &= e a \iiint x \,\del(x) \,\del(y) \,\del[ z - \cos(\omg t) ] \ddx \ddy \ddz
		= 0, \\
		\py &= e a \iiint y \,\del(x) \,\del(y) \,\del[ z - \cos(\omg t) ] \ddx \ddy \ddz
		= 0, \\
		\pz &= e a \iiint z \,\del(x) \,\del(y) \,\del[ z - \cos(\omg t) ] \ddx \ddy \ddz
		= e a \cos(\omg t).
	}
	In complex notation, $\vp = e a \,e^{-i\omg t} \,\zh$.  Substituting into Eq.~\refeq{ans1a}, we find
	\eq{
		\dv{\vL}{t} = \Re\!\brac{ \frac{i k^3}{12\pi \epso} e^2 a^2 e^{-i \omg t} e^{i \omg t} (\zh \cross \zh) }
		= \ans{ \vO. }
	}
	So we see that a charge undergoing linear motion does not lead to a radiated angular momentum, which is sensible.
}

%
%	Jackson 9.8(d)
%

\prob{}{
	What are the results corresponding to Probs.~{1(a)} and {1(b)} for magnetic dipole radiation?
}

\sol{
	The radiation fields for a magnetic dipole are given by Jackson~(19.35--36),
	\al{
		\vH &= \frac{1}{4\pi} \curly{ k^2 (\nh \cross \vm) \cross \nh \frac{e^{i k r}}{r} + [ 3 \nh (\nh \vdot \vm) - \vm ] \paren{ \frac{1}{r^3} - \frac{i k}{r^2} } e^{i k r} }, &
		\vE &= -\frac{\Zo}{4\pi} k^2 (\nh \cross \vm) \frac{e^{i k r}}{r} \paren{ 1 - \frac{1}{i k r} }.
	}
	\clearpage
	Comparing with Eq.~\refeq{fields1}, we see that $\vH \to -\vE / \Zo$, $\vE \to \Zo \vH$, and $\vp \to \vm / c$ as stated in the book~\cite[p.~413]{Jackson}.  Making these substitutions, the results of Probs.~{1.1(a)} and {(b)} become
	\al{
		\ans{ \dv{\vL}{t}\ }&\ans{= \frac{\muo k^3}{12\pi} \Im[ \vms \cross \vm ], } &
		\ans{ \frac{\dv*{\vL}{t}}{\dv*{E}{t}}\ }&\ans{= \frac{\Im[ \vms \cross \vm ]}{\omg \abs{\vm}^2} }
	}
	where we have used $\mu = 1 / \epso c^2$.
}






\state{Landau theory of phase transitions}{
	A ferroelectric crystal is one that supports a macroscopic polarization $P$, which usually arises because the underlying crystal structure does not have inversion symmetry.  However, as temperature or pressure is changed, the crystal may recover the inversion symmetry.  This can be modeled by Landau's theory of second order phase transitions, where we postulate a form for the free energy density (per unit volume)
	\eqn{given2}{
		\cF = \frac{a}{2} P^2 + \frac{b}{4} P^4 + \frac{c}{6} P^6 + \cdots,
	}
	where the coefficient $a = \ao (T - \Tc)$ is temperature dependent and all the other coefficients are constant.  Although the polarization $P$ is of course a vector, we assume that it can point only in a symmetry direction of the crystal, and so it is replaced by a scalar.
}

\prob{
	Assume that $b > 0$ and $c = 0$.  Use Eq.~\refeq{given2} to determine the form for the equilibrium $\PT$.
}

\sol{
	When $b > 0$ and $c = 0$, Eq.~\refeq{given2} becomes
	\eq{
		\cF = \frac{a}{2} P^2 + \frac{b}{4} P^4.
	}
	The equilibrium $\PT$ occurs at the minima of $\cF$, where $\dv*{\cF}{P} = 0$~\cite{Ferroelectricity}:
	\eq{
		\dv{\cF}{P} = a P + b P^3 = 0.
	}
	This implies
	\al{
		P &= 0, &
		P &= \pm \sqrt{-\frac{a}{b}}.
	}
	Note, however, that $P = 0$ is a local maximum of $\cF$:
	\eq{
		\left. \dv[2]{\cF}{P} \right|_{P = 0} = \brac{ a + 2 b P^2 }_{P = 0}
		= \ao (T - \Tc)
		< 0 \quad \text{when } T < \Tc,
	}
	which is the regime we are interested in for a ferroelectric~\cite[p.~556]{Ashcroft}\cite{Ferroelectricity}.  Thus the equiliibrium $\PT$ is given by
	\eq{
		\ans{ \PT = \pm \sqrt{\frac{\ao}{b} (\Tc - T)}. }
	}
	\vfix
}



\prob{ \label{2c}
	Including in $\cF$ the energy of the polarization coupled to an external electric field $E$, determine the dielectric susceptibility $\chi = \dv*{P}{E}$ both above and below the critical temperature.
}



\prob{
	Sketch curves for $\PT$, $\chiinvT$, and $\chiT$.
}



\prob{
	In a different material, the free energy is described by a similar form to Eq.~\refeq{given2}, but with $b < 0$ and $c > 0$.  By sketching $\cF$ at different temperatures, discuss the behavior of the equilibrium polarization and the linear susceptibility, contrasting the results with those found in \ref{2c}.
}





\clearpage
\state{Reflectivity of metals}{
	The phase velocity of light in a conducting medium is the speed of light divided by the complex dielectric constant $\Nomg = \sqrt{\epsomg}$ where we may use for $\eps$ the Drude result
	\eqn{eps}{
		\epsomg = 1 - \frac{\omgp^2}{\omg^2 + i \omg / \tau}.
	}
	In a good Drude metal, we have $1 / \tau \ll \omgp$.
}

\prob{
	Sketch curves of
	\begin{enumerate}
		\item $\Re[\sigomg]$,
		\item $\Re[\epsomg]$,
		\item $\Im[1 / \epsomg]$.
	\end{enumerate}
}

\sol{
	The conductivity is defined in (5.25) of the lecture notes:
	\eq{
		\sigomg = \frac{\omgp^2}{4\pi (1 / \tau - i \omg)}.
	}
	Thus
	\eqn{3ai}{
		\Re[\sigomg] = \frac{\omgp^2}{4\pi \tau} \frac{1}{1 / \tau^2 + \omg^2}.
	}
	Note also that
	\eqn{3aii}{
		\Re[\epsomg] = 1 - \frac{\omgp^2}{1 / \tau^2 + \omg^2}.
	}
	and that
	\eqn{3ax}{
		\Im[\epsomg] = -\frac{\omgp^2}{\tau \omg} \frac{1}{1 / \tau^2 + \omg^2},
	}
	so
	\eqn{3aiii}{
		\Im[1 / \epsomg] = -\frac{\tau \omg}{\omgp^2} \paren{ \frac{1}{\tau^2} + \omg^2 }.
	}
	Figure~\ref{3a} shows plots of $\Re[\sigomg]$~(blue), $\Re[\epsomg]$~(gold), and $\Im[1 / \epsomg]$~(green).
	
	\begin{figure}[t] \centering
		\includegraphics[width=0.6\textwidth,trim=1.5cm 0 0 0,clip]{3a}
		\caption{Plots of $\Re[\sigomg]$~(blue), $\Re[\epsomg]$~(gold), and $\Im[1 / \epsomg]$~(green).  These expressions are given by Eqs.~\refeq{3ai}, \refeq{3aii}, and \refeq{3aiii}, respectively.}
		\label{3a}
	\end{figure}
}



\prob{
	Consider a light wave with the electric field polarized in the $x$ direction at normal incidence from the vacuum on a good Drude metal occupying the region $z > 0$.  In the vacuum ($z < 0$), the incident $\Eq$ and reflected $\Ew$ waves give rise to a field
	\eq{
		\Ex = \Eq e^{i \omg (z / c - t)} + \Ew e^{-i \omg (z / c + t)},
	}
	whereas in the medium, the electric field is
	\eq{
		\Ex = \Eo e^{i \omg [ \Nomg z / c - t ]}.
	}
	Matching the electric and magnetic fields on the boundary, show that
	\al{
		\Eo &= \Eq + \Ew, &
		N \Eo &= \Eq - \Ew,
	}
	and hence show that the reflection coefficient satisfies
	\eqn{R}{
		R = \abs{\frac{\Ew}{\Eq}}^2
		= \abs{\frac{1 - N}{1 + N}}^2.
	}
}

\sol{
	By (5.15) of the lecture notes, the boundary condition is
	\eq{
		\epspar \Epar - \Dpar,
	}
	where $\epspar$ is given by Eq.~\refeq{eps}.  In this problem the surface of the metal is the $xy$ plane.  We require
	\eq{
		\Eq e^{i \omg (z / c - t)} + \Ew e^{-i \omg (z / c + t)} = \epsomg \Eo e^{i \omg [ \Nomg z / c - t ]}.
	}
	First making the ansatz $\Eo = \Eq + \Ew$,
	\al{
		\Eq e^{i \omg (z / c - t)} + \Ew e^{-i \omg (z / c + t)} &= \epsomg (\Eq + \Ew) e^{i \omg [ \Nomg z / c - t ]} \\
		\Eq [ e^{i \omg (z / c - t)} - \epsomg e^{i \omg [ \Nomg z / c - t ]} ] &= -\Ew [ e^{-i \omg (z / c + t)} - \epsomg e^{i \omg [ \Nomg z / c - t ]} ] \\
		\Eq [ e^{-2 i \omg z / c} - \epsomg e^{i \omg z [ \Nomg + 1 ] / c} ] &= -\Ew [ 1  - \epsomg e^{i \omg z [ \Nomg + 1 ] / c} ] \\
	}
	
	\hl{??????}
}



\prob{
	Using the Drude formula above, show that
	\eq{
		R \approx \begin{cases}
			1 - 2 \sqrt{\dfrac{\omg}{2\pi \sigo}} & \omg \ll 1 / \tau, \\[3ex]
			1 - \dfrac{2}{\omgp \tau} & 1 / \tau \ll \omg \ll \omgp, \\[3ex]
			0 & \omgp \ll \omg,
		\end{cases}
	}
	and sketch the reflectivity $\Romg$.
}

\sol{
	From Eq.~\refeq{R}, we can write
	\al{
		R &= \abs{\frac{1 - N}{1 + N}}^2 \\
		&= \frac{(1 - \Re[N])^2 + \Im[N]^2}{(1 + \Re[N])^2 + \Im[N]^2} \\
		&= \frac{(1 - \sqrt{\Re[\epsomg]})^2 + \sqrt{\Im[\epsomg]}^2}{(1 + \sqrt{\Re[\epsomg]})^2 + \sqrt{\Im[\epsomg]}^2} \\
		&= \frac{1 - 2 \sqrt{\Re[\epsomg]} + \Re[\epsomg] + \Im[\epsomg]}{1 + 2 \sqrt{\Re[\epsomg]} + \Re[\epsomg] + \Im[\epsomg]},
	}
	where we have used Ashcroft \& Mermin~(K.6).  Feeding in Eqs.~\refeq{3aii} and \refeq{3ax},
	\eq{
		R = \paren{ 2 - 2 \sqrt{1 - \frac{\omgp^2}{1 / \tau^2 + \omg^2}} - \omgp^2 \frac{1 + 1 / \tau \omg}{1 / \tau^2 + \omg^2} } \paren{ 2 + 2 \sqrt{1 - \frac{\omgp^2}{1 / \tau^2 + \omg^2}} - \omgp^2 \frac{1 + 1 / \tau \omg}{1 / \tau^2 + \omg^2} }^{-1}.
	}
	When $\omg \ll 1 / \tau$, $\tau \omg \ll 1$.  Then
	\al{
		R &= \paren{ 2 - 2 \sqrt{1 - \frac{\omgp^2 / \omg^2}{1 / \tau^2 \omg^2 + 1}} - \frac{\omgp^2}{\omg} \frac{1 / \omg + 1 / \tau \omg}{1 / \tau^2 \omg^2 + 1} } \paren{ 2 + 2 \sqrt{1 - \frac{\omgp^2 / \omg^2}{1 / \tau^2 \omg^2 + 1}} - \frac{\omgp^2}{\omg} \frac{1 / \omg + 1 / \tau \omg}{1 / \tau^2 \omg^2 + 1} }^{-1} \\
		&\approx \frac{ 2 - 2 \sqrt{1 - (\tau \omg)^2 \omgp^2 / \omg^2} - (\tau \omg)^2 \omgp^2 (1 + 1 / \tau) / \omg^2 }{ 2 + 2 \sqrt{1 - (\tau \omg)^2 \omgp^2 / \omg^2} - (\tau \omg)^2 \omgp^2 (1 + 1 / \tau) / \omg^2 } \\
		&\approx \frac{ [ \omgp^2 / \omg^2 - \omgp^2 (1 + 1 / \tau) / \omg^2 ] (\tau \omg)^2 }{ 4 - [ \omgp^2 / \omg^2 + \omgp^2 (1 + 1 / \tau) / \omg^2 ] (\tau \omg)^2 } \\
		&= \frac{ (\omgp^2 / \tau \omg^2) (\tau \omg)^2 }{ 4 - [ \omgp^2 / \omg^2 + \omgp^2 (1 + 1 / \tau) / \omg^2 ] (\tau \omg)^2 }
	}
	\hl{???}
	From (5.27) in the lecture notes,
	\eq{
		\sigo = \frac{\omgp^2 \tau}{4\pi}.
	}
	\eq{
		1 - 2 \sqrt{\frac{\omg}{2 \pi \sigo}} = 1 - 2 \sqrt{\frac{2 \omg}{\omgp^2 \tau}}
	}
	
	When $\omg \ll \omgp$ and $1 / \tau \ll \omg$, $\omgp / \omg \gg 1$ and $\tau \omg \gg 1$:
	\al{
		R &= \paren{ 2 - 2 \sqrt{1 - \frac{\omgp^2 / \omg^2}{1 / \tau^2 \omg^2 + 1}} - \frac{\omgp^2}{\omg^2} \frac{1 + 1 / \tau}{1 / \tau^2 \omg^2 + 1} } \paren{ 2 + 2 \sqrt{1 - \frac{\omgp^2 / \omg^2}{1 / \tau^2 \omg^2 + 1}} - \frac{\omgp^2}{\omg^2} \frac{1 + 1 / \tau}{1 / \tau^2 \omg^2 + 1} }^{-1} \\
		&\approx \frac{ 2 - 2 \omgp/ \omg - \omgp^2/ \omg^2 }{ 2 + 2 \omgp/ \omg - \omgp^2/ \omg^2 }
	}
	\hl{idek}
	
	When $\omgp \ll \omg$, $\omgp / \omg \ll 1$:
	\al{
		R &= \paren{ 2 - 2 \sqrt{1 - \frac{\omgp^2 / \omg^2}{1 / \tau^2 \omg^2 + 1}} - \frac{\omgp^2}{\omg^2} \frac{\omg^2 + \omg / \tau}{1 / \tau^2 + \omg^2} } \paren{ 2 + 2 \sqrt{1 - \frac{\omgp^2 / \omg^2}{1 / \tau^2 \omg^2 + 1}} - \frac{\omgp^2}{\omg^2} \frac{\omg^2 + \omg / \tau}{1 / \tau^2 + \omg^2} }^{-1} \\
		&\approx \frac{2 - 2 \sqrt{1}}{2 + 2 \sqrt{1}} \\
		&= \ans{ 0 },
	}
	as desired. \qed
}





\clearpage
\state{Phonons}{
	From Eq.~(5.8) construct $\Im[\chi]$ in the limit that $\gam \to 0$.  Use the Kramers--{\Kronig} relation to then reconstruct $\Re[\chi]$ from $\Im[\chi]$ in the same limit.
}

\sol{
	Equation~(5.8) in the lecture notes is
	\eq{
		\chivqomg = \frac{1}{-\rho \omg^2 + i \gam \omg + K q^2}.
	}
	Then
	\eq{
		\Im[\chi] = -\frac{\gam \omg}{(K q^2 - \rho \omg)^2 + \gam^2 \omg^2}.
	}
	In the limit $\gam \to 0$,
	\eq{
		\Im[\chi] = -\frac{\gam \omg}{(K q^2 - \rho \omg^2)^2}.
	}
	The relevant Kramers--{\Kronig} relation is given by (5.39),
	\eq{
		\Re[\kapomg] = \PrV \int \ddomgpf \frac{\Im[\kapomgp]}{\omg' - \omg}.
	}
	Then
	\eq{
		\Re[\kapomg] = -\PrV \int \ddomgpf \frac{1}{\omg' - \omg} \frac{\gam \omg'}{(K q^2 - \rho {\omg'}^2)^2}
		= 
	}
	\hl{partial fraction decomposition???}
}





%\clearpage
%\state{Screened Coulomb interaction}{
%	Consider a nucleus of charge $Z$ producing a potential
%	\eq{
%		\Vextvq = -\frac{4\pi Z e^2}{q^2}.
%	}
%	Using the long-wavelength limit of the dielectric function, show that the screened potential satisfies
%	\eq{
%		\Vscrvqo = -\frac{2}{3} \Omg \EF,
%	}
%	where $\Omg$ is the volume of the unit cell and $\EF$ is the Fermi energy for $Z$ free electrons per unit cell.
%}






%\state{Peierls transition}{
%	By rewriting the term containing $\nkpq$ (replace $\vk + \vq \to -\vk'$ and then relabel $\vk'$ as $\vk$), show that the static density response function can be written
%	\eq{
%		\chiovqo = 2 \sumklkF \frac{1}{\epskpq - \epsk}.
%	}
%	In one dimension, make a linear approximation to the electronic dispersion near $\kF$, i.e.~$\epsk = \vF \abs{k}$, and consider the response for $q = 2 \kF + p$, where $p \ll 2 \kF$.  By considering terms in the sum over $k$ near $k \approx -\kF$, show that
%	\eq{
%		\chio(2 \kF + p) \approx \frac{1}{2\pi \vF} \ln\abs{ \frac{\kF}{p} }.
%	}
%	Explain why this result suggests that a one-dimensional metal will be unstable to a lattice distortion with wavevector $2 \kF$.
%}






%\state{Optical properties}{
%	Discuss why, at optical frequencies, glass is transparent and silver is shiny, while graphite appears black and powdered sugar is white.
%}





%\state{Metals and insulators}{
%	Explain the differences between a metal and an insulator.  Your discussion should include reference to single particle properties, screening of the Coulomb interaction, optical properties, and electrical and thermal properties.
%}


\makebib

\end{document}
