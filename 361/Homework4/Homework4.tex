\documentclass[11pt]{article}
\usepackage{homework}

\classname{361}
\homeworknum{4}



\begin{document}

% Environments

\newcommand{\state}[2]{\begin{statement}{#1} #2 \end{statement}}
\newcommand{\prob}[2]{\begin{problem}{#1} #2 \end{problem}}
\newcommand{\subprob}[1]{\begin{subproblem} #1 \end{subproblem}}
\newcommand{\sol}[1]{\begin{solution} #1 \end{solution}}
\newcommand{\fig}[2]{\begin{figure} \centering #2  \label{#1} \end{figure}}

\newcommand{\makebib}{
	\vfill
	\color{black}
	\bibliography{references}{}
	\bibliographystyle{lucas_unsrt}
}
	

% Implication

\newcommand{\qwhere}{\quad \text{where} \quad}
\newcommand{\qimplies}{\quad \implies \quad}
\newcommand{\impliesq}{\implies \quad}



% Brackets

\newcommand{\paren}[1]{\left( #1 \right)}
\newcommand{\brac}[1]{\left[ #1 \right]}


% Greek

\newcommand{\alp}{\alpha}
\newcommand{\bet}{\beta}
\newcommand{\gam}{\gamma}
\newcommand{\del}{\delta}
\newcommand{\eps}{\epsilon}
\newcommand{\zet}{\zeta}
\newcommand{\tht}{\theta}
\newcommand{\kap}{\kappa}
\newcommand{\lam}{\lambda}
\newcommand{\sig}{\sigma}
\newcommand{\ups}{\upsilon}
\newcommand{\omg}{\omega}

\newcommand{\Gam}{\Gamma}
\newcommand{\Del}{\Delta}
\newcommand{\Tht}{\Theta}
\newcommand{\Lam}{\Lambda}
\newcommand{\Sig}{\Sigma}
\newcommand{\Omg}{\Omega}
% Problem 1

\newcommand{\Psii}{\Psi^i}
\newcommand{\Psiix}{\Psii(x)}

\newcommand{\Pii}{\Pi^i}

\newcommand{\Phii}{\Phi^i}
\newcommand{\Phiix}{\Phii(x)}
\newcommand{\PhiN}{\Phi^N}
\newcommand{\PhiNx}{\PhiN(x)}
\newcommand{\Phiq}{\Phi^1}
\newcommand{\Phiw}{\Phi^2}

\newcommand{\ddcx}{\dd[3]{x}}

\newcommand{\delij}{\del^{i j}}
\newcommand{\delkl}{\del^{k l}}
\newcommand{\delil}{\del^{i l}}
\newcommand{\deljk}{\del^{j k}}
\newcommand{\delik}{\del^{i k}}
\newcommand{\deljl}{\del^{j l}}

\newcommand{\DF}{D_F}

\newcommand{\sigx}{\sig(x)}

\newcommand{\pii}{\pi^i}
\newcommand{\pij}{\pi^j}
\newcommand{\pik}{\pi^k}
\newcommand{\pil}{\pi^l}
\newcommand{\piix}{\pi(x)}

\newcommand{\pq}{p_1}
\newcommand{\pw}{p_2}
\newcommand{\pe}{p_3}
\newcommand{\pr}{p_4}

\newcommand{\vp}{\vb{p}}
\newcommand{\vpsi}{\vp_i}

\newcommand{\mpi}{m_\pi}

\state{(Jackson 9.8)}{\ 
	%\emph{Hint:} The electromagnetic angular momentum density comes from more than the transverse (radiation zone) components of the fields.
}

%
%	Jackson 9.8(a)
%

\prob{}{
	Show that a classical oscillating electric dipole $\vp$ with fields given by
	\aln{ \label{fields1}
		\vH &= \frac{c k^2}{4\pi} (\nh \cross \vp) \frac{e^{i k r}}{r} \paren{ 1 - \frac{1}{i k r} }, &
		\vE &= \frac{1}{4\pi \epso} \curly{ k^2 (\nh \cross \vp) \cross \nh \frac{e^{i k r}}{r} + [ 3 \nh (\nh \vdot \vp) - \vp ] \paren{ \frac{1}{r^3} - \frac{i k}{r^2} } e^{i k r} },
	}
	radiates electromagnetic angular momentum to infinity at the rate
	\eq{
		\dv{\vL}{t} = \frac{k^3}{12 \pi \epso} \Im[ \vp^* \cross \vp ].
	}
	\vfix
}

\sol{
	According to Jackson~(9.20), the time-averaged angular momentum density is
	\eq{
		\vl = \frac{\Re[ \vx \cross (\vE \cross \vHs)}{2 c^2}.
	}
	One of the vector identities on the inside cover of Jackson is $\vaa \cross (\vbb \cross \vcc) = (\vaa \vdot \vcc) \vbb - (\vaa \vdot \vbb) \vcc$, so
	\eqn{l1}{
		\vl = \frac{(\vx \vdot \vHs) \vE - (\vx \vdot \vE) \vHs}{2 c^2}.
	}
	From Eq.~\refeq{fields1}, note that
	\eq{
		\vx \vdot \vHs \propto \vx \vdot (\nh \cross \vps)
		= \vps \vdot (\vx \cross \nh)
		= \vO,
	}
	where we have used the identity $\vaa \vdot (\vbb \cross \vcc) = \vcc \vdot (\vaa \cross \vbb)$ and the fact that $\nh$ points in the $\vx$ direction.  For $\vx \vdot \vE$, note that
	\al{
		\vx \vdot [ (\nh \cross \vp) \cross \nh ] &= -\vx \vdot [ \nh \cross (\nh \cross \vp) ]
		= -\vx \vdot [ (\nh \vdot \vp) \nh - (\nh \vdot \nh) \vp ]
		= -(\nh \vdot \vp) (\vx \vdot \nh) + \vx \vdot \vp \\
		&= -r (\nh \vdot \vp) + \vx \vdot \vp
		= \vx \vdot \vp - \vx \vdot \vp
		= 0, \\[1.5ex]
		\vx \vdot [ 3 \nh (\nh \vdot \vp) - \vp ] &= 3 (\vx \vdot \nh) (\nh \vdot \vp) - \vx \vdot \vp
		= 3r (\nh \vdot \vp) - \vx \vdot \vp
		= 3(\vx \vdot \vp) - \vx \vdot \vp
		= 2(\vx \vdot \vp),
	}
	since $\abs{\vx} = r$ and $\vx = r \,\nh$.  Then
	\eq{
		\vx \vdot \vE = \frac{1}{2\pi \epso} (\vx \vdot \vp) \paren{ \frac{1}{r^3} - \frac{i k}{r^2} } e^{i k r}
		= \frac{1}{2\pi \epso} (\nh \vdot \vp) \paren{ \frac{1}{r^2} - \frac{i k}{r} } e^{i k r}.
	}
	
	With these substitutions, Eq.~\refeq{l1} becomes
	\al{
		\vl &= -\frac{(\vx \vdot \vE) \vHs}{c^2}
		= -\frac{1}{4\pi \epso c^2} (\nh \vdot \vp) \paren{ \frac{1}{r^2} - \frac{i k}{r} } e^{i k r} \frac{c k^2}{4\pi} (\nh \cross \vps) \frac{e^{-i k r}}{r} \paren{ 1 + \frac{1}{i k r} } \\
		&= -\frac{k^2}{16\pi^2 \epso c r} (\nh \vdot \vp) (\nh \cross \vps) \paren{ \frac{1}{r^2} - \frac{i k}{r} } \paren{ 1 - \frac{i}{k r} }
		= -\frac{k^2}{16\pi^2 \epso c} (\nh \vdot \vp) (\nh \cross \vps) \paren{ \frac{1}{r^2} - \frac{i}{k r^3} - \frac{i k}{r} - \frac{1}{r^2} } \\
		&= -\frac{i k^2}{16\pi^2 \epso c r} (\nh \vdot \vp) (\nh \cross \vps) \paren{ \frac{1}{k r^3} + \frac{k}{r^2} }
		= \frac{i k^3}{16\pi^2 \epso c r^2} (\nh \vdot \vp) (\nh \cross \vps) \paren{ \frac{1}{k^2 r^2} + 1 }.
	}
	
	Let $\vL$ be the angular momentum radiated to a distance $R$.  Then
	\eq{
		\vL = \int_R \vl(r) \ddcx
		= \intopi \intotp \intoR \vl(r) \,r^2 \sin\tht \ddr \ddphi \dd\tht,
	}
	and the time derivative is
	\aln{
		\dv{\vL}{t} &= \dv{t}(\intopi \intotp \intoR \vl(r) \,r^2 \sin\tht \ddr \ddphi \dd\tht)
		= \dv{r}{t} \dv{r}(\intopi \intotp \intoR \vl(r) \,r^2 \sin\tht \ddr \ddphi \dd\tht) \notag \\
		&= c \intopi \intotp \vl(r) \,r^2 \sin\tht \ddphi \dd\tht
		= \frac{i k^3}{16\pi^2 \epso} \paren{ \frac{1}{k^2 r^2} + 1 } \intopi \intotp (\nh \vdot \vp) (\nh \cross \vps) \sin\tht \ddphi \dd\tht. \label{dLdt}
	}
	Note that
	\eq{
		[ (\nh \vdot \vp) (\nh \cross \vps) ]_i = \sumje n_j p_j (\nh \cross \vps)_i
		= \sumje \sumke \sumle \epsikl n_j p_j n_k p_l^*,
	}
	so
	\eq{
		\dv{L_i}{t} \propto \sumje \sumke \sumle \epsikl p_j p_l^* \int n_j p_k \ddOmg
		= \sumje \sumke \sumle \epsikl p_j p_l^* \frac{4\pi}{3} \del_{jk}
		= \frac{4\pi}{3} \epsikl p_k p_l^*
		= \frac{4\pi}{3} (\vp \cross \vps)_i,
	}
	where we have used Jackson~(9.47), $\int n_\bet n_\gam \ddOmg = 4\pi \del_{\bet \gam} / 3$.  Making this substitution into Eq.~\refeq{dLdt},
	\eq{
		\dv{\vL}{t} = \frac{i k^3}{6\pi \epso} \paren{ \frac{1}{k^2 r^2} + 1 } (\vp \cross \vps).
	}
	Taking the limit as $r \to \infty$, we find
	\eqn{ans1a}{
		\dv{\vL}{t} = \Re\!\brac{ \frac{i k^3}{12\pi \epso} (\vp \cross \vps) }
		= \Re\!\brac{ -\frac{i k^3}{12\pi \epso} (\vps \cross \vp) }
		= \ans{ \frac{k^3}{12\pi \epso} \Im[ \vps \cross \vp ], }
	}
	as desired. \qed
}

%
%	Jackson 9.8(b)
%

\prob{}{
	What is the ratio of angular momentum radiated to energy radiated?  Interpret.
}

\sol{
	According to Jackson~(9.24), the total power radiated by an oscillating electric dipole $\vp$ is
	\eq{
		P = \dv{E}{t}
		= \frac{c^2 \Zo k^4}{12 \pi} \abs{\vp}^2.
	}
	Then the ratio of angular momentum radiated to energy radiated is
	\eq{
		\frac{\dv*{\vL}{t}}{\dv*{E}{t}} = \frac{k^3}{12\pi \epso} \Im[ \vps \cross \vp ] \frac{12 \pi}{c^2 \Zo k^4 \abs{\vp}^2}
		= \frac{1}{\epso} \Im[ \vps \cross \vp ] \frac{1}{c^2 \Zo k \abs{\vp}^2}
		= \ans{ \frac{\Im[ \vps \cross \vp ]}{\omg \abs{\vp}^2}, }
	}
	where we have used $\Zo = \sqrt{\muo / \epso} = 1 / \sqrt{\epso^2 c^2} = 1 / \epso c$, $c^2 = 1 / (\epso \muo)$, and $\omg = k c$.
	
	In the limit of high frequency, $(\dv*{\vL}{t}) / (\dv*{E}{t}) \to 0$.  In this scenario, the energy radiated dominates over the angular momentum radiated.  Likewise, in the limit of low frequency, $(\dv*{\vL}{t}) / (\dv*{E}{t}) \to \infty$, meaning that angular momentum radiation dominates.  This is sensible because rotational kinetic energy $E \propto \omg^2$, while angular momentum $L \propto \omg$.
}

%
%	Jackson 9.8(c)
%

\prob{}{
	For a charge $e$ rotating in the $xy$ plane at radius $a$ and angular speed $\omg$, show that there is only a $z$ component of radiated angular momentum with magnitude $\dv*{\Lz}{t} = e^2 k^3 a^2 / 6 \pi \epso$.  What about a charge oscillating along the $z$ axis?
}

\sol{
	We know from Homework~5 that the position of a point charge rotating counterclockwise in the $xy$ plane is
	\eq{
		\vx(t) = a \cos(\omg t) \,\vx + a \sin(\omg t) \,\yh.
	}
	\clearpage
	Then the charge distribution is
	\eq{
		\rho(\vx, t) = e \del[ x - a \cos(\omg t) ] \,\del[ y - a \sin(\omg t) ] \,\del(z).
	}
	
	According to Jackson~(4.8), the dipole moment is defined
	\eq{
		\vp = \int \vx' \,\rho(\vx') \ddcxp.
	}
	The components of $\vp$ for the point charge are then
	\al{
		\px &= e \iiint x \,\del[ x - a \cos(\omg t) ] \,\del[ y - a \sin(\omg t) ] \,\del(z) \ddx \ddy \ddz
		= e a \cos(\omg t), \\
		\py &= e \iiint y \,\del[ x - a \cos(\omg t) ] \,\del[ y - a \sin(\omg t) ] \,\del(z) \ddx \ddy \ddz
		= e a \sin(\omg t), \\
		\pz &= e \iiint z \,\del[ x - a \cos(\omg t) ] \,\del[ y - a \sin(\omg t) ] \,\del(z) \ddx \ddy \ddz
		= 0,
	}
	so we can write $\vp = e a \,e^{-i \omg t} (\xh + i\,\yh).$  Substituting into Eq.~\refeq{ans1a},
	\al{
		\dv{\vL}{t} &= \Re\!\brac{ \frac{i k^3}{12\pi \epso} e^2 a^2 e^{-i \omg t} e^{i \omg t} [ (\xh + i\,\yh) \cross (\xh - i\,\yh) ] }
		= \Re\!\brac{ \frac{i e^2 k^3 a^2}{12\pi \epso} (-2i \,\xh \cross \yh) }
		= \Re\!\brac{ \frac{e^2 k^3 a^2}{6\pi \epso} \,\zh } \\
		&= \ans{ \frac{e^2 k^3 a^2}{6\pi \epso} \cos(\omg t) \,\zh, }
	}
	as desired. \qed
	
	A charge oscillating along the $z$ axis with amplitude $a$ has the charge density
	\eq{
		\rho(\vx, t) = e a \,\del(x) \,\del(y) \,\del[ z - \cos(\omg t) ],
	}
	which gives the dipole moment
	\al{
		\px &= e a \iiint x \,\del(x) \,\del(y) \,\del[ z - \cos(\omg t) ] \ddx \ddy \ddz
		= 0, \\
		\py &= e a \iiint y \,\del(x) \,\del(y) \,\del[ z - \cos(\omg t) ] \ddx \ddy \ddz
		= 0, \\
		\pz &= e a \iiint z \,\del(x) \,\del(y) \,\del[ z - \cos(\omg t) ] \ddx \ddy \ddz
		= e a \cos(\omg t).
	}
	In complex notation, $\vp = e a \,e^{-i\omg t} \,\zh$.  Substituting into Eq.~\refeq{ans1a}, we find
	\eq{
		\dv{\vL}{t} = \Re\!\brac{ \frac{i k^3}{12\pi \epso} e^2 a^2 e^{-i \omg t} e^{i \omg t} (\zh \cross \zh) }
		= \ans{ \vO. }
	}
	So we see that a charge undergoing linear motion does not lead to a radiated angular momentum, which is sensible.
}

%
%	Jackson 9.8(d)
%

\prob{}{
	What are the results corresponding to Probs.~{1(a)} and {1(b)} for magnetic dipole radiation?
}

\sol{
	The radiation fields for a magnetic dipole are given by Jackson~(19.35--36),
	\al{
		\vH &= \frac{1}{4\pi} \curly{ k^2 (\nh \cross \vm) \cross \nh \frac{e^{i k r}}{r} + [ 3 \nh (\nh \vdot \vm) - \vm ] \paren{ \frac{1}{r^3} - \frac{i k}{r^2} } e^{i k r} }, &
		\vE &= -\frac{\Zo}{4\pi} k^2 (\nh \cross \vm) \frac{e^{i k r}}{r} \paren{ 1 - \frac{1}{i k r} }.
	}
	\clearpage
	Comparing with Eq.~\refeq{fields1}, we see that $\vH \to -\vE / \Zo$, $\vE \to \Zo \vH$, and $\vp \to \vm / c$ as stated in the book~\cite[p.~413]{Jackson}.  Making these substitutions, the results of Probs.~{1.1(a)} and {(b)} become
	\al{
		\ans{ \dv{\vL}{t}\ }&\ans{= \frac{\muo k^3}{12\pi} \Im[ \vms \cross \vm ], } &
		\ans{ \frac{\dv*{\vL}{t}}{\dv*{E}{t}}\ }&\ans{= \frac{\Im[ \vms \cross \vm ]}{\omg \abs{\vm}^2} }
	}
	where we have used $\mu = 1 / \epso c^2$.
}
\state{Beta function of the Gross-Neveu model~(P\&S~12.2)}{
	Compute $\bet(g)$ in the two-dimensional Gross-Neveu model studied in Problem~11.3,
	\eq{
		\cL = \psibsi i \ptsl \psisi + \frac{1}{2} g^2 (\psibsi \psisi)^2,
	}
	with $i = 1, \ldots, N$.  You should find that this model is asymptotically free.  How was that fact reflected in the solution to Problem~11.3?
}

\sol{
	We saw in Problem~2 of Homework~4 that this Lagrangian can be written as
	\eq{
		\cL = \psibsi i \ptsl \psisi - \sig \psibsi \psisi - \frac{1}{2 g^2} \sig^2,
	}
	where $\sig$ is a new scalar field with no kinetic energy terms.  In the modified minimal subtraction scheme, we found the effective potential was
	\eqn{Veff}{
		\Veff = \sig^2 \curly{ \frac{1}{2 g^2} + \frac{N}{4\pi} \brac{ \ln(\frac{\sig^2}{M^2}) - 1 } }.
	}
	Since $\Gam[ \phicl ] = -(V T) \Veff(\phi)$ by P\&S~(11.50), we have
	\eqn{Gam}{
		\Gam[ \sigcl ] = -(V T)  \sig^2 \curly{ \frac{1}{2 g^2} + \frac{N}{4\pi} \brac{ \ln(\frac{\sig^2}{M^2}) - 1 } }.
	}
	Referring to p.~3 of Lecture~11, we can apply the Callan-Symanzik equation to $\Gam$.   The Callan-Symanzik equation is P\&S~(12.41),
	\eq{
		\brac{ M \pdv{M} + \bet(\lam) \pdv{\lam} + n \gam(\lam) } G^{(n)}(\{ x_i \}; M, \lam) = 0.
	}
	For our problem, $\gam$ is 0 because there are no field insertions.  That is, we have
	\eq{
		\brac{ M \pdv{M} + \bet(g) \pdv{g} } \Gam[ \phicl ] = 0.
	}
	Using Eq.~\refeq{Gam}, note that
	\al{
		\pdv{\Gam}{M} &= (V T) \frac{N \sig^2}{2 \pi M}, &
		\pdv{\Gam}{g} &= (V T) \frac{\sig^2}{g^3}.
	}
	Then
	\eq{
		0 = (V T) \paren{ \frac{N \sig^2}{2 \pi} + \bet(g) \frac{\sig^2}{g^3} }
		\qimplies
		\ans{ \betg = -\frac{N g^3}{2\pi}. }
	}
	This model is asymptotically free because the $\bet$ function is proportional to $-g^3$~\cite[pp.~424--425]{Peskin}.
	
	In 2(e) of Homework~4, we found that the vacuum expectation value of $\sig$ was
	\eq{
		\sig = \pm M e^{-\pi / N g^2} = \pm v.
	}
	We showed that the vacuum expectation value does not depend on the renormalization condition chosen.  This means that we can increase $M \to 0$ while holding $\sig$ constant, and see that $g \to 0$ logarithmically.  This is indicative of an asymptotically-free theory~\cite[p.~425]{Peskin}. \qed
}
\newcommand{\kq}{\ket{1}}
\newcommand{\kw}{\ket{2}}
\newcommand{\ke}{\ket{3}}

\newcommand{\vq}{v_1}
\newcommand{\vw}{v_2}
\newcommand{\ve}{v_3}

\newcommand{\vqs}{\vq^*}
\newcommand{\vws}{\vw^*}

\newcommand{\Heff}{H_\text{eff}}
\newcommand{\Eo}{E\suo}
\newcommand{\Eod}{\Eo_D}

\newcommand{\Pq}{P_1}

%\clearpage
\begin{statement}{}
	Consider the Hamiltonian $\Ho$ acting on a three-dimensional Hilbert space spanned by the orthonormal basis $\{\kq, \kw, \ke\}$.  $\Ho = \sum_{i = 3}^3 E_i \ketbra{i}$, with energy eigenvalues $\Eoq, \Eow, \Eoe$.  Assume $\Eoq = \Eow = \Eod$.  To $\Ho$, we add a perturbation
	\beq
		V = \vq \ketbra{1}{3} + \vqs \ketbra{3}{1} + \vw \ketbra{2}{3} + \vws \ketbra{3}{2}.
	\eeq
	Here, $\vq$ and $\vw$ are complex constants and small compared to $\Ee$.
\end{statement}

\begin{problem}
	To second order in $V$, write down the explicit form of the effective Hamiltonian acting on the subspace spanned by $\{\kq, \kw\}$.
\end{problem}

\begin{solution}
	We have
	\begin{align*}
		\Ho &= \mqty[ \Eod & 0 & 0 \\ 0 & \Eod & 0 \\ 0 & 0 & \Eoe ], &
		V &= \mqty[ 0 & 0 & \vq \\ 0 & 0 & \vw \\ \vqs & \vws & 0 ], &
		H &= \Ho + \lam V = \mqty[ \Eod & 0 & \lam \vq \\ 0 & \Eod & \lam \vw \\ \vqs & \vws & \Eoe].
	\end{align*}
	From the lecture notes and (5.2.12) in Sakurai, the effective Hamiltonian is given to second order in $\lam$ by
	\beq
		\Heff = \Eod + \lam \Po V \Po + \lam^2 \Po V \Pq (\Eod - \Ho)^{-1} \Pq V \Po,
	\eeq
	where $\Po$ is the projection onto the degenerate subspace, $\Pq$ is the projection onto the nondegenerate subspace, and $\Eod$ is the degenerate energy.  Here, $\Po$ projects onto the subspace spanned by $\{ \kq, \kw \}$ and $\Pq$ onto that spanned by $\{ \ke \}$.
	
	Note that
	\beq
		Po V \Po = \mqty[ 1 & 0 & 0 \\ 0 & 1 & 0 \\ 0 & 0 & 0 ] \mqty[ 0 & 0 & \vq \\ 0 & 0 & \vw \\ \vqs & \vws & 0 ] \mqty[ 1 & 0 & 0 \\ 0 & 1 & 0 \\ 0 & 0 & 0 ]
		= \mqty[ 0 & 0 & 0 \\ 0 & 0 & 0 \\ 0 & 0 & 0 ],
	\eeq
	and
	\begin{align*}
		\Po V \Pq (\Eod - \Ho)^{-1} &\Pq V \Po = \frac{1}{\Eod - \Eoe} \Po \mqty[ 0 & 0 & \vq \\ 0 & 0 & \vw \\ \vqs & \vws & 0 ] \mqty[ 0 & 0 & 0 \\ 0 & 0 & 0 \\ 0 & 0 & 1] \mqty[ 0 & 0 & \vq \\ 0 & 0 & \vw \\ \vqs & \vws & 0 ] \Po \\
		&= \frac{1}{\Eod - \Eoe} \mqty[ 1 & 0 & 0 \\ 0 & 1 & 0 \\ 0 & 0 & 0 ] \mqty[|\vq|^2 & \vq \vws & 0 \\ \vqs \vw & |\vq|^2 & 0 \\ 0 & 0 & 0] \mqty[ 1 & 0 & 0 \\ 0 & 1 & 0 \\ 0 & 0 & 0 ]
		= \frac{1}{\Eod - \Eoe} \mqty[|\vq|^2 & \vq \vws & 0 \\ \vqs \vw & |\vq|^2 & 0 \\ 0 & 0 & 0].
	\end{align*}
	
	In the degenerate subspace, we have
	\begin{align*}
		\Heff = \mqty[ \Eod + |\vq|^2/(\Eod - \Eoe) & \vq \vws / (\Eod - \Eoe) \\ \vqs \vw / (\Eod - \Eoe) & \Eod + |\vw|^2/(\Eod - \Eoe)].
	\end{align*}
\end{solution}


%\clearpage
\newcommand{\uq}{u_1}
\newcommand{\uw}{u_2}
\newcommand{\wq}{w_1}
\newcommand{\ww}{w_2}

\begin{problem}
	By solving the effective Hamiltonian, construct the approximate solution for the eigenvalues and eigenfunctions of $\Ho + V$.  (The eigenkets only need to be constructed within the degenerate subspace.)
\end{problem}

\begin{solution}
	Let $E$ be the eigenvalues of $\Heff$.  We need to solve the characteristic equation
	\begin{align*}
		0 &= \det(\Heff - E I)
		= \vmqty{ \Eod + |\vq|^2/(\Eod - \Eoe) - E & \vq \vws / (\Eod - \Eoe) \\ \vqs \vw / (\Eod - \Eoe) & \Eod + |\vw|^2/(\Eod - \Eoe) - E } \\
		&= \left( \Eod + \frac{|\vq|^2}{\Eod - \Eoe} - E \right) \left( \Eod + \frac{|\vw|^2}{\Eod - \Eoe} - E \right) - \frac{|\vq|^2 |\vw|^2}{(\Eod - \Eoe)^2} \\
		&= {\Eod}^2 + \Eod \frac{|\vw|^2}{\Eod - \Eoe} - \Eod E + \Eod \frac{|\vq|^2}{\Eod - \Eoe} - E \frac{|\vq|^2}{\Eod - \Eoe} - \Eod E - E \frac{|\vw|^2}{\Eod - \Eoe} + E^2 \\
		&= E^2 - \Eod E - E \frac{|\vq|^2 + |\vw|^2}{\Eod - \Eoe} - \Eod E + {\Eod}^2 + \Eod \frac{|\vq|^2 + |\vw|^2}{\Eod - \Eoe} \\
		&= (E - \Eod) \left( E - \Eod - \frac{|\vq|^2 + |\vw|^2}{(\Eod - \Eoe)^2} \right),
	\end{align*}
	so the eigenvalues are
	\begin{align*}
		\Eq &= \Eod, &
		\Ew &= \Eod + \frac{|\vq|^2 + |\vw|^2}{(\Eod - \Eoe)^2}.
	\end{align*}
	
	The eigenvector corresponding to $\Eq$ can be found by
	\beq
		\mqty[ \Eod + |\vq|^2 / (\Eod - \Eoe) & \vq \vws / (\Eod - \Eoe) \\ \vqs \vw / (\Eod - \Eoe) & \Eod + |\vw|^2 / (\Eod - \Eoe) ] \mqty[ \uq \\ \uw ] = \Eod \mqty[ \uq \\ \uw ]
	\eeq
	which is equivalent to the system of equations
	\begin{align*}
		\left( \Eod + \frac{|\vq|^2}{\Eod - \Eoe} \right) \uq + \frac{\vq \vws}{\Eod - \Eoe} \uw &= \Eod \uq, &
		\frac{\vqs \vw}{\Eod - \Eoe} \uq + \left( \Eod + \frac{|\vw|^2}{\Eod - \Eoe} \right) \uw &= \Eod \uw.
	\end{align*}
	By inspection, these are satisfied when $\uq = -\vws$ and $\uw = \vqs$.
	
	For the eigenvector corresponding to $\Ew$, we have
	\beq
		\mqty[ \Eod + |\vq|^2 / (\Eod - \Eoe) & \vq \vws / (\Eod - \Eoe) \\ \vqs \vw / (\Eod - \Eoe) & \Eod + |\vw|^2 / (\Eod - \Eoe) ] \mqty[ \wq \\ \ww ] = \left( \Eod + \frac{|\vq|^2 + |\vw|^2}{\Eod - \Eoe} \right)\mqty[ \wq \\ \ww ]
	\eeq
	which is equivalent to the system of equations
	\begin{align*}
		\left( \Eod + \frac{|\vq|^2}{\Eod - \Eoe} \right) \wq + \frac{\vq \vws}{\Eod - \Eoe} \ww &= \left( \Eod + \frac{|\vq|^2 + |\vw|^2}{\Eod - \Eoe} \right) \wq, \\
		\frac{\vqs \vw}{\Eod - \Eoe} \wq + \left( \Eod + \frac{|\vw|^2}{\Eod - \Eoe} \right) \ww &= \left( \Eod + \frac{|\vq|^2 + |\vw|^2}{\Eod - \Eoe} \right) \ww.
	\end{align*}
	By inspection, these are satisfied when $\wq = \vq$ and $\ww = \ww$.  So we have the eigenvectors
	\begin{align*}
		\ket*{\Eq} &= \mqty[ -\vws \\ \vqs ], &
		\ket*{\Ew} &= \mqty[ \vq \\ \vw ].
	\end{align*}



\state{Phonons}{
	From Eq.~(5.8) construct $\Im[\chi]$ in the limit that $\gam \to 0$.  Use the Kramers--{\Kronig} relation to then reconstruct $\Re[\chi]$ from $\Im[\chi]$ in the same limit.
}

\sol{
	Equation~(5.8) in the lecture notes is
	\eq{
		\chivqomg = \frac{1}{-\rho \omg^2 + i \gam \omg + K q^2}.
	}
	Define $\chi' = -\rho \chi$.  Then
	\eq{
		\chi' = \frac{1}{\omg^2 - i \gam \omg / \rho - K q^2 / \rho}
		\equiv \frac{1}{\omg^2 - \omgo^2 - i b \omg}
	}
	where we have defined $\omgo^2 = K q^2 / \rho$ and $b = \gam / \rho$.  Then $\chi'$ has poles at
	\eq{
		\omg = \frac{i b \pm \sqrt{4 \omgo^2 - b^2}}{2}
		\equiv i \alp \pm \bet,
	}
	where we have defined $\alp = b / 2$ and $\bet = \sqrt{4 \omgo^2 - b^2} / 2$.  We note that $\bet$ is real since $b \propto \gam \to 0$.  Then
	\eq{
		\chi' = \frac{1}{(\omg - \bet - i\alp) (\omg + \bet - i \alp)}.
	}
	Note that as $\gam \to 0$, $\alp \to 0$ and $\bet \to 0$.  Equation (5.85) in the lecture notes is
	\eq{
		\lim_{\eta \to 0^+} \frac{1}{x + i \eta} = \PrV \frac{1}{x} - i \pi \del(x).
	}
	Applying this,
	\eq{
		\chi' = \paren{ \PrV \frac{1}{\omg - \bet} - i \pi \del(\omg - \bet) } \paren{ \PrV \frac{1}{\omg + \bet} - i \pi \del(\omg + \bet) }
	}
	
	
%	Then
%	\al{
%		\Re[\chi] &= \frac{K q^2 - \rho \omg^2}{(K q^2 - \rho \omg)^2 + \gam^2 \omg^2}, &
%		\Im[\chi] &= -\frac{\gam \omg}{(K q^2 - \rho \omg)^2 + \gam^2 \omg^2}.
%	}
%	In the limit $\gam \to 0$,
%	\al{
%		\Re[\chi] &= \frac{K q^2 - \rho \omg^2}{(K q^2 - \rho \omg)^2}, &
%		\Im[\chi] &= -\frac{\gam \omg}{(K q^2 - \rho \omg^2)^2}.
%	}
%	The relevant Kramers--{\Kronig} relation is given by (5.39),
%	\eq{
%		\Re[\kapomg] = \PrV \int \ddomgpf \frac{\Im[\kapomgp]}{\omg' - \omg}.
%	}
%	Then
%	\eq{
%		\Re[\kapomg] = -\PrV \int \ddomgpf \frac{1}{\omg' - \omg} \frac{\gam \omg'}{(K q^2 - \rho {\omg'}^2)^2}
%		= 
%	}
%	\hl{partial fraction decomposition???}
}





\clearpage
\state{Acoustic and optic phonons in the diatomic chain}{
	In the diatomic chain, we take the unit cell to be of length $a$, and take $\xA$ and $\xB$ to be the coordinates of the A and B atoms within the unit cell.  Hence, in the $n$th cell,
	\al{
		\rnA &= n a + \xA; &
		\rnB &= n a + \xB
	}
	\vfix
}

\prob{}{
	In the equations of motion Eq.~(2.30), look for solutions of the form
	\eqn{5a}{
		\unalp = \ealpq \exp( i [ q \rnalp - \omgq t ] ) + \ealpsq \exp( i [-q \rnalp + \omgq t] )
	}
	where $\alp = A$ or $B$, and $\ealp$ are complex numbers that give the amplitude and phase of the oscillation of the two atoms.
	
	Separating out the terms that have the same time dependence, show that (for equal masses, ${\mA = \mB = m}$)
	\al{
		m \omgsq \eAq &= \DAAq \eAq + \DABq \eBq, \\
		m \omgsq \eBq &= \DBAq \eAq + \DBBq \eBq,
	}
	where
	\al{
		\DAAq &= \DBBq = K + K', \\
		-\DABq &= K \exp( i q [ \rnB - \rnA ] ) + K' \exp( i q [ \rnmqB - \rnA ] ), \\
		-\DBAq &= K \exp( i q [ \rnA - \rnB ] ) + K' \exp( i q [ \rnpqA - \rnB ] ).
	}
	Check that $\DAB = \DsBA$.
}

\sol{
	Equation~(2.30) is
	\aln{ \label{thing5a}
		\mA \pdv[2]{\unA}{t} &= K (\unB - \unA) + K' (\unmqB - \unA), &
		\mB \pdv[2]{\unB}{t} &= K' (\unpqA - \unB) + K (\unA - \unB).
	}
	Note that
	\al{
		\pdv[2]{\unalp}{t} &= \pdv{t}\{ -i \omgq \ealpq \exp( i [ q \rnalp - \omgq t ] ) + i \omgq \ealpsq \exp( i [-q \rnalp + \omgq t] ) \} \\
		&= -\omgsq \{ \ealpq \exp( i [ q \rnalp - \omgq t ] ) + \ealpsq \exp( i [-q \rnalp + \omgq t] ) \} \\
		&= -\omgsq \unalp,
	}
	so the first of Eq.~\refeq{5a} can be written
	\al{
		[ K + K' - \mA \omgsq ] \unA &= K \unB + K' \unmqB, \\[1ex]
		&= K \eBq e^{i q \rnB} e^{-i \omgq t} + K \eBsq e^{-i q \rnB} e^{i \omgq t} + K' \eBq e^{i q \rnmqB} e^{-i \omgq t} \\
		&\hspace{5em} \phantom{=\ } + K' \eBsq e^{-i q \rnmqB} e^{i \omgq t} \\[1ex]
		&= K \eBq e^{i q [ na + \xB ]} e^{-i \omgq t} + K' \eBq e^{i q [ (n - 1) a + \xB ]} e^{-i \omgq t} + K \eBsq e^{-i q [ na + \xB ]} e^{i \omgq t} \\
		&\hspace{5em} \phantom{=\ } + K' \eBsq e^{-i q [ (n - 1) a + \xB ]} e^{i \omgq t} \\[1ex]
		&= (K + e^{-i q a} K') \eBq e^{i q (na + \xB)} e^{-i \omgq t} + (K + e^{-i q a} K') \eBsq e^{-i q (na + \xB)} e^{i \omgq t} \\[1ex]
		&= (K + e^{-i q a} K') \unB.
	}
	Generalizing this, we have
	\al{
		[ K + K' - \mA \omgsq ] \unA &= (K + e^{-i q a} K') \unB, &
		[ K + K' - \mB \omgsq ] \unB &= (K + e^{i q a} K') \unA.
	}
	
	Collecting terms of like time dependence yields
	\aln{
		[ K + K' - m \omgsq ] \eAq e^{i q \rnA} &= (K + e^{-i q a} K') \eBq e^{i q \rnB}, \label{thing5.a1} \\
		[ K + K' - m \omgsq ] \eBq e^{i q \rnB} &= (K + e^{i q a} K') \eAq e^{i q \rnA}, \label{thing5.a2}
	}
	for $e^{-i \omg t}$, and
	\al{
		[ K + K' - m \omgsq ] \eAsq e^{-i q \rnA} &= (K + e^{-i q a} K') \eBsq e^{-i q \rnB}, \\
		[ K + K' - m \omgsq ] \eBsq e^{-i q \rnB} &= (K + e^{-i q a} K') \eAsq e^{-i q \rnA}.
	}
	for $e^{i \omg t}$.
	
	Rearranging Eqs.~\refeq{thing5.a1} and \refeq{thing5.a2}, we have
	\al{
		m \omgsq \eAq &= (K + K') \eAq - (K + e^{-i q a} K') e^{i q (\rnB - \rnA)} \eBq \\
		&= (K + K') \eAq - (e^{i q (\rnB - \rnA)} K + e^{i q (\rnmqB - \rnA)} K') \eBq, \\[1ex]
		m \omgsq \eBq &= -(K + e^{i q a} K') e^{i q (\rnA - \rnB)} \eAq - (K + K') \eBq \\
		&= -(e^{i q (\rnA - \rnB)} K + e^{i q (\rnpqA - \rnB)} K') \eAq - (K + K') \eBq,
	}
	which gives us
	\ans{ \al{
		\DAAq &= \DBBq = K + K', \\
		\DABq &= -e^{i q (\rnB - \rnA)} K - e^{i q (\rnmqB - \rnA)} K', \\
		\DBAq &= -e^{i q (\rnA - \rnB)} K - e^{i q (\rnpqA - \rnB)} K',
	}}%
	as we wanted to show. \qed
	
	Finally, note that
	\al{
		\DsBA &= [ -e^{i q (\rnA - \rnB)} K - e^{i q (\rnpqA - \rnB)} K' ]^*
		= -e^{i q (\rnB - \rnA)} K - e^{i q (\rnB - \rnpqA)} K' \\
		&= -e^{i q (\rnB - \rnA)} K - e^{i q (\rnB - \rnA)} e^{-i q a} K'
		= -e^{i q (\rnB - \rnA)} K - e^{i q (\rnmqB - \rnA)} K' \\
		&= \ans{ \DAB }
	}
	as desired. \qed
}



\prob{}{
	The $2 \times 2$ matrix equation can have a nontrivial solution if the determinant vanishes:
	\eq{
		\mqty| 	\DAAq - m \omgsq & \DABq \\
				\DBAq & \DBBq - m \omgsq |
		= 0.
	}
	Hence show that the frequencies of the modes are given by
	\eq{
		m \omgsq = K + K' \pm \sqrt{ (K + K')^2 - 4 K K' \sin[2]( \frac{q a}{2} ) }.
	}
	\vfix
}

\sol{
	The determinant is
	\eq{
		0 = [ \DAAq - m \omgsq ] [ \DBBq - m \omgsq ] - \DABq \DBAq
		= [ \DAAq - m \omgsq ]^2 - \DABq \DBAq,
	}
	which implies
	{\allowdisplaybreaks
	\aln{
		m \omgsq &= \DAAq \pm \sqrt{\DABq \DBAq} \notag \\
		&= K + K' \pm \sqrt{(e^{i q (\rnB - \rnA)} K + e^{i q (\rnmqB - \rnA)} K') (e^{i q (\rnA - \rnB)} K + e^{i q (\rnpqA - \rnB)} K')} \notag \\
		&= K + K' \pm \sqrt{(K + e^{-i q a} K') e^{i q (\rnB - \rnA)} (K + e^{i q a} K') e^{i q (\rnA - \rnB)}} \notag \\
		&= K + K' \pm \sqrt{(K + e^{-i q a} K') (K + e^{i q a} K')}
		= K + K' \pm \sqrt{K^2 + (e^{i q a} + e^{-i q a}) K K' + {K'}^2} \notag \\
		&= K + K' \pm \sqrt{K^2 + 2\cos(q a) K K' + {K'}^2} \label{5bo} \\
		&= K + K' \pm \sqrt{K^2 + \brac{ 2 - 4 \sin[2](\frac{q a}{2}) } K K' + {K'}^2} \notag \\
		&= K + K' \pm \sqrt{K^2 + 2 K K' + {K'}^2 - 4 \sin[2](\frac{q a}{2}) K K'} \notag \\
		&= \ans{ K + K' \pm \sqrt{ (K + K')^2 - 4 K K' \sin[2]( \frac{q a}{2} ) }, } \label{5b}
	}}
	where we have used the double-angle formula $\cos(2x) = 1 - 2 \sin[2](x)$~\cite{DoubleAngle}. \qed
}



\prob{}{
	Sketch the dispersion relations when $K / K' =  2$.
}

\sol{
	There are two dispersion curves since there are two solutions in Eq.~\refeq{5b}.  The expressions for the branches are
	\eqn{5ceq}{
		\omgq = \frac{1}{\sqrt{m}} \sqrt{ K + K' \pm \sqrt{ (K + K')^2 - 4 K K' \sin[2]( \frac{q a}{2} ) } }
		\begin{cases}
			\text{optical}, \\
			\text{acoustic},
		\end{cases}
	}
	where the acoustic~(optical) branch corresponds to the upper~(lower) sign.  Both branches are shown in Fig.~\ref{5c}, with the $K / K' =  2$ case on the left and the $K = K'$ case on the right.
	
	\fig{5c}{
		\includegraphics[width=0.5\textwidth,trim=1.5cm 0 0 0,clip]{5c}
		\caption{Dispersion curves for $K / K' =  2$.  The optical branch~(blue) corresponds to the upper sign in Eq.~\refeq{5ceq}, and the acoustic branch~(gold) to the lower sign.}
	}
}



\prob{}{
	  Discuss what happens if $K = K'$.
}

\sol{
	If $K = K'$, then not only are the masses of the two atoms identical, but so are their restorative forces.  Thus, the system is essentially reduced to a monatomic chain~\cite[p.~437]{Ashcroft}.  Picking up from Eq.~\refeq{5bo},
	\eq{
		m \omgsq = 2 K \pm \sqrt{2 K^2 + 2 \cos(q a) K^2}
		= 2 K \pm K \sqrt{4 \cos[2](\frac{q a}{2})}
		= 2 K \brac{ 1 - \cos(\frac{q a}{2}) }
		= 4 K \sin[2](\frac{q a}{4}),
	}
	where we have used the double-angle formula $\cos(2 x) = 2 \cos[2](x) - 1$~\cite{DoubleAngle}.  This is Eq.~\refeq{2.25} with $q a \to q a / 2$.  So in this limit, the diatomic chain is reduced to a monatomic chain with lattice constant $a / 2$~\cite[p.~437]{Ashcroft}.
}

\state{Partition function as a generating functional}{
	Consider the Gibbs distribution of the system described in Problem~5.  For simplicity neglect the kinetic energy. Let $n(x) = \sumi \del(x - \xii)$ be the density, and $\evnx$ its expectation value. Let $C(x, y) = \ev{\del n(x) \,\del n(y)}$, where $\del n(x) = n(x) - \evn$, be the two-point correlation function.
}

%
%	6.1
%

\prob{}{
	Show that $\evnx = -T \,\deldvs{\ln Z}{U(x)}$, where $Z[U(x)]$ is the partition function of the Gibbs distribution treated as a functional of the potential $U$.
}

\sol{
	The expectation value of $n(x)$ is
	\eqn{evn}{
		\evnx = \frac{1}{Z} \int n(x) \,e^{-\bet H} \prodjN \ddxjj
		= \frac{1}{Z} \int \sumiN \del(x - \xii) \,e^{-\bet H} \prodjN \ddxjj,
	}
	where $Z$ is the partition function.
	\clearpage
	Adapting the Hamiltonian in Eq.~\refeq{Ham5}, we have
	\eq{
		H = \sumiN U(\xii) + \sumiN \sumji V(\xii - \xjj).
	}
	The partition function of the Gibbs distribution for this system is then
	\al{
		Z &= \int e^{-\bet H} \prodjN \ddxjj
		= \int \exp( \bet \sumiN U(\xii) + \bet \sumiN \sumji V(\xii - \xjj) ) \prodkN \ddxkk.
	}
	The basic definition of the functional derivative in one dimension is~\cite[p.~289]{Peskin}
	\al{
		\deldv{J(y)}{J(x)} &= \del(x - y), &
		\deldv{}{J(x)} \int J(y) \,\phi(y) \dd{y} &= \int \del(x - y) \,\phi(y) \dd{y}
		= \phi(x).
	}
	Note that
	\eqn{thing5}{
		\deldv{\ln Z}{U(x)}
		= \pdv{\ln Z}{Z} \deldv{Z}{U(x)}
		= \frac{1}{Z} \deldv{Z}{U(x)},
	}
	and that
	\aln{
		\deldv{Z}{U(x)} &= \deldv{}{U(x)} \int \exp( \bet \sumiN U(\xii) + \bet \sumiN \sumji V(\xii - \xjj) ) \prodkN \ddxkk
%		&= \int -\bet \sumiN \deldv{U(\xii)}{U(x)} \exp( \bet \sumiN U(\xii) + \bet \sumiN \sumji V(\xii - \xjj) ) \prodkN \ddxkk
		= \int -\bet \sumiN \deldv{U(\xii)}{U(x)} \, e^{-\bet H} \prodjN \ddxjj \notag \\
		&= -\frac{1}{T} \int \sumiN \del(x - \xii) \,e^{-\bet H} \prodjN \ddxjj
		= -\frac{Z}{T} \evnx, \label{ans6.1}
	}
	where we have used Eq.~\refeq{evn}.  Then, from Eq.~\refeq{thing5}, we have
	\eq{
		-T \deldv{\ln Z}{U(x)} = \frac{T}{Z} \deldv{Z}{U(x)}
		= \ans{ \evnx, }
	}
	as desired. \qed
}

%
%	6.2
%

\prob{}{
	Show that
	\eqn{show6.2}{
		C(x, y) = T^2 \deldvm{\ln Z}{U(x)}{U(y)}
		= -T \deldv{\evnx}{U(y)}
		= -T \deldv{\evny}{U(x)}.
	}
}

\sol{
	Firstly,
	\eq{
		C(x, y) = \ev{n(x) \,n(y)} - \evn^2
		= \frac{1}{Z} \int n(x) \,n(y) \,e^{-\bet H} \prodjN \ddxjj - \evn^2.
	}
	The final two equalities of Eq.~\refeq{show6.2} follow directly from Eq.~\refeq{ans6.1} and the fact that the order of the derivatives is interchangeable:
	\al{
		T^2 \deldvm{\ln Z}{U(x)}{U(y)} &= T \deldv{}{U(y)} \paren{ T \deldv{\ln Z}{U(x)} }
		= \ans{ -T \deldv{\evnx}{U(y)}, } \\[2ex]
		T^2 \deldvm{\ln Z}{U(x)}{U(y)} &= T \deldv{}{U(x)} \paren{ T \deldv{\ln Z}{U(y)} }
		= \ans{ -T \deldv{\evny}{U(x)}. }
	}
	To prove the first equality, we will show that $C(x, y) = -T \,\del\evnx / \del U(y)$.  Note that
	\aln{
		\deldv{\evnx}{U(y)} &= \deldv{}{U(y)} \paren{ \frac{1}{Z} \int n(x) \,e^{-\bet H} \prodjN \ddxjj } \notag \\
		&= \deldv{}{U(y)} \paren{ \frac{1}{Z} } \int n(x) \,e^{-\bet H} \prodjN \ddxjj + \frac{1}{Z} \deldv{}{U(y)} \paren{ \int n(x) \,e^{-\bet H} \prodjN \ddxjj }. \label{thing6.2}
	}
	For the first term,
	\eq{
		\deldv{}{U(y)} \paren{ \frac{1}{Z} } = \pdv{(1/Z)}{Z} \deldv{Z}{U(y)}
		= -\frac{1}{Z^2} \deldv{Z}{U(y)}
		= \frac{\evny}{Z T},
	}
	where we have used Eq.~\refeq{ans6.1}.  For the second term,
	\al{
		\deldv{}{U(y)} \int n(x) \,e^{-\bet H} \prodjN \ddxjj &= \deldv{}{U(y)} \int n(x) \exp( \bet \sumiN U(\xii) + \bet \sumiN \sumji V(\xii - \xjj) ) \prodkN \ddxkk \\
		&= \int -\bet n(x) \sumiN \deldv{U(\xii)}{U(y)} e^{-\bet H} \prodjN \ddxjj
		= -\frac{1}{T} \int n(x) \,n(y) e^{-\bet H} \prodjN \ddxjj.
	}
	Substituting back into Eq.~\refeq{thing6.2},
	\eq{
		\deldv{\evnx}{U(y)} = \frac{\evny}{Z T} \int n(x) \,e^{-\bet H} \prodjN \ddxjj - \frac{1}{Z T} \int n(x) \,n(y) e^{-\bet H} \prodjN \ddxjj
		= \frac{\evny \evnx}{T} - \frac{\ev{n(x) \,n(y)}}{T}.
	}
	Then
	\eq{
		-T \deldv{\evnx}{U(y)} = \ev{n(x) \,n(y)} - \evn^2
		= \ans{ C(x, y), }
	}
	as desired.  So we have proven Eq.~\refeq{show6.2} in its entirety. \qed
}

\state{Optical properties}{	\label{7}
	Discuss why, at optical frequencies, glass is transparent and silver is shiny, while graphite appears black and powdered sugar is white.
}

\sol{
	\begin{figure}[t]
		\centering
		\includegraphics[width=.5\textwidth]{medium.png}
		\caption{Band structure of graphite, from Ref.~\cite{Graphite}.}
		\label{f7}
	\end{figure}

	Glass is an amorphous solid; that is, it does not have a crystal structure~\cite[pp.~573--674]{Kittel}.  Only solids with crystal structure have an energy band structure~\cite[p.~161]{Kittel}.  In order to absorb visible light, glass would need a band structure with the spacing between bands corresponding to optical frequencies.   Glass does not have this (or any) band structure, and a large gap between energy levels, so it cannot absorb visible light.  Moreover, glass is a nonmetal and so does not have free electrons, meaning it cannot appreciably reflect light.  Finally, since glass is created by cooling a liquid without crystallization~\cite[pp.~573--574]{Kittel}, it does not have structural impurities and thus does not scatter light.  Since glass does not appreciably absorb, reflect, or scatter visible light, it must only transmit it; this makes it transparent.
	
	Silver, in contrast, is a metal and therefore characterized by its abundance of free electrons in a partially filled energy band~\cite[p.~562]{Ashcroft}.  This means it has a high plasma frequency $\omgp^2 \propto n$ from (5.27), where $n$ is the free electron concentration.  We know from Fig.~\ref{f3d}~(left) that a Drude metal reflects light of all frequencies up to the plasma frequency; since $\omgp$ is large, it must be above the optical range.  This enables silver to reflect much of the optical light that is incident upon it.  In addition, the reflected waves interfere with light that would be transmitted.  So silver, like other metals, appears shiny because it almost exclusively reflects light.
	
	The band structure of graphite is shown in Fig.~\ref{f7}, which is from Ref.~\cite{Graphite}.  The de Broglie energy of visible light ranges from approximately 1.6 to \SI{3.3}{\electronvolt}~\cite[p.~1051]{YF}.  As the figure illustrates, graphite has many energy bands, and the spacing between them allows for band-to-band transmissions at nearly all optical frequencies.  This means that nearly all visible light that is incident upon graphite is absorbed, creating its black appearance.
	
	Table sugar is very pure, crystalline sucrose.  Granulated sugar is made up of small, nearly-transparent sucrose crystals with smooth edges~\cite{Sucrose}.  Powdered sugar is the result of grinding these crystals into a fine powder.  Since the grinding process results in many structural imperfections and harsh edges on the surface of the powdered sugar particles, they scatter visible light.  This gives powdered sugar its white appearance.
}





\state{Metals and insulators}{
	Explain the differences between a metal and an insulator.  Your discussion should include reference to single particle properties, screening of the Coulomb interaction, optical properties, and electrical and thermal properties.
}


\makebib

\end{document}
