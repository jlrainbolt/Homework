\state{Metals and insulators}{
	Explain the differences between a metal and an insulator.  Your discussion should include reference to single particle properties, screening of the Coulomb interaction, optical properties, and electrical and thermal properties.
}

\sol{
	An insulator has a completely filled set of bands and empty higher-energy bands.  Thus, an insulator has a band gap.  A metal, however, has a set of bands that is partially filled.  The occupied states with highest energy sit at the Fermi surface in momentum space.  The number of electrons per unit cell in a metal may be even or odd, although it is typically odd.  In contrast, an insulator must have an even number of electrons per unit cell~(as mentioned on p.~39 of the lecture notes).
	
	In metals, the Coulomb interaction is screened; the external potential $\Vext(\vr) = Q / r$ is ``seen'' by the electrons in the metal as $V(\vr) = Q e^{-\qTF r} / r$~\cite[p.~342]{Ashcroft}.  This is because the charge distribution of the free electrons is affected by the potential.  Since insulators, not being conductors, do not have freely-flowing electrons within them, the effect of screening are small.
	
	Plasma oscillations or ``plasmons'' occur within an electron gas (p.~75 of the lecture notes), meaning the free electrons in a metal.  Metals are very reflective because they have a high plasma frequency due to their large number of free electrons, as explained in~\ref{7}.  Insulators, on the other hand, do not have such a freely-flowing electron gas; instead, they have band gaps that allow them to absorb light.  If the band gaps are of the proper width to absorb optical frequencies, they will not be visibly reflective.
	
	Metals readily conduct electricity thanks to their abundance of free electrons.  Insulators, by definition, do not conduct electricity; their electrons are too limited in their range of motion.  Insulators, however, can become electrically polarized.  The specific heat of the electron gas is given by (2.16), where we see $c_v \propto T$.  This means that, as the temperature of a metal increases, it costs more energy to increase its temperature.
}