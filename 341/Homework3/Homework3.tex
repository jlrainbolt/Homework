\documentclass[11pt]{article}
\usepackage{geometry, titlesec}
\usepackage[parfill]{parskip}
\usepackage{physics, amsfonts, amsthm}
\usepackage[cm]{fullpage}
\usepackage{fancyhdr}
\usepackage{enumitem}
\usepackage{xcolor, soul}
%\allowdisplaybreaks

\makeatletter
\renewcommand*\env@cases[1][1.2]{%
  \let\@ifnextchar\new@ifnextchar
  \left\lbrace
  \def\arraystretch{#1}%
  \array{@{}l@{\quad}l@{}}%
}
\makeatother
 
 
\renewcommand{\footrulewidth}{.2pt}
%\setlist[enumerate]{leftmargin=*}
\pagestyle{fancy}
\fancyhf{}
\lhead{\textbf{Physics 341 Homework 3}}
\rhead{Lacey Rainbolt}
\setlength{\headheight}{11pt}
\setlength{\headsep}{11pt}
\setlength{\footskip}{24pt}
\lfoot{\today}
\rfoot{\thepage}

\titleformat{\subsection}[runin]{\normalfont\large\bfseries}{\thesubsection}{1em}{}
\newcommand{\refeq}[1]{(\ref{#1})}

\newcommand{\beq}{\begin{equation*}}
\newcommand{\eeq}{\end{equation*}}

\newcommand{\beqn}{\begin{equation}}
\newcommand{\eeqn}{\end{equation}}


\newenvironment{statement}
{
    \color{darkgray}
    \ignorespaces
}
{
%    \smallskip
}

\newenvironment{problem}
{
    \color{darkgray}
    \subsection{}
    \ignorespaces
}


\newenvironment{solution}
{
    \paragraph{Solution.}
    \ignorespaces
}
{
%    \smallskip
}

\newcommand{\Schrodinger}{Schr\"{o}dinger}


\begin{document}

\newcommand{\lam}{\lambda}
\newcommand{\klam}{\ket{\lam}}
\newcommand{\ad}{a^\dagger}

\newcommand{\ko}{\ket{0}}
\newcommand{\bkxlam}{\braket{x}{\lam}}
\newcommand{\bx}{\bra{x}}

\newcommand{\ihb}{i\hbar}
\newcommand{\bkxo}{\braket{x}{0}}

\newcommand{\Dx}{\Delta x}
\newcommand{\Dp}{\Delta p}
	
\section{Problem 1}
\begin{statement}
	Let's consider coherent states of a one-dimensional quantum particle with mass $m$ confined in a one-dimensional harmonic potential $V(x) = m \omega^2 x^2 / 2$:
	\begin{align*}
		a \klam &= \lam \klam, &
		\klam &= \exp(-\frac{1}{2} |\lam|^2) \exp(\lam \ad) \ko.
	\end{align*}
	Here, $\lam$ is a complex paramter.
\end{statement}

\begin{problem}
	Compute $\bkxlam$.
\end{problem}

\begin{solution}
	In terms of the position and momentum operators $X$ and $P$,
	\begin{align*}
		a &= \sqrt{\frac{m\omega}{2\hbar}} \left( X + \frac{iP}{m\omega} \right), &
		\ad &= \sqrt{\frac{m\omega}{2\hbar}} \left( X - \frac{iP}{m\omega} \right),
	\end{align*}
	so
	\beqn \label{ugh}
		\bkxlam = \exp(-\frac{|\lam|^2}{2}) \bx \exp(\lam \ad) \ko = \exp(-\frac{|\lam|^2}{2}) \bx \exp{\lam \sqrt{\frac{m\omega}{2\hbar}} \left( X - \frac{iP}{m\omega} \right)} \ko.
	\eeqn
	Note that for two operators $A$ and $B$, $e^{A + B} = e^{-[A, B] / 2} e^A e^B$ if $[A, B]$ commutes with each $A$ and $B$.  Note also that
	\beq
		\left[ X, -\frac{i P}{m \omega} \right] = -\frac{i}{m \omega} [X, P] = \frac{\hbar}{m \omega}.	
	\eeq
	Thus,
	\beq
		\exp{\lam \sqrt{\frac{m\omega}{2\hbar}} \left( X - \frac{iP}{m\omega} \right)} = \exp(-\frac{\hbar \lambda}{2 m \omega} \sqrt{\frac{m\omega}{2\hbar}}) \exp(\lam \sqrt{\frac{m\omega}{2\hbar}} X) \exp(-\frac{i \lambda}{m \omega} \sqrt{\frac{m\omega}{2\hbar}} P)
	\eeq
	so \refeq{ugh} becomes
	\begin{align*}
		\bkxlam &= \exp(-\frac{|\lam|^2}{2}) \exp(-\frac{\hbar \lambda}{2 m \omega}) \bx \exp(\lam \sqrt{\frac{m\omega}{2\hbar}} X) \exp(-\frac{i \lambda}{m \omega} \sqrt{\frac{m\omega}{2\hbar}} P) \ko \\
		&= \exp(-\frac{|\lam|^2}{2}) \exp(-\frac{\hbar \lambda}{2 m \omega}) \exp(\lam \sqrt{\frac{m\omega}{2\hbar}} x) \exp(-\frac{\hbar \lambda}{m \omega} \sqrt{\frac{m\omega}{2\hbar}} \pdv{}{x}) \bkxo
	\end{align*}
	From (2.3.30) in Sakurai,
	\beq
		\bkxo = \left( \frac{m \omega}{\pi \hbar} \right)^{1/4} \exp(-\frac{m \omega}{2 \hbar} x^2).
	\eeq
%	
%	$$ \exp(-\frac{|\lam|^2}{2}) \exp[\lam \sqrt{\frac{m\omega}{2\hbar}} \left( x - \frac{\hbar}{m\omega} \pdv{}{x} \right)] \bkxo$$
%	\beq
%		\exp[\lam \sqrt{\frac{m\omega}{2\hbar}} \left( x - \frac{\hbar}{m\omega} \pdv{}{x} \right)] = \exp(
%	\eeq
%	so \refeq{ugh} becomes
%	\beq
%		\bkxlam = \left( \frac{m \omega}{\pi \hbar} \right)^{1/4} \exp(-\frac{|\lam|^2}{2}) \exp[\lam \sqrt{\frac{m\omega}{2\hbar}} \left( x - \frac{\hbar}{m\omega} \pdv{}{x} \right)] \exp(-\frac{m \omega}{2 \hbar} x^2).
%	\eeq
%	\hl{???}
\end{solution}

\begin{problem}
	Compute $\ev{x}{\lam}$, $\ev{p}{\lam}$, $\ev{x^2}{\lam}$, and $\ev{p^2}{\lam}$.  Also, compute $\ev{(\Dx)^2}_\lam \ev{(\Dp)^2}_\lam$ where $\Delta A = A \ev{A}$.
\end{problem}

\begin{solution}
%	\begin{align*}
%	
%	\end{align*}
\end{solution}
	
\end{document}