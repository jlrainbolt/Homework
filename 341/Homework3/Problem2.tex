\newcommand{\Xq}{X_1}
\newcommand{\Pq}{P_1}
\newcommand{\Xw}{X_2}
\newcommand{\Pw}{P_2}
\newcommand{\Hc}{H_c}

\newcommand{\Xqo}{\Xq(0)}
\newcommand{\Pqo}{\Pq(0)}
\newcommand{\Xwo}{\Xw(0)}
\newcommand{\Pwo}{\Pw(0)}

\newcommand{\Xqt}{\Xq(t)}
\newcommand{\Pqt}{\Pq(t)}
\newcommand{\Xwt}{\Xw(t)}
\newcommand{\Pwt}{\Pw(t)}

\newcommand{\XqT}{\Xq(T)}
\newcommand{\PqT}{\Pq(T)}
\newcommand{\XwT}{\Xw(T)}
\newcommand{\PwT}{\Pw(T)}

\newcommand{\kPsio}{\ket{\Psi(0)}}
\newcommand{\phio}{\phi_1(0) \,\phi_2(0)}
\newcommand{\kphiqo}{\ket{\phi_1(0)}}
\newcommand{\kphiwo}{\ket{\phi_2(0)}}

\newcommand{\kphio}{\ket{\phio}}


\newcommand{\Ot}{O(t)}
%\newcommand{\Ut}{U(t)}
\newcommand{\Udt}{U^\dagger(t)}

\section{Problem 2}
\begin{statement}
	Consider a quantum system which has coordinate $\Xq$ and momentum $\Pq$, and another system which has coordinate $\Xw$ and momentum $\Pw$.  (An operator from the first system always commutes with an operator of the second system.)  We think of the second system as a ``probe'' which we can use to detect the properties of the first system.  For a short time $T$, the two systems are coupled by a coupling Hamiltonian $\Hc$, given by
	\beq
		\Hc = \frac{\Xq \Pw}{T}.
	\eeq
	The coupling between the two systems disturbs the momentum of the first system.  The disturbance operator is defined to be
	\beqn \label{dist}
		D \equiv \Pq(T) - \Pqo.
	\eeqn
	The probe introduces measurement error or ``noise'' into the system.  The noise operator is defined by
	\beq
		N \equiv \Xw(T) - \Xqo.
	\eeq
	The stste of the system at $t = 0$ is $\kPsio = \kphio$, and all expectation values are taken in this state.
\end{statement}

\begin{problem}
	With $\Hc$ as the Hamiltonian, find the Heisenberg operators $\Xqt$, $\Pqt$, $\Xwt$, and $\Pwt$ in terms of $\Xqo$, $\Pqo$, $\Xwo$, and $\Pwo$.  Time is restricted to the range $t \in [0, T]$.
\end{problem}

\begin{solution}
	In general, a Heisenberg operator $\Ot$ is defined by
	\beq
		\Ot = \Udt \, O(0) \, \Ut,
	\eeq
	where $\Ut$ is the time evolution operator.  For $\Hc$, it is given by
	\beq
		\Ut = \exp(-\frac{i \Hc t}{\hbar}) = \exp(-\frac{i t}{\hbar T} \, \Xqo \, \Pwo).
	\eeq
	(2.2.23b) in Sakurai gives the commutation relations
	\begin{align*}
		[X_i, F(\vec{P})] &= \ihb \pdv{F}{P_i} &
		[P_i, G(\vec{X})] &= -\ihb \pdv{G}{X_i}.
	\end{align*}
	Using these, we have
	\begin{align*}
		[\Xqo, \Ut] &= 0, \\
		[\Xwo, \Ut] &= \ihb \left( -\frac{i t}{\hbar T} \, \Xqo \right) \Ut = \frac{t}{T} \, \Xqo \, \Ut = \frac{t}{T} \, \Ut \, \Xqo, \\
		[\Pqo, \Ut] &= -\ihb \left( -\frac{i t}{\hbar T} \, \Pwo \right) \Ut = -\frac{t}{T} \, \Pwo \, \Ut = -\frac{t}{T} \, \Ut \, \Pwo, \\
		[\Pwo, \Ut] &= 0.
	\end{align*}
	Then
	\begin{align}
		\Xqt &= \Udt \, \Xqo \, \Ut = \Xqo, \label{Xqt} \\
		\Pqt &= \Udt \, \Pqo \, \Ut = \Udt \left( \Ut \, \Pqo - \frac{t}{T} \Ut \, \Pwo \right) = \Pqo - \frac{t}{T} \, \Pwo, \label{Pqt} \\
		\Xwt &= \Udt \, \Xwo \, \Ut = \Udt \left( \Ut \, \Xwo + \frac{t}{T} \, \Ut \, \Xqo  \right) = \Xwo + \frac{t}{T} \, \Xqo, \label{Xwt} \\
		\Pwt &= \Udt \, \Pwo \, \Ut = \Pwo. \label{Pwt}
	\end{align}
\end{solution}

\newcommand{\sigD}{\sigma(D)}
\newcommand{\sigN}{\sigma(N)}

\begin{problem}
	Derive an expression for $\sigD$ which involves only the standard deviations of $\Xqo$, $\Pqo$, $\Xwo$, and $\Pwo$.  Here, we denote the standard deviation of an operator $O$ as $\sigma(O) = \sqrt{\ev{(O - \ev{O})^2}}$.
\end{problem}

\begin{solution}
	Substituting \refeq{Pwt} into \refeq{dist},
	\beq
		D = \Pqo - \frac{T}{T} \Pwo - \Pwo = -\Pwo.
	\eeq
	Note that for an operator $O$,
	\beq
		\sigma(-O) = \sqrt{\ev{(-O - \ev{-O})^2}} = \sqrt{\ev{(\ev{O} - O)^2}} = \sigma(O),
	\eeq
	so
	\beqn \label{sigD}
		\sigD = \sigma\big( \Pwo \big).
	\eeqn
\end{solution}

\begin{problem}
	Derive an expression for $\sigN$ which involves only the standard deviations of $\Xqo$, $\Pqo$, $\Xwo$, and $\Pwo$.
\end{problem}

\begin{solution}
	Substituting \refeq{Xwt} into \refeq{dist},
	\beq
		N = \Xwo + \frac{T}{T} \Xqo - \Xqo = \Xwo.
	\eeq
	which implies
	\beqn \label{sigN}
		\sigN = \sigma\big( \Xwo \big).
	\eeqn
\end{solution}

\begin{problem}
	Now consider the product $\sigN \, \sigD$.  Assume
	\begin{align*}
		\sigma\big( \Xqo \big) \, \sigma\big( \Pqo \big) &\geq \frac{\hbar}{2}, &
		\sigma\big( \Xwo \big) \, \sigma\big( \Pwo \big) &\geq \frac{\hbar}{2}
	\end{align*}
	both hold.  Is $\sigN \, \sigD \geq \hbar / 2$ satisfied?  What conditions are required for equality?
\end{problem}

\begin{solution}
	From \refeq{sigD} and \refeq{sigN},
	\beq
		\sigN \, \sigD = \sigma\big( \Pwo \big) \, \sigma\big( \Xwo \big) \geq \frac{\hbar}{2},
	\eeq
	where the final inequality is satisfied by assumption.  For equality, we would need
	\beq
		\sigma\big( \Xwo \big) \, \sigma\big( \Pwo \big) = \frac{\hbar}{2}.
	\eeq
%	\begin{align*}
%		\frac{\hbar}{2} &= \sigma\big( \Xwo \big) \, \sigma\big( \Pwo \big) \\
%		\frac{\hbar^2}{4} &= \ev{\big( \Xwo - \ev{\Xwo}\big)^2} \ev{\big( \Pwo - \ev{\Pwo}\big)^2} %= z \big( \ev{\Xwo^2} - \ev{\Xwo}^2 \big) \big( \ev{\Pwo^2} - \ev{\Pwo}^2 \big)
%	\end{align*}
	
\end{solution}