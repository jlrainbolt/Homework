\newcommand{\lam}{\lambda}
\newcommand{\blam}{\bra{\lam}}
\newcommand{\klam}{\ket{\lam}}
\newcommand{\ad}{a^\dagger}

\newcommand{\bo}{\bra{0}}
\newcommand{\ko}{\ket{0}}
\newcommand{\kn}{\ket{n}}
\newcommand{\bkxlam}{\braket{x}{\lam}}
\newcommand{\bx}{\bra{x}}

\newcommand{\ihb}{i\hbar}
\newcommand{\bkxo}{\braket{x}{0}}
\newcommand{\bkxn}{\braket{x}{n}}
\newcommand{\xo}{x_0}

\newcommand{\DA}{\Delta A}
\newcommand{\DX}{\Delta X}
\newcommand{\DP}{\Delta P}

\newcommand{\bkxbo}{\braket{x-b}{0}}
	
\section{Problem 1}
\begin{statement}
	Let's consider coherent states of a one-dimensional quantum particle with mass $m$ confined in a one-dimensional harmonic potential $V(X) = m \omega^2 X^2 / 2$:
	\begin{align*}
		a \klam &= \lam \klam, &
		\klam &= \exp(-\frac{1}{2} |\lam|^2) \exp(\lam \ad) \ko.
	\end{align*}
	Here, $\lam$ is a complex parameter.
\end{statement}

\begin{problem}
	Compute $\bkxlam$.
\end{problem}

\begin{solution}
	Since $a \ko = 0 \ko$, $\exp(\lam a) \ko = \ko$ and therefore we can write
	\beqn \label{ugh1}
		\bkxlam = \exp(-\frac{1}{2} |\lam|^2) \bx \exp(\lam \ad) \exp(\lam a) \ko.
	\eeqn
	For two operators $A$ and $B$, $e^{A + B} = e^{-[A, B] / 2} e^A e^B$ if $[A, B]$ commutes with each $A$ and $B$.  Here, we have
	\beq
		\exp[\lam(\ad + a)] = \exp(\frac{\lam^2}{2}) \exp(\lam \ad) \exp(\lam a) \implies \exp(\lam \ad) \exp(\lam a) = \exp(-\frac{\lam^2}{2}) \exp[\lam(\ad + a)],
	\eeq
	where we have used $[a, \ad] = 1$.  From (2.3.24) in Sakurai,
	\begin{align} \label{xpa}
		X &= \sqrt{\frac{\hbar}{2 m \omega}} (a + \ad), &
		P &= i \sqrt{\frac{\hbar m \omega}{2}} (\ad - a),
	\end{align}
	so
	\beq
		\exp[\lam(\ad + a)] = \exp(\lam X \sqrt{\frac{2 m \omega}{\hbar}}).
	\eeq
	Making these substitutions into \refeq{ugh1} yields
	\begin{align}
		\bkxlam &= \exp(-\frac{1}{2} |\lam|^2) \exp(-\frac{\lam^2}{2}) \bx \exp(\lam X \sqrt{\frac{2 m \omega}{\hbar}}) \ko \notag \\
		&= \exp(-\frac{1}{2} |\lam|^2) \exp(-\frac{\lam^2}{2}) \exp(\lam x \sqrt{\frac{2 m \omega}{\hbar}}) \bkxo. \label{ugh2}
	\end{align}
	From (2.3.30) in Sakurai,
	\beq
		\bkxo = \left( \frac{m \omega}{\pi \hbar} \right)^{1/4} \exp(-\frac{m \omega}{2 \hbar} x^2)
	\eeq
	so \refeq{ugh2} becomes
	\beq
		\bkxlam = \left( \frac{m \omega}{\pi \hbar} \right)^{1/4} \exp(-\frac{1}{2} |\lam|^2 - \frac{\lam^2}{2} + \lam x \sqrt{\frac{2 m \omega}{\hbar}} - \frac{m \omega}{2 \hbar} x^2).
	\eeq
\end{solution}

\clearpage
\begin{problem}
	Compute $\ev{X}{\lam}$, $\ev{P}{\lam}$, $\ev{X^2}{\lam}$, and $\ev{P^2}{\lam}$.  Also, compute $\ev{(\DX)^2}{\lam} \ev{(\DP)^2}{\lam}$ where $\DA = A - \ev{A}$.
\end{problem}

\begin{solution}
	For $\ev{X}{\lam}$,
	\begin{align}
		\ev{X}{\lam} &= \frac{1}{|\lam|^2} \ev{\ad X a}{\lam} \notag \\
		&= \frac{1}{|\lam|^2} \sqrt{\frac{\hbar}{2 m \omega}} \ev{\ad (a + \ad) a}{\lam}
		= \frac{1}{|\lam|^2} \sqrt{\frac{\hbar}{2 m \omega}} \ev{(\ad a^2 + {\ad}^2 a}{\lam}
		= \frac{|\lam|^2 (\lam^* + \lam)}{|\lam|^2} \sqrt{\frac{\hbar}{2 m \omega}} \notag \\
		&= 2 \Re(\lam) \sqrt{\frac{\hbar}{2 m \omega}}, \label{evX}
	\end{align}
	where we have again used \refeq{xpa}.  For $\ev{P}{\lam}$,
	\begin{align}
		\ev{P}{\lam} &= \frac{1}{\lambda^2} \ev{\ad P a}{\lam} \notag \\
		&= \frac{i}{|\lam|^2} \sqrt{\frac{\hbar m \omega}{2}} \ev{\ad(\ad - a)a}{\lam} = \frac{i}{|\lam|^2} \sqrt{\frac{\hbar m \omega}{2}} \ev{{\ad}^2 a - \ad a^2}{\lam}
		= \frac{i |\lam|^2 (\lam^* - \lam)}{|\lam|^2} \sqrt{\frac{\hbar m \omega}{2}} \notag \\
		&= 2 \Im(\lam) \sqrt{\frac{\hbar m \omega}{2}}. \label{evP}
	\end{align}
	From \refeq{xpa}, note that
	\begin{align*}
		X^2 &= \frac{\hbar}{2 m \omega} (a^2 + a \ad + \ad a + {\ad}^2), &
		P^2 &= -\frac{\hbar m \omega}{2} ({\ad}^2 - \ad a - a \ad + a^2).
	\end{align*}
	Then for $\ev{X^2}{\lam}$,
	\begin{align}
		\ev{X^2}{\lam} &= \frac{1}{|\lam|^2} \ev{\ad X^2 a}{\lam}
		= \frac{1}{|\lam|^2} \frac{\hbar}{2 m \omega} \ev{\ad (a^2 + a \ad + \ad a + {\ad}^2) a}{\lam} \notag \\
		&= \frac{1}{|\lam|^2} \frac{\hbar}{2 m \omega} \ev{(\ad a^3 + \ad a \ad a + {\ad}^2 a^2 + {\ad}^3 a)}{\lam}
		= \frac{1}{|\lam|^2} \frac{\hbar}{2 m \omega} \ev{(\ad a^3 + \ad a + 2 {\ad}^2 a^2 + {\ad}^3 a)}{\lam} \notag \\
		&= (\lam^2 + 1 + 2 |\lam|^2 + {\lam^*}^2) \frac{\hbar}{2 m \omega}
		= \left( 1 + 2 \left[ \Re(\lam)^2 + \Im(\lam)^2 \right] + 2 \left[ \Re(\lam)^2 - \Im(\lam)^2 \right] \right) \frac{\hbar}{2 m \omega} \notag \\
		&= [1 + 4 \Re(\lam)^2 ] \frac{\hbar}{2 m \omega} \label{evXsq},
	\end{align}
	where we have again used $[a, \ad] = 1$.  For $\ev{P^2}{\lam}$,
	\begin{align}
		\ev{P^2}{\lam} &= \frac{1}{|\lam|^2} \ev{\ad P^2 a}{\lam}
		= -\frac{1}{|\lam|^2} \frac{\hbar m \omega}{2} \ev{\ad ({\ad}^2 - \ad a - a \ad + a^2) a}{\lam} \notag \\
		&= -\frac{1}{|\lam|^2} \frac{\hbar m \omega}{2} \ev{({\ad}^2 a - {\ad}^2 a^2 - \ad a \ad a + \ad a^3}{\lam}
		= -\frac{1}{|\lam|^2} \frac{\hbar m \omega}{2} \ev{({\ad}^3 a - \ad a - 2 {\ad}^2 a^2 + \ad a^3}{\lam} \notag \\
		&= -({\lam^*}^2 - 1 - 2 |\lam|^2 + \lam^2) \frac{\hbar m \omega}{2}
		= \left( 1 + 2 \left[ \Re(\lam)^2 + \Im(\lam)^2 \right] - 2 [\Re(\lam)^2 - \Im(\lam)^2] 
\right) \frac{\hbar m \omega}{2} \notag \\
		&= [1 + 4 \Im(\lam)^2] \frac{\hbar m \omega}{2}. \label{evPsq}
	\end{align}
	
	From (1.4.51) in Sakurai, $\ev{(\DA)^2} = \ev{A^2} - \ev{A}^2$.  Then
	\beq
		\ev{(\DX)^2}{\lam} = \ev{X^2}{\lam} - \ev{X}{\lam}^2 = [1 + 4 \Re(\lam)^2 ] \frac{\hbar}{2 m \omega} - 4 \Re(\lam)^2 \frac{\hbar}{2 m \omega} = \frac{\hbar}{2 m \omega},
	\eeq
	where we have used \refeq{evX} and \refeq{evXsq}, and
	\beq
		\ev{(\DP)^2}{\lam} = \ev{P^2}{\lam} - \ev{P}{\lam}^2 = [1 + 4 \Im(\lam)^2 ] \frac{\hbar m \omega}{2} - 4 \Im(\lam)^2 \frac{\hbar m \omega}{2} = \frac{\hbar m \omega}{2},
	\eeq
	where we have used \refeq{evP} and \refeq{evPsq}.  Finally,
	\beq
		\ev{(\DX)^2}{\lam} \ev{(\DP)^2}{\lam} = \frac{\hbar^2}{4},
	\eeq
	which shows that the coherent state $\klam$ satisfies the minimum uncertainty relation.
\end{solution}


\newcommand{\psit}{\psi(t)}
\newcommand{\kpsit}{\ket{\psit}}
\newcommand{\Ut}{U(t)}
\newcommand{\lamt}{\lam e^{-i \omega t}}
\newcommand{\klamt}{\ket{\lamt}}
\newcommand{\blamt}{\bra{\lamt}}

\begin{problem} \label{timeevo}
	Starting from $\ket{\psi(0)} = \klam$ at $t = 0$, we let $\kpsit$ evolve in time.  What is the state $\kpsit$ for $t > 0$?
\end{problem}

\begin{solution}
	The Hamiltonian for the harmonic oscillator,
	\beqn \label{ham}
		H = \frac{P^2}{2 m} + \frac{m \omega^2 X^2}{2},
	\eeqn
	is time independent, so the time evolution operator $U(t)$ for the coherent state in general is given by
	\beq
		U(t) = \exp(-\frac{i H t}{\hbar}),
	\eeq
	which is (2.1.28) in Sakurai.  Rewriting $\klam$ in the power series representation,
	\beqn \label{pow}
		\klam = \exp(-\frac{|\lam|^2}{2}) \sum_{n=0}^\infty \frac{\lambda^n {\ad}^n}{n!} \ko
		= \exp(-\frac{|\lam|^2}{2}) \sum_{n=0}^\infty \frac{\lambda^n}{n!} \kn.
	\eeqn
	The time evolution operator $U(t)$ for an energy eigenket $\kn$ of the harmonic oscillator is given by
	\beq
		U(t) \kn = \exp(-\frac{i E_n t}{\hbar}) \kn
		= \exp[-i \left( n + \frac{1}{2} \right) \omega t] \kn
		= e^{-i n \omega t} e^{-i \omega t / 2},
	\eeq
	where $E_n$ are given by (2.3.9) in Sakurai.  Then, using \refeq{pow}, we have
	\begin{align}
		\kpsit &= \Ut \klam
		= \exp(-\frac{|\lam|^2}{2}) \sum_{n=0}^\infty \frac{\lambda^n}{n!} e^{-i n \omega t} e^{-i \omega t / 2} \kn
		= e^{-i \omega t / 2} \exp(-\frac{|\lam|^2}{2}) \sum_{n=0}^\infty \frac{(\lambda e^{-i \omega t})^n}{n!} \kn \notag \\
		&= e^{-i \omega t / 2} \klamt, \label{klamt}
	\end{align}
	where $\lambda e^{-i \omega t}$ is a complex number (albeit one that is changing in time).  Thus, $\klamt$ is another coherent state.
\end{solution}

\begin{problem}
	Compute $\ev{X}{\psit}$ and $\ev{P}{\psit}$, and their time derivatives $\dv*{\ev{X}}{t}$ and $\dv*{\ev{P}}{t}$.
\end{problem}

\begin{solution}
	From \refeq{klamt}, we have
	\begin{align*}
		\ev{X}{\psit} &= \blamt e^{i \omega t / 2} X e^{-i \omega t / 2} \klamt = \ev{X}{\lamt} = 2 \Re(\lam) \sqrt{\frac{\hbar}{2 m \omega}}, \\
		\ev{P}{\psit} &= \blam e^{i \omega t} P e^{-i \omega t} \klam = \ev{P}{\lam} = 2 \Im(\lam) \sqrt{\frac{\hbar m \omega}{2}},
	\end{align*}
	where we have used \refeq{evX} and \refeq{evP}.
	
	For the time derivatives, the harmonic oscillator Hamiltonian is given by \refeq{ham}.  Using the Ehrenfest theorem and the other results of problem~4.1 of the previous homework,
	\begin{align*}
		\dv{\ev{X}}{t} &= -\frac{i}{\hbar} \ev{[X, H]}{\psit} = \frac{1}{m} \ev{P}{\psit} = 2 \Im(\lam) \sqrt{\frac{\hbar \omega}{2 m}}, \\
		\dv{\ev{P}}{t} &= -\frac{i}{\hbar} \ev{[P, H]}{\psit} = - m \omega^2 \ev{X}{\psit} = -2 \Re(\lam) \sqrt{\frac{\hbar m \omega^3}{2}},
	\end{align*}
	which again are similar to the classical equations of motion.
\end{solution}

\newcommand{\klamp}{\ket{\lam'}}
\newcommand{\bkll}{\braket*{\lam''}{\lam'}}

\begin{problem}
	Compute $\mel*{\lambda''}{\exp(-i H t / \hbar)}{\lambda'}$.
\end{problem}

\begin{solution}
	Note that $\Ut = \exp(-i H t / \hbar)$ where $\Ut$ is the time evolution operator.  From problem \ref{timeevo},
	\beq
		\kpsit = \Ut \klam \implies \exp(-\frac{i H t}{\hbar}) \klamp = e^{-i \omega t} \klamp,
	\eeq
	so
	\beq
		\mel*{\lam''}{\exp(-\frac{i H t}{\hbar})}{\lam'} = e^{-i \omega t} \bkll.
	\eeq
	\hl{Using the power series representation,}
	\beq
		\klam = \exp(-\frac{|\lam|^2}{2}) \sum_{n=0}^\infty \frac{\lambda^n {\ad}^n}{n!} \ko = \exp(-\frac{|\lam|^2}{2}) \sum_{n=0}^\infty \frac{\lambda^n}{n!} \kn,
	\eeq
	so
	\beq
		\bkll = \exp(-\frac{|\lam''|^2}{2}) \exp(-\frac{|\lam'|^2}{2}) \sum_{n=0}^\infty \frac{({\lam''}^* \lam')^n}{n!} \braket{n} = \exp(-\frac{|\lam''|^2}{2} + {\lam''}^* \lam' - \frac{|\lam'|^2}{2}).
	\eeq
	Finally,
	\beq
		\mel*{\lam''}{\exp(-\frac{i H t}{\hbar})}{\lam'} = \exp(-i \omega t - \frac{|\lam''|^2}{2} + {\lam''}^* \lam' - \frac{|\lam'|^2}{2}).
	\eeq
\end{solution}