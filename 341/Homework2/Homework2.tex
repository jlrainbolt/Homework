\documentclass[10pt]{article}
\usepackage{geometry, titlesec}
\usepackage[parfill]{parskip}
\usepackage{physics, amsfonts, amsthm}
\usepackage{fullpage}
\usepackage{fancyhdr}
\usepackage{enumitem}
\usepackage{xcolor, soul}
%\allowdisplaybreaks

\makeatletter
\renewcommand*\env@cases[1][1.2]{%
  \let\@ifnextchar\new@ifnextchar
  \left\lbrace
  \def\arraystretch{#1}%
  \array{@{}l@{\quad}l@{}}%
}
\makeatother
 
 
\renewcommand{\footrulewidth}{.2pt}
\setlist[enumerate]{leftmargin=*}
\pagestyle{fancy}
\fancyhf{}
\lhead{\textbf{Physics 341 Homework 1}}
\rhead{Lacey Rainbolt}
\setlength{\headheight}{14pt}
\setlength{\headsep}{12pt}
\lfoot{\today}
\rfoot{\thepage}

\titleformat{\subsection}[runin]{\normalfont\large\bfseries}{\thesubsection}{1em}{}
\newcommand{\refeq}[1]{(\ref{#1})}


\newenvironment{statement}
{
    \color{darkgray}
    \ignorespaces
}
{
%    \smallskip
}

\newenvironment{problem}
{
    \color{darkgray}
    \subsection{}
    \ignorespaces
}


\newenvironment{solution}
{
    \paragraph{Solution.}
    \ignorespaces
}
{
    \bigskip
}

\newcommand{\Schrodinger}{Schr\"{o}dinger}


\setcounter{enumi}{2}
\begin{document}

\newcommand{\ihb}{i \hbar}
\newcommand{\partt}{\partial_t}
\newcommand{\partx}{\partial_x}

\newcommand{\bkPsix}{\braket{\Psi(t)}{x}}
\newcommand{\bkxPsi}{\braket{x}{\Psi(t)}}
\newcommand{\bPsi}{\bra{\Psi(t)}}
\newcommand{\kPsi}{\ket{\Psi(t)}}
\newcommand{\bx}{\bra{x}}
\newcommand{\kx}{\ket{x}}

\newcommand{\bkPhix}{\braket{\Phi(t)}{x}}
\newcommand{\bPhi}{\bra{\Phi(t)}}
\newcommand{\kPhi}{\ket{\Phi(t)}}

\section{Problem 3}
\begin{statement}
	Consider a particle moving in one dimension with the Hamiltonian
	\begin{equation} \label{hamiltonian}
		H = \frac{p^2}{2m} + V(x).
	\end{equation}
\end{statement}

\begin{problem}
	Verify the following:
	
	\renewcommand{\theenumi}{\alph{enumi}}
	\begin{enumerate}
		\item $\ihb \partt \bkPsix = - \mel{\Psi(t)}{H}{x}$,
		\item $\ihb \partt \bkPhix \bkxPsi = \bkPhix \mel{x}{H}{\Psi(t)} - \mel{\Phi(t)}{H}{x} \bkxPsi$,
		\item $\ihb \partt \bkPhix \bkxPsi = -\dfrac{\hbar^2}{2m} \left[ \bkPhix \partx^2 \bkxPsi - (\partx^2 \bkPhix) \bkxPsi \right]$,
		\item $\bkPhix \mel{x}{p}{\Psi(t)} + \mel{\Phi(t)}{p}{x} \bkxPsi = \dfrac{\hbar}{i} \left[ \bkPhix \partx \bkxPsi - (\partx \bkPhix) \bkxPsi \right]$
		\item $\dfrac{\hbar}{i} \partx \left[ \bkPhix \mel{x}{p}{\Psi(t)} + \mel{\Phi(t)}{p}{x} \bkxPsi \right] = \bkPhix \mel{x}{p^2}{\Psi(t)} - mel{\Phi(t)}{p^2}{x} \bkxPsi$
	\end{enumerate}
\end{problem}
	
\begin{solution}
	\renewcommand{\theenumi}{\alph{enumi}}
	\begin{enumerate} \label{a}
		\item Beginning with \Schrodinger's equation, note that
			\begin{align}
				\ihb \partt \kPsi &= H \kPsi \\
				\ihb \partt \bkxPsi &= \mel{x}{H}{\Psi(t)} \label{schrox} \\
				(\ihb \partt \bkxPsi)^* &= (\mel{x}{H}{\Psi(t)})^* \\
				-\ihb \partt \bkPsix &= \mel{\Psi(t)}{H}{x} \label{hermitianH} \\
				\ihb \partt \bkPsix &= -\mel{\Psi(t)}{H}{x}, \label{res1a}
			\end{align}
			where in going to \refeq{hermitianH} we have used the fact that $H$ is Hermitian, and \refeq{res1a} is what we sought to prove. \qed
			
		\item Beginning with what was proven in (a),
			\begin{align}
				\ihb \partt \bkPhix &= -\mel{\Phi(t)}{H}{x} \\
				\ihb (\partt \bkPhix) \bkxPsi &= -\mel{\Phi(t)}{H}{x} \bkxPsi. \label{term1}
			\end{align}
			From \refeq{schrox}, we can write
			\begin{equation}
				\bkPhix \ihb \partt \bkxPsi = \bkPhix \mel{x}{H}{\Psi(t)}. \label{term2}
			\end{equation}
			Adding \refeq{term1} and \refeq{term2} yields
			\begin{align}
				\bkPhix \ihb \partt \bkxPsi + \ihb (\partt \bkPhix) \bkxPsi &= \bkPhix \mel{x}{H}{\Psi(t)} - \mel{\Phi(t)}{H}{x} \bkxPsi \\
				\ihb \partt \bkPhix \bkxPsi &= \bkPhix \mel{x}{H}{\Psi(t)} - \mel{\Phi(t)}{H}{x} \bkxPsi, \label{res1b}
			\end{align}
			where in going to \refeq{res1b} we have used the product rule of differentiation.   \refeq{res1b} is what we sought to prove. \qed
			
		\item Using \refeq{hamiltonian}, note that:
			\begin{align}
				\mel{x}{H}{\Psi(t)} &= \bx \left[ \frac{p^2}{2m} + V(x) \right] \kPsi \\
				&= \frac{1}{2m} \mel{x}{p^2}{\Psi(t)} + \mel{x}{V(x)}{\Psi(t)} \\
				&= \frac{(-\ihb \partx)^2}{2m} \bkxPsi + V(x) \bkxPsi \label{commutation} \\
				&= -\frac{\hbar^2}{2m} \partx^2 \bkxPsi + V(x) \bkxPsi, \label{term1}
			\end{align}
			where in going to \refeq{commutation} we have used the fact that
			\begin{equation} \label{pbra}
				\mel{x}{p}{\Psi(x)} = -\ihb \partx \bkxPsi.
			\end{equation}
			Similarly, note that
			\begin{equation} \label{term2}
				\mel{\Phi(t)}{H}{x} = -\frac{\hbar^2}{2m} \partx^2 \bkPhix + V(x) \bkPhix
			\end{equation}
			where we have used
			\begin{equation} \label{pket}
				 \mel{\Phi(t)}{p}{x} = \ihb \partx \bkPhix.
			\end{equation}
			Making the substitutions \refeq{term1} and \refeq{term2} into what was proven in (b),
			\begin{align}
				\ihb \partt \bkPhix \bkxPsi &= \bkPhix \left[ -\frac{\hbar^2}{2m} \partx^2 \bkxPsi + V(x) \bkxPsi \right] \notag \\
    				&\phantom{=\ } - \left[ -\frac{\hbar^2}{2m} \partx^2 \bkPhix + V(x) \bkPhix \right] \bkxPsi \\
				&= -\frac{\hbar^2}{2m} \left[ \bkPhix \partx^2 \bkPhix - (\partx^2 \bkPhix) \bkxPsi \right] \notag \\
				&\phantom{=\ } + V(x) \bkPhix \bkxPsi - V(x) \bkPhix \bkxPsi \\
				&= -\dfrac{\hbar^2}{2m} \left[ \bkPhix \partx^2 \bkxPsi - (\partx^2 \bkPhix) \bkxPsi \right],
			\end{align}
			as we sought to prove. \qed
			
		\item Applying \refeq{pbra} and \refeq{pket} to the left-hand side of (d),
			\begin{align}
				\bkPhix \mel{x}{p}{\Psi(t)} + \mel{\Phi(t)}{p}{x} \bkxPsi &= \bkPhix (-\ihb \partx \bkxPsi) + (\ihb \partx \bkPhix) \bkxPsi \\
				&= \dfrac{\hbar}{i} \left[ \bkPhix \partx \bkxPsi - (\partx \bkPhix) \bkxPsi \right]
			\end{align}
			as we sought to prove. \qed
			
		\item Beginning with the first term of the left-hand side of the expression in (e), applying the product rule of differentiation yields
		\begin{equation}
			\partx (\bkPhix \mel{x}{p}{\Psi(t)}) = (\partx \bkPhix) \mel{x}{p}{\Psi(t)} + \bkPhix \partx \mel{x}{p}{\Psi(t)}
		\end{equation}
		Multiplying through by $\hbar/i$,
		\begin{align}
			\frac{\hbar}{i} \partx (\bkPhix \mel{x}{p}{\Psi(t)}) &= (-\ihb \partx \bkPhix) \mel{x}{p}{\Psi(t)} - \bkPhix \ihb \partx \mel{x}{p}{\Psi(t)} \\
			&= -\mel{\Phi(t)}{p}{x} \mel{x}{p}{\Psi(t)} + \bkPhix \mel{x}{p^2}{\Psi(t)}, \label{res1e}
		\end{align}
		where in going to \refeq{res1e} we have used \refeq{pbra} and \refeq{pket}.  Using a similar procedure for the second term of the left-hand side of (e),
		\begin{align}
			\frac{\hbar}{i} \partx (\mel{\Phi(t)}{p}{x} \bkxPsi) &= (-\ihb \partx \mel{\Phi(t)}{p}{x}) \bkxPsi - \mel{\Phi(t)}{p}{x} \ihb \partx \bkxPsi \\
			&= -\mel{\Phi(t)}{p^2}{x} \bkxPsi + \mel{\Phi(t)}{p}{x} \mel{x}{p}{\Psi(t)}. \label{res2e}
		\end{align}
		Adding the results of \refeq{res1e} and \refeq{res2e},
		\begin{align}
			\frac{\hbar}{i} \partx \left[ \bkPhix \mel{x}{p}{\Psi(t)} + \mel{\Phi(t)}{p}{x} \bkxPsi \right] &= \bkPhix \mel{x}{p^2}{\Psi(t)} - \mel{\Phi(t)}{p}{x} \mel{x}{p}{\Psi(t)} \notag \\
			&\phantom{=\ } + \mel{\Phi(t)}{p}{x} \mel{x}{p}{\Psi(t)} - \mel{\Phi(t)}{p^2}{x} \bkxPsi \\
			&= \bkPhix \mel{x}{p^2}{\Psi(t)} - \mel{\Phi(t)}{p^2}{x} \bkxPsi
		\end{align}
		as we sought to prove. \qed
	\end{enumerate}
\end{solution}

\begin{problem}
	Define
	\begin{align}
		\rho(x, t) &= \bkPhix \bkxPsi, \label{rhodef} \\
		J_x(x, t) &= \frac{1}{2m} \left[ \bkPhix \mel{x}{p}{\Psi(t)} + \mel{\Phi(t)}{p}{x} \bkxPsi \right]. \label{Jdef}
	\end{align}
	Show that $\rho(x, t) + \partx J_x(x, t) = 0.$
\end{problem}

\begin{solution}
	From \refeq{rhodef},
	\begin{equation} \label{rho1}
		\partt \rho(x, t) = \partt (\bkPhix \bkxPsi),
	\end{equation}
	and from what was proven in 1(c),
	\begin{align} 
		\partt ( \bkPhix \bkxPsi) &= -\frac{1}{\ihb} \left[ \bkPhix \partx^2 \bkxPsi - (\partx^2 \bkPhix) \bkxPsi \right] \\
		&= -\frac{1}{2m} \frac{i}{\hbar} \left[ \bkPhix \mel{x}{p^2}{\Psi(t)} - \mel{\Phi(t)}{p^2}{x} \bkxPsi \right], \label{psq}
	\end{align}
	where we have applied \refeq{pbra} and \refeq{pket} in going to \refeq{psq}.  Equating \refeq{rho1} and \refeq{psq},
	\begin{equation} \label{rhoexp}
		\partt \rho(x, t) = -\frac{1}{2m} \frac{i}{\hbar} \left[ \bkPhix \mel{x}{p^2}{\Psi(t)} - \mel{\Phi(t)}{p^2}{x} \bkxPsi \right].
	\end{equation}
	Beginning from \refeq{Jdef},
	\begin{align}
		\partx J_x(x, t) &= \frac{1}{2m} \partx \left[ \bkPhix \mel{x}{p}{\Psi(t)} + \mel{\Phi(t)}{p}{x} \bkxPsi \right] \\
		&= \frac{1}{2m} \frac{i}{\hbar} \left[ \bkPhix \mel{x}{p^2}{\Psi(t)} - \mel{\Phi(t)}{p^2}{x} \bkxPsi \right] \label{Jexp},
	\end{align}
	where in going to \refeq{Jexp} we have used what was proven in 1(e).  Summing \refeq{rhoexp} and \refeq{Jexp}, we have
	\begin{align}
		\partt \rho(x, t) + \partx J_x(x, t) &= -\frac{1}{2m} \frac{i}{\hbar} \left[ \bkPhix \mel{x}{p^2}{\Psi(t)} - \mel{\Phi(t)}{p^2}{x} \bkxPsi \right] \notag \\
		&\phantom{=\ } + \frac{1}{2m} \frac{i}{\hbar} \left[ \bkPhix \mel{x}{p^2}{\Psi(t)} - \mel{\Phi(t)}{p^2}{x} \bkxPsi \right] \\
		&= 0
	\end{align}
	as we sought to prove. \qed
\end{solution}


\end{document}