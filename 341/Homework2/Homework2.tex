\documentclass[10pt]{article}
\usepackage{geometry, titlesec}
\usepackage[parfill]{parskip}
\usepackage{physics, amsfonts, amsthm}
\usepackage{fullpage}
\usepackage{fancyhdr}
\usepackage{enumitem}
\usepackage{xcolor, soul}
%\allowdisplaybreaks

\makeatletter
\renewcommand*\env@cases[1][1.2]{%
  \let\@ifnextchar\new@ifnextchar
  \left\lbrace
  \def\arraystretch{#1}%
  \array{@{}l@{\quad}l@{}}%
}
\makeatother
 
 
\renewcommand{\footrulewidth}{.2pt}
\setlist[enumerate]{leftmargin=*}
\pagestyle{fancy}
\fancyhf{}
\lhead{\textbf{Physics 341 Homework 1}}
\rhead{Lacey Rainbolt}
\setlength{\headheight}{14pt}
\setlength{\headsep}{12pt}
\lfoot{\today}
\rfoot{\thepage}

\titleformat{\subsection}[runin]{\normalfont\large\bfseries}{\thesubsection}{1em}{}
\newcommand{\refeq}[1]{(\ref{#1})}


\newenvironment{statement}
{
    \color{darkgray}
    \ignorespaces
}
{
%    \smallskip
}

\newenvironment{problem}
{
    \color{darkgray}
    \subsection{}
    \ignorespaces
}


\newenvironment{solution}
{
    \paragraph{Solution.}
    \ignorespaces
}
{
    \bigskip
}

\newcommand{\Schrodinger}{Schr\"{o}dinger}


\setcounter{enumi}{2}
\begin{document}

\newcommand{\ihb}{i \hbar}
\newcommand{\partt}{\partial_t}
\newcommand{\partx}{\partial_x}

\newcommand{\bkPsix}{\braket{\Psi(t)}{x}}
\newcommand{\bkxPsi}{\braket{x}{\Psi(t)}}
\newcommand{\bPsi}{\bra{\Psi(t)}}
\newcommand{\kPsi}{\ket{\Psi(t)}}
\newcommand{\kx}{\ket{x}}

\newcommand{\bkPhix}{\braket{\Phi(t)}{x}}

\section{Problem 3}
\begin{statement}
	Consider a particle moving in one dimension with the Hamiltonian
	\begin{equation}
		H = \frac{p^2}{2m} + V(x).
	\end{equation}
\end{statement}

\begin{problem}
	Verify the following:
	
	\renewcommand{\theenumi}{\alph{enumi}}
	\begin{enumerate}
		\item $\ihb \partt \bkPsix = - \mel{\Psi(t)}{H}{x}$,
		\item $\ihb \partt \bkPhix \bkxPsi = \bkPhix \mel{x}{H}{\Psi(t)} - \mel{\Phi(t)}{H}{x} \bkxPsi$,
		\item $\ihb \partt \bkPhix \bkxPsi = -\dfrac{\hbar^2}{2m} \left( \bkPhix \partx^2 \bkxPsi - (\partx^2 \bkPhix) \bkxPsi \right)$,
		\item 
	\end{enumerate}
\end{problem}
	
\begin{solution}
	\renewcommand{\theenumi}{\alph{enumi}}
	\begin{enumerate} \label{a}
		\item Beginning with \Schrodinger's equation, note that
			\begin{align}
				\ihb \partt \kPsi &= H \kPsi \\
				\ihb \partt \bkxPsi &= \mel{x}{H}{\Psi(t)} \label{schrox} \\
				(\ihb \partt \bkxPsi)^* &= (\mel{x}{H}{\Psi(t)})^* \\
				-\ihb \partt \bkPsix &= \mel{\Psi(t)}{H}{x} \label{hermitianH} \\
				\ihb \partt \bkPsix &= -\mel{\Psi(t)}{H}{x},
			\end{align}
			where in going to \refeq{hermitianH} we have used the fact that $H$ is Hermitian. \qed
			
		\item Beginning with what was proven in (a),
			\begin{align}
				\ihb \partt \bkPhix &= -\mel{\Phi(t)}{H}{x} \\
				\ihb (\partt \bkPhix) \bkxPsi &= -\mel{\Phi(t)}{H}{x} \bkxPsi. \label{term1}
			\end{align}
			From \refeq{schrox}, we can write
			\begin{equation}
				\bkPhix \ihb \partt \bkxPsi = \bkPhix \mel{x}{H}{\Psi(t)}. \label{term2}
			\end{equation}
			Adding \refeq{term1} and \refeq{term2} yields
			\begin{align}
				\bkPhix \ihb \partt \bkxPsi + \ihb (\partt \bkPhix) \bkxPsi &= \bkPhix \mel{x}{H}{\Psi(t)} - \mel{\Phi(t)}{H}{x} \bkxPsi \\
				\ihb \partt \bkPhix \bkxPsi &= \bkPhix \mel{x}{H}{\Psi(t)} - \mel{\Phi(t)}{H}{x} \bkxPsi \label{product},
			\end{align}
			where in going to \refeq{product} we have used the product rule of differentiation. \qed
	\end{enumerate}
\end{solution}


\end{document}