\documentclass[10pt]{article}
\usepackage{geometry, titlesec}
\usepackage[parfill]{parskip}
\usepackage{physics, amsfonts, amsthm}
\usepackage{fullpage}
\usepackage{fancyhdr}
\usepackage{enumitem}
\usepackage{xcolor, soul}
%\allowdisplaybreaks

\makeatletter
\renewcommand*\env@cases[1][1.2]{%
  \let\@ifnextchar\new@ifnextchar
  \left\lbrace
  \def\arraystretch{#1}%
  \array{@{}l@{\quad}l@{}}%
}
\makeatother
 
 
\renewcommand{\footrulewidth}{.2pt}
\setlist[enumerate]{leftmargin=*}
\pagestyle{fancy}
\fancyhf{}
\lhead{\textbf{Physics 341 Homework 2}}
\rhead{Lacey Rainbolt}
\setlength{\headheight}{14pt}
\setlength{\headsep}{12pt}
\lfoot{\today}
\rfoot{\thepage}

\titleformat{\subsection}[runin]{\normalfont\large\bfseries}{\thesubsection}{1em}{}
\newcommand{\refeq}[1]{(\ref{#1})}


\newenvironment{statement}
{
    \color{darkgray}
    \ignorespaces
}
{
%    \smallskip
}

\newenvironment{problem}
{
    \color{darkgray}
    \subsection{}
    \ignorespaces
}


\newenvironment{solution}
{
    \paragraph{Solution.}
    \ignorespaces
}
{
    \bigskip
}

\newcommand{\Schrodinger}{Schr\"{o}dinger}


\setcounter{enumi}{2}
\begin{document}

\newcommand{\ihb}{i \hbar}
\newcommand{\partt}{\partial_t}
\newcommand{\partx}{\partial_x}

\newcommand{\bkPsix}{\braket{\Psi(t)}{x}}
\newcommand{\bkxPsi}{\braket{x}{\Psi(t)}}
\newcommand{\bPsi}{\bra{\Psi(t)}}
\newcommand{\kPsi}{\ket{\Psi(t)}}
\newcommand{\bx}{\bra{x}}
\newcommand{\kx}{\ket{x}}

\newcommand{\bkPhix}{\braket{\Phi(t)}{x}}
\newcommand{\bPhi}{\bra{\Phi(t)}}
\newcommand{\kPhi}{\ket{\Phi(t)}}

\section{Problem 3}
\begin{statement}
	Consider a particle moving in one dimension with the Hamiltonian
	\begin{equation} \label{hamiltonian}
		H = \frac{p^2}{2m} + V(x).
	\end{equation}
\end{statement}

\begin{problem}
	Verify the following:
	
	\renewcommand{\theenumi}{\alph{enumi}}
	\begin{enumerate}
		\item $\ihb \partt \bkPsix = - \mel{\Psi(t)}{H}{x}$,
		\item $\ihb \partt \bkPhix \bkxPsi = \bkPhix \mel{x}{H}{\Psi(t)} - \mel{\Phi(t)}{H}{x} \bkxPsi$,
		\item $\ihb \partt \bkPhix \bkxPsi = -\dfrac{\hbar^2}{2m} \left[ \bkPhix \partx^2 \bkxPsi - (\partx^2 \bkPhix) \bkxPsi \right]$,
		\item $\bkPhix \mel{x}{p}{\Psi(t)} + \mel{\Phi(t)}{p}{x} \bkxPsi = \dfrac{\hbar}{i} \left[ \bkPhix \partx \bkxPsi - (\partx \bkPhix) \bkxPsi \right]$
		\item $\dfrac{\hbar}{i} \partx \left[ \bkPhix \mel{x}{p}{\Psi(t)} + \mel{\Phi(t)}{p}{x} \bkxPsi \right] = \bkPhix \mel{x}{p^2}{\Psi(t)} - mel{\Phi(t)}{p^2}{x} \bkxPsi$
	\end{enumerate}
\end{problem}
	
\begin{solution}
	\renewcommand{\theenumi}{\alph{enumi}}
	\begin{enumerate} \label{a}
		\item Beginning with \Schrodinger's equation, note that
			\begin{align}
				\ihb \partt \kPsi &= H \kPsi \\
				\ihb \mel{x}{\partt}{\Psi(t)} &= \mel{x}{H}{\Psi(t)} \label{schrox} \\
				(\ihb \mel{x}{\partt}{\Psi(t)})^\dagger &= (\mel{x}{H}{\Psi(t)})^\dagger \\
				-\ihb \mel{\Psi(t)}{\partt}{x} &= \mel{\Psi(t)}{H}{x} \label{hermitianH} \\
				\ihb \partt \bkPsix &= -\mel{\Psi(t)}{H}{x}, \label{res1a}
			\end{align}
			where in going to \refeq{hermitianH} we are assuming that $H$ is Hermitian.  Note also that $\partt$ is Hermitian because $t$ is merely a parameter of the system.  \refeq{res1a} is what we sought to prove. \qed
			
		\item Rewriting what was proven in (a) with $\Psi \mapsto \Phi$ and then multiplying by $\Psi(x, t)$ on the right,
			\begin{align}
				\ihb \partt \bkPhix &= -\mel{\Phi(t)}{H}{x} \\
				\ihb (\partt \bkPhix) \bkxPsi &= -\mel{\Phi(t)}{H}{x} \bkxPsi. \label{term01}
			\end{align}
			Multiplying \refeq{schrox} by $\Phi^*(x, t)$ on the left,
			\begin{equation}
				\bkPhix \ihb \partt \bkxPsi = \bkPhix \mel{x}{H}{\Psi(t)}. \label{term02}
			\end{equation}
			Adding \refeq{term02} and \refeq{term01} yields
			\begin{align}
				\bkPhix \ihb \partt \bkxPsi + \ihb (\partt \bkPhix) \bkxPsi &= \bkPhix \mel{x}{H}{\Psi(t)} - \mel{\Phi(t)}{H}{x} \bkxPsi \\
				\ihb \partt \bkPhix \bkxPsi &= \bkPhix \mel{x}{H}{\Psi(t)} - \mel{\Phi(t)}{H}{x} \bkxPsi, \label{res1b}
			\end{align}
			where in going to \refeq{res1b} we have used the product rule of differentiation on the left-hand side.   \refeq{res1b} is what we sought to prove. \qed
			
		\item Using \refeq{hamiltonian}, note that:
			\begin{align}
				\mel{x}{H}{\Psi(t)} &= \bx \left[ \frac{p^2}{2m} + V(x) \right] \kPsi \\
				&= \frac{1}{2m} \mel{x}{p^2}{\Psi(t)} + \mel{x}{V(x)}{\Psi(t)} \\
				&= \frac{(-\ihb \partx)^2}{2m} \bkxPsi + V(x) \bkxPsi \label{commutation} \\
				&= -\frac{\hbar^2}{2m} \partx^2 \bkxPsi + V(x) \bkxPsi, \label{term1}
			\end{align}
			where in going to \refeq{commutation} we have (twice) used the fact that
			\begin{equation} \label{pbra}
				\mel{x}{p}{\Psi(x)} = -\ihb \partx \bkxPsi.
			\end{equation}
			Similarly, note that
			\begin{equation} \label{term2}
				\mel{\Phi(t)}{H}{x} = -\frac{\hbar^2}{2m} \partx^2 \bkPhix + V(x) \bkPhix
			\end{equation}
			where we have (twice) used the adjoint of \refeq{pbra} with $\Psi \mapsto \Phi$,
			\begin{equation} \label{pket}
				 \mel{\Phi(t)}{p}{x} = \ihb \partx \bkPhix.
			\end{equation}
			This follows because $p$ is Hermitian.  Making the substitutions \refeq{term1} and \refeq{term2} into what was proven in (b),
			\begin{align}
				\ihb \partt \bkPhix \bkxPsi &= \bkPhix \left[ -\frac{\hbar^2}{2m} \partx^2 \bkxPsi + V(x) \bkxPsi \right] \notag \\
    				&\phantom{=\ } - \left[ -\frac{\hbar^2}{2m} \partx^2 \bkPhix + V(x) \bkPhix \right] \bkxPsi \\
				&= -\frac{\hbar^2}{2m} \left[ \bkPhix \partx^2 \bkPhix - (\partx^2 \bkPhix) \bkxPsi \right] \notag \\
				&\phantom{=\ } + V(x) \bkPhix \bkxPsi - V(x) \bkPhix \bkxPsi \\
				&= -\dfrac{\hbar^2}{2m} \left[ \bkPhix \partx^2 \bkxPsi - (\partx^2 \bkPhix) \bkxPsi \right],
			\end{align}
			as we sought to prove. \qed
			
		\item Applying \refeq{pbra} and \refeq{pket} to the left-hand side of (d),
			\begin{align}
				\bkPhix \mel{x}{p}{\Psi(t)} + \mel{\Phi(t)}{p}{x} \bkxPsi &= \bkPhix (-\ihb \partx \bkxPsi) + (\ihb \partx \bkPhix) \bkxPsi \\
				&= \dfrac{\hbar}{i} \left[ \bkPhix \partx \bkxPsi - (\partx \bkPhix) \bkxPsi \right]
			\end{align}
			as we sought to prove. \qed
			
		\item Beginning with the first term of the left-hand side of the expression in (e), applying the product rule of differentiation yields
		\begin{equation}
			\partx (\bkPhix \mel{x}{p}{\Psi(t)}) = (\partx \bkPhix) \mel{x}{p}{\Psi(t)} + \bkPhix \partx \mel{x}{p}{\Psi(t)}
		\end{equation}
		Multiplying through by $\hbar/i$,
		\begin{align}
			\frac{\hbar}{i} \partx (\bkPhix \mel{x}{p}{\Psi(t)}) &= (-\ihb \partx \bkPhix) \mel{x}{p}{\Psi(t)} - \bkPhix \ihb \partx \mel{x}{p}{\Psi(t)} \\
			&= -\mel{\Phi(t)}{p}{x} \mel{x}{p}{\Psi(t)} + \bkPhix \mel{x}{p^2}{\Psi(t)}, \label{res1e}
		\end{align}
		where in going to \refeq{res1e} we have used \refeq{pbra} and \refeq{pket}.  Using a similar procedure for the second term of the left-hand side of (e),
		\begin{align}
			\frac{\hbar}{i} \partx (\mel{\Phi(t)}{p}{x} \bkxPsi) &= (-\ihb \partx \mel{\Phi(t)}{p}{x}) \bkxPsi - \mel{\Phi(t)}{p}{x} \ihb \partx \bkxPsi \\
			&= -\mel{\Phi(t)}{p^2}{x} \bkxPsi + \mel{\Phi(t)}{p}{x} \mel{x}{p}{\Psi(t)}. \label{res2e}
		\end{align}
		Adding the results of \refeq{res1e} and \refeq{res2e},
		\begin{align}
			\frac{\hbar}{i} \partx \left[ \bkPhix \mel{x}{p}{\Psi(t)} + \mel{\Phi(t)}{p}{x} \bkxPsi \right] &= \bkPhix \mel{x}{p^2}{\Psi(t)} - \mel{\Phi(t)}{p}{x} \mel{x}{p}{\Psi(t)} \notag \\
			&\phantom{=\ } + \mel{\Phi(t)}{p}{x} \mel{x}{p}{\Psi(t)} - \mel{\Phi(t)}{p^2}{x} \bkxPsi \\
			&= \bkPhix \mel{x}{p^2}{\Psi(t)} - \mel{\Phi(t)}{p^2}{x} \bkxPsi
		\end{align}
		as we sought to prove. \qed
	\end{enumerate}
\end{solution}

\begin{problem}
	Define
	\begin{align}
		\rho(x, t) &= \bkPhix \bkxPsi, \label{rhodef} \\
		J_x(x, t) &= \frac{1}{2m} \left[ \bkPhix \mel{x}{p}{\Psi(t)} + \mel{\Phi(t)}{p}{x} \bkxPsi \right]. \label{Jdef}
	\end{align}
	Show that $\rho(x, t) + \partx J_x(x, t) = 0.$
\end{problem}

\begin{solution}
	From \refeq{rhodef},
	\begin{equation} \label{rho1}
		\partt \rho(x, t) = \partt (\bkPhix \bkxPsi),
	\end{equation}
	and from what was proven in 1(c),
	\begin{align} 
		\partt ( \bkPhix \bkxPsi) &= -\frac{1}{\ihb} \left[ \bkPhix \partx^2 \bkxPsi - (\partx^2 \bkPhix) \bkxPsi \right] \\
		&= -\frac{1}{2m} \frac{i}{\hbar} \left[ \bkPhix \mel{x}{p^2}{\Psi(t)} - \mel{\Phi(t)}{p^2}{x} \bkxPsi \right], \label{psq}
	\end{align}
	where we have applied \refeq{pbra} and \refeq{pket} in going to \refeq{psq}.  Equating \refeq{rho1} and \refeq{psq},
	\begin{equation} \label{rhoexp}
		\partt \rho(x, t) = -\frac{1}{2m} \frac{i}{\hbar} \left[ \bkPhix \mel{x}{p^2}{\Psi(t)} - \mel{\Phi(t)}{p^2}{x} \bkxPsi \right].
	\end{equation}
	Beginning from \refeq{Jdef},
	\begin{align}
		\partx J_x(x, t) &= \frac{1}{2m} \partx \left[ \bkPhix \mel{x}{p}{\Psi(t)} + \mel{\Phi(t)}{p}{x} \bkxPsi \right] \\
		&= \frac{1}{2m} \frac{i}{\hbar} \left[ \bkPhix \mel{x}{p^2}{\Psi(t)} - \mel{\Phi(t)}{p^2}{x} \bkxPsi \right] \label{Jexp},
	\end{align}
	where in going to \refeq{Jexp} we have used what was proven in 1(e).  Summing \refeq{rhoexp} and \refeq{Jexp}, we have
	\begin{align}
		\partt \rho(x, t) + \partx J_x(x, t) &= -\frac{1}{2m} \frac{i}{\hbar} \left[ \bkPhix \mel{x}{p^2}{\Psi(t)} - \mel{\Phi(t)}{p^2}{x} \bkxPsi \right] \notag \\
		&\phantom{=\ } + \frac{1}{2m} \frac{i}{\hbar} \left[ \bkPhix \mel{x}{p^2}{\Psi(t)} - \mel{\Phi(t)}{p^2}{x} \bkxPsi \right] \\
		&= 0
	\end{align}
	as we sought to prove.  This is is the continuity equation for probability. \qed
\end{solution}

\newcommand{\Le}{L_3}
\newcommand{\Lz}{L_z}
\newcommand{\Py}{P_y}
\newcommand{\Px}{P_x}

\newcommand{\lei}{l_{3,i}}
\newcommand{\klei}{\ket{\lei}}

\newcommand{\Ud}{U^\dagger}
\newcommand{\Uii}{U_{ii}}

\newcommand{\Xpm}{X_\pm}
\newcommand{\Xp}{X_+}
\newcommand{\Xm}{X_-}
\newcommand{\Ypm}{Y_\pm}

\newcommand{\Cpm}{C_\pm}

\section{Problem 4}
\begin{statement}
	Consider a particle moving in three dimensions.  Consider an operator
	\begin{align} \label{defu}
		U(\phi) &= \exp(-\frac{i}{\hbar} \Le \phi), & \Le &= \Lz = X \Py - Y \Px,
	\end{align}
	where $X, Y$ and $\Px, \Py$ are position and momentum operators, respectively.  Define new operators
	\begin{align} \label{defxy}
		X(\phi) &= \Ud(\phi) X U(\phi), &
		Y(\phi) &= \Ud(\phi) Y U(\phi).
	\end{align}
	Note that $X(0) = Y(0) = 0$.
\end{statement}

\begin{problem}
	Derive the equation
	\begin{equation}
		\dv{X(\phi)}{\phi} = \frac{i}{\hbar} \Ud(\phi) [\Le, X] U(\phi) = -Y(\phi), \label{show2}
	\end{equation}
	and a similar equation for $\dv*{Y(\phi)}{\phi}$.
\end{problem}

\begin{solution}
	Using the definition of $X(\phi)$ in \refeq{defxy} and applying the product rule of differentiation,
	\begin{align}
		\dv{X(\phi)}{\phi} &= \dv{}{\phi} \left( \Ud X U \right) = \dv{\Ud}{\phi} X U + \Ud \dv{}{\phi} (X U) \\
		&= \dv{\Ud}{\phi} X U + \Ud \dv{X}{\phi} U + \Ud X \dv{U}{\phi}. \label{product}
	\end{align}
	We know immediately that $\dv*{X}{\phi} = 0$ because $\phi$ is not a parameter of the position operator $X$.  Let $\klei$ denote the eigenbasis of $\Le$ and $\lei$ its eigenvalues.  $\Le$ \hl{is Hermitian so an orthonormal basis is guaranteed to exist.  USE POWER SERIES INSTEAD.}  In this basis, $U(\phi)$ is diagonal and its nonzero matrix elements are given by
	\begin{equation} \label{diag}
		\Uii = \exp(-\frac{i}{\hbar} \lei \phi)
	\end{equation}
	which implies
	\begin{align}
		\dv{\Uii}{\phi} &= -\frac{i}{\hbar} \lei \exp(-\frac{i}{\hbar} \lei \phi) = -\exp(-\frac{i}{\hbar} \lei \phi) \frac{i}{\hbar} \lei \\
		&= -\frac{i}{\hbar} \lei \Uii = -\frac{i}{\hbar} \Uii \lei. \label{eigencomm}
	\end{align}
	The power-series representation of $e^{x}$ allows us to retrieve from \refeq{eigencomm} the operator relationships
	\begin{equation} \label{oper}
		\dv{U}{\phi} = -\frac{i}{\hbar} \Le U = -\frac{i}{\hbar} U \Le, \\
	\end{equation}
	which informs us that $[\Le, U] = 0$.  In a similar fashion, note that
	\begin{equation}
		\Ud = \exp(\frac{i}{\hbar} \Le \phi) \implies \dv{\Ud}{\phi} = \frac{i}{\hbar} \Le \exp(\frac{i}{\hbar} \Le \phi) = \frac{i}{\hbar} \Le \Ud = \frac{i}{\hbar} \Ud \Le
	\end{equation}
	and $[\Le, \Ud] = 0$ as well.  Then \refeq{product} becomes
	\begin{equation}
		\dv{X(\phi)}{\phi} = \frac{i}{\hbar} \Ud \Le X U - \frac{i}{\hbar} \Ud X \Le U = \frac{i}{\hbar} \Ud (\Le X - X \Le) U = \frac{i}{\hbar} \Ud(\phi) [\Le, X] U(\phi), \label{middle}
	\end{equation}
	which is the first equality of what we wanted to show in \refeq{show2}.
	
	From the definition of $\Le$ in \refeq{defu},
	\begin{align}
		[\Le, X] &= \Le X - X \Le = (X \Py - Y \Px) X - X (X \Py - Y \Px) \\
		&= X \Py X - Y \Px X - X X \Py + X Y \Px = Y X \Px - Y \Px X \label{[X, Py]} \\
		&= Y [X, \Px] = \ihb Y \label{[X, Px]}
	\end{align}
	where in \refeq{[X, Py]} we have used $[X, \Py] = [X, Y] = 0$, and in \refeq{[X, Px]} we have used $[X, \Px] = \ihb$.  Making the substitution \refeq{[X, Px]} into \refeq{middle}, we have
	\begin{equation}
		\dv{X(\phi)}{\phi} = \frac{i}{\hbar} \Ud(\phi) (\ihb Y) U(\phi) = -\Ud(\phi) Y U(\phi) = -Y(\phi),
	\end{equation}
	where the last equality is from the definition of $Y(\phi)$ in \refeq{defxy}.  This is the second equality of what we wanted to show in \refeq{show2}, which completes the proof.
	
	For $\dv*{Y(\phi)}{\phi}$, we can make the substitutions $X(\phi) \mapsto Y(\phi), X \mapsto Y$ in \refeq{product} and \refeq{middle} to obtain
	\begin{equation} \label{middleY}
		\dv{Y(\phi)}{\phi} = \frac{i}{\hbar} \Ud(\phi) [\Le, Y] U(\phi).
	\end{equation}
	Then making similar use of commutators $[Y, \Px] = [X, Y] = 0$ and $[Y, \Py] = \ihb$ as for \refeq{[X, Py]} and \refeq{[X, Px]},
	\begin{align}
		[\Le, Y] &= \Le Y - Y \Le = (X \Py - Y \Px) Y - Y (X \Py - Y \Px) \\
		&= X \Py Y - Y \Px Y - Y X \Py + Y Y \Px = X \Py Y - X Y \Py \\
		&= X [\Py, Y] = -X [Y, \Py] = -\ihb X. \label{[Y, Py]}
	\end{align}
	Substituting \refeq{[Y, Py]} into \refeq{middleY},
		\begin{equation}
		\dv{Y(\phi)}{\phi} = \frac{i}{\hbar} \Ud(\phi) (-\ihb X) U(\phi) = X(\phi),
	\end{equation}
	and so we have derived
	\begin{equation} \label{show2Y}
		\dv{Y(\phi)}{\phi} = \frac{i}{\hbar} \Ud(\phi) [\Le, Y] U(\phi) = X(\phi).
	\end{equation}
	and \refeq{show2} as desired. \qed
\end{solution}

\begin{problem}
	Define $\Xpm(\phi) = X(\phi) \pm i Y(\phi)$.  From the results of previous parts, show $\Xp(\phi) = e^{i\phi} \Xp$ where $\Xp = \Xp(0)$.  Derive the similar expression for $\Xm(\phi)$.
\end{problem}

\begin{solution}
	Differentiating $\Xpm(\phi)$ and making use of \refeq{show2} and \refeq{show2Y},
	\begin{align}
		\dv{\Xpm(\phi)}{\phi} &= \dv{X(\phi)}{\phi} \pm i \dv{Y(\phi)}{\phi} = -Y(\phi) \pm i X(\phi) = \pm i \left[ X(\phi) \pm i Y(\phi) \right] \\
		&= \pm i \Xpm(\phi). \label{expform}
	\end{align}
	The differential equation \refeq{expform} has solutions given by exponential functions of $\pm i \phi$.  We will make the ansatz
	\begin{equation} \label{ansatz}
		\Xpm(\phi) = e^{\pm i\phi} \Cpm,
	\end{equation}
	where $\Cpm$ is an operator ``constant'' in $\phi$ (that is, independent of it) and is fixed by an initial condition.  Inspecting \refeq{ansatz}, clearly $\Xpm(0) = \Cpm$ where it is defined $\Xpm(0) \equiv \Xpm$.  All that remains is to show that \refeq{ansatz} obeys the relation \refeq{expform}, as follows:
	\begin{equation}
		\dv{\Xpm(\phi)}{\phi} = \dv{}{\phi} \left( e^{\pm i\phi} \right) \Cpm = \pm i e^{\pm i\phi} \Cpm = \pm i \Xpm(\phi).
	\end{equation}
	Thus, we have derived
	\begin{align}
		\Xp(\phi) &= e^{i\phi} \Xp, &
		\Xm(\phi) &= e^{-i\phi} \Xm
	\end{align}
	as desired. \qed
\end{solution}

\begin{problem}
	Show that $[\Le, \Xp] = \hbar \Xp$.  Derive the similar expression for $[\Le, \Xm]$.
\end{problem}

\begin{solution}
	Firstly, note that
	\begin{equation}
		\Xpm = \Xpm(0) = X(0) \pm i Y(0) = \Ud(0) X U(0) \pm i \Ud(0) Y U(0) = X \pm i Y
	\end{equation}
	because $U(0) = \Ud(0) = I$.  Also applying the definition of $\Le$ in \refeq{defu}, we have
	\begin{align}
		[\Le, \Xpm] &= [X \Py - Y \Px, X \pm i Y] = (X \Py - Y \Px) (X \pm i Y) - (X \pm i Y) (X \Py - Y \Px) \\
		&= X \Py X \pm i X \Py Y - Y \Px X \mp i Y \Px Y - X X \Py + X Y \Px \mp i Y X \Py \pm i Y Y \Px \\
		&= \pm i X \Py Y - Y \Px X + X Y \Px \mp i Y X \Py = \pm i X [\Py, Y] + Y [X, \Px] \\
		&= \pm \hbar X + \ihb Y = \pm \hbar [X \pm i Y] = \pm \hbar \Xpm.
	\end{align}
	Thus, we have shown
	\begin{align}
		[\Le, \Xp] &= \hbar \Xp, & [\Le, \Xm] &= -\hbar \Xm
	\end{align}
	as desired. \qed
\end{solution}
\end{document}