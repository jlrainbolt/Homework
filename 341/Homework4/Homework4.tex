\documentclass[11pt]{article}
\usepackage{geometry, titlesec}
\usepackage[parfill]{parskip}
\usepackage[italicdiff]{physics}
\usepackage{amsfonts, amsthm}
\usepackage[cm]{fullpage}
\usepackage{fancyhdr}
\usepackage{enumitem}
\usepackage{xcolor, soul}
\usepackage{kbordermatrix}
\allowdisplaybreaks

\makeatletter
\renewcommand*\env@cases[1][1.2]{%
  \let\@ifnextchar\new@ifnextchar
  \left\lbrace
  \def\arraystretch{#1}%
  \array{@{}l@{\quad}l@{}}%
}
\makeatother
 
 
\renewcommand{\footrulewidth}{.2pt}
%\setlist[enumerate]{leftmargin=*}
\pagestyle{fancy}
\fancyhf{}
\lhead{\textbf{Physics 341 Homework 4}}
\rhead{Lacey Rainbolt}
\setlength{\headheight}{11pt}
\setlength{\headsep}{11pt}
\setlength{\footskip}{24pt}
\lfoot{\today}
\rfoot{\thepage}

\titleformat{\subsection}[runin]{\normalfont\large\bfseries}{\thesubsection}{1em}{}
\newcommand{\refeq}[1]{(\ref{#1})}

\newcommand{\beq}{\begin{equation*}}
\newcommand{\eeq}{\end{equation*}}

\newcommand{\beqn}{\begin{equation}}
\newcommand{\eeqn}{\end{equation}}

\renewcommand{\vec}[1]{\mathbf{#1}}


\newenvironment{statement}
{
    \color{darkgray}
    \ignorespaces
}
{
%    \smallskip
}

\newenvironment{problem}
{
    \color{darkgray}
    \subsection{}
    \ignorespaces
}


\newenvironment{solution}
{
    \paragraph{Solution.}
    \ignorespaces
}
{
%    \smallskip
}

\newcommand{\Schrodinger}{Schr\"{o}dinger}


\begin{document}

\state{(Jackson 9.8)}{\ 
	%\emph{Hint:} The electromagnetic angular momentum density comes from more than the transverse (radiation zone) components of the fields.
}

%
%	Jackson 9.8(a)
%

\prob{}{
	Show that a classical oscillating electric dipole $\vp$ with fields given by
	\aln{ \label{fields1}
		\vH &= \frac{c k^2}{4\pi} (\nh \cross \vp) \frac{e^{i k r}}{r} \paren{ 1 - \frac{1}{i k r} }, &
		\vE &= \frac{1}{4\pi \epso} \curly{ k^2 (\nh \cross \vp) \cross \nh \frac{e^{i k r}}{r} + [ 3 \nh (\nh \vdot \vp) - \vp ] \paren{ \frac{1}{r^3} - \frac{i k}{r^2} } e^{i k r} },
	}
	radiates electromagnetic angular momentum to infinity at the rate
	\eq{
		\dv{\vL}{t} = \frac{k^3}{12 \pi \epso} \Im[ \vp^* \cross \vp ].
	}
	\vfix
}

\sol{
	According to Jackson~(9.20), the time-averaged angular momentum density is
	\eq{
		\vl = \frac{\Re[ \vx \cross (\vE \cross \vHs)}{2 c^2}.
	}
	One of the vector identities on the inside cover of Jackson is $\vaa \cross (\vbb \cross \vcc) = (\vaa \vdot \vcc) \vbb - (\vaa \vdot \vbb) \vcc$, so
	\eqn{l1}{
		\vl = \frac{(\vx \vdot \vHs) \vE - (\vx \vdot \vE) \vHs}{2 c^2}.
	}
	From Eq.~\refeq{fields1}, note that
	\eq{
		\vx \vdot \vHs \propto \vx \vdot (\nh \cross \vps)
		= \vps \vdot (\vx \cross \nh)
		= \vO,
	}
	where we have used the identity $\vaa \vdot (\vbb \cross \vcc) = \vcc \vdot (\vaa \cross \vbb)$ and the fact that $\nh$ points in the $\vx$ direction.  For $\vx \vdot \vE$, note that
	\al{
		\vx \vdot [ (\nh \cross \vp) \cross \nh ] &= -\vx \vdot [ \nh \cross (\nh \cross \vp) ]
		= -\vx \vdot [ (\nh \vdot \vp) \nh - (\nh \vdot \nh) \vp ]
		= -(\nh \vdot \vp) (\vx \vdot \nh) + \vx \vdot \vp \\
		&= -r (\nh \vdot \vp) + \vx \vdot \vp
		= \vx \vdot \vp - \vx \vdot \vp
		= 0, \\[1.5ex]
		\vx \vdot [ 3 \nh (\nh \vdot \vp) - \vp ] &= 3 (\vx \vdot \nh) (\nh \vdot \vp) - \vx \vdot \vp
		= 3r (\nh \vdot \vp) - \vx \vdot \vp
		= 3(\vx \vdot \vp) - \vx \vdot \vp
		= 2(\vx \vdot \vp),
	}
	since $\abs{\vx} = r$ and $\vx = r \,\nh$.  Then
	\eq{
		\vx \vdot \vE = \frac{1}{2\pi \epso} (\vx \vdot \vp) \paren{ \frac{1}{r^3} - \frac{i k}{r^2} } e^{i k r}
		= \frac{1}{2\pi \epso} (\nh \vdot \vp) \paren{ \frac{1}{r^2} - \frac{i k}{r} } e^{i k r}.
	}
	
	With these substitutions, Eq.~\refeq{l1} becomes
	\al{
		\vl &= -\frac{(\vx \vdot \vE) \vHs}{c^2}
		= -\frac{1}{4\pi \epso c^2} (\nh \vdot \vp) \paren{ \frac{1}{r^2} - \frac{i k}{r} } e^{i k r} \frac{c k^2}{4\pi} (\nh \cross \vps) \frac{e^{-i k r}}{r} \paren{ 1 + \frac{1}{i k r} } \\
		&= -\frac{k^2}{16\pi^2 \epso c r} (\nh \vdot \vp) (\nh \cross \vps) \paren{ \frac{1}{r^2} - \frac{i k}{r} } \paren{ 1 - \frac{i}{k r} }
		= -\frac{k^2}{16\pi^2 \epso c} (\nh \vdot \vp) (\nh \cross \vps) \paren{ \frac{1}{r^2} - \frac{i}{k r^3} - \frac{i k}{r} - \frac{1}{r^2} } \\
		&= -\frac{i k^2}{16\pi^2 \epso c r} (\nh \vdot \vp) (\nh \cross \vps) \paren{ \frac{1}{k r^3} + \frac{k}{r^2} }
		= \frac{i k^3}{16\pi^2 \epso c r^2} (\nh \vdot \vp) (\nh \cross \vps) \paren{ \frac{1}{k^2 r^2} + 1 }.
	}
	
	Let $\vL$ be the angular momentum radiated to a distance $R$.  Then
	\eq{
		\vL = \int_R \vl(r) \ddcx
		= \intopi \intotp \intoR \vl(r) \,r^2 \sin\tht \ddr \ddphi \dd\tht,
	}
	and the time derivative is
	\aln{
		\dv{\vL}{t} &= \dv{t}(\intopi \intotp \intoR \vl(r) \,r^2 \sin\tht \ddr \ddphi \dd\tht)
		= \dv{r}{t} \dv{r}(\intopi \intotp \intoR \vl(r) \,r^2 \sin\tht \ddr \ddphi \dd\tht) \notag \\
		&= c \intopi \intotp \vl(r) \,r^2 \sin\tht \ddphi \dd\tht
		= \frac{i k^3}{16\pi^2 \epso} \paren{ \frac{1}{k^2 r^2} + 1 } \intopi \intotp (\nh \vdot \vp) (\nh \cross \vps) \sin\tht \ddphi \dd\tht. \label{dLdt}
	}
	Note that
	\eq{
		[ (\nh \vdot \vp) (\nh \cross \vps) ]_i = \sumje n_j p_j (\nh \cross \vps)_i
		= \sumje \sumke \sumle \epsikl n_j p_j n_k p_l^*,
	}
	so
	\eq{
		\dv{L_i}{t} \propto \sumje \sumke \sumle \epsikl p_j p_l^* \int n_j p_k \ddOmg
		= \sumje \sumke \sumle \epsikl p_j p_l^* \frac{4\pi}{3} \del_{jk}
		= \frac{4\pi}{3} \epsikl p_k p_l^*
		= \frac{4\pi}{3} (\vp \cross \vps)_i,
	}
	where we have used Jackson~(9.47), $\int n_\bet n_\gam \ddOmg = 4\pi \del_{\bet \gam} / 3$.  Making this substitution into Eq.~\refeq{dLdt},
	\eq{
		\dv{\vL}{t} = \frac{i k^3}{6\pi \epso} \paren{ \frac{1}{k^2 r^2} + 1 } (\vp \cross \vps).
	}
	Taking the limit as $r \to \infty$, we find
	\eqn{ans1a}{
		\dv{\vL}{t} = \Re\!\brac{ \frac{i k^3}{12\pi \epso} (\vp \cross \vps) }
		= \Re\!\brac{ -\frac{i k^3}{12\pi \epso} (\vps \cross \vp) }
		= \ans{ \frac{k^3}{12\pi \epso} \Im[ \vps \cross \vp ], }
	}
	as desired. \qed
}

%
%	Jackson 9.8(b)
%

\prob{}{
	What is the ratio of angular momentum radiated to energy radiated?  Interpret.
}

\sol{
	According to Jackson~(9.24), the total power radiated by an oscillating electric dipole $\vp$ is
	\eq{
		P = \dv{E}{t}
		= \frac{c^2 \Zo k^4}{12 \pi} \abs{\vp}^2.
	}
	Then the ratio of angular momentum radiated to energy radiated is
	\eq{
		\frac{\dv*{\vL}{t}}{\dv*{E}{t}} = \frac{k^3}{12\pi \epso} \Im[ \vps \cross \vp ] \frac{12 \pi}{c^2 \Zo k^4 \abs{\vp}^2}
		= \frac{1}{\epso} \Im[ \vps \cross \vp ] \frac{1}{c^2 \Zo k \abs{\vp}^2}
		= \ans{ \frac{\Im[ \vps \cross \vp ]}{\omg \abs{\vp}^2}, }
	}
	where we have used $\Zo = \sqrt{\muo / \epso} = 1 / \sqrt{\epso^2 c^2} = 1 / \epso c$, $c^2 = 1 / (\epso \muo)$, and $\omg = k c$.
	
	In the limit of high frequency, $(\dv*{\vL}{t}) / (\dv*{E}{t}) \to 0$.  In this scenario, the energy radiated dominates over the angular momentum radiated.  Likewise, in the limit of low frequency, $(\dv*{\vL}{t}) / (\dv*{E}{t}) \to \infty$, meaning that angular momentum radiation dominates.  This is sensible because rotational kinetic energy $E \propto \omg^2$, while angular momentum $L \propto \omg$.
}

%
%	Jackson 9.8(c)
%

\prob{}{
	For a charge $e$ rotating in the $xy$ plane at radius $a$ and angular speed $\omg$, show that there is only a $z$ component of radiated angular momentum with magnitude $\dv*{\Lz}{t} = e^2 k^3 a^2 / 6 \pi \epso$.  What about a charge oscillating along the $z$ axis?
}

\sol{
	We know from Homework~5 that the position of a point charge rotating counterclockwise in the $xy$ plane is
	\eq{
		\vx(t) = a \cos(\omg t) \,\vx + a \sin(\omg t) \,\yh.
	}
	\clearpage
	Then the charge distribution is
	\eq{
		\rho(\vx, t) = e \del[ x - a \cos(\omg t) ] \,\del[ y - a \sin(\omg t) ] \,\del(z).
	}
	
	According to Jackson~(4.8), the dipole moment is defined
	\eq{
		\vp = \int \vx' \,\rho(\vx') \ddcxp.
	}
	The components of $\vp$ for the point charge are then
	\al{
		\px &= e \iiint x \,\del[ x - a \cos(\omg t) ] \,\del[ y - a \sin(\omg t) ] \,\del(z) \ddx \ddy \ddz
		= e a \cos(\omg t), \\
		\py &= e \iiint y \,\del[ x - a \cos(\omg t) ] \,\del[ y - a \sin(\omg t) ] \,\del(z) \ddx \ddy \ddz
		= e a \sin(\omg t), \\
		\pz &= e \iiint z \,\del[ x - a \cos(\omg t) ] \,\del[ y - a \sin(\omg t) ] \,\del(z) \ddx \ddy \ddz
		= 0,
	}
	so we can write $\vp = e a \,e^{-i \omg t} (\xh + i\,\yh).$  Substituting into Eq.~\refeq{ans1a},
	\al{
		\dv{\vL}{t} &= \Re\!\brac{ \frac{i k^3}{12\pi \epso} e^2 a^2 e^{-i \omg t} e^{i \omg t} [ (\xh + i\,\yh) \cross (\xh - i\,\yh) ] }
		= \Re\!\brac{ \frac{i e^2 k^3 a^2}{12\pi \epso} (-2i \,\xh \cross \yh) }
		= \Re\!\brac{ \frac{e^2 k^3 a^2}{6\pi \epso} \,\zh } \\
		&= \ans{ \frac{e^2 k^3 a^2}{6\pi \epso} \cos(\omg t) \,\zh, }
	}
	as desired. \qed
	
	A charge oscillating along the $z$ axis with amplitude $a$ has the charge density
	\eq{
		\rho(\vx, t) = e a \,\del(x) \,\del(y) \,\del[ z - \cos(\omg t) ],
	}
	which gives the dipole moment
	\al{
		\px &= e a \iiint x \,\del(x) \,\del(y) \,\del[ z - \cos(\omg t) ] \ddx \ddy \ddz
		= 0, \\
		\py &= e a \iiint y \,\del(x) \,\del(y) \,\del[ z - \cos(\omg t) ] \ddx \ddy \ddz
		= 0, \\
		\pz &= e a \iiint z \,\del(x) \,\del(y) \,\del[ z - \cos(\omg t) ] \ddx \ddy \ddz
		= e a \cos(\omg t).
	}
	In complex notation, $\vp = e a \,e^{-i\omg t} \,\zh$.  Substituting into Eq.~\refeq{ans1a}, we find
	\eq{
		\dv{\vL}{t} = \Re\!\brac{ \frac{i k^3}{12\pi \epso} e^2 a^2 e^{-i \omg t} e^{i \omg t} (\zh \cross \zh) }
		= \ans{ \vO. }
	}
	So we see that a charge undergoing linear motion does not lead to a radiated angular momentum, which is sensible.
}

%
%	Jackson 9.8(d)
%

\prob{}{
	What are the results corresponding to Probs.~{1(a)} and {1(b)} for magnetic dipole radiation?
}

\sol{
	The radiation fields for a magnetic dipole are given by Jackson~(19.35--36),
	\al{
		\vH &= \frac{1}{4\pi} \curly{ k^2 (\nh \cross \vm) \cross \nh \frac{e^{i k r}}{r} + [ 3 \nh (\nh \vdot \vm) - \vm ] \paren{ \frac{1}{r^3} - \frac{i k}{r^2} } e^{i k r} }, &
		\vE &= -\frac{\Zo}{4\pi} k^2 (\nh \cross \vm) \frac{e^{i k r}}{r} \paren{ 1 - \frac{1}{i k r} }.
	}
	\clearpage
	Comparing with Eq.~\refeq{fields1}, we see that $\vH \to -\vE / \Zo$, $\vE \to \Zo \vH$, and $\vp \to \vm / c$ as stated in the book~\cite[p.~413]{Jackson}.  Making these substitutions, the results of Probs.~{1.1(a)} and {(b)} become
	\al{
		\ans{ \dv{\vL}{t}\ }&\ans{= \frac{\muo k^3}{12\pi} \Im[ \vms \cross \vm ], } &
		\ans{ \frac{\dv*{\vL}{t}}{\dv*{E}{t}}\ }&\ans{= \frac{\Im[ \vms \cross \vm ]}{\omg \abs{\vm}^2} }
	}
	where we have used $\mu = 1 / \epso c^2$.
}


\newcommand{\vn}{\vec{n}}
\newcommand{\nx}{n_x}
\newcommand{\ny}{n_y}
\newcommand{\nz}{n_z}

\newcommand{\sig}{\sigma}
\newcommand{\vsig}{\boldsymbol{\sig}}
\newcommand{\sigx}{\sig_x}
\newcommand{\sigy}{\sig_y}
\newcommand{\sigz}{\sig_z}
\newcommand{\nds}{\vn \cdot \vsig}

\newcommand{\lam}{\lambda}
\newcommand{\lamq}{\lam_1}
\newcommand{\lamw}{\lam_2}
\newcommand{\lampmq}{\lam_{\pm1}}
\newcommand{\lampmw}{\lam_{\pm2}}
\newcommand{\klamq}{\ket{\lamq}}
\newcommand{\klamw}{\ket{\lamw}}

\section{Problem 2}
\begin{problem}
	Consider $\nds$, where $\vn$ is a three-dimensional unit vector and $\vsig = (\sigx, \sigy, \sigz)$ represents the Pauli matrices.  Compute the eigenvalues $\lamq, \lamw$ and the corresponding eigenvectors $\klamq, \klamw$ of $\nds$.  Use them to obtain the spectrum decomposition of $\nds$.
\end{problem}

\begin{solution}
	From (3.2.32) in Sakurai, the Pauli matrices are
	\begin{align*}
		\sigx &= \mqty[ 0 & 1 \\ 1 & 0 ], &
		\sigy &= \mqty[ 0 & -i \\ i & 0 ], &
		\sigz &= \mqty[ 1 & 0 \\ 0 & -1 ].
	\end{align*}
	Let $\vn = (\nx, \ny, \nz)$.  Then
	\beq
		\nds = \mqty[ 0 & \nx \\ \nx & 0 ] + i \mqty[ 0 & -\ny \\ \ny & 0 ] + \mqty[ \nz & 0 \\ 0 & -\nz ]
		= \mqty[ \nz & \nx - i \ny \\ \nx + i \ny & -\nz ].
	\eeq
	The eigenvalues of $\nds$ are the solutions to the characteristic polynomial equation
	\beq
		0 = \det(\nds - \lam I) = \mdet{\nz - \lam & \nx - i \ny \\ \nx + i \ny & -(\nz + \lam)} = -(\nz - \lam) (\nz + \lam) - (\nx - i \ny) (\nx + i \ny) = \lam^2 - \nx^2 - \ny^2 - \nz^2.
	\eeq
	Since $|\vn|^2 = \nx^2 + \ny^2 + \nz^2$, we have $\lam = \pm |\vn| = \pm 1$.  Let $\lamq = 1$ and $\lamw = -1$.
	
	For the eigenvectors, let the elements of $\klamq$ be $\lam_{+1}, \lam_{+2}$ and the elements of $\klamw$ be $\lam_{-1}, \lam_{-2}$.  Then
	\beq
		\mqty[ \nz & \nx - i \ny \\ \nx + i \ny & -\nz ] \mqty[ \lampmq \\ \lampmw ] = \pm \mqty[ \lampmq \\ \lampmw ],
	\eeq
	which is equivalent to the system of equations
	\begin{align*}
		\nz \lampmq + (\nx - i \ny) \lampmw &= \pm\lampmq, &
		(\nx + i \ny) \lampmq - \nz \lampmw &= \pm\lampmw.
	\end{align*}
%	We may let $\lampmw = 1$ without loss of generality.  Then $\lampmq = (\nz \pm 1) / (\nx + i \ny)$, so
%	\begin{align*}
%		\klamq &= \mqty[ (\nz + 1) / (\nx + i \ny) \\ 1 ], &
%		\klamw &= \mqty[ (\nz - 1) / (\nx + i \ny) \\ 1 ].
%	\end{align*}
	We may let $\lampmw = \nx + i \ny$ without loss of generality.  Then $\lampmq = \nz \pm 1$, so
	\begin{align*}
		\klamq &= \mqty[ \nz + 1 \\ \nx + i \ny ], &
		\klamw &= \mqty[ \nz - 1 \\ \nx + i \ny ].
	\end{align*}
\end{solution}



\newcommand{\up}{\uparrow}
\newcommand{\dn}{\downarrow}
\newcommand{\kpsi}{\ket{\psi}}
\newcommand{\kup}{\ket{\uparrow}}
\newcommand{\kdn}{\ket{\downarrow}}
\newcommand{\cq}{c_1}
\newcommand{\cw}{c_2}
\newcommand{\cqs}{c_1^*}
\newcommand{\cws}{c_2^*}
\newcommand{\Sz}{S_z}

\section{Problem 3}
\begin{statement}
	Consider a spin $1/2$ state $\kpsi = \cq \kup + \cw \kdn$, where $\kup$ and $\kdn$ are the $\Sz$ eigenstates with eigenvalues $+\hbar/2$ and $-\hbar/2$, respectively.
\end{statement}

\begin{problem}
	Consider the operator $\rho = \ketbra{\psi}$.  Write down the matrix elements of $\rho$ is the basis $\{ \kup, \kdn \}$.
\end{problem}

\begin{solution}
	From the definition of $\kpsi$,
	\begin{align*}
		\braket{\up}{\psi} &= \cq, &
		\braket{\psi}{\up} &= \cqs, &
		\braket{\dn}{\psi} &= \cw, &
		\braket{\psi}{\dn} &= \cws.
	\end{align*}
	Using these,
	\begin{align*}
		\mel{\up}{\rho}{\up} &= \braket{\up}{\psi} \braket{\psi}{\up} = \cq \cqs = |\cq|^2, &
		\mel{\up}{\rho}{\dn} &= \braket{\up}{\psi} \braket{\psi}{\dn} = \cq \cws, \\
		\mel{\dn}{\rho}{\up} &= \braket{\dn}{\psi} \braket{\psi}{\up} = \cqs \cw, &
		\mel{\dn}{\rho}{\dn} &= \braket{\dn}{\psi} \braket{\psi}{\dn} = \cw \cws = |\cw|^2.
	\end{align*}
\end{solution}

\newcommand{\soo}{s_0}
\newcommand{\sq}{s_1}
\newcommand{\sw}{s_2}
\newcommand{\se}{s_3}
\newcommand{\vs}{\vec{s}}

\begin{problem}
	In the $\Sz$ eigenbasis, express $\rho$ by using the Pauli matrices.  That is, write $\rho$ as
	\beq
		\rho = \frac{\soo}{2} I + \frac{1}{2} \vs \cdot \vsig,
	\eeq
	and express $\soo, \sq, \sw, \se$ in terms of $\cq$ and $\cw$.
\end{problem}



\vfill
While writing up these solutions, I consulted Sakurai's \emph{Modern Quantum Mechanics} and Shankar's \emph{Principles of Quantum Mechanics}.

\end{document}