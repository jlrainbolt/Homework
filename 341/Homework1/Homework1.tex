\documentclass[11pt]{article}
\usepackage{geometry}
\usepackage[parfill]{parskip}
\usepackage{physics, amsfonts, amsthm}
\usepackage{fullpage}
\usepackage{fancyhdr}
\usepackage{enumitem}
\usepackage{xcolor, soul}
%\allowdisplaybreaks

\makeatletter
\renewcommand*\env@cases[1][1.2]{%
  \let\@ifnextchar\new@ifnextchar
  \left\lbrace
  \def\arraystretch{#1}%
  \array{@{}l@{\quad}l@{}}%
}
\makeatother
 
 
\setlist[enumerate]{leftmargin=*}
\pagestyle{fancy}
\fancyhf{}
\lhead{\textbf{Physics 316 Homework 1}}
\rhead{Lacey Rainbolt}
\setlength{\headheight}{14pt}
\setlength{\headsep}{12pt}

\newcommand{\pder}[2]{\frac{\partial#1}{\partial#2}}
\newcommand{\pmder}[3]{\frac{\partial^2#1}{\partial#2 \, \partial#3}}
\newcommand{\der}[2]{\frac{d#1}{d#2}}
\newcommand{\refeq}[1]{(\ref{#1})}


\newenvironment{statement}
{
    \color{darkgray}
    \ignorespaces
}

\newenvironment{problem}
{
    \color{darkgray}
    \subsection{}
    \ignorespaces
}


\newenvironment{solution}
{
    \paragraph{Solution.}
    \ignorespaces
}
{
    \bigskip\bigskip
}



\begin{document}

\newcommand{\keq}{\ket{e_1}}
\newcommand{\beq}{\bra{e_1}}
\newcommand{\kew}{\ket{e_2}}
\newcommand{\bew}{\bra{e_2}}
\newcommand{\kee}{\ket{e_3}}
\newcommand{\bee}{\bra{e_3}}

\section{Problem 1}
\begin{statement}
	Consider operators $J$ and $K$ acting in a three-dimensional space as
	\begin{align}
		J \keq &= i \kew, & J \kew &= -i \keq, & J \kee &= 0, \\
		K \keq &= 0, & K \kew &= i \kee, & K \kee &= -i \kew,
	\end{align}
	where $\keq, \kew, \kee$ for a complete orthonormal basis.
\end{statement}

\begin{problem}
	Compute the matrix elements of $J$ and $K$.
\end{problem}
	
\begin{solution} The matrix elements of $J$ are
	\begin{align}
		J_{11} &= \mel{e_1}{J}{e_1} = i \braket{e_1}{e_2} = 0, \label{J11} \\
		J_{12} &= \mel{e_1}{J}{e_2} = -i \braket{e_1}{e_1} = -i, \\
		J_{13} &= \mel{e_1}{J}{e_3} = 0, \\
		J_{21} &= \mel{e_2}{J}{e_1} = i \braket{e_2}{e_2} = i, \\
		J_{22} &= \mel{e_2}{J}{e_2} = -i \braket{e_2}{e_1} = 0, \\
		J_{23} &= \mel{e_2}{J}{e_3} = 0, \\
		J_{31} &= \mel{e_3}{J}{e_1} = i \braket{e_3}{e_2} = 0, \\
		J_{32} &= \mel{e_3}{J}{e_2} = -i \braket{e_3}{e_1} = 0, \\
		J_{33} &= \mel{e_3}{J}{e_3} = 0.
	\end{align}
	
	The matrix elements of $K$ are
	\begin{align}
		K_{11} &= \mel{e_1}{K}{e_1} = 0, \\
		K_{12} &= \mel{e_1}{K}{e_2} = i \braket{e_1}{e_3} = 0, \\
		K_{13} &= \mel{e_1}{K}{e_3} = -i \braket{e_1}{e_2} = 0, \\
		K_{21} &= \mel{e_2}{K}{e_1} = 0, \\
		K_{22} &= \mel{e_2}{K}{e_2} = i \braket{e_2}{e_3} = 0, \\
		K_{23} &= \mel{e_2}{K}{e_3} = -i \braket{e_2}{e_2} = -i, \\
		K_{31} &= \mel{e_3}{K}{e_1} = 0, \\
		K_{32} &= \mel{e_3}{K}{e_2} = i \braket{e_3}{e_3} = i, \\
		K_{33} &= \mel{e_3}{K}{e_3} = -i \braket{e_3}{e_2} = 0. \label{J33}
	\end{align}
\end{solution}

\begin{problem}
	Consider $O = AJ + BK$ where $A, B$ are real numbers.  Show that $O$ is Hermitian.
\end{problem}

\begin{solution}
	Using \refeq{J11}--\refeq{K33}, the matrix elements of $O$ are
	\begin{align}
		O_{11} &= 0, \\
		O_{12} &= -iA, \\
		O_{13} &= 0, \\
		O_{21} &= iA, \\
		O_{22} &= 0, \\
		O_{23} &= -iA, \\
		O_{31} &= 0, \\
		O_{32} &= iB, \\
		O_{33} &= 0.
	\end{align}
\end{solution}


\end{document}