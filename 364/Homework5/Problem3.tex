\state{Weyl curvature tensor~(MCP 25.12)}{
	Show that the Weyl curvature tensor~(25.48) has vanishing contraction on all its slots and has the same symmetries as Riemann: Eqs.~(25.45).  From these properties, show that Weyl has just 10 independent components.  Write the Riemann tensor in terms of the Weyl tensor, the Ricci tensor, and the scalar curvature.
}

\sol{
	MCP~(25.48) is
	\eqn{weyl}{
		\fv{C}{\mu \nu}{\rho \sig} = \fv{R}{\mu \nu}{\rho \sig} - 2 \fv{\sg}{[ \mu}{[ \rho} \fv{R}{\nu ]}{\sig ]} + \frac{1}{3} \fv{\sg}{[ \mu}{[ \rho} \fv{\sg}{\nu ]}{\sig ]} R,
	}
	where the square brackets denote antisymmetrization, $A_{[ \alp \bet ]} = (A_{\alp \bet} - A_{\bet \alp}) / 2$, and $R = \fv{R}{\alp}{\alp}$.  Lowering the indices and expanding out the antisymmetrizations gives us
	\aln{
		C_{\mu \nu \rho \sig} &= R_{\mu \nu \rho \sig} - \paren{ \sg_{\mu \ [ \rho} R_{\nu \ \sig ]} - \sg_{\nu \ [ \rho} R_{\mu \ \sig ]} } + \frac{1}{6} \paren{ \sg_{\mu \ [ \rho} \sg_{\nu \ \sig ]} - \sg_{\nu \ [ \rho} \sg_{\mu \ \sig ]} } R \notag \\
		&= R_{\mu \nu \rho \sig} - \frac{1}{2} \paren{ \sg_{\mu \rho} R_{\nu \sig} - \sg_{\mu \sig} R_{\nu \rho} - \sg_{\nu \rho} R_{\mu \sig} + \sg_{\nu \sig} R_{\mu \rho} } + \frac{1}{12} \paren{ \sg_{\mu \rho} \sg_{\nu \sig} - \sg_{\mu \sig} \sg_{\nu \rho} - \sg_{\nu \rho} \sg_{\mu \sig} + \sg_{\nu \sig} \sg_{\mu \rho} } R. \label{lowC}
	}
	
	We begin by interchanging indices since it makes the other proofs easier.  Interchanging the first two indices,
	\al{
		C_{\nu \mu \rho \sig} &= R_{\nu \mu \rho \sig} - \frac{1}{2} \paren{ \sg_{\nu \rho} R_{\mu \sig} - \sg_{\nu \sig} R_{\mu \rho} - \sg_{\mu \rho} R_{\nu \sig} + \sg_{\mu \sig} R_{\nu \rho} } + \frac{1}{12} \paren{ \sg_{\nu \rho} \sg_{\mu \sig} - \sg_{\nu \sig} \sg_{\mu \rho} - \sg_{\mu \rho} \sg_{\nu \sig} + \sg_{\mu \sig} \sg_{\nu \rho} } R \\
		&= -\brac{ R_{\mu \nu \rho \sig} - \frac{1}{2} \paren{ \sg_{\mu \rho} R_{\nu \sig} - \sg_{\mu \sig} R_{\nu \rho} - \sg_{\nu \rho} R_{\mu \sig} + \sg_{\nu \sig} R_{\mu \rho} } + \frac{1}{12} \paren{ \sg_{\mu \rho} \sg_{\nu \sig} - \sg_{\mu \sig} \sg_{\nu \rho} - \sg_{\nu \rho} \sg_{\mu \sig} + \sg_{\nu \sig} \sg_{\mu \rho} } R } \\
		&= - C_{\mu \nu \rho \sig},
	}
	where we have used $R_{\alp \bet \gam \del} = -R_{\bet \alp \gam \del}$ from MCP~(25.45a).
	
	Interchanging the last two indices,
	\al{
		C_{\mu \nu \sig \rho} &= R_{\mu \nu \sig \rho} - \frac{1}{2} \paren{ \sg_{\mu \sig} R_{\nu \rho} - \sg_{\mu \rho} R_{\nu \sig} - \sg_{\nu \sig} R_{\mu \rho} + \sg_{\nu \rho} R_{\mu \sig} } + \frac{1}{12} \paren{ \sg_{\mu \sig} \sg_{\nu \rho} - \sg_{\mu \rho} \sg_{\nu \sig} - \sg_{\nu \sig} \sg_{\mu \rho} + \sg_{\nu \rho} \sg_{\mu \sig} } R \\
		&= -\brac{ R_{\mu \nu \rho \sig} - \frac{1}{2} \paren{ \sg_{\mu \rho} R_{\nu \sig} - \sg_{\mu \sig} R_{\nu \rho} - \sg_{\nu \rho} R_{\mu \sig} + \sg_{\nu \sig} R_{\mu \rho} } + \frac{1}{12} \paren{ \sg_{\mu \rho} \sg_{\nu \sig} - \sg_{\mu \sig} \sg_{\nu \rho} - \sg_{\nu \rho} \sg_{\mu \sig} + \sg_{\nu \sig} \sg_{\mu \rho} } R } \\
		&= - \fv{C}{\mu \nu}{\rho \sig},
	}
	where we have used $R_{\alp \bet \gam \del} = -R_{\alp \bet \del \gam}$ from MCP~(25.45a).
	
	Interchanging the first and second pair of indices,
	\al{
		C_{\rho \sig \mu \nu} &= R_{\rho \sig \mu \nu} - \frac{1}{2} \paren{ \sg_{\rho \mu} R_{\sig \nu} - \sg_{\rho \nu} R_{\sig \mu} - \sg_{\sig \mu} R_{\rho \nu} + \sg_{\sig \nu} R_{\rho \mu} } + \frac{1}{12} \paren{ \sg_{\rho \mu} \sg_{\sig \nu} - \sg_{\rho \nu} \sg_{\sig \mu} - \sg_{\sig \mu} \sg_{\rho \nu} + \sg_{\sig \nu} \sg_{\rho \mu} } R \\
		&= R_{\mu \nu \rho \sig} - \frac{1}{2} \paren{ \sg_{\mu \rho} R_{\nu \sig} - \sg_{\mu \sig} R_{\nu \rho} - \sg_{\nu \rho} R_{\mu \sig} + \sg_{\nu \sig} R_{\mu \rho} } + \frac{1}{12} \paren{ \sg_{\mu \rho} \sg_{\nu \sig} - \sg_{\mu \sig} \sg_{\nu \rho} - \sg_{\nu \rho} \sg_{\mu \sig} + \sg_{\nu \sig} \sg_{\mu \rho} } R \\
		&= \fv{C}{\mu \nu}{\rho \sig},
	}
	where we have used the symmetry of the metric, $\fv{\sg}{\mu}{\nu} = \fv{\sg}{\nu}{\mu}$, and $R_{\alp \bet \gam \del} = +R_{\gam \del \alp \bet}$ from MCP~(25.45a).

	Note also that
	\al{
		C_{\mu \rho \sig \nu} &= R_{\mu \rho \sig \nu} - \frac{1}{2} \paren{ \sg_{\mu \sig} R_{\rho \nu} - \sg_{\mu \nu} R_{\rho \sig} - \sg_{\rho \sig} R_{\mu \nu} + \sg_{\rho \nu} R_{\mu \sig} } + \frac{1}{12} \paren{ \sg_{\mu \sig} \sg_{\rho \nu} - \sg_{\mu \nu} \sg_{\rho \sig} - \sg_{\rho \sig} \sg_{\mu \nu} + \sg_{\rho \nu} \sg_{\mu \sig} } R, \\
		%
		C_{\mu \sig \nu \rho} &= R_{\mu \sig \nu \rho} - \frac{1}{2} \paren{ \sg_{\mu \nu} R_{\sig \rho} - \sg_{\mu \rho} R_{\sig \nu} - \sg_{\sig \nu} R_{\mu \rho} + \sg_{\sig \rho} R_{\mu \nu} } + \frac{1}{12} \paren{ \sg_{\mu \nu} \sg_{\sig \rho} - \sg_{\mu \rho} \sg_{\sig \nu} - \sg_{\sig \nu} \sg_{\mu \rho} + \sg_{\sig \rho} \sg_{\mu \nu} } R.
	}
	Then for the equivalent to MCP~(25.45b),
	\al{
		C_{\mu \nu \rho \sig} + C_{\mu \rho \sig \nu} + C_{\mu \sig \nu \rho}
		&= R_{\mu \nu \rho \sig} + R_{\mu \rho \sig \nu} + R_{\mu \sig \nu \rho} - \frac{1}{2} \paren{ \sg_{\mu \rho} R_{\nu \sig} - \sg_{\mu \sig} R_{\nu \rho} - \sg_{\nu \rho} R_{\mu \sig} + \sg_{\nu \sig} R_{\mu \rho} } \\
		&\hspace{5em} \phantom{=} - \frac{1}{2} \paren{ \sg_{\mu \sig} R_{\rho \nu} - \sg_{\mu \nu} R_{\rho \sig} - \sg_{\rho \sig} R_{\mu \nu} + \sg_{\rho \nu} R_{\mu \sig} } \\
		&\hspace{5em} \phantom{=} - \frac{1}{2} \paren{ \sg_{\mu \nu} R_{\sig \rho} - \sg_{\mu \rho} R_{\sig \nu} - \sg_{\sig \nu} R_{\mu \rho} + \sg_{\sig \rho} R_{\mu \nu} } \\
		&\hspace{5em} \phantom{=} + \frac{1}{12} \paren{ \sg_{\mu \rho} \sg_{\nu \sig} - \sg_{\mu \sig} \sg_{\nu \rho} - \sg_{\nu \rho} \sg_{\mu \sig} + \sg_{\nu \sig} \sg_{\mu \rho} } R \\
		&\hspace{5em} \phantom{=} + \frac{1}{12} \paren{ \sg_{\mu \sig} \sg_{\rho \nu} - \sg_{\mu \nu} \sg_{\rho \sig} - \sg_{\rho \sig} \sg_{\mu \nu} + \sg_{\rho \nu} \sg_{\mu \sig} } R \\
		&\hspace{5em} \phantom{=} + \frac{1}{12} \paren{ \sg_{\mu \nu} \sg_{\sig \rho} - \sg_{\mu \rho} \sg_{\sig \nu} - \sg_{\sig \nu} \sg_{\mu \rho} + \sg_{\sig \rho} \sg_{\mu \nu} } R \\
		&= 0,
	}
	where we have used $R_{\alp \bet \gam \del} + R_{\alp \gam \del \bet} + R_{\alp \del \bet \gam} = 0$ from MCP~(25.45b), and the symmetry of the metric tensor.  Thus we have shown that the Weyl curvature tensor has the same symmetries as Riemann in MCP~(25.45): \vfix
	\ans{\aln{ \label{Csymm}
		C_{\nu \mu \rho \sig} &= -C_{\mu \nu \rho \sig}, &
		C_{\mu \nu \sig \rho} &= -C_{\mu \nu \rho \sig}, &
		C_{\rho \sig \mu \nu} &= +C_{\mu \nu \rho \sig}, &
		C_{\mu \nu \rho \sig} + C_{\mu \rho \sig \nu} + C_{\mu \sig \nu \rho} &= 0.
	}}%
	Now for the contractions.  Contracting the indices within each pair,
	\aln{ \label{thingy2}
		\sg^{\mu \nu} C_{\mu \nu \rho \sig} &= \sfvt{C}{\mu}{\mu}{\rho \sig}
		= -\sfvt{C}{\mu}{\mu}{\rho \sig}
		= 0, &
		%
		\sg^{\rho \sig} C_{\mu \nu \rho \sig} &= \sfv{C}{\mu \nu \rho}{\rho}
		= -\sfv{C}{\mu \nu \rho}{\rho}
		= 0,
	}
	where we have used $C_{\nu \mu \rho \sig} = -C_{\mu \nu \rho \sig}$ and our ability to swap covariant and contravariant for summed indices.
	
	Contracting the first and third indices,
	\al{
		\sg^{\mu \rho} C_{\mu \nu \rho \sig} &= \sg^{\mu \rho} \brac{ R_{\mu \nu \rho \sig} - \frac{1}{2} \paren{ \sg_{\mu \rho} R_{\nu \sig} - \sg_{\mu \sig} R_{\nu \rho} - \sg_{\nu \rho} R_{\mu \sig} + \sg_{\nu \sig} R_{\mu \rho} } + \frac{1}{12} \paren{ \sg_{\mu \rho} \sg_{\nu \sig} - \sg_{\mu \sig} \sg_{\nu \rho} - \sg_{\nu \rho} \sg_{\mu \sig} + \sg_{\nu \sig} \sg_{\mu \rho} } R } \\
		&= \sfvt{R}{\mu \nu}{\mu}{\sig} - \frac{1}{2} \paren{ \sfv{\sg}{\mu}{\mu} R_{\nu \sig} - \sg_{\mu \sig} \sfv{R}{\nu}{\mu} - \sfv{\sg}{\nu}{\mu} R_{\mu \sig} + \sg_{\nu \sig} \sfv{R}{\mu}{\mu} } + \frac{1}{12} \paren{ \sfv{\sg}{\mu}{\mu} \sg_{\nu \sig} - \sg_{\mu \sig} \sfv{\sg}{\nu}{\mu} - \sfv{\sg}{\nu}{\mu} \sg_{\mu \sig} + \sg_{\nu \sig} \sfv{\sg}{\mu}{\mu} } R \\
		&= R_{\nu \sig} - \frac{1}{2} \paren{ 4 R_{\nu \sig} - R_{\nu \sig} - R_{\nu \sig} + \sg_{\nu \sig} R } + \frac{1}{12} \paren{ 4 \sg_{\nu \sig} - \sg_{\nu \sig} - \sg_{\nu \sig} + 4 \sg_{\nu \sig} } R \\
		&= R_{\nu \sig} - R_{\nu \sig} - \frac{1}{2} \sg_{\nu \sig} R + \frac{1}{2} \sg_{\nu \sig} R \\
		&= 0,
	}
	where we have used MCP~(24.10), $\sg^{\mu \bet} \sg_{\bet \nu} = \fv{\del}{\mu}{\nu}$, to find $\fv{\sg}{\mu}{\mu} = 4$.  Then, using Eq.~\refeq{Csymm}, we have
	\eq{
		0 = \sfvt{C}{\mu \nu}{\mu}{\sig}
		= -\sfv{C}{\mu \nu \sig}{\mu}
		= -\sfvt{C}{\nu \mu}{\mu}{\sig}
		= \sfv{C}{\nu \mu \sig}{\mu}.
	}
	Rewriting these as contractions, and incorporating our results in Eq.~\refeq{thingy2}, we have shown
	\eq{
		\ans{ 0 = \sg^{\mu \nu} C_{\mu \nu \rho \sig}
		= \sg^{\mu \rho} C_{\mu \nu \rho \sig}
		= \sg^{\mu \sig} C_{\mu \nu \rho \sig}
		= \sg^{\nu \rho} C_{\mu \nu \rho \sig}
		= \sg^{\nu \sig} C_{\mu \nu \rho \sig}
		= \sg^{\rho \sig} C_{\mu \nu \rho \sig}; }
	}
	that is, the Weyl curvature tensor has vanishing contraction on all its slots.
	
	To show that Weyl has just 10 independent components, we refer to the discussion at the end of Lecture~11.  If there were no symmetries, in $n = 4$ spacetime dimensions $C_{\alp \bet \gam \del}$ would have $n^4 = 4^4$ independent components.  Each symmetry gives rise to some number of independent components:
	\al{
		\alp \lrarrow \bet \antisymmetry &\qimplies \frac{n (n - 1)}{2} = 6 \components, \\
		\gam \lrarrow \del \antisymmetry &\qimplies \frac{n (n - 1)}{2} = 6 \components, \\
		\alp \bet \lrarrow \gam \del \symmetry &\qimplies \text{components reduced by } \frac{1}{2}.
	}
	Combining the three (anti)symmetries, we have a total of
	\eq{
		\paren{ \frac{n (n - 1)}{2} } \paren{ \frac{n (n - 1)}{2} + 1 } \frac{1}{2} = 21 \components.
	}
	Now we account for the constraints imposed by $C_{\mu \nu \rho \sig} + C_{\mu \rho \sig \nu} + C_{\mu \sig \nu \rho} = 0$.  This expression is redundant if any of the indices are the same; for example,
	\eq{
		C_{\mu \mu \rho \sig} + C_{\mu \rho \sig \mu} + C_{\mu \sig \mu \rho}
		= 0 + C_{\sig \mu \mu \rho} + C_{\mu \sig \mu \rho}
		= C_{\sig \mu \mu \rho} - C_{\sig \mu \mu \rho}
		= 0,
	}
	which we already knew from the other symmetries.  Since there are four indices, this requirement gives us
	\eq{
		{ n \choose 4 } = 1 \text{ constraint}.
	}
	Finally we account for the constraints imposed by the vanishing contractions.  Only one of the contractions is not redundant under the other symmetries; say, $\fv{C}{\mu}{\nu \mu \sig} = 0$.  This imposes~\cite[p.~146]{Weinberg}
	\eq{
		\frac{n (n + 1)}{2} = 10 \constraints.
	}
	Finally, we find the number of independent components by
	\eq{
		N_\text{components} - N_\text{constraints} = 21 - (1 + 10) = \ans{ 10 \text{ independent components} }
	}
	as we wanted to show. \qed
	
	Rewriting Eq.~\refeq{weyl},
	\eq{
		\ans{ \fv{R}{\mu \nu}{\rho \sig} = \fv{C}{\mu \nu}{\rho \sig} - 2 \fv{\sg}{[ \mu}{[ \rho} \fv{R}{\nu ]}{\sig ]} + \frac{1}{3} \fv{\sg}{[ \mu}{[ \rho} \fv{\sg}{\nu ]}{\sig ]} R, }
	}
	which makes it clear that 10 of the 20 independent components of the Riemann tensor come from the Weyl tensor and the remaining 10 come from the Ricci tensor.
}