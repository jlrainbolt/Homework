\documentclass[11pt]{article}
\usepackage{homework}

\classname{364}
\homeworknum{5}


\DeclareMathAlphabet{\mathsfit}{T1}{\sfdefault}{\mddefault}{\sldefault}



\begin{document}

% Environments

\newcommand{\state}[2]{\begin{statement}{#1} #2 \end{statement}}
\newcommand{\prob}[2]{\begin{problem}{#1} #2 \end{problem}}
\newcommand{\subprob}[1]{\begin{subproblem} #1 \end{subproblem}}
\newcommand{\sol}[1]{\begin{solution} #1 \end{solution}}
\newcommand{\fig}[2]{\begin{figure} \centering #2  \label{#1} \end{figure}}

\newcommand{\makebib}{
	\vfill
	\color{black}
	\bibliography{references}{}
	\bibliographystyle{lucas_unsrt}
}
	

% Implication

\newcommand{\qwhere}{\quad \text{where} \quad}
\newcommand{\qimplies}{\quad \implies \quad}
\newcommand{\impliesq}{\implies \quad}



% Brackets

\newcommand{\paren}[1]{\left( #1 \right)}
\newcommand{\brac}[1]{\left[ #1 \right]}


% Greek

\newcommand{\alp}{\alpha}
\newcommand{\bet}{\beta}
\newcommand{\gam}{\gamma}
\newcommand{\del}{\delta}
\newcommand{\eps}{\epsilon}
\newcommand{\zet}{\zeta}
\newcommand{\tht}{\theta}
\newcommand{\kap}{\kappa}
\newcommand{\lam}{\lambda}
\newcommand{\sig}{\sigma}
\newcommand{\ups}{\upsilon}
\newcommand{\omg}{\omega}

\newcommand{\Gam}{\Gamma}
\newcommand{\Del}{\Delta}
\newcommand{\Tht}{\Theta}
\newcommand{\Lam}{\Lambda}
\newcommand{\Sig}{\Sigma}
\newcommand{\Omg}{\Omega}
% Problem 1

\newcommand{\Psii}{\Psi^i}
\newcommand{\Psiix}{\Psii(x)}

\newcommand{\Pii}{\Pi^i}

\newcommand{\Phii}{\Phi^i}
\newcommand{\Phiix}{\Phii(x)}
\newcommand{\PhiN}{\Phi^N}
\newcommand{\PhiNx}{\PhiN(x)}
\newcommand{\Phiq}{\Phi^1}
\newcommand{\Phiw}{\Phi^2}

\newcommand{\ddcx}{\dd[3]{x}}

\newcommand{\delij}{\del^{i j}}
\newcommand{\delkl}{\del^{k l}}
\newcommand{\delil}{\del^{i l}}
\newcommand{\deljk}{\del^{j k}}
\newcommand{\delik}{\del^{i k}}
\newcommand{\deljl}{\del^{j l}}

\newcommand{\DF}{D_F}

\newcommand{\sigx}{\sig(x)}

\newcommand{\pii}{\pi^i}
\newcommand{\pij}{\pi^j}
\newcommand{\pik}{\pi^k}
\newcommand{\pil}{\pi^l}
\newcommand{\piix}{\pi(x)}

\newcommand{\pq}{p_1}
\newcommand{\pw}{p_2}
\newcommand{\pe}{p_3}
\newcommand{\pr}{p_4}

\newcommand{\vp}{\vb{p}}
\newcommand{\vpsi}{\vp_i}

\newcommand{\mpi}{m_\pi}



\state{Ocean tides~(MCP 25.9)}{\hfix}

\prob{
	Place a local Lorentz frame at the center of Earth, and let $\cEsjk$ be the tidal field there, produced by the Newtonian gravitational fields of the Sun and the Moon.  For simplicity, treat Earth as precisely spherical.  Show that the gravitational acceleration (relative to Earth's center) at some location on or near Earth's surface (radius $r$) is
	\eq{
		\gsj = -\frac{G M}{r^2} \nj - \cEjsk r \nk,
	}
	where $M$ is Earth's mass, and $\nj$ is a unit vector pointing from Earth's center to the location at which $\gsj$ is evaluated.
}



\prob{
	Show that this gravitational acceleration is minus the gradient of the Newtonian potential
	\eq{
		\Phi = -\frac{G M}{r} + \frac{1}{2} \cEsjk r^2 \nj \nk.
	}
}



\prob{
	Consider regions of Earth's oceans that are far from any coast and have ocean depth large compared to the heights of ocean tides.  If Earth were nonrotating, then explain why the analysis of Sec~13.3 predicts that the ocean surface in these regions would be a surface of constant $\Phi$.  Explain why this remains true to good accuracy also for the rotating Earth.
}



\prob{
	Show that in these ocean regions, the Moon creates high tides pointing toward and away from itself and low tides in the transverse directions on Earth; and similarly for the Sun.  Compute the difference between high and low tides produced by the Moon and by the Sun, and the difference of the total tide when the Moon and the Sun are in approximately the same direction in the sky.  Your answers are reasonably accurate for deep-ocean regions far from any coast, but near a coast, the tides are typically larger and sometimes far larger, and they are shifted in phase relative to the positions of the moon and Sun.  Why?
}






\clearpage
\state{Beta function of the Gross-Neveu model~(P\&S~12.2)}{
	Compute $\bet(g)$ in the two-dimensional Gross-Neveu model studied in Problem~11.3,
	\eq{
		\cL = \psibsi i \ptsl \psisi + \frac{1}{2} g^2 (\psibsi \psisi)^2,
	}
	with $i = 1, \ldots, N$.  You should find that this model is asymptotically free.  How was that fact reflected in the solution to Problem~11.3?
}

\sol{
	We saw in Problem~2 of Homework~4 that this Lagrangian can be written as
	\eq{
		\cL = \psibsi i \ptsl \psisi - \sig \psibsi \psisi - \frac{1}{2 g^2} \sig^2,
	}
	where $\sig$ is a new scalar field with no kinetic energy terms.  In the modified minimal subtraction scheme, we found the effective potential was
	\eqn{Veff}{
		\Veff = \sig^2 \curly{ \frac{1}{2 g^2} + \frac{N}{4\pi} \brac{ \ln(\frac{\sig^2}{M^2}) - 1 } }.
	}
	Since $\Gam[ \phicl ] = -(V T) \Veff(\phi)$ by P\&S~(11.50), we have
	\eqn{Gam}{
		\Gam[ \sigcl ] = -(V T)  \sig^2 \curly{ \frac{1}{2 g^2} + \frac{N}{4\pi} \brac{ \ln(\frac{\sig^2}{M^2}) - 1 } }.
	}
	Referring to p.~3 of Lecture~11, we can apply the Callan-Symanzik equation to $\Gam$.   The Callan-Symanzik equation is P\&S~(12.41),
	\eq{
		\brac{ M \pdv{M} + \bet(\lam) \pdv{\lam} + n \gam(\lam) } G^{(n)}(\{ x_i \}; M, \lam) = 0.
	}
	For our problem, $\gam$ is 0 because there are no field insertions.  That is, we have
	\eq{
		\brac{ M \pdv{M} + \bet(g) \pdv{g} } \Gam[ \phicl ] = 0.
	}
	Using Eq.~\refeq{Gam}, note that
	\al{
		\pdv{\Gam}{M} &= (V T) \frac{N \sig^2}{2 \pi M}, &
		\pdv{\Gam}{g} &= (V T) \frac{\sig^2}{g^3}.
	}
	Then
	\eq{
		0 = (V T) \paren{ \frac{N \sig^2}{2 \pi} + \bet(g) \frac{\sig^2}{g^3} }
		\qimplies
		\ans{ \betg = -\frac{N g^3}{2\pi}. }
	}
	This model is asymptotically free because the $\bet$ function is proportional to $-g^3$~\cite[pp.~424--425]{Peskin}.
	
	In 2(e) of Homework~4, we found that the vacuum expectation value of $\sig$ was
	\eq{
		\sig = \pm M e^{-\pi / N g^2} = \pm v.
	}
	We showed that the vacuum expectation value does not depend on the renormalization condition chosen.  This means that we can increase $M \to 0$ while holding $\sig$ constant, and see that $g \to 0$ logarithmically.  This is indicative of an asymptotically-free theory~\cite[p.~425]{Peskin}. \qed
}






\clearpage
\state{Weyl curvature tensor~(MCP 25.12)}{
	Show that the Weyl curvature tensor~(25.48) has vanishing contraction on all its slots and has the same symmetries as Riemann: Eqs.~(25.45).  From these properties, show that Weyl has just 10 independent components.  Write the Riemann tensor in terms of the Weyl tensor, the Ricci tensor, and the scalar curvature.
}

\sol{
	MCP~(25.48) is
	\eqn{weyl}{
		\fv{C}{\mu \nu}{\rho \sig} = \fv{R}{\mu \nu}{\rho \sig} - 2 \fv{\sg}{[ \mu}{[ \rho} \fv{R}{\nu ]}{\sig ]} + \frac{1}{3} \fv{\sg}{[ \mu}{[ \rho} \fv{\sg}{\nu ]}{\sig ]} R,
	}
	where the square brackets denote antisymmetrization, $A_{[ \alp \bet ]} = (A_{\alp \bet} - A_{\bet \alp}) / 2$, and $R = \fv{R}{\alp}{\alp}$.  Lowering the indices and expanding out the antisymmetrizations gives us
	\aln{
		C_{\mu \nu \rho \sig} &= R_{\mu \nu \rho \sig} - \paren{ \sg_{\mu \ [ \rho} R_{\nu \ \sig ]} - \sg_{\nu \ [ \rho} R_{\mu \ \sig ]} } + \frac{1}{6} \paren{ \sg_{\mu \ [ \rho} \sg_{\nu \ \sig ]} - \sg_{\nu \ [ \rho} \sg_{\mu \ \sig ]} } R \notag \\
		&= R_{\mu \nu \rho \sig} - \frac{1}{2} \paren{ \sg_{\mu \rho} R_{\nu \sig} - \sg_{\mu \sig} R_{\nu \rho} - \sg_{\nu \rho} R_{\mu \sig} + \sg_{\nu \sig} R_{\mu \rho} } + \frac{1}{12} \paren{ \sg_{\mu \rho} \sg_{\nu \sig} - \sg_{\mu \sig} \sg_{\nu \rho} - \sg_{\nu \rho} \sg_{\mu \sig} + \sg_{\nu \sig} \sg_{\mu \rho} } R. \label{lowC}
	}
	
	We begin by interchanging indices since it makes the other proofs easier.  Interchanging the first two indices,
	\al{
		C_{\nu \mu \rho \sig} &= R_{\nu \mu \rho \sig} - \frac{1}{2} \paren{ \sg_{\nu \rho} R_{\mu \sig} - \sg_{\nu \sig} R_{\mu \rho} - \sg_{\mu \rho} R_{\nu \sig} + \sg_{\mu \sig} R_{\nu \rho} } + \frac{1}{12} \paren{ \sg_{\nu \rho} \sg_{\mu \sig} - \sg_{\nu \sig} \sg_{\mu \rho} - \sg_{\mu \rho} \sg_{\nu \sig} + \sg_{\mu \sig} \sg_{\nu \rho} } R \\
		&= -\brac{ R_{\mu \nu \rho \sig} - \frac{1}{2} \paren{ \sg_{\mu \rho} R_{\nu \sig} - \sg_{\mu \sig} R_{\nu \rho} - \sg_{\nu \rho} R_{\mu \sig} + \sg_{\nu \sig} R_{\mu \rho} } + \frac{1}{12} \paren{ \sg_{\mu \rho} \sg_{\nu \sig} - \sg_{\mu \sig} \sg_{\nu \rho} - \sg_{\nu \rho} \sg_{\mu \sig} + \sg_{\nu \sig} \sg_{\mu \rho} } R } \\
		&= - C_{\mu \nu \rho \sig},
	}
	where we have used $R_{\alp \bet \gam \del} = -R_{\bet \alp \gam \del}$ from MCP~(25.45a).
	
	Interchanging the last two indices,
	\al{
		C_{\mu \nu \sig \rho} &= R_{\mu \nu \sig \rho} - \frac{1}{2} \paren{ \sg_{\mu \sig} R_{\nu \rho} - \sg_{\mu \rho} R_{\nu \sig} - \sg_{\nu \sig} R_{\mu \rho} + \sg_{\nu \rho} R_{\mu \sig} } + \frac{1}{12} \paren{ \sg_{\mu \sig} \sg_{\nu \rho} - \sg_{\mu \rho} \sg_{\nu \sig} - \sg_{\nu \sig} \sg_{\mu \rho} + \sg_{\nu \rho} \sg_{\mu \sig} } R \\
		&= -\brac{ R_{\mu \nu \rho \sig} - \frac{1}{2} \paren{ \sg_{\mu \rho} R_{\nu \sig} - \sg_{\mu \sig} R_{\nu \rho} - \sg_{\nu \rho} R_{\mu \sig} + \sg_{\nu \sig} R_{\mu \rho} } + \frac{1}{12} \paren{ \sg_{\mu \rho} \sg_{\nu \sig} - \sg_{\mu \sig} \sg_{\nu \rho} - \sg_{\nu \rho} \sg_{\mu \sig} + \sg_{\nu \sig} \sg_{\mu \rho} } R } \\
		&= - \fv{C}{\mu \nu}{\rho \sig},
	}
	where we have used $R_{\alp \bet \gam \del} = -R_{\alp \bet \del \gam}$ from MCP~(25.45a).
	
	Interchanging the first and second pair of indices,
	\al{
		C_{\rho \sig \mu \nu} &= R_{\rho \sig \mu \nu} - \frac{1}{2} \paren{ \sg_{\rho \mu} R_{\sig \nu} - \sg_{\rho \nu} R_{\sig \mu} - \sg_{\sig \mu} R_{\rho \nu} + \sg_{\sig \nu} R_{\rho \mu} } + \frac{1}{12} \paren{ \sg_{\rho \mu} \sg_{\sig \nu} - \sg_{\rho \nu} \sg_{\sig \mu} - \sg_{\sig \mu} \sg_{\rho \nu} + \sg_{\sig \nu} \sg_{\rho \mu} } R \\
		&= R_{\mu \nu \rho \sig} - \frac{1}{2} \paren{ \sg_{\mu \rho} R_{\nu \sig} - \sg_{\mu \sig} R_{\nu \rho} - \sg_{\nu \rho} R_{\mu \sig} + \sg_{\nu \sig} R_{\mu \rho} } + \frac{1}{12} \paren{ \sg_{\mu \rho} \sg_{\nu \sig} - \sg_{\mu \sig} \sg_{\nu \rho} - \sg_{\nu \rho} \sg_{\mu \sig} + \sg_{\nu \sig} \sg_{\mu \rho} } R \\
		&= \fv{C}{\mu \nu}{\rho \sig},
	}
	where we have used the symmetry of the metric, $\fv{\sg}{\mu}{\nu} = \fv{\sg}{\nu}{\mu}$, and $R_{\alp \bet \gam \del} = +R_{\gam \del \alp \bet}$ from MCP~(25.45a).

	Note also that
	\al{
		C_{\mu \rho \sig \nu} &= R_{\mu \rho \sig \nu} - \frac{1}{2} \paren{ \sg_{\mu \sig} R_{\rho \nu} - \sg_{\mu \nu} R_{\rho \sig} - \sg_{\rho \sig} R_{\mu \nu} + \sg_{\rho \nu} R_{\mu \sig} } + \frac{1}{12} \paren{ \sg_{\mu \sig} \sg_{\rho \nu} - \sg_{\mu \nu} \sg_{\rho \sig} - \sg_{\rho \sig} \sg_{\mu \nu} + \sg_{\rho \nu} \sg_{\mu \sig} } R, \\
		%
		C_{\mu \sig \nu \rho} &= R_{\mu \sig \nu \rho} - \frac{1}{2} \paren{ \sg_{\mu \nu} R_{\sig \rho} - \sg_{\mu \rho} R_{\sig \nu} - \sg_{\sig \nu} R_{\mu \rho} + \sg_{\sig \rho} R_{\mu \nu} } + \frac{1}{12} \paren{ \sg_{\mu \nu} \sg_{\sig \rho} - \sg_{\mu \rho} \sg_{\sig \nu} - \sg_{\sig \nu} \sg_{\mu \rho} + \sg_{\sig \rho} \sg_{\mu \nu} } R.
	}
	Then for the equivalent to MCP~(25.45b),
	\al{
		C_{\mu \nu \rho \sig} + C_{\mu \rho \sig \nu} + C_{\mu \sig \nu \rho}
		&= R_{\mu \nu \rho \sig} + R_{\mu \rho \sig \nu} + R_{\mu \sig \nu \rho} - \frac{1}{2} \paren{ \sg_{\mu \rho} R_{\nu \sig} - \sg_{\mu \sig} R_{\nu \rho} - \sg_{\nu \rho} R_{\mu \sig} + \sg_{\nu \sig} R_{\mu \rho} } \\
		&\hspace{5em} \phantom{=} - \frac{1}{2} \paren{ \sg_{\mu \sig} R_{\rho \nu} - \sg_{\mu \nu} R_{\rho \sig} - \sg_{\rho \sig} R_{\mu \nu} + \sg_{\rho \nu} R_{\mu \sig} } \\
		&\hspace{5em} \phantom{=} - \frac{1}{2} \paren{ \sg_{\mu \nu} R_{\sig \rho} - \sg_{\mu \rho} R_{\sig \nu} - \sg_{\sig \nu} R_{\mu \rho} + \sg_{\sig \rho} R_{\mu \nu} } \\
		&\hspace{5em} \phantom{=} + \frac{1}{12} \paren{ \sg_{\mu \rho} \sg_{\nu \sig} - \sg_{\mu \sig} \sg_{\nu \rho} - \sg_{\nu \rho} \sg_{\mu \sig} + \sg_{\nu \sig} \sg_{\mu \rho} } R \\
		&\hspace{5em} \phantom{=} + \frac{1}{12} \paren{ \sg_{\mu \sig} \sg_{\rho \nu} - \sg_{\mu \nu} \sg_{\rho \sig} - \sg_{\rho \sig} \sg_{\mu \nu} + \sg_{\rho \nu} \sg_{\mu \sig} } R \\
		&\hspace{5em} \phantom{=} + \frac{1}{12} \paren{ \sg_{\mu \nu} \sg_{\sig \rho} - \sg_{\mu \rho} \sg_{\sig \nu} - \sg_{\sig \nu} \sg_{\mu \rho} + \sg_{\sig \rho} \sg_{\mu \nu} } R \\
		&= 0,
	}
	where we have used $R_{\alp \bet \gam \del} + R_{\alp \gam \del \bet} + R_{\alp \del \bet \gam} = 0$ from MCP~(25.45b), and the symmetry of the metric tensor.  Thus we have shown that the Weyl curvature tensor has the same symmetries as Riemann in MCP~(25.45): \vfix
	\ans{\aln{ \label{Csymm}
		C_{\nu \mu \rho \sig} &= -C_{\mu \nu \rho \sig}, &
		C_{\mu \nu \sig \rho} &= -C_{\mu \nu \rho \sig}, &
		C_{\rho \sig \mu \nu} &= +C_{\mu \nu \rho \sig}, &
		C_{\mu \nu \rho \sig} + C_{\mu \rho \sig \nu} + C_{\mu \sig \nu \rho} &= 0.
	}}%
	Now for the contractions.  Contracting the indices within each pair,
	\aln{ \label{thingy2}
		\sg^{\mu \nu} C_{\mu \nu \rho \sig} &= \sfvt{C}{\mu}{\mu}{\rho \sig}
		= -\sfvt{C}{\mu}{\mu}{\rho \sig}
		= 0, &
		%
		\sg^{\rho \sig} C_{\mu \nu \rho \sig} &= \sfv{C}{\mu \nu \rho}{\rho}
		= -\sfv{C}{\mu \nu \rho}{\rho}
		= 0,
	}
	where we have used $C_{\nu \mu \rho \sig} = -C_{\mu \nu \rho \sig}$ and our ability to swap covariant and contravariant for summed indices.
	
	Contracting the first and third indices,
	\al{
		\sg^{\mu \rho} C_{\mu \nu \rho \sig} &= \sg^{\mu \rho} \brac{ R_{\mu \nu \rho \sig} - \frac{1}{2} \paren{ \sg_{\mu \rho} R_{\nu \sig} - \sg_{\mu \sig} R_{\nu \rho} - \sg_{\nu \rho} R_{\mu \sig} + \sg_{\nu \sig} R_{\mu \rho} } + \frac{1}{12} \paren{ \sg_{\mu \rho} \sg_{\nu \sig} - \sg_{\mu \sig} \sg_{\nu \rho} - \sg_{\nu \rho} \sg_{\mu \sig} + \sg_{\nu \sig} \sg_{\mu \rho} } R } \\
		&= \sfvt{R}{\mu \nu}{\mu}{\sig} - \frac{1}{2} \paren{ \sfv{\sg}{\mu}{\mu} R_{\nu \sig} - \sg_{\mu \sig} \sfv{R}{\nu}{\mu} - \sfv{\sg}{\nu}{\mu} R_{\mu \sig} + \sg_{\nu \sig} \sfv{R}{\mu}{\mu} } + \frac{1}{12} \paren{ \sfv{\sg}{\mu}{\mu} \sg_{\nu \sig} - \sg_{\mu \sig} \sfv{\sg}{\nu}{\mu} - \sfv{\sg}{\nu}{\mu} \sg_{\mu \sig} + \sg_{\nu \sig} \sfv{\sg}{\mu}{\mu} } R \\
		&= R_{\nu \sig} - \frac{1}{2} \paren{ 4 R_{\nu \sig} - R_{\nu \sig} - R_{\nu \sig} + \sg_{\nu \sig} R } + \frac{1}{12} \paren{ 4 \sg_{\nu \sig} - \sg_{\nu \sig} - \sg_{\nu \sig} + 4 \sg_{\nu \sig} } R \\
		&= R_{\nu \sig} - R_{\nu \sig} - \frac{1}{2} \sg_{\nu \sig} R + \frac{1}{2} \sg_{\nu \sig} R \\
		&= 0,
	}
	where we have used MCP~(24.10), $\sg^{\mu \bet} \sg_{\bet \nu} = \fv{\del}{\mu}{\nu}$, to find $\fv{\sg}{\mu}{\mu} = 4$.  Then, using Eq.~\refeq{Csymm}, we have
	\eq{
		0 = \sfvt{C}{\mu \nu}{\mu}{\sig}
		= -\sfv{C}{\mu \nu \sig}{\mu}
		= -\sfvt{C}{\nu \mu}{\mu}{\sig}
		= \sfv{C}{\nu \mu \sig}{\mu}.
	}
	Rewriting these as contractions, and incorporating our results in Eq.~\refeq{thingy2}, we have shown
	\eq{
		\ans{ 0 = \sg^{\mu \nu} C_{\mu \nu \rho \sig}
		= \sg^{\mu \rho} C_{\mu \nu \rho \sig}
		= \sg^{\mu \sig} C_{\mu \nu \rho \sig}
		= \sg^{\nu \rho} C_{\mu \nu \rho \sig}
		= \sg^{\nu \sig} C_{\mu \nu \rho \sig}
		= \sg^{\rho \sig} C_{\mu \nu \rho \sig}; }
	}
	that is, the Weyl curvature tensor has vanishing contraction on all its slots.
	
	To show that Weyl has just 10 independent components, we refer to the discussion at the end of Lecture~11.  If there were no symmetries, in $n = 4$ spacetime dimensions $C_{\alp \bet \gam \del}$ would have $n^4 = 4^4$ independent components.  Each symmetry gives rise to some number of independent components:
	\al{
		\alp \lrarrow \bet \antisymmetry &\qimplies \frac{n (n - 1)}{2} = 6 \components, \\
		\gam \lrarrow \del \antisymmetry &\qimplies \frac{n (n - 1)}{2} = 6 \components, \\
		\alp \bet \lrarrow \gam \del \symmetry &\qimplies \text{components reduced by } \frac{1}{2}.
	}
	Combining the three (anti)symmetries, we have a total of
	\eq{
		\paren{ \frac{n (n - 1)}{2} } \paren{ \frac{n (n - 1)}{2} + 1 } \frac{1}{2} = 21 \components.
	}
	Now we account for the constraints imposed by $C_{\mu \nu \rho \sig} + C_{\mu \rho \sig \nu} + C_{\mu \sig \nu \rho} = 0$.  This expression is redundant if any of the indices are the same; for example,
	\eq{
		C_{\mu \mu \rho \sig} + C_{\mu \rho \sig \mu} + C_{\mu \sig \mu \rho}
		= 0 + C_{\sig \mu \mu \rho} + C_{\mu \sig \mu \rho}
		= C_{\sig \mu \mu \rho} - C_{\sig \mu \mu \rho}
		= 0,
	}
	which we already knew from the other symmetries.  Since there are four indices, this requirement gives us
	\eq{
		{ n \choose 4 } = 1 \text{ constraint}.
	}
	Finally we account for the constraints imposed by the vanishing contractions.  After accounting for the (anti)symmetries, we are left with two unique vanishing contractions, which impose
	\eq{
		\frac{n (n + 1)}{2} = 10 \constraints.
	}
	\hl{this probably is wrong but eh}
	
	Finally, we find the number of independent components by
	\eq{
		N_\text{components} - N_\text{constraints} = 21 - (1 + 10) = \ans{ 10 \text{ independent components} }
	}
	as we wanted to show. \qed
	
	\hl{general form of 25.48?}
}






\clearpage
\newcommand{\lap}{\nabla^2}
\newcommand{\vF}{\vec{F}}
\newcommand{\nabx}{\nabla_{\!x}}
\newcommand{\absxp}{\abs{\vx'}}
\newcommand{\nh}{\vec{\hat{n}}}
\newcommand{\rh}{\vec{\hat{r}}}
\newcommand{\Gd}{G_D}
\newcommand{\Gdxxp}{\Gd(\vx,\vx')}

\begin{statement}{}
	A point charge of charge $q$ is placed at point $\vx'$ inside a conducting spherical shell of radius $R$.  There is no net charge on the conductor.  The potential inside the sphere is thus given by $q \, \Gdxxp$, where the explicit formula for $\Gdxxp$ for a spherical cavity is given in the lecture notes.
\end{statement}

\begin{problem}
	Find the surface charge density $\sigtv$ on the conducting shell.
\end{problem}

\begin{solution}
	The Green's function for a spherical cavity is given by Eq.~(2.91),
	\beq
		\Gdxxp = \frac{1}{\abs{\vx - \vx'}} + \frac{\alp}{\abs{\vx - \vx''}} \qq{where} \vx'' = \vx' \frac{R^2}{\absxp^2} \qand \alp = - \frac{R}{\absxp}.
	\eeq
	The surface charge density can be found from Eq.~(2.86),
	\beqn \label{scdeq}
		\vE \cdot \nh = 4\pi \sig,
	\eeqn
	where $\vE = -\nabla \phi$ in electrostatics.
	
	We will begin by finding $\vE$.  We will orient our coordinate system such that $\vx'$ (and consequently $\vx''$) points along the $z$ axis.  Note that
	\beq
		\Gdxxp = \frac{1}{\abs{\vx - \vx'}} - \frac{R}{\absxp \abs{\vx - \dfrac{R^2}{\absxp^2} \vx'}}
		= \frac{1}{\sqrt{\vx^2 - 2 \vx \cdot \vx' + {\vx'}^2}} - \frac{R}{\absxp \sqrt{\vx^2 - 2 \dfrac{R^2}{{\vx'}^2} \vx \cdot \vx' + \dfrac{R^4}{{\vx'}^4} {\vx'}^2}}.
	\eeq
	In spherical coordinates, we have
	\beq
		\Gdxxp = \frac{1}{\sqrt{r^2 - 2 r r' \cost + {r'}^2}} - \frac{R}{r'} \frac{1}{\sqrt{r^2 - 2 R^2 r \cost / r' + R^4 / {r'}^2}},
	\eeq
	where we note that $\tht$ is the angle between $\vx$ and the $z$ axis.  The gradient in spherical coordinates is given by
	\beq
		\nabla = \pdv{}{r} \,\rh + \frac{1}{r} \pdv{}{\tht} \,\thh + \frac{1}{r \sint} \pdv{}{\vph} \, \phh.
	\eeq
	The $r$ component of the electric field inside the conductor is then
	\beq
		\Er(\vx) = -q \pdv{\Gdxxp}{r}
		= q \left( \frac{r - r' \cost}{(r^2 - 2 r r' \cost + {r'}^2)^{3/2}} - \frac{R}{r'} \frac{r - R^2 \cost / r'}{(r^2 - 2 R^2 r \cost / r' + R^4 / {r'}^2)^{3/2}} \right).
	\eeq
	Since $\nh = -\rh$ for the inner surface of a sphere, we are interested in only the $r$ component of the field.  On the surface of the sphere, the field is $\Er(r=R) \,\rh$.  So we have
	\begin{align*}
		\Er(r=R) &= q \left( \frac{R - r' \cost}{(R^2 - 2 R r' \cost + {r'}^2)^{3/2}} - \frac{R}{r'} \frac{R - R^2 \cost / r'}{(R^2 - 2 R^3 \cost / r' + R^4 / {r'}^2)^{3/2}} \right) \\
		&= q \left( \frac{R - r' \cost}{{r'}^3 (R^2 / {r'}^2 - 2 R \cost / r' + 1)^{3/2}} - \frac{R}{r'} \frac{R - R^2 \cost / r'}{R^3 (1 - 2 R \cost / r' + R^2 / {r'}^2)^{3/2}} \right) \\
		&= \frac{q}{r'} \frac{R^3 - R^2 r' \cost - R {r'}^2 + R^2 r' \cost}{R^2 {r'}^2 (R^2 / {r'}^2 - 2 R \cost / r' + 1)^{3/2}}
		= \frac{q}{R {r'}^3} \frac{R^2 - {r'}^2}{(R^2 / {r'}^2 - 2 R \cost / r' + 1)^{3/2}}.
	\end{align*}
	Finally, feeding this into \refeq{scdeq},
	\beq
		\sig = -\frac{\vE \cdot \rh}{4\pi}
		= \frac{q}{4\pi R {r'}^3} \frac{{r'}^2 - R^2}{(R^2 / {r'}^2 - 2 R \cost / r' + 1)^{3/2}}
		= \frac{q}{4\pi R \absxp^3} \frac{\absxp^2 - R^2}{(R^2 / \absxp^2 - 2 R \cost / \absxp + 1)^{3/2}}.
	\eeq
\end{solution}
\vfix


\newcommand{\vEo}{\vE_0}
\newcommand{\del}{\delta}
\newcommand{\Etht}{E_\tht}
\newcommand{\Fr}{F_r}

\begin{problem}
	Find the force $\vF$ that must be exerted on the point charge in order to hold it in place.
\end{problem}

\begin{solution}
	The total force on a charge distribution arises only from the external electric field $\vEo$, and is given by Eq.~(2.42) in the lecture notes:
	\beq
		\vF = \int \rhox \, \vEo(\vx) \dcx.
	\eeq
	The force required to keep the point charge in place is equal and opposite to this force, so we need to insert a minus sign.  We also need the $\tht$ component of the field inside the conductor, which is
	\beq
		\Etht(\vx) = -\frac{q}{r} \pdv{\Gdxxp}{\tht}
		= -q \left( \frac{r' \sint}{(r^2 - 2 r r' \cost + {r'}^2)^{3/2}} - \frac{R^3 \sint}{{r'}^2 (r^2 - 2 R^2 r \cost / r' + R^4 / {r'}^2)^{3/2}} \right).
	\eeq
	The charge density for a point charge located at $\vx'$ is given by $\rhox = q \, \delta(\vx - \vx')$.  Evaluating the integral, we have
	\beq
		\vF = -\int q \, \delta(\vx - \vx') \, \vE(\vx) \dcx
		= -q \vE(\vx').
	\eeq
	Recall that we chose $\vx'$ to point along the $z$ axis, so $\tht' = 0$.  The $\tht$ component of $\vF$ is then $0$, and the $r$ component is
	\begin{align*}
		\Fr &= -q^2 \left( \frac{r' - r'}{({r'}^2 - 2 {r'}^2 + {r'}^2)^{3/2}} - \frac{R}{r'} \frac{r' - R^2 / r'}{({r'}^2 - 2 R^2 + R^4 / {r'}^2)^{3/2}} \right)
		= q^2 R {r'}^2 \frac{r' - R^2 / r'}{({r'}^4 - 2 R^2 {r'}^2 + R^4)^{3/2}} \\
		&= -q^2 R {r'}^2 \frac{({r'}^2 - R^2) / r'}{({r'}^2 - R^2)^3}
		= -q^2 \frac{R r'}{({r'}^2 - R^2)^2}.
	\end{align*}
	Since only the $r$ component of $\vF$ is nonzero, it points in the $z$ direction, which we chose to be equivalent to the unit vector $\vx' / \absxp$.  Therefore,
	\beq
		\vF = -q^2 \frac{R \absxp}{(R^2 - \absxp^2)^2} \frac{\vx'}{\absxp}
		= -q^2 \frac{R}{(R^2 - \absxp^2)^2} \vx'.
	\eeq
\end{solution}






\clearpage
\state{Geodesic deviation on a sphere~(MCP 25.14)}{
	Consider two neighboring geodesics (great circles) on a sphere of radius $a$, one the equator and the other a geodesic slightly displaced from the equator (by $\Del\tht = b$) and parallel to it at $\phi = 0$.  Let $\vxi$ be the separation vector betwen the two geodesics, and note that at $\phi = 0$, $\vxi = b \pdv*{\tht}$.  Let $l$ be proper distance along the equatorial geodesic, so $\dv*{l} = \vuu$ is its tangent vector.
}

\prob{
	Show that $l = a \phi$ along the equatorial geodesic.
}

\sol{
	We know from Eq.~\refeq{given4} that, on the equator where $\tht = \pi / 2$,
	\eq{
		\dds^2 = a^2 \sin[2](\frac{\pi}{2}) \ddphi^2
		= a^2 \ddphi^2.
	}
	Integrating both sides of the equation,
	\eq{
		\int_0^l \dds = \int_0^\phi a \ddphi
		\qimplies
		\ans{ l = a \phi. }
	}
	This is merely the expression for the arc length of a circle of radius $a$ for a segment of angle $\tht$. \qed
}



\prob{
	Show that the equation of geodesic deviation (25.31) reduces to
	\aln{ \label{show5b}
		\dv[2]{\xit}{\phi} &= -\xit, &
		\dv[2]{\xip}{\phi} &= 0.
	}
	\vfix
}

\sol{
	The equation of geodesic deviation is
	\eqn{gd}{
		(\fv{\xi}{\alp}{; \bet} p^\bet)_{; \gam} p^\gam = -\fv{R}{\alp}{\bet \gam \del} p^\bet \xi^\gam p^\del,
	}
	where $\vp$ is a 4-momentum tangent problem and is equal to $\vuu$ in this problem.  By definition, $\vuu$ has only a component in the $\phi$ direction, which is equal to 1.  Then Eq.~\refeq{gd} reduces to
	\eqn{thing5b}{
		(\fv{\xi}{\alp}{; \phi})_{; \phi} p^\phi = -\fv{R}{\alp}{\phi \gam \phi} \xi^\gam
		= -\fv{R}{\alp}{\phi \tht \phi} \xit,
	}
	since the only nonzero $R_{\alp \bet \gam \del}$ have two $\tht$ and two $\phi$ indices.  Applying Eq.~\refeq{thing2} to the left-hand side of \ref{thing5b},
	\eqn{lhs5}{
		(\fv{\xi}{\alp}{; \phi})_{; \phi} = (\fv{\xi}{\alp}{, \phi} + \xi^\mu \fv{\Gam}{\alp}{\mu \phi})_{, \phi} + (\fv{\xi}{\nu}{, \phi} + \xi^\mu \fv{\Gam}{\nu}{\mu \phi}) \fv{\Gam}{\alp}{\nu \phi}.
	}
	For $\alp = \tht$,
	\al{
		(\fv{\xi}{\tht}{; \phi})_{; \phi} &= (\fv{\xi}{\tht}{, \phi} + \xi^\tht \fv{\Gam}{\tht}{\tht \phi} + \xi^\phi \fv{\Gam}{\tht}{\phi \phi})_{, \phi} + (\fv{\xi}{\tht}{, \phi} + \xi^\tht \fv{\Gam}{\tht}{\tht \phi} + \xi^\phi \fv{\Gam}{\tht}{\phi \phi}) \fv{\Gam}{\tht}{\tht \phi} + (\xi^\phi \fv{\Gam}{\phi}{\tht \phi} + \xi^\phi \fv{\Gam}{\phi}{\phi \phi}) \fv{\Gam}{\tht}{\phi \phi} \\
		&= (\fv{\xi}{\tht}{, \phi} + \xi^\phi \fv{\Gam}{\tht}{\phi \phi})_{, \phi} + \xi^\phi \fv{\Gam}{\phi}{\tht \phi} \fv{\Gam}{\tht}{\phi \phi} \\
		&= \paren{ \fv{\xi}{\tht}{, \phi} - \frac{\cot\tht}{a} \xi^\phi }_{, \phi} - \frac{\cot^2\tht}{a^2} \xi^\phi \\
		&= \frac{1}{a^2} \dv[2]{\xit}{\phi},
	}
	where we have used Eq.~\refeq{Gammas} and $\tht = \pi / 2$.  Combining with the right-hand side of Eq.~\refeq{thing5b} then gives us
	\eq{
		\frac{1}{a^2} \dv[2]{\xit}{\phi} = -\fv{R}{\tht}{\phi \tht \phi} \xit
		= -\frac{1}{a^2} \xit
		\qimplies
		\ans{ \dv[2]{\xit}{\phi} = -\xit, }
	}
	where we have used Eq.~\refeq{show4d}.
	
	For $\alp = \phi$, Eq.~\refeq{lhs5} gives us
	\al{
		(\fv{\xi}{\phi}{; \phi})_{; \phi} &= (\fv{\xi}{\phi}{, \phi} + \xi^\tht \fv{\Gam}{\phi}{\tht \phi} + \xi^\phi \fv{\Gam}{\phi}{\phi \phi})_{, \phi} + (\fv{\xi}{\tht}{, \phi} + \xi^\tht \fv{\Gam}{\tht}{\tht \phi} + \xi^\phi \fv{\Gam}{\tht}{\phi \phi}) \fv{\Gam}{\phi}{\tht \phi} + (\fv{\xi}{\phi}{, \phi} + \xi^\tht \fv{\Gam}{\phi}{\tht \phi} + \xi^\phi \fv{\Gam}{\phi}{\phi \phi}) \fv{\Gam}{\phi}{\phi \phi} \\
		&= (\fv{\xi}{\phi}{, \phi} + \xi^\tht \fv{\Gam}{\phi}{\tht \phi})_{, \phi} + (\fv{\xi}{\tht}{, \phi} + \xi^\phi \fv{\Gam}{\tht}{\phi \phi}) \fv{\Gam}{\phi}{\tht \phi} \\
		&= \paren{ \fv{\xi}{\phi}{, \phi} + \frac{\cot\tht}{r} \xi^\tht }_{, \phi} + \paren{ \fv{\xi}{\tht}{, \phi} - \frac{\cot\tht}{r} \xi^\phi } \frac{\cot\tht}{r} \\
		&= \frac{1}{a^2} \dv[2]{\xip}{\phi},
	}
	and combining with the right-hand side of Eq.~\refeq{thing5b},
	\eq{
		\frac{1}{a^2} \dv[2]{\xip}{\phi} = 0
		\qimplies
		\ans{ \dv[2]{\xip}{\phi} = 0, }
	}
	so we have proven Eq.~\refeq{show5b}. \qed
}



\prob{
	Solve Eq.~\refeq{show5b}, subject to the above initial conditions, to obtain
	\al{
		\xit &= b \cos\phi, &
		\xip &= 0.
	}
	Verify, by drawing a picture, that this is precisely what one would expect for the separation vector between two great circles.
}

\sol{
	The differential equations have the solutions~\cite[pp.~206--207]{Swartz}
	\al{
		\dv[2]{\xit}{\phi} = -\xit
		&\qimplies
		\xit(\phi) = A \sin\phi + B \cos\phi, &
		%
		\dv[2]{\xip}{\phi} = 0
		&\qimplies
		\xip(\phi) = C + D \phi.
	}
	For the initial conditions, as stated $\vxi(\phi = 0) = b \vest$, so $\xit(0) = b$ and $\xip(0) = 0$.  This means
	\al{
		b &= B, &
		0 &= C.
	}
	It is also clear that
	\al{
		\left. \dv{\xit}{\phi} \right|_{\phi = 0} &= \left. \dv{b}{\phi} \right|_{\phi = 0} = 0, &
		\left. \dv{\xip}{\phi} \right|_{\phi = 0} &= \left. \dv{0}{\phi} \right|_{\phi = 0} = 0,
	}
	which gives us
	\al{
		0 &= \left. \dv{\xit}{\phi} \right|_{\phi = 0}
		= \bigg[ A \cos\phi - b \sin\phi \bigg]_{\phi = 0}
		= A, &
		0 &= \left. \dv{\xip}{\phi} \right|_{\phi = 0}
		= \bigg[ D \bigg]_{\phi = 0}
		= D.
	}
	So we have the solutions
	\ans{\al{
		\xit &= b \cos\phi, &
		\xip &= 0
	}}%
	as we wanted to show. \qed
	
	\hl{how the heck to draw this picture}
}





\clearpage
\state{Curvature-coupling torque~(MCP 25.16)}{\hfix}

\prob{
	In the Newtonian theory of gravity, consider an axisymmetric, spinning body (e.g., Earth) with spin angular momentum $\Sj$ an dtimei-independent mass distribution $\rho(\bx)$, interacting with an externally produced tidal gravitational field $\cEsjk$ (e.g., that of the Sun and the Moon).  Show that the torque around the body's center of mass, exerted by the tidal field, and the resulting evolution of the body's spin are
	\eqn{show6a}{
		\dv{\Ssi}{t} = -\epssijk \cIsjl \cEskl.
	}
	Here
	\eq{
		\cIskl = \int \rho \paren{ \xsk \xsl - \frac{1}{3} r^2 \delskl } \ddV
	}
	is the body's mass quadrupole moment, with $r = \sqrt{\delsij \xsi \xsj}$ the distance from the center of mass.
}



\prob{
	For the centrifugally flattened Earth interacting with the tidal fields of the Moon and the Sun, estimate in order of magnitude the spin-precession period produced by this torque.  [The observed precession period is 26,000 years.]
}



\prob{
	Show that when rewritten in the language of general relativity, and in frame-independent, geometric language, Eq.~\refeq{show6a} takes the form (25.59) discussed in the text.  As part of showing this, explain the meaning of $\cIsbm$ in that equation.
}

\makebib

\end{document}
