\documentclass[11pt]{article}
\usepackage{homework}

\classname{364}
\homeworknum{3}

\DeclareMathAlphabet{\mathsfit}{T1}{\sfdefault}{\mddefault}{\sldefault}


\begin{document}

% Environments

\newcommand{\state}[2]{\begin{statement}{#1} #2 \end{statement}}
\newcommand{\prob}[2]{\begin{problem}{#1} #2 \end{problem}}
\newcommand{\subprob}[1]{\begin{subproblem} #1 \end{subproblem}}
\newcommand{\sol}[1]{\begin{solution} #1 \end{solution}}
\newcommand{\fig}[2]{\begin{figure} \centering #2  \label{#1} \end{figure}}

\newcommand{\makebib}{
	\vfill
	\color{black}
	\bibliography{references}{}
	\bibliographystyle{lucas_unsrt}
}
	

% Implication

\newcommand{\qwhere}{\quad \text{where} \quad}
\newcommand{\qimplies}{\quad \implies \quad}
\newcommand{\impliesq}{\implies \quad}



% Brackets

\newcommand{\paren}[1]{\left( #1 \right)}
\newcommand{\brac}[1]{\left[ #1 \right]}


% Greek

\newcommand{\alp}{\alpha}
\newcommand{\bet}{\beta}
\newcommand{\gam}{\gamma}
\newcommand{\del}{\delta}
\newcommand{\eps}{\epsilon}
\newcommand{\zet}{\zeta}
\newcommand{\tht}{\theta}
\newcommand{\kap}{\kappa}
\newcommand{\lam}{\lambda}
\newcommand{\sig}{\sigma}
\newcommand{\ups}{\upsilon}
\newcommand{\omg}{\omega}

\newcommand{\Gam}{\Gamma}
\newcommand{\Del}{\Delta}
\newcommand{\Tht}{\Theta}
\newcommand{\Lam}{\Lambda}
\newcommand{\Sig}{\Sigma}
\newcommand{\Omg}{\Omega}
% Problem 1

\newcommand{\Psii}{\Psi^i}
\newcommand{\Psiix}{\Psii(x)}

\newcommand{\Pii}{\Pi^i}

\newcommand{\Phii}{\Phi^i}
\newcommand{\Phiix}{\Phii(x)}
\newcommand{\PhiN}{\Phi^N}
\newcommand{\PhiNx}{\PhiN(x)}
\newcommand{\Phiq}{\Phi^1}
\newcommand{\Phiw}{\Phi^2}

\newcommand{\ddcx}{\dd[3]{x}}

\newcommand{\delij}{\del^{i j}}
\newcommand{\delkl}{\del^{k l}}
\newcommand{\delil}{\del^{i l}}
\newcommand{\deljk}{\del^{j k}}
\newcommand{\delik}{\del^{i k}}
\newcommand{\deljl}{\del^{j l}}

\newcommand{\DF}{D_F}

\newcommand{\sigx}{\sig(x)}

\newcommand{\pii}{\pi^i}
\newcommand{\pij}{\pi^j}
\newcommand{\pik}{\pi^k}
\newcommand{\pil}{\pi^l}
\newcommand{\piix}{\pi(x)}

\newcommand{\pq}{p_1}
\newcommand{\pw}{p_2}
\newcommand{\pe}{p_3}
\newcommand{\pr}{p_4}

\newcommand{\vp}{\vb{p}}
\newcommand{\vpsi}{\vp_i}

\newcommand{\mpi}{m_\pi}



\state{Connection coefficients for spherical polar coordinates~(MCP 24.9)}{\hfix}

\prob{ \label{1a}
	Consider spherical polar coordinates in 3-dimensional space, and verify that the nonzero connection coefficients, assuming an orthonormal basis, are given by Eq.~(11.71).
}

\sol{
	We follow the procedure on pp.~1171--1172 of MCP for computing the connection coefficients.  We first evaluate the commutation coefficients $\csabsr$ using MCP~(24.38a),
	\eqn{commcoeff}{
		\csabsr = \ver \cdot [ \vesa, \vesb ],
	}
	We lower the last index using (24.38b),
	\eq{
		\csabg = \csabsr \sgsrg.
	}
	Then we use (24.38c) to compute
	\eqn{Gamthing}{
		\Gamsabg = \frac{1}{2} (\sgsabdg + \sgsagdb - \sgsbgda + \csabg + \csagb - \csbga),
	}
	and raise the first index using (24.38d),
	\eq{
		\Gammsbg = \sgma \Gamsabg.
	}
	From the lecture, the commutator is given by
	\eqn{comm}{
		[ \vA, \vB ] = \nabsvA \vB - \nabsvB \vA.
	}
	We also note that $\sgsab = \vesa \cdot \vesb$~\cite[p.~1161]{MCP}.
	
	For an orthonormal basis $\{ \rh, \thh, \phh \}$, $\sg$ is the identity matrix~\cite[p.~614]{MCP}.  In spherical coordinates, the gradient is
	\eq{
		\grad = \rh \pdv{r} + \frac{1}{r} \thh \pdv{\tht} + \frac{1}{r \sin\tht} \phh \pdv{\phi},
	}
	and its components are~\cite{Spherical} \hl{(better double check)}
	\al{
		\nabsr \rh &= \bo, &
		\nabst \rh &= \frac{1}{r} \thh, &
		\nabsp \rh &= \frac{1}{r} \phh, \\
		\nabsr \thh &= \bo, &
		\nabst \thh &= -\frac{1}{r} \rh, &
		\nabsp \thh &= \frac{1}{r \sin \tht} \phh, \\
		\nabsr \phh &= \bo, &
		\nabst \phh &= \bo, &
		\nabsp \phh &= -\frac{1}{r \sin\tht} \thh - \frac{1}{r} \rh.
	}
	Applying Eq.~\refeq{comm} and the above, we find
	\al{
		[ \rh, \rh ] &= \nabsr \rh - \nabsr \rh = \bo, &
		[ \rh, \thh ] &= \nabsr \thh - \nabst \rh = -\frac{1}{r} \thh, &
		[ \rh, \phh ] &= \nabsr \phh - \nabsp \rh = -\frac{1}{r} \phh, \\
		%
		[ \thh, \rh ] &= -[ \rh, \thh] = \frac{1}{r} \thh, &
		[ \thh, \thh ] &= \nabst \thh - \nabst \thh = \bo, &
		[ \thh, \phh ] &= \nabst \phh - \nabsp \thh = -\frac{1}{r \sin \tht} \phh, \\
		%
		[ \phh, \rh ] &= -[ \rh, \phh ] = \frac{1}{r} \phh, &
		[ \phh, \thh ] &= -[ \thh, \phh ] = \frac{1}{r \sin \tht} \phh, &
		[ \phh, \phh ] &= \nabsp \phh - \nabsp \phh = \bo.
	}
	Since $\sg$ is the identity, we can immediately write from Eq.~\refeq{commcoeff}
	\al{
		c_{r r r} &= [ \rh, \rh ] \vdot \rh = 0, &
		c_{r \tht r} &= [ \rh, \thh ] \vdot \rh = 0, &
		c_{r \phi r} &= [ \rh, \phh ] \vdot \rh = 0, \\
		%
		c_{\tht r r} &= -c_{r \tht r} = 0, &
		c_{\tht \tht r} &= [ \thh, \thh ] \vdot \rh = 0, &
		c_{\tht \phi r} &= [ \thh, \phh ] \vdot \rh = 0, \\
		%
		c_{\phi r r} &= -c_{r \phi r} = 0, &
		c_{\phi \tht r} &= -c_{ \tht \phi r} = 0, &
		c_{\phi \phi r} &= [ \phh, \phh ] \vdot \rh = 0,
	}
	\al{
		c_{r r \tht} &= [ \rh, \rh ] \vdot \thh = 0, &
		c_{r \tht \tht} &= [ \rh, \thh ] \vdot \thh = -\frac{1}{r}, &
		c_{r \phi \tht} &= [ \rh, \phh ] \vdot \thh = 0, \\
		%
		c_{\tht r \tht} &= -c_{r \tht \tht} = \frac{1}{r}, &
		c_{\tht \tht \tht} &= [ \thh, \thh ] \vdot \thh = 0, &
		c_{\tht \phi \tht} &= [ \thh, \phh ] \vdot \thh = 0, \\
		%
		c_{\phi r \tht} &= -c_{r \phi \phi} = \frac{1}{r}, &
		c_{\phi \tht \tht} &= -c_{\tht \phi \tht} = 0, &
		c_{\phi \phi \tht} &= [ \phh, \phh ] \vdot \thh = 0, \\[2ex]
		%
		%
		c_{r r \phi} &= [ \rh, \rh ] \vdot \phh = 0, &
		c_{r \tht \phi} &= [ \rh, \thh ] \vdot \phh = 0, &
		c_{r \phi \phi} &= [ \rh, \phh ] \vdot \phh = -\frac{1}{r}, \\
		%
		c_{\tht r \phi} &= -c_{r \tht \phi} = 0, &
		c_{\tht \tht \phi} &= [ \thh, \thh ] \vdot \phh = 0, &
		c_{\tht \phi \phi} &= [ \thh, \phh ] \vdot \phh = -\frac{1}{r \sin \tht}, \\
		%
		c_{\phi r \phi} &= -c_{r \phi \phi} = \frac{1}{r}, &
		c_{\phi \tht \phi} &= -c_{\tht \phi \tht} = \frac{1}{r \sin \tht}, &
		c_{\phi \phi \phi} &= [ \phh, \phh ] \vdot \phh = 0.
	}
	From Eq.~\refeq{Gamthing} we again use the fact that $\sg$ is the identity to write
	\al{
		\Gam_{r r r} &= \frac{c_{r r r} + c_{r r r} - c_{r r r}}{2} = 0, &
		\Gam_{r r \tht} &= \frac{c_{r r \tht} + c_{r \tht r} - c_{r \tht r}}{2} = 0, &
		\Gam_{r r \phi} &= \frac{c_{r r \phi} + c_{r \phi r} - c_{r \phi r}}{2} = 0, \\
		%
		\Gam_{r \tht r} &= \frac{c_{r \tht r} + c_{r r \tht} - c_{\tht r r}}{2} = 0, &
		\Gam_{r \tht \tht} &= \frac{c_{r \tht \tht} + c_{r \tht \tht} - c_{\tht \tht r}}{2} = -\frac{1}{r}, &
		\Gam_{r \tht \phi} &= \frac{c_{r \tht \phi} + c_{r \phi \tht} - c_{\tht \phi r}}{2} = 0, \\
		%
		\Gam_{r \phi r} &= \frac{c_{r \phi r} + c_{r r \phi} - c_{\phi r r}}{2} = 0, &
		\Gam_{r \phi \tht} &= \frac{c_{r \phi \tht} + c_{r \tht \phi} - c_{\phi \tht r}}{2} = 0, &
		\Gam_{r \phi \phi} &= \frac{c_{r \phi \phi} + c_{r \phi \phi} - c_{\phi \phi r}}{2} = -\frac{1}{r}, \\
	}
}


\clearpage
\prob{ \label{1b}
	Repeat the exercise in \ref{1a} assuming a coordinate basis with
	\al{
		\besr &\equiv \pdv{r}, &
		\bestht &\equiv \pdv{\tht}, &
		\besphi &\equiv \pdv{\phi}.
	}
	\vfix
}



\prob{
	Repeat both computations in \ref{1a} and \ref{1b} using symbolic manipulation software on a computer.
}





\clearpage
\state{}{
	 Let $V$ be a vector field.  Prove the covariant divergence formula valid in a coordinate basis
	 \eq{
	 	\nablasa \Va = \frac{1}{\sqabsg} \ptsa (\sqabsg \Va),
	 }
	 where $g$ is the determinant of the metric.
}





\clearpage
\state{}{
	 In this problem you will explore the geometry of a sphere $S^2$ of radius $R$.
}

\prob{
	A vector $\vV = \Vtht \vestht + \Vphi \vesphi$ is defined at a point $(\tht, \phi)$ on the sphere.  It is then parallel transported around the circle of constant $\tht$ with $\phi \to \phi + 2\pi$.  What are its resulting components?  What is its length?
}



\prob{
	Write the geodesic equation in $(\tht, \phi)$ angular coordinates.  Show that the solutions are \emph{great circles}, i.e.~circles on the sphere of largest diameter.
}



\prob{
	Consider a disk of radius $\eps$ on the sphere.  Working in the limit of small $\eps$, compute the area of the disk to order $\eps^4$.  Compare your results to $\bbR^2$ with the flat metric.
}



\prob{
	A spherical triangle is made from three points on the sphere pairwise connected by geodesics.  Let the angles on the triangle be $\alp$, $\bet$, and $\gam$.  By drawing pictures, show that $\alp + \bet + \gam$ can be larger than $\pi$.
}



\prob{
	Define the excess angle $E$ of a spherical triangle by $E = \alp + \bet + \gam - \pi$.  Prove that the area of the triangle is $R^2 E$.
}






\clearpage
\state{}{
	In this problem you will explore the geometry on the space of possible inertial velocities.
}

\prob{ \label{4a}
	Suppose two inertial frames move with 3-velocities $\vvq$ and $\vvw$ relative to a fixed inertial frame.  Show that their relative velocity $\vv$ has magnitude $v$ given by
	\eq{
		v^2 = \frac{(\vvq - \vvw)^2 - (\vvq \times \vvw)^2}{(1 - \vvq \cdot \vvw)^2}.
	}
}



\prob{
	We define a metric on the space of all possible 3-velocities by defining the distance between two nearby velocities to be their relative velocity.  Using the result from \ref{4a}, show that this metric is
	\eq{
		\dds^2 = \ddchi^2 + \sinh[2](\chi) (\ddtht^2 + \sin[2](\tht) \ddphi^2),
	}
	where $\chi$ is the rapidity $v = \tanh(\chi)$, and $\tht, \phi$ are polar and azimuthal angles defined relative to $\vv$.
}



\prob{
	Show that the geodesics of this metric are paths of minimum fuel use for a rocket ship changing its velocity.
}



\prob{
	A rocket ship in interstellar travel with velocity $\vvq$ relative to earth changes to a new velocity $\vvw$ in a manner that uses the least amount of fuel.  What is the ship’s smallest velocity relative to earth during the change?
}


\makebib

\end{document}
