\state{Gravitational redshift~(MCP 24.16)}{
	Inside a laboratory on Earth's surface the effects of spacetime curvature are so small that current technology cannot measure them.  Therefore, experiments performed in the laboratory can be analyzed using special relativity. %  (This fact is embodied in Einstein's equivalence principle.)
}

\prob{
	Explain why the spacetime metric in the proper reference frame of the laboratory's floor has the form
	\eqn{metric2}{
		\dds^2 = (1 + 2 g z) (\ddxoh)^2 + \ddx^2 + \ddy^2 + \ddz^2,
	}
	plus terms due to the slow rotation of the laboratory walls, which we neglect in this exercise.  Here $g$ is the acceleration of gravity measured on the floor.
}

\sol{
	We can transform coordinates from the proper reference frame of the laboratory floor to another inertial frame.  We choose this other frame such that it is only a very small ``distance'' away at $\vx = 0$ in the proper frame.  That is, the frames are identical in the immediate vicinity of event (small $\vx$).  Then the coordinate transformation from the proper reference frame to the other inertial frame is given by MCP~(24.60a),
	\al{
		\xii &= \xiih + \frac{1}{2} \aih (\xoh)^2 + \epsisjkh \Omgjh \xkh \xoh, &
		\xo &= \xoh (1 + \asjh \xjh),
	}
	where terms to quadratic order in $\xah$ are included, and $\Omgjh$ is the rotational angular velocity of the laboratory.  Since the metric in the inertial frame is $\dds^2 = -(\ddxo)^2 + \delsij \ddxii \ddxj$~\cite[p.~1183]{MCP}, the metric in the proper reference frame is given by MCP~(24.60b),
	\eq{
		\dds^2 = -(1 + 2 \ba \vdot \bx) (\ddxoh)^2 + 2 (\bOmg \cross \bx) \vdot \ddbx \ddxoh + \delsjk \ddxjh \ddxkh,
	}
	which is accurate to linear order in $\xah$.  For this problem, we ignore rotations so $\bOmg \to \bo$, eliminating the second term.  Also, $\ba = g \zh$ so $\ba \vdot \bx = -g z$.  Finally, we note that $\xiih \in \{ x, y, z \}$ since these coordinates coincide with the inertial frame~\cite[p.~1186]{MCP}.  Then the spacetime metric is
	\eq{
		\dds^2 = -(1 + 2 g z) (\ddxoh)^2 + \ddx^2 + \ddy^2 + \ddz^2,
	}
	which is what we want.  Evidently, there is a typo (missing minus sign) in the problem statement. \qed
}


\prob{
	An electromagnetic wave is emitted from the floor, where it is measured to have wavelength $\lamo$, and is received at the ceiling.  Using the metric of Eq.~\refeq{metric2}, show that, as measured in the proper reference frame of an observer on the ceiling, the received wave has wavelength $\lamr = \lamo (1 + g h)$, where $h$ is the height of the ceiling above the floor (i.e., the light is \emph{gravitationally redshifted} by $\Del\lam / \lamo = g h$).
%	
%	Hint: Show that all crests of the wave must travel along world lines that have the same shape, $z = F(\xoh - \xohe)$, where $F$ is some function, and $\xohe$ is the coordinate time at which the crest is emitted from the floor. %  You can compute the shape function $F$ if you wish, but it is not needed to derive the gravitational redshift; only its universality is needed.]
}

\sol{
	As measured by an observer on the floor, say that one crest of the wave is emitted at some time $\xoh = t$, and the next crest is emitted at $\xoh = \tw$.  Call these events $\cPq$ and $\cPw$.  Both events occur on the floor at $z = 0$.  The interval between the two events is
	\eq{
		(\ddsqw)^2 = -(\tw - \tq)^2
		\equiv \To^2,
	}
	where $\To$ is the period of the wave as measured at the floor.  We know that this interval represents the period because it is a timelike separation~\cite[p.~45]{MCP}.  At some time later $\te$, the first crest hits the ceiling; at $\trr$, the second crest hits the ceiling.  The interval between these two events $\cPe$ and $\cPr$, which both occur at $z = R + h$, is
	\eq{
		(\ddser)^2 = -(1 + 2 g h) (\trr - \te)^2
		\equiv \Trr^2,
	}
	where $\Trr$ is the period of the wave as measured at the ceiling.
	
	We are concerned with the propagation of an electromagnetic wave, which means each crest travels at the speed of light.  Thus, it must be true that $\tw - \tq = \trr - \te$.  Noting that $\Trr / \To = \lamr / \lamo$, it follows that
	\eq{
		\frac{\lamr^2}{\lamo^2} = \frac{(1 + 2 g h) (\trr - \te)^2}{(\tw - \tq)^2}
		= 1 + 2 g h
		\qimplies
		\frac{\lamr}{\lamo} = \sqrt{1 + 2 g h} \approx 1 + g h,
	}
	in the limit that $g h \ll 1$ (where $c = 1$).  Thus we have shown that \ans{$\lamr = \lamo (1 + g h)$}. \qed
}