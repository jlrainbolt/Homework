\documentclass[11pt]{article}
\usepackage{homework}

\classname{364}
\homeworknum{4}

\DeclareMathAlphabet{\mathsfit}{T1}{\sfdefault}{\mddefault}{\sldefault}


\begin{document}

% Environments

\newcommand{\state}[2]{\begin{statement}{#1} #2 \end{statement}}
\newcommand{\prob}[2]{\begin{problem}{#1} #2 \end{problem}}
\newcommand{\subprob}[1]{\begin{subproblem} #1 \end{subproblem}}
\newcommand{\sol}[1]{\begin{solution} #1 \end{solution}}
\newcommand{\fig}[2]{\begin{figure} \centering #2  \label{#1} \end{figure}}

\newcommand{\makebib}{
	\vfill
	\color{black}
	\nocite{*}
	\bibliography{references}{}
	\bibliographystyle{lucas_unsrt}
}
	

% Implication

\newcommand{\qwhere}{\quad \text{where} \quad}
\newcommand{\qimplies}{\quad \implies \quad}
\newcommand{\impliesq}{\implies \quad}



% Brackets

\newcommand{\paren}[1]{\left( #1 \right)}
\newcommand{\brac}[1]{\left[ #1 \right]}
\newcommand{\curly}[1]{\left\{ #1 \right\}}


% Greek

\newcommand{\alp}{\alpha}
\newcommand{\bet}{\beta}
\newcommand{\gam}{\gamma}
\newcommand{\del}{\delta}
\newcommand{\eps}{\epsilon}
\newcommand{\zet}{\zeta}
\newcommand{\tht}{\theta}
\newcommand{\kap}{\kappa}
\newcommand{\lam}{\lambda}
\newcommand{\sig}{\sigma}
\newcommand{\ups}{\upsilon}
\newcommand{\omg}{\omega}

\newcommand{\Gam}{\Gamma}
\newcommand{\Del}{\Delta}
\newcommand{\Tht}{\Theta}
\newcommand{\Lam}{\Lambda}
\newcommand{\Sig}{\Sigma}
\newcommand{\Omg}{\Omega}


% Text

\newcommand{\where}{\text{where }}

% Problem 1

\newcommand{\Hint}{H_\text{int}}
\newcommand{\ddcx}{\dd[3]{x}}
\newcommand{\psib}{\bar{\psi}}

\newcommand{\mh}{m_h}
\newcommand{\mmu}{m_\mu}
\newcommand{\me}{m_e}
\newcommand{\ma}{m_a}

\newcommand{\aexpt}{a_\text{expt.}}
\newcommand{\aQED}{a_\text{QED}}
\renewcommand{\GeV}{\giga\electronvolt}

\newcommand{\gamt}{\gam^5}

\state{}{
	Consider a particle which, as viewed by an observer in an inertial lab, is in a circular orbit in the $(x, y)$ plane with angular velocity $\omg$ and radius $r$.  Suppose that this particle carries a spin angular momentum $\vS$ (treated classically in this problem) which is Fermi-Walker transported.  Compute the time dependence of this angular momentum $\vSt$ where $t$ is the inertial time in the laboratory frame.  Show that in the non-relativistic limit, the complex vector $\Sx + i \Sy$ precesses about the $z$ axis with frequency $\omgT = r^2 \omg^3 / 2$.
}

\sol{
	Since the particle is in a circular orbit in the $xy$ plane, we can write its position in the lab frame as
	\eq{
		\vx = ( t, r \cos(\omg \tau), r \sin(\omg \tau), 0 ).
	}
	Then from MCP~(2.7), its velocity is
	\eqn{U}{
		\vU = \dv{\vx}{\tau}
		= ( \gam, -r \omg \sin(\omg \tau), r \omg \cos(\omg \tau), 0 ),
	}
	since $\ddtau = \ddt / \gam$~\cite[p.~201]{Resnick}.  The particle's acceleration is
	\eq{
		\vaa = \dv{\vU}{\tau}
		= ( \gam \gamd, -r \omg^2 \cos(\omg \tau), -r \omg^2 \sin(\omg \tau), 0 ),
	}
	where $\gamd = \dv*{\gam}{t}$~\cite{Acceleration}.  However, we do not need to bother computing this component explicitly since the Fermi-Walker transport is, as given by MCP~(24.62),
	\eqn{FW}{
		\nabsU \vS = \vU (\vaa \cdot \vS),
	}
	and we know that the spin vector is always orthogonal to the particle's 4-velocity~\cite[p.~1184]{MCP}.  This means that the spin vector can be written
	\eq{
		\vS = (0, \bS)
	}
	since $\vU \cdot \vS = 0$ is Lorentz invariant, and the spatial components of the particle's velocity are zero in its rest frame.  Moreover, this means
	\eq{
		\nabsU \vS = -\dv{\vS}{\tau},
	}
	so Eq.~\refeq{FW} can be written
	\eq{
		\dv{\vS}{\tau} = -\vU (\vaa \cdot \vS)
		= r \omg^2 [ \Sx \cos(\omg \tau) + \Sy \sin(\omg \tau) ] \vU.
	}
	Feeding in the relevant components of Eq.~\refeq{U}, we have the system of coupled differential equations
	\al{
		\dv{\Sx}{\tau} &= -r^2 \omg^3 \sin(\omg \tau) [ \Sx \cos(\omg \tau) + \Sy \sin(\omg \tau) ], \\
		\dv{\Sy}{\tau} &= r^2 \omg^3 \cos(\omg \tau) [ \Sx \cos(\omg \tau) + \Sy \sin(\omg \tau) ].
	}
%	which is equivalent to the matrix equation
%	\eq{
%		\dv{\tau} \mqty[ \Sx \\ \Sy ] = r^2 \omg^3 \mqty[
%			-\sin(\omg \tau) \cos(\omg \tau) & -\sin^2(\omg \tau) \\
%			\cos^2(\omg \tau) & \sin(\omg \tau) \cos(\omg \tau)
%			] \mqty[ \Sx \\ \Sy ].
%	}
%	We can solve this using the characteristic equation of the eigenvalue problem.  According to Mathematica, the eigenvalues are $\lamq = \lamw = 0$, and the eigenvectors are $\vq = ( -\tan(\omg \tau), 1)$ and $\vw = (0, 0)$.  This gives us the solutions~\cite[pp.~130--131]{Strogatz}
%	\eq{
%		\mqty[ \Sx \\ \Sy ] = r^2 \omg^3 \paren{ \cq e^{\lam1 \tau} \vq + \cw e^{\lam2 \tau} \vw }
%		= \cq r^2 \omg^3 \mqty[ -\tan(\omg \tau) \\ 1 ],
%	}
%	where $\cq$ is a constant determined by initial conditions.  We can choose $\Sx(0) = 0$ and $\Sy(0) = S$, where $S = |S|$.
%	
%	\hl{this is not right}
%	
%	Let $\omg' = \omg / \gam$, so
%	\eq{
%		\vU(t) = \gam ( 1, -\omg' r \sin(\omg' t), \omg' r \cos(\omg' t), 0 ).
%	}
	
}





\clearpage
\state{Gravitational redshift~(MCP 24.16)}{
	Inside a laboratory on Earth's surface the effects of spacetime curvature are so small that current technology cannot measure them.  Therefore, experiments performed in the laboratory can be analyzed using special relativity. %  (This fact is embodied in Einstein's equivalence principle.)
}

\prob{
	Explain why the spacetime metric in the proper reference frame of the laboratory's floor has the form
	\eqn{metric2}{
		\dds^2 = (1 + 2 g z) (\ddxoh)^2 + \ddx^2 + \ddy^2 + \ddz^2,
	}
	plus terms due to the slow rotation of the laboratory walls, which we neglect in this exercise.  Here $g$ is the acceleration of gravity measured on the floor.
}

\sol{
	We can transform coordinates from the proper reference frame of the laboratory floor to another inertial frame.  We choose this other frame such that it is only a very small ``distance'' away at $\vx = 0$ in the proper frame.  That is, the frames are identical in the immediate vicinity of event (small $\vx$).  Then the coordinate transformation from the proper reference frame to the other inertial frame is given by MCP~(24.60a),
	\al{
		\xii &= \xiih + \frac{1}{2} \aih (\xoh)^2 + \epsisjkh \Omgjh \xkh \xoh, &
		\xo &= \xoh (1 + \asjh \xjh),
	}
	where terms to quadratic order in $\xah$ are included, and $\Omgjh$ is the rotational angular velocity of the laboratory.  Since the metric in the inertial frame is $\dds^2 = -(\ddxo)^2 + \delsij \ddxii \ddxj$~\cite[p.~1183]{MCP}, the metric in the proper reference frame is given by MCP~(24.60b),
	\eq{
		\dds^2 = -(1 + 2 \ba \vdot \bx) (\ddxoh)^2 + 2 (\bOmg \cross \bx) \vdot \ddbx \ddxoh + \delsjk \ddxjh \ddxkh,
	}
	which is accurate to linear order in $\xah$.  For this problem, we ignore rotations so $\bOmg \to \bo$, eliminating the second term.  Also, $\ba = g \zh$ so $\ba \vdot \bx = -g z$.  Finally, we note that $\xiih \in \{ x, y, z \}$ since these coordinates coincide with the inertial frame~\cite[p.~1186]{MCP}.  Then the spacetime metric is
	\eq{
		\dds^2 = -(1 + 2 g z) (\ddxoh)^2 + \ddx^2 + \ddy^2 + \ddz^2,
	}
	\hl{which is not the same as what we want, but I can't figure out what's going on with the minus sign.}
}


\prob{
	An electromagnetic wave is emitted from the floor, where it is measured to have wavelength $\lamo$, and is received at the ceiling.  Using the metric of Eq.~\refeq{metric2}, show that, as measured in the proper reference frame of an observer on the ceiling, the received wave has wavelength $\lamr = \lamo (1 + g h)$, where $h$ is the height of the ceiling above the floor (i.e., the light is \emph{gravitationally redshifted} by $\Del\lam / \lamo = g h$).
	
	Hint: Show that all crests of the wave must travel along world lines that have the same shape, $z = F(\xoh - \xohe)$, where $F$ is some function, and $\xohe$ is the coordinate time at which the crest is emitted from the floor. %  You can compute the shape function $F$ if you wish, but it is not needed to derive the gravitational redshift; only its universality is needed.]
}





\clearpage
\state{Rigidly rotating disk~(MCP 24.17)}{
	Consider a thin disk with radius $R$ at $z = 0$ in a Lorentz reference frame.  The disk rotates rigidly with angular velocity $\Omg$.  In the early years of special relativity there was much confusion over the geometry of the disk: In the inertial frame it has physical radius (proper distance from center to edge) $R$ and physical circumference $\sC = 2\pi R$.  But Lorentz contraction dictates that, as measured on the disk, the circumference should be $\sqrt{1 - v^2} \sC$ (with $v = \Omg R$), and the physical radius, $R$, should be unchanged.  This seemed weird.  How could an obviously flat disk in spacetime have a curved, non-Euclidean geometry, with physical circumference divided by physical radius smaller than $2\pi$?  In this exercise you will explore this issue.
}

\prob{
	Consider a family of observers who ride on the edge of the disk.  Construct a circular curve, orthogonal to their world lines, that travels around the disk (at $\sqrt{x^2 + y^2} = R$).  This curve can be thought of as lying in a 3-surface of constant time $\xoh$ of the observers' proper reference frames.  Show that it spirals upward in a Lorentz-frame spacetime diagram, so it cannot close on itself after traveling around the disk.  Thus the 3-planes, orthogonal to the observers' world lines at the edge of the disk, cannot mesh globally to form global 3-planes.
}

\sol{
	\hl{I understand the family's world lines, but wtf is orthogonal?  Is it not just spiraling in the opposite direction?  Why not just use the world line of one observer then??}
}



\prob{
	Next, consider a 2-dimensional family of observers who ride on the surface of the rotating disk.  Show that at each radius $\sqrt{x^2 + y^2} = \const$, the constant-radius curve that is orthogonal to their world lines spirals upward in spacetime with a different slope.  Show that this means that even locally, the 3-planes orthogonal to each of their world lines cannot mesh to form larger 3-planes---thus there does not reside in spacetime any 3-surface orthogonal to these observers' world lines.  There is no 3-surface that has the claimed non-Euclidean geometry.
}

\sol{
	\hl{obviously the world lines themselves are going to have a greater slope closer to the edge of the disk.  But what is orthogonal???}
}






\clearpage
\state{Constant of geodesic motion in a spacetime with symmetry~(MCP 25.4)}{\hfix}

\prob{
	Suppose that in some coordinate system the metric coefficients are independent of some specific coordinate $\xA$: $\sgsabA = 0$ (e.g., in spherical polar coordinates $\{ t, r, \tht, \phi \}$ in flat spacetime $\sgsabphi = 0$, so we could set $\xA = \phi$).  Show that
	\eq{
		\psA \equiv \vp \cdot \pdv{\xA}
	}
	is a constant of the motion for a freely moving particle [$\psp = (\text{conserved $z$~component of angular momentum})$] in the above, spherically symmetric example.]
	
	Hint: Show that the geodesic equation can be written in the form
	\eq{
		\dv{\psa}{\zet} - \Gamsman \pmu \pnu = 0,
	}
	where $\Gamsman$ is the covariant connection of Eqs.~(24.38c), (24.38d) with $\csabg = 0$, because we are using a coordinate basis.
	
	Note the analogy of the constant of motion $\psA$ with Hamiltonian mechanics; there, if the Hamiltonian is independent of $\xA$, then the generalized momentum $\psA$ is conserved; here, if the metric coefficients are independent of $\xA$, then the covariant component $\psA$ of the momentum is conserved.
}



\prob{
	As an example, consider a particle moving freely through a time-independent, Newtonian gravitational field.  In Ex.~25.18, we learn that such a gravitational field can be described in the language of general relativity by the spacetime metric
	\eq{
		\dds^2 = -(1 + 2 \Phi) \ddt^2 + (\delsjk + \hsjk) \ddxj \ddxk,
	}
	where $\Phi(x, y, z)$ is the time-independent Newtonian potential, and $\hsjk$ are contributions to the metric that are independent of the time coordinate $t$ and have magnitude of order $\abs{\Phi}$.  That the gravitational field is weak means $\abs{\Phi} \ll 1$ (or, in conventional units, $\abs{\Phi / c^2} \ll 1$).  The coordinates being used are Lorentz, aside from tiny corrections of order $\abs{\Phi}$, and as this exercise and Ex.~25.18 show, they coincide with the coordinates of the Newtonian theory of gravity.  Suppose that the particle has velocity $\vj \equiv \dd*{\xj}{t}$ through this coordinate system that is $\lesssim \abs{\Phi}^{1/2}$ and thus is small compared to the speed of light.  Because the metric is independent of the time coordinate $t$, the component $\pst$ of the particle's 4-momentum must be conserved along its world line.  Since throughout physics, the conserved quantity associated with time-translation invariance is always the energy, we expect that $\pst$, when evaluated accurate to first order in $\abs{\Phi}$, must be equal to the particle's conserved Newtonian energy, $E = m \Phi + m \vj \vk \delsjk / 2$, aside from some multiplicative and additive constants.  Show that this, indeed, is true, and evaluate the constants.
}





\clearpage
\state{Killing vector field~(MCP 25.5)}{
	A \emph{Killing vector field} is a coordinate-independent tool for exhibiting symmetries of the metric.  It is any vector field $\vxi$ that satisfies
	\eq{
		\xisascb + \xisbsca = 0
	}
	(i.e., any vector field whose symmetrized gradient vanishes).
}

\prob{ \label{5a}
	Let $\vxi$ be a vector field that might or might not be Killing.  Show, by construction, that it is possible to introduce a coordinate system in which $\vxi = \pdv*{\xA}$ for some coordinate $\xA$.
}

\sol{
	If we treat $\vxi$ as a singular vector, it must exist in a space tangent to some manifold at a point we call $\cP$.  We can imagine a projection of $\vxi$ onto the manifold; such a projection is a curve on the manifold.  We may choose the direction of this curve as a coordinate of the manifold and call it $\xA$.  The curve itself we call $\cP(\xA)$.  Then, by construction, $\vxi$ is tangent to $\cP(\xA)$.  This means that $\vxi$ must be the directional derivative along $\cP(\xA)$, which is $\pdv*{\xA}$.  Thus, $\vxi = \pdv*{\xA}$~\cite[pp.~1166--1167]{MCP}. \qed
}



\prob{
	Show that in the coordinate system of \ref{5a} the symmetrized gradient of $\vxi$ is $\xisascb + \xisbsca = \pdv{\sgsab}{\xA}$.  From this infer that a vector field $\vxi$ is Killing if and only if there exists a coordinate system in which (i) $\vxi = \pdv*{\xA}$ and (ii) the metric is independent of $\xA$.
}

\sol{
	By MCP~(24.36),
	\eq{
		A_{\alp; \bet} = A_{\alp, \bet} - \Gammsab A_\mu.
	}
	So
	\al{
		\xisascb &= \xisacb - \Gammsab \xism
		= \xisacb - \pdv{\GamAsab}{\xA}, &
		\xisbsca &= \xisbca - \Gamnsba \xisn
		= \xisbca - \pdv{\GamAsba}{\xA},
	}
	since $\vxi$ has a component in only the $\xA$ direction.  The basis of \ref{5a} is a coordinate basis, so the $c$ terms vanish in MCP~(24.38), which becomes
	\eq{
		\Gamsabg = \frac{1}{2} (\sgsabg + \sgsagb - \sgsbga),
	}
	and from (24.38d),
	\eq{
		\Gammsbg = \frac{1}{2} \sgma (\sgsabg + \sgsagb - \sgsbga)
	}
	
	
%	the $\Gammsab$ are symmetric in their last two indices~\cite[p.~1172]{MCP}.  This means
%	\eq{
%		\xisascb + \xisbsca = \ptsvesb \xisa + \ptsvesa \xisb - 2 \Gammsab \xism
%	}
}



\prob{
	Use Killing's equation~(25.18) to show, without introducing a coordinate system, that, if $\vxi$ is a Killing vector field and $\vp$ is the 4-momentum of a freely-falling particle, then $\vxi \cdot \vp$ is conserved along the particle's geodesic world line.  This is the same conservation law as we proved in Ex.~25.4a using a coordinate-dependent calculation.
}


%\makebib

\end{document}
