\state{}{
	Consider a particle which, as viewed by an observer in an inertial lab, is in a circular orbit in the $(x, y)$ plane with angular velocity $\omg$ and radius $r$.  Suppose that this particle carries a spin angular momentum $\vS$ (treated classically in this problem) which is Fermi-Walker transported.  Compute the time dependence of this angular momentum $\vSt$ where $t$ is the inertial time in the laboratory frame.  Show that in the non-relativistic limit, the complex vector $\Sx + i \Sy$ precesses about the $z$ axis with frequency $\omgT = r^2 \omg^3 / 2$.
}

\sol{
	Since the particle is in a circular orbit in the $xy$ plane, we can write its position in the lab frame as
	\eq{
		\vx = ( t, r \cos(\omg t), r \sin(\omg t), 0 ).
	}
	Then from MCP~(2.7), its velocity is
	\eqn{U}{
		\vU = \dv{\vx}{\tau}
		= \gam ( 1, -r \omg \sin(\omg t), r \omg \cos(\omg t), 0 ),
	}
	since $\ddtau = \ddt / \gam$~\cite[p.~201]{Resnick}.  The particle's acceleration is
	\eq{
		\vaa = \dv{\vU}{\tau}
		= -r \omg^2 \gam^2 ( 0, \cos(\omg t), \sin(\omg t), 0 ).
	}
	The Fermi-Walker transport is, as given by MCP~(24.62),
	\eqn{FW}{
		\nabsU \vS = \vU (\vaa \cdot \vS),
	}
	and we know that the spin vector is always orthogonal to the particle's 4-velocity~\cite[p.~1184]{MCP}.  This means that the spin vector can be written
	\eq{
		\vS = (0, \bS)
	}
	since $\vU \cdot \vS = 0$ is Lorentz invariant, and the spatial components of the particle's velocity are zero in its rest frame.  Moreover, this means
	\eq{
		\nabsU \vS = \dv{\vS}{\tau}
		= \gam \dv{\vS}{t},
	}
	so Eq.~\refeq{FW} can be written
	\eq{
		\dv{\vS}{t} = \frac{1}{\gam} \vU (\vaa \cdot \vS)
		= -\gam r \omg^2 [ \Sx \cos(\omg t) + \Sy \sin(\omg t) ] \vU.
	}
	Feeding in the relevant components of Eq.~\refeq{U}, we have the system of coupled differential equations
	\al{
		\dv{\Sx}{t} &= \gam^2 r^2 \omg^3 \sin(\omg t) [ \Sx \cos(\omg t) + \Sy \sin(\omg t) ], \\
		\dv{\Sy}{t} &= -\gam^2 r^2 \omg^3 \cos(\omg t) [ \Sx \cos(\omg t) + \Sy \sin(\omg t) ],
	}
	or, in matrix form,
	\eqn{sysxy}{
		\dv{t} \mqty[ \Sx \\ \Sy ] = \gam^2 r^2 \omg^3 \mqty[
				\cos(\omg t) \sin(\omg t) & \sin[2](\omg t) \\
				-\cos[2](\omg t) & -\cos(\omg t) \sin(\omg t)
			] \mqty[ \Sx \\ \Sy ].
	}
	To solve the system, we define the polar components of $\vS$ by~\cite[inside cover]{Griffiths}
	\al{
		\Sr &= \Sx \cos(\omg t) + \Sy \sin(\omg t), &
		\Stht &= -\Sx \sin(\omg t) + \Sy \cos(\omg t).
	}
	Then we can transform into these polar coordinates using the matrix~\cite[inside cover]{Griffiths}
	\eqn{sub1}{
		\mqty[ \Sx \\ \Sy ] = \mqty[
				\cos(\omg t) & -\sin(\omg t) \\
				\sin(\omg t) & \cos(\omg t)
			] \mqty[ \Sr \\ \Stht ],
	}
	which implies
	\aln{
		\dv{t} \mqty[ \Sx \\ \Sy ] &= \dv{t} \mqty[
				\cos(\omg t) & -\sin(\omg t) \\
				\sin(\omg t) & \cos(\omg t)
			] \mqty[ \Sr \\ \Stht ] + \mqty[
				\cos(\omg t) & -\sin(\omg t) \\
				\sin(\omg t) & \cos(\omg t)
			] \dv{t} \mqty[ \Sr \\ \Stht ] \notag \\
		&= -\omg \mqty[
			\sin(\omg t) & \cos(\omg t) \\
			-\cos(\omg t) & \sin(\omg t)
		] \mqty[ \Sr \\ \Stht ] + \mqty[
				\cos(\omg t) & -\sin(\omg t) \\
				\sin(\omg t) & \cos(\omg t)
			] \dv{t} \mqty[ \Sr \\ \Stht ]. \label{sub2}
	}
	Substituting Eqs.~\refeq{sub1} and \refeq{sub2} into Eq.~\refeq{sysxy} yields
	\al{
		-\omg &\mqty[
				\sin(\omg t) & \cos(\omg t) \\
				-\cos(\omg t) & \sin(\omg t)
			] \mqty[ \Sr \\ \Stht ] + \mqty[
				\cos(\omg t) & -\sin(\omg t) \\
				\sin(\omg t) & \cos(\omg t)
			] \dv{t} \mqty[ \Sr \\ \Stht ] \\
		&\hspace{10em} = \gam^2 r^2 \omg^3 \mqty[
				\cos(\omg t) \sin(\omg t) & \sin[2](\omg t) \\
				-\cos[2](\omg t) & -\cos(\omg t) \sin(\omg t)
			] \mqty[
				\cos(\omg t) & -\sin(\omg t) \\
				\sin(\omg t) & \cos(\omg t)
			] \mqty[ \Sr \\ \Stht ] \\
		&\hspace{10em} = \gam^2 r^2 \omg^3 \mqty[
				\sin(\omg t) & 0 \\
				-\cos(\omg t) & 0
			] \mqty[ \Sr \\ \Stht ],
	}
	where the matrix multiplication has been carried out with Mathematica.  This implies
	\al{
		\mqty[
				\cos(\omg t) & -\sin(\omg t) \\
				\sin(\omg t) & \cos(\omg t)
			] \dv{t} \mqty[ \Sr \\ \Stht ]
		&= \paren{ \omg \mqty[
				\sin(\omg t) & \cos(\omg t) \\
				-\cos(\omg t) & \sin(\omg t)
			] + \gam^2 r^2 \omg^3 \mqty[
				\sin(\omg t) & 0 \\
				-\cos(\omg t) & 0
			] } \mqty[ \Sr \\ \Stht ] \\
		&= \omg \mqty[
				(1 + \gam^2 r^2 \omg^2) \sin(\omg t) & \cos(\omg t) \\
				-(1 + \gam^2 r^2 \omg^2) \cos(\omg t) & \sin(\omg t)
			] \mqty[ \Sr \\ \Stht ].
	}
	Note that
	\eq{
		\gam^2 = \frac{1}{1 - r^2 \omg^2}
		\qimplies
		\gam^2 = \gam^2 r^2 \omg^2 + 1.
	}
	Making this substiution and multiplying both sides by the inverse of the first matrix, we find
	\eq{
		\dv{t} \mqty[ \Sr \\ \Stht ] = \omg \mqty[
				\cos(\omg t) & \sin(\omg t) \\
				-\sin(\omg t) & \cos(\omg t)
			] \mqty[
				\gam^2 \sin(\omg t) & \cos(\omg t) \\
				-\gam^2 \cos(\omg t) & \sin(\omg t)
			] \mqty[ \Sr \\ \Stht ]
		= \omg \mqty[
				0 & 1 \\
				-\gam^2 & 0
			] \mqty[ \Sr \\ \Stht ].
	}
	In other words, we have a system of coupled first-order ODEs:
	\aln{ \label{ODEs}
		\dv{\Sr}{t} &= \omg \Stht, &
		\dv{\Stht}{t} &= -\omg \gam^2 \Sr.
	}
	Differentiating each equation by $t$ once more and substituting, we get a system of uncoupled second-order ODEs with well-known solutions:
	\al{
		\dv[2]{\Sr}{t} &= \omg \dv{\Stht}{t}
		= -\omg^2 \gam^2 \Sr, &
		%
		\dv[2]{\Stht}{t} &= -\omg \gam^2 \dv{\Sr}{t}
		= -\omg^2 \gam^2 \Stht.
	}
	The solutions are~\cite[p.~207]{Swartz}
	\al{
		\Sr(t) &= \Cq \cos(\omg \gam t) + \Cw \sin(\omg \gam t), &
		\Stht(t) &= \Dq \cos(\omg \gam t) + \Dw \sin(\omg \gam t).
	}
	We choose the initial conditions $\Sr(0) = S$ and $\Stht(0) = 0$.  This gives us
	\al{
		S &= \Sr(0) = \Cq, &
		0 &= \Stht(0) = \Dq.
	}
	Then by Eq.~\refeq{ODEs},
	\al{
		\omg \Dw \sin(\omg \gam t) &= \dv{\Sr}{t} = \omg \gam [ -S \sin(\omg \gam t) + \Cw \cos(\omg \gam t) ], \\
		-\omg \gam^2 [ S \cos(\omg \gam t) + \Cw \sin(\omg \gam t) ] &= \dv{\Stht}{t} = \omg \gam \Dw \cos(\omg \gam t).
	}
	Solving this system of equations with Mathematica yields $\Cw = 0$ and $\Dw = -\gam S$.  So our equations are
	\al{
		\Sr(t) &= S \cos(\omg \gam t), &
		\Stht(t) &= -\gam S \sin(\omg \gam t).
	}
	Transforming back into Cartesian components using Eq.~\refeq{sub1}, we have
	\eq{
		\mqty[ \Sx \\ \Sy ] = S \mqty[
				\cos(\omg t) & -\sin(\omg t) \\
				\sin(\omg t) & \cos(\omg t)
			] \mqty[
				\cos(\omg \gam t) \\
				-\gam \sin(\omg \gam t)
			]
		= S \mqty[
				\cos(\omg t) \cos(\omg \gam t) + \gam \sin(\omg t) \sin(\omg \gam t) \\
				\sin(\omg t) \cos(\omg \gam t) - \gam \cos(\omg t) \sin(\omg \gam t)
			].
	}
	We can make these expressions look a little nicer by using the following product-to-sum identities~\cite{Trig}:
	\al{
		2 \cos\tht \cos\phi &= \cos(\tht - \phi) + \cos(\tht + \phi), &
		2 \sin\tht \sin\phi &= \cos(\tht - \phi) - \cos(\tht + \phi), \\
		2 \sin\tht \cos\phi &= \sin(\tht + \phi) + \sin(\tht - \phi), &
		2 \cos\tht \sin\phi &= \sin(\tht + \phi) - \sin(\tht - \phi).
	}
	Then we have
	\al{
		\Sx(t) &\propto \frac{\cos(\omg t - \omg \gam t) + \cos(\omg t + \omg \gam t)}{2} + \gam \frac{\cos(\omg t - \omg \gam t) - \cos(\omg t + \omg \gam t)}{2}, \\
		\Sy(t) &\propto \frac{\sin(\omg t + \omg \gam t) + \sin(\omg t - \omg \gam t)}{2} - \gam \frac{\sin(\omg t + \omg \gam t) - \sin(\omg t - \omg \gam t)}{2},
	}
	and finally
	\ans{\al{
		\Sx(t) &= S \frac{(1 + \gam) \cos[(1 - \gam) \omg t] + (1 - \gam) \cos[(1 + \gam) \omg t]}{2}, \\
		\Sy(t) &= S \frac{(1 + \gam) \sin[(1 - \gam) \omg t] + (1 - \gam) \sin[(1 + \gam) \omg t]}{2}.
	}}%
	is the time dependence of $\vS$.
	
	Then
	\aln{
		\Sx + i \Sy &= S \brac{ (1 + \gam) \frac{\cos[(1 - \gam) \omg t] + i \sin[(1 - \gam) \omg t]}{2} + (1 - \gam) \frac{\cos[(1 + \gam) \omg t] + i \sin[(1 + \gam) \omg t]}{2} } \notag \\
		&= \frac{S}{2} \brac{ (1 + \gam) e^{i (1 - \gam) \omg t} + (1 - \gam) e^{i (1 - \gam) \omg t} }. \label{thing1z}
	}
	In the non-relativistic limit, $r \omg \ll 1$ so
	\eq{
		\gam = \frac{1}{\sqrt{1 - r^2 \omg^2}}
		\approx 1 + \frac{r^2 \omg^2}{2},
	}
	where we have evaluated the series expansion with Mathematica.  Using also $\gam \to 1$ in this limit, Eq.~\refeq{thing1z} becomes
	\eq{
		\ans{ \Sx + i \Sy }\to S e^{i (1 - \gam) \omg t}
		\approx S \exp[ i \paren{ 1 - 1 - \frac{r^2 \omg^2}{2} } \omg t ]
		\ans{\ = S e^{-i r^2 \omg^3 t / 2}. }
	}
	So we have shown that $\Sx + i \Sy$ precesses about the $z$ axis with frequency $\omgT = r^2 \omg^3 / 2$. \qed
}