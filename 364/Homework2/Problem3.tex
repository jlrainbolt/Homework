\state{Index-manipulation rules from duality (MCP 24.4)}{
	For an arbitrary basis $\{ \vesa \}$ and its dual basis $\{ \vem \}$, use (i)~the duality relation~(24.8), (ii)~the definition~(24.9) of components of a tensor, and (iii)~the relation $\vA \cdot \vB = \sg(\vA, \vB)$ between the metric and the inner product to deduce the following results.
}

\prob{ \label{3a}
	The relations
	\al{
		\vem &= \sgma \vesa, &
		\vesa &= \sgsam \vem.
	}
	\vfix
}

\sol{
	MCP~(24.8) is
	\eqn{24.8}{
		\vem \cdot \vesb = \sg(\vem, \vesb)
		= \delmsb,
	}
	and MCP~(24.9) is
	\aln{ \label{24.9}
		\Fmn &= \sF(\vem, \ven), &
		\Fsab &= \sF(\vesa, \vesb), &
		\Fmsb &= \sF(\vem, \vesb).
	}
	For the first relation, we take the dot product of both sides with $\veb$.  Beginning with the right-hand side,
	\eq{
		(\sgma \vesa) \cdot \veb = \sgma (\veb \cdot \vesa)
		= \sgma \sg(\veb, \vesa)
		= \sgma \delbsa
		= \sgmb
		= \sg(\vem, \vesb)
		= \vem \cdot \veb
	}
	which proves that \ans{$\vem = \sgma \vesa$.} \qed
	
	For the second relation, the proof follows the same path:
	\eq{
		(\sgsam \vem) \cdot \vesb = \sgsam (\vesb \cdot \vem)
		= \sgsam \sg(\vesb, \vem)
		= \sgsam \delsbsm
		= \sgsab
		= \sg(\vesa, \vesb)
		= \vesa \cdot \vesb,
	}
	which proves that \ans{$\vesa = \sgsam \vem$.} \qed
}



\prob{
	The fact that indices on the components of tensors can be raised and lowered using the components of the metric:
	\al{
		\Fmn &= \sgma \Fsasn, &
		\psa &= \sgsab \pbet.
	}
	\vfix
}

\sol{
	We once again apply Eqs.~\refeq{24.8} and \refeq{24.9}.  For the first relation, we also use the linearity of tensors~\cite[p.~11]{MCP} and the result of \ref{3a}.  Beginning with the right-hand side,
	\eq{
		\ans{ \sgma \Fsasn =\ } \sgma \sF(\vesa, \ven)
		= \sF(\sgma \vesa, \ven)
		= \sF(\vem, \ven)
		\ans{\ = \Fmn }
	}
	as we wanted to show. \qed
	
	For the second relation, the proof is similar:
	\eq{
		\ans{ \sgsab \pbet =\ } \sgsab p(\veb)
		= p(\sgsab \veb)
		= p(\vesa)
		\ans{\ = \psa, }
	}
	as we wanted to show. \qed
}



\prob{
	The fact that a tensor can be reconstructed from its components in the manner of Eq.~(24.11).
}

\sol{
	MCP~(24.11) is
	\eq{
		\sF = \Fmn \vesm \otimes \vesn
		= \Fsab \vea \otimes \veb
		= \Fmsb \vesm \otimes \veb.
	}
	In addition to Eqs.~\refeq{24.8} and \refeq{24.9}, we invoke the definition of the tensor product given by MCP~(1.5a):
	\eq{
		\bA \otimes \bB \otimes \bC(\bE, \bF, \bG) = \bA(\bE) \bB(\bF) \bC(\bG)
		= (\bA \cdot \bE) (\bB \cdot \bF) (\bC \cdot \bG).
	}
	Then~\cite[p.~55]{MCP}
	\eq{
		\sF(\vem, \ven) = \Fmn
		= \sgma \sgnb \Fsab
		= \Fsab (\vea \cdot \vem) (\veb \cdot \ven)
		= \Fsab \vea \otimes \veb(\vem, \ven).
	}
	Comparing the first and last expressions, we have shown that \ans{ $\sF = \Fsab \vea \otimes \veb$ } as desired.  The proofs for the other two expressions are equivalent. \qed
}