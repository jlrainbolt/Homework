\state{Transformation matrices for circular polar bases (MCP 24.5)}{
	Consider the circular polar coordinate system $\{ \vpi, \phi \}$ and its coordinate bases and orthonormal bases as shown in Fig.~24.3 and discussed in the associated text.  These coordinates are related to Cartesian coordinates $\{ x, y \}$ by the usual relations: $x = \vpi \cos\phi$, $y = \vpi \sin\phi$.
}

\prob{
	Evaluate the components ($\Lxsvpi$, etc.) of the transformation matrix that links the two coordinate bases $\{ \vesx, \vesy \}$ and $\{ \vesvpi, \vesphi \}$.  Also evaluate the components ($\Lvpisx$, etc.) of the inverse transformation matrix. 
}

\sol{
	MCP~(24.17) states that
	\aln{ \label{24.17}
		\vesa &= \vesmb \Lmbsa, &
		\vesmb &= \vesa \Lasmb,
	}
	and	MCP~(24.20) states that
	\aln{ \label{24.20}
		\Lmbsa &= \pdv{\xmb}{\xa}, &
		\Lasmb &= \pdv{\xa}{\xmb}.
	}
	Then
	\ans{\al{
		\Lxsvpi &= \pdv{x}{\vpi}
		= \cos\phi, &
		\Lysvpi &= \pdv{y}{\vpi}
		= \sin\phi, &
		\Lxsphi &= \pdv{x}{\phi}
		= -\vpi \sin\phi, &
		\Lysphi &= \pdv{y}{\phi}
		= \vpi \cos\phi.
	}}%
	We know the inverse transformation matrix must be the inverse of the matrix we just found~\cite[p.~1164]{MCP}.  We can write
	\eqn{mateq}{
		\mqty[
			\vesvpi \\ \vesphi
		] = \mqty[
			\cos\phi & -\vpi \sin\phi \\
			\sin\phi & \vpi \cos\phi
		] \mqty[
			\vesx \\ \vesy
		] = \mqty[
			\Lxsvpi & \Lxsphi \\
			\Lysvpi & \Lysphi
		] \mqty[
			\vesx \\ \vesy
		] \equiv \sL \mqty[
			\vesx \\ \vesy
		],
	}
	where we have defined the matrix $\sL$.  The inverse of $2 \times 2$ matrix can be found using the general expression~\cite{Inverse}
	\eqn{inv}{
		\sA^{-1} = \frac{1}{\abs{\sA}} \mqty[
			A_{22} & -A_{12} \\
			-A_{21} & A_{11}
		].
	}
	Our determinant is
	\eq{
		\abs{\sL} = \Lxsvpi \Lysphi - \Lxsphi \Lysvpi
		= \vpi \cos^2\phi + \vpi \sin^2 \psi
		= \vpi,
	}
	so
	\eq{
		\sL^{-1} = \frac{1}{\vpi} \mqty[
			\Lysphi & -\Lxsphi \\
			-\Lysvpi & \Lxsvpi
		] = \frac{1}{\vpi} \mqty[
			\vpi \cos\phi & \vpi \sin\phi \\
			-\sin\phi & \cos\phi
		] = \mqty[
			\cos\phi & \sin\phi \\
			-\sin(\phi) / \vpi & \cos(\phi) / \vpi
		].
	}
	In other words,
	\ans{\aln{ \label{inva}
		\Lvpisx &= \cos\phi, &
		\Lvpisy &= \sin\phi, &
		\Lphisx &= -\frac{\sin\phi}{\vpi}, &
		\Lphisy &= \frac{\cos\phi}{\vpi}.
	}}%
	\vfix
}



\prob{
	Similarly, evaluate the components of the transformation matrix and its inverse linking the bases $\{ \vesx, \vesy \}$ and $\{ \vesvpih, \vesphih \}$.
}

\sol{
	We know that $\vesphih = (1 / \vpi) \vesphi$ and $\vesvpih = \vesvpih$~\cite[p.~1163]{MCP}.  Applying Eq.~\refeq{24.20} and the chain rule, we find
	\al{
		\ans{ \Lxsvpih }&\ans{\ =\ } \pdv{x}{\vpih}
		= \pdv{\vpi}{\vpih} \pdv{x}{\vpi}
		\ans{\ = \cos\phi, } &
		%
		\ans{ \Lysvpih }&\ans{\ =\ } \pdv{y}{\vpih}
		= \pdv{\vpi}{\vpih} \pdv{y}{\vpi}
		\ans{\ = \sin\phi, } \\
		%
		\ans{ \Lxsphih }&\ans{\ =\ } \pdv{x}{\phih}
		= \pdv{\phi}{\phih} \pdv{x}{\vpi}
		\ans{\ = -\sin\phi, } &
		%
		\ans{ \Lysphih }&\ans{\ =\ } \pdv{y}{\phih}
		= \pdv{\phi}{\phih} \pdv{y}{\vpi}
		\ans{\ = \cos\phi. }
	}
	For the inverse, we may apply Eq.~\refeq{inv} once more.  Our determinant is
	\eq{
		\abs*{\sLh} = \Lxsvpih \Lysphih - \Lxsphih \Lysvpih
		= \cos^2\phi + \sin^2\phi
		= 1,
	}
	so
	\eq{
		\sLh^{-1} = \mqty[
			\Lysphih & -\Lxsphih \\
			-\Lysvpih & \Lxsvpih
		] = \mqty[
			\cos\phi & \sin\phi \\
			-\sin\phi & \cos(\phi) / \vpi
		].
	}
	In other words,
	\ans{\aln{ \label{invb}
		\Lvpihsx &= \cos\phi, &
		\Lvpihsy &= \sin\phi, &
		\Lphihsx &= -\sin\phi, &
		\Lphihsy &= \cos\phi.
	}}%
	\vfix
}



\prob{
	Consider the vector $\vA = \vesx + 2 \vesy$.  What are its components in the other two bases?
}

\sol{
	Applying Eqs.~\refeq{24.17} and \refeq{inva},
	\al{
		\vesx &= \vesvpi \Lvpisx + \vesphi \Lphisx
		= \cos\phi \vesvpi - \frac{\sin\phi}{\vpi} \vesphi, &
		\vesy &= \vesvpi \Lvpisy + \vesphi \Lphisy
		= \sin\phi \vesvpi + \frac{\cos\phi}{\vpi} \vesphi,
	}
	so
	\eq{
		\vA = (\cos\phi \vesvpi - \frac{\sin\phi}{\vpi} \vesphi) + 2 (\sin\phi \vesvpi + \frac{\cos\phi}{\vpi} \vesphi)
		= \ans{ (\cos\phi + 2 \sin \phi) \vesvpi + \frac{1}{\vpi} (2 \cos\phi - \sin\phi) \vesphi. }
	}
	Now applying Eqs.~\refeq{24.17} and \refeq{invb},
		\al{
		\vesx &= \vesvpih \Lvpihsx + \vesphih \Lphihsx
		= \cos\phi \vesvpih - \sin\phi \vesphih, &
		\vesy &= \vesvpih \Lvpihsy + \vesphih \Lphihsy
		= \sin\phi \vesvpih + \cos\phi \vesphih,
	}
	so
	\eq{
		\vA = (\cos\phi \vesvpih - \sin\phi \vesphih) + 2 (\sin\phi \vesvpih + \cos\phi \vesphih)
		= \ans{ (\cos\phi + 2 \sin \phi) \vesvpih + (2 \cos\phi - \sin\phi) \vesphih. }
	}
}