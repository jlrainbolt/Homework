\documentclass[11pt]{article}
\usepackage{homework}

\classname{364}
\homeworknum{2}


\DeclareMathAlphabet{\mathsfit}{T1}{\sfdefault}{\mddefault}{\sldefault}



\begin{document}

% Environments

\newcommand{\state}[2]{\begin{statement}{#1} #2 \end{statement}}
\newcommand{\prob}[2]{\begin{problem}{#1} #2 \end{problem}}
\newcommand{\subprob}[1]{\begin{subproblem} #1 \end{subproblem}}
\newcommand{\sol}[1]{\begin{solution} #1 \end{solution}}
\newcommand{\fig}[2]{\begin{figure} \centering #2  \label{#1} \end{figure}}

\newcommand{\makebib}{
	\vfill
	\color{black}
	\bibliography{references}{}
	\bibliographystyle{lucas_unsrt}
}
	

% Implication

\newcommand{\qwhere}{\quad \text{where} \quad}
\newcommand{\qimplies}{\quad \implies \quad}
\newcommand{\impliesq}{\implies \quad}



% Brackets

\newcommand{\paren}[1]{\left( #1 \right)}
\newcommand{\brac}[1]{\left[ #1 \right]}


% Greek

\newcommand{\alp}{\alpha}
\newcommand{\bet}{\beta}
\newcommand{\gam}{\gamma}
\newcommand{\del}{\delta}
\newcommand{\eps}{\epsilon}
\newcommand{\zet}{\zeta}
\newcommand{\tht}{\theta}
\newcommand{\kap}{\kappa}
\newcommand{\lam}{\lambda}
\newcommand{\sig}{\sigma}
\newcommand{\ups}{\upsilon}
\newcommand{\omg}{\omega}

\newcommand{\Gam}{\Gamma}
\newcommand{\Del}{\Delta}
\newcommand{\Tht}{\Theta}
\newcommand{\Lam}{\Lambda}
\newcommand{\Sig}{\Sigma}
\newcommand{\Omg}{\Omega}
% Problem 1

\newcommand{\Psii}{\Psi^i}
\newcommand{\Psiix}{\Psii(x)}

\newcommand{\Pii}{\Pi^i}

\newcommand{\Phii}{\Phi^i}
\newcommand{\Phiix}{\Phii(x)}
\newcommand{\PhiN}{\Phi^N}
\newcommand{\PhiNx}{\PhiN(x)}
\newcommand{\Phiq}{\Phi^1}
\newcommand{\Phiw}{\Phi^2}

\newcommand{\ddcx}{\dd[3]{x}}

\newcommand{\delij}{\del^{i j}}
\newcommand{\delkl}{\del^{k l}}
\newcommand{\delil}{\del^{i l}}
\newcommand{\deljk}{\del^{j k}}
\newcommand{\delik}{\del^{i k}}
\newcommand{\deljl}{\del^{j l}}

\newcommand{\DF}{D_F}

\newcommand{\sigx}{\sig(x)}

\newcommand{\pii}{\pi^i}
\newcommand{\pij}{\pi^j}
\newcommand{\pik}{\pi^k}
\newcommand{\pil}{\pi^l}
\newcommand{\piix}{\pi(x)}

\newcommand{\pq}{p_1}
\newcommand{\pw}{p_2}
\newcommand{\pe}{p_3}
\newcommand{\pr}{p_4}

\newcommand{\vp}{\vb{p}}
\newcommand{\vpsi}{\vp_i}

\newcommand{\mpi}{m_\pi}


\state{Stress-energy tensor and energy-momentum conservation for a perfect fluid (MCP 2.26)}{\hfix}

\prob{
	Derive the frame-independent expression~(2.74b) for the perfect fluid stress-energy tensor from its rest-frame components~(2.74a).
}

\sol{
	From MCP~(2.74a), the nonzero rest-frame components of the tensor are
	\al{
		\Too &= \rho, &
		\Tjk &= P \deljk.
	}
	We know from MCP~(2.23c) that $\gab = \etaab$, and from (2.22) that $\eta_{00} = -1$, $\eta_{11} = \eta_{22} = \eta_{33} = 1$.  Then, using the form $\sT \equiv \Tab \vesa \otimes \vesb$ of (2.23a), we can write
	\al{
		\sT = (\rho + P) \veo \otimes \veo + P \sg.
	}
	Note that in the local rest frame, the fluid is stationary so its 4-velocity is $(1, 0, 0, 0)$.  That is, $\uv \otimes \uv$ simplifies to $\veo \otimes \veo$ in the local rest frame.  So the frame-independent expression is
	\eq{
		\ans{ \sT = (\rho + P) \uv \otimes \uv + P \sg, }
	}
	which is identical to (2.74b). \qed
}



\prob{
	Explain why the projection of $\vgrad \cdot \sT = 0$ along the fluid 4-velocity, $\uv \cdot (\vgrad \cdot \sT) = 0$, should represent energy conservation as viewed by the fluid itself.  Show that this equation reduces to
	\eq{
		\dv{\rho}{\tau} = -(\rho + P) \vgrad \cdot \uv.
	}
	With the aid of Eq.~(2.65), bring this into the form
	\eq{
		\dv{(\rho V)}{\tau} = -P \dv{V}{\tau},
	}
	where $V$ is the 3-volume of some small fluid element as measured in the fluid's local rest frame.  What are the physical interpretations of the left- and right-hand sides of this equation, and how is it related to the first law of thermodynamics?
}

\sol{
	We know that $\vgrad \cdot \sT$ represents the energy-momentum flow of the system, and that $\vgrad \cdot \sT = 0$ rells us that 4-momentum is conserved~\cite[pp.~83--85]{MCP}.  So $\uv \cdot (\vgrad \cdot \sT)$ tells us how the energy-momentum flow looks in the local rest frame of the fluid, and $\uv \cdot (\vgrad \cdot \sT) = 0$ thus indicates that 4-momentum is conserved in that frame.
	
	Applying (2.74b) and the product rule, note that
	\aln{
		\vgrad \cdot \sT &= \ptsa \Tab \notag \\
		&= \ptsa [ (\rho + P) \ua \ub + P \gab ] \notag \\
		&= \ua \ub \ptsa(\rho + P) + (\rho + P) (\ub \ptsa \ua + \ua \ptsa \ub) + \gab \ptsa P \notag \\
		&= \ua \ub \ptsa(\rho + P) + (\rho + P) (\ub \ptsa \ua + \ua \ptsa \ub) + \ptb P. \label{tabthing}
	}
	Then
	\aln{
		\uv \cdot (\vgrad \cdot \sT) &= \usb \ptsa \Tab \notag \\
		&= \usb \ua \ub \ptsa(\rho + P) + \usb (\rho + P) (\ub \ptsa \ua + \ua \ptsa \ub) + \usb \ptb P \notag \\
		&= -\ua \ptsa(\rho + P) + (\rho + P) (-\ptsa \ua + \usb \ua \ptsa \ub) + \usb \ptb P \notag \\
		&= -\ua \ptsa(\rho + P) - (\rho + P) \ptsa \ua + \usb \ptb P, \label{thing1b}
	}
	where we have used MCP~(2.9), $\uv^2 = -1$, and that as a consequence~\cite[p.~36]{Carroll},
	\eqn{vthing}{
		0 = \ptsa(\usb \ub) = \usb \ptsa \ub + \ub \ptsa \usb = 2 \usb \ptsa \ub.
	}
	Picking back up at Eq.~\refeq{thing1b},
	\eq{
		\uv \cdot (\vgrad \cdot \sT) = -\ua \ptsa \rho - \ua \ptsa P - (\rho + P) \ptsa \ua + \usb \ptb P
		= -\ua \ptsa \rho - (\rho + P) \ptsa \ua.
	}
	Applying $\uv \cdot (\vgrad \cdot \sT) = 0$ and \hl{ specifying the fluid's rest frame in which } $\uv = (1, 0, 0, 0)$,
	\eqn{ans1b}{
		\ua \ptsa \rho = -(\rho + P) \ptsa \ua
		\qimplies
		\dv{\rho}{\tau} + \ui \dv{\rho}{\xii} = -(\rho + P) \vgrad \cdot \uv
		\qimplies
		\ans{ \dv{\rho}{\tau} = -(\rho + P) \vgrad \cdot \uv, }
	}
	as we wanted to show. \qed
	
	MCP~(2.65) states
	\eq{
		\vgrad \cdot \uv = \frac{1}{V} \dv{V}{\tau}.
	}
	Feeding this into Eq.~\refeq{ans1b} and applying the product rule yields
	\eqn{ans1b2}{
		\dv{\rho}{\tau} = -(\rho + P) \frac{1}{V} \dv{V}{\tau}
		\qimplies
		V \dv{\rho}{\tau} = -(\rho + P) \dv{V}{\tau}
		\qimplies
		V \dv{\rho}{\tau} + \rho \dv{V}{\tau} = \ans{ \dv{(\rho V)}{\tau} = -P \dv{V}{\tau}, }
	}
	as we wanted to show. \qed
	
	The left-hand side of Eq.~\refeq{ans1b2} represents the rate of change of energy ($ =\rho V$).  The right-hand side represents the product of pressure and the rate of change of volume.  Essentially, the equation is relating the change in energy of the fluid to the change in its volume under constant pressure.  The first law of thermodynamics is given by MCP~(5.7):
	\eq{
		\dd{\cE} = T \dd{S} + \tmu \dd{N} - P \dd{V},
	}
	where $\cE$ is energy, $T$ is temperature, $S$ is entropy, $\tmu$ is chemical potential, and $N$ is number of particles.  Eq.~\refeq{ans1b2} is the first law of thermodynamics for a perfect fluid in the case of constant entropy and constant number of particles.
}



\prob{
	Read the discussion in Ex.~2.10 about the tensor $\sP = \sg + \uv \otimes \uv$ that projects into the 3-space of the fluid's rest frame.  Explain why $\Psab \Tabsb = 0$ should represent the law of force balance (momentum conservation) as seen by the fluid.  Show that this equation reduces to
	\eq{
		(\rho + P) \av = -\sP \cdot \grad P,
	}
	where $\av = \dv*{\uv}{\tau}$ is the fluid's 4-acceleration.  This equation is a relativistic version of Newton's $\Fb = m \ab$.    Explain the physical meanings of the left- and right-hand sides.  Infer that $\rho + P$ must be the fluid's inertial mass per unit volume. % It is also the enthalpy per unit volume, including the contribution of rest mass; see Ex.~5.5 and Box~13.2.
}

\sol{
	\hl{explain why}
	
	We write
	\eq{
		\Psab \Tabsb = \Psab \ptsg \Tag
	}
	and apply Eq.~\refeq{tabthing} to $\ptsg \Tag$.  Now we invoke MCP~(2.31a), which we can write as $\Psab = \gsab + \usa \usb$:
	\al{
		\Psab \Tabsb &= (\gsab + \usa \usb) [ \ugam \ub \ptsg(\rho + P) + (\rho + P) (\ua \ptsg \ugam + \ugam \ptsg \ua) + \pta P ] \\
		&= \ugam \usa \ptsg(\rho + P) - \usa \ugam \ptsg(\rho + P) + (\gsab + \usa \usb) [ (\rho + P) (\ua \ptsg \ugam + \ugam \ptsg \ua) + \pta P ] \\
		&= (\rho + P) (\usb \ptsg \ugam + \ugam \ptsg \usb - \usb \ptsg \ugam + \usb \ugam \usa \ptsg \ua) + (\gsab + \usa \usb) \pta P \\
		&= (\rho + P) \ugam \ptsg \usb + \Psab \pta P,
	}
	where we have again used $\uv^2 = -1$ and Eq.~\refeq{vthing}.  Applying $\Psab \Tabsb = 0$ and \hl{ specifying the fluid's rest frame in which} $\uv = (1, 0, 0, 0)$, we have
	\eqn{ans1c}{
		-\Psab \pta P = (\rho + P) \ptst \usb = (\rho + P) \asb
		\qimplies
		\ans{ (\rho + P) \av = -\sP \cdot \grad P, }
	}
	as we wanted to show. \qed
	
	The left-hand side of Eq.~\refeq{ans1c} represents the fluid's inertial force per unit volume, since $\rho + P$ is its inertial mass per unit volume.  The right-hand side is \hl{the projection of the divergence of the pressure in the fluid's rest frame?}
	
	\hl{infer?}
}





\clearpage
\state{Inertial mass per unit volume (MCP 2.27)}{
	Suppose that some medium has a rest frame (unprimed frame) in which its energy flux and momentum density vanish, $\Toj = \Tjo = 0$.  Suppose that the medium moves in the $x$ direction with speed very small compared to light, $v \ll 1$, as seen in a (primed) laboratory frame, and ignore factors of order $v^2$.  The ratio of the medium's momentum density $\Gjp = \Tjpop$ (as measured in the laboratory frame) to its velocity $\vsi = \delsix$ is called its total \emph{inertial mass per unit volume} and is denoted $\rhosjiinert$:
	\eq{
		\Tjpop = \rhosjiinert \vsi.
	}
	In other words, $\rhosjiinert$ is the 3-dimensional tensor that gives the momentum density $\Gjp$ when the medium's small velocity is put into its second slot.
}

\prob{
	Using a Lorentz transformation from the medium's (unprimed) rest frame to the (primed) laboratory frame, show that
	\eq{
		\rhosjiinert = \Too \delsji + \Tsji.
	}
}



\prob{
	Give a physical explanation of the contribution $\Tsji$ to the momentum density.
}



\prob{
	Show that for a perfect fluid [Eq.~(2.74b)] the inertial mass per unit volume is isotropic and has magnitude $\rho + P$, where $\rho$ is the mass-energy density, and $P$ is the pressure measured in the fluid's rest frame:
	\eq{
		\rhosjiinert = (\rho + P) \delsji.
	}
%	See Ex.~2.26 for this inertial-mass role of $\rho + P$ in the law of force balance.
}





\clearpage
\newcommand{\kq}{\ket{1}}
\newcommand{\kw}{\ket{2}}
\newcommand{\ke}{\ket{3}}

\newcommand{\vq}{v_1}
\newcommand{\vw}{v_2}
\newcommand{\ve}{v_3}

\newcommand{\vqs}{\vq^*}
\newcommand{\vws}{\vw^*}

\newcommand{\Heff}{H_\text{eff}}
\newcommand{\Eo}{E\suo}
\newcommand{\Eod}{\Eo_D}

\newcommand{\Pq}{P_1}

%\clearpage
\begin{statement}{}
	Consider the Hamiltonian $\Ho$ acting on a three-dimensional Hilbert space spanned by the orthonormal basis $\{\kq, \kw, \ke\}$.  $\Ho = \sum_{i = 3}^3 E_i \ketbra{i}$, with energy eigenvalues $\Eoq, \Eow, \Eoe$.  Assume $\Eoq = \Eow = \Eod$.  To $\Ho$, we add a perturbation
	\beq
		V = \vq \ketbra{1}{3} + \vqs \ketbra{3}{1} + \vw \ketbra{2}{3} + \vws \ketbra{3}{2}.
	\eeq
	Here, $\vq$ and $\vw$ are complex constants and small compared to $\Ee$.
\end{statement}

\begin{problem}
	To second order in $V$, write down the explicit form of the effective Hamiltonian acting on the subspace spanned by $\{\kq, \kw\}$.
\end{problem}

\begin{solution}
	We have
	\begin{align*}
		\Ho &= \mqty[ \Eod & 0 & 0 \\ 0 & \Eod & 0 \\ 0 & 0 & \Eoe ], &
		V &= \mqty[ 0 & 0 & \vq \\ 0 & 0 & \vw \\ \vqs & \vws & 0 ], &
		H &= \Ho + \lam V = \mqty[ \Eod & 0 & \lam \vq \\ 0 & \Eod & \lam \vw \\ \vqs & \vws & \Eoe].
	\end{align*}
	From the lecture notes and (5.2.12) in Sakurai, the effective Hamiltonian is given to second order in $\lam$ by
	\beq
		\Heff = \Eod + \lam \Po V \Po + \lam^2 \Po V \Pq (\Eod - \Ho)^{-1} \Pq V \Po,
	\eeq
	where $\Po$ is the projection onto the degenerate subspace, $\Pq$ is the projection onto the nondegenerate subspace, and $\Eod$ is the degenerate energy.  Here, $\Po$ projects onto the subspace spanned by $\{ \kq, \kw \}$ and $\Pq$ onto that spanned by $\{ \ke \}$.
	
	Note that
	\beq
		Po V \Po = \mqty[ 1 & 0 & 0 \\ 0 & 1 & 0 \\ 0 & 0 & 0 ] \mqty[ 0 & 0 & \vq \\ 0 & 0 & \vw \\ \vqs & \vws & 0 ] \mqty[ 1 & 0 & 0 \\ 0 & 1 & 0 \\ 0 & 0 & 0 ]
		= \mqty[ 0 & 0 & 0 \\ 0 & 0 & 0 \\ 0 & 0 & 0 ],
	\eeq
	and
	\begin{align*}
		\Po V \Pq (\Eod - \Ho)^{-1} &\Pq V \Po = \frac{1}{\Eod - \Eoe} \Po \mqty[ 0 & 0 & \vq \\ 0 & 0 & \vw \\ \vqs & \vws & 0 ] \mqty[ 0 & 0 & 0 \\ 0 & 0 & 0 \\ 0 & 0 & 1] \mqty[ 0 & 0 & \vq \\ 0 & 0 & \vw \\ \vqs & \vws & 0 ] \Po \\
		&= \frac{1}{\Eod - \Eoe} \mqty[ 1 & 0 & 0 \\ 0 & 1 & 0 \\ 0 & 0 & 0 ] \mqty[|\vq|^2 & \vq \vws & 0 \\ \vqs \vw & |\vq|^2 & 0 \\ 0 & 0 & 0] \mqty[ 1 & 0 & 0 \\ 0 & 1 & 0 \\ 0 & 0 & 0 ]
		= \frac{1}{\Eod - \Eoe} \mqty[|\vq|^2 & \vq \vws & 0 \\ \vqs \vw & |\vq|^2 & 0 \\ 0 & 0 & 0].
	\end{align*}
	
	In the degenerate subspace, we have
	\begin{align*}
		\Heff = \mqty[ \Eod + |\vq|^2/(\Eod - \Eoe) & \vq \vws / (\Eod - \Eoe) \\ \vqs \vw / (\Eod - \Eoe) & \Eod + |\vw|^2/(\Eod - \Eoe)].
	\end{align*}
\end{solution}


%\clearpage
\newcommand{\uq}{u_1}
\newcommand{\uw}{u_2}
\newcommand{\wq}{w_1}
\newcommand{\ww}{w_2}

\begin{problem}
	By solving the effective Hamiltonian, construct the approximate solution for the eigenvalues and eigenfunctions of $\Ho + V$.  (The eigenkets only need to be constructed within the degenerate subspace.)
\end{problem}

\begin{solution}
	Let $E$ be the eigenvalues of $\Heff$.  We need to solve the characteristic equation
	\begin{align*}
		0 &= \det(\Heff - E I)
		= \vmqty{ \Eod + |\vq|^2/(\Eod - \Eoe) - E & \vq \vws / (\Eod - \Eoe) \\ \vqs \vw / (\Eod - \Eoe) & \Eod + |\vw|^2/(\Eod - \Eoe) - E } \\
		&= \left( \Eod + \frac{|\vq|^2}{\Eod - \Eoe} - E \right) \left( \Eod + \frac{|\vw|^2}{\Eod - \Eoe} - E \right) - \frac{|\vq|^2 |\vw|^2}{(\Eod - \Eoe)^2} \\
		&= {\Eod}^2 + \Eod \frac{|\vw|^2}{\Eod - \Eoe} - \Eod E + \Eod \frac{|\vq|^2}{\Eod - \Eoe} - E \frac{|\vq|^2}{\Eod - \Eoe} - \Eod E - E \frac{|\vw|^2}{\Eod - \Eoe} + E^2 \\
		&= E^2 - \Eod E - E \frac{|\vq|^2 + |\vw|^2}{\Eod - \Eoe} - \Eod E + {\Eod}^2 + \Eod \frac{|\vq|^2 + |\vw|^2}{\Eod - \Eoe} \\
		&= (E - \Eod) \left( E - \Eod - \frac{|\vq|^2 + |\vw|^2}{(\Eod - \Eoe)^2} \right),
	\end{align*}
	so the eigenvalues are
	\begin{align*}
		\Eq &= \Eod, &
		\Ew &= \Eod + \frac{|\vq|^2 + |\vw|^2}{(\Eod - \Eoe)^2}.
	\end{align*}
	
	The eigenvector corresponding to $\Eq$ can be found by
	\beq
		\mqty[ \Eod + |\vq|^2 / (\Eod - \Eoe) & \vq \vws / (\Eod - \Eoe) \\ \vqs \vw / (\Eod - \Eoe) & \Eod + |\vw|^2 / (\Eod - \Eoe) ] \mqty[ \uq \\ \uw ] = \Eod \mqty[ \uq \\ \uw ]
	\eeq
	which is equivalent to the system of equations
	\begin{align*}
		\left( \Eod + \frac{|\vq|^2}{\Eod - \Eoe} \right) \uq + \frac{\vq \vws}{\Eod - \Eoe} \uw &= \Eod \uq, &
		\frac{\vqs \vw}{\Eod - \Eoe} \uq + \left( \Eod + \frac{|\vw|^2}{\Eod - \Eoe} \right) \uw &= \Eod \uw.
	\end{align*}
	By inspection, these are satisfied when $\uq = -\vws$ and $\uw = \vqs$.
	
	For the eigenvector corresponding to $\Ew$, we have
	\beq
		\mqty[ \Eod + |\vq|^2 / (\Eod - \Eoe) & \vq \vws / (\Eod - \Eoe) \\ \vqs \vw / (\Eod - \Eoe) & \Eod + |\vw|^2 / (\Eod - \Eoe) ] \mqty[ \wq \\ \ww ] = \left( \Eod + \frac{|\vq|^2 + |\vw|^2}{\Eod - \Eoe} \right)\mqty[ \wq \\ \ww ]
	\eeq
	which is equivalent to the system of equations
	\begin{align*}
		\left( \Eod + \frac{|\vq|^2}{\Eod - \Eoe} \right) \wq + \frac{\vq \vws}{\Eod - \Eoe} \ww &= \left( \Eod + \frac{|\vq|^2 + |\vw|^2}{\Eod - \Eoe} \right) \wq, \\
		\frac{\vqs \vw}{\Eod - \Eoe} \wq + \left( \Eod + \frac{|\vw|^2}{\Eod - \Eoe} \right) \ww &= \left( \Eod + \frac{|\vq|^2 + |\vw|^2}{\Eod - \Eoe} \right) \ww.
	\end{align*}
	By inspection, these are satisfied when $\wq = \vq$ and $\ww = \ww$.  So we have the eigenvectors
	\begin{align*}
		\ket*{\Eq} &= \mqty[ -\vws \\ \vqs ], &
		\ket*{\Ew} &= \mqty[ \vq \\ \vw ].
	\end{align*}
\newcommand{\lap}{\nabla^2}
\newcommand{\vF}{\vec{F}}
\newcommand{\nabx}{\nabla_{\!x}}
\newcommand{\absxp}{\abs{\vx'}}
\newcommand{\nh}{\vec{\hat{n}}}
\newcommand{\rh}{\vec{\hat{r}}}
\newcommand{\Gd}{G_D}
\newcommand{\Gdxxp}{\Gd(\vx,\vx')}

\begin{statement}{}
	A point charge of charge $q$ is placed at point $\vx'$ inside a conducting spherical shell of radius $R$.  There is no net charge on the conductor.  The potential inside the sphere is thus given by $q \, \Gdxxp$, where the explicit formula for $\Gdxxp$ for a spherical cavity is given in the lecture notes.
\end{statement}

\begin{problem}
	Find the surface charge density $\sigtv$ on the conducting shell.
\end{problem}

\begin{solution}
	The Green's function for a spherical cavity is given by Eq.~(2.91),
	\beq
		\Gdxxp = \frac{1}{\abs{\vx - \vx'}} + \frac{\alp}{\abs{\vx - \vx''}} \qq{where} \vx'' = \vx' \frac{R^2}{\absxp^2} \qand \alp = - \frac{R}{\absxp}.
	\eeq
	The surface charge density can be found from Eq.~(2.86),
	\beqn \label{scdeq}
		\vE \cdot \nh = 4\pi \sig,
	\eeqn
	where $\vE = -\nabla \phi$ in electrostatics.
	
	We will begin by finding $\vE$.  We will orient our coordinate system such that $\vx'$ (and consequently $\vx''$) points along the $z$ axis.  Note that
	\beq
		\Gdxxp = \frac{1}{\abs{\vx - \vx'}} - \frac{R}{\absxp \abs{\vx - \dfrac{R^2}{\absxp^2} \vx'}}
		= \frac{1}{\sqrt{\vx^2 - 2 \vx \cdot \vx' + {\vx'}^2}} - \frac{R}{\absxp \sqrt{\vx^2 - 2 \dfrac{R^2}{{\vx'}^2} \vx \cdot \vx' + \dfrac{R^4}{{\vx'}^4} {\vx'}^2}}.
	\eeq
	In spherical coordinates, we have
	\beq
		\Gdxxp = \frac{1}{\sqrt{r^2 - 2 r r' \cost + {r'}^2}} - \frac{R}{r'} \frac{1}{\sqrt{r^2 - 2 R^2 r \cost / r' + R^4 / {r'}^2}},
	\eeq
	where we note that $\tht$ is the angle between $\vx$ and the $z$ axis.  The gradient in spherical coordinates is given by
	\beq
		\nabla = \pdv{}{r} \,\rh + \frac{1}{r} \pdv{}{\tht} \,\thh + \frac{1}{r \sint} \pdv{}{\vph} \, \phh.
	\eeq
	The $r$ component of the electric field inside the conductor is then
	\beq
		\Er(\vx) = -q \pdv{\Gdxxp}{r}
		= q \left( \frac{r - r' \cost}{(r^2 - 2 r r' \cost + {r'}^2)^{3/2}} - \frac{R}{r'} \frac{r - R^2 \cost / r'}{(r^2 - 2 R^2 r \cost / r' + R^4 / {r'}^2)^{3/2}} \right).
	\eeq
	Since $\nh = -\rh$ for the inner surface of a sphere, we are interested in only the $r$ component of the field.  On the surface of the sphere, the field is $\Er(r=R) \,\rh$.  So we have
	\begin{align*}
		\Er(r=R) &= q \left( \frac{R - r' \cost}{(R^2 - 2 R r' \cost + {r'}^2)^{3/2}} - \frac{R}{r'} \frac{R - R^2 \cost / r'}{(R^2 - 2 R^3 \cost / r' + R^4 / {r'}^2)^{3/2}} \right) \\
		&= q \left( \frac{R - r' \cost}{{r'}^3 (R^2 / {r'}^2 - 2 R \cost / r' + 1)^{3/2}} - \frac{R}{r'} \frac{R - R^2 \cost / r'}{R^3 (1 - 2 R \cost / r' + R^2 / {r'}^2)^{3/2}} \right) \\
		&= \frac{q}{r'} \frac{R^3 - R^2 r' \cost - R {r'}^2 + R^2 r' \cost}{R^2 {r'}^2 (R^2 / {r'}^2 - 2 R \cost / r' + 1)^{3/2}}
		= \frac{q}{R {r'}^3} \frac{R^2 - {r'}^2}{(R^2 / {r'}^2 - 2 R \cost / r' + 1)^{3/2}}.
	\end{align*}
	Finally, feeding this into \refeq{scdeq},
	\beq
		\sig = -\frac{\vE \cdot \rh}{4\pi}
		= \frac{q}{4\pi R {r'}^3} \frac{{r'}^2 - R^2}{(R^2 / {r'}^2 - 2 R \cost / r' + 1)^{3/2}}
		= \frac{q}{4\pi R \absxp^3} \frac{\absxp^2 - R^2}{(R^2 / \absxp^2 - 2 R \cost / \absxp + 1)^{3/2}}.
	\eeq
\end{solution}
\vfix


\newcommand{\vEo}{\vE_0}
\newcommand{\del}{\delta}
\newcommand{\Etht}{E_\tht}
\newcommand{\Fr}{F_r}

\begin{problem}
	Find the force $\vF$ that must be exerted on the point charge in order to hold it in place.
\end{problem}

\begin{solution}
	The total force on a charge distribution arises only from the external electric field $\vEo$, and is given by Eq.~(2.42) in the lecture notes:
	\beq
		\vF = \int \rhox \, \vEo(\vx) \dcx.
	\eeq
	The force required to keep the point charge in place is equal and opposite to this force, so we need to insert a minus sign.  We also need the $\tht$ component of the field inside the conductor, which is
	\beq
		\Etht(\vx) = -\frac{q}{r} \pdv{\Gdxxp}{\tht}
		= -q \left( \frac{r' \sint}{(r^2 - 2 r r' \cost + {r'}^2)^{3/2}} - \frac{R^3 \sint}{{r'}^2 (r^2 - 2 R^2 r \cost / r' + R^4 / {r'}^2)^{3/2}} \right).
	\eeq
	The charge density for a point charge located at $\vx'$ is given by $\rhox = q \, \delta(\vx - \vx')$.  Evaluating the integral, we have
	\beq
		\vF = -\int q \, \delta(\vx - \vx') \, \vE(\vx) \dcx
		= -q \vE(\vx').
	\eeq
	Recall that we chose $\vx'$ to point along the $z$ axis, so $\tht' = 0$.  The $\tht$ component of $\vF$ is then $0$, and the $r$ component is
	\begin{align*}
		\Fr &= -q^2 \left( \frac{r' - r'}{({r'}^2 - 2 {r'}^2 + {r'}^2)^{3/2}} - \frac{R}{r'} \frac{r' - R^2 / r'}{({r'}^2 - 2 R^2 + R^4 / {r'}^2)^{3/2}} \right)
		= q^2 R {r'}^2 \frac{r' - R^2 / r'}{({r'}^4 - 2 R^2 {r'}^2 + R^4)^{3/2}} \\
		&= -q^2 R {r'}^2 \frac{({r'}^2 - R^2) / r'}{({r'}^2 - R^2)^3}
		= -q^2 \frac{R r'}{({r'}^2 - R^2)^2}.
	\end{align*}
	Since only the $r$ component of $\vF$ is nonzero, it points in the $z$ direction, which we chose to be equivalent to the unit vector $\vx' / \absxp$.  Therefore,
	\beq
		\vF = -q^2 \frac{R \absxp}{(R^2 - \absxp^2)^2} \frac{\vx'}{\absxp}
		= -q^2 \frac{R}{(R^2 - \absxp^2)^2} \vx'.
	\eeq
\end{solution}







\state{Gauss's theorem (MCP 24.11)}{
	In 3-dimensional Euclidean space Maxwell's equation $\grad \vdot \bE = \rhoe / \epso$ can be combined with Gauss's theorem to show that the electric flux through the surface $\pt\cV$ of a sphere is equal to the charge in the sphere's interior $\cV$ divided by $\epso$:
	\eq{
		\intdV \bE \vdot d\bSig = \intV \frac{\rhoe}{\epso} \ddV.
	}
	Introduce spherical polar coordinates so the sphere's surface is at some radius $r = R$.  Consider a surface element on the sphere's surface with vectorial legs $\ddphi \pdv*{\phi}$ and $\ddtht \pdv*{\tht}$.  Evaluate the components $\ddSigj$ of the surface integration element $\ddbSig = \beps(\ldots, \ddtht \pdv*{\tht}, \ddphi \pdv*{\phi})$.  (Here $\beps$ is the Levi-Civita tensor.)  Similarly, evaluate $\ddV$ in terms of vectorial legs in the sphere's interior.  Then use these results for $\ddSigj$ and $\ddV$ to convert Eq.~(24.47) into an explicit form in terms of integrals over $r$, $\tht$, and $\phi$.  The final answer should be obvious, but the above steps in deriving it are informative.
}


\makebib

\end{document}
