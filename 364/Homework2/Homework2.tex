\documentclass[11pt]{article}
\usepackage{homework}

\classname{364}
\homeworknum{2}


\DeclareMathAlphabet{\mathsfit}{T1}{\sfdefault}{\mddefault}{\sldefault}



\begin{document}

% Environments

\newcommand{\state}[2]{\begin{statement}{#1} #2 \end{statement}}
\newcommand{\prob}[2]{\begin{problem}{#1} #2 \end{problem}}
\newcommand{\subprob}[1]{\begin{subproblem} #1 \end{subproblem}}
\newcommand{\sol}[1]{\begin{solution} #1 \end{solution}}
\newcommand{\fig}[2]{\begin{figure} \centering #2  \label{#1} \end{figure}}

\newcommand{\makebib}{
	\vfill
	\color{black}
	\nocite{*}
	\bibliography{references}{}
	\bibliographystyle{lucas_unsrt}
}
	

% Implication

\newcommand{\qwhere}{\quad \text{where} \quad}
\newcommand{\qimplies}{\quad \implies \quad}
\newcommand{\impliesq}{\implies \quad}



% Brackets

\newcommand{\paren}[1]{\left( #1 \right)}
\newcommand{\brac}[1]{\left[ #1 \right]}
\newcommand{\curly}[1]{\left\{ #1 \right\}}


% Greek

\newcommand{\alp}{\alpha}
\newcommand{\bet}{\beta}
\newcommand{\gam}{\gamma}
\newcommand{\del}{\delta}
\newcommand{\eps}{\epsilon}
\newcommand{\zet}{\zeta}
\newcommand{\tht}{\theta}
\newcommand{\kap}{\kappa}
\newcommand{\lam}{\lambda}
\newcommand{\sig}{\sigma}
\newcommand{\ups}{\upsilon}
\newcommand{\omg}{\omega}

\newcommand{\Gam}{\Gamma}
\newcommand{\Del}{\Delta}
\newcommand{\Tht}{\Theta}
\newcommand{\Lam}{\Lambda}
\newcommand{\Sig}{\Sigma}
\newcommand{\Omg}{\Omega}


% Text

\newcommand{\where}{\text{where }}

% Problem 1

\newcommand{\Hint}{H_\text{int}}
\newcommand{\ddcx}{\dd[3]{x}}
\newcommand{\psib}{\bar{\psi}}

\newcommand{\mh}{m_h}
\newcommand{\mmu}{m_\mu}
\newcommand{\me}{m_e}
\newcommand{\ma}{m_a}

\newcommand{\aexpt}{a_\text{expt.}}
\newcommand{\aQED}{a_\text{QED}}
\renewcommand{\GeV}{\giga\electronvolt}

\newcommand{\gamt}{\gam^5}


\state{Stress-energy tensor and energy-momentum conservation for a perfect fluid (MCP 2.26)}{\hfix}

\prob{
	Derive the frame-independent expression~(2.74b) for the perfect fluid stress-energy tensor from its rest-frame components~(2.74a).
}

\sol{
	From MCP~(2.74a), the nonzero rest-frame components of the tensor are
	\al{
		\Too &= \rho, &
		\Tjk &= P \deljk.
	}
	We know from MCP~(2.23c) that $\gab = \etaab$, and from (2.22) that $\eta_{00} = -1$, $\eta_{11} = \eta_{22} = \eta_{33} = 1$.  Then, using the form $\sT \equiv \Tab \vesa \otimes \vesb$ of (2.23a), we can write
	\al{
		\sT = (\rho + P) \veo \otimes \veo + P \sg.
	}
	Note that in the local rest frame, the fluid is stationary so its 4-velocity is $(1, 0, 0, 0)$.  That is, $\uv \otimes \uv$ simplifies to $\veo \otimes \veo$ in the local rest frame.  So the frame-independent expression is
	\eq{
		\ans{ \sT = (\rho + P) \uv \otimes \uv + P \sg, }
	}
	which is identical to (2.74b). \qed
}



\prob{
	Explain why the projection of $\vgrad \cdot \sT = 0$ along the fluid 4-velocity, $\uv \cdot (\vgrad \cdot \sT) = 0$, should represent energy conservation as viewed by the fluid itself.  Show that this equation reduces to
	\eq{
		\dv{\rho}{\tau} = -(\rho + P) \vgrad \cdot \uv.
	}
	With the aid of Eq.~(2.65), bring this into the form
	\eq{
		\dv{(\rho V)}{\tau} = -P \dv{V}{\tau},
	}
	where $V$ is the 3-volume of some small fluid element as measured in the fluid's local rest frame.  What are the physical interpretations of the left- and right-hand sides of this equation, and how is it related to the first law of thermodynamics?
}

\sol{
	We know that $\vgrad \cdot \sT$ represents the energy-momentum flow of the system, and that $\vgrad \cdot \sT = 0$ rells us that 4-momentum is conserved~\cite[pp.~83--85]{MCP}.  So $\uv \cdot (\vgrad \cdot \sT)$ tells us how the energy-momentum flow looks in the local rest frame of the fluid, and $\uv \cdot (\vgrad \cdot \sT) = 0$ thus indicates that 4-momentum is conserved in that frame.
	
	Applying (2.74b) and the product rule, note that
	\aln{
		\vgrad \cdot \sT &= \ptsa \Tab \notag \\
		&= \ptsa [ (\rho + P) \ua \ub + P \gab ] \notag \\
		&= \ua \ub \ptsa(\rho + P) + (\rho + P) (\ub \ptsa \ua + \ua \ptsa \ub) + \gab \ptsa P \notag \\
		&= \ua \ub \ptsa(\rho + P) + (\rho + P) (\ub \ptsa \ua + \ua \ptsa \ub) + \ptb P. \label{tabthing}
	}
	Then
	\aln{
		\uv \cdot (\vgrad \cdot \sT) &= \usb \ptsa \Tab \notag \\
		&= \usb \ua \ub \ptsa(\rho + P) + \usb (\rho + P) (\ub \ptsa \ua + \ua \ptsa \ub) + \usb \ptb P \notag \\
		&= -\ua \ptsa(\rho + P) + (\rho + P) (-\ptsa \ua + \usb \ua \ptsa \ub) + \usb \ptb P \notag \\
		&= -\ua \ptsa(\rho + P) - (\rho + P) \ptsa \ua + \usb \ptb P, \label{thing1b}
	}
	where we have used MCP~(2.9), $\uv^2 = -1$, and that as a consequence~\cite[p.~36]{Carroll},
	\eqn{vthing}{
		0 = \ptsa(\usb \ub) = \usb \ptsa \ub + \ub \ptsa \usb = 2 \usb \ptsa \ub.
	}
	Picking back up at Eq.~\refeq{thing1b},
	\eq{
		\uv \cdot (\vgrad \cdot \sT) = -\ua \ptsa \rho - \ua \ptsa P - (\rho + P) \ptsa \ua + \usb \ptb P
		= -\ua \ptsa \rho - (\rho + P) \ptsa \ua.
	}
	Applying $\uv \cdot (\vgrad \cdot \sT) = 0$ and \hl{ specifying the fluid's rest frame in which } $\uv = (1, 0, 0, 0)$,
	\eqn{ans1b}{
		\ua \ptsa \rho = -(\rho + P) \ptsa \ua
		\qimplies
		\dv{\rho}{\tau} + \ui \dv{\rho}{\xii} = -(\rho + P) \vgrad \cdot \uv
		\qimplies
		\ans{ \dv{\rho}{\tau} = -(\rho + P) \vgrad \cdot \uv, }
	}
	as we wanted to show. \qed
	
	MCP~(2.65) states
	\eq{
		\vgrad \cdot \uv = \frac{1}{V} \dv{V}{\tau}.
	}
	Feeding this into Eq.~\refeq{ans1b} and applying the product rule yields
	\eqn{ans1b2}{
		\dv{\rho}{\tau} = -(\rho + P) \frac{1}{V} \dv{V}{\tau}
		\qimplies
		V \dv{\rho}{\tau} = -(\rho + P) \dv{V}{\tau}
		\qimplies
		V \dv{\rho}{\tau} + \rho \dv{V}{\tau} = \ans{ \dv{(\rho V)}{\tau} = -P \dv{V}{\tau}, }
	}
	as we wanted to show. \qed
	
	The left-hand side of Eq.~\refeq{ans1b2} represents the rate of change of energy ($ =\rho V$).  The right-hand side represents the product of pressure and the rate of change of volume.  Essentially, the equation is relating the change in energy of the fluid to the change in its volume under constant pressure.  The first law of thermodynamics is given by MCP~(5.7):
	\eq{
		\dd{\cE} = T \dd{S} + \tmu \dd{N} - P \dd{V},
	}
	where $\cE$ is energy, $T$ is temperature, $S$ is entropy, $\tmu$ is chemical potential, and $N$ is number of particles.  Eq.~\refeq{ans1b2} is the first law of thermodynamics for a perfect fluid in the case of constant entropy and constant number of particles.
}



\prob{
	Read the discussion in Ex.~2.10 about the tensor $\sP = \sg + \uv \otimes \uv$ that projects into the 3-space of the fluid's rest frame.  Explain why $\Psab \Tabsb = 0$ should represent the law of force balance (momentum conservation) as seen by the fluid.  Show that this equation reduces to
	\eq{
		(\rho + P) \av = -\sP \cdot \grad P,
	}
	where $\av = \dv*{\uv}{\tau}$ is the fluid's 4-acceleration.  This equation is a relativistic version of Newton's $\Fb = m \ab$.    Explain the physical meanings of the left- and right-hand sides.  Infer that $\rho + P$ must be the fluid's inertial mass per unit volume. % It is also the enthalpy per unit volume, including the contribution of rest mass; see Ex.~5.5 and Box~13.2.
}

\sol{
	\hl{explain why}
	
	We write
	\eq{
		\Psab \Tabsb = \Psab \ptsg \Tag
	}
	and apply Eq.~\refeq{tabthing} to $\ptsg \Tag$.  Now we invoke MCP~(2.31a), which we can write as $\Psab = \gsab + \usa \usb$:
	\al{
		\Psab \Tabsb &= (\gsab + \usa \usb) [ \ugam \ub \ptsg(\rho + P) + (\rho + P) (\ua \ptsg \ugam + \ugam \ptsg \ua) + \pta P ] \\
		&= \ugam \usa \ptsg(\rho + P) - \usa \ugam \ptsg(\rho + P) + (\gsab + \usa \usb) [ (\rho + P) (\ua \ptsg \ugam + \ugam \ptsg \ua) + \pta P ] \\
		&= (\rho + P) (\usb \ptsg \ugam + \ugam \ptsg \usb - \usb \ptsg \ugam + \usb \ugam \usa \ptsg \ua) + (\gsab + \usa \usb) \pta P \\
		&= (\rho + P) \ugam \ptsg \usb + \Psab \pta P,
	}
	where we have again used $\uv^2 = -1$ and Eq.~\refeq{vthing}.  Applying $\Psab \Tabsb = 0$ and \hl{ specifying the fluid's rest frame in which} $\uv = (1, 0, 0, 0)$, we have
	\eqn{ans1c}{
		-\Psab \pta P = (\rho + P) \ptst \usb = (\rho + P) \asb
		\qimplies
		\ans{ (\rho + P) \av = -\sP \cdot \grad P, }
	}
	as we wanted to show. \qed
	
	The left-hand side of Eq.~\refeq{ans1c} represents the fluid's inertial force per unit volume, since $\rho + P$ is its inertial mass per unit volume.  The right-hand side is \hl{the projection of the divergence of the pressure in the fluid's rest frame?}
	
	\hl{infer?}
}





\clearpage
\state{Inertial mass per unit volume (MCP 2.27)}{
	Suppose that some medium has a rest frame (unprimed frame) in which its energy flux and momentum density vanish, $\Toj = \Tjo = 0$.  Suppose that the medium moves in the $x$ direction with speed very small compared to light, $v \ll 1$, as seen in a (primed) laboratory frame, and ignore factors of order $v^2$.  The ratio of the medium's momentum density $\Gjp = \Tjpop$ (as measured in the laboratory frame) to its velocity $\vsi = \delsix$ is called its total \emph{inertial mass per unit volume} and is denoted $\rhosjiinert$:
	\eq{
		\Tjpop = \rhosjiinert \vsi.
	}
	In other words, $\rhosjiinert$ is the 3-dimensional tensor that gives the momentum density $\Gjp$ when the medium's small velocity is put into its second slot.
}

\prob{
	Using a Lorentz transformation from the medium's (unprimed) rest frame to the (primed) laboratory frame, show that
	\eq{
		\rhosjiinert = \Too \delsji + \Tsji.
	}
}



\prob{
	Give a physical explanation of the contribution $\Tsji$ to the momentum density.
}



\prob{
	Show that for a perfect fluid [Eq.~(2.74b)] the inertial mass per unit volume is isotropic and has magnitude $\rho + P$, where $\rho$ is the mass-energy density, and $P$ is the pressure measured in the fluid's rest frame:
	\eq{
		\rhosjiinert = (\rho + P) \delsji.
	}
%	See Ex.~2.26 for this inertial-mass role of $\rho + P$ in the law of force balance.
}





\clearpage
\newcommand{\tht}{\theta}
\newcommand{\thtxx}{\tht_{xx}}
\newcommand{\thtyy}{\tht_{yy}}
\newcommand{\thttt}{\tht_{tt}}
\newcommand{\thtt}{\tht_t}
\newcommand{\thtx}{\tht_x}
\newcommand{\thty}{\tht_y}
\newcommand{\sint}{\sin{\tht}}
\newcommand{\dxdydt}{\dxdy \dd{t}}

\begin{statement}{}
	The nondimensionalized, multidimensional Sine-Gordon equation,
	\beq
		\thtxx + \thtyy - \thttt = \sint
	\eeq
	for $\tht(x, y, t)$, is the Euler-Lagrange equation for the action integral
	\beq
		S[\tht] = \intR \left\{ \frac{1}{2} \left[ \thtt^2 - (\nabla\tht)^2 \right] - \sint \right\} \dxdydt
	\eeq
	with $\nabla\tht = (\pdv*{\tht}{x}, \pdv*{\tht}{y})$.  The functional $S[\tht]$ is invariant under translation of $x$, $y$, and $t$.  Find the associated energy-momentum tensor and energy-momentum vector.
\end{statement}

\begin{solution}
	Expanding out $(\nabla\tht)^2$, the Lagrangian density is
	\beqn \label{lagr3}
		\Ld = \frac{1}{2} (\thtt^2 - \thtx^2 - \thty^2) - \sint.
	\eeqn
	The energy-momentum tensor is defined by
	\beq
		T_{ij} = \pdv{\Ld}{\tht_{x_i}} \pdv{\tht}{x_j} - \Ld \, \delta_{ij},
	\eeq
	where $x_i \in \{ x_0, x_1, x_2 \} = \{ t, x, y \}$.  The diagonal elements of $T$ are then
	\begin{align*}
		T_{00} &= \pdv{\Ld}{\thtt} \pdv{\tht}{t} - \Ld = \thtt^2 - \frac{1}{2} (\thtt^2 - \thtx^2 - \thty^2) + \sint = \frac{1}{2} (\thtt^2 + \thtx^2 + \thty^2) + \sint, \\
		T_{11} &= \pdv{\Ld}{\thtx} \pdv{\tht}{x} - \Ld = -\thtx^2 - \frac{1}{2} (\thtt^2 - \thtx^2 - \thty^2) + \sint = -\frac{1}{2} (\thtt^2 + \thtx^2 - \thty^2) + \sint, \\
		T_{22} &= \pdv{\Ld}{\thty} \pdv{\tht}{y} - \Ld = -\thty^2 - \frac{1}{2} (\thtt^2 - \thtx^2 - \thty^2) + \sint = -\frac{1}{2} (\thtt^2 - \thtx^2 + \thty^2) + \sint,
	\end{align*}
	and the nondiagonal elements are
	\begin{align*}
		T_{01} &= \pdv{\Ld}{\thtt} \pdv{\tht}{x} = \thtt \thtx, &
		T_{02} &= \pdv{\Ld}{\thtt} \pdv{\tht}{y} = \thtt \thty, &
		T_{12} &= \pdv{\Ld}{\thtx} \pdv{\tht}{y} = -\thtx \thty, \\
		T_{10} &= \pdv{\Ld}{\thtx} \pdv{\tht}{t} = -\thtt \thtx, &
		T_{20} &= \pdv{\Ld}{\thty} \pdv{\tht}{t} = -\thtt \thty, &
		T_{21} &= \pdv{\Ld}{\thty} \pdv{\tht}{x} = -\thtx \thty.
	\end{align*}
	In matrix form, we have
	\beq
		T = \mqty[(\thtt^2 + \thtx^2 + \thty^2) / 2 + \sint & \thtt \thtx & \thtt \thty \\
				-\thtt \thtx & -(\thtt^2 + \thtx^2 - \thty^2) / 2 + \sint & -\thtx \thty \\
				-\thtt \thty & -\thtx \thty & -(\thtt^2 - \thtx^2 + \thty^2) / 2 + \sint ].
	\eeq
	The energy-momentum vector is defined by
	\beq
		P_j = \int T_{0j} \dd{x_1} \dd{x_2}.
	\eeq
	Its components are then
	\begin{align*}
		P_0 &= \int \left[ \frac{1}{2} (\thtt^2 + \thtx^2 + \thty^2) + \sint \right] \dxdy, &
		P_1 &= \int \thtt \thtx \dxdy, &
		P_2 &= \int \thtt \thty \dxdy.
	\end{align*}
\vfix
\end{solution}
\state{Degenerate Bose gas}{\hfix}

%
%	4.1
%

\prob{}{
	The chemical potential of the degenerate Bose gas vanishes below $\Ts$ (the critical temperature of the BEC).  Find its temperature dependence at temperatures slightly above $\Ts$.
}

\sol{
	In three dimensions, the energy distribution of a Bose gas is~\cite[p.~149]{Landau}
	\eqn{ddNeps}{
		\ddNeps = \frac{g V}{\pi^2 \hbar^3} \sqrt{\frac{m^3}{2}} \frac{\sqrt{\eps}}{e^{(\eps - \mu) / T} - 1} \ddeps.
	}
	Integrating over all energies, we find the total number of molecules~\cite[p.~149]{Landau}.  This gives an expression relating the chemical potential $\mu$ and the density $\nb$~\cite[p.~159]{Landau}:
	\eqn{nb}{
		\nb = \frac{g}{\pi^2 \hbar^3} \sqrt{\frac{m^3}{2}} \intoi \frac{\sqrt{\eps}}{e^{(\eps - \mu) / T} - 1} \ddeps.
	}
	The critical temperature $\Ts$ satisfies this relation for $\mu = 0$, and can be found by making the substitution $z = \eps / T$:
	\eq{
		\nb = \frac{g}{\pi^2 \hbar^3} \sqrt{\frac{m^3}{2}} \intoi \frac{\sqrt{\eps}}{e^{\eps / T} - 1} \ddeps
		= \frac{g}{\pi^2 \hbar^3} \sqrt{\frac{m^3 T^3}{2}} \intoi \frac{\sqrt{z}}{e^z - 1} \ddeps.
	}
	The integral may be evaluated using the formula~\cite[p.~156]{Landau}
	\eqn{formula}{
		\intoi \frac{z^{x - 1}}{e^z - 1} \ddz = \Gam(x) \zeta(x),
	}
	with $x > 1$.  The relevant values are $\Gam(3/2) = \sqrt{\pi} / 2$, and $\zeta(3/2) = 2.612$~\cite[p.~156]{Landau}.  Thus,
	\eq{
		\nb = \frac{g}{\pi^2 \hbar^3} \sqrt{\frac{m^3 T^3}{2}} (2.612)\frac{\sqrt{\pi}}{2}
		= \frac{g}{\pi^2 \hbar^3} \sqrt{\frac{m^3 T^3}{2}} (2.612)\frac{\sqrt{\pi}}{2}
		= \frac{0.9235 \,g}{\hbar^3} \paren{ \frac{m T}{\pi} }^{3/2},
	}
	and
	\eq{
		\paren{ \frac{m \Ts}{\pi} }^{3/2} = \frac{\nb \hbar^3}{0.9235 \,g}
		\qimplies
		\Ts = \frac{\pi}{m} \paren{ \frac{\nb \hbar^3}{0.9235 \,g} }^{2/3}
		= \frac{1.054\, \pi}{m \hbar^2} \paren{ \frac{\nb}{g} }^{2/3}.
	}
	
	Define the function
	\eq{
		\nbs(T) = \frac{g}{\pi^2 \hbar^3} \sqrt{\frac{m^3}{2}} \intoi \frac{\sqrt{\eps}}{e^{\eps / T} - 1} \ddeps
		= \frac{0.9235 \,g}{\hbar^3} \paren{ \frac{m T}{\pi} }^{3/2},
	}
	and note that $\nbs(\Ts) = \nb$.  Then we can rewrite Eq.~\refeq{nb} as
	\eq{
		\nb = \nbs(T) + \frac{g}{\pi^2 \hbar^3} \sqrt{\frac{m^3}{2}} \intoi \frac{\sqrt{\eps}}{e^{(\eps - \mu) / T} - 1} \ddeps - \nbs(T)
		= \nbs(T) + \frac{g}{\pi^2 \hbar^3} \sqrt{\frac{m^3}{2}} \intoi \paren{ \frac{\sqrt{\eps}}{e^{(\eps - \mu) / T} - 1} - \frac{\sqrt{\eps}}{e^{\eps / T} - 1} } \ddeps.
	}
	Expanding the integrand for small exponential powers using $e^x \approx 1 + x$, which is vaild for $T - \Ts \ll 1$, we find
	\eq{
		\frac{\sqrt{\eps}}{e^{(\eps - \mu) / T} - 1} - \frac{\sqrt{\eps}}{e^{\eps / T} - 1}
		\approx \frac{\sqrt{\eps}}{1 + (\eps - \mu) / T - 1} - \frac{\sqrt{\eps}}{1 + \eps / T - 1}
		= \frac{T \sqrt{\eps}}{\eps - \mu} - \frac{T}{\sqrt{\eps}}
		= \frac{T \eps - T (\eps - \mu)}{\sqrt{\eps} (\eps - \mu)}
		= \frac{T \mu}{\sqrt{\eps} (\eps - \mu)}.
	}
	Then the integral is
	\eq{
		T \mu \intoi \frac{\ddeps}{\sqrt{\eps} (\eps - \mu)} = T \mu \frac{\pi}{\sqrt{-\mu}}
		= \pi T \sqrt{-\mu},
	}
	so long as $\mu < 0$, which is true for the Bose distribution~\cite[p.~145]{Landau}.  Making this substitution and solving for $\mu$, we find
	\eqn{density}{
		\nb = \nbs(T) - \frac{g T}{\pi \hbar^3} \sqrt{\frac{-\mu m^3}{2}}
		\qimplies
		\mu = -\frac{2}{m^3} \paren{ \frac{\pi \hbar^3 [ \nbs(T) - \nb ]}{g T} }^2
		= -\frac{2 \pi^2 \hbar^6 [ \nbs(T) - \nb ]^2}{m^3 g^2 T^2}.
	}
	Note that
	\eq{
		\nbs(T) - \nb = \nb \paren{ \frac{\nbs(T)}{\nb} - 1 }
		= \nb \paren{ \frac{\nbs(T)}{\nbs(\Ts)} - 1 }
		= \nb \paren{ \frac{T^{3/2}}{{\Ts}^{3/2}} - 1 },
	}
	since $\nbs(\Ts) = \nb$.  Then the relationship between chemical potential and temperature is
	\eqn{mu}{
		\mu = -\frac{2 \pi^2 \hbar^6 \nb^2}{m^3 g^2 T^2} \paren{ \frac{T^{3/2}}{{\Ts}^{3/2}} - 1 }^2
		= \ans{ -\frac{2 \pi^2 \hbar^6 \nb^2}{m^3 g^2} \paren{ \frac{T^{1/2}}{{\Ts}^{3/2}} - \frac{1}{T} }^2. }
	}
	Since $T / \Ts \approx 1$, the leading behavior is \ans{$\mu \sim -1 / T^2$.}
}

%
%	4.2
%

\prob{}{
	Find the discontinuities in the derivatives of thermodynamic quantities (energy, particle density, entropy, thermodynamic potential, and specific heat) at the BEC transition.  Which order is this phase transition?
}

\sol{
	Using Eq.~\refeq{ddNeps}, the energy of the Bose gas is
	\eq{
		E = \intoi \eps \ddNeps
		= \frac{g V}{\pi^2 \hbar^3} \sqrt{\frac{m^3}{2}} \intoi \frac{\eps^{3/2}}{e^{(\eps - \mu) / T} - 1} \ddeps.
	}
	\clearpage
	The thermodynamic potential for a Bose gas is~\cite[p.~146]{Landau}
	\eq{
		\Omg = T \sumk \ln(1 - e^{(\mu - \epsk) / T}).
	}
	Transforming the sum to an integral as in Prob.~{3.2}, we have~\cite[p.~149]{Landau}
	\al{
		\Omg &= \frac{g V T}{\pi^2 \hbar^3} \sqrt{\frac{m^3}{2}} \intoi \sqrt{\eps} \ln(1 - e^{(\mu - \eps) / T}) \ddeps \\
		&= \frac{g V T}{\pi^2 \hbar^3} \sqrt{\frac{m^3}{2}} \paren{ \brac{ \frac{2}{3} \eps^{3/2} \ln(1 - e^{(\mu - \epsk) / T}) }\oi - \frac{2}{3 T} \intoi \frac{\eps^{3/2}}{e^{(\eps - \mu) / T} - 1} \ddeps } \\
	&= -\frac{3 g V T}{\pi^2 \hbar^3} \paren{ \frac{m}{2} }^{3/2} \intoi \frac{\eps^{3/2}}{e^{(\eps - \mu) / T} - 1} \ddeps
	= -\frac{2}{3} E.
	}
	
	Note that $N = -(\pdv*{\Omg}{\mu})_{T, V}$~\cite[p.~24]{Landau}.  Then~\cite[p.~161]{Landau}
	\eq{
		\nb = -\frac{1}{V} \pdv{\Omg}{\mu}
		= \frac{2}{3 V} \pdv{E}{\mu}
		\approx \nbs,
	}
	since the contribution to $\nb$ is small for $\mu \ll 1$.  This gives us
	\al{
		\Omg &= \Omgs - \nbs V \mu, &
		E &= \Es + \frac{3}{2} \nbs V \mu,
	}
	where $\Omgs$ and $\Es$ are the thermodynamic potential and the energy at $\mu = 0$.  Using Eq.~\refeq{formula},
	\al{
		\Es &= \frac{g V}{\pi^2 \hbar^3} \sqrt{\frac{m^3}{2}} \intoi \frac{\eps^{3/2}}{e^{\eps / T} - 1} \ddeps
		= \frac{g V}{\pi^2 \hbar^3} \sqrt{\frac{m^3 T^5}{2}} \intoi \frac{z^{3/2}}{e^z - 1} \ddz
		= \frac{g V}{\pi^2 \hbar^3} \sqrt{\frac{m^3 T^5}{2}} \Gam(5/2) \zeta(5/2) \\
		&= \frac{0.711 \,g V}{\hbar^3} \sqrt{\frac{m^3 T^5}{\pi^3}}, \\[2ex]
		\Omgs &= -\frac{0.474 \,g V}{\hbar^3} \sqrt{\frac{m^3 T^5}{\pi^3}},
	}
	both of which are continuously differentiable in $T$.  So the discontinuities in the $T$ derivatives of $\Omg$ and $E$ stem from $\mu$, given by Eq.~\refeq{mu}.  Since
	\eq{
		\pdv{\mu}{T} \sim -\pdv{T}(\frac{1}{T^2})
		\propto -\frac{1}{T^3},
	}
	where $T$ is, by definition, slightly larger than $\Ts$, we conclude that
	\ans{
	\al{
		\pdv{\Omg}{T} &\sim \frac{1}{(T - \Ts)^3}, &
		\pdv{E}{T} &\sim -\frac{1}{(T - \Ts)^3},
	}
	which both have infinite discontinuities at $T = \Ts$.
	}
	
	The particle density is given in Eq.~\refeq{density}.  Differentiating with respect to chemical potential, we see that
	\eq{
		\pdv{\nb}{\mu} = \pdv{\mu}(\nbs(T) - \frac{g T}{\pi \hbar^3} \sqrt{\frac{-\mu m^3}{2}})
		\propto \ans{ \frac{1}{\sqrt{-\mu}}, }
	}
	\ans{which diverges as $\mu \to 0$ from the left} (and is negative for real $\mu$).
	
	Entropy can be found by $S = -(\pdv*{\Omg}{T})_{V, \mu}$~\cite[p.~150]{Landau}, and the specific heat by $\Cv = (\pdv*{E}{T})_V$~\cite[p.~165]{Landau}.  Since
	\al{
		S &\sim -\frac{1}{T^3}, &
		\Cv &\sim -\frac{1}{T^3},
	}
	again for small $T - \Ts$,
	\ans{
	\al{
		\pdv{S}{T} &\sim -\pdv{T}(\frac{1}{T^3})
		\sim -\frac{1}{(T - \Ts)^4}, &
		\pdv{\Cv}{T} &\sim -\frac{1}{(T - \Ts)^4},
	}
	which both have infinite discontinuities at $T = \Ts$.
	}
	
	The order of a phase transition is determined by whether the first or the second derivative of the free energy with respect to some thermodynamic quantity is discontinuous~\cite{Wikipedia}.  The free energy can be found by $F - \mu N = \Omg$~\cite[p.~69]{Landau}.  Since $\pdv*{\mu}{T}$ and $\pdv*{\Omg}{T}$ are discontinuous, so is $\pdv*{F}{T}$.  Thus, this is a \ans{ first-order phase transition. }
}

%
%	4.3
%

\prob{}{
	Can the ideal Bose gas condense in spatial dimensions 1 and 2?  Discuss what happens in these cases. 
}

\sol{
	The ideal Bose gas can condense if the equivalent of Eq.~\refeq{nb} can be solved with $\mu = 0$ to obtain an expression for $\Ts$.  The number of quantum states in the interval $\ddp$ is the same as for a Fermi gas, and so is given by Eq.~\refeq{Nstates}~\cite[p.~148]{Landau}.  Transforming this to the number of states in the interval $\ddeps$ by Eq.~\refeq{transform}, we obtain
	\aln{ \label{thing4}
		\frac{g L}{2\pi \hbar} \sqrt{\frac{m}{2}} \frac{1}{\sqrt{\eps}} \ddeps
		&\quad (d = 1), &
		\frac{m g A}{2\pi \hbar^2} \ddeps
		&\quad (d = 2), &
		\frac{g V}{\pi^2 \hbar^3} \sqrt{\frac{m^3}{2}} \eps^{3/2} \ddeps
		&\quad (d = 3).
	}
	Applying the expression for the total number of particles in a Bose gas~\cite[p.~146]{Landau},
	\eq{
		N = \sumk \frac{1}{e^{(\epsk - \mu) / T} - 1},
	}
	replacing the sum by an integral over $p \in (0, \infty)$, and transforming coordinates to $z = \eps / \Ts$ as in Prob.~{4.1}, we obtain
	\al{
		(d = 1) \quad
		\nb &= \frac{g L}{2\pi \hbar} \sqrt{\frac{m}{2}} \intoi \frac{\ddeps}{\sqrt{\eps} (e^{\eps / \Ts} - 1)}
		= \frac{g L}{2\pi \hbar} \sqrt{\frac{m \Ts}{2}} \intoi \frac{\ddz}{\sqrt{z} (e^z - 1)}
		\to \infty, \\[2ex]
		(d = 2) \quad
		\nb &= \frac{m g A}{2\pi \hbar^2} \intoi \frac{\ddeps}{e^{\eps / \Ts} - 1}
		= \frac{m g A \Ts}{2\pi \hbar^2} \intoi \frac{\ddz}{e^z - 1}
		\to \infty.
	}
	Both integrals diverge, making it impossible to solve for $\Ts$ in either case.
	
	However, these integrals will converge in the limit that $z \to \infty$, which is equivalent to $T \to 0$.  In this limit,
		\al{
		(d = 1) \quad
		\limTo \nb &= \frac{g L}{2\pi \hbar} \sqrt{\frac{m \Ts}{2}} \intoi \frac{\ddz}{e^z \sqrt{z}}
		= \frac{g L}{2 \hbar} \sqrt{\frac{m \Ts}{2 \pi}}, \\[2ex]
		(d = 2) \quad
		\limTo \nb &= \frac{m g A \Ts}{2 \pi \hbar^3} \intoi \frac{\ddz}{e^z}
		= \frac{m g A \Ts}{2 \pi \hbar^3}.
	}
	Thus, we conclude that, \ans{it is not possible for the 1D and 2D ideal Bose gases to condense above $T = 0$.}
	
	Referring back to Eq.~\refeq{thing4}, for $d = 1$ the number of states in the interval $\ddeps$ diverges as $\eps \to 0$.  For $d = 2$, the number of states is independent of $\eps$.  For $d = 3$, the number of states approaches 0 as $\eps \to 0$.  It would seem that, in 1D and in 2D, there are many states with very low energy that may be occupied instead of $\eps = 0$, while this is not the case in 3D.  Since the particles are therefore not ``forced'' into the ground state at nonzero temperature, the gas will not condense.
}






\state{Gauss's theorem (MCP 24.11)}{
	In 3-dimensional Euclidean space Maxwell's equation $\grad \vdot \bE = \rhoe / \epso$ can be combined with Gauss's theorem to show that the electric flux through the surface $\pt\cV$ of a sphere is equal to the charge in the sphere's interior $\cV$ divided by $\epso$:
	\eq{
		\intdV \bE \vdot d\bSig = \intV \frac{\rhoe}{\epso} \ddV.
	}
	Introduce spherical polar coordinates so the sphere's surface is at some radius $r = R$.  Consider a surface element on the sphere's surface with vectorial legs $\ddphi \pdv*{\phi}$ and $\ddtht \pdv*{\tht}$.  Evaluate the components $\ddSigj$ of the surface integration element $\ddbSig = \beps(\ldots, \ddtht \pdv*{\tht}, \ddphi \pdv*{\phi})$.  (Here $\beps$ is the Levi-Civita tensor.)  Similarly, evaluate $\ddV$ in terms of vectorial legs in the sphere's interior.  Then use these results for $\ddSigj$ and $\ddV$ to convert Eq.~(24.47) into an explicit form in terms of integrals over $r$, $\tht$, and $\phi$.  The final answer should be obvious, but the above steps in deriving it are informative.
}


\makebib

\end{document}
