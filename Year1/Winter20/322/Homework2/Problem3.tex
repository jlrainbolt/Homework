\newcommand{\Gxxp}{G(\vx, \vx')}
\newcommand{\intoi}{\int_0^\infty}

\begin{statement}{}
	Charge is distributed on a (nonconducting) sphere of radius $R$, i.e., the charge density throughout space is of the form $\rhox = \sig\tv \, \delta(r - R)$.  The surface charge distribution $\sig$ on the sphere is chosen in such a way that the electrostatic potential on the sphere is $\phi(r=R, \tht, \vph) = \alp \cost$, where $\alp$ is a constant.
\end{statement}

\begin{problem} \label{3a}
	Find the electrostatic potential $\phix$ at all $r \leq R$.
\end{problem}

\begin{solution}
	The electrostatic potential can be found using the Green's function for electrostatics, $\Gxxp$, as given by Eq.~(2.23),
	\beq
		\phix = \int \Gxxp \, \rhoxp \dcxp.
	\eeq
	$\Gxxp$ can be expanded in spherical harmonics according to Eq.~(2.78):
	\beq
		\Gxxp = \frac{1}{\abs{\vx - \vx'}}
		= \begin{cases} \sum_{l,m} \dfrac{4\pi}{2l + 1} \dfrac{r^l}{{r'}^{l + 1}} \Ylm^*\tvp \, \Ylm\tv & \text{if } r < r', \\
		\sum_{l,m} \dfrac{4\pi}{2l + 1} \dfrac{{r'}^l}{r^{l + 1}} \Ylm^*\tvp \, \Ylm\tv & \text{if } r > r'. \end{cases}
	\eeq
	We can also write $\phi$ in terms of spherical harmonics:
	\beq
		\phi(r=R, \tht, \vph) = \alp \sqfr{4\pi}{3} Y_{1 0}\tv.
	\eeq
	
	For $r \leq r'$, the potential is
	\begin{align}
		\phix &= \intotp \intono \intoi \sig\tvp \, \delta(r' - R) \sum_{l,m} \frac{4\pi}{2l + 1} \frac{r^l}{{r'}^{l + 1}} \Ylm^*\tvp \, \Ylm\tv {r'}^2 \drp \dctp \dvp \notag \\
		&= \sum_{l,m} \frac{4\pi}{2l + 1} r^l \Ylm\tv \intoi \delta(r' - R) \frac{1}{{r'}^{l - 1}} \drp \intotp \intono \sig\tvp \, \Ylm^*\tvp \dctp \dvp \notag \\
		&= \sum_{l,m} \frac{4\pi}{2l + 1} r^l \Ylm\tv \frac{1}{R^{l - 1}} \intotp \intono \sig\tvp \, \Ylm^*\tvp \dctp \dvp. \label{pot1ax}
	\end{align}
	Plugging in $r = R$, we may apply the boundary condition:
	\beq
		\alp \sqfr{4\pi}{3} Y_{1 0}\tv = \sum_{l,m} \frac{4\pi}{2l + 1} R \Ylm\tv \intotp \intono \sig\tvp \, \Ylm^*\tvp \dctp \dvp,
	\eeq
	which implies that $l = 1$ and $m = 0$ are the only $\Ylm$ with nonzero coefficients.  Therefore,
	\beq
		\alp \sqfr{4\pi}{3} Y_{1 0}\tv = \frac{4\pi}{3} R \, Y_{1 0}\tv \intotp \intono \sig\tvp \, \Ylm^*\tvp \dctp \dvp,
	\eeq
	which implies
	\beqn \label{integ}
		\intotp \intono \sig\tvp \, \Ylm^*\tvp \dctp \dvp = \sqfr{3}{4\pi} \frac{\alp}{R},
	\eeqn
	so \refeq{pot1ax} becomes
	\beq
		\phix = \frac{4\pi}{3} r \sqfr{3}{4\pi} \cost \frac{1}{R^{l - 1}} \sqfr{3}{4\pi} \frac{\alp}{R}
		= \alp \frac{r}{R} \cost.
	\eeq
\end{solution}
\vfix


\begin{problem} \label{3b}
	Find the electrostatic potential $\phix$ at all $r \geq R$.
\end{problem}

\begin{solution}
	For $r \geq r'$, the potential is
	\begin{align*}
		\phix &= \sum_{l,m}\frac{4\pi}{2l + 1} \frac{1}{r^{l + 1}} \Ylm\tv \intoi \delta(r' - R) {r'}^{l+2} \drp \intotp \intono \sig\tvp \Ylm^*\tvp \dctp \dvp \\
		&= \sum_{l,m}\frac{4\pi}{2l + 1} \frac{1}{r^{l + 1}} \Ylm\tv R^{l+2} \intotp \intono \sig\tvp \Ylm^*\tvp \dctp \dvp.
	\end{align*}
	By the same arguments as in \ref{3a}, we restrict ourselves to $l = 0$ and $m = 1$ and make the substitution \refeq{integ}.  This gives us
	\beq
		\phix = \frac{4\pi}{3} \frac{R^3}{r^2} \sqfr{3}{4\pi} \cost \sqfr{3}{4\pi} \frac{\alp}{R}
		= \alp \frac{R^2}{r^2} \cost.
	\eeq
\end{solution}
\vfix


\newcommand{\Er}{E_r}
\newcommand{\thh}{\boldsymbol{\hat{\tht}}}
\newcommand{\phh}{\boldsymbol{\hat{\vph}}}

\begin{problem} \label{3c}
	Find the surface charge density $\sigtv$ that was required in order to produce this potential $\phi$.
\end{problem}

\begin{solution}
	From \refeq{integ} and the fact that $l = 1$ and $m = 0$, we need $\sig\tv = C \,Y_{1 0}\tv$ where $C$ is a constant.  Then
	\beq
		\sqfr{3}{4\pi} \frac{\alp}{R} = C \intotp \intono Y_{1 0}\tvp Y_{1 0}^*\tvp \dctp \dvp
		= C,
	\eeq
	which implies
	\beq
		\sig\tv = \sqfr{3}{4\pi} \frac{\alp}{R} \sqfr{3}{4\pi} \cost = \frac{3}{4\pi} \frac{\alp}{R} \cost.
	\eeq
\end{solution}
\vfix


\newcommand{\dr}{\dd{r}}
\newcommand{\dct}{\dd{(\cost)}}
\newcommand{\dph}{\dd{\vph}}

\begin{problem}
	Find the total electrostatic energy.
\end{problem}

\begin{solution}
	The total energy is given by \refeq{tote}.  Since $\rho$ is nonzero only on the boundary, we can use the given expression for $\phi$ on the boundary.  Feeding in our result from \ref{3c}, 
	\begin{align*}
		\sE &= \frac{1}{2} \int \phi \rho \dcx
		= \frac{1}{2} \intotp \intono \intoi \frac{3}{4\pi} \frac{\alp}{R} \cost \, \delta(r - R) \, \alp \cost r^2 \dr \dct \dph \\
		&= \frac{3}{8\pi} \frac{\alp^2}{R} \intotp \dph \intono \cos^2\tht \dct \intoi \delta(r - R) \, r^2 \dr
		= \frac{3}{8\pi} \frac{\alp^2}{R} \bigg[ \vph \bigg]_0^{2\pi} \left[ \frac{\cos^3\tht}{3} \right]_{-1}^1 R^2
		= \frac{3}{8\pi} \alp^2 R (2\pi) \frac{2}{3} \\
		&= \frac{1}{2} \alp^2 R.
	\end{align*}
\end{solution}
\vfix