\newcommand{\Vt}{V(t)}
\newcommand{\psiE}{\psi_E}
\newcommand{\qp}{^{(1)}}
\newcommand{\cnq}{c_n\qp}
\newcommand{\cnqt}{\cnq(t)}
\newcommand{\too}{t_0}
\newcommand{\inttot}{\int_{\too}^t}
\newcommand{\omg}{\omega}
\newcommand{\omgni}{\omg_{n i}}
\newcommand{\dtp}{\dd{t'}}
\newcommand{\Ho}{H_0}
\newcommand{\psiq}{\psi_1}
\newcommand{\psiw}{\psi_2}
\newcommand{\psiqt}{\psiq(t)}
\newcommand{\psiwt}{\psiw(t)}
\newcommand{\Eq}{E_1}
\newcommand{\Ew}{E_2}
\newcommand{\dxp}{\dd{x'}}
\newcommand{\dxpp}{\dd{x''}}
\newcommand{\intoi}{\int_0^\infty}
\newcommand{\intoL}{\int_0^L}
\newcommand{\intopi}{\int_0^\pi}
\newcommand{\intoepi}{\int_0^{3\pi}}
\newcommand{\du}{\dd{u}}
\newcommand{\ddv}{\dd{v}}
\newcommand{\intot}{\int_0^t}


\begin{statement}{}
	A particle is initially in the the ground state of an infinite one-dimensional potential box with walls at $x = 0$ and $x = L$.  During the time interval $0 \leq t \leq \infty$, the particle is subject to a perturbation $\Vt = x^2 e^{-t/\tau}$, where $\tau$ is a time constant.  Calculate, to first order in perturbation theory, the probability of finding the particle in its first excited state as a result of this perturbation.
\end{statement}

\begin{solution}
	The wave functions and energy eigenstates for a particle in an infinite one-dimensional box are given by Eq.~(A.2.4) in Sakurai:
	\begin{align*}
		\psiE(x) &= \sqrt{\frac{2}{L}} \sin(\frac{n \pi x}{L}), &
		E &= \frac{\hbar^2 n^2 \pi^2}{2 m L^2},
	\end{align*}
	where $n = 1, 2, 3, \ldots$  Equation~(5.6.19) gives the general expression for the transition probability from state $i$ to state $n$, which is
	\beq
		P(i \to n) = \abs*{\cnqt + c_n^{(2)}(t) + \cdots}^2.
	\eeq
	We are looking for the first order contribution, $\cnqt$, which may be found using Eq.~(5.6.17):
	\beqn \label{cn1}
		\cnqt = -\frac{i}{\hbar} \inttot \mel{n}{V_I(t')}{t} \dtp
		= -\frac{i}{\hbar} \inttot e^{i \omgni t'} V_{n i}(t') \dtp,
	\eeqn
	where
	\beq
		e^{i (E_n - E_i) t / \hbar} = e^{i \omgni t}
	\eeq
	from Eq.~(5.6.18).
	
	Let $\psi_n$ denote the wavefunctions corresponding to the eigenstates of $\Ho$ with eigenvalue $n$.  We are interested in the transition probability from $i = 1$ to $n = 2$, so the relevant wavefunctions are
	\begin{align*}
		\psiqt &= \sqrt{\frac{2}{L}} \sin(\frac{\pi x}{L}), &
		\psiwt &= \sqrt{\frac{2}{L}} \sin(\frac{2\pi x}{L}),
	\end{align*}
	and the corresponding energies are
	\begin{align*}
		\Eq &= \frac{\hbar^2 \pi^2}{2 m L^2}, &
		\Ew &= \frac{2 \hbar^2 \pi^2}{m L^2}.
	\end{align*}
	The relevant matrix element of $\Vt$ is
	\begin{align*}
		\mel{2}{\Vt}{1} &= \intoi \intoi \braket{\psiw}{x'} \mel{x'}{V}{x''} \braket{x''}{\psiq} \dxp \dxpp \\
		&= \frac{2}{L} e^{-t / \tau} \intoL \intoL \sin(\frac{2\pi x'}{L}) {x'}^2 \delta(x' - x'') \sin(\frac{\pi x''}{L}) \dxp \dxpp \\
		&= \frac{2}{L} e^{-t / \tau} \intoL \sin(\frac{2\pi x'}{L}) {x'}^2 \sin(\frac{\pi x'}{L}) \dxp
		= \frac{4}{L} e^{-t / \tau} \intoL {x'}^2 \sin[2](\frac{\pi x'}{L}) \cos(\frac{\pi x'}{L}) \dxp.
	\end{align*}
	Let $u = \pi x' / L$.  Then
	\begin{align*}
		\mel{2}{\Vt}{1} &= \frac{4 L^2}{\pi^3} e^{-t / \tau} \intopi u^2 \sin^2{u} \cos{u} \du
		= \frac{4 L^2}{\pi^3} e^{-t / \tau} \intopi u^2 (\cos{u} - \cos^3{u}) \du \\
		&= \frac{4 L^2}{\pi^3} e^{-t / \tau} \intopi u^2 \left( \cos{u} - \frac{3}{4} \cos{u} - \frac{1}{4} \cos{3u} \right) \du
		= \frac{L^2}{\pi^3} e^{-t / \tau} \intopi u^2 \left( \cos{u} - \cos{3u} \right) \du.
	\end{align*}
	For the first integral, we integrate by parts twice:
	\beq
		\intopi u^2 \cos{u} \du = \bigg[ u^2 \sin{u} \bigg]_0^\pi - 2 \intopi u \sin{u} \du
		= 2 \bigg[ u \cos{u} \bigg]_0^\pi + 2 \intopi \cos{u} \du
		= -2 \pi + 2 \bigg[ \sin{u} \bigg]_0^\pi
		= -2 \pi.
	\eeq
	For the second, let $v = 3u$.  Then we again integrate by parts twice:
	\begin{align*}
		\intopi u^2 \cos{3u} \du &= \frac{1}{27} \intoepi v^2 \cos{v} \ddv
		= \frac{1}{27} \bigg[ v^2 \sin{v} \bigg]_0^{3\pi} - \frac{2}{27} \intoepi v \sin{v} \ddv
		= \frac{2}{27} \bigg[ v \cos{v} \bigg]_0^{3\pi} + \frac{2}{27} \intoepi \cos{v} \ddv \\
		&= -\frac{2\pi}{9} + \frac{2}{27} \bigg[ \sin{v} \bigg]_0^{3\pi}
		= -\frac{2\pi}{9}.
	\end{align*}
	Then our matrix element is
	\beq
		\mel{2}{\Vt}{1} = -\frac{L^2}{\pi^2} e^{-t / \tau} \frac{16 \pi}{9}
		= -\frac{16 L^2}{9 \pi^2} e^{-t / \tau}.
	\eeq
	 Returning to \refeq{cn1}, we may now find the first-order coefficient.  First note that
	 \beq
	 	\omgni = \frac{\Ew - \Eq}{\hbar}
	 	= \frac{3 \hbar \pi^3}{2 m L^2}.
	 \eeq
	 Then
	 \begin{align*}
	 	\cnqt &= -\frac{i}{\hbar} \inttot e^{i \omgni t'} V_{2 1}(t') \dtp
	 	= \frac{i}{\hbar} \frac{16 L^2}{9 \pi^2} \intot e^{i \omgni t'}  \, e^{-t' / \tau} \dtp 
	 	= \frac{i}{\hbar} \frac{16 L^2}{9 \pi^2} \intot \exp[\left(i \omgni - \frac{1}{\tau} \right) t'] \dtp \\
	 	&= \frac{i}{\hbar} \frac{16 L^2}{9 \pi^2} \frac{1}{i\omgni - 1/\tau} \left[ \exp{\left(i \omgni - \frac{1}{\tau} \right) t'} \right]_0^t
	 	= \frac{i}{\hbar} \frac{16 L^2}{9 \pi^2} \frac{1}{i\omgni - 1/\tau} \left( e^{i\omgni t} e^{-t/\tau} - 1 \right),
	 \end{align*}
	 so the transition probability is
	 \begin{align*}
	 	\abs*{\cnqt}^2 &= \left[ \frac{i}{\hbar} \frac{16 L^2}{9 \pi^2} \frac{1}{i\omgni - 1/\tau} \left( e^{i\omgni t} e^{-t/\tau} - 1 \right) \right] \left[ \frac{i}{\hbar} \frac{16 L^2}{9 \pi^2} \frac{1}{i\omgni + 1/\tau} \left( e^{-i\omgni t} e^{-t/\tau} - 1 \right) \right] \\
	 	&= \frac{16^2 L^2}{9^2 \pi^2 \hbar^2} \frac{\tau^2}{\omgni^2 \tau^2 + 1} \left( e^{-2t/\tau} - e^{i\omgni t} e^{-t/\tau} - e^{-i\omgni t} e^{-t/tau} + 1 \right) \\
	 	&= \frac{16^2 L^2}{9^2 \pi^2 \hbar^2} \frac{\tau^2}{\omgni^2 \tau^2 + 1} \left( e^{-2t/\tau} - 2 e^{-t/\tau} \cos(\omgni t) + 1 \right).
	 \end{align*}
\end{solution}
\vfix
