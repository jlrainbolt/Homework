\newcommand{\Ha}{H_a}
\newcommand{\Hb}{H_b}
\newcommand{\Hi}{H_i}
\newcommand{\vp}{\vb{p}}
\newcommand{\vpi}{\vp_i}
\newcommand{\alp}{\alpha}
\newcommand{\ri}{r_i}
\newcommand{\ra}{r_a}
\newcommand{\rb}{r_b}
\newcommand{\rab}{r_{a b}}
\newcommand{\vx}{\vb{x}}
\newcommand{\vxi}{\vx_i}
\newcommand{\vxa}{\vx_a}
\newcommand{\vxb}{\vx_b}

\begin{statement}{}
	Consider a system of two electrons, which is described by the Hamiltonian
	\begin{align*}
		H &= \Ha + \Hb + V, &
		\Hi &= \frac{\vpi^2}{2m} - \frac{Z \alp \hbar c}{\ri}, &
		V = \frac{\alp \hbar c}{\rab}.
	\end{align*}
	Here, we label two electrons by $i = a, b$; $\ri = \abs{\vxi}$ and $\rab = \abs{\vxa - \vxb}$ where $\vxi$ is the spatial coordinate for electron $i$; and $Z$ and $\alp$ are constants.  To find an approximate ground state of $H$, let us try a variational wave function
	\beq
		\Psi(\vxa, \vxb) = \frac{A}{4\pi} e^{-B(\ra + \rb)},
	\eeq
	where $A$ is a normalization constant and $B$ is your variational parameter.
\end{statement}


\newcommand{\Hbar}{\bar{H}}
\newcommand{\ot}{\tilde{0}}
\newcommand{\kot}{\ket{\ot}}
\newcommand{\vpa}{\vp_a}
\newcommand{\vpb}{\vp_b}
\newcommand{\vr}{\vb{r}}
\newcommand{\vra}{\vec{r}_a}
\newcommand{\vrb}{\vec{r}_b}
\newcommand{\dcxa}{\dd[3]{\vxa}}
\newcommand{\dcxb}{\dd[3]{\vxb}}
\newcommand{\dcxap}{\dd[3]{\vxa'}}
\newcommand{\dcxbp}{\dd[3]{\vxb'}}
\newcommand{\dra}{\dd{\ra}}
\newcommand{\drb}{\dd{\rb}}
\newcommand{\drap}{\dd{\ra'}}
\newcommand{\drbp}{\dd{\rb'}}
\newcommand{\dr}{\dd{r}}
\newcommand{\tht}{\theta}
\newcommand{\tha}{\tht_a}
\newcommand{\dcta}{\dd{(\cos\tha)}}
\newcommand{\intono}{\int_{-1}^1}
\newcommand{\intab}{\int_{(\ra - \rb)^2}^{(\ra + \rb)^2}}

\begin{problem}
	Compute the variational energy for the given variational parameter $B$.
\end{problem}

\begin{solution}
	The general expression for the variational energy $\Hbar$ is (5.4.1) in Sakurai:
	\beqn \label{Hbar}
		\Hbar = \frac{\ev{H}{\ot}}{\braket{\ot}},
	\eeqn
	where $\kot$ is our trial ket.
	
	For this problem, the numerator of \refeq{Hbar} is
	\begin{align*}
		\ev{H}{\ot} &= \ev{H}{\Psi}
		= \iint \braket{\Psi}{\vxa, \vxb} \mel{\vxa, \vxb}{H}{\vxa', \vxb'} \braket{\vxa', \vxb'}{\Psi} \\
		&= \iint \iint \Psi(\vxa, \vxb) \mel{\vxa, \vxb}{H}{\vxa', \vxb'} \Psi(\vxa', \vxb') \dcxa \dcxb \dcxap \dcxbp,
	\end{align*}
	where
	\beq
		H = \frac{\vpa^2}{2m} + \frac{\vpb^2}{2m} - \frac{Z \alp \hbar c}{\abs{\vxa}} - \frac{Z \alp \hbar c}{\abs{\vxb}} + \frac{\alp \hbar c}{\abs{\vxa - \vxb}},
	\eeq
	so we have five integrals.  For the first,
	\begin{align*}
		\frac{A^2}{32 \pi^2 m} &\iint \iint e^{-B(\ra + \rb)} \mel{\vxa, \vxb}{\vpa^2}{\vxa', \vxb'}^2 e^{-B(\ra' + \rb')} \dcxa \dcxb \dcxap \dcxbp \\
		&= \frac{A^2}{32 \pi^2 m} \iint \iint e^{-B(\ra + \rb)} \bigg( i^2 \hbar^2 \delta(\vxa - \vxa') \delta(\vxb - \vxb') \laplacian_{a'} \bigg) e^{-B(\ra' + \rb')} \dcxa \dcxb \dcxap \dcxbp \\
		&= -\frac{A^2 \hbar^2}{2 m} \iint e^{-B(\ra + \rb)} \left( \pdv[2]{}{\ra} e^{-B(\ra + \rb)} \right) \ra^2 \rb^2 \dra \drb
		= -\frac{A^2 B^2 \hbar^2}{2 m} \intoi \ra^2 e^{-2 B \ra} \dra \intoi \rb^2 e^{-2 B \rb} \drb \\
		&= \frac{A^2 \hbar^2}{32 B^4 m},
	\end{align*}
	where we have used
	\begin{align*}
		\intoi r^2 e^{-2B r} \dr &= \left[ -\frac{r^2 e^{-2 B r}}{2 B} \right]_0^\infty + \frac{1}{B} \intoi r e^{-2 B r} \dr
		= \frac{1}{B} \left[ -\frac{r e^{-2 B r}}{2 B} \right]_0^\infty + \frac{1}{2 B^2} \intoi e^{-2 B r} \dr
		= \frac{1}{2 B^2} \left[ -\frac{e^{-2 B r}}{2 B} \right]_0^\infty \\
		&= \frac{1}{4 B^3}.
	\end{align*}
	For the second integral, we also have
	\beq
		\frac{A^2}{16 \pi^2} \frac{1}{2 m} \iint \iint e^{-B(\ra + \rb)} \mel{\vxa, \vxb}{\vpb^2}{\vxa', \vxb'}^2 e^{-B(\ra' + \rb')} \dcxa \dcxb \dcxap \dcxbp = \frac{A^2 \hbar^2}{32 B^4 m}.
	\eeq
	
	For the third integral,
	\begin{align*}
		-\frac{Z \alp \hbar c}{16\pi^2} &\iint \iint e^{-B(\ra + \rb)} \mel{\vxa, \vxb}{\frac{1}{\abs{\vxa}}}{\vxa', \vxb'} e^{-B(\ra' + \rb')} \dcxa \dcxb \dcxap \dcxbp \\
		&= -\frac{Z \alp \hbar c}{16\pi^2}\iint \iint e^{-B(\ra + \rb)} \left( \delta(\vxa - \vxa') \delta(\vxb - \vxb') \frac{1}{\abs{\vxa}} \right) e^{-B(\ra' + \rb')} \dra \drb \drap \drbp \\
		&= -A^2 Z \alp \hbar c \iint \frac{e^{-2B(\ra + \rb)}}{\ra} \ra^2 \rb^2 \dra \drb
		= -A^2 Z \alp \hbar c \intoi \ra e^{-2 B \ra} \dra \intoi \rb^2 e^{-2 B \rb} \drb \\
		&= -\frac{A^2 Z \alp \hbar c}{16 B^5}.
	\end{align*}
	For the fourth integral, we also have
	\beq
		-\frac{Z \alp \hbar c}{16\pi^2} \iint \iint e^{-B(\ra + \rb)} \mel{\vxa, \vxb}{\frac{1}{\abs{\vxb}}}{\vxa', \vxb'} e^{-B(\ra' + \rb')} \dcxa \dcxb \dcxap \dcxbp = -\frac{A^2 Z \alp \hbar c}{16 B^5}.
	\eeq
	
	For the fifth integral, we will orient our coordinate system such that $\vxb$ points in the $z$ direction.  Then
	\beq
		\frac{1}{\abs{\vxa - \vxb}} = \frac{1}{\sqrt{\vxa^2 - 2 \vxa \cdot \vxb + \vxb^2}}
		= \frac{1}{\sqrt{\ra^2 - 2 \ra \rb \cos\tha + \rb^2}},
	\eeq
	and so
	\begin{align}
		\frac{\alp \hbar c}{16\pi^2} &\iint \iint e^{-B(\ra + \rb)} \mel{\vxa, \vxb}{\frac{1}{\abs{\vxa - \vxb}}}{\vxa', \vxb'} e^{-B(\ra' + \rb')} \dcxa \dcxb \dcxap \dcxbp \notag \\
		&= \frac{A^2 \alp \hbar c}{16\pi^2} \iint \iint e^{-B(\ra + \rb)}  \left( \delta(\vxa - \vxa') \delta(\vxb - \vxb') \frac{1}{\abs{\vxa - \vxb}} \right) e^{-B(\ra' + \rb')} \dcxa \dcxb \dcxap \dcxbp \notag \\
		&= \frac{A^2 \alp \hbar c}{2} \intoi \intono \intoi \frac{e^{-2B(\ra + \rb)}}{\sqrt{\ra^2 - 2 \ra \rb \cos\tha + \rb^2}} \ra^2 \rb^2 \dra \dcta \drb \notag \\
		&= \frac{A^2 \alp \hbar c}{2} \intoi \intoi \ra^2 \rb^2 e^{-2B(\ra + \rb)} \intono \frac{\dcta}{\sqrt{\ra^2 - 2 \ra \rb \cos\tha + \rb^2}} \dra \drb. \label{leftoff}
	\end{align}
	For the innermost integral, let $u = \ra^2 - 2 \ra \rb \cos\tha + \rb^2$.  Then
	\beq
		\dcta = -\frac{\du}{2 \ra \rb},
	\eeq
	and we are integrating from $\ra^2 + 2 \ra \rb + \rb^2 = (\ra + \rb)^2$ to $\ra^2 - 2 \ra \rb + \rb^2 = (\ra - \rb)^2$.  So the innermost integral becomes
	\begin{align*}
		\intono \frac{\dcta}{\sqrt{\ra^2 - 2 \ra \rb \cos\tha + \rb^2}} &= \frac{1}{2 \ra \rb} \intab \frac{\du}{\sqrt{u}}
		= \frac{1}{2 \ra \rb} \bigg[ 2 \sqrt{u} \bigg]_{(\ra - \rb)^2}^{(\ra + \rb)^2}
		= \frac{\abs{\ra + \rb} - \abs{\ra - \rb}}{\ra \rb} \\
		&= \begin{cases}
			\dfrac{\ra + \rb - \ra + \rb}{\ra \rb} = \dfrac{2}{\ra} & \text{if } \ra > \rb, \\[2ex]
			\dfrac{\ra + \rb + \ra - \rb}{\ra \rb} = \dfrac{2}{\rb} & \text{if } \rb > \ra,
		\end{cases}
	\end{align*}
	where we have used $\ra, \rb > 0$.  Picking up from \refeq{leftoff}, we now have
	\begin{align}
		\frac{A^2 \alp \hbar c}{16\pi^2} &\iint \iint e^{-B(\ra + \rb)} \mel{\vxa, \vxb}{\frac{1}{\abs{\vxa - \vxb}}}{\vxa', \vxb'} e^{-B(\ra' + \rb')} \dcxa \dcxb \dcxap \dcxbp \notag \\
		&= A^2 \alp \hbar c \intoi \ra^2 e^{-2B \ra} \left( \frac{1}{\ra} \int_0^{\ra} \rb^2 e^{-2B \rb} \drb + \int_{\ra}^\infty \rb e^{-2B \rb} \drb \right) \dra, \label{leftoff2}
	\end{align}
	where the first inner integral is
	\begin{align*}
		\frac{1}{\ra} \int_0^{\ra} \rb^2 e^{-2B \rb} \drb &= \frac{1}{\ra} \left[ -\frac{\rb^2 e^{-2 B \rb}}{2 B} \right]_0^{\ra} + \frac{1}{B \ra} \int_0^{\ra} \rb e^{-2B \rb} \drb \\
		&= -\frac{\ra e^{-2B \ra}}{2 B} + \frac{1}{B \ra} \left[ -\frac{\rb e^{-2B \rb}}{2 B} \right]_0^{\ra} + \frac{1}{2 B^2 \ra} \int_0^{\ra} e^{-2B \rb} \drb \\
		&= -\frac{\ra e^{-2B \ra}}{2 B} - \frac{e^{-2B \ra}}{2 B^2} + \frac{1}{2 B^2 \ra} \left[ -\frac{e^{-2B \rb}}{2B} \right]_{\ra}^\infty \\
		&= -\frac{\ra e^{-2B \ra}}{2 B} - \frac{e^{-2B \ra}}{2 B^2} - \frac{e^{-2B \ra}}{4 B^3 \ra} + \frac{1}{4 B^3 \ra},
	\end{align*}
	and the second is
	\beq
		\int_{\ra}^\infty \rb e^{-2B \rb} \drb = \left[ -\frac{\rb e^{-2B \rb}}{2 B} \right]_{\ra}^\infty + \frac{1}{2B} \int_{\ra}^\infty e^{-2B \rb} \drb
		= \frac{\ra e^{-2 \ra}}{2B} + \frac{1}{2B} \left[ -\frac{e^{-2B \rb}}{2 B} \right]_{\ra}^\infty
		= \frac{\ra e^{-2B \ra}}{2 B} + \frac{e^{-2B \ra}}{4 B^2}.
	\eeq
	Picking up from \refeq{leftoff2},
	\begin{align*}
		\frac{A^2 \alp \hbar c}{16\pi^2} &\iint \iint e^{-B(\ra + \rb)} \mel{\vxa, \vxb}{\frac{1}{\abs{\vxa - \vxb}}}{\vxa', \vxb'} e^{-B(\ra' + \rb')} \dcxa \dcxb \dcxap \dcxbp \\
		&= A^2 \alp \hbar c \intoi \ra^2 e^{-2B \ra} \left( \frac{1}{4 B^3 \ra} - \frac{e^{-2B \ra}}{4 B^3 \ra} - \frac{e^{-2B \ra}}{4 B^2} \right) \dra \\
		&= \frac{A^2 \alp \hbar c}{4B^3} \left( \intoi \ra e^{-2B \ra} \dra - \intoi \ra e^{-4B \ra} \dra - B \intoi \ra^2 e^{-4B \ra} \dra \right) \\
		&= \frac{A^2 \alp \hbar c}{4B^3} \left( \frac{1}{4 B^2} - \frac{1}{16 B^2} - \frac{1}{32 B^2} \right)
		= \frac{5 A^2 \alp \hbar c}{128 B^5}.
	\end{align*}
	Putting this all together,
	\beq
		\ev{H}{\ot} = \frac{5}{8} \frac{A^2 \alp \hbar c}{16 B^5} + \frac{A^2 B \hbar^2 / m}{16 B^5} - 2 \frac{A^2 Z \alp \hbar c}{16 B^5}
		= \frac{A^2}{16 B^5} \left[ \left( \frac{5}{8} - 2 Z \right) \alp \hbar c + \frac{B \hbar^2}{m} \right].
	\eeq
	For the denominator of \refeq{Hbar},
	\begin{align*}
		\braket{\ot} &= \iint \braket{\Psi}{\vxa, \vxb} \braket{\vxa, \vxb}{\Psi} \dcxa \dcxb
		= \frac{A^2}{16 \pi^2} \iint e^{-B(\ra + \rb)} e^{-B(\ra + \rb)} \ra^2 \rb^2 \dra \drb \\
		&= A^2 \intoi \ra^2 e^{-2B \ra} \dra \intoi \rb^2 e^{-2B \rb} \drb
		= \frac{A^2}{16 B^6}.
	\end{align*}
	Finally,
	\beqn \label{Hsol}
		\Hbar = B \left( \frac{5}{8} - 2 Z \right) \alp \hbar c + \frac{B^2 \hbar^2}{m}.
	\eeqn
\end{solution}
\vfix



\begin{problem}
	By minimizing the variational energy, find the optimal value of $B$.
\end{problem}

\begin{solution}
	By (5.4.9) in Sakurai, we can minimize $\Hbar$ by setting to zero its derivative with respect to $B$.  From \refeq{Hsol}, we have
	\beq
		\pdv{\Hbar}{B} = A^2 (1 - 2 Z) \alp \hbar c + 2 \frac{A^2 B \hbar^2}{m} = 0
	\eeq
	which implies
	\beq
		\left( \frac{5}{8} - 2 Z \right) \alp c = -2 \frac{B \hbar}{m}
		\implies
		B = \frac{2 Z - 5/8}{2 \hbar} \alp c m
	\eeq
	is the optimal value of $B$.
\end{solution}