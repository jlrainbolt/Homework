\documentclass[11pt]{article}
\usepackage{geometry, titlesec}
\usepackage[parfill]{parskip}
\usepackage[italicdiff]{physics}
\usepackage{amsfonts, amsthm}
\usepackage{fullpage}
\usepackage{fancyhdr}
\usepackage{xcolor}
\usepackage{siunitx}
\usepackage{enumitem}
%\allowdisplaybreaks
 
\renewcommand{\footrulewidth}{.2pt}
\setlist[enumerate]{leftmargin=*}
\pagestyle{fancy}
\fancyhf{}
\lhead{Homework 4}
\rhead{Physics 133-B}
\setlength{\headheight}{11pt}
\setlength{\headsep}{11pt}
\setlength{\footskip}{24pt}
\cfoot{\today}
%\rfoot{\thepage}

\titleformat{\section}[runin]{\normalfont\large\bfseries}{Problem \thesection.}{1em}{}
\titleformat{\subsection}[runin]{\normalfont\large\bfseries}{\thesubsection}{1em}{}
\titleformat{\subparagraph}[leftmargin]{\normalfont\normalsize\bfseries}{}{0pt}{}
\newcommand{\refeq}[1]{(\ref{#1})}

\newcommand{\beq}{\begin{equation*}}
\newcommand{\eeq}{\end{equation*}}

\newcommand{\beqn}{\begin{equation}}
\newcommand{\eeqn}{\end{equation}}

\newcommand{\qimplies}{\quad \implies \quad}



\begin{document}

\begin{enumerate}

\newcommand{\freq}{\SI{97.1}{\mega\Hz}}
\newcommand{\length}{\SI{50.0}{\meter}}

% Y&F 32.11,  index of refraction
\item Chicago's radio station 97.1 FM ``The Drive'' uses the frequency {\freq}.  Imagine that you are in the pool at the Ratner athletics center, and you want to tune in to this station using an underwater radio.  What is the frequency of the broadcast under the water?  (The index of refraction of water is about $1.333$.)  How long would it take the broadcast to travel from one end of the {\length} pool to the other?  How long would it take to travel the same distance in vacuum?


\newcommand{\critang}{\SI{50.0}{\degree}}
\newcommand{\intang}{\SI{40.0}{\degree}}

% Y&F 33.17
\item Consider the interface between air and some liquid.  The critical angle for total internal reflection at this interface is {\critang}.  A ray of light traveling in the liquid hits the interface at incident angle {\intang} and is refracted in the air.  What is the angle between the refracted ray and the normal to the surface?  Now say that a ray traveling in the \emph{air} hits the interface at incident angle {\intang} and is refracted in the \emph{liquid}.  What is the angle between the refracted ray and the normal?

\end{enumerate}




\end{document}