\newcommand{\vr}{\vb{r}}
\newcommand{\vrperp}{\vr_\perp}

\begin{statement}{(Jackson 11.18)}
	The electric and magnetic fields of a particle of charge $q$ moving in a straight line with speed $v = \bet c$, given by \refeq{fields}, become more and more concentrated as $\bet \to 1$.  Choose axes so that the charge moves along the $z$ axis in the positive direction, passing the origin at $t = 0$.  Let the spatial coordinates of the observation point be $(x, y, z)$ and define the transverse vector $\vrperp$, with components $x$ and $y$.  Consider the fields and the source in the limit of $\bet = 1$.
\end{statement}

\newcommand{\rperp}{r_\perp}
\newcommand{\vE}{\vb{E}}
\newcommand{\vB}{\vb{B}}
\newcommand{\vh}{\vb{\hat{v}}}
\newcommand{\delctz}{\del(ct - z)}

\begin{problem} \label{3.a}
	Show that the fields can be written as
	\begin{align} \label{show3.a}
		\vE &= 2q \frac{\vrperp}{\rperp^2} \delctz, &
		\vB &= 2q \frac{\vh \cross \vrperp}{\rperp^2} \delctz,
	\end{align}
	where $\vh$ is a unit vector in the direction of the particle's velocity.
\end{problem}


\newcommand{\xh}{\vb{\hat{x}}}
\newcommand{\yh}{\vb{\hat{y}}}
\newcommand{\zh}{\vb{\hat{z}}}
\newcommand{\intii}{\int_{-\infty}^\infty}
\newcommand{\limgam}{\lim_{\gam \to \infty}}
\newcommand{\du}{\dd{u}}
\newcommand{\dz}{\dd{z}}
\newcommand{\vEperp}{\vE_\perp}

\begin{solution}
	Let $K'$ denote the rest frame of this charge.  In this frame,
	\begin{align*}
		\vE' &= \frac{q}{{r'}^3} \vr', &
		\vB' &= \vo,
	\end{align*}
	where $\vr'$ is the observation point in $K'$.
\clearpage
	First we will boost into the lab frame.  Jackson~(11.148) gives the transformations of the electric field for a boost in the $x$ direction.  Adapting these for the $z$ direction, we have
	\begin{align*}
		\Eq &= \gam (\Eq' - \bet \Be') &
		\Ew &= \gam (\Ew' + \bet \Bw') &
		\Ee &= \Ee',
	\end{align*}
	where $\bet < 0$ indicates a boost in the $-z$ direction.  Note also that
	\begin{align*}
		ct' &= \gam (c t + \bet z), &
		x' &= x, &
		y' &= y, &
		z' &= \gam (z + \bet c t).
	\end{align*}
	Let $\bet = v / c$, so $\bet < 0$.  Then
	\begin{align*}
		{r'}^2 &= {x'}^2 + {y'}^2 + {z'}^2
		= x^2 + y^2 + \gam^2 (z - \bet c t)^2
		= \rperp^2 + \gam^2 (z - v t)^2, \\
		\vr' &= x' \,\xh' + y' \,\yh' + z' \,\zh'
		= x \,\xh + y \,\yh + \gam (z - \bet c t)\,\zh
		= \vrperp + \gam (z - v t)\,\zh,
	\end{align*}
	and in the lab frame, the perpendicular component of the electric field is
	\beq
		\vEperp = \gam q \frac{\vrperp}{(\rperp^2 + \gam^2 (z - v t)^2)^{3/2}}.
	\eeq
	Taking the limit of this expression as $\bet \to 1$ is identical to taking the limit as $\gam \to \infty$.  Doing so, we find
	\beq
		\limgam \vEperp = q \vrperp \limgam \frac{\gam}{(\rperp^2 + \gam^2 (z - v t)^2)^{3/2}}
		= \begin{cases}
			\infty & \text{if } z = vt, \\
			0 & \text{otherwise},
		\end{cases}
	\eeq
	so we can conclude that
	\beq
		\limgam \vEperp = k q \,\del(vt - z) \,\vrperp
		= k q \,\del(ct - z) \,\vrperp
	\eeq
	where $k$ is some constant, and we have made the replacement $v \to c$ as $\bet \to 1$.  To find $k$, we use the fact that the integral of $\delctz$ from $z = -\infty$ to $z = \infty$ must be 1.  Let $u = \gam (z - vt)$.  Then $\dz = \du / \gam$, and
	\beq
		k = \intii \frac{\gam}{(\rperp^2 + \gam^2 (z - v t)^2)^{3/2}} \dz
		= \intii \frac{\du}{(\rperp^2 + u^2)^{3/2}}
		= \left[ \frac{u}{\rperp^2 \sqrt{\rperp^2 + u^2}} \right]_{-\infty}^\infty
		= \frac{2}{\rperp^2}.
	\eeq
	Finally, we have shown
	\beq
		\vEperp = 2 q \frac{\vrperp}{\rperp^2} \,\delctz
	\eeq
	as desired.
	
	For the magnetic field, Jackson~(11.150) states that $\vB = \vbet \cross \vE$.  In the limit $\bet \to 1$, $\vbet \to \zh = \vh$.  Then we have
	\beq
		\vB = 2q \frac{\vh \cross \vrperp}{\rperp^2} \delctz,
	\eeq
	as desired.  This completes our proof of \refeq{show3.a}. \qed
\end{solution}



\newcommand{\Ja}{J^\alp}
\newcommand{\Jb}{J^\bet}
\newcommand{\dela}{\partial_\alp}
\newcommand{\valp}{v^\alp}
\newcommand{\delrp}{\del^2(\vrperp)}

\begin{problem}
	Show that by substitution into the Maxwell equations that these fields are consistent with a 4-vector source density,
	\beq
		\Ja = q c \valp \delrp \,\delctz,
	\eeq
	where the 4-vector $\valp = (1, \vh)$.
\end{problem}

\newcommand{\cV}{\mathcal{V}}
\newcommand{\dcV}{\partial \cV}
\newcommand{\dS}{\partial S}
\newcommand{\nh}{\vb{\hat{n}}}
\newcommand{\vro}{\vr_0}
\newcommand{\grperp}{\grad_{\!\perp}}

\newcommand{\dtht}{\dd{\tht}}
\newcommand{\ddcV}{\dd{(\dcV)}}
\newcommand{\ddS}{\dd{(\dS)}}
\newcommand{\intotp}{\int_0^{2\pi}}
\newcommand{\intS}{\int_S}
\newcommand{\intdS}{\int_{\dS}}
\newcommand{\intcV}{\int_\cV}
\newcommand{\intdcV}{\int_{\dcV}}
\newcommand{\dcr}{\dd[3]{r}}
\newcommand{\dsr}{\dd[2]{r}}

\newcommand{\Jo}{J^0}
\newcommand{\vJ}{\vb{J}}
\newcommand{\vaa}{\vb{a}}
\newcommand{\rperpx}{{\rperp}_x}
\newcommand{\rperpy}{{\rperp}_y}


\begin{solution}
	From Jackson~(11.128), $\Ja = (c \rho, \vJ)$.  From Wald~(5.4) and (5.5), the inhomogeneous Maxwell equations are
	\begin{align} \label{maxwell}
		\grad \vdot \vE &= 4\pi \rho, &
		\grad \cross \vB - \frac{1}{c} \pdv{\vE}{t} &= \frac{4\pi}{c} \vJ.
	\end{align}
\clearpage
	For the first equation,
	\beq
		\grad \vdot \vE = 2 q \,\delctz \left( \grperp \vdot \frac{\vrperp}{\rperp^2} \right).
	\eeq
	Gauss's theorem is given by Wald~(2.6),
	\beq
		\intcV \grad \vdot \vv \dcr = \intdcV \vv \vdot \nh \ddcV,
	\eeq
	where $\cV$ is a three-dimensional bounded region with surface $\dcV$, $\vv$ is an arbitrary vector field, and $\nh$ is an outward pointing unit vector.  In two dimensions, this becomes
	\beq
		\intS \grperp \vdot \vv \dsr = \intdS \vv \vdot \nh \ddS,
	\eeq
	where $S$ is a two-dimensional bounded region with boundary $\dS$, and $\grperp$ is the two-dimensional gradient.  Taking the surface as a circle of radius $\rperp$ in the $xy$ plane, $\ddS = \rperp \dtht$ in plane polar coordinates.  Then
	\beq
		\intS \grperp \vdot \frac{\vrperp}{\rperp^2} \dsr = \intdS \frac{\vrperp}{\rperp^2} \vdot \frac{\vrperp}{\rperp} \ddS
		= \intdS \frac{\ddS}{\rperp}
		= \intotp \frac{\rperp}{\rperp} \dtht
		= \intotp \dtht
		= 2\pi,
	\eeq
	which implies
	\beq
		\grperp \vdot \frac{\vrperp}{\rperp^2} = 2\pi \,\delrp.
	\eeq
	Substituting into \refeq{maxwell}, we have
	\beqn \label{rho}
		4\pi q \,\delrp \,\delctz = 4\pi \rho
		\qimplies
		\rho = q \,\delrp \,\delctz
		\qimplies
		\Jo = c q \,\delrp \,\delctz.
	\eeqn
	
	For the second equation of \refeq{maxwell}, note that $\pdv*{\vE}{t} = 0$, and
	\beq
		\grad \cross \vB = 2q \,\delctz \left[ \grad \cross \left( \vh \cross \frac{\vrperp}{\rperp^2} \right) \right].
	\eeq
	From the inside cover of Jackson,
	\beq
		\grad \cross (\vaa \cross \vbb) = \vaa (\grad \vdot \vbb) - \vbb (\grad \vdot \vaa) + (\vbb \vdot \grad) \vaa - (\vaa \vdot \grad) \vbb.
	\eeq
	Note that
	\begin{align*}
		\grad \vdot \frac{\vrperp}{\rperp^2} &= 2\pi \,\delrp, &
		\grad \vdot \vh &= 0, &
		\left( \frac{\vrperp}{\rperp^2} \vdot \grad \right) \vv &= 0, &
		\vv \vdot \grad &= 0,
	\end{align*}
	so
	\beq
		\grad \cross \left( \vh \cross \frac{\vrperp}{\rperp^2} \right) = 2\pi \vh \,\delrp.
	\eeq
	Substituting into \refeq{maxwell}, we find
	\beq
		4\pi q \vh \,\delrp \,\delctz = \frac{4\pi}{c} \vJ
		\qimplies
		\vJ = q c \vh \,\delrp \,\delctz.
	\eeq
	Combining this result with \refeq{rho}, we have shown
	\beq
		\Ja = (c q \,\delrp \,\delctz, q c \vh \,\delrp \,\delctz)
		= q c \valp \delrp \,\delctz,
	\eeq
	as desired. \qed
\end{solution}



\newcommand{\Ao}{A^0}
\newcommand{\Az}{A^z}
\newcommand{\vA}{\vb{A}}
\newcommand{\vAperp}{\vA_\perp}
\newcommand{\Thtctz}{\Tht(ct - z)}

\begin{problem}
	Show that the fields of \refeq{3.a} are derivable from either of the following 4-vector potentials,
	\begin{align} \label{given3.c.1}
		\Ao &= \Az = -2q \,\delctz \ln(\lam \rperp), &
		\vAperp &= \vo,
	\end{align}
	or
	\begin{align} \label{given3.c.2}
		\Ao &= 0 = \Az, &
		\vAperp &= -2q \,\Thtctz \,\grperp \ln(\lam \rperp),
	\end{align}
	where $\lam$ is an irrelevant parameter setting the scale of the logarithm.
	
	Show that the two potentials differ by a gauge transformation and find the gauge function, $\chi$.
\end{problem}

\newcommand{\Aa}{A^\alp}
\newcommand{\Apa}{{A'}^\alp}
\newcommand{\Apo}{{A'}^0}
\newcommand{\Apz}{{A'}^z}

\begin{solution}
	From Jackson~(11.132), $\Aa = (\phi, \vA)$.  From Wald~(5.2) and (5.3), the fields are derived from the potentials as
	\begin{align} \label{potentials}
		\vE &= -\grad\phi - \frac{1}{c} \pdv{\vA}{t}, &
		\vB &= \grad \cross \vA.
	\end{align}
	
	For the potentials of \refeq{given3.c.1}, note that
	\begin{align*}
		\grad\Ao &= -2q \,\delctz \left( \pdv{x} \xh + \pdv{y} \yh + \pdv{z} \zh \right) \ln( \lam \sqrt{x^2 + y^2})
		= -2q \,\delctz \left( \frac{x}{x^2 + y^2} \xh + \frac{y}{x^2 + y^2}\yh \right) \\
		&= -2q \frac{\vrperp}{\rperp^2} \delctz, \\[2ex]
		\grad \cross \vA &= \left( \pdv{x} \xh + \pdv{y} \yh + \pdv{z} \right) \cross \Az \,\zh
		= \left( \pdv{y} \xh - \pdv{x} \yh \right) \Az
		= -2q \,\delctz \left( \pdv{y} \xh - \pdv{x} \yh \right) \ln( \lam \sqrt{x^2 + y^2}) \\
		&= -2q \,\delctz \left( \frac{y}{x^2 + y^2} \xh - \frac{x}{x^2 + y^2} \yh \right)
		= -2q \frac{\vrperp \cross \zh}{\rperp^2} \,\delctz
		= 2q \frac{\vh \cross \vrperp}{\rperp^2} \,\delctz,
	\end{align*}
	while $\pdv*{\vAperp}{t} = \vo$.  Substitution into \refeq{potentials} yields
	\begin{align*}
		\vE &= 2q \frac{\vrperp}{\rperp^2} \delctz, &
		\vB &= 2q \frac{\vh \cross \vrperp}{\rperp^2} \,\delctz,
	\end{align*}
	which are identical to \refeq{show3.a}, as desired.
	
	For the potentials of \refeq{given3.c.2}, we assume that $\Tht(x)$ denotes the Heaviside step function.  Then, according to Wolfram Mathworld,
	\beq
		\dv{\Tht(x)}{x} = \del(x).
	\eeq
	Note also that
	\beq
		\vAperp = -2q \,\Thtctz \left( \pdv{x} \xh + \pdv{y} \yh \right) \ln(\lam \sqrt{x^2 + y^2})
		= -2q \frac{\vrperp}{\rperp^2} \,\Thtctz,
	\eeq
	so
	\begin{align*}
		\grad \cross \vA &= -2q \left( \pdv{x} \xh + \pdv{y} \yh + \pdv{z} \zh \right) \cross \Thtctz \left( \frac{x}{x^2 + y^2} \xh + \frac{y}{x^2 + y^2} \yh \right) \\
		&= -2q \left[ \Thtctz \left( \pdv{x} \frac{y}{x^2 + y^2} - \pdv{y} \frac{x}{x^2 + y^2} \right) \zh + \left( \frac{x}{x^2 + y^2} \yh - \frac{y}{x^2 + y^2} \xh \right) \pdv{z} \Thtctz \right] \\
		&= -2q \left[ \Thtctz \left( -\frac{2 x y}{(x^2 + y^2)^2} + \frac{2 x y}{(x^2 + y^2)^2} \right) \zh + \left( \frac{y}{x^2 + y^2} \xh - \frac{x}{x^2 + y^2} \yh \right) \delctz \right] \\
		&= -2q \frac{\vrperp \cross \zh}{\rperp^2} \,\delctz
		= 2q \frac{\vh \cross \vrperp}{\rperp^2} \,\delctz,
	\end{align*}
	and
	\beq
		\pdv{\vAperp}{t} = -2q c \frac{\vrperp}{\rperp^2} \delctz,
	\eeq
	while $\grperp\Ao = \vo$.  Substituting into \refeq{potentials}, we once again recover \refeq{show3.a}.
	
	According to Wald~(1.13), the general gauge transformations for $\Aa$ are given by
	\begin{align*}
		\phi' &= \phi - \frac{1}{c} \pdv{\chi}{t}, &
		\vA' &= \vA + \grad\chi,
	\end{align*}
	where $\chi = \chi(t, \vr)$ is a gauge function.  Inspecting \refeq{given3.c.1} and \refeq{given3.c.2}, we make the ansatz
	\beqn \label{chi}
		{\color{blue} \chi = -2q \,\Thtctz \ln(\lam \rperp)},
	\eeqn
	which we will now prove.  Denote \refeq{given3.c.1} by $\Aa$ and \refeq{given3.c.2} by $\Apa$.  Then we have
	\begin{align*}
		\Ao - \frac{1}{c} \pdv{\chi}{t} &= -2q \,\delctz \ln(\lam \rperp) + \frac{2q}{c} \ln(\lam \rperp) \pdv{t} \Thtctz
		= -2q \,\delctz \ln(\lam \rperp) + 2q \,\delctz \ln(\lam \rperp) \\
		&= 0
		= \Apo, \\[2ex]
		\vAperp + \grperp \chi &= -2q \,\Thtctz \, \grperp \ln(\lam \rperp)
		= \vAperp', \\[2ex]
		\Az + \pdv{\chi}{z} &= -2q \,\delctz \ln(\lam \rperp) - 2q \ln(\lam \rperp) \pdv{z} \Thtctz
		= -2q \,\delctz \ln(\lam \rperp) + 2q \,\delctz \ln(\lam \rperp) \\
		&= 0
		= \Apz,
	\end{align*}
	so we have shown that \refeq{given3.c.1} and \refeq{given3.c.2} differ by the gauge transformation in \refeq{chi}. \qed
\end{solution}