\state{Degenerate Fermi gas}{
	Consider a Fermi gas in 1, 2, and 3 spatial dimensions with density $\nb = N / V$.
}

%
%	3.1
%

\prob{}{
	First, set the temperature to zero ($T = 0$) and find the Fermi momentum, Fermi energy, and the total energy in all three cases as a function of density.
}

\sol{
	The particles in a completely degenerate Fermi gas ($T = 0$) are distributed among the lowest energy states, which correspond to the lowest momentum states.  These states have momentum less than or equal to the Fermi momentum $\po$.
	\clearpage
	The number of quantum states in the interval $(p, p + \ddp)$ is, in each case~\cite[p.~152]{Landau},
	\aln{ \label{Nstates}
		\frac{g L}{2\pi \hbar} \ddp &\quad (d = 1), &
		\frac{2\pi g A}{(2\pi \hbar)^2} p \ddp &\quad (d = 2), &
		\frac{4\pi g V}{(2\pi \hbar)^3} p^2 \ddp &\quad (d = 3),
	}
	where $g = 2s +1$ with $s$ being the spin of the particle, and $L$, $A$, and $V$ indicate the volume in 1, 2, and 3 spatial dimensions.
	
	Let $N$ be the number of particles occupying these states, which is found by integrating these quantities over $p \in (0, \po)$.  For each case,
	\al{
		(d = 1) \quad
		N &= \frac{g L}{2\pi \hbar} \intopo \ddp
		= \frac{g L}{2\pi \hbar} \bigg[ p \bigg]\opo
		= \frac{g L \po}{2\pi \hbar}, \\[2ex]
		%
		(d = 2) \quad
		N &= \frac{2\pi g A}{(2\pi \hbar)^2} \intopo p \ddp
		= \frac{2\pi g A}{(2\pi \hbar)^2} \brac{ \frac{p^2}{2} }\opo
		= \frac{g A \po^2}{4\pi \hbar^2}, \\[2ex]
		%
		(d = 3) \quad
		N &= \frac{4\pi g V}{(2\pi \hbar)^3} \intopo p^2 \ddp
		= \frac{4\pi g V}{(2\pi \hbar)^3} \brac{ \frac{p^3}{3} }\opo
		= \frac{g V \po^3}{6 \pi^2 \hbar^3}.
	}
	Solving each case for $\po$,
	\aln{
		(d = 1) \quad
		\po &= \frac{2\pi \hbar N}{g L}
		= \ans{ \frac{2\pi \hbar \nb}{g}, } \notag \\[2ex]
		%
		(d = 2) \quad
		\po &= \sqrt{\frac{4\pi \hbar^2 N}{g A}}
		= \ans{ 2 \hbar \sqrt{\frac{\pi \nb}{g}}, } \label{p0} \\[2ex]
		%
		(d = 3) \quad
		\po &= \paren{ \frac{6 \pi^2 \hbar^3 N}{g V} }^{1/3}
		= \ans{ \hbar \paren{ \frac{6 \pi^2 \nb}{g} }^{1/3}. } \notag
	}
	
	The Fermi energy is found by $\epso = \po^2 / 2m$ in all cases~\cite[p.~152]{Landau}.  Thus, we have
		\aln{
		(d = 1) \quad
		\epso &= \frac{1}{2m} \paren{  \frac{2\pi \hbar \nb}{g} }^2
		= \ans{ \frac{2 \pi^2 \hbar^2 \nb^2}{m g^2}, } \notag \\[2ex]
		%
		(d = 2) \quad
		\epso &= \frac{1}{2m} \paren{ 2 \hbar \sqrt{\frac{\pi \nb}{g}} }^2
		= \ans{ \frac{2 \pi \hbar^2 \nb}{m g}, } \label{eps0} \\[2ex]
		%
		(d = 3) \quad
		\epso &= \frac{1}{2m} \brac{ \hbar \paren{ \frac{6 \pi^2 \nb}{g} }^{1/3} }^2
		= \ans{ \frac{\hbar^2}{2m} \paren{ \frac{6 \pi^2 \nb}{g} }^{2/3}. } \notag
	}
	
	The total energy of the gas is found by multiplying Eq.~\refeq{Nstates} by $\eps = p^2 / m$ and integrating over $p \in (0, \po)$~\cite[p.~153]{Landau}:
	\al{
		(d = 1) \quad
		E &= \frac{g}{2m} \frac{L}{2\pi \hbar} \intopo p^2 \ddp
		= \frac{g}{2m} \frac{L}{2\pi \hbar} \brac{ \frac{p^3}{3} }\opo
		= \frac{g}{6m} \frac{L}{2\pi \hbar} \paren{ \frac{2\pi \hbar \nb}{g} }^3
		= \frac{(2\pi \hbar)^2 L}{6m g^2} \nb^3
		= \ans{ \frac{2 \pi^2 \hbar^2 N \nb^2}{3 m g^2}, } \\[2ex]
		%
		(d = 2) \quad
		E &= \frac{g}{2m} \frac{2\pi A}{(2\pi \hbar)^2} \intopo p^3 \ddp
		= \frac{g}{2m} \frac{2\pi A}{(2\pi \hbar)^2} \brac{ \frac{p^4}{4} }\opo
		= \frac{g}{8m} \frac{2\pi A}{(2\pi \hbar)^2} \paren{ 2\pi \hbar \sqrt{\frac{\nb}{\pi g}} }^4
		= \frac{(2\pi \hbar)^2 A}{4\pi m g} \nb^2
		= \ans{ \frac{\pi \hbar^2 N \nb}{m g} },
	}
	\al{
		(d = 3) \quad
		E &= \frac{g}{2m} \frac{4\pi V}{(2\pi \hbar)^3} \intopo p^4 \ddp
		= \frac{g}{2m} \frac{4\pi V}{(2\pi \hbar)^3} \brac{ \frac{p^5}{5} }\opo
		= \frac{g}{10m} \frac{4\pi V}{(2\pi \hbar)^3} \brac{ 2\pi \hbar \paren{ \frac{3 \nb}{4\pi g} }^{1/3} }^5 \\
		&= \frac{4\pi (2\pi \hbar)^2 g V}{10 m} \paren{ \frac{3 \nb}{4\pi g} }^{5/3}
		= \ans{ \frac{3 \hbar^2}{10 m} \paren{ \frac{6 \pi^2 \nb}{g} }^{2/3}, }
	}
	where we have used Eq.~\refeq{p0}.
}

%
%	3.2
%

\prob{}{
	Then compute the leading terms of the small temperature corrections to the basic thermodynamic quantities: thermodynamic potential, free energy, energy, pressure, entropy, and specific heat.
}

\sol{
	The thermodynamic potential for a Fermi gas is~\cite[p.~145]{Landau}
	\eq{
		\Omg = -T \sumk \ln(1 + e^{(\mu - \epsk) / T}),
	}
	where $\mu$ is the chemical potential of the gas.  We may replace the sum by an integral over $p \in (0, \infty)$ using Eq.~\refeq{Nstates}, transform variables to $\eps$, and integrate by parts~\cite[pp.~148--149]{Landau}.  Note that
	\eqn{transform}{
		\eps = \frac{p^2}{2m}
		\qimplies
		2 m \ddeps = 2 p \ddp
		\qimplies
		\ddp = \frac{m}{p} \ddeps
		= \frac{m}{\sqrt{2m \eps}} \ddeps
		= \sqrt{\frac{m}{2\eps}} \ddeps.
	}
	Then in each case, we find
	\al{
		(d = 1) \quad
		\Omg &= -g T \frac{L}{2\pi \hbar} \intoi \ln(1 + e^{(\mu - \eps) / T}) \ddp
		= -g T \sqrt{\frac{m}{2}} \frac{L}{2\pi \hbar} \intoi \frac{1}{\sqrt{\eps}} \ln(1 + e^{(\mu - \eps) / T}) \ddeps \\
		&= -g T \sqrt{\frac{m}{2}} \frac{L}{2\pi \hbar} \paren{ \brac{ 2 \sqrt{\eps} \ln(1 + e^{(\mu - \eps) / T}) }\oi + \frac{2}{T} \intoi \frac{\sqrt{\eps}}{1 + e^{(\eps - \mu) / T}} \ddeps } \\
		&= -g \sqrt{2 m} \frac{L}{2\pi \hbar} \intoi \frac{\sqrt{\eps}}{1 + e^{(\eps - \mu) / T}} \ddeps, \\[2ex]
		%
		(d = 2) \quad
		\Omg &= -g T \frac{2\pi A}{(2\pi \hbar)^2} \intoi p \ln(1 + e^{(\mu - \eps) / T}) \ddp
%		= -g T \sqrt{\frac{m}{2}} \frac{2\pi A}{(2\pi \hbar)^2} \intoi \sqrt{\frac{2m \eps}{\eps}} \ln(1 + e^{(\mu - \eps) / T}) \ddeps
		= -g T m \frac{2\pi A}{(2\pi \hbar)^2} \intoi \ln(1 + e^{(\mu - \eps) / T}) \ddeps \\
		&= -g T m \frac{2\pi A}{(2\pi \hbar)^2} \paren{ \brac{ \eps \ln(1 + e^{(\mu - \eps) / T}) }\oi + \frac{1}{T} \intoi \frac{\eps}{1 + e^{(\eps - \mu) / T}} \ddeps } \\
		&= -g m \frac{2\pi A}{(2\pi \hbar)^2} \intoi \frac{\eps}{1 + e^{(\eps - \mu) / T}} \ddeps, \\[2ex]
		(d = 3) \quad
		\Omg &= -g T \frac{4\pi V}{(2\pi \hbar)^3} \intoi p^2 \ln(1 + e^{(\mu - \eps) / T}) \ddp
%		= -g T \sqrt{\frac{m}{2}} \frac{4\pi V}{(2\pi \hbar)^3} \intoi 2m \frac{\eps}{\sqrt{\eps}} \ln(1 + e^{(\mu - \eps) / T}) \ddeps
		= -g T \sqrt{2 m^3} \frac{4\pi V}{(2\pi \hbar)^3} \intoi \sqrt{\eps} \ln(1 + e^{(\mu - \eps) / T}) \ddeps \\
		&= -g T \sqrt{2 m^3} \frac{4\pi V}{(2\pi \hbar)^3} \paren{ \brac{ \frac{2}{3} \eps^{3/2} \ln(1 + e^{(\mu - \eps) / T}) }\oi + \frac{2}{3 T} \intoi \frac{\eps^{3/2}}{1 + e^{(\eps - \mu) / T}} \ddeps } \\
		&= -g \sqrt{2 m^3} \frac{8\pi V}{3 (2\pi \hbar)^3} \intoi \frac{\eps^{3/2}}{1 + e^{(\eps - \mu) / T}} \ddeps,
	}
	where we have used
	\eq{
		\dv{\eps}(\ln(1 + e^{(\mu - \eps) / T})) = -\frac{1}{T} \frac{e^{(\mu - \eps) / T}}{1 + e^{(\mu - \eps) / T}}
		= -\frac{1}{T} \frac{1}{1 + e^{(\eps - \mu) / T}}.
	}
	
	All three expressions have integrals of the form
	\eq{
		I = \intoi \frac{f(\eps)}{1 + e^{(\eps - \mu) / T}} \ddeps
		= T \int_{-\mu / T}^\infty \frac{f(\mu + T z)}{1 + e^z} \ddz,
	}
	where we have made the substitution $\eps - \mu = T z$.  The first two terms of the Taylor series for this integral are~\cite[p.~155]{Landau}
	\eq{
		I \approx \intom f(\eps) \ddeps + \frac{\pi^2 T^2}{6} f'(\mu).
	}
	Thus, the leading term of the correction is given by the second term.
	
	Let $\Omgo$ be the thermodynamic potential at $T = 0$.  Then the leading corrections are given by
	\al{
		(d = 1) \quad
		\Omg &= \Omgo - g \sqrt{2 m} \frac{L}{2\pi \hbar} \frac{\pi^2 T^2}{6} \pdv{\mu}(\sqrt{\mu})
%		= \Omgo - g \sqrt{2 m} \frac{N}{2\pi \hbar} \frac{\pi^2 T^2}{12} \frac{1}{\nb \sqrt{\mu}}
		= \Omgo - \frac{\pi^2}{12} \sqrt{\frac{2 m}{\mu}} \frac{g N T^2}{(2\pi \hbar) \nb}
		= \ans{ \Omgo - \frac{\pi g N T^2}{6 \hbar \nb} \sqrt{\frac{2 m}{\mu}}, } \\[2ex]
		%
		(d = 2) \quad
		\Omg &= \Omgo - g m \frac{2\pi A}{(2\pi \hbar)^2} \frac{\pi^2 T^2}{6} \pdv{\mu}{\mu}
		= \Omgo - \frac{\pi^3}{3} \frac{m g N T^2}{(2\pi \hbar)^2 \nb}
		= \ans{ \Omgo - \frac{\pi m g N T^2}{12 \hbar^2 \nb}, } \\[2ex]
		%
		(d = 3) \quad
		\Omg &= \Omgo - g \sqrt{2 m^3} \frac{8\pi V}{3 (2\pi \hbar)^3} \frac{\pi^2 T^2}{6} \pdv{\mu}(\mu^{3/2})
		= \Omgo - g \sqrt{2 m^3 \mu} \frac{2\pi^3 N T^2}{3 (2\pi \hbar)^3 \nb}
		= \ans{ \Omgo - \frac{g N T^2}{12 \hbar^3 \nb} \sqrt{2 m^3 \mu}. }
	}
	
	For the free energy, we will use the relation $(\del F)_{T, V, N} = (\del \Omg)_{T, V, \mu}$~\cite[pp.~69, 156]{Landau}.  In order to express the correction to $\Omg$ in terms of $T$, $V$, and $N$ only, we will make the approximation $\mu = \epso$, which is exact at $T = 0$~\cite[p.~153]{Landau}.  Applying Eq.~\refeq{eps0} and letting $\Fo$ denote the free energy at $T = 0$, we have
	\al{
		(d = 1) \quad
		F &= \Fo - \frac{\pi g N T^2}{6 \hbar \nb} \sqrt{2 m^3 \frac{m g^2}{2 \pi^2 \hbar^2 \nb^2}}
		= \Fo - \frac{\pi g N T^2}{6 \hbar \nb} \frac{m^2 g}{\pi \hbar \nb}
		= \ans{ \Fo - \frac{m^2 g^2 N T^2}{6 \pi \hbar^2 \nb^2}, } \\[2ex]
		%
		(d = 2) \quad
		F &= \ans{ \Fo - \frac{\pi m g N T^2}{12 \hbar^2 \nb}, } \\[2ex]
		%
		(d = 3) \quad
		F &= \Fo - \frac{g N T^2}{12 \hbar^3 \nb} \sqrt{2 m^3 \frac{\hbar^2}{2m} \paren{ \frac{6 \pi^2 \nb}{g} }^{2/3}}
		= \Fo - \frac{g N T^2}{12 \hbar^3 \nb} m \hbar \paren{ \frac{6 \pi^2 \nb}{g} }^{1/3}
		= \ans{ \Fo - \frac{m N T^2}{2 \hbar^2} \paren{ \frac{\pi g}{6 \nb} }^{2/3}. }
	}
	
	Energy may be calculated from free energy by $E = -T^2 (\pdv*{(F/T)}{T})_{V}$~\cite[p.~47]{Landau}.  This gives us
	\al{
		(d = 1) \quad
		E &= \Eo + T^2 \pdv{T}(\frac{m^2 g^2 N T}{6 \pi \hbar^2 \nb^2})
		= \ans{ \Eo + \frac{m^2 g^2 N T^2}{6 \pi \hbar^2 \nb^2}, } \\[2ex]
		(d = 2) \quad
		E &= \Eo + T^2 \pdv{T}(\frac{\pi m g N T}{12 \hbar^2 \nb})
		= \ans{ \Eo + \frac{\pi m g N T^2}{12 \hbar^2 \nb}, } \\[2ex]
		%
		(d = 3) \quad
		E &= \Eo + T^2 \pdv{T}(\frac{m N T}{2 \hbar^2} \paren{ \frac{\pi g}{6 \nb} }^{2/3})
		= \ans{ \Eo + \frac{m N T^2}{2 \hbar^2} \paren{ \frac{\pi g}{6 \nb} }^{2/3}, }
	}
	where $\Eo$ is the energy at $T = 0$.
	
	The pressure may be found by the definition of the thermodynamic potential, $\Omg = -P V$~\cite[p.~69]{Landau}.  Again using $\mu = \epso$ and letting $\Po$ be the pressure at $T = 0$, we have
	\al{
		(d = 1) \quad
		P &= \Po + \frac{1}{V} \frac{\pi g N T^2}{6 \hbar \nb} \sqrt{2 m^3 \frac{m g^2}{2 \pi^2 \hbar^2 \nb^2}}
		= \ans{ \Po + \frac{\pi g N T^2}{6 \hbar \nb} \sqrt{2 m^3 \frac{m g^2}{2 \pi^2 \hbar^2 \nb}}, } \\[2ex]
		%
		(d = 2) \quad
		\Omg &= \Po + \frac{1}{V} \frac{\pi m g N T^2}{12 \hbar^2 \nb}
		= \ans{ \Po + \frac{\pi m g T^2}{12 \hbar^2}, }
	}
	\al{
		(d = 3) \quad
		\Omg &= \Po + \frac{1}{V} \frac{m N T^2}{2 \hbar^2} \paren{ \frac{\pi g}{6 \nb} }^{2/3}
		= \ans{ \Po + \frac{m T^2}{2 \hbar^2} \nb^{1/3} \paren{ \frac{\pi g}{6} }^{2/3}. }
	}
	
	Entropy may be calculated from free energy by $S = -(\pdv*{F}{T})_{V}$~\cite[p.~46]{Landau}.  The entropy is zero at $T = 0$ for any system due to Nernst's theorem~\cite[p.~66]{Landau}.  Then the leading-order corrections to the entropy are
	\al{
		(d = 1) \quad
		S &= \pdv{T}(\frac{m^2 g^2 N T^2}{6 \pi \hbar^2 \nb^2})
		= \ans{ \frac{m^2 g^2 N T}{3 \pi \hbar^2 \nb^2}, } \\[2ex]
		%
		(d = 2) \quad
		S &= \pdv{T}(\frac{\pi m g N T^2}{12 \hbar^2 \nb})
		= \ans{ \frac{\pi m g N T}{6 \hbar^2 \nb}, } \\[2ex]
		%
		(d = 3) \quad
		S &= \pdv{T}(\frac{m N T^2}{2 \hbar^2} \paren{ \frac{\pi g}{6 \nb} }^{2/3})
		= \ans{ \frac{m N T}{\hbar^2} \paren{ \frac{\pi g}{6 \nb} }^{2/3}. }
	}
	
	Another consequence of Nernst's theorem is that $\Cp = \Cv$ for $T \to 0$, so we can find the specific heat $C$ by $\Cv = T (\pdv*{S}{T})_V$~\cite[pp.~45, 66]{Landau}.  So we have
	\al{
		(d = 1) \quad
		C &= T \pdv{T}(\frac{m^2 g^2 N T}{3 \pi \hbar^2 \nb^2})
		= \ans{ \frac{m^2 g^2 N T}{3 \pi \hbar^2 \nb^2}, } \\[2ex]
		%
		(d = 2) \quad
		C &= T \pdv{T}(\frac{\pi m g N T}{6 \hbar^2 \nb})
		= \ans{ \frac{\pi m g N T}{6 \hbar^2 \nb}, } \\[2ex]
		%
		(d = 3) \quad
		C &= T \pdv{T}(\frac{m N T}{\hbar^2} \paren{ \frac{\pi g}{6 \nb} }^{2/3})
		= \ans{ \frac{m N T}{\hbar^2} \paren{ \frac{\pi g}{6 \nb} }^{2/3}. }
	}
	\vfix
}