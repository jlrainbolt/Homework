\newcommand{\qd}{\dot{q}}
\newcommand{\pd}{\dot{p}}
\newcommand{\qs}{q^*}
\newcommand{\ps}{p^*}

\newcommand{\Vq}{V(q)}
\newcommand{\fq}{f(q)}
\newcommand{\gq}{g(q)}

\section{Problem 1}
\begin{statement}
	The motion of a particle in a cubic potential is governed by the Hamiltonian
	\beqn \label{ham1}
		H(q, p) = \frac{p^2}{2m} + \frac{k^2}{2} q^2 - \frac{A}{3} q^3.
	\eeqn
	Here $m$ is the particle mass, $k$ is the spring constant, and $A$ is a positive dimensional constant.
\end{statement}

\begin{problem}
	Sketch the potential and the contours of $H$.  Identify any fixed points (mechanical equilibrium states) that exist.  Classify them as stable (elliptic) or unstable (hyperbolic).
\end{problem}

\begin{solution}
	Define the potential of \refeq{ham1} as
	\beqn \label{pot1}
		\Vq \equiv \frac{k^2}{2} q^2 - \frac{A}{3} q^3 \equiv \gq + \gq,
	\eeqn
	where we have defined $\fq = k^2 q^2 / 2$ and $\gq = -A q^3 / 3$.  Figures~\ref{f} and \ref{g} and show sketches of $\fq$ and $\gq$, respectively.  Their sum $\Vq$ may be obtained by summing them graphically, and is shown in figure~\ref{V}.
	
	Fixed points are located where $\dv*{V}{q} |_{\qs} = 0$.  They are stable where $\Vq$ has a local minimum ($\dv*[2]{V}{q} |_{\qs} > 0$) and unstable where $\Vq$ has a local maximum ($\dv*[2]{V}{q} |_{\qs} < 0$).  There are two fixed points, indicated by circles in figure~\ref{V}.  The stable~(unstable) fixed point is indicated by a closed~(open) circle.

	Hamilton's equations for \refeq{ham1} are given by
	\begin{align}
		\qd &= \pdv{H}{p} = \frac{p}{m} \implies p = m \qd, \label{x1} \\
		\pd &= -\pdv{H}{q} = k^2 q - A q^2. \notag
	\end{align}
	Fixed points occur where $\qd = \pd = 0$; that is, the solutions of the equation
	\beq
		\ps = k^2 \qs - A {\qs}^2.
	\eeq
	From \refeq{x1}, $\qd = 0 \implies \pd = 0$.  Thus, the stable fixed point is located at $(\qs, \ps) = 0$, and the unstable fixed point is located at $(\qs, \ps) = (k^2 / A, 0)$.
	
	Contours are curves in the phase plane for which $H$ is constant.  Several contours are shown in figure~\ref{contours}.
\end{solution}

\unitlength=.2in
\begin{figure}[p] \centering
	\begin{picture}(10.5,10.5)(-5,-5)
		{\color{lightgray}
		\thinlines
		\multiput(-8,-4)(0,1){9}{\line(1,0){16}}
		\multiput(-7,-5)(1,0){15}{\line(0,1){10}}
		}
		\thicklines
		\put(-8,0){\vector(1,0){16.2}}
		\put(0,-5){\vector(0,1){10.2}}
		\put(8.3,0){\makebox(1,0)[l]{$q$}}
		\put(0,5.3){\makebox(0,1)[c]{$\fq$}}
	\end{picture}
	\caption{Sketch of $\fq$ as defined in \refeq{pot1}.}
	\label{f}
\end{figure}

\unitlength=.2in
\begin{figure} \centering
	\begin{picture}(10.5,10.5)(-5,-5)
		{\color{lightgray}
		\thinlines
		\multiput(-8,-4)(0,1){9}{\line(1,0){16}}
		\multiput(-7,-5)(1,0){15}{\line(0,1){10}}
		}
		\thicklines
		\put(-8,0){\vector(1,0){16.2}}
		\put(0,-5){\vector(0,1){10.2}}
		\put(8.3,0){\makebox(1,0)[l]{$q$}}
		\put(0,5.3){\makebox(0,1)[c]{$\gq$}}
	\end{picture}
	\caption{Sketch of $\gq$ as defined in \refeq{pot1}.}
	\label{g}
\end{figure}

\unitlength=.2in
\begin{figure} \centering
	\begin{picture}(10.5,10.5)(-5,-5)
		{\color{lightgray}
		\thinlines
		\multiput(-8,-4)(0,1){9}{\line(1,0){16}}
		\multiput(-7,-5)(1,0){15}{\line(0,1){10}}
		}
		\thicklines
		\put(-8,0){\vector(1,0){16.2}}
		\put(0,-5){\vector(0,1){10.2}}
		\put(8.3,0){\makebox(1,0)[l]{$q$}}
		\put(0,5.3){\makebox(0,1)[c]{$\Vq$}}
	\end{picture}
	\caption{Sketch of $\Vq$ obtained by summing $\fq$ and $\gq$.  The stable~(unstable) fixed point is represented by a closed~(open) circle.}
	\label{V}
\end{figure}

\unitlength=.3in
\begin{figure}[p] \centering
	\begin{picture}(10.5,10.5)(-5,-5)
		{\color{lightgray}
		\thinlines
		\multiput(-8,-4)(0,1){9}{\line(1,0){16}}
		\multiput(-7,-5)(1,0){15}{\line(0,1){10}}
		}
		\thicklines
		\put(-8,0){\vector(1,0){16.2}}
		\put(0,-5){\vector(0,1){10.2}}
		\put(8.3,0){\makebox(1,0)[l]{$q$}}
		\put(0,5.3){\makebox(0,1)[c]{$p$}}
	\end{picture}
	\caption{Contours of $H$.  The stable~(unstable) fixed point is represented by a closed~(open) circle.}
	\label{contours}
\end{figure}

\unitlength=.3in
\begin{figure} \centering
	\begin{picture}(10.5,10.5)(-5,-5)
		{\color{lightgray}
		\thinlines
		\multiput(-8,-4)(0,1){9}{\line(1,0){16}}
		\multiput(-7,-5)(1,0){15}{\line(0,1){10}}
		}
		\thicklines
		\put(-8,0){\vector(1,0){16.2}}
		\put(0,-5){\vector(0,1){10.2}}
		\put(8.3,0){\makebox(1,0)[l]{$q$}}
		\put(0,5.3){\makebox(0,1)[c]{$p$}}
	\end{picture}
	\caption{Trajectories of $H$, with the direction of time evolution indicated by arrows.  The stable~(unstable) fixed point is represented by a closed~(open) circle.  The separatrix is drawn in red.}
	\label{trajectories}
\end{figure}

\begin{problem}
	Sketch qualitatively both representative and interesting trajectories in the phase space.  If there is a separatrix, a trajectory separating qualitatively different types of motion, specify the equation governing its shape.
\end{problem}

\begin{solution}
	Trajectories lie along contours of $H$.  The directions of the trajectories may be deduced by \refeq{x1}, which indicates that time evolution flows in the $+q$~($-q$) direction when $p > 0$~($< 0$).  This corresponds to the top~(bottom) half of the phase plane.  Representative trajectories corresponding to some of the contours in figure~\ref{contours} are shown in figure~\ref{trajectories}.
	
	There is a separatrix in figure~\ref{trajectories}, shown in red.  The separatrix passes through the unstable fixed point at $(\qs, \ps) = (k^2 / A, 0)$.  Feeding these values into \refeq{ham1}, we obtain
	\beq
		E \equiv \frac{k^2}{2} \left(\frac{k^2}{A}\right)^2 - \frac{A}{3} \left(\frac{k^2}{A}\right)^3 = \frac{1}{6} \frac{k^6}{A^2}
	\eeq
	as the constant energy of the separatrix.  Substituting once more into \refeq{ham1} yields
	\beq
		\frac{1}{6} \frac{k^6}{A^2} = \frac{p^2}{2m} + \frac{k^2}{2} q^2 - \frac{A}{3} q^3 \iff p^3 = m \left( \frac{1}{3} \frac{k^6}{A^2} - k^2 q^2 + \frac{2}{3} A q^3 \right)
	\eeq
	as the equation governing the shape of the separatrix.
\end{solution}