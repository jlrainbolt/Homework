\documentclass[11pt]{article}
\usepackage{geometry, titlesec}
\usepackage[parfill]{parskip}
\usepackage{physics, amsfonts, amsthm}
\usepackage{fullpage}
\usepackage{fancyhdr}
\usepackage{enumitem}
\usepackage{xcolor, soul}
%\allowdisplaybreaks

\makeatletter
\renewcommand*\env@cases[1][1.2]{%
  \let\@ifnextchar\new@ifnextchar
  \left\lbrace
  \def\arraystretch{#1}%
  \array{@{}l@{\quad}l@{}}%
}
\makeatother
 
 
\setlist[enumerate]{leftmargin=*}
\pagestyle{fancy}
\fancyhf{}
\lhead{\textbf{Physics 341 Homework 1}}
\rhead{Lacey Rainbolt}
\setlength{\headheight}{14pt}
\setlength{\headsep}{12pt}

\titleformat{\subsection}[runin]{\normalfont\large\bfseries}{\thesubsection}{1em}{}
\newcommand{\refeq}[1]{(\ref{#1})}


\newenvironment{statement}
{
    \color{darkgray}
    \ignorespaces
}
{
    \bigskip
}

\newenvironment{problem}
{
    \color{darkgray}
    \subsection{}
    \ignorespaces
}


\newenvironment{solution}
{
    \paragraph{Solution.}
    \ignorespaces
}
{
    \bigskip
}



\begin{document}

\newcommand{\keq}{\ket{e_1}}
\newcommand{\beq}{\bra{e_1}}
\newcommand{\kew}{\ket{e_2}}
\newcommand{\bew}{\bra{e_2}}
\newcommand{\kee}{\ket{e_3}}
\newcommand{\bee}{\bra{e_3}}

\section{Problem 1}
\begin{statement}
	Consider operators $J$ and $K$ acting in a three-dimensional space as
	\begin{align}
		J \keq &= i \kew, & J \kew &= -i \keq, & J \kee &= 0, \\
		K \keq &= 0, & K \kew &= i \kee, & K \kee &= -i \kew,
	\end{align}
	where $\keq, \kew, \kee$ form a complete orthonormal basis.
\end{statement}

\begin{problem}
	Compute the matrix elements of $J$ and $K$.
\end{problem}
	
\begin{solution} The matrix elements of $J$ are
	\begin{align}
		J_{11} &= \mel{e_1}{J}{e_1} = i \braket{e_1}{e_2} = 0, \label{J11} &
		J_{12} &= \mel{e_1}{J}{e_2} = -i \braket{e_1}{e_1} = -i, &
		J_{13} &= \mel{e_1}{J}{e_3} = 0, \\
		J_{21} &= \mel{e_2}{J}{e_1} = i \braket{e_2}{e_2} = i, &
		J_{22} &= \mel{e_2}{J}{e_2} = -i \braket{e_2}{e_1} = 0, &
		J_{23} &= \mel{e_2}{J}{e_3} = 0, \\
		J_{31} &= \mel{e_3}{J}{e_1} = i \braket{e_3}{e_2} = 0, &
		J_{32} &= \mel{e_3}{J}{e_2} = -i \braket{e_3}{e_1} = 0, &
		J_{33} &= \mel{e_3}{J}{e_3} = 0.
	\end{align}
	The matrix elements of $K$ are
	\begin{align}
		K_{11} &= \mel{e_1}{K}{e_1} = 0, &
		K_{12} &= \mel{e_1}{K}{e_2} = i \braket{e_1}{e_3} = 0, &
		K_{13} &= \mel{e_1}{K}{e_3} = -i \braket{e_1}{e_2} = 0, \\
		K_{21} &= \mel{e_2}{K}{e_1} = 0, &
		K_{22} &= \mel{e_2}{K}{e_2} = i \braket{e_2}{e_3} = 0, &
		K_{23} &= \mel{e_2}{K}{e_3} = -i \braket{e_2}{e_2} = -i,\!\\
		K_{31} &= \mel{e_3}{K}{e_1} = 0, &
		K_{32} &= \mel{e_3}{K}{e_2} = i \braket{e_3}{e_3} = i, &
		K_{33} &= \mel{e_3}{K}{e_3} = -i \braket{e_3}{e_2} = 0. \label{K33}
	\end{align}
\end{solution}

\begin{problem}
	Consider $O = AJ + BK$ where $A, B$ are real numbers.  Show that $O$ is Hermitian.
\end{problem}

\begin{solution}
	Using \refeq{J11}--\refeq{K33}, the matrix elements of $O$ are
	\begin{align}
		O_{11} &= O_{13} = O_{22} = O_{31} = O_{33} = 0, \label{O11} \\
		O_{12} &= -iA, \\
		O_{21} &= iA, \\
		O_{23} &= -iB, \\
		O_{32} &= iB. \label{O32}
	\end{align}
	$O$ is Hermitian if and only if $O_{ij} = O_{ji}^*$ for all $O_{ij}$.  Recall that $(z^*)^* = z$ for any $z \in \mathbb{C}$.  From \refeq{O11}--\refeq{O32}, note that
	\begin{align}
		O_{11} &= 0 = O_{11}^*, \\
		O_{12} &= -iA = (iA)^* = O_{21}^*, \\
		O_{13} &= 0 = O_{31}^*, \\
		O_{22} &= 0 = O_{22}^*, \\
		O_{23} &= -iB = (iB)^* = O_{32}^*, \\
		O_{33} &= 0 = O_{33}^*,
	\end{align}
	so $O$ is indeed Hermitian. \qed
\end{solution}

\newcommand{\kpl}{\ket{p_\lambda}}
\newcommand{\lp}{\lambda_+}
\newcommand{\lo}{\lambda_0}
\newcommand{\lm}{\lambda_-}
\newcommand{\lpm}{\lambda_\pm}
\newcommand{\kpp}{\ket{p_+}}
\newcommand{\kpo}{\ket{p_0}}
\newcommand{\kpm}{\ket{p_-}}
\newcommand{\kppm}{\ket{p_\pm}}

\newcommand{\klo}{\ket{\lo}}
\newcommand{\klp}{\ket{\lp}}
\newcommand{\klm}{\ket{\lm}}
\newcommand{\klpm}{\ket{\lpm}}
\newcommand{\uoq}{\lambda_{0,1}}
\newcommand{\uow}{\lambda_{0,2}}
\newcommand{\uoe}{\lambda_{0,3}}
\newcommand{\mO}{\mqty[0 & -iA & 0 \\ iA & 0 & -iB \\ 0 & iB & 0]}

\newcommand{\upmq}{\lambda_{\pm,1}}
\newcommand{\upmw}{\lambda_{\pm,2}}
\newcommand{\upme}{\lambda_{\pm,3}}
\newcommand{\sqab}{\sqrt{A^2 + B^2}}
\newcommand{\pmsq}{\pm \sqab}
\newcommand{\pmisq}{\pm i \sqab}
\newcommand{\mpisq}{\mp i \sqab}
\newcommand{\isq}{i \sqab}

\begin{problem}
	If $\kpl$ is an eignevector of $O$, we have $O \kpl = \lambda \kpl$ where $\lambda$ is the corresponding eigenvalue. $\kpl$ can be expanded as $\kpl = \sum{i=1}^3 u_{\lambda, i} \ket{e_i}$.  Denote the three eigenvalues and the corresponding normalized eigenvectors of $O$ as $\lp, \lo, \lm$ and $\kpp, \kpo, \kpm$ where $\lp$~($\lm$) is the largest~(smallest) eigenvalue.  Find $\lp, \lo, \lm$ and $\kpp, \kpo, \kpm$.
\end{problem}

\begin{solution}
	Using a matrix representation in the $\keq, \kew, \kee$ basis, we can write
	\begin{equation}
		O = \mO.
	\end{equation}
	$\lambda$ is an eigenvalue of $O$ if $\det(O - \lambda I) = 0$, where $I$ is the identity matrix.  That is,
	\begin{align}
		0 &= \mdet{-\lambda & -iA & 0 \\ iA & -\lambda & -iB \\ 0 & iB & -\lambda} \\
		&= (-\lambda)^3 - (-\lambda)(-iB)(iB) - (-iA)(iA)(-\lambda) \\
		&= \lambda(\lambda^2 - A^2 - B^2) \label{lambda0} \\
		&= \lambda^2 - A^2 - B^2 \label{lambdapm}.
	\end{align}
	From \refeq{lambda0} we obtain $\lo = 0$, and from \refeq{lambdapm} we obtain $\lpm = \pm \sqab$.

	Let $\klo, \klpm$ be the not-necessarily-normalized eigenvectors corresponding to $\lo, \lpm$.  Beginning with $\lo$, we will find the corresponding eigenvector $\klo = \uoq \keq + \uow \kew + \uoe \kee$.  We seek $\uoq, \uow, \uoe$ such that
	\begin{equation} \label{ev0}
		\mO \mqty[\uoq \\ \uow \\ \uoe] = 0 \mqty[\uoq \\ \uow \\ \uoe].
	\end{equation}
	The algebraic equations corresponding to \refeq{ev0} are
	\begin{align}
		-iA \,\uow &= 0, \label{u02a} \\
		iA \,\uoq - iB \,\uoe &= 0, \label{u01} \\
		iB \,\uow &= 0. \label{u02b}
	\end{align}
	\refeq{u02a} and \refeq{u02b} imply that $\uow = 0$.  We may fix $\uoe = A$ without loss of generality.  Then \refeq{u01} implies $\uoq = B$.  Thus, $\klo = B \keq + A \kee$.

	For $\klpm = \upmq \keq + \upmw \kew + \upme \kee$, we seek $\upmq, \upmw, \upme$ such that
	\begin{equation} \label{evpm}
		\mO \mqty[\upmq \\ \upmw \\ \upme] = \pmsq \mqty[\upmq \\ \upmw \\ \upme].
	\end{equation}
	The algrabraic equations corresponding to \refeq{evpm} are
	\begin{align}
		-iA \,\upmw &= \pmsq \,\upmq, \label{upm1} \\
		iA \,\upmq - iB \,\upme &= \pmsq \,\upmw, \label{upm2} \\
		iB \,\upmw &= \pmsq \,\upme. \label{upm3}
	\end{align}
	Summing \refeq{upm1}, \refeq{upm2}, and \refeq{upm3}, we have
	\begin{align}
		\pmsq (\upmq + \upmw + \upme) &= iA (\upmq - \upmw) + iB (\upmw - \upme) \\
		\pmisq (\upmq + \upmw + \upme) &= A (\upmw - \upmq) - B (\upmw - \upme). \label{pmform}
	\end{align}
	From the form of \refeq{pmform}, we make the ansatz $\upmq = -A, \upme = B$.  Making the relevant substituions in \refeq{upm1} and \refeq{upm3}, we have
	\begin{align}
		-iA \,\upmw &= \pm A \sqab, \\ 
		iB \,\upmw &= \pm B \sqab
	\end{align}
	which both imply $\upmw = \mpisq$.  Therefore, $\klpm = -A \keq \mpisq \kew + B \kee$.
	
	Now we will compute $\kpp, \kpo, \kpm$ by normalizing $\klo, \klpm$.  Note that
	\begin{align}
		\norm{\lo}^2 &= \braket{\lo}{\lo} = A^2 + B^2, \\
		\norm{\lpm}^2 &= \braket{\lpm}{\lpm} = A^2 + (A^2 + B^2) + B^2 = 2 A^2 + 2 B^2,
	\end{align}
	so
	\begin{align}
		\kpp &= \frac{\klp}{\norm{\lp}} = \frac{-A \keq - \isq \kew + B \kee}{\sqrt{2} \sqab}, \\
		\kpo &= \frac{\klo}{\norm{\lo}} = \frac{B \keq + A \kee}{\sqab}, \label{p0} \\
		\kpm &= \frac{\klm}{\norm{\lm}} = \frac{-A \keq + \isq \kew + B \kee}{\sqrt{2} \sqab}.
	\end{align}
\end{solution}

\newcommand{\bpo}{\bra{p_0}}
\newcommand{\kepq}{\ket{e_1'}}
\newcommand{\khq}{\ket{h_1}}
\newcommand{\ab}{A^2 + B^2}

\begin{problem}
	Define a new state $\kepq$ by $\kepq= \khq / \norm{h_1}$ where $\norm{h_1} = \sqrt{\braket{h_1}{h_1}}$ and $\khq = (1 - \kpo \bpo) \keq$.  Find the probability that the state $\kepq$ is found to have the eigenvalue $\lp, \lo, \lm$.
\end{problem}

\begin{solution}
	First, we can find an $\kepq$ using the result \refeq{p0} for $\kpo$.  Beginning with $\khq$, we have
	\begin{align}
		\khq &= \keq - \braket{p_0}{e_1} \kpo = \keq - \frac{B}{\sqab} \kpo = \keq - \frac{B}{\sqab} \left(\frac{B \keq + A \kee}{\sqab}\right) \\
		&= \left(1 - \frac{B^2}{\ab}\right) \keq - \frac{AB}{\ab} \kee.
	\end{align}
	Then
	\begin{align}
		\norm{h_1}^2 &= \left(1 - \frac{B^2}{\ab}\right)^2 - \left(\frac{AB}{\ab}\right)^2 = 1 - \frac{2B^2}{\ab} + \frac{B^4}{(\ab)^2} - \frac{A^2 B^2}{(\ab)^2} \\
		&= \frac{(\ab)^2 - 2B^2(\ab) + B^4 - A^2B^2}{(\ab)^2} = \frac{A^2 (\ab)}{(\ab)^2} \\
		&= \frac{A^2}{A^2 + B^2}
	\end{align}
	so
	\begin{align}
		\kepq&= \frac{\khq}{\norm{h_1}} = \frac{\sqab}{A} \left[\left(1 - \frac{B^2}{\ab}\right) \keq - \frac{AB}{\ab} \kee \right] \\
%		&= \left(\frac{\sqab}{A} - \frac{B^2}{A \sqab}\right) \keq - \frac{B}{\sqab} \kee \\
		&= \frac{A}{\sqab} \keq - \frac{B}{\sqab} \kee
	\end{align}
	The probability that $\kepq$ has the eigenvalue $\lambda$ is $\qty|\braket{p_\lambda}{e_1'}|^2$.  Thus,
	\begin{align}
		\qty|\braket{p_\pm}{e_1'}|^2 &= \left| -\frac{A}{\sqab} \frac{A}{\sqrt{2}\sqab} - \frac{B}{\sqab} \frac{B}{\sqrt{2}\sqab} \right|^2 = \left| -\frac{1}{\sqrt{2}} \right|^2 = \frac{1}{2}, \\
		\qty|\braket{p_0}{e_1'}|^2 &= \left| -\frac{A}{\sqab} \frac{B}{\sqab} + \frac{B}{\sqab} \frac{A}{\sqab} \right|^2 = 0.
	\end{align}
	As expected, $\qty|\braket{p_+}{e_1'}|^2 + \qty|\braket{p_0}{e_1'}|^2 + \qty|\braket{p_0}{e_1'}|^2 = 1$.
\end{solution}


\section{Problem 2}
\begin{statement}
	Consider an operator $A$ acting in a two-dimensional space as
	\begin{equation} \label{A}
		A\keq = i \kew, \quad A\kew = -i \keq,
	\end{equation}
	where $\keq, \kew$ form a complete orthonormal basis.
\end{statement}

\begin{problem}
	Find the matrix elements $A_{ij}$ ($i,j = 1,2$) of $A$ with respect to $\keq, \kew$.
\end{problem}

\begin{solution}
	Using \refeq{A}, the matrix elements of $A$ are
	\begin{align}
		A_{11} &= \mel{e_1}{A}{e_1} = i \braket{e_1}{e_2} = 0, &
		A_{12} &= \mel{e_1}{A}{e_2} = -i \braket{e_1}{e_1} = -i, \\
		A_{21} &= \mel{e_2}{A}{e_1} = i \braket{e_2}{e_2} = i, &\
		A_{22} &= \mel{e_2}{A}{e_2} = -i \braket{e_2}{e_1} = 0.
	\end{align}
\end{solution}

\newcommand{\kepw}{\ket{e_2'}}
\newcommand{\mA}{\mqty[0 & -i \\ i & 0]}
\newcommand{\lpmq}{\lambda_{\pm1}}
\newcommand{\lpmw}{\lambda_{\pm2}}


\begin{problem} \label{2.2}
	The eigenvalues of $A$ are $\pm1$.  Find the corresponding eigenvectors $\kepq, \kepw$ and represent them in terms of $\keq, \kew$.
\end{problem}

\begin{solution}
	Using a matrix representation in the $\keq, \kew$ basis, we can write
	\begin{equation} \label{matA}
		A = \mA.
	\end{equation}
	Let $\klpm$ be the not-necessarily-normalized eigenvector corresponding to the eigenvalue $\pm1$.  We seek $\lpmq, \lpmw$ such that
	\begin{equation} \label{evpm}
		\mA \mqty[\lpmq \\ \lpmw] = \pm \mqty[\lpmq \\ \lpmw].
	\end{equation}
	The algebraic equations corresponding to \refeq{evpm} are
	\begin{equation} \label{lpm}
		-i \,\lpmw = \pm \,\lpmq, \quad\quad\quad i \,\lpmq = \pm \,\lpmw.
	\end{equation}
	By inspection of \refeq{lpm}, $\lpmq = \mp i$ and $\lpmw = 1$.  Thus $\klpm = \mp i \keq + \kew$.
	
	Let $\kepq$~($\kepw$) be the normalized eigenvector corresponding to eigenvalue 1~($-1$).  Then
	\begin{equation} \label{eprime}
		\kepq = \frac{\klp}{\norm{\lp}} = \frac{-i \keq + \kew}{\sqrt{2}}, \quad\quad\quad
		\kepw = \frac{\klm}{\norm{\lm}} = \frac{i \keq + \kew}{\sqrt{2}}.
	\end{equation}
\end{solution}

\begin{problem}
	Let $U$ be the unitary operator such that $\ket{e_i'} = U \ket{e_i}$.  Find the matrix elements $U_{ij}$ of $U$ with respect to $\keq, \kew$.
\end{problem}

\newcommand{\Ud}{U^\dagger}

\begin{solution}
	Using $\refeq{eprime}$, the matrix elements of $U$ are
	\begin{align}
		U_{11} &= \mel{e_1}{U}{e_1} = \braket{e_1}{e_1'} = -\frac{i}{\sqrt{2}}, &
		U_{12} &= \mel{e_1}{U}{e_2} = \braket{e_1}{e_2'} = \frac{i}{\sqrt{2}}, \label{melua} \\
		U_{21} &= \mel{e_2}{U}{e_1} = \braket{e_2}{e_1'} = \frac{1}{\sqrt{2}}, &\
		U_{22} &= \mel{e_2}{U}{e_2} = \braket{e_2}{e_2'} = \frac{1}{\sqrt{2}}. \label{melub}
	\end{align}
	$U$ is a unitary operator if and only if $U \Ud = \Ud U = I$ where $I$ is the identity matrix.  Using a matrix representation in the $\keq, \kew$ basis, we have
	\begin{equation} \label{matrep}
		U = \frac{1}{\sqrt{2}} \mqty[-i & i \\ 1 & 1], \quad\quad\quad \Ud = \frac{1}{\sqrt{2}} \mqty[i & 1 \\ -i & 1]
	\end{equation}
	so
	\begin{align}
		U \Ud &= \frac{1}{2} \mqty[-i & i \\ 1 & 1] \mqty[i & 1 \\ -i & 1] = \frac{1}{2} \mqty[2 & 0 \\ 0 & 2] = \mqty[1 & 0 \\ 0 & 1], \\
		\Ud U &= \frac{1}{2} \mqty[i & 1 \\ -i & 1] \mqty[-i & i \\ 1 & 1] = \frac{1}{2} \mqty[2 & 0 \\ 0 & 2] = \mqty[1 & 0 \\ 0 & 1]
	\end{align}
	so $U$ is indeed unitary.
\end{solution}

\begin{problem}
	Consider the matrix elements of $A$ in the $\kepq, \kepw$ basis.  Represent $A_{ij}'$ using $A_{ij}$ and $U_{ij}$.  (Numerical values of $A_{ij}'$ are not required.)
\end{problem}

\begin{solution}
	Recall that $\keq, \kew$ form a complete orthonormal basis, so $\sum_{i=1}^2 \ket{e_i}\bra{e_i} = I$.  This allows us to write
	\begin{equation}
		A = \sum_{n=1}^2 \sum_{m=1}^2 \ket{e_n} \mel{e_n}{A}{e_m} \bra{e_m} = \sum_{n=1}^2 \sum_{m=1}^2 \ket{e_n} A_{nm} \bra{e_m}.
	\end{equation}
	Then the matrix elements $A_{ij}'$ are
	\begin{equation} \label{rep1}
		A_{ij}' = \mel*{e_i'}{A}{e_j'} = \sum_{n=1}^2 \sum_{m=1}^2 \braket{e_i'}{e_n} A_{nm} \braket{e_m}{e_j'}.
	\end{equation}
	From \refeq{melua} and \refeq{melub} we know that
	\begin{equation} \label{thing1}
		\braket*{e_m}{e_j'} = \mel{e_m}{U}{e_j} = U_{mj}.
	\end{equation}
	Similarly,
	\begin{equation} \label{thing2}
		\braket*{e_i'}{e_n} = (\braket{e_n}{e_i'})^* = (\mel{e_n}{U}{e_i})^* = U^*_{ni}.
	\end{equation}
	Making the substitutions \refeq{thing1} and \refeq{thing2}, \refeq{rep1} becomes
	\begin{equation}
		A_{ij}' = \sum_{n=1}^2 \sum_{m=1}^2 U_{in}^* A_{nm} U_{mj}.
	\end{equation}
	Explicitly in terms of $i, j$, this is
	\begin{align}
		A_{ij}' = U_{ii}^* A_{ii} U_{ij} + U_{ii}^* A_{ij} U_{jj} + U_{ij}^* A_{ji} U_{ij} + U_{ij}^* A_{jj} U_{jj}.
	\end{align}

\end{solution}

\end{document}