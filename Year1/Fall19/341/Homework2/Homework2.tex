\documentclass[11pt]{article}
\usepackage{geometry, titlesec}
\usepackage[parfill]{parskip}
\usepackage{physics, amsfonts, amsthm}
\usepackage[cm]{fullpage}
\usepackage{fancyhdr}
\usepackage{enumitem}
\usepackage{xcolor, soul}
%\allowdisplaybreaks

\makeatletter
\renewcommand*\env@cases[1][1.2]{%
  \let\@ifnextchar\new@ifnextchar
  \left\lbrace
  \def\arraystretch{#1}%
  \array{@{}l@{\quad}l@{}}%
}
\makeatother
 
 
\renewcommand{\footrulewidth}{.2pt}
%\setlist[enumerate]{leftmargin=*}
\pagestyle{fancy}
\fancyhf{}
\lhead{\textbf{Physics 341 Homework 2}}
\rhead{Lacey Rainbolt}
\setlength{\headheight}{11pt}
\setlength{\headsep}{11pt}
\setlength{\footskip}{24pt}
\lfoot{\today}
\rfoot{\thepage}

\titleformat{\subsection}[runin]{\normalfont\large\bfseries}{\thesubsection}{1em}{}
\newcommand{\refeq}[1]{(\ref{#1})}


\newenvironment{statement}
{
    \color{darkgray}
    \ignorespaces
}
{
%    \smallskip
}

\newenvironment{problem}
{
    \color{darkgray}
    \subsection{}
    \ignorespaces
}


\newenvironment{solution}
{
    \paragraph{Solution.}
    \ignorespaces
}
{
%    \smallskip
}

\newcommand{\Schrodinger}{Schr\"{o}dinger}


\setcounter{enumi}{2}
\begin{document}

\newcommand{\ihb}{i \hbar}
\newcommand{\partt}{\partial_t}
\newcommand{\partx}{\partial_x}

\newcommand{\bkPsix}{\braket{\Psi(t)}{x}}
\newcommand{\bkxPsi}{\braket{x}{\Psi(t)}}
\newcommand{\bPsit}{\bra{\Psi(t)}}
\newcommand{\kPsit}{\ket{\Psi(t)}}
\newcommand{\bPsi}{\bra{\Psi}}
\newcommand{\kPsi}{\ket{\Psi}}
\newcommand{\bx}{\bra{x}}
\newcommand{\kx}{\ket{x}}

\newcommand{\bkPhix}{\braket{\Phi(t)}{x}}
\newcommand{\bPhi}{\bra{\Phi(t)}}
\newcommand{\kPhi}{\ket{\Phi(t)}}

\newcommand{\Ei}{E_i}
\newcommand{\kEi}{\ket{E_i}}
\newcommand{\bEi}{\bra{E_i}}
\newcommand{\bkEiPsi}{\braket{E_i}{\Psi}}
\newcommand{\bkPsiEi}{\braket{\Psi}{E_i}}

\section{Problem 3}
\begin{statement}
	Consider a particle moving in one dimension with the Hamiltonian
	\begin{equation} \label{hamiltonian}
		H = \frac{p^2}{2m} + V(x).
	\end{equation}
\end{statement}

\begin{problem}
	Verify the following:
	
	\renewcommand{\theenumi}{\alph{enumi}}
	\begin{enumerate}
		\item $\ihb \partt \bkPsix = - \mel{\Psi(t)}{H}{x}$,
		\item $\ihb \partt \bkPhix \bkxPsi = \bkPhix \mel{x}{H}{\Psi(t)} - \mel{\Phi(t)}{H}{x} \bkxPsi$,
		\item $\ihb \partt \bkPhix \bkxPsi = -\dfrac{\hbar^2}{2m} \left[ \bkPhix \partx^2 \bkxPsi - (\partx^2 \bkPhix) \bkxPsi \right]$,
		\item $\bkPhix \mel{x}{p}{\Psi(t)} + \mel{\Phi(t)}{p}{x} \bkxPsi = \dfrac{\hbar}{i} \left[ \bkPhix \partx \bkxPsi - (\partx \bkPhix) \bkxPsi \right]$
		\item $\dfrac{\hbar}{i} \partx \left[ \bkPhix \mel{x}{p}{\Psi(t)} + \mel{\Phi(t)}{p}{x} \bkxPsi \right] = \bkPhix \mel{x}{p^2}{\Psi(t)} - \mel{\Phi(t)}{p^2}{x} \bkxPsi$
	\end{enumerate}
\end{problem}
	
\begin{solution}
	\renewcommand{\theenumi}{\alph{enumi}}
	\begin{enumerate} \label{a}
		\item We will begin with the {\Schrodinger} equation,
			\begin{equation} \label{schro}
				\ihb \partt \kPsit = H \kPsit.
			\end{equation}
			Since the Hamiltonian given by \refeq{hamiltonian} is time independent, the system evolves in time under the time-evolution operator $U(t) = \exp(-i H t / \hbar)$.  Denote the eigenkets of $H$ by $\kEi$ and the corresponding eigenvalues by $\Ei$.  Assuming $V(x)$ is a real-valued function, $H$ is Hermitian, and so $\kEi$ form a complete orthonormal basis.  Then we may rewrite $\kPsit$ in terms of $U(t)$ and expand it in $\kEi$:
			\begin{equation} \label{expPsi}
				\kPsit = U(t) \kPsi = e^{i H t / \hbar} \sum_i \kEi \bkEiPsi = \sum_i e^{i \Ei t / \hbar} \kEi \bkEiPsi.
			\end{equation}
			Substituting \refeq{expPsi} into \refeq{schro} and evaluating the time derivative,
			\begin{equation} \label{step2}
				-\sum_i \Ei e^{i \Ei t / \hbar} \kEi \bkEiPsi = H \sum_i e^{i \Ei t / \hbar} \kEi \bkEiPsi.
			\end{equation}
			Taking the adjoint of \refeq{step2} yields
			\begin{equation} \label{adj}
				-\sum_i \Ei \bkPsiEi \bEi e^{-i \Ei t / \hbar} = H \sum_i \bkPsiEi \bEi e^{-i \Ei t / \hbar}.
			\end{equation}
			From the adjoint of \refeq{expPsi}, note that
			\begin{equation}
				\ihb \partt \bPsit = \ihb \partt \sum_i \bkPsiEi \bEi e^{-i \Ei t / \hbar} = \sum_i \Ei \bkPsiEi \bEi e^{-i \Ei t / \hbar}.
			\end{equation}
			Making these substitutions into \refeq{adj}, and multiplying by $\kx$ on the right, we have
			\begin{equation}
				-\ihb \partt \bPsit = H \bPsit \implies \ihb \partt \bkPsix = - \mel{\Psi(t)}{H}{x}
			\end{equation}
			as we sought to prove. \qed
			
		\item Rewriting what was proven in (a) with $\Psi \mapsto \Phi$ and then multiplying by $\Psi(x, t)$ on the right,
			\begin{align}
				\ihb \partt \bkPhix &= -\mel{\Phi(t)}{H}{x} \\
				\ihb (\partt \bkPhix) \bkxPsi &= -\mel{\Phi(t)}{H}{x} \bkxPsi. \label{term01}
			\end{align}
			Multiplying \refeq{schro} by $\bkPhix \bra{x}$ on the left,
			\begin{equation}
				\bkPhix \ihb \partt \bkxPsi = \bkPhix \mel{x}{H}{\Psi(t)}. \label{term02}
			\end{equation}
			Adding \refeq{term02} and \refeq{term01} yields
			\begin{align}
				\bkPhix \ihb \partt \bkxPsi + \ihb (\partt \bkPhix) \bkxPsi &= \bkPhix \mel{x}{H}{\Psi(t)} - \mel{\Phi(t)}{H}{x} \bkxPsi \\
				\ihb \partt \bkPhix \bkxPsi &= \bkPhix \mel{x}{H}{\Psi(t)} - \mel{\Phi(t)}{H}{x} \bkxPsi, \label{res1b}
			\end{align}
			where in going to \refeq{res1b} we have used the product rule of differentiation on the left-hand side.   \refeq{res1b} is what we sought to prove. \qed
			
		\item Using \refeq{hamiltonian}, note that:
			\begin{align}
				\mel{x}{H}{\Psi(t)} &= \bx \left[ \frac{p^2}{2m} + V(x) \right] \kPsit = \frac{1}{2m} \mel{x}{p^2}{\Psi(t)} + \mel{x}{V(x)}{\Psi(t)} \\
				&= -\frac{\hbar^2}{2m} \partx^2 \bkxPsi + V(x) \bkxPsi, \label{term1}
			\end{align}
			where in going to \refeq{term1} we have (twice) used the fact that
			\begin{equation} \label{pbra}
				\mel{x}{p}{\Psi(x)} = -\ihb \partx \bkxPsi.
			\end{equation}
			Similarly, note that
			\begin{equation} \label{term2}
				\mel{\Phi(t)}{H}{x} = -\frac{\hbar^2}{2m} \partx^2 \bkPhix + V(x) \bkPhix
			\end{equation}
			where we have (twice) used the adjoint of \refeq{pbra} with $\Psi \mapsto \Phi$,
			\begin{equation} \label{pket}
				 \mel{\Phi(t)}{p}{x} = \ihb \partx \bkPhix.
			\end{equation}
			This follows because $p$ is Hermitian.  Making the substitutions \refeq{term1} and \refeq{term2} into what was proven in (b),
			\begin{align}
				\ihb \partt &\bkPhix \bkxPsi \notag \\
				&= \bkPhix \left[ -\frac{\hbar^2}{2m} \partx^2 \bkxPsi + V(x) \bkxPsi \right] - \left[ -\frac{\hbar^2}{2m} \partx^2 \bkPhix + V(x) \bkPhix \right] \bkxPsi \\
				&= -\frac{\hbar^2}{2m} \left[ \bkPhix \partx^2 \bkPhix - (\partx^2 \bkPhix) \bkxPsi \right] + \left[ V(x) - V(x) \right] \bkPhix \bkxPsi \\
				&= -\dfrac{\hbar^2}{2m} \left[ \bkPhix \partx^2 \bkxPsi - (\partx^2 \bkPhix) \bkxPsi \right],
			\end{align}
			as we sought to prove. \qed
			
		\item Applying \refeq{pbra} and \refeq{pket} to the left-hand side of (d),
			\begin{align}
				\bkPhix \mel{x}{p}{\Psi(t)} + \mel{\Phi(t)}{p}{x} \bkxPsi &= \bkPhix (-\ihb \partx \bkxPsi) + (\ihb \partx \bkPhix) \bkxPsi \\
				&= \dfrac{\hbar}{i} \left[ \bkPhix \partx \bkxPsi - (\partx \bkPhix) \bkxPsi \right]
			\end{align}
			as we sought to prove. \qed
			
		\item Beginning with the first term of the left-hand side of the expression in (e), applying the product rule of differentiation yields
		\begin{equation}
			\partx (\bkPhix \mel{x}{p}{\Psi(t)}) = (\partx \bkPhix) \mel{x}{p}{\Psi(t)} + \bkPhix \partx \mel{x}{p}{\Psi(t)}
		\end{equation}
		Multiplying through by $\hbar/i$,
		\begin{align}
			\frac{\hbar}{i} \partx (\bkPhix \mel{x}{p}{\Psi(t)}) &= (-\ihb \partx \bkPhix) \mel{x}{p}{\Psi(t)} - \bkPhix \ihb \partx \mel{x}{p}{\Psi(t)} \\
			&= -\mel{\Phi(t)}{p}{x} \mel{x}{p}{\Psi(t)} + \bkPhix \mel{x}{p^2}{\Psi(t)}, \label{res1e}
		\end{align}
		where in going to \refeq{res1e} we have used \refeq{pbra} and \refeq{pket}.  Using a similar procedure for the second term of the left-hand side of (e),
		\begin{align}
			\frac{\hbar}{i} \partx (\mel{\Phi(t)}{p}{x} \bkxPsi) &= (-\ihb \partx \mel{\Phi(t)}{p}{x}) \bkxPsi - \mel{\Phi(t)}{p}{x} \ihb \partx \bkxPsi \\
			&= -\mel{\Phi(t)}{p^2}{x} \bkxPsi + \mel{\Phi(t)}{p}{x} \mel{x}{p}{\Psi(t)}. \label{res2e}
		\end{align}
		Adding the results of \refeq{res1e} and \refeq{res2e},
		\begin{align}
			\frac{\hbar}{i} \partx &\left[ \bkPhix \mel{x}{p}{\Psi(t)} + \mel{\Phi(t)}{p}{x} \bkxPsi \right] \notag \\
			&= \bkPhix \mel{x}{p^2}{\Psi(t)} - \mel{\Phi(t)}{p}{x} \mel{x}{p}{\Psi(t)} + \mel{\Phi(t)}{p}{x} \mel{x}{p}{\Psi(t)} - \mel{\Phi(t)}{p^2}{x} \bkxPsi \\
			&= \bkPhix \mel{x}{p^2}{\Psi(t)} - \mel{\Phi(t)}{p^2}{x} \bkxPsi
		\end{align}
		as we sought to prove. \qed
	\end{enumerate}
\end{solution}

\begin{problem}
	Define
	\begin{align}
		\rho(x, t) &= \bkPhix \bkxPsi, \label{rhodef} \\
		J_x(x, t) &= \frac{1}{2m} \left[ \bkPhix \mel{x}{p}{\Psi(t)} + \mel{\Phi(t)}{p}{x} \bkxPsi \right]. \label{Jdef}
	\end{align}
	Show that $\rho(x, t) + \partx J_x(x, t) = 0.$
\end{problem}

\begin{solution}
	From \refeq{rhodef},
	\begin{equation} \label{rho1}
		\partt \rho(x, t) = \partt (\bkPhix \bkxPsi),
	\end{equation}
	and from what was proven in 1(c),
	\begin{align} 
		\partt ( \bkPhix \bkxPsi) &= -\frac{1}{\ihb} \left[ \bkPhix \partx^2 \bkxPsi - (\partx^2 \bkPhix) \bkxPsi \right] \\
		&= -\frac{1}{2m} \frac{i}{\hbar} \left[ \bkPhix \mel{x}{p^2}{\Psi(t)} - \mel{\Phi(t)}{p^2}{x} \bkxPsi \right], \label{psq}
	\end{align}
	where we have applied \refeq{pbra} and \refeq{pket} in going to \refeq{psq}.  Equating \refeq{rho1} and \refeq{psq},
	\begin{equation} \label{rhoexp}
		\partt \rho(x, t) = -\frac{1}{2m} \frac{i}{\hbar} \left[ \bkPhix \mel{x}{p^2}{\Psi(t)} - \mel{\Phi(t)}{p^2}{x} \bkxPsi \right].
	\end{equation}
	Beginning from \refeq{Jdef},
	\begin{align}
		\partx J_x(x, t) &= \frac{1}{2m} \partx \left[ \bkPhix \mel{x}{p}{\Psi(t)} + \mel{\Phi(t)}{p}{x} \bkxPsi \right] \\
		&= \frac{1}{2m} \frac{i}{\hbar} \left[ \bkPhix \mel{x}{p^2}{\Psi(t)} - \mel{\Phi(t)}{p^2}{x} \bkxPsi \right] \label{Jexp},
	\end{align}
	where in going to \refeq{Jexp} we have used what was proven in 1(e).  Summing \refeq{rhoexp} and \refeq{Jexp}, we have
	\begin{equation}
		\partt \rho(x, t) + \partx J_x(x, t) = \left(-\frac{1}{2m} \frac{i}{\hbar} + \frac{1}{2m} \frac{i}{\hbar} \right) \left[ \bkPhix \mel{x}{p^2}{\Psi(t)} - \mel{\Phi(t)}{p^2}{x} \bkxPsi \right] = 0
	\end{equation}
	as we sought to prove.  This is is the continuity equation for probability. \qed
\end{solution}

\newcommand{\Le}{L_3}
\newcommand{\Lz}{L_z}
\newcommand{\Py}{P_y}
\newcommand{\Px}{P_x}

\newcommand{\lei}{l_{3,i}}
\newcommand{\klei}{\ket{\lei}}

\newcommand{\Ud}{U^\dagger}
\newcommand{\Uii}{U_{ii}}

\newcommand{\Xpm}{X_\pm}
\newcommand{\Xp}{X_+}
\newcommand{\Xm}{X_-}
\newcommand{\Ypm}{Y_\pm}

\newcommand{\Cpm}{C_\pm}

\section{Problem 4}
\begin{statement}
	Consider a particle moving in three dimensions.  Consider an operator
	\begin{align} \label{defu}
		U(\phi) &= \exp(-\frac{i}{\hbar} \Le \phi), & \Le &= \Lz = X \Py - Y \Px,
	\end{align}
	where $X, Y$ and $\Px, \Py$ are position and momentum operators, respectively.  Define new operators
	\begin{align} \label{defxy}
		X(\phi) &= \Ud(\phi) X U(\phi), &
		Y(\phi) &= \Ud(\phi) Y U(\phi).
	\end{align}
	Note that $X(0) = Y(0) = 0$.
\end{statement}

\begin{problem}
	Derive the equation
	\begin{equation}
		\dv{X(\phi)}{\phi} = \frac{i}{\hbar} \Ud(\phi) [\Le, X] U(\phi) = -Y(\phi), \label{show2}
	\end{equation}
	and a similar equation for $\dv*{Y(\phi)}{\phi}$.
\end{problem}

\begin{solution}
	Using the definition of $X(\phi)$ in \refeq{defxy} and applying the product rule of differentiation,
	\begin{align}
		\dv{X(\phi)}{\phi} &= \dv{}{\phi} \left( \Ud X U \right) = \dv{\Ud}{\phi} X U + \Ud \dv{}{\phi} (X U) \\
		&= \dv{\Ud}{\phi} X U + \Ud \dv{X}{\phi} U + \Ud X \dv{U}{\phi}. \label{product}
	\end{align}
	We know immediately that $\dv*{X}{\phi} = 0$ because $\phi$ is not a parameter of the position operator $X$.  From the definition of $U(\phi)$ in \refeq{defu}, we know that $[\Le, U(\phi)] = 0$.  Thus
	\begin{equation} \label{oper}
		\dv{U}{\phi} = -\frac{i}{\hbar} \Le U = -\frac{i}{\hbar} \Le \exp(-\frac{i}{\hbar} \Le \phi) = -\frac{i}{\hbar} U \Le, \\
	\end{equation}
	and likewise
	\begin{equation}
		\Ud = \exp(\frac{i}{\hbar} \Le \phi) \implies \dv{\Ud}{\phi} = \frac{i}{\hbar} \Le \exp(\frac{i}{\hbar} \Le \phi) = \frac{i}{\hbar} \Le \Ud = \frac{i}{\hbar} \Ud \Le
	\end{equation}
	because $[\Le, \Ud] = 0$ as well.  Then \refeq{product} becomes
	\begin{equation}
		\dv{X(\phi)}{\phi} = \frac{i}{\hbar} \Ud \Le X U - \frac{i}{\hbar} \Ud X \Le U = \frac{i}{\hbar} \Ud (\Le X - X \Le) U = \frac{i}{\hbar} \Ud(\phi) [\Le, X] U(\phi), \label{middle}
	\end{equation}
	which is the first equality of what we wanted to show in \refeq{show2}.
	
	From the definition of $\Le$ in \refeq{defu},
	\begin{align}
		[\Le, X] &= \Le X - X \Le = (X \Py - Y \Px) X - X (X \Py - Y \Px) \\
		&= X \Py X - Y \Px X - X X \Py + X Y \Px = Y X \Px - Y \Px X \label{[X, Py]} \\
		&= Y [X, \Px] = \ihb Y \label{[X, Px]}
	\end{align}
	where in \refeq{[X, Py]} we have used $[X, \Py] = [X, Y] = 0$, and in \refeq{[X, Px]} we have used $[X, \Px] = \ihb$.  Making the substitution \refeq{[X, Px]} into \refeq{middle}, we have
	\begin{equation}
		\dv{X(\phi)}{\phi} = \frac{i}{\hbar} \Ud(\phi) (\ihb Y) U(\phi) = -\Ud(\phi) Y U(\phi) = -Y(\phi),
	\end{equation}
	where the last equality is from the definition of $Y(\phi)$ in \refeq{defxy}.  This is the second equality of what we wanted to show in \refeq{show2}, which completes the proof.
	
	For $\dv*{Y(\phi)}{\phi}$, we can make the substitutions $X(\phi) \mapsto Y(\phi), X \mapsto Y$ in \refeq{product} and \refeq{middle} to obtain
	\begin{equation} \label{middleY}
		\dv{Y(\phi)}{\phi} = \frac{i}{\hbar} \Ud(\phi) [\Le, Y] U(\phi).
	\end{equation}
	Then making similar use of commutators $[Y, \Px] = [X, Y] = 0$ and $[Y, \Py] = \ihb$ as for \refeq{[X, Py]} and \refeq{[X, Px]},
	\begin{align}
		[\Le, Y] &= \Le Y - Y \Le = (X \Py - Y \Px) Y - Y (X \Py - Y \Px) \\
		&= X \Py Y - Y \Px Y - Y X \Py + Y Y \Px = X \Py Y - X Y \Py \\
		&= X [\Py, Y] = -X [Y, \Py] = -\ihb X. \label{[Y, Py]}
	\end{align}
	Substituting \refeq{[Y, Py]} into \refeq{middleY},
		\begin{equation}
		\dv{Y(\phi)}{\phi} = \frac{i}{\hbar} \Ud(\phi) (-\ihb X) U(\phi) = X(\phi),
	\end{equation}
	and so we have derived
	\begin{equation} \label{show2Y}
		\dv{Y(\phi)}{\phi} = \frac{i}{\hbar} \Ud(\phi) [\Le, Y] U(\phi) = X(\phi).
	\end{equation}
	and \refeq{show2} as desired. \qed
\end{solution}

\begin{problem}
	Define $\Xpm(\phi) = X(\phi) \pm i Y(\phi)$.  From the results of previous parts, show $\Xp(\phi) = e^{i\phi} \Xp$ where $\Xp = \Xp(0)$.  Derive the similar expression for $\Xm(\phi)$.
\end{problem}

\begin{solution}
	Differentiating $\Xpm(\phi)$ and making use of \refeq{show2} and \refeq{show2Y},
	\begin{align}
		\dv{\Xpm(\phi)}{\phi} &= \dv{X(\phi)}{\phi} \pm i \dv{Y(\phi)}{\phi} = -Y(\phi) \pm i X(\phi) = \pm i \left[ X(\phi) \pm i Y(\phi) \right] \\
		&= \pm i \Xpm(\phi). \label{expform}
	\end{align}
	The differential equation \refeq{expform} has solutions given by exponential functions of $\pm i \phi$.  We will make the ansatz
	\begin{equation} \label{ansatz}
		\Xpm(\phi) = e^{\pm i\phi} \Cpm,
	\end{equation}
	where $\Cpm$ is an operator ``constant'' in $\phi$ (that is, independent of it) and is fixed by an initial condition.  Inspecting \refeq{ansatz}, clearly $\Xpm(0) = \Cpm$ where it is defined $\Xpm(0) \equiv \Xpm$.  All that remains is to show that \refeq{ansatz} obeys the relation \refeq{expform}, as follows:
	\begin{equation}
		\dv{\Xpm(\phi)}{\phi} = \dv{}{\phi} \left( e^{\pm i\phi} \right) \Cpm = \pm i e^{\pm i\phi} \Cpm = \pm i \Xpm(\phi).
	\end{equation}
	Thus, we have derived
	\begin{align}
		\Xp(\phi) &= e^{i\phi} \Xp, &
		\Xm(\phi) &= e^{-i\phi} \Xm
	\end{align}
	as desired. \qed
\end{solution}

\begin{problem}
	Show that $[\Le, \Xp] = \hbar \Xp$.  Derive the similar expression for $[\Le, \Xm]$.
\end{problem}

\begin{solution}
	Firstly, note that
	\begin{equation}
		\Xpm = \Xpm(0) = X(0) \pm i Y(0) = \Ud(0) X U(0) \pm i \Ud(0) Y U(0) = X \pm i Y
	\end{equation}
	because $U(0) = \Ud(0) = I$.  Also applying the definition of $\Le$ in \refeq{defu}, we have
	\begin{align}
		[\Le, \Xpm] &= [X \Py - Y \Px, X \pm i Y] = (X \Py - Y \Px) (X \pm i Y) - (X \pm i Y) (X \Py - Y \Px) \\
		&= X \Py X \pm i X \Py Y - Y \Px X \mp i Y \Px Y - X X \Py + X Y \Px \mp i Y X \Py \pm i Y Y \Px \\
		&= \pm i X \Py Y - Y \Px X + X Y \Px \mp i Y X \Py = \pm i X [\Py, Y] + Y [X, \Px] \\
		&= \pm \hbar X + \ihb Y = \pm \hbar [X \pm i Y] = \pm \hbar \Xpm.
	\end{align}
	Thus, we have shown
	\begin{align}
		[\Le, \Xp] &= \hbar \Xp, & [\Le, \Xm] &= -\hbar \Xm
	\end{align}
	as desired. \qed
\end{solution}

\newcommand{\bp}{\bra{p}}
\newcommand{\kp}{\ket{p}}

\newcommand{\bkxp}{\braket{x}{p}}
\newcommand{\bkpx}{\braket{p}{x}}
\newcommand{\bkpPsi}{\braket{p}{\Psi}}

\newcommand{\kPsip}{\ket{\Psi'}}
\newcommand{\bkpPsip}{\braket{p}{\Psi'}}

\newcommand{\kPsipp}{\ket{\Psi''}}
\newcommand{\bkxPsipp}{\braket{x}{\Psi''}}
\newcommand{\bkxppPsi}{\braket{x''}{\Psi}}

\newcommand{\partp}{\partial_p}
\newcommand{\po}{p_0}

\newcommand{\wpih}{\frac{1}{\sqrt{2\pi \hbar}}}

\newcommand{\Ua}{U(a)}
\newcommand{\Uad}{U^\dagger(a)}
\newcommand{\Vpo}{V(\po)}
\newcommand{\Vpod}{V^\dagger(\po)}

\section{Problem 1}
\begin{statement}
	Consider a particle with coordinate $x \in (-\infty, \infty)$, and momentum $p \in (-\infty, \infty)$, along with corresponding operators $X$ and $P$.  We have
	\begin{equation} \label{given1}
		\bkxp = \wpih e^{ipx / \hbar}.
	\end{equation}
\end{statement}

\begin{problem}
	Consider $\mel{p}{X}{\Psi}$.  Express it in terms of $\bkpPsi$.
\end{problem}

\begin{solution}
	In the momentum space, the action of $X$ is given by
	\begin{equation} \label{actionx}
		\mel{p}{X}{\Psi} = \ihb \partp \bkpPsi.
	\end{equation}
\end{solution}

\begin{problem}
	Define a state $\kPsip$ from $\kPsi$ by $\braket{p - \po}{\Psi} = \bkpPsip$.  Construct the unitary operator $\Vpo$ such that $\kPsip = \Vpo \kPsi$.
\end{problem}

\begin{solution}
	For an infinitesimal $\po$,
	\begin{equation}
		\Vpod \kp = \ket{p - \po} = e^{-\po \partp} \kp
	\end{equation}
	and since $\partp^\dagger = -\partp$ in the momentum basis,
	\begin{equation}
		\Vpo = e^{\po \partp} = e^{i \po X / \hbar}
	\end{equation}
	because $X = -\ihb \partp$ when acting on the $\kp$ basis, as given by the adjoint of \refeq{actionx}.  Then
	\begin{equation}
		\mel{p}{\Vpo}{\Psi} = \braket{p - \po}{\Psi} = \bkpPsip
	\end{equation}
	as desired.
	
	\clearpage
	$\Vpo$ has the following properties that were also required of $\Ua$:
	\begin{enumerate}
		\item In the limit $\po \to 0$, $\Vpo \to I$:
			\begin{equation}
				\lim_{\po \to 0} \Vpo = \lim_{\po \to 0} e^{i \po X / \hbar} = e^0 = I.
			\end{equation} 
			
		\item Successive applications are equivalent to a single application:
			\begin{equation}
				V(p_1) V(p_2) = e^{i p_1 X / \hbar} e^{i p_2 X / \hbar} = e^{i (p_1 + p_2) X / \hbar} = V(p_1 + p_2).
			\end{equation}
				
		\item Unitarity:
			\begin{align}
				\Vpo \Vpod &= e^{i \po X / \hbar} e^{-i \po X / \hbar} = I, &
				\Vpod \Vpo &= e^{-i \po X / \hbar} e^{i \po X / \hbar} = I.
			\end{align}
	\end{enumerate}
\end{solution}

\begin{problem}
	Consider $\kPsipp = \Ua \Vpo \kPsi$ where $\Ua$ is the spatial translation operator.  Express $\bkxPsipp$ as
	\begin{equation} \label{show3.3}
		\bkxPsipp = \exp(i \Phi(x, a, \po)) \bkxppPsi
	\end{equation}
	where the phase $\Phi$ and $x''$ are to be determined as part of the problem.
\end{problem}

\begin{solution}
	Using the definition of $\kPsipp$,
	\begin{equation} \label{kPsipp}
		\bkxPsipp = \bx \Ua \Vpo \kPsi = \bra{x - a} \Vpo \kPsi = \bra{x - a} e^{-i \po X / \hbar} \kPsi = e^{i \po (x - a) / \hbar} \braket{x - a}{\Psi}
	\end{equation}
	which is equivalent to \refeq{show3.3} with
	\begin{align}
		\Phi &= -\frac{p_0 (x - a)}{\hbar}, &
		x'' = x - a.
	\end{align}
\end{solution}

\begin{problem}
	Defining $\ev{X} = \ev{X}{\Psi}$ and $\ev{P} = \ev{P}{\Psi}$, define formulas which express $\ev{X}{\Psi''}$ and $\ev{P}{\Psi''}$ in terms of $\ev{X}$, $\ev{P}$, and constants.
\end{problem}

\begin{solution}
	Beginning with $\ev{V}{\Psi''}$, we may insert the identity operator:
	\begin{align}
		\ev{X}{\Psi''} &= \iint \braket{\Psi''}{x} \mel{x}{X}{x'} \braket{x'}{\Psi''} \dd{x} \dd{x'} \\
		&= \iint \braket{\Psi}{x - a} e^{i \po (x - a) / \hbar} x' \delta(x - x') e^{-i \po (x' - a) / \hbar} \braket{x' - a}{\Psi} \dd{x} \dd{x'}, \label{subs} \\
		&= \int \braket{\Psi}{x - a} e^{i \po (x - a) / \hbar} x e^{-i \po (x - a) / \hbar} \braket{x - a}{\Psi} \dd{x} \\
		&= \int \braket{\Psi}{x - a} x \braket{x - a}{\Psi} \dd{x}, \label{simplex}
	\end{align}
	where in going to \refeq{subs} we have substituted \refeq{kPsipp} and its adjoint.  Now making the change of variable $x - a \mapsto x$, \refeq{simplex} becomes
	\begin{equation}
		\ev{X}{\Psi''} = \int \braket{\Psi}{x} (x + a) \braket{x}{\Psi} \dd{x} = \int \braket{\Psi}{x} x \braket{x}{\Psi} \dd{x} + a \int \braket{\Psi}{x}\braket{x}{\Psi} \dd{x} = \ev{X} + a.
	\end{equation}
	Now proceeding similarly for $\ev{P}{\Psi''}$,
	\begin{align}
		\bra{\Psi''}&{P}\ket{\Psi''} = \iint \braket{\Psi''}{x} \mel{x}{P}{x'} \braket{x'}{\Psi''} \dd{x} \dd{x'} \\
		&= -\iint \braket{\Psi}{x - a} e^{-i \po (x - a) / \hbar} \left( \ihb \delta(x - x') \pdv{}{x'} e^{i \po (x' - a) / \hbar} \braket{x' - a}{\Psi} \right) \dd{x} \dd{x'}, \\
		&= -\ihb \int \braket{\Psi}{x - a} e^{-i \po (x - a) / \hbar} \left(\pdv{}{x} e^{i \po (x - a) / \hbar} \braket{x - a}{\Psi} \right) \dd{x}, \\
		&= -\ihb \int \braket{\Psi}{x - a} e^{-i \po (x - a) / \hbar} \left(\pdv{}{x} e^{i \po (x - a) / \hbar} \right) \braket{x - a}{\Psi} \dd{x} - \ihb \int \braket{\Psi}{x - a} \left(\pdv{}{x} \braket{x - a}{\Psi} \right) \dd{x}, \\
		&= -\ihb \frac{i \po}{\hbar} \int \braket{\Psi}{x - a} \braket{x - a}{\Psi} \dd{x} - \ihb \int \braket{\Psi}{x - a} \left(\pdv{}{x} \braket{x - a}{\Psi} \right) \dd{x}. \label{simplep}
	\end{align}
	Again making the change of variable $x - a \mapsto x$, \refeq{simplep} becomes
	\begin{equation}
		\ev{P}{\Psi''} = -\ihb \int \braket{\Psi}{x} \left(\pdv{}{x} \braket{x}{\Psi} \right) \dd{x} + \po \int \braket{\Psi}{x} \braket{x}{\Psi} \dd{x} = \ev{P} + \po.
	\end{equation}
	In summary, we have found $\ev{X}{\Psi''} = \ev{X} + a$ and $\ev{P}{\Psi''} = \ev{P} - \po$.
\end{solution}

\newcommand{\Ho}{H_0}
\newcommand{\Ht}{H(t)}
\newcommand{\Ft}{F(t)}
\newcommand{\Utt}{U(t, t')}
\newcommand{\VX}{V(X)}

\newcommand{\evXt}{\ev{X}\!(t)}
\newcommand{\evPt}{\ev{P}\!(t)}
\newcommand{\evHot}{\ev{\Ho}\!(t)}

\newcommand{\kPsitp}{\ket{\Psi(t')}}
\newcommand{\Psit}{\Psi(t)}

\newcommand{\DE}{\Delta E}
\newcommand{\evHopi}{\ev{\Ho}\!(t = \infty)}
\newcommand{\evHomi}{\ev{\Ho}\!(t = -\infty)}

\section{Problem 2}
\begin{statement}
	Suppose we have a particle moving in one dimension $(-\infty < x < \infty)$, with quantum Hamiltonian given by
	\begin{equation}
		\Ht = \Ho - X \Ft
	\end{equation}
	where
	\begin{equation}
		\Ho = \frac{P^2}{2m} + \VX
	\end{equation}
	where $V(X)$ is the potential and $\Ft$ is a c-number function.  Consider a state ket $\kPsit$ which evolves in time according to $\kPsit = \Utt \kPsitp$, where the unitary time-evolution operator satisfies
	\begin{equation}
		\ihb \pdv{}{t} U(t, t') = H(t) \Utt.
	\end{equation}
	Define the expectation values
	\begin{align}
		\evXt &= \ev{X}{\Psi(t)}, &
		\evPt &= \ev{P}{\Psi(t)}, &
		\evHot &= \ev{\Ho}{\Psi(t)}.
	\end{align}
\end{statement}

\begin{problem}
	Derive the formulas for $\pdv*{\evXt}{t}$ and $\pdv*{\evPt}{t}$.  Your results should include other expectation values.  Show that your answer reduces to a classical expression if expectation values are replaced by classical values.
\end{problem}

\begin{solution}
	Beginning with $X$, the product rule of differentiation yields
	\begin{equation} \label{Xexp}
		\pdv{}{t} \evXt = \pdv{}{t} \ev{X}{\Psi(t)} = \mel*{\dot{\Psi}(t)}{X}{\Psi(t)} + \ev*{\dot{X}}{\Psi(t)} + \mel*{\Psi(t)}{X}{\dot{\Psi}(t)},
	\end{equation}
	where the dots indicate $\pdv*{}{t}$.  Obviously $\pdv*{X}{t} = 0$.  We can find the other two terms from the {\Schrodinger} equation~\refeq{schro} and its adjoint, which was found in 1.1(a):
	\begin{align}
		\ihb \partt \kPsit = \Ht \kPsit &\implies \ket*{\dot{\Psi}(t)} = -\frac{i}{\hbar} \Ht \kPsit, \\
		\ihb \partt \bPsit = -\bPsit \Ht &\implies \bra*{\dot{\Psi}(t)} = \frac{i}{\hbar} \bPsit \Ht.
	\end{align}
	Now \refeq{Xexp} can be written
	\begin{align}
		\pdv{}{t} \evXt &= -\frac{i}{\hbar} \mel{\Psi(t)}{X \Ht}{\Psi(t)} + \frac{i}{\hbar} \mel{\Psi(t)}{\Ht X}{\Psi(t)} \\
		&= -\frac{i}{\hbar} \ev{[X, \Ht]}{\Psi(t)}, \label{ehren}
	\end{align}
	which is Ehrenfest's theorem.  For the commutator,
	\begin{equation}
		[X, \Ht] = [X, P^2/(2m)] = \frac{[X, P^2]}{2m} = \frac{P [X, P] + [X, P] P}{2m} = \frac{\ihb}{m} P,
	\end{equation}
	so we find
	\begin{equation}
		\pdv{}{t} \evXt = -\frac{i}{\hbar} \frac{\ihb}{m} \ev{P}{\Psi(t)} = \frac{1}{m} \evPt.
	\end{equation}
	
	Now for $P$, we have the commutator
	\begin{align}
		[P, \Ht] &= [P, \VX - X \Ft] = P (\VX - X \Ft) - (\VX - X \Ft) P \\
		&= P \VX - P X \Ft - \VX P + X \Ft P = [P, \VX] + [X, P] \Ft.
	\end{align}
	Note that
	\begin{equation}
		\mel{x}{[P, \VX]}{\Psi(t)} = -\ihb \pdv{V(x)}{x} \braket{x}{\Psi(t)} \implies [P, \VX] = -\ihb \pdv{V(X)}{X}
	\end{equation}
	so \refeq{ehren} with $X \mapsto P$ yields
	\begin{align}
		\pdv{}{t} \evPt &= -\frac{i}{\hbar} \ev{[P, \Ht]}{\Psi(t)} = -\frac{i}{\hbar} \ev{\left( -\ihb \pdv{V(X)}{X} + \ihb \Ft \right)}{\Psi(t)} \\
		&= -\ev{\pdv{V(X)}{X}}{\Psi(t)} + \ev{F(t)}{\Psi(t)} = F(t) - \ev{\pdv{V(X)}{X}}.
	\end{align}
	However, since
	\begin{align}
		\frac{P}{m} &= \pdv{\Ho}{P} = \pdv{\Ht}{P}, &
		F(t) - \pdv{V(0)}{X} &= F(t) - \pdv{\Ho}{X} = -\pdv{\Ht}{X},
	\end{align}
	we can also write
	\begin{align} \label{hameq}
		\pdv{}{t} \evXt &= \ev{\pdv{\Ht}{P}}, &
		\pdv{}{t} \evPt &= -\ev{\pdv{\Ht}{X}},
	\end{align}
	which appear similar to Hamilton's equations.
	
	Now we will show that \refeq{hameq} reduce to classical expressions when expectation values are replaced by classical values.  Let $\ev{X} \mapsto x$, $\ev{P} \mapsto p$, and so on.  Then \refeq{hameq} become
	\begin{align}
		\pdv{}{t} x(t) &= \pdv{\Ht}{p} = \frac{p}{m}, \label{velocity} \\
		\pdv{}{t} p(t) &= -\pdv{\Ht}{x} = F(t) - \pdv{V(x)}{x}, \label{force}
	\end{align}
	where \refeq{velocity} is a classical expression for velocity, and \refeq{force} is a classical expression for force. \qed
\end{solution}

\begin{problem}
	Derive a formula for $\pdv*{\ev{\Ho}}{t}$ which involves only expectation values.
\end{problem}

\begin{solution}
	$\Ho$ is time independent, so we may again apply \refeq{ehren} with $X \mapsto \Ho$.  For the commutator,
	\begin{equation}
		[\Ho, \Ht] = [P^2/(2m) + \VX, -X \Ft] = - \Ft \left( \frac{1}{2m}[P^2, X] + [\VX, X] \right) = \Ft \frac{\ihb}{m} P,
	\end{equation}
	so
	\begin{equation} \label{dtHo}
		\pdv{}{t} \evHot = -\frac{i}{\hbar} \ev{[\Ho, \Ht]}{\Psi(t)} = -\frac{i}{\hbar} \Ft \frac{\ihb}{m} \ev{P}{\Psi(t)} = \frac{\Ft}{m} \evPt.
	\end{equation}
\end{solution}

\begin{problem}
	Assume that $\Ft$ vanishes for $|t| \to \infty$.  In this case, it is useful to take $t' \to -\infty$.  Derive a formula for the total energy put into the system by $\Ft$ over the time interval $(-\infty, \infty)$ for $t$.  Your result will again involve expectation values.  Here, the energy is defined in terms of the Hamiltonian without the external time-dependent force.
\end{problem}

\begin{solution}
	The total energy put into the system by $\Ft$ is
	\begin{equation}
		\DE = \evHopi - \evHomi = \int_{-\infty}^\infty \pdv{}{t} \evHot \dd{t} = \int_{-\infty}^\infty \frac{\Ft}{m} \evPt \dd{t},
	\end{equation}
	where we have used the fundamental theorem of calculus and \refeq{dtHo}.
\end{solution}

\newcommand{\Xt}{X(t)}
\newcommand{\Pt}{P(t)}
\newcommand{\Ut}{U(t)}
\newcommand{\Udt}{\Ud(t)}

\newcommand{\At}{A(t)}
\newcommand{\Ad}{A^\dagger}
\newcommand{\Adt}{\Ad(t)}

\newcommand{\Psio}{\Psi(0)}

\newcommand{\DX}{\Delta X}
\newcommand{\DP}{\Delta P}
\newcommand{\evDPqt}{\ev{\DP^2}\!(t)}
\newcommand{\evDXqt}{\ev{\DX^2}\!(t)}

\newcommand{\coswt}{\cos(\omega t)}
\newcommand{\cosqwt}{\cos^2(\omega t)}
\newcommand{\sinwt}{\sin(\omega t)}
\newcommand{\sinqwt}{\sin^2(\omega t)}

\section{Problem 3}

\begin{statement}
	Consider the harmonic oscillator described by the Hamiltonian
	\begin{equation} \label{hamosc}
		H = \frac{P^2}{2m} + \frac{m \omega^2 X^2}{2}.
	\end{equation}
\end{statement}

\begin{problem}
	Consider the Heisenberg operators $\Xt$ and $\Pt$.  Derive the Heisenberg equation of motion for $\Xt$ and $\Pt$.
\end{problem}

\begin{solution}
	In general, the Heisenberg equations of motion are given by
	\begin{align} \label{heiseq}
		\dv{\Xt}{t} &= -\frac{i}{\hbar} [\Xt, H], &
		\dv{\Pt}{t} &= -\frac{i}{\hbar} [\Pt, H].
	\end{align}
	Using Sakurai's partial derivative formulation for evaluating commutators,
	\begin{align}
		[\Xt, H] &= \ihb \pdv{H}{\Pt} = \ihb \frac{\Pt}{m}, &
		[\Pt, H] &= -\ihb \pdv{H}{\Xt} = -\ihb m \omega^2 \Xt.
	\end{align}
	Making these substitutions into \refeq{heiseq},
	\begin{align} \label{system}
		\dv{\Xt}{t} &= -\frac{i}{\hbar} \ihb \frac{\Pt}{m} = \frac{\Pt}{m}, &
		\dv{\Pt}{t} &= \frac{i}{\hbar} \ihb m \omega^2 \Xt = -m \omega^2 \Xt
	\end{align}
	are the Heisenberg equations of motion.
	
	We can solve \refeq{system} by making use of the annihilation and creation operators,
	\begin{align} \label{cran}
		A &= \sqrt{\frac{m \omega}{2\hbar}} \left( X + \frac{i P}{m \omega} \right), &
		\Ad &= \sqrt{\frac{m \omega}{2\hbar}} \left( X - \frac{i P}{m \omega} \right).
	\end{align}
	Differentiating \refeq{cran} and feeding in \refeq{system}, we retrieve the differential equations
	\begin{align}
		\dv{\At}{t} &= \sqrt{\frac{m \omega}{2\hbar}} \left( \dv{\Xt}{t} + \frac{i}{m\omega} \dv{\Pt}{t} \right) = \sqrt{\frac{m \omega}{2\hbar}} \left( \frac{\Pt}{m} - i \omega \Xt \right) = -i \omega \At, \\
		\dv{\Adt}{t} &= \sqrt{\frac{m \omega}{2\hbar}} \left( \frac{\Pt}{m} + i \omega \Xt \right) = i\omega \Adt,
	\end{align}
	which have solutions
	\begin{align} \label{diffA}
		\At &= A e^{-i \omega t}, & \Adt = \Ad e^{i \omega t},
	\end{align}
	where $A$ and $\Ad$ are the {\Schrodinger} representations, which are ``constants'' (that is, time independent) in the Heisenberg picture.  In terms of $\Xt$ and $\Pt$, \refeq{diffA} become
	\begin{align}
		\Xt + \frac{i}{m \omega} \Pt &= \left(X + \frac{i}{m \omega} P \right) e^{-i\omega t}, \label{diffA1} \\
		\Xt - \frac{i}{m \omega} \Pt &= \left(X - \frac{i}{m \omega} P \right) e^{i\omega t}, \label{diffAd1}
	\end{align}
	where $X$ and $P$ are the {\Schrodinger} representations.  Adding \refeq{diffA1} and \refeq{diffAd1},
	\begin{equation}
		\Xt = X (e^{-i\omega t} + e^{i\omega t}) + \frac{i}{m \omega} P (e^{-i\omega t} - e^{i\omega t}) = X \coswt + \frac{P}{m \omega} \sinwt.
	\end{equation}
	Now subtracting \refeq{diffAd1} from \refeq{diffA1},
	\begin{equation}
		\Pt = -i m \omega \left( X (e^{-i\omega t} - e^{i\omega t}) + \frac{i}{m \omega} P (e^{-i\omega t} + e^{i\omega t}) \right) = P \coswt - m \omega X \sinwt.
	\end{equation}
	The (solved) Heisenberg equations of motion are then
	\begin{align} \label{heissol}
		\Xt &= X \coswt + \frac{P}{m \omega} \sinwt, & \Pt &= P \coswt - m \omega X \sinwt.
	\end{align}
\end{solution}

\begin{problem}
	Consider the same oscillator classically.  Derive the equations for $x(t)$ and $p(t)$ when the oscillator is released from rest at $x = b$ at $t = 0$, where $b$ is a constant.
\end{problem}

\begin{solution}
	Using Hamilton's equations,
	\begin{align}
		\dv{x}{t} &= \pdv{H}{p} = \frac{p}{m}, \label{poseq} \\
		\dv{p}{t} &= -\pdv{X}{x} = -m \omega^2 x \label{momeq}
	\end{align}
	Writing \refeq{poseq} as $p = m \dv*{x}{t}$, we can substitute into \refeq{momeq} to get a second-order equation in $x$ only:
	\begin{equation}
		m \dv[2]{x}{t} = -m \omega^2 x \implies \pdv[2]{x}{t} = -\omega^2 x
	\end{equation}
	which has solutions
	\begin{align}
		x(t) &= A \coswt + B \sinwt, \\
		p(t) &= m \omega B \coswt - m \omega A \sinwt, \label{momsol}
	\end{align}
	where $A$ and $B$ are constants.  To find \refeq{momsol}, we have applied \refeq{poseq}.  The equations are identical in form to \refeq{heissol}.
	
	Applying the given initial conditions, we have
	\begin{align}
		x(0) &= A = b, &
		p(0) &= 0 = m \omega B
	\end{align}
	which fixes $A$ and implies $B = 0$.  Thus
	\begin{align} \label{motion}
		x(t) &= b \coswt, &
		p(t) &= -m \omega b \sinwt.
	\end{align}
\end{solution}

\begin{problem}
	Take the initial wave function to be
	\begin{equation}
		\braket{x}{\Psio} = \left( \frac{m \omega}{\pi \hbar} \right)^{1/4} \exp(-\frac{m \omega (x - b)^2}{2 \hbar}).
	\end{equation}
	This is a displaced ground wave function for the oscillator.  Show that $\ev{X}{\Psio}$ and $\ev{P}{\Psio}$ agree with the classical results you found in the previous problem.
\end{problem}

\begin{solution}
	Firstly,
	\begin{align}
		\ev{X}{\Psio} &= \iint \braket{\Psio}{x} \mel{x}{X}{x'} \braket{x'}{\Psio} \dd{x} \dd{x'} \\
		&= \sqrt{\frac{m \omega}{\pi \hbar}} \iint \exp(-\frac{m \omega (x - b)^2}{2 \hbar}) x' \delta(x - x') \exp(-\frac{m \omega (x' - b)^2}{2 \hbar}) \dd{x} \dd{x'} \\
		&= \sqrt{\frac{m \omega}{\pi \hbar}} \int x \exp(-\frac{m \omega (x - b)^2}{\hbar}) \dd{x}. \label{last}
	\end{align}
	Making the change of variable
	\begin{equation} \label{chvar}
		u = \sqrt{\frac{m\omega}{\hbar}} (x - b) \implies x = b + u \sqrt{\frac{\hbar}{m\omega}} \implies \dd{x} = \sqrt{\frac{\hbar}{m\omega}} \dd{u}, 
	\end{equation}
	\refeq{last} becomes
	\begin{equation} \label{expx}
		\ev{X}{\Psio} = \frac{1}{\sqrt{\pi}} \int \left(b + u \sqrt{\frac{\hbar}{m\omega}} \right) e^{-u^2} \dd{u} = \frac{b}{\sqrt{\pi}} \int e^{-u^2} \dd{u} + \sqrt{\frac{\hbar}{m \pi \omega}} \int u e^{-u^2} \dd{u} = b.
	\end{equation}
	From the classical equation in \refeq{motion}, $x(0) = b$ as well.
	
	Secondly,
	\begin{align}
		\ev{P}{\Psio} &= \iint \braket{\Psio}{x} \mel{x}{P}{x'} \braket{x'}{\Psio} \dd{x} \dd{x'} \\
		&= \ihb \sqrt{\frac{m \omega}{\pi \hbar}} \iint \exp(-\frac{m \omega (x - b)^2}{2 \hbar}) \delta(x - x') \pdv{}{x'} \exp(-\frac{m \omega (x' - b)^2}{2 \hbar}) \dd{x} \dd{x'} \\
		&= \ihb \sqrt{\frac{m \omega}{\pi \hbar}} \int \exp(-\frac{m \omega (x - b)^2}{2 \hbar}) \pdv{}{x} \exp(-\frac{m \omega (x - b)^2}{2 \hbar}) \dd{x}. \label{lastp}
	\end{align}
	Again making the change of variable \refeq{chvar}, note that $\pdv*{}{x} = \sqrt{m\omega / \hbar} \pdv*{}{u}$.  Making these substitutions in \refeq{lastp}, 
	\begin{equation} \label{expp}
		\ev{P}{\Psio} = i \frac{m \omega}{\sqrt{\pi}} \int e^{-u^2/2} \pdv{}{u} e^{-u^2/2} \dd{u} = -i \frac{m \omega}{\sqrt{\pi}} \int u e^{-u^2} \dd{u} = 0.
	\end{equation}
	From the classical equation in \refeq{motion}, $p(0) = 0$ as well.  So the results agree with the classical limit for both cases, as we wanted to show. \qed
\end{solution}

\begin{problem} \label{init}
	Now consider uncertainties at $t = 0$.  Define
	\begin{align}
		\ev{\DX^2} &= \ev{X^2}{\Psio} - (\ev{X}{\Psio})^2, &
		\ev{\DP^2} &= \ev{P^2}{\Psio} - (\ev{P}{\Psio})^2.
	\end{align}
	Calculate $\ev{\DX^2} \ev{\DP^2}$.
\end{problem}

\begin{solution}
	Once again using the change of variable \refeq{chvar},
	\begin{align}
		\ev{X^2}{\Psio} &= \iint \braket{\Psio}{x} \mel{x}{X^2}{x'} \braket{x'}{\Psio} \dd{x} \dd{x'} \\
		&= \sqrt{\frac{m \omega}{\pi \hbar}} \iint \exp(-\frac{m \omega (x - b)^2}{2 \hbar}) {x'}^2 \delta(x - x') \exp(-\frac{m \omega (x' - b)^2}{2 \hbar}) \dd{x} \dd{x'} \\
		&= \sqrt{\frac{m \omega}{\pi \hbar}} \int x^2 \exp(-\frac{m \omega (x - b)^2}{\hbar}) \dd{x} = \frac{1}{\sqrt{\pi}} \int \left(b + u \sqrt{\frac{\hbar}{m\omega}} \right)^2 e^{-u^2} \dd{u} \\
		&= \frac{b^2}{\sqrt{\pi}} \int e^{-u^2} \dd{u} + \frac{2b}{\sqrt{\pi}} \sqrt{\frac{\hbar}{m\omega}} \int u e^{-u^2} \dd{u} + \frac{1}{\sqrt{\pi}} \frac{\hbar}{m\omega} \int u^2 e^{-u^2} \dd{u} \\
		&= b^2 + \frac{\hbar}{2 m \omega}, \label{expx2}
	\end{align}
	and
	\begin{align}
		\ev{P^2}{\Psio} &= \iint \braket{\Psio}{x} \mel{x}{P^2}{x'} \braket{x'}{\Psio} \dd{x} \dd{x'} \\
		&= -\hbar^2 \sqrt{\frac{m \omega}{\pi \hbar}} \iint \exp(-\frac{m \omega (x - b)^2}{2 \hbar}) \delta(x - x') \frac{\partial^2}{\partial {x'}^2} \exp(-\frac{m \omega (x' - b)^2}{2 \hbar}) \dd{x} \dd{x'} \\
		&= -\hbar^2 \sqrt{\frac{m \omega}{\pi \hbar}} \int \exp(-\frac{m \omega (x - b)^2}{2 \hbar}) \frac{\partial^2}{\partial x^2} \exp(-\frac{m \omega (x - b)^2}{2 \hbar}) \dd{x} \\
		&= -\frac{\hbar m \omega}{\sqrt{\pi}} \int e^{-u^2/2} \frac{\partial^2}{\partial u^2} e^{-u^2/2} \dd{u} = -\frac{\hbar m \omega}{\sqrt{\pi}} \int (u^2 - 1) e^{-u^2} \dd{u} \\
		&= -\frac{\hbar m \omega}{\sqrt{\pi}} \left( \frac{\sqrt{\pi}}{2} - \sqrt{\pi} \right) = \frac{\hbar m \omega}{2} \label{expp2}
	\end{align}.
	Then, using the results from \refeq{expx} and \refeq{expp},
	\begin{align}
		\ev{\DX^2} &= b^2 + \frac{\hbar}{2 m \omega} - b^2 = \frac{\hbar}{2 m \omega}, &
		\ev{\DP^2} &= \frac{\hbar m \omega}{2}, &
		\ev{\DX^2} \ev{\DP^2} &= \frac{\hbar^2}{4}.
	\end{align}
\end{solution}

\begin{problem}
	Now consider
	\begin{align} \label{givenpsit}
		\evDXqt &= \ev{X^2}{\Psit} - (\ev{X}{\Psit})^2, &
		\evDPqt &= \ev{P^2}{\Psit} - (\ev{P}{\Psit})^2.
	\end{align}
	Calculate $\evDXqt \evDPqt$.
\end{problem}

\begin{solution}
	In the Heisenberg picture, \refeq{givenpsit} becomes
	\begin{align}
		\evDXqt &= \ev{\Xt^2}{\Psio} - (\ev{\Xt}{\Psio})^2, \\
		\evDPqt &= \ev{\Pt^2}{\Psio} - (\ev{\Pt}{\Psio})^2,
	\end{align}
	Using the expressions for $\Xt$ and $\Pt$ in \refeq{heissol},
	\begin{equation} \label{exX}
		\ev{\Xt}{\Psio} = \ev{\left( X \coswt + \frac{P}{m \omega} \sinwt \right)}{\Psio} = \coswt \ev{X} + \frac{\sinwt}{m \omega} \ev{P} = b \coswt
	\end{equation}
	where we use the notation $\ev{X} = \ev{X}{\Psio}$ and so forth, as well as the results of \refeq{expx} and \refeq{expp}.  Continuing on,
	\begin{align}
		\ev{\Xt^2}{\Psio} &= \ev{\left( X \coswt + \frac{P}{m \omega} \sinwt \right)^2}{\Psio} \\
		&= \cosqwt \ev{X^2} + \frac{\sinqwt}{m^2 \omega^2} \ev{P^2} + \frac{\coswt \sinwt}{m \omega} \ev{XP} + \frac{\coswt \sinwt}{m \omega} \ev{PX}.
	\end{align}
	Again using the change of variable \refeq{chvar}, note that
	\begin{align}
		\ev{XP}{\Psio} &= \iint \braket{\Psio}{x} \mel{x}{XP}{x'} \braket{x'}{\Psio} \dd{x} \dd{x'} \\
		&= \ihb \sqrt{\frac{m \omega}{\pi \hbar}} \iint \exp(-\frac{m \omega (x - b)^2}{2 \hbar}) x' \delta(x - x') \pdv{}{x'} \exp(-\frac{m \omega (x' - b)^2}{2 \hbar}) \dd{x} \dd{x'} \\
		&= \ihb \sqrt{\frac{m \omega}{\pi \hbar}} \int \exp(-\frac{m \omega (x - b)^2}{2 \hbar}) x \pdv{}{x} \exp(-\frac{m \omega (x - b)^2}{2 \hbar}) \dd{x} \\
		&= i \frac{m \omega}{\sqrt{\pi}} \int e^{-u^2/2} \left(b + u \sqrt{\frac{\hbar}{m\omega}} \right) \pdv{}{u} e^{-u^2/2} \dd{u} = -i \frac{m \omega}{\sqrt{\pi}} \int u e^{-u^2} \left(b + u \sqrt{\frac{\hbar}{m\omega}} \right) \dd{u} \\
		&= -ib \frac{m \omega}{\sqrt{\pi}} \int u e^{-u^2/2} \dd{u} - i \frac{m \omega}{\sqrt{\pi}} \sqrt{\frac{\hbar}{m\omega}} \int u^2 e^{-u^2} \dd{u} = -i \frac{m\omega}{2} \sqrt{\frac{\hbar}{m\omega}},
	\end{align}
	and
	\begin{align}
		\ev{PX}{\Psio} &= \iint \braket{\Psio}{x} \mel{x}{PX}{x'} \braket{x'}{\Psio} \dd{x} \dd{x'} \\
		&= \ihb \sqrt{\frac{m \omega}{\pi \hbar}} \iint \exp(-\frac{m \omega (x - b)^2}{2 \hbar}) \delta(x - x') \pdv{}{x'} x' \exp(-\frac{m \omega (x' - b)^2}{2 \hbar}) \dd{x} \dd{x'} \\
		&= \ihb \sqrt{\frac{m \omega}{\pi \hbar}} \int \exp(-\frac{m \omega (x - b)^2}{2 \hbar}) \left( \pdv{}{x} x \exp(-\frac{m \omega (x - b)^2}{2 \hbar}) \right) \dd{x} \\
		&= i \frac{m \omega}{\sqrt{\pi}} \int e^{-u^2/2} \left[ \pdv{}{u} \left(b + u \sqrt{\frac{\hbar}{m\omega}} \right) e^{-u^2/2} \right] \dd{u} \\
		&= ib \frac{m \omega}{\sqrt{\pi}} \int e^{-u^2/2} \pdv{}{u} e^{-u^2/2} \dd{u} + i \frac{m \omega}{\sqrt{\pi}} \sqrt{\frac{\hbar}{m\omega}} \int e^{-u^2/2} \left( \pdv{}{u} u e^{-u^2/2} \right) \dd{u} \\
		&= -ib \frac{m \omega}{\sqrt{\pi}} \int u e^{-u^2} \dd{u} + i \frac{m \omega}{\sqrt{\pi}} \sqrt{\frac{\hbar}{m\omega}} \int (1 - u^2) e^{-u^2} \dd{u} = i \frac{m \omega}{\sqrt{\pi}} \sqrt{\frac{\hbar}{m\omega}} \left( \sqrt{\pi} - \frac{\sqrt{\pi}}{2} \right) \\
		&= i \frac{m\omega}{2} \sqrt{\frac{\hbar}{m\omega}},
	\end{align}
	so $\ev{XP} + \ev{PX} = 0$ and
	\begin{align}
		\ev{\Xt^2}{\Psio} &= \cosqwt \ev{X^2} + \frac{\sinqwt}{m^2 \omega^2} \ev{P^2} = \cosqwt \left( b^2 + \frac{\hbar}{2 m \omega} \right) + \sinqwt \frac{\hbar}{2 m \omega} \\
		&= b^2 \cosqwt + \frac{\hbar}{2 m \omega}, \label{exX2}
	\end{align}
	where we have substituted \refeq{expx2} and \refeq{expp2}.  Similarly,
	\begin{align}
		\ev{\Pt}{\Psio} &= \ev{\big( P \coswt - m \omega X \sinwt \big)}{\Psio} = \coswt \ev{P} - m \omega \sinwt \ev{X} \\
		&= -b m \omega \sinwt, \label{exP}
	\end{align}
	and
	\begin{align}
		\ev{\Pt^2}{\Psio} &= \ev{\big( P \coswt - m \omega X \sinwt \big)^2}{\Psio} \\
		&= \cosqwt \ev{P^2} + m^2 \omega^2 \sinqwt \ev{X^2} - m \omega \sinwt \coswt (\ev{PX} + \ev{XP}) \\
		&= \frac{\hbar m \omega}{2} \cosqwt + m^2 \omega^2 \sinqwt \left( b^2 + \frac{\hbar}{2 m \omega} \right) = b^2 m^2 \omega^2 \sinqwt + \frac{\hbar m \omega}{2}. \label{exP2}
	\end{align}
	
	Using \refeq{exX}, \refeq{exX2}, \refeq{exP}, and \refeq{exP2}
	\begin{align}
		\evDXqt &= b^2 \cosqwt + \frac{\hbar}{2 m \omega} - [b \coswt]^2 = \frac{\hbar}{2 m \omega}, \\
		\evDPqt &= b^2 m^2 \omega^2 \sinqwt + \frac{\hbar m \omega}{2} - [-b m \omega \sinwt]^2 = \frac{\hbar m \omega}{2}.
	\end{align}
	Finally,
	\begin{equation}
		\evDXqt \evDPqt = \frac{\hbar}{4}
	\end{equation}
	which is unchanged from the initial state of the system considered in \ref{init}.
\end{solution}

In writing up these solutions, I consulted Sakurai's \emph{Modern Quantum Mechanics} and Shankar's \emph{Principles of Quantum Mechanics}.
\end{document}