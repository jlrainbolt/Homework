\documentclass[11pt]{article}
\usepackage{geometry}
\usepackage[parfill]{parskip}
\usepackage{physics, amsfonts, amsthm}
\usepackage{fullpage}
\usepackage{fancyhdr}
\usepackage{enumitem}
\usepackage{xcolor, soul}
%\allowdisplaybreaks
 
 
\setlist[enumerate]{leftmargin=*}
\pagestyle{fancy}
\fancyhf{}
\lhead{\textbf{Physics 316 Homework 1}}
\rhead{Lacey Rainbolt}
\setlength{\headheight}{14pt}
\setlength{\headsep}{12pt}

\newcommand{\pder}[2]{\frac{\partial#1}{\partial#2}}
\newcommand{\pmder}[3]{\frac{\partial^2#1}{\partial#2 \, \partial#3}}
\newcommand{\der}[2]{\frac{d#1}{d#2}}
\newcommand{\refeq}[1]{(\ref{#1})}


\newenvironment{problem}
{
    \color{darkgray}
    \textbf{Problem.}\quad
    \ignorespaces
}


\newenvironment{solution}
{
    \paragraph{Solution.}
    \ignorespaces
}
{
    \bigskip\bigskip
}



\begin{document}
\begin{enumerate}

\newcommand{\qd}{\dot{q}}
\newcommand{\Qd}{\dot{Q}}

    \item \begin{problem}
        Suppose we have a mechanical system with $n$ degrees of freedom.  Let $q_1(t), q_2(t), \ldots, q_n(t)$ be its generalized coordinates.  Now consider a time-dependent coordinate transformation
        $$Q_i = Q_i(t, q_1, q_2, \ldots, q_n) \quad\quad i = 1, 2, \ldots, n.$$
        Show that if $q_i(t)$ solve a system of Euler-Lagrange equations involving a Lagrangian $L(t, q_i, \qd_i)$, then $Q_i(t)$ solves the Euler-Lagrange equations involving $L(t, Q_i, \Qd_i)$ provided the time-dependent coordinate transformation fulfills some minimal standard of good behavior.  Specify this ``minimal standard of good behavior.''
    \end{problem}
    
    \begin{solution}
        Suppose that
        \begin{equation} \label{given}
        \pder{L}{q_i} - \der{}{t} \pder{L}{\qd_i} = 0;
        \end{equation}
        that is, $q_i(t)$ solve a system of Euler-Lagrange equations.  We want to show that
        \begin{equation} \label{show}
        \pder{L}{Q_i} - \der{}{t} \pder{L}{\Qd_i} = 0.
        \end{equation}
        
	Beginning with the first term of~\refeq{show}, we can use the chain rule to write
	\begin{equation} \label{showleft}
	\pder{L}{Q_i} = \pder{L}{q_j} \pder{q_j}{Q_i} + \pder{L}{\qd_j} \pder{\qd_j}{Q_i}.
	\end{equation}
	However, $\partial q_j / \partial Q_i$ and $\partial \qd_j / \partial Q_i$ are only guaranteed to exist if there is an inverse transformation
	\begin{equation}
	q_i = q_i(t, Q_1, Q_2, \ldots, Q_n) \quad\quad i = 1, 2, \ldots, n.
	\end{equation}
	This is only possible if there is a one-to-one correspondence between $q_i(t)$ and $Q_i(t)$, which is the ``minimal standard of good behavior'' for the transformation.

	Assuming this is the case, we can write
	\begin{equation} \label{qdot}
	\qd_i = \pder{q_i}{Q_j} \Qd_j + \pder{q_i}{t}
	\end{equation}
	so \refeq{showleft} becomes
	\begin{equation} \label{showleft2}
%	\pder{L}{Q_i} = \pder{L}{q_j} \pder{q_j}{Q_i} + \pder{L}{\qd_j} \pder{}{Q_i} \left( \pder{q_j}{Q_k} \Qd_k + \pder{q_j}{t} \right).
	\pder{L}{Q_i} = \pder{L}{q_j} \pder{q_j}{Q_i} + \pder{L}{\qd_j} \left(\pmder{q_j}{Q_i}{Q_k} \Qd_k + \pmder{q_j}{t}{Q_i} \right).
	\end{equation}
	
	For the second term of \refeq{show}, we have
	\begin{equation}
	\pder{L}{\Qd_i} = \pder{L}{\qd_j} \pder{\qd_j}{\Qd_i}  = \pder{L}{\qd_j} \pder{q_j}{Q_i}
	\end{equation}
	where the right-hand side comes from applying \refeq{qdot}.  Then, using the product rule to take the time derivative,
	\begin{equation} \label{showright}
	\der{}{t} \pder{L}{\Qd_i} = \pder{q_j}{Q_i} \der{}{t} \pder{L}{\qd_j} + \pder{L}{\qd_j} \der{}{t} \pder{q_j}{Q_i}.
	\end{equation}
	\hl{For the second term of} \refeq{showright}, \hl{the chain rule gives}
	\begin{equation}
	\pder{L}{\qd_j} \der{}{t} \pder{q_j}{Q_i} =  \pder{L}{\qd_j} \left( \pmder{q_j}{t}{Q_i} + \pmder{q_j}{Q_i}{Q_k} \Qd_k \right).
	\end{equation}
	The second term on the right side also appeared in \refeq{showleft2}, so substituting back into \refeq{showright} we now have
	\begin{equation}
	\der{}{t} \pder{L}{\Qd_i} = \pder{q_j}{Q_i} \der{}{t} \pder{L}{\qd_j} + \pder{L}{Q_i} - \pder{L}{q_j} \pder{q_j}{Q_i}.
	\end{equation}
	Rearranged, this is
	\begin{equation} \label{showright2}
	\pder{L}{Q_i} - \der{}{t} \pder{L}{\Qd_i} = \pder{q_j}{Q_i} \left( \pder{L}{q_j} - \der{}{t} \pder{L}{\qd_j} \right).
	\end{equation}
	Finally, substituting the original assumption \refeq{given}, we have
	\begin{equation} \label{showright2}
	\pder{L}{Q_i} - \der{}{t} \pder{L}{\Qd_i} = 0
	\end{equation}
	which is what we sought to prove. \qed
	

       
    \end{solution}


\end{enumerate}
\end{document}