\documentclass[11pt]{article}
\usepackage{geometry}
\usepackage[parfill]{parskip}
\usepackage{physics, amsfonts, amsthm}
\usepackage{fullpage}
\usepackage{fancyhdr}
\usepackage{enumitem}
\usepackage{xcolor, soul}
%\allowdisplaybreaks

\makeatletter
\renewcommand*\env@cases[1][1.2]{%
  \let\@ifnextchar\new@ifnextchar
  \left\lbrace
  \def\arraystretch{#1}%
  \array{@{}l@{\quad}l@{}}%
}
\makeatother
 
 
\setlist[enumerate]{leftmargin=*}
\pagestyle{fancy}
\fancyhf{}
\lhead{\textbf{Physics 316 Homework 1}}
\rhead{Lacey Rainbolt}
\setlength{\headheight}{14pt}
\setlength{\headsep}{12pt}

\newcommand{\pder}[2]{\frac{\partial#1}{\partial#2}}
\newcommand{\pmder}[3]{\frac{\partial^2#1}{\partial#2 \, \partial#3}}
\newcommand{\der}[2]{\frac{d#1}{d#2}}
\newcommand{\refeq}[1]{(\ref{#1})}


\newenvironment{problem}
{
    \color{darkgray}
    \textbf{Problem.}\quad
    \ignorespaces
}


\newenvironment{solution}
{
    \paragraph{Solution.}
    \ignorespaces
}
{
    \bigskip\bigskip
}



\begin{document}
\begin{enumerate}

\newcommand{\qd}{\dot{q}}
\newcommand{\qdd}{\ddot{q}}
\newcommand{\Qd}{\dot{Q}}
\newcommand{\Qdd}{\ddot{Q}}

	\item \begin{problem}
		Suppose we have a mechanical system with $n$ degrees of freedom.  Let $q_1(t), q_2(t), \ldots, q_n(t)$ be its generalized coordinates.  Now consider a time-dependent coordinate transformation
		$$Q_i = Q_i(t, q_1, q_2, \ldots, q_n) \quad\quad i = 1, 2, \ldots, n.$$
		Show that if $q_i(t)$ solves a system of Euler-Lagrange equations involving a Lagrangian $L(t, q_i, \qd_i)$, then $Q_i(t)$ solves the Euler-Lagrange equations involving $L(t, Q_i, \Qd_i)$ provided the time-dependent coordinate transformation fulfills some minimal standard of good behavior.  Specify this ``minimal standard of good behavior.''
	\end{problem}
    
	\begin{solution}
		Suppose that
		\begin{equation} \label{given}
			\pder{L}{q_i} - \der{}{t} \pder{L}{\qd_i} = 0;
		\end{equation}
		that is, $q_i(t)$ solve a system of Euler-Lagrange equations.  We want to show that
		\begin{equation} \label{show}
			\pder{L}{Q_i} - \der{}{t} \pder{L}{\Qd_i} = 0.
		\end{equation}
        
		Beginning with the first term of~\refeq{given}, we can use the chain rule for $L(t, Q_i, \Qd_i)$ to write
		\begin{equation} \label{givenleft}
			\pder{L}{q_i} = \pder{L}{Q_j} \pder{Q_j}{q_i} + \pder{L}{\Qd_j} \pder{\Qd_j}{q_i},
		\end{equation}
%		provided $Q_j$ and $\Qd_j$ are continuously differentiable with respect to $q_1, q_2, \ldots, q_n$.
		provided there exists an inverse transformation
		\begin{equation}
			q_i = q_i(t, Q_1, Q_2, \ldots, Q_n) \quad\quad i = 1, 2, \ldots, n
		\end{equation}
		that allows us to write $L(t, q_i, \qd_i)$ in terms of $t$, $Q_i$, and $\Qd_i$.  This is only possible if there is a one-to-one correspondence between $q_i(t)$ and $Q_i(t)$, which is the ``minimal standard of good behavior'' for the transformation.  We will assume the transformation is so well behaved.

		Again using the chain rule for $Q_j = Q_j(t, q_1, q_2, \ldots, q_n)$, note that
		\begin{equation} \label{qdot}
			\Qd_j =  \pder{Q_j}{t} + \pder{Q_j}{q_i} \qd_i
		\end{equation}
		so \refeq{givenleft} becomes
		\begin{equation} \label{givenleft2}
			\pder{L}{q_i} = \pder{L}{Q_j} \pder{Q_j}{q_i} + \pder{L}{\Qd_j} \left(\pmder{Q_j}{q_i}{t} + \pmder{Q_j}{q_i}{q_k} \qd_k \right).
		\end{equation}
	
		Applying the chain rule now to the second term of \refeq{given}, we have
		\begin{equation}
			\pder{L}{\qd_i} = \pder{L}{\Qd_j} \pder{\Qd_j}{\qd_i}  = \pder{L}{\Qd_j} \pder{Q_j}{q_i}
		\end{equation}
		where the right-hand side comes from \refeq{qdot}.  Then, using the product rule to take the time derivative,
		\begin{equation} \label{givenright}
			\der{}{t} \pder{L}{\qd_i} = \left( \der{}{t} \pder{L}{\Qd_j} \right) \pder{Q_j}{q_i} + \pder{L}{\Qd_j} \left( \der{}{t} \pder{Q_j}{q_i} \right).
		\end{equation} \label{givenright2}
		For the second term of \refeq{givenright}, the chain rule for $Q_j = Q_j(t, q_1, q_2, \ldots, q_n)$ gives
		\begin{equation} \label{givenright3}
			\der{}{t} \pder{Q_j}{q_i} = \pmder{Q_j}{t}{q_i} + \pmder{Q_j}{q_i}{q_k} \qd_k.
		\end{equation}
		Substituting \refeq{givenright3} into \refeq{givenright}, we have
		\begin{equation}
			\der{}{t} \pder{L}{\qd_i} = \left( \der{}{t} \pder{L}{\Qd_j} \right) \pder{Q_j}{q_i} + \pder{L}{\Qd_j} \left( \pmder{Q_j}{t}{q_i} + \pmder{Q_j}{q_k}{q_i} \qd_k \right)
		\end{equation}
		where the terms on the far right appeared in \refeq{givenleft2}.  Making this substitution,
		\begin{equation}
			\der{}{t} \pder{L}{\qd_i} = \left( \der{}{t} \pder{L}{\Qd_j} \right) \pder{Q_j}{q_i} + \pder{L}{q_i} - \pder{L}{Q_j} \pder{Q_j}{q_i},
		\end{equation}
		and rearranging,
		\begin{equation}
		\pder{Q_j}{q_i} \left( \pder{L}{Q_j} - \der{}{t} \pder{L}{\Qd_j} \right) = \pder{L}{q_i} - \der{}{t} \pder{L}{\qd_i},
		\end{equation}
		Finally, substituting the original assumption \refeq{given}, we find
		\begin{equation}
			\pder{L}{Q_j} - \der{}{t} \pder{L}{\Qd_j} = 0
		\end{equation}
		which is what we sought to prove. \qed
	\end{solution}


	\item \begin{problem}
		Look at the Lagrangian
		$$L = e^{\sigma t} \left( \frac{m \qd^2}{2} - \frac{kq^2}{2} \right)$$
		for one-dimensional motion.
		
	\begin{enumerate}
		\item Write down the associated Euler-Lagrange ODE.
		
		\item Now perform a point transformation
		$$Q = e^{\sigma t / 2} q$$
		where the new position coordinate $Q$ is a function of $t$ and $q$.  What is the equation of motion for $Q(t)$?  Are there conserved quantities?
	\end{enumerate}
	\end{problem}
	
	\begin{solution}
	\begin{enumerate}
		\item Beginning from the general expression for the Euler-Lagrange equations,
			\begin{equation}
				\pder{L}{q} - \der{}{t} \pder{L}{\qd} = -e^{\sigma t} kq - \der{}{t}\left( e^{\sigma t} m\qd \right) = -m e^{\sigma t} \left( \qdd + \sigma \qd + \frac{k}{m} q \right)
			\end{equation}
			so the ODE is
			\begin{equation} \label{ode}
				0 = \qdd + \sigma \qd + \frac{k}{m} q.
			\end{equation}
		
		\item It is possible to invert this transformation and write $q = q(t, Q)$.  Explicitly, this is
			\begin{equation} \label{inversion}
				q = Q e^{-\sigma t / 2}
			\end{equation}
			so
			\begin{equation}
				\qd = e^{-\sigma t / 2} \left(\Qd - \frac{\sigma}{2} Q \right).
			\end{equation}
			Rewriting the Lagrangian such that $L = L(t, Q, \Qd)$ results in
			\begin{align}
				L &= e^{\sigma t} \left(\frac{m}{2} \left(e^{-\sigma t / 2} \left( \Qd - \frac{\sigma}{2} Q \right) \right)^2 - \frac{k}{2} \left( Q e^{-\sigma t / 2} \right)^2 \right) \\
				&= \frac{m}{2} \left( \Qd - \frac{\sigma}{2} Q \right)^2 - \frac{k}{2} Q^2 \\
				&= \label{lagrangian} \frac{m}{2} \left(\Qd^2 - \sigma \Qd Q + \left(\frac{\sigma^2}{4} - \frac{k}{m} \right) Q^2 \right).
			\end{align}
			Then the Euler-Lagrange equations are given by
			\begin{equation}
				0 = \pder{L}{Q} - \der{}{t} \pder{L}{\Qd} = \frac{m}{2} \left( -\sigma \Qd + 2 \left( \frac{\sigma^2}{4} - \frac{k}{m} \right) Q - \der{}{t} \left(2 \Qd - \sigma Q \right) \right)
			\end{equation}
			which simplifies to
			\begin{equation} \label{newode} 
				0 = \Qdd + \left( \frac{k}{m} - \frac{\sigma^2}{4} \right) Q.
			\end{equation}
			
			The solutions to \refeq{newode} have the form
			\begin{equation}
				Q(t) = \begin{cases}[2.8]
						A_1 \sin\left(\sqrt{\dfrac{k}{m} - \dfrac{\sigma^2}{4}}\,t\right) + A_2 \cos\left(\sqrt{\dfrac{k}{m} - \dfrac{\sigma^2}{4}}\,t\right) \quad \quad &\text{if}\quad \dfrac{k}{m} > \dfrac{\sigma^2}{4}, \\
						B_1 + B_2 t \quad \quad &\text{if}\quad \dfrac{k}{m} = \dfrac{\sigma^2}{4}, \\
						C_1 \exp{-\sqrt{\dfrac{k}{m} - \dfrac{\sigma^2}{4}}\,t} + C_2 \exp{\sqrt{\dfrac{k}{m} - \dfrac{\sigma^2}{4}}\,t} \quad \quad &\text{if} \quad \dfrac{k}{m} < \dfrac{\sigma^2}{4}, \\
					\end{cases}
			\end{equation}
			where $A_i, B_i, C_i$ are real constants.
			
			The Lagrangian in \refeq{lagrangian} does not explicitly depend on time.  Thus, the total energy $H$ of the system is conserved.  Explicitly,
			\begin{equation}
				H = \Qd \pder{L}{\Qd} - L = \frac{m}{2} \left( \Qd^2 - \left(\frac{\sigma^2}{4} - \frac{k}{m} \right) Q^2 \right)
			\end{equation}
			is a conserved quantity.

	\end{enumerate}
	\end{solution}
	
	\renewcommand{\vec}[1]{\mathbf{#1}}
	\newcommand{\vr}{\vec{r}}
	\newcommand{\vrd}{\vec{\dot{r}}}
	\newcommand{\vrdd}{\vec{\ddot{r}}}
	\newcommand{\vR}{\vec{R}}
	\newcommand{\vRd}{\vec{\dot{R}}}
	\newcommand{\vRdd}{\vec{\ddot{R}}}
	
	\item \begin{problem}
		Let $U(\alpha \vr_1, \ldots, \alpha \vr_N)$ be a potential for $N$ particles that satisfies the relation
		$$U(\alpha \vr_1, \ldots, \alpha \vr_N) = \alpha^k U(\vr_1, \ldots, \vr_N).$$
		The factor $\alpha$ can be any nonzero real number.  The exponent $k$ is an integer.
		
		\begin{enumerate}
			\item Show that the equations of motion associated with such a potential remain unchanged under a dilation of the distance scale if the time scale is also dilated by some other factor $\beta$.  Find $\beta$ as a function of $\alpha$ and $k$.
			
			\item If $k = 2$, the forces correspond to a system of harmonic oscillators coupled to each other.  Show that the result in part (a) implies the frequencies of such a system are independent of the oscillation amplitude.
			
			\item If $k = -1$, we have an inverse square force law, such as that which arises in mutual gravitational attraction.  Show that the result in part (a) implies Kepler's third law: the square of the orbital period of a planet is directly proporitonal to the cube of the semi-major axis of its orbit.
		\end{enumerate}
		
	\end{problem}
	
	\begin{solution}
		The Lagrangian $L = L(t, \vr_i, \vrd_i)$ for the system of $N$ particles is
		\begin{equation} \label{lagr1}
			L = T - U = \frac{1}{2} m_i \vrd_i \cdot \vrd_i - U(\vr_1, \ldots, \vr_N)
		\end{equation}
		where $m_i$ is the mass of the particle located at $\vr_i$.  The Euler-Lagrange equations for this Lagrangian are
		\begin{equation} \label{el1}
			\pder{L}{\vr_i} - \der{}{t} \pder{L}{\vrd_i} = 0 \implies \pder{U}{\vr_i} + m_i \vrdd_i = 0.
		\end{equation}

		 Define the time scale transformation
		 \begin{equation}
		 	T = \beta t,
		\end{equation}
		and define the coordinate transformation
		\begin{equation}
			\vR_i = \vR_i(T) = \alpha \vr_i
		\end{equation}
		for all $N$ particles.  Using these coordinates, the Lagrangian $L = L(T, \vR_i, \vRd_i)$ is
		\begin{equation} \label{lagr2}
			L = \frac{1}{2} m_i \vRd_i \cdot \vRd_i - U(\vR_1, \ldots, \vR_N)
		\end{equation}
		and the Euler-Lagrange equations are
		\begin{equation} \label{el2}
			\pder{L}{\vR_i} - \der{}{T} \pder{L}{\vRd_i} = 0 \implies \pder{U}{\vR_i} + m_i \vRdd_i = 0.
		\end{equation}
		
		\begin{enumerate}
			\item The equations of motion associated to the Lagrangians \refeq{lagr1} and \refeq{lagr2} are identical if the Euler-Lagrange equations in \refeq{el1} and \refeq{el2} are identical.  We will now show that this is the case.
			
			The transformation $\vR_i = \alpha \vr_i$ is invertible, so $\vr_i = \vR_i / \alpha$.  Likewise, $t = T / \beta$.  By the chain rule,
			\begin{equation} \label{transT}
				\der{}{T} = \der{}{t} \der{t}{T} = \frac{1}{\beta} \der{}{t}
			\end{equation}
			so
			\begin{equation}
				\vRd = \alpha \der{\vr_i}{T} = \frac{\alpha}{\beta} \vrd_i
			\end{equation}
			and, likewise,
			\begin{equation} \label{secondder}
				\vRdd = \frac{\alpha}{\beta^2} \vrdd_i.
			\end{equation}
			From the given relationship for $U$, note that
			\begin{equation} \label{urel}
				U(\vR_1, \ldots, \vR_N) = \alpha^k U(\vr_1, \ldots, \vr_N)
			\end{equation}
			and again using the chain rule,
			\begin{equation} \label{rpart}
				\pder{}{\vR_i} = \pder{}{\vr_i} \der{\vr_i}{\vR_i} = \frac{1}{\alpha} \pder{}{\vr_i}.
			\end{equation}
			
			Making use of \refeq{secondder}, \refeq{urel}, and \refeq{rpart}, we can rewrite \refeq{el2} in terms of the original coordinates:
			\begin{equation}
			0 = \frac{\alpha^k}{\alpha} \pder{U}{\vr_i} + m_i \frac{\alpha}{\beta^2} \vrdd_i \implies \alpha^k \beta^2 \pder{U}{\vr_i} + m_i \vrdd_i = 0
			\end{equation}
			which is equivalent to \refeq{el1} so long as
			\begin{equation}
				\alpha^k \beta^2 = 1 \implies \beta = \pm\alpha^{-k/2}.
			\end{equation}
			\qed
			
			
			\item Fixing $k=2$ requires that $U$ is of the form
			\begin{equation}
				U(\vr_1, \ldots, \vr_N) = \frac{k_{ij}}{2} \vr_i \cdot \vr_j
			\end{equation}
			where $k_ij$ are real constants.  Then
			\begin{equation} \label{upart}
				\pder{U}{\vr_i} = \frac{k_{ii}}{2} \vr_i + \frac{k_{ij}}{2} \vr_j.
			\end{equation}
			Substituting \refeq{upart} into \refeq{el1}, the Euler-Lagrange equations for this system are
			\begin{equation} \label{el3}
				m_i \vrdd_i + \frac{k_{ii}}{2} \vr_i + \frac{k_{ij}}{2} \vr_j = 0.
			\end{equation}
			We make an ansatz for the solutions,
			\begin{equation}
				\vr_i(t) = A_i \cos(\omega t)
			\end{equation}
			where $A_i$ are constants representing the amplitude of oscillation and $\omega$ are the normal mode frequencies.  Then
			\begin{equation}
				\vrdd_i = -A_i \omega^2 \cos(\omega t)
			\end{equation}
			so \refeq{el3} becomes
			\begin{equation}
				-m_i A_i \omega^2 \cos(\omega t) + \frac{k_{ii}}{2} A_i \cos(\omega t) + \frac{k_{ij}}{2} A_j \cos(\omega t) = 0 \implies m_i A_i \omega^2 - \frac{k_{ii}}{2} A_i + \frac{k_{ij}}{2} A_j = 0
			\end{equation}
			\hl{???}
			
			
			\item \hl{help}
		\end{enumerate}
	\end{solution}
	
	
	\item \begin{problem}
		A particle in three-dimensional space is confined in a central potential
		$$U(r) = -U_0 \left( \frac{r_0}{r} \right)^n.$$
		Here $r = |\vr|$ where $\vr(t)$ is the location of the particle at time $t$, $U_0$ is a characteristic energy scale and $r_0$ is a characteristic length scale.  Show that the particle motion is confined to a two-dimensional orbital plane.  For what values of $n$ are circular orbits stable?
	
	\end{problem}
	

\end{enumerate}
\end{document}