\documentclass[11pt]{article}
\usepackage{geometry}
\usepackage[parfill]{parskip}
\usepackage{physics, amsfonts, amsthm}
\usepackage{fullpage}
\usepackage{fancyhdr}
\usepackage{enumitem}
\usepackage{xcolor, soul}
%\allowdisplaybreaks
 
 
\setlist[enumerate]{leftmargin=*}
\pagestyle{fancy}
\fancyhf{}
\lhead{\textbf{Physics 316 Homework 1}}
\rhead{Lacey Rainbolt}
\setlength{\headheight}{14pt}
\setlength{\headsep}{12pt}

\newcommand{\pder}[2]{\frac{\partial#1}{\partial#2}}
\newcommand{\pmder}[3]{\frac{\partial^2#1}{\partial#2 \, \partial#3}}
\newcommand{\der}[2]{\frac{d#1}{d#2}}
\newcommand{\refeq}[1]{(\ref{#1})}


\newenvironment{problem}
{
    \color{darkgray}
    \textbf{Problem.}\quad
    \ignorespaces
}


\newenvironment{solution}
{
    \paragraph{Solution.}
    \ignorespaces
}
{
    \bigskip\bigskip
}



\begin{document}
\begin{enumerate}

\newcommand{\qd}{\dot{q}}
\newcommand{\qdd}{\ddot{q}}
\newcommand{\Qd}{\dot{Q}}

	\item \begin{problem}
		Suppose we have a mechanical system with $n$ degrees of freedom.  Let $q_1(t), q_2(t), \ldots, q_n(t)$ be its generalized coordinates.  Now consider a time-dependent coordinate transformation
		$$Q_i = Q_i(t, q_1, q_2, \ldots, q_n) \quad\quad i = 1, 2, \ldots, n.$$
		Show that if $q_i(t)$ solves a system of Euler-Lagrange equations involving a Lagrangian $L(t, q_i, \qd_i)$, then $Q_i(t)$ solves the Euler-Lagrange equations involving $L(t, Q_i, \Qd_i)$ provided the time-dependent coordinate transformation fulfills some minimal standard of good behavior.  Specify this ``minimal standard of good behavior.''
	\end{problem}
    
	\begin{solution}
		Suppose that
		\begin{equation} \label{given}
			\pder{L}{q_i} - \der{}{t} \pder{L}{\qd_i} = 0;
		\end{equation}
		that is, $q_i(t)$ solve a system of Euler-Lagrange equations.  We want to show that
		\begin{equation} \label{show}
			\pder{L}{Q_i} - \der{}{t} \pder{L}{\Qd_i} = 0.
		\end{equation}
        
		Beginning with the first term of~\refeq{given}, we can use the chain rule for $L(t, Q_i, \Qd_i)$ to write
		\begin{equation} \label{givenleft}
			\pder{L}{q_i} = \pder{L}{Q_j} \pder{Q_j}{q_i} + \pder{L}{\Qd_j} \pder{\Qd_j}{q_i},
		\end{equation}
%		provided $Q_j$ and $\Qd_j$ are continuously differentiable with respect to $q_1, q_2, \ldots, q_n$.
		provided there exists an inverse transformation
		\begin{equation}
			q_i = q_i(t, Q_1, Q_2, \ldots, Q_n) \quad\quad i = 1, 2, \ldots, n
		\end{equation}
		that allows us to write $L(t, q_i, \qd_i)$ in terms of $t$, $Q_i$, and $\Qd_i$.  This is only possible if there is a one-to-one correspondence between $q_i(t)$ and $Q_i(t)$, which is the ``minimal standard of good behavior'' for the transformation.  We will assume the transformation is so well behaved.

		Again using the chain rule for $Q_j = Q_j(t, q_1, q_2, \ldots, q_n)$, note that
		\begin{equation} \label{qdot}
			\Qd_j =  \pder{Q_j}{t} + \pder{Q_j}{q_i} \qd_i
		\end{equation}
		so \refeq{givenleft} becomes
		\begin{equation} \label{givenleft2}
			\pder{L}{q_i} = \pder{L}{Q_j} \pder{Q_j}{q_i} + \pder{L}{\Qd_j} \left(\pmder{Q_j}{q_i}{t} + \pmder{Q_j}{q_i}{q_k} \qd_k \right).
		\end{equation}
	
		Applying the chain rule now to the second term of \refeq{given}, we have
		\begin{equation}
			\pder{L}{\qd_i} = \pder{L}{\Qd_j} \pder{\Qd_j}{\qd_i}  = \pder{L}{\Qd_j} \pder{Q_j}{q_i}
		\end{equation}
		where the right-hand side comes from \refeq{qdot}.  Then, using the product rule to take the time derivative,
		\begin{equation} \label{givenright}
			\der{}{t} \pder{L}{\qd_i} = \left( \der{}{t} \pder{L}{\Qd_j} \right) \pder{Q_j}{q_i} + \pder{L}{\Qd_j} \left( \der{}{t} \pder{Q_j}{q_i} \right).
		\end{equation} \label{givenright2}
		For the second term of \refeq{givenright}, the chain rule for $Q_j = Q_j(t, q_1, q_2, \ldots, q_n)$ gives
		\begin{equation} \label{givenright3}
			\der{}{t} \pder{Q_j}{q_i} = \pmder{Q_j}{t}{q_i} + \pmder{Q_j}{q_i}{q_k} \qd_k.
		\end{equation}
		Substituting \refeq{givenright3} into \refeq{givenright}, we have
		\begin{equation}
			\der{}{t} \pder{L}{\qd_i} = \left( \der{}{t} \pder{L}{\Qd_j} \right) \pder{Q_j}{q_i} + \pder{L}{\Qd_j} \left( \pmder{Q_j}{t}{q_i} + \pmder{Q_j}{q_k}{q_i} \qd_k \right)
		\end{equation}
		where the terms on the far right appeared in \refeq{givenleft2}.  Making this substitution,
		\begin{equation}
			\der{}{t} \pder{L}{\qd_i} = \left( \der{}{t} \pder{L}{\Qd_j} \right) \pder{Q_j}{q_i} + \pder{L}{q_i} - \pder{L}{Q_j} \pder{Q_j}{q_i},
		\end{equation}
		and rearranging,
		\begin{equation}
		\pder{Q_j}{q_i} \left( \pder{L}{Q_j} - \der{}{t} \pder{L}{\Qd_j} \right) = \pder{L}{q_i} - \der{}{t} \pder{L}{\qd_i},
		\end{equation}
		Finally, substituting the original assumption \refeq{given}, we find
		\begin{equation}
			\pder{L}{Q_j} - \der{}{t} \pder{L}{\Qd_j} = 0
		\end{equation}
		which is what we sought to prove. \qed
	\end{solution}


	\item \begin{problem}
		Look at the Lagrangian
		$$L = e^{\sigma t} \left( \frac{m \qd^2}{2} - \frac{kq^2}{2} \right)$$
		for one-dimensional motion.
		
	\begin{enumerate}
		\item Write down the associated Euler-Lagrange ODE.
		
		\item Now perform a point transformation
		$$Q = e^{\sigma t / 2} q$$
		where the new position coordinate $Q$ is a function of $t$ and $q$.  What is the equation of motion for $Q(t)$?  Are there conserved quantities?
	\end{enumerate}
	\end{problem}
	
	\begin{solution}
	\begin{enumerate}
		\item Beginning from the general expression for the Euler-Lagrange equations,
			\begin{equation}
				\pder{L}{q} - \der{}{t} \pder{L}{\qd} = -e^{\sigma t} kq - \der{}{t}\left( e^{\sigma t} m\qd \right) = -m e^{\sigma t} \left( \qdd + \sigma \qd + \frac{k}{m} q \right)
			\end{equation}
			so the ODE is
			\begin{equation} \label{ode}
				0 = \qdd + \sigma \qd + \frac{k}{m} q.
			\end{equation}
		
		\item It is possible to invert this transformation and write $q = q(t, Q)$.  This is
			\begin{equation} \label{inversion}
				q = Q e^{-\sigma t / 2}
			\end{equation}
			which implies
			\begin{equation}
				\qd = e^{-\sigma t / 2} \left(\Qd -\frac{\sigma t}{2} Q \right)
			\end{equation}
			and
			\begin{equation}
				\qdd = e^{-\sigma t / 2} \left(\Qd -\frac{\sigma t}{2} Q \right)
			\end{equation}
			Now we can write \refeq{ode} in terms of $Q$ and $\Qd$:
			\begin{equation}
				0 = \qdd + \sigma \qd + \frac{k}{m} q.
			\end{equation}

	\end{enumerate}
	\end{solution}
	

\end{enumerate}
\end{document}