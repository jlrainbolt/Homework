\documentclass[11pt]{article}
\usepackage{geometry, titlesec}
\usepackage[parfill]{parskip}
\usepackage[italicdiff]{physics}
\usepackage{amsfonts, amsthm}
\usepackage[cm]{fullpage}
\usepackage{fancyhdr}
\usepackage{enumitem}
\usepackage{xcolor, soul}
%\allowdisplaybreaks

\renewcommand{\thesubsection}{\thesection.\alph{subsection}}

\makeatletter
\renewcommand*\env@cases[1][1.2]{%
  \let\@ifnextchar\new@ifnextchar
  \left\lbrace
  \def\arraystretch{#1}%
  \array{@{}l@{\quad}l@{}}%
}
\makeatother
 
\renewcommand{\footrulewidth}{.2pt}
%\setlist[enumerate]{leftmargin=*}
\pagestyle{fancy}
\fancyhf{}
\lhead{\textbf{Physics 316 Homework 5}}
\rhead{Lacey Rainbolt}
\setlength{\headheight}{11pt}
\setlength{\headsep}{11pt}
\setlength{\footskip}{24pt}
\lfoot{\today}
\rfoot{\thepage}

\titleformat{\subsection}[runin]{\normalfont\large\bfseries}{\thesubsection}{1em}{}
\newcommand{\refeq}[1]{(\ref{#1})}

\newcommand{\beq}{\begin{equation*}}
\newcommand{\eeq}{\end{equation*}}

\newcommand{\beqn}{\begin{equation}}
\newcommand{\eeqn}{\end{equation}}

\newcommand{\blg}{\begin{align*}}
\newcommand{\elg}{\end{align*}}


\newenvironment{statement}
{
    \color{darkgray}
    \ignorespaces
}
{
%    \smallskip
}

\newenvironment{problem}
{
    \subsection{}
    \color{darkgray}
    \ignorespaces
}

\newenvironment{solution}
{
    \paragraph{Solution.}
    \ignorespaces
}
{
%    \bigskip
}



\begin{document}


\renewcommand{\vec}[1]{\mathbf{#1}}
\newcommand{\intr}{\int_R}
\newcommand{\dR}{\partial R}
\newcommand{\dr}{\dd{x} \dd{y} \dd{z}}
\newcommand{\ux}{u_x}
\newcommand{\uy}{u_y}
\newcommand{\uz}{u_z}

\newcommand{\uxx}{u_{xx}}
\newcommand{\uyx}{u_{yx}}
\newcommand{\uzx}{u_{zx}}

\newcommand{\uxy}{u_{xy}}
\newcommand{\uyy}{u_{yy}}
\newcommand{\uzy}{u_{zy}}

\newcommand{\uxz}{u_{xz}}
\newcommand{\uyz}{u_{yz}}
\newcommand{\uzz}{u_{zz}}

\newcommand{\Ld}{\mathcal{L}}

\section{}
\begin{statement}
	Find the Euler-Lagrange equation associated with the functional
	\beq
		J[u(x, y, z)] = \intr \sqrt{1 + \ux^2 + \uy^2 + \uz^2} \dr,
	\eeq
	where $R$ is a region in three-dimensional space.
\end{statement}

\begin{solution}
	We will assume $u(x, y, z)$ has explicit values on the boundary of $R$, $\dr$.  By the definition of the action,
	\beq
		J[u] = \intr \Ld \dr \implies \Ld = \sqrt{1 + \ux^2 + \uy^2 + \uz^2}.
	\eeq
	In general, the Euler-Lagrange equation is
	\beqn \label{el1}
		0 = \pdv{\Ld}{u} - \pdv{}{x} \pdv{\Ld}{\ux} - \pdv{}{y} \pdv{\Ld}{\uy} - \pdv{}{z} \pdv{\Ld}{\uz}.
	\eeqn
	Here,
	\begin{align*}
		\pdv{\Ld}{u} &= 0, &
		\pdv{\Ld}{\ux} &= \pdv{\Ld}{\ux^2} \pdv{\ux^2}{\ux} = \frac{\ux}{\sqrt{1 + \ux^2 + \uy^2 + \uz^2}} = \frac{\ux}{\Ld} &
		\pdv{\Ld}{\uy} &= \frac{\uy}{\Ld}, &
		\pdv{\Ld}{\uz} &= \frac{\uz}{\Ld}.
	\end{align*}
	For the $\pdv*{}{x}$ term of \refeq{el1},
	\beq
		\pdv{}{x} \pdv{\Ld}{\ux} = \pdv{}{x} \frac{\ux}{\Ld} = \pdv{\ux}{x} \pdv{}{\ux} \frac{\ux}{\Ld} + \pdv{\uy}{x} \pdv{}{\uy} \frac{\ux}{\Ld} + \pdv{\uz}{x} \pdv{}{\uz} \frac{\ux}{\Ld}
	\eeq
	where
	\begin{align}
		\pdv{}{\ux} \frac{\ux}{\Ld} &= \frac{1}{\Ld^2} \left(\pdv{\ux}{\ux} \Ld - \ux \pdv{\Ld}{\ux} \right) = \frac{1}{\Ld^2} \left( \Ld - \ux \frac{\ux}{\Ld} \right) = \frac{\Ld^2 - \ux^2}{\Ld^3}, \label{uxx} \\
		\pdv{}{\uy} \frac{\ux}{\Ld} &= \frac{1}{\Ld^2} \left(\pdv{\ux}{\uy} \Ld - \ux \pdv{\Ld}{\uy} \right) = -\frac{\ux \uy}{\Ld^3}, \\
		\pdv{}{\uz} \frac{\ux}{\Ld} &= -\frac{\ux \uz}{\Ld^3}, \label{uxz}
	\end{align}
	Generalizing \refeq{uxx}--\refeq{uxz} to the $\pdv*{}{y}$ and $\pdv*{}{z}$ terms,
	\begin{align*}
		\pdv{}{x} \pdv{\Ld}{\ux} &= \uxx \frac{\Ld^2 - \ux^2}{\Ld^3} - \uyx \frac{\ux \uy}{\Ld^3} - \uzx \frac{\ux \uz}{\Ld^3}, \\
		\pdv{}{y} \pdv{\Ld}{\uy} &= \uyy \frac{\Ld^2 - \uy^2}{\Ld^3} - \uxy \frac{\ux \uy}{\Ld^3} - \uzy \frac{\uy \uz}{\Ld^3}, \\
		\pdv{}{z} \pdv{\Ld}{\uz} &= \uzz \frac{\Ld^2 - \uz^2}{\Ld^3} - \uxz \frac{\ux \uz}{\Ld^3} - \uyz \frac{\uy \uz}{\Ld^3}.
	\end{align*}
	Then, assuming $\uxy = \uyx$, $\uyz = \uzy$, and $\uxz = \uzx$, \refeq{el1} becomes
	\begin{align*}
		0 &= \uxx (\Ld^4 - \ux^2) + \uyy (\Ld^4 - \uy^2) + \uzz (\Ld^4 - \uz^2) - 2 \uxy \ux \uy - 2 \uyz \uy \uz - 2 \uxz \ux \uz \\
		&= (\uxx + \uyy + \uzz) (1 + \ux^2 + \uy^2 + \uz^2) - \uxx \ux^2 - \uyy \uy^2 - \uzz \uz^2 - 2 \uxy \ux \uy - 2 \uyz \uy \uz - 2 \uxz \ux \uz \\
		&= \uxx (1 + \uy^2 + \uz^2) + \uyy (1 + \ux^2 + \uz^2) + \uzz(1 + \ux^2 + \uy^2) - 2 \uxy \ux \uy - 2 \uyz \uy \uz - 2 \uxz \ux \uz.
	\end{align*}
\end{solution}


\newcommand{\intdr}{\int_{\partial R}}

\section{Plate vibrations (preliminaries)} \label{2}

\begin{statement}
	Start from Green's theorem
	\beqn \label{green}
		\intr \left( \pdv{Q}{x} - \pdv{P}{y} \right) \dd{x} \dd{y} = \intdr (P \dd{x} + Q \dd{y}),
	\eeqn
	where $R$ is the region in the $xy$ plane spanned by the plate, and $\dr$ its boundary.
\end{statement}

\begin{problem}
	Show that
	\beqn \label{greenxx}
		\intr \phi \pdv[2]{\psi}{x} \dd{x} \dd{y} = \intr \psi \pdv[2]{\phi}{x} \dd{x} \dd{y} + \intdr \left( \phi \pdv{\psi}{x} - \psi \pdv{\phi}{x} \right) \dd{y}.
	\eeqn
\end{problem}

\begin{solution}
	In \refeq{green}, let
	\begin{align*}
		Q &= \phi \pdv{\psi}{x} - \psi \pdv{\phi}{x}, &
		P &= 0.
	\end{align*}
	Then
	\begin{align*}
		\pdv{Q}{x} &= \pdv{\phi}{x} \pdv{\psi}{x} + \phi \pdv[2]{\psi}{x} - \pdv{\psi}{x} \pdv{\phi}{x} - \psi \pdv[2]{\phi}{x} = \phi \pdv[2]{\psi}{x} - \psi \pdv[2]{\phi}{x}, &
		\pdv{P}{y} &= 0.
	\end{align*}
	Making these substitutions into \refeq{green} gives
	\begin{align*}
		\intr \left( \phi \pdv[2]{\psi}{x} - \psi \pdv[2]{\phi}{x} \right) \dd{x} \dd{y} &= \intdr \left( \phi \pdv{\psi}{x} - \psi \pdv{\phi}{x} \right) \dd{y} \\
		\iff \intr \phi \pdv[2]{\psi}{x} \dd{x} \dd{y} &= \intr \psi \pdv[2]{\phi}{x} \dd{x} \dd{y} + \intdr \left( \phi \pdv{\psi}{x} - \psi \pdv{\phi}{x} \right) \dd{y}
	\end{align*}
	as desired. \qed
\end{solution}

\begin{problem}
	Work out analogous expressions for
	\begin{align}
		&\intr \phi \pdv[2]{\psi}{y} \dd{x} \dd{y}, \label{2bfirst} \\
		&\intr \phi \pdv{\psi}{x}{y} \dd{x} \dd{y}. \label{2bsecond}
	\end{align}
\end{problem}

\begin{solution}
	For \refeq{2bfirst}, let
	\begin{align*}
		Q &= 0, &
		P &= \psi \pdv{\phi}{y} - \phi \pdv{\psi}{y},
	\end{align*}
	in \refeq{green}.  Then, similarly to the proof for \refeq{greenxx},
	\begin{align*}
		\pdv{Q}{x} &= 0, &
		\pdv{P}{y} &= \psi \pdv[2]{\phi}{y} - \phi \pdv[2]{\psi}{y}.
	\end{align*}
	Substituting into \refeq{green},
	\begin{align}
		\intr \left( \phi \pdv[2]{\psi}{y} - \psi \pdv[2]{\phi}{y} \right) \dd{x} \dd{y} &= \intdr \left( \psi \pdv{\phi}{y} - \phi \pdv{\psi}{y} \right) \dd{x} \notag \\
		\iff \intr \phi \pdv[2]{\psi}{y} \dd{x} \dd{y} &= \intr \psi \pdv[2]{\phi}{y} \dd{x} \dd{y} + \intdr \left( \psi \pdv{\phi}{y} - \phi \pdv{\psi}{y} \right) \dd{y}. \label{greenyy}
	\end{align}
	
	For \refeq{2bsecond}, let
	\begin{align*}
		2Q &= \phi \pdv{\psi}{y} - \psi \pdv{\phi}{y}, &
		2P &= \psi \pdv{\phi}{x} - \phi \pdv{\psi}{x}.
	\end{align*}
	Then
	\begin{align*}
		2 \pdv{Q}{x} &= \pdv{\phi}{x} \pdv{\psi}{y} + \phi \pdv{\psi}{x}{y} - \pdv{\psi}{x} \pdv{\phi}{y} - \psi \pdv{\phi}{x}{y} = \phi \pdv{\psi}{x}{y} - \psi \pdv{\phi}{x}{y}, \\
		2 \pdv{P}{y} &= \pdv{\psi}{y} \pdv{\phi}{x} + \psi \pdv{\phi}{x}{y} - \pdv{\phi}{y} \pdv{\psi}{x} - \phi \pdv{\psi}{x}{y} = \psi \pdv{\phi}{x}{y} - \phi \pdv{\psi}{x}{y}.
	\end{align*}
	Substituting into \refeq{green}, we have
	\begin{align}
		\frac{1}{2} \intr &\left( \phi \pdv{\psi}{x}{y} - \psi \pdv{\phi}{x}{y} - \psi \pdv{\phi}{x}{y} + \phi \pdv{\psi}{x}{y} \right) \dd{x} \dd{y} = \frac{1}{2} \intdr \left( \psi \pdv{\phi}{x} - \phi \pdv{\psi}{x} \right) \dd{x} + \frac{1}{2} \intdr \left( \phi \pdv{\psi}{y} - \psi \pdv{\phi}{y} \right) \dd{y} \notag \\
		&\iff \intr \phi \pdv{\psi}{x}{y} \dd{x} \dd{y} = \intr \psi \pdv{\phi}{x}{y} \dd{x} \dd{y} + \frac{1}{2} \intdr \left( \psi \pdv{\phi}{x} - \phi \pdv{\psi}{x} \right) \dd{x} + \frac{1}{2} \intdr \left( \phi \pdv{\psi}{y} - \psi \pdv{\phi}{y} \right) \dd{y}. \label{greenxy}
	\end{align}
\end{solution}


\newcommand{\dS}{\delta S}
\newcommand{\DS}{\Delta S}
\newcommand{\eps}{\epsilon}
\newcommand{\intt}{\int_{t_0}^{t_1}}
\newcommand{\kaq}{\kappa_1}
\newcommand{\lapw}{\nabla^4}
\newcommand{\Pu}{P(u)}
\newcommand{\Mu}{M(u)}
\newcommand{\uo}{u^0}

\newcommand{\ut}{u_t}
\newcommand{\utt}{u_{tt}}
\newcommand{\uxxx}{u_{xxx}}
\newcommand{\uxxxx}{u_{xxxx}}
\newcommand{\uxxyy}{u_{xxyy}}
\newcommand{\uxxy}{u_{xxy}}
\newcommand{\uxyy}{u_{xyy}}
\newcommand{\uyyy}{u_{yyy}}
\newcommand{\uyyyy}{u_{yyyy}}

\newcommand{\psit}{\psi_t}
\newcommand{\psix}{\psi_{x}}
\newcommand{\psiy}{\psi_{y}}
\newcommand{\psixx}{\psi_{xx}}
\newcommand{\psixy}{\psi_{xy}}
\newcommand{\psiyy}{\psi_{yy}}
\newcommand{\psixxy}{\psi_{xxy}}
\newcommand{\psixyy}{\psi_{xyy}}
\newcommand{\psil}{\psi_\ell}
\newcommand{\psin}{\psi_n}

\newcommand{\inttr}{\intt \!\!\intr}
\newcommand{\inttdr}{\intt \!\!\intdr}
\newcommand{\drt}{\dd{x} \dd{y} \dd{t}}
\newcommand{\dlt}{\dd{\ell} \dd{t}}

\newcommand{\vn}{\vec{\hat{n}}}
\newcommand{\vl}{\boldsymbol{\hat{\ell}}}
\newcommand{\xn}{x_n}
\newcommand{\yn}{y_n}
\newcommand{\xl}{x_\ell}
\newcommand{\yl}{y_\ell}
\newcommand{\lap}{\nabla^2}

\section{Plate vibrations}

\begin{statement}
	Start with the action for a vibrating plate whose potential energy is dominated by bending,
	\beqn \label{action3}
		S[u(x, y, t)] = \frac{1}{2} \inttr \left\{ \rho \ut^2 - \kaq \left[ (\uxx^2 + \uyy^2) - 2 (1 - \mu) (\uxx \uyy - \uxy^2) \right] \right\} \dd{x} \dd{y} \dd{t},
	\eeqn
	where $\rho$ is the mass density per unit area, $\kaq$ has the dimension of energy and is sometimes called flexural rigidity, and $\mu$ is a dimensionless material constant called Poisson's ratio.  For isotropic material, $\mu = 1/4$.  Notice that there is \emph{no} external bending moment applied to the plate boundary.  There is also \emph{no} external forcing.
\end{statement}

\begin{problem}
Using the results of problem~\ref{2}, show that the variation generated by going from a solution $\uo$ to $\uo + \eps\psi$ has the form
	\beqn \label{daction3}
		\dS = \eps \inttr \left(-\rho \utt - \kaq \lapw u \right) \psi \drt + \eps \inttdr \left( \Pu \psi + \Mu \pdv{\psi}{n} \right) \dlt.
	\eeqn
	Specify $\Pu$ and $\Mu$.
\end{problem}

\begin{solution}
	Making the substitution $u \mapsto u + \eps \psi$ into \refeq{action3},
	\begin{align*}
		S[u + \eps \psi] &= \inttr \left\{ \frac{\rho}{2} (\ut + \eps \psit)^2 - \frac{\kaq}{2} \left[ (\uxx + \eps \psixx)^2 + (\uyy + \eps \psiyy)^2 \right] \right\} \drt \\
		&\phantom{mmm} + \kaq (1 - \mu) \inttr \left[ (\uxx + \eps \psixx) (\uyy + \eps \psiyy) - (\uxy + \eps \psixy)^2 \right] \drt \\
		&= \inttr \left[ \frac{\rho}{2} (\ut^2 + 2 \eps \ut \psit + \eps^2 \psit^2) - \frac{\kaq}{2} (\uxx^2 + 2 \eps \uxx \psixx + \eps^2 \psixx^2 + \uyy^2 + 2 \eps \uyy \psiyy + \eps^2 \psiyy^2) \right] \drt \\
		&\phantom{mmm} + \kaq (1 - \mu) \inttr (\uxx \uyy + \eps \uxx \psiyy + \eps \uyy \psixx + \eps^2 \psixx \psiyy - \uxy^2 - 2 \eps \uxy \psixy - \eps^2 \psixy^2 ) \drt.
	\end{align*}
	Then
	\begin{align*}
		\DS &= S[u + \eps \psi] - S[u] \\
		&=  \inttr \left[ \frac{\rho}{2} (2 \eps \ut \psit + \eps^2 \psit^2) - \frac{\kaq}{2} (2 \eps \uxx \psixx + \eps^2 \psixx^2 + 2 \eps \uyy \psiyy + \eps^2 \psiyy^2) \right] \drt \\
		&\phantom{mmmmmmmmm} + \kaq (1 - \mu) \inttr (\eps \uxx \psiyy + \eps \uyy \psixx + \eps^2 \psixx \psiyy - 2 \eps \uxy \psixy - \eps^2 \psixy^2 ) \drt,
	\end{align*}
	and so, dropping terms of $\order{\eps^2}$,
	\beqn \label{dS1}
		\dS = \eps \inttr \left\{ \rho \ut \psit - \kaq \left[ \uxx \psixx + \uyy \psiyy - (1 - \mu) (\uxx \psiyy + \uyy \psixx - 2 \uxy \psixy) \right] \right\} \drt.
	\eeqn
	
	For the first term in the integrand of \refeq{dS1}, using the product rule of differentiation yields
	\beq
		\ut \psit = \pdv{}{t} (\ut \psi) - \utt \psi.
	\eeq
	For the second two terms, we may apply what was proven in problem~\ref{2}.  Letting $\phi \mapsto \uxx$ and $\psi \mapsto \psi$ in \refeq{greenxx} and \refeq{greenyy}, we have
	\begin{align*}
		\inttr \uxx \psixx \drt &= \inttr \psi \uxxxx \drt + \inttdr \left( \uxx \psix - \psi \uxxx \right) \dd{y} \dd{t}, \\
		\inttr \uxx \psiyy \drt &= \inttr \psi \uxxyy \drt - \inttdr \left( \uxx \psiy - \psi \uxxy \right) \dd{x} \dd{t}.
	\end{align*}
	Now with $\phi \mapsto \uyy$,
	\begin{align*}
		\inttr \uyy \psixx \drt &= \inttr \psi \uxxyy \drt + \inttdr \left( \uyy \psix - \psi \uxyy \right) \dd{y} \dd{t}, \\
		\inttr \uyy \psiyy \drt &= \inttr \psi \uyyyy \drt - \inttdr \left( \uyy \psiy - \psi \uyyy \right) \dd{x} \dd{t}.
	\end{align*}
	Finally, with $\phi \mapsto \uxy$ and $\psi \mapsto \psi$ in \refeq{greenxy}, we have
	\begin{align*}
		\inttr \uxy \psixy \drt = \inttr \psi \uxxyy \drt - \frac{1}{2} \inttdr &\left( \uxy \psix - \psi \uxxy \right) \dd{x} \dd{t} \\
		&+ \frac{1}{2} \inttdr \left( \uxy \psiy - \psi \uxyy \right) \dd{y} \dd{t}.
	\end{align*}
	Making these substitutions into \refeq{dS1},
	\begin{align*}
		\frac{\dS}{\eps} &= \inttr \psi \left\{ -\rho \utt - \kaq \left[ \uxxxx + \uyyyy - (1 - \mu) (\uxxyy + \uxxyy - 2 \uxxyy) \right] \right\} \drt + \rho \inttr \pdv{}{t} (\ut \psi) \drt \notag \\
		&\phantom{=\ }- \kaq \inttdr \left[ \uxx \psix - \psi \uxxx - (1 - \mu) (\uyy \psix - \psi \uxyy - \uxy \psiy + \psi \uxyy) \right] \dd{y} \dd{t} \notag \\
		&\phantom{=\ } + \kaq \inttdr \left[ \uyy \psiy - \psi \uyy - (1 - \mu) (\uxx \psiy - \psi \uxxy - \uxy \psix + \psi \uxxy) \right] \dd{x} \dd{t} \\
		&= \inttr \psi \left\{ -\rho \utt - \kaq \left[ \uxxxx + \uyyyy - (1 - \mu) (\uxxyy + \uxxyy - 2 \uxxyy) \right] \right\} \drt + \rho \intr \bigg[\ut \psi\bigg]_{t_0}^{t_1} \dd{x} \dd{y} \\
		&\phantom{=\ } - \kaq \inttdr \left[ \uxx \psix - \psi \uxxx - (1 - \mu) (\uyy \psix - \uxy \psiy) \right] \dd{y} \dd{t} \\
		&\phantom{=\ } + \kaq \inttdr \left[ \uyy \psiy - \psi \uyyy - (1 - \mu) (\uxx \psiy - \uxy \psix) \right] \dd{x} \dd{t}.
	\end{align*}
	Note that
	\beqn \label{lapla}
		\lapw u = \pdv[4]{u}{x} + 2 \frac{\partial^4 u}{\partial x^2 \partial y^2} + \pdv[4]{u}{y} = \pdv[4]{u}{x} + \pdv[4]{u}{y},
	\eeqn
	for a real solution $u$.  Note also that
	\beq
		\rho \intr \bigg[\ut \psi\bigg]_{t_0}^{t_1} \dd{x} \dd{y} = 0
	\eeq
	because $\psi(t_0) = \psi(t_1) = 0$.  Then
	\beqn \label{final}
		\frac{\dS}{\eps} = \inttr \psi (-\rho \utt - \kaq \lapw u) \drt + \kaq \intt L \dd{t},
	\eeqn
	where we have defined $L$ as all of the surface integrals.

	Define $\vn = (\xn, \yn)$ as the unit vector normal to the surface and $\vl = (\xl, \yl)$ as the unit vector tangent to the surface.  Then we have the directional derivatives
	\begin{align} \label{ders}
		\pdv{}{n} &= \vn \cdot \nabla = \xn \pdv{}{x} + \yn \pdv{}{y}, &
		\pdv{}{\ell} &= \vl \cdot \nabla = -\xl \pdv{}{x} + \yl \pdv{}{y}.
	\end{align}
	For the surface integrals, we have the differentials
	\begin{align} \label{diffs}
		\dd{x} &= \yn \dd{\ell}, &
		\dd{y} &= -\xn \dd{\ell}.
	\end{align}
	Using \refeq{diffs}, we can rewrite $L$ in terms of $\dd{\ell}$:
	\beq
		L = \intdr \left[ \xn \uxx \psix - \xn \psi \uxxx + \yn \uyy \psiy - \yn \psi \uyyy + (1 - \mu) (-\xn \uyy \psix + \xn \uxy \psiy - \yn \uxx \psiy + \yn \uxy \psix) \right] \dd{\ell}.
	\eeq
	Applying \refeq{ders} to $\psi$, we can rewrite this in terms of $\psil$ and $\psiy$:
\begin{align*}
		L &= \intdr [ - (\xn \psi \uxxx + \yn \psi \uyyy) + \xn^2 \uxx \psin + \xn \yl \uxx \psil - \xl \yn \uyy \psil + \yn^2 \uyy \psin \\
		&\phantom{mmmmmmmmmm} + (1 - \mu) (-\xn^2 \uyy \psin - \xn \yl \uyy \psil - \xn \xl \uxy \psil + \xn \yn \uxy \psin \\
		&\phantom{mmmmmmmmmmmmmmmmmmmm} + \xl \yn \uxx \psil - \yn^2 \uxx \psin + \xn \yn \uxy \psin - \yl \yn \uxy \psil) \dd{\ell} \\
		&= \intdr \{ - (\xn \psi \uxxx + \yn \psi \uyyy) + \psin [\xn^2 \uxx + \yn^2 \uyy + (1 - \mu) (-\xn^2 \uyy + \xn \yn \uxy - \yn^2 \uxx + \xn \yn \uxy)] \\
		&\phantom{mmmmmmmmmm} + \psil [\xn \yl \uxx - \xl \yn \uyy + (1 - \mu) (-\xn \yl \uyy - \xn \xl \uxy + \xl \yn \uxx + \yn \yl \uxy)] \} \dd{\ell}.
	\end{align*}
	From the product rule,
	\beq
		\intdr F \psil \dd{\ell} = \bigg[ F \psi \bigg]_\ell - \intdr \psi \left( \pdv{F}{s} \right) \dd{\ell} = -\intdr \psi \left( \pdv{F}{\ell} \right) \dd{\ell},
	\eeq
	because $\psi = 0$ on the surface $\ell$.  Then we have
	\begin{align}
		L &= \intdr \bigg\{ \psi \left[\xn \uxxx + \yn \uyyy + \pdv{}{\ell} [\xn \yl \uxx - \xl \yn \uyy + (1 - \mu) (-\xn \yl \uyy - \xn \xl \uxy + \xl \yn \uxx + \yn \yl \uxy)] \right] \\
		&\phantom{mmmmmmmmmm} - \pdv{\psi}{n} [\xn^2 \uxx - \yn^2 \uyy + (1 - \mu) (\xn^2 \uyy - 2 \xn \yn \uxy + \yn^2 \uxx)] \bigg\} \dd{\ell} \notag \\
		&= \frac{1}{\kaq} \intdr [ \psi \Pu + \pdv{\psi}{n} \Mu ] \dd{\ell}. \label{L}
	\end{align}
	Applying \refeq{ders} to $\Pu$, we have
	\begin{align}
		\frac{\Pu}{\kaq} &= \xn \uxxx + \yn \uyyy + \pdv{}{\ell} [\xn \yl \uxx - \xl \yn \uyy + (1 - \mu) (-\xn \yl \uyy - \xn \xl \uxy + \xl \yn \uxx + \yn \yl \uxy)] \notag \\
		&= \xn^2 \pdv{\uxx}{n} + \xn \yl \pdv{\uxx}{\ell} + \yn^2 \pdv{\uyy}{n} - \xl \yn \pdv{\uyy}{\ell} - \xn \yl \pdv{\uxx}{\ell} + \xl \yn \pdv{\uyy}{\ell} \notag \\
		&\phantom{mmmmmmmmmm} + \pdv{}{\ell} [(1 - \mu) (-\xn \yl \uyy - \xn \xl \uxy + \xl \yn \uxx + \yn \yl \uxy)] \notag \\
		&= \pdv{}{n} \lap u + (1 - \mu) \pdv{}{\ell} (\xn \yn \uxx + (\xn \yn \uyy + \xl \yn) \uxy + \yn \yl \uxy). \label{\Pu}
	\end{align}
	For $\Mu$, we have
	\beqn \label{Mu}
		-\frac{\Mu}{\kaq} = \xn^2 \uxx + \yn^2 \uyy + (1 - \mu) (\xn^2 \uyy + 2 \xn \yn \uxy + \yn^2 \uxx)
	\eeqn
	
	Finally, combining \refeq{final} and \refeq{L}, we have shown that
	\beqn
		\dS = -\eps \inttr \left(\rho \utt + \kaq \lapw u \right) \psi \drt + \eps \inttdr \left( \Pu \psi + \Mu \pdv{\psi}{n} \right) \dlt,
	\eeqn
	with $\Pu$ given by \refeq{Pu} and $\Mu$ given by \refeq{Mu}. \qed
	
	(Honestly, I shuffled a lot of signs around in this problem to make things work.  But the main ideas are all present.)
\end{solution}

\newcommand{\ello}{\ell^0}

\begin{problem}
	Finally, derive the Euler-Lagrange equation and the associated boundary conditions.
\end{problem}

\begin{solution}
	We begin by making the strong assumption that the boundary of the plate remains fixed.  Mathematically, we assume that the solution $\uo$ does not vary on the boundary of the plate, denoted by $\ell \in \dR$.  We further assume that the edges of the plate cannot move; that is, the first derivative of $\uo$ normal to the plate does not vary either.  These assumptions constrain $\psi = \psi(\ell, t)$:
	\begin{align*}
		\uo(\ell, t) = 0 &\implies \psi(\ell, t) = 0, &
		\pdv{\uo(\ell, t)}{n} = 0 &\implies \pdv{\psi(\ell, t)}{n} = 0.
	\end{align*}
	Making these assumptions, the entire surface integral of \refeq{daction3} vanishes, and we are left with
	\beq
		\dS = -\eps \inttr \left(\rho \utt + \kaq \lapw u \right) \psi \drt.
	\eeq
	By Hamilton's principle, this gives us
	\beq \label{el3}
		0 = \rho \utt + \kaq \lapw u
	\eeq
	as the Euler-Lgrange equation.
	
	Now we use \refeq{el3} as our assumption and return to \refeq{daction3}, which becomes
	\beq
		\dS = \eps \inttdr \left( \Pu \psi + \Mu \pdv{\psi}{n} \right) \dlt.
	\eeq
	Once again invoking Hamilton's principle, we find the boundary conditions
	\begin{align}
		\Mu &= 0, & \Pu &= 0.
	\end{align}
\end{solution}


\newcommand{\lam}{\lambda}
\newcommand{\vrt}{v(r, \theta)}
\newcommand{\gt}{g(t)}
\newcommand{\fnr}{f_n(r)}

\section{Vibrations of a circular disk}

\begin{statement}
	The only scenario in which plate vibrations can be described analytically in terms of known functions is a circular disk.  Work with polar coordinates $(r, \theta)$, the Euler-Lagrange equation
	\beqn \label{el4}
		\utt + \lam \lapw u = 0,
	\eeqn
	and the boundary conditions
	\begin{align} \label{bc4}
		u &= 0, & \pdv{u}{n} &= 0.
	\end{align}
\end{statement}


\newcommand{\Cq}{C_1}
\newcommand{\Cw}{C_2}

\newcommand{\Dq}{D_1}
\newcommand{\Dw}{D_2}

\newcommand{\sqmu}{\sqrt{\mu}}

\begin{problem}
	Show that this problem reduces to an eigenvalue problem if we assume that $u(r, \theta, t)$ is separable:
	\beqn \label{udef}
		u = \vrt\, \gt.
	\eeqn
	Write down the general form of $g(t)$.
\end{problem}

\begin{solution}
	Substituting the ansatz \refeq{udef} into \refeq{el4}, we have
	\beqn \label{pde4}
		v \pdv[2]{g}{t} + \lambda g \,\lapw v = 0 \implies \frac{1}{g} \pdv[2]{g}{t} = -\lambda \frac{1}{v} \lapw v \equiv -\lambda^2
	\eeqn
	where we have fixed $\lambda^2$.  We may then separate \refeq{pde4} into two differential equations,
	\begin{align}
		\lapw v - \lambda v &= 0, \label{evp} \\
		\pdv[2]{g}{t} + \lambda^2 g &= 0. \label{time}
	\end{align}
	The eigenvalue problem is \refeq{evp}, which we may solve for the eigenvalues $\lambda_n$ and obtain the eigenfunctions $v_n(r, \theta)$.  Then we simply feed $\lambda_n$ into \refeq{time} to obtain $g_n(t)$, which have the general form
	\beqn \label{gt}
		g(t) = \Cq + \Cw t - \frac{\lambda^2}{6} t^3,
	\eeqn
	where $\Cq$ and $\Cw$ are arbitrary constants.  Finally, the solutions to \refeq{el4} are $u_n(r, \theta, t) = v_n(r, \theta) \, g_n(t)$.
\end{solution}

\begin{problem}
	Now consider the eigenvalue problem
	\beqn \label{evpb}
		(\lapw - k^4) \vrt = 0,
	\eeqn
	with $\lam$ set to be $k^4$.  Notice that it factors into
	\beqn \label{sep}
		(\lap - k^2)(\lap + k^2) \vrt = 0,
	\eeqn
	with
	\beq
		\lap = \pdv[2]{}{r} + \frac{1}{r} \pdv{}{r} + \frac{1}{r^2} \pdv[2]{}{\theta}.
	\eeq
	Since the disk is circular, we expect the vibration modes to be periodic in $\theta$.  This suggests the ansatz
	\beqn \label{ansatz4}
		v = \sum_{n = -\infty}^\infty \fnr \, e^{in\theta}.
	\eeqn
	Obtain the ODE governing $\fnr$.
\end{problem}

\newcommand{\fm}{f_m}
\newcommand{\fp}{f_p}

\newcommand{\km}{k_m}
\newcommand{\kp}{k_m}

\newcommand{\eint}{e^{in\theta}}
\newcommand{\eimt}{e^{im\theta}}
\newcommand{\eipt}{e^{ip\theta}}

\begin{solution}
	Firstly, note that
		\beq
		\lapw = \left( \pdv[2]{}{r} + \frac{1}{r} \pdv{}{r} + \frac{1}{r^2} \pdv[2]{}{\theta} \right)^2 = \pdv[4]{}{r} + \frac{2}{r} \pdv[3]{}{r} + \frac{1}{r^2} \pdv[2]{}{r} + \frac{2}{r^2} \pdv[2]{}{r} \pdv[2]{}{\theta} + \frac{2}{r^3} \pdv{}{r} \pdv[2]{}{\theta} + \frac{1}{r^4} \pdv[4]{}{\theta}.
	\eeq
	Substituting the ansatz of \refeq{ansatz4} into \refeq{evpb} yields
	\begin{align*}
		k^4 \fnr \,\eint &= -\lapw \fnr \,\eint \\
		&= \left( \pdv[4]{}{r} + \frac{2}{r} \pdv[3]{}{r} + \frac{1}{r^2} \pdv[2]{}{r} + \frac{2}{r^2} \pdv[2]{}{r} \pdv[2]{}{\theta} + \frac{2}{r^3} \pdv{}{r} \pdv[2]{}{\theta} + \frac{1}{r^4} \pdv[4]{}{\theta} \right) \fnr \,\eint \\
		&= \eint \left( \pdv[4]{}{r} + \frac{2}{r} \pdv[3]{}{r} + \frac{1}{r^2} \pdv[2]{}{r} - \frac{2n^2}{r^2} \pdv[2]{}{r} - \frac{2n^2}{r^3} \pdv{}{r} + \frac{n^4}{r^4} \right) \fnr.
	\end{align*}
	Dividing out $\eint$, we have
	\beq
		k^4 \fnr = \pdv[4]{\fnr}{r} + \frac{2}{r} \pdv[3]{\fnr}{r} + \frac{1 - 2n^2}{r^2} \pdv[2]{\fnr}{r} - \frac{2n^2}{r^3} \pdv{\fnr}{r} + \frac{n^4}{r^4} \fnr
	\eeq
	as the ODE governing $\fnr$.
\end{solution}

\begin{problem}
	What are the appropriate boundary conditions on $\fnr$?
\end{problem}

\begin{solution}
	From \refeq{udef} and \refeq{ansatz4}, the solution $u$ is defined
	\beq
		u = \vrt \,\gt = \gt \sum_{n = -\infty}^\infty \fnr e^{i n \theta}.
	\eeq
	From \refeq{bc4},
	\begin{align}
		u = 0 \implies v = 0 \implies \fnr = 0, \\
		\pdv{u}{n} = 0 \implies \pdv{v}{n} = 0 \implies \pdv{\fnr}{r} = 0,
	\end{align}
	for all $n \in (-\infty, \infty)$ on the boundry $\dR$ of the plate.  Note that $\pdv*{}{n}$ is the normal derivative.
\end{solution}

\newcommand{\Ub}{U_b}
\newcommand{\vvu}{\vec{u}}
\newcommand{\AT}{A^T}
\newcommand{\uq}{u_1}
\newcommand{\uw}{u_2}

\newcommand{\uqx}{{u_1}_x}
\newcommand{\uqy}{{u_1}_y}
\newcommand{\uqxx}{{u_1}_{xx}}
\newcommand{\uqxy}{{u_1}_{xy}}
\newcommand{\uqyy}{{u_1}_{yy}}
\newcommand{\uwx}{{u_2}_x}
\newcommand{\uwy}{{u_2}_y}
\newcommand{\uwxx}{{u_2}_{xx}}
\newcommand{\uwxy}{{u_2}_{xy}}
\newcommand{\uwyy}{{u_2}_{yy}}

\section{Big brother}
\begin{problem}
	Let $\vvu(x, y) = [\uq(x, y), \uw(x, y)]$ be the \emph{unknown} two-dimensional warp map corresponding to a grayscale photograph.  Find the Euler-Lagrange equations associated with the elastic energy functional
	\beq
		\Ub[\vvu] = \intr \left[ \lam \tr((A + \AT)^2) + \mu \tr(A) \tr(\AT) \right] \dd{x} \dd{y},
	\eeq
	where $\lambda$ and $\mu$ are elastic constant, the deviation $A$ is given by
	\beq
		A = \mqty[ \pdv*{\uq}{x} & \pdv*{\uq}{y} \\
					\pdv*{\uw}{x} & \pdv*{\uw}{y} ],
	\eeq
	and $R$ is the region spanned by a photograph.
\end{problem}

\begin{solution}
	Firstly, note that
	\beq
		\AT = \mqty[ \pdv*{\uq}{x} & \pdv*{\uw}{x} \\
					\pdv*{\uw}{y} & \pdv*{\uw}{y} ],
	\eeq
	so
	\begin{align*}
		A + \AT &= \mqty[ 2 \pdv*{\uq}{x} & \pdv*{\uq}{y} + \pdv*{\uw}{x} \\
					\pdv*{\uw}{x} + \pdv*{\uq}{y} & 2 \pdv*{\uw}{y} ], \\
		(A + \AT)^2 &= \mqty[ 4 (\pdv*{\uq}{x})^2 + (\pdv*{\uq}{y} + \pdv*{\uw}{x})^2 & \\
					 & 4 (\pdv*{\uw}{y})^2 + (\pdv*{\uq}{y} + \pdv*{\uw}{x})^2 ],
	\end{align*}
	where only the diagonal terms of $(A + \AT)^2$ are of interest.  Then
	\begin{align*}
		\tr((A + \AT)^2) &= 4 \left( \pdv{\uq}{x} \right)^2 + 2 \left( \pdv{\uq}{y} + \pdv{\uw}{x} \right)^2 + 4 \left( \pdv{\uw}{y} \right)^2 \\
		&= 4 \left( \pdv{\uq}{x} \right)^2 + 2 \left( \pdv{\uq}{y} \right)^2 + 4 \pdv{\uq}{y} \pdv{\uw}{x} + 2 \left( \pdv{\uw}{x} \right)^2 + 4 \left( \pdv{\uw}{y} \right)^2, \\
		\tr(A) \tr(\AT) &= \left( \pdv{\uq}{x} + \pdv{\uw}{y} \right)^2 = \left( \pdv{\uq}{x} \right)^2 + 2 \pdv{\uq}{x} \pdv{\uw}{y} + \left( \pdv{\uw}{y} \right)^2,
	\end{align*}
	and
	\beqn \label{ld5}
		\Ub[\vvu] = \intr \left\{ \lam \left[ 4 \left( \pdv{\uq}{x} \right)^2 + 2 \left( \pdv{\uq}{y} + \pdv{\uw}{x} \right)^2 + 4 \left( \pdv{\uw}{y} \right)^2 \right] + \mu \left( \pdv{\uq}{x} + \pdv{\uw}{y} \right)^2 \right\} \dd{x} \dd{y} \equiv \intr \Ld \dd{x} \dd{y},
	\eeqn
	where we have defined the Lagrangian density $\Ld$.
	
	We will have two Euler-Lagrange equations, one for each $\uq$ and $\uw$.  In general, they are given by
	\begin{align*}
		0 &= \pdv{\Ld}{\uq} - \pdv{}{x} \pdv{\Ld}{\uqx} - \pdv{}{y} \pdv{\Ld}{\uqy}, &
		0 &= \pdv{\Ld}{\uw} - \pdv{}{x} \pdv{\Ld}{\uwx} - \pdv{}{y} \pdv{\Ld}{\uwy},
	\end{align*}
	where $\uqx = \pdv*{\uq}{x}$, and so on.  From \refeq{ld5},
	\begin{align*}
		\pdv{\Ld}{\uq} &= 0, &
		\pdv{\Ld}{\uqx} &= 2 (4 \lambda + \mu) \uqx + 2 \mu \uwy, &
		\pdv{\Ld}{\uqy} &= 4 \lambda \uqy + 4 \lambda \uwx, \\
		\pdv{\Ld}{\uw} &= 0, &
		\pdv{\Ld}{\uwx} &= 4 \lambda \uwx + 4 \lambda \uqy, &
		\pdv{\Ld}{\uwy} &= 2 (4 \lambda + \mu) \uqy + 2 \mu \uwx.
	\end{align*}
	Then
	\begin{align*}
		\pdv{}{x} \pdv{\Ld}{\uqx} &= 2 (4 \lambda + \mu) \uqxx + 2 \mu \uwxy, &
		\pdv{}{y} \pdv{\Ld}{\uqy} &= 4 \lambda \uqyy + 4 \lambda \uwxy, \\
		\pdv{}{x} \pdv{\Ld}{\uwx} &= 4 \lambda \uwxx + 4 \lambda \uqxy, &
		\pdv{}{y} \pdv{\Ld}{\uwy} &= 2 (4 \lambda + \mu) \uwyy + 2 \mu \uqxy.
	\end{align*}
	So the Euler-Lagrange equations are
	\begin{align*}
		0 &= 2 (4 \lambda + \mu) \pdv[2]{\uq}{x} + 4 \lambda \pdv[2]{\uq}{y} + 2 (2\lambda + \mu) \pdv{\uw}{x}{y}, \\
		0 &= 2 (2 \lambda + \mu) \pdv{\uq}{x}{y} + 4 \lambda \pdv[2]{\uw}{x} + 2 (4 \lambda + \mu) \pdv[2]{\uw}{y},
	\end{align*}
	which are coupled.
\end{solution}



In writing these solutions, I consulted Gelfand and Fomin's \emph{Calculus of Variations} and Olmstead and Volpert's \emph{Differential Equations in Applied Mathemtics}.

\end{document}