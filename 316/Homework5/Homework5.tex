\documentclass[11pt]{article}
\usepackage{geometry, titlesec}
\usepackage[parfill]{parskip}
\usepackage{physics, amsfonts, amsthm}
\usepackage[cm]{fullpage}
\usepackage{fancyhdr}
\usepackage{enumitem}
\usepackage{xcolor, soul}
%\allowdisplaybreaks

\renewcommand{\thesubsection}{\thesection.\alph{subsection}}

\makeatletter
\renewcommand*\env@cases[1][1.2]{%
  \let\@ifnextchar\new@ifnextchar
  \left\lbrace
  \def\arraystretch{#1}%
  \array{@{}l@{\quad}l@{}}%
}
\makeatother
 
\renewcommand{\footrulewidth}{.2pt}
%\setlist[enumerate]{leftmargin=*}
\pagestyle{fancy}
\fancyhf{}
\lhead{\textbf{Physics 316 Homework 5}}
\rhead{Lacey Rainbolt}
\setlength{\headheight}{11pt}
\setlength{\headsep}{11pt}
\setlength{\footskip}{24pt}
\lfoot{\today}
\rfoot{\thepage}

\titleformat{\subsection}[runin]{\normalfont\large\bfseries}{\thesubsection}{1em}{}
\newcommand{\refeq}[1]{(\ref{#1})}

\newcommand{\beq}{\begin{equation*}}
\newcommand{\eeq}{\end{equation*}}

\newcommand{\beqn}{\begin{equation}}
\newcommand{\eeqn}{\end{equation}}

\newcommand{\blg}{\begin{align*}}
\newcommand{\elg}{\end{align*}}


\newenvironment{statement}
{
    \color{darkgray}
    \ignorespaces
}
{
%    \smallskip
}

\newenvironment{problem}
{
    \subsection{}
    \color{darkgray}
    \ignorespaces
}

\newenvironment{solution}
{
    \paragraph{Solution.}
    \ignorespaces
}
{
%    \bigskip
}



\begin{document}


\renewcommand{\vec}[1]{\mathbf{#1}}
\newcommand{\intr}{\int_R}
\newcommand{\dR}{\partial R}
\newcommand{\dr}{\dd{x} \dd{y} \dd{z}}
\newcommand{\ux}{u_x}
\newcommand{\uy}{u_y}
\newcommand{\uz}{u_z}

\newcommand{\uxx}{u_{xx}}
\newcommand{\uyx}{u_{yx}}
\newcommand{\uzx}{u_{zx}}

\newcommand{\uxy}{u_{xy}}
\newcommand{\uyy}{u_{yy}}
\newcommand{\uzy}{u_{zy}}

\newcommand{\uxz}{u_{xz}}
\newcommand{\uyz}{u_{yz}}
\newcommand{\uzz}{u_{zz}}

\newcommand{\Ld}{\mathcal{L}}

\section{}
\begin{statement}
	Find the Euler-Lagrange equation associated with the functional
	\beq
		J[u(x, y, z)] = \intr \sqrt{1 + \ux^2 + \uy^2 + \uz^2} \dr,
	\eeq
	where $R$ is a region in three-dimensional space.
\end{statement}

\begin{solution}
	We will assume $u(x, y, z)$ has explicit values on the boundary of $R$, $\dR$.  By the definition of the action,
	\beq
		J[u] = \intr \Ld \dr \implies \Ld = \sqrt{1 + \ux^2 + \uy^2 + \uz^2}.
	\eeq
	In general, the Euler-Lagrange equation is
	\beqn \label{el1}
		0 = \pdv{\Ld}{u} - \pdv{}{x} \pdv{\Ld}{\ux} - \pdv{}{y} \pdv{\Ld}{\uy} - \pdv{}{z} \pdv{\Ld}{\uz}.
	\eeqn
	Here,
	\begin{align*}
		\pdv{\Ld}{u} &= 0, &
		\pdv{\Ld}{\ux} &= \pdv{\Ld}{\ux^2} \pdv{\ux^2}{\ux} = \frac{\ux}{\sqrt{1 + \ux^2 + \uy^2 + \uz^2}} = \frac{\ux}{\Ld} &
		\pdv{\Ld}{\uy} &= \frac{\uy}{\Ld}, &
		\pdv{\Ld}{\uz} &= \frac{\uz}{\Ld}.
	\end{align*}
	For the $\pdv*{}{x}$ term of \refeq{el1},
	\beq
		\pdv{}{x} \pdv{\Ld}{\ux} = \pdv{}{x} \frac{\ux}{\Ld} = \pdv{\ux}{x} \pdv{}{\ux} \frac{\ux}{\Ld} + \pdv{\uy}{x} \pdv{}{\uy} \frac{\ux}{\Ld} + \pdv{\uz}{x} \pdv{}{\uz} \frac{\ux}{\Ld}
	\eeq
	where
	\begin{align}
		\pdv{}{\ux} \frac{\ux}{\Ld} &= \frac{1}{\Ld^2} \left(\pdv{\ux}{\ux} \Ld - \ux \pdv{\Ld}{\ux} \right) = \frac{1}{\Ld^2} \left( \Ld - \ux \frac{\ux}{\Ld} \right) = \frac{\Ld^2 - \ux^2}{\Ld^3}, \label{uxx} \\
		\pdv{}{\uy} \frac{\ux}{\Ld} &= \frac{1}{\Ld^2} \left(\pdv{\ux}{\uy} \Ld - \ux \pdv{\Ld}{\uy} \right) = -\frac{\ux \uy}{\Ld^3}, \\
		\pdv{}{\uz} \frac{\ux}{\Ld} &= -\frac{\ux \uz}{\Ld^3}, \label{uxz}
	\end{align}
	Generalizing \refeq{uxx}--\refeq{uxz} to the $\pdv*{}{y}$ and $\pdv*{}{z}$ terms,
	\begin{align*}
		\pdv{}{x} \pdv{\Ld}{\ux} &= \uxx \frac{\Ld^2 - \ux^2}{\Ld^3} - \uyx \frac{\ux \uy}{\Ld^3} - \uzx \frac{\ux \uz}{\Ld^3}, \\
		\pdv{}{y} \pdv{\Ld}{\uy} &= \uyy \frac{\Ld^2 - \uy^2}{\Ld^3} - \uxy \frac{\ux \uy}{\Ld^3} - \uzy \frac{\uy \uz}{\Ld^3}, \\
		\pdv{}{z} \pdv{\Ld}{\uz} &= \uzz \frac{\Ld^2 - \uz^2}{\Ld^3} - \uxz \frac{\ux \uz}{\Ld^3} - \uyz \frac{\uy \uz}{\Ld^3}.
	\end{align*}
	Then, assuming $\uxy = \uyx$, $\uyz = \uzy$, and $\uxz = \uzx$, \refeq{el1} becomes
	\begin{align*}
		0 &= \uxx (\Ld^4 - \ux^2) + \uyy (\Ld^4 - \uy^2) + \uzz (\Ld^4 - \uz^2) - 2 \uxy \ux \uy - 2 \uyz \uy \uz - 2 \uxz \ux \uz \\
		&= (\uxx + \uyy + \uzz) (1 + \ux^2 + \uy^2 + \uz^2) - \uxx \ux^2 - \uyy \uy^2 - \uzz \uz^2 - 2 \uxy \ux \uy - 2 \uyz \uy \uz - 2 \uxz \ux \uz \\
		&= \uxx (1 + \uy^2 + \uz^2) + \uyy (1 + \ux^2 + \uz^2) + \uzz(1 + \ux^2 + \uy^2) - 2 \uxy \ux \uy - 2 \uyz \uy \uz - 2 \uxz \ux \uz.
	\end{align*}
\end{solution}


\newcommand{\intdr}{\int_{\partial R}}

\section{Plate vibrations (preliminaries)} \label{2}

\begin{statement}
	Start from Green's theorem
	\beq
		\intr \left( \pdv{Q}{x} - \pdv{P}{y} \right) \dd{x} \dd{y} = \intdr (P \dd{x} + Q \dd{y}),
	\eeq
	where $R$ is the region in the $xy$ plane spanned by the plate, and $\dR$ its boundary.
\end{statement}

\begin{problem}
	Show that
	\beq
		\intr \phi \pdv[2]{\psi}{x} \dd{x} \dd{y} = \intr \psi \pdv[2]{\phi}{x} \dd{x} \dd{y} + \intdr \left( \phi \pdv{\psi}{x} - \psi \pdv{\phi}{x} \right) \dd{y}.
	\eeq
\end{problem}

%\begin{solution}
%	
%\end{solution}


\newcommand{\eps}{\epsilon}
\newcommand{\intt}{\int_{t_0}^{t_1}}
\newcommand{\ut}{u_t}
\newcommand{\utt}{u_{tt}}
\newcommand{\kaq}{\kappa_1}
\newcommand{\lapw}{\nabla^4}
\newcommand{\Pu}{P(u)}
\newcommand{\Mu}{M(u)}
\newcommand{\dlt}{\dd{l} \dd{t}}
\newcommand{\uo}{u_0}

\section{Plate vibrations}

\begin{statement}
	Start with the action for a vibrating plate whose potential energy is dominated by bending,
	\beq
		S[u(x, y, t)] = \eps \intt \intr \left\{ \rho \ut^2 - \kaq \left[ (\uxx^2 + \uyy^2) - 2 (1 - \mu) (\uxx \uyy - \uxy^2) \right] \right\} \dr,
	\eeq
	where $\rho$ is the mass density per unit area, $\kaq$ has the dimension of energy and is sometimes called flexural rigidity, and $\mu$ is a dimensionless material constant called Poisson's ratio.  For isotropic material, $\rho = 1/4$.  Notice that there is \emph{no} external bending moment applied to the plate boundary.  There is also \emph{no} external forcing.
\end{statement}

\begin{problem}
Using the results of problem~\ref{2}, show that the variation generated by going from a solution $\uo$ to $\uo + \eps\psi$ had the form
	\beq
		S[u(x, y, t)] = \eps \intt \intr \left(\rho \utt - \kaq \lapw u \right) \psi \dr + \eps \intt \intdr \left( \Pu \psi + \Mu \pdv{\psi}{n} \right) \dlt.
	\eeq
	Specify $\Pu$ and $\Mu$.
\end{problem}

\begin{problem}
	Finally, derive the Euler-Lagrange equation and the associated boundary conditions.
\end{problem}


\newcommand{\lam}{\lambda}
\newcommand{\vrt}{v(r, \theta)}
\newcommand{\gt}{g(t)}
\newcommand{\lap}{\nabla^2}
\newcommand{\fnr}{f_n(r)}

\section{Vibrations of a circular disk}

\begin{statement}
	The only scenario in which plate vibrations can be described analytically in terms of known functions is a circular disk.  Work with polar coordinates $(r, \theta)$, the Euler-Lagrange equation
	\beqn \label{el4}
		\utt - \lam \lapw u = 0,
	\eeqn
	and the boundary conditions
	\begin{align*}
		u &= 0, & \pdv{u}{n} &= 0.
	\end{align*}
\end{statement}


\newcommand{\Cq}{C_1}
\newcommand{\Cw}{C_2}

\newcommand{\Dq}{D_1}
\newcommand{\Dw}{D_2}

\newcommand{\sqmu}{\sqrt{\mu}}

\begin{problem}
	Show that this problem reduces to an eigenvalue problem if we assume that $u(r, \theta, t)$ is separable:
	\beqn \label{udef}
		u = \vrt\, \gt.
	\eeqn
	Write down the general form of $g(t)$.
\end{problem}

\begin{solution}
	Substituting the ansatz \refeq{udef} into \refeq{el4}, we have
	\beqn \label{pde4}
		v \pdv[2]{g}{t} - \lambda g \,\lapw v = 0 \implies \frac{1}{g} \pdv[2]{g}{t} = \lambda \frac{1}{v} \lapw v \equiv -\mu
	\eeqn
	where we have defined some constant $\mu$.  We may then separate \refeq{pde4} into two differential equations,
	\begin{align}
		\lambda \lapw v + \mu v &= 0, \label{evp} \\
		\pdv[2]{g}{t} + \mu g &= 0. \label{time}
	\end{align}
	The eigenvalue problem is \refeq{evp}, which we may solve for the eigenvalues $\mu_n$ and obtain the eigenfunctions $v_n(r, \theta)$.  Then we simply feed $\mu_n$ into \refeq{time} to obtain $g_n(t)$, which have the general form
	\beqn \label{gt}
		g(t) = \Cq e^{\sqmu x} + \Cw e^{-\sqmu x},
	\eeqn
	where we note that $\sqmu$ may be imaginary.  If so, \refeq{gt} may be written in terms of sines and cosines.  Finally, the solutions to \refeq{el4} are $u_n(r, \theta, t) = v_n(r, \theta) \, g_n(t)$.
\end{solution}

\begin{problem}
	Now consider the eigenvalue problem
	\beqn \label{evpb}
		(\lapw - k^4) \vrt = 0,
	\eeqn
	with $\lam$ set to be $k^4$.  Notice that it factors into
	\beqn \label{sep}
		(\lap - k^2)(\lap + k^2) \vrt = 0,
	\eeqn
	with
	\beq
		\lap = \pdv[2]{}{r} + \frac{1}{r} \pdv{}{r} + \frac{1}{r^2} \pdv[2]{}{\theta}.
	\eeq
	Since the disk is circular, we expect the vibration modes to be periodic in $\theta$.  This suggests the ansatz
	\beqn \label{ansatz4}
		v = \sum_{n = -\infty}^\infty \fnr \, e^{in\theta}.
	\eeqn
	Obtain the ODE governing $\fnr$.
\end{problem}

\newcommand{\fm}{f_m}
\newcommand{\fp}{f_p}

\newcommand{\km}{k_m}
\newcommand{\kp}{k_m}

\newcommand{\eint}{e^{in\theta}}
\newcommand{\eimt}{e^{im\theta}}
\newcommand{\eipt}{e^{ip\theta}}

\begin{solution}
	Firstly, note that
		\beq
		\lapw = \left( \pdv[2]{}{r} + \frac{1}{r} \pdv{}{r} + \frac{1}{r^2} \pdv[2]{}{\theta} \right)^2 = \pdv[4]{}{r} + \frac{2}{r} \pdv[3]{}{r} + \frac{1}{r^2} \pdv[2]{}{r} + \frac{2}{r^2} \pdv[2]{}{r} \pdv[2]{}{\theta} + \frac{2}{r^3} \pdv{}{r} \pdv[2]{}{\theta} + \frac{1}{r^4} \pdv[4]{}{\theta}.
	\eeq
	Substituting the ansatz of \refeq{ansatz4} into \refeq{evpb} yields
	\begin{align*}
		k^4 \fnr \,\eint &= -\lapw \fnr \,\eint \\
		&= \left( \pdv[4]{}{r} + \frac{2}{r} \pdv[3]{}{r} + \frac{1}{r^2} \pdv[2]{}{r} + \frac{2}{r^2} \pdv[2]{}{r} \pdv[2]{}{\theta} + \frac{2}{r^3} \pdv{}{r} \pdv[2]{}{\theta} + \frac{1}{r^4} \pdv[4]{}{\theta} \right) \fnr \,\eint \\
		&= \eint \left( \pdv[4]{}{r} + \frac{2}{r} \pdv[3]{}{r} + \frac{1}{r^2} \pdv[2]{}{r} - \frac{2n^2}{r^2} \pdv[2]{}{r} - \frac{2n^2}{r^3} \pdv{}{r} + \frac{n^4}{r^4} \right) \fnr.
	\end{align*}
	Dividing out $\eint$, we have
	\beq
		k^4 \fnr = \pdv[4]{\fnr}{r} + \frac{2}{r} \pdv[3]{\fnr}{r} + \frac{1 - 2n^2}{r^2} \pdv[2]{\fnr}{r} - \frac{2n^2}{r^3} \pdv{\fnr}{r} + \frac{n^4}{r^4} \fnr
	\eeq
	as the ODE governing $\fnr$.
\end{solution}

\begin{problem}
	What are the appropriate boundary conditions on $\fnr$?
\end{problem}

\begin{solution}
	Firstly, note that \refeq{evpb} may be separated into the two eigenvalue problems
	\begin{align}
		0 &= \lapw v - k^2 v, \label{kneg} \\
		0 &= \lapw v + k^2 v. \label{kpos}
	\end{align}
	Any $k$ that corresponds to a nontrivial solution of \refeq{evpb} must also correspond to a nontrivial solution of \refeq{kneg} and of \refeq{kpos}.  We will proceed by solving \refeq{kneg} for $k_m$ and \refeq{kpos} for $k_p$.  Then, any $k_n \in k_m \cap k_p$ that nontrivially solves both \refeq{kpos} and \refeq{kneg} must also nontrivially solve \refeq{evpb}.
	
	Beginning with \refeq{kneg}, we have
	\beq
		0 = \left( \pdv[2]{}{r} + \frac{1}{r} \pdv{}{r} + \frac{1}{r^2} \pdv[2]{}{\theta} \right) \fm(r) \, \eimt - \km^2 \fm(r) \, \eimt = \left( \pdv[2]{}{r} + \frac{1}{r} \pdv{}{r} \right) \fm \, \eimt - \frac{1}{r^2} m^2 \fm \, \eimt - \km^2 \fm \, \eimt
	\eeq
	where we have substituted the ansatz \refeq{ansatz4}, here $v_m = \fm(r) \, \eimt$.  Dividing out $\eimt$, this becomes
	\beqn \label{modbess}
		0 = r^2 \pdv[2]{\fm}{r} + r \pdv{\fm}{r} - (\km^2 r^2 + m^2) \fm,
	\eeqn
	which is the modified Bessel equation of order $m$.  It has solutions
	\beq
		\fm(r) = \Cq I_m(kr) + \Cw K_p(kr),
	\eeq
	where $\Cq$ and $\Cw$ are constants, $I_m$ is the modified Bessel function of the first kind, and $K_m$ is the modified Bessel function of the second kind.  Both functions are of order $m$.
	
	Proceeding similarly for \refeq{kpos}, we obtain
	\beqn \label{bess}
		0 = r^2 \pdv[2]{\fp}{r} + r \pdv{\fp}{r} + (\kp^2 r^2 - p^2) \fp,
	\eeqn
	which is the Bessel equation of order $p$, and has solutions
	\beq
		\fp(r) = \Dq J_p(kr) + \Dw Y_p(kr),
	\eeq
	where $\Dq$ and $\Dw$ are constants, $J_p$ is the Bessel function of the first kind, $Y_p$ is the Bessel function of the second kind, and both are of order $p$.

	Both $Y_n$ and $K_n$ diverge as $r \to 0$ for all $n$, so we do not want them in our solution.
\end{solution}



In writing these solutions, I consulted Gelfand and Fomin's \emph{Calculus of Variations}, Olmstead and Volpert's \emph{Differential Equations in Applied Mathemtics}.

\end{document}