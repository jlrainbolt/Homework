\newcommand{\gam}{\gamma}
\newcommand{\xo}{x_0}
\newcommand{\xq}{x_1}
\newcommand{\tlt}{\tilde{t}}

\section{Extra credit}
\begin{problem}
	Verify that the nondimensionalized, one-dimensional Sine-Gordon equation,
	\beqn \label{sg}
		\thtxx - \thttt = \sint,
	\eeqn
	is also invariant under a Lorentz transformation on $(\xo = t, \xq = x)$.  The transformation is given by
	\beq
		\mqty[	\gam & -\gam \nu \\
				-\gam \nu & \gam ],
	\eeq
	where $\gam = 1 / \sqrt{1 - \nu^2}$.
\end{problem}

\begin{solution}
	Define $(\tlt, \xt)$ as the transformed coordinates.  \refeq{sg} is invariant if it has the same form under the substitution $\tht(t, x) \mapsto \tht(\tlt, \xt)$.  The new coordinates are given by
	\beq
		\mqty[	\tlt \\ \xt ]
		= \mqty[	\gam & -\gam \nu \\
				-\gam \nu & \gam ]
		\mqty[	t \\ x ]
		= \mqty[	\gam (t - \nu x) \\
					\gam (x - \nu t) ],
	\eeq
	or
	\begin{align*}
		\tlt &= \gam (t - \nu x), &
		\xt &= \gam (x - \nu t).
	\end{align*}
	Proceeding similarly to problem 1, the chain rule gives us
	\begin{align*}
		\pdv{}{t} &= \pdv{}{\tlt} \pdv{\tlt}{t} + \pdv{}{\xt} \pdv{\xt}{t}
		= \gam \left( \pdv{}{\tlt} - \nu \pdv{}{\xt} \right), &
		\pdv{}{x} &= \pdv{}{\tlt} \pdv{\tlt}{x} + \pdv{}{\xt} \pdv{\xt}{x}
		= \gam \left( \pdv{}{\xt} - \nu \pdv{}{\tlt} \right).
	\end{align*}
	For the second derivatives,
	\begin{align*}
		\pdv[2]{}{t} &= \gam^2 \left( \pdv{}{\tlt} - \nu \pdv{}{\xt} \right)^2
		= \gam^2 \left( \pdv[2]{}{\tlt} - 2 \nu \pdv{}{\tlt}{\xt} + \nu^2 \pdv[2]{}{\xt} \right), \\
		\pdv[2]{}{x} &= \gam^2 \left( \pdv{}{\xt} - \nu \pdv{}{\tlt} \right)^2
		= \gam^2 \left( \pdv[2]{}{\xt} - 2 \nu \pdv{}{\tlt}{\xt} + \nu^2 \pdv[2]{}{\tlt} \right).
	\end{align*}
	Making these substitutions, \refeq{sg} becomes
	\begin{align*}
		\sint &= \pdv[2]{\tht}{x} - \pdv[2]{\tht}{t} \\
		&= \gam^2 \left( \pdv[2]{\tht}{\xt} - 2 \nu \pdv{\tht}{\tlt}{\xt} + \nu^2 \pdv[2]{\tht}{\tlt} \right) - \gam^2 \left( \pdv[2]{\tht}{\tlt} - 2 \nu \pdv{\tht}{\tlt}{\xt} + \nu^2 \pdv[2]{\tht}{\xt} \right) \\
		&= \gam^2 \left[ (1 - \nu^2) \pdv[2]{\tht}{\xt} - (1 - \nu^2) \pdv[2]{\tht}{\tlt} \right] \\
		&= \pdv[2]{\tht}{\xt} - \pdv[2]{\tht}{\tt},
	\end{align*}
	because $\gam^2 = 1 / (1 - \nu^2)$.  Thus, we have demonstrated the invariance of \refeq{sg}. \qed
\end{solution}

\begin{problem}
	Find the associated conserved quantity.  Is it analogous to a common conserved quantity in classical mechanics?
\end{problem}

\begin{solution}
	By analogy to problem 3, the Lagrangian for this system is given by
	\beq
		\Ld = \frac{1}{2} (\thtt^2 - \thtx^2) - \sint
	\eeq
	which is like \refeq{lagr3}, but with only one spatial dimension.  Continuing the analogy, the components of the energy-momentum vector are
	\begin{align*}
		P_0 &= \int \left[ \frac{1}{2} (\thtt^2 + \thtx^2) + \sint \right] \dd{x}, &
		P_1 &= \int \thtt \thtx \dd{x}. &
	\end{align*}
	These are the conserved quantitites, or ``currents.''  The component $P_0$ is analagous to the calssical Hamiltonian, or the total energy of the system.  This corresponds to $\Ld$'s having no explicit $t$ dependence.  The component $P_1$ is like the momentum conjugate to $x$, since it corresponds to $\Ld$'s having no explicit $x$ dependence.  Since we are concerned with only one spatial dimension, $P_1$ is analogous to the classical total (linear) momentum of the system.
\end{solution}