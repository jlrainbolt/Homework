\newcommand{\vB}{\vec{B}}
\newcommand{\vA}{\vec{A}}
\newcommand{\px}{p_x}
\newcommand{\py}{p_y}
\newcommand{\pdx}{\dot{p}_x}
\newcommand{\pdy}{\dot{p}_y}

\begin{statement}{Charged particle in a magnetic field}
\subparagraph{}
	Suppose a charged particle moves in a two-dimensional plane while experiencing a magnetic field $\vB = (0, 0, B)$.  Use the vector potential $\vA = (-By, 0, 0)$.  The Hamiltonian for the particle is
	\beq
		H = \frac{1}{2m} \left( \px + \frac{e B}{c} y \right)^2 + \frac{\py^2}{2m}.
	\eeq
\vfix
\end{statement}

\begin{problem}
	Write down Hamilton's equations.  Verify that by appropriate manipulation we have
	\begin{align*}
		\py + \frac{e B}{c} x &= a, &
		\px &= m \xd - \frac{e B}{c} y = b,
	\end{align*}
	where $a$ and $b$ are constants.
\end{problem}

\begin{solution}
	Note that
	\beq
		H = \frac{1}{2m} \left( \px^2 + 2 \frac{e B}{c} \px y + \frac{e^2 B^2}{c^2} y^2 \right) + \frac{\py^2}{2m}
		= \frac{\px^2}{2m} + \frac{e B}{c} \frac{\px y}{m} + \frac{e^2 B^2}{c^2} \frac{y^2}{2m} + \frac{\py^2}{2m}
	\eeq
	Hamilton's equations are
	\begin{align}
		\xd &= \pdv{H}{\px} = \frac{1}{m} \left( \px + \frac{e B}{c} y \right), \label{xd} \\
		\pdx &= -\pdv{H}{x} = 0, \label{pdx} \\
		\yd &= \pdv{H}{\py} = \frac{\py}{m}, \label{yd} \\
		\pdy &= -\pdv{H}{y} = -\frac{e B}{c} \frac{1}{m} \left( \px + \frac{e B}{c} y \right). \label{pdy}
	\end{align}
	
	Substituting \refeq{xd} into \refeq{pdy},
	\beq
		\pdy = -\frac{e B}{c} \xd.
	\eeq
	By integrating with respect to $t$, we obtain
	\beq
		\py = -\frac{e B}{c} \int \xd \dd{t} = -\frac{e B}{c} x + a,
	\eeq 
	where $a$ is some constant.  Therefore, we have
	\beqn \label{a}
		\py + \frac{e B}{c} x = a,
	\eeqn
	as desired.
	
	From \refeq{xd},
	\beq
		m \xd = \px + \frac{e B}{c} y
		\iff
		\px = m \xd - \frac{e B}{c} y,
	\eeq
	and from \refeq{pdx},
	\beq
		\px = \int 0 \dd{t} = b,
	\eeq
	where $b$ is some constant.  Combining these, we have
	\beqn \label{b}
		\px = m \xd - \frac{e B}{c} y = b
	\eeqn
	as desired. \qed
\end{solution}

\newcommand{\xdd}{\ddot{x}}
\newcommand{\ydd}{\ddot{y}}

\newcommand{\xc}{x_c}
\newcommand{\yc}{y_c}
\newcommand{\xddc}{\ddot{x}_c}
\newcommand{\yddc}{\ddot{y}_c}

\newcommand{\xxt}{x(t)}
\newcommand{\yyt}{y(t)}
\newcommand{\xct}{\xc(t)}
\newcommand{\yct}{\yc(t)}
\newcommand{\xpt}{x_p(t)}
\newcommand{\ypt}{y_p(t)}

\begin{problem}
	Using the relations above and the equations of motion, verify that the charged particle moves in a circle in the $(x, y)$ plane and that the circling frequency $\omg$ is given by
	\beq
		\omg = \frac{e B}{m c}.
	\eeq
This is called the \emph{Larmor frequency}.
\end{problem}

\begin{solution}
	Substituting \refeq{b} into \refeq{xd} yields
	\beqn \label{xd2}
		\xd = \frac{1}{m} \left( b + \frac{e B}{c} y \right)
		= \frac{e B}{m c} \left( \frac{c}{e B} b + y \right).
	\eeqn
	Similarly, solving \refeq{a} for $\py$ and substituting into \refeq{yd} gives us
	\beqn \label{yd2}
		\yd = \frac{1}{m} \left( a - \frac{e B}{c} x \right)
		= \frac{e B}{m c} \left( \frac{c}{e B} a - x \right).
	\eeqn
	Differentiating \refeq{xd2} and \refeq{yd2} by $t$, we obtain two uncoupled second-order equations:
	\begin{align} \label{sys}
		\xdd &= \frac{e B}{m c} \yd
		= -\frac{e^2 B^2}{m^2 c^2} \left( x - \frac{c}{e B} a \right), &
		\ydd &= -\frac{e B}{m c} \xd
		= -\frac{e^2 B^2}{m^2 c^2} \left( y + \frac{c}{e B} b \right).
	\end{align}
	Let $\xt$ and $\yt$ be new coordinates such that
	\begin{align*}
		\xt &\equiv x - \frac{c}{e B} a, &
		\yt &\equiv y + \frac{c}{e B} b.
	\end{align*}
	Then
	\begin{align*}
		\dv{\xt}{t} &= \xd, &
		\dv[2]{\xt}{t} &= \xdd, &
		\dv{\yt}{t} &= \yd, &
		\dv[2]{\yt}{t} &= \ydd,
	\end{align*}
	and the equations \refeq{sys} can be rewritten in terms of $\xt$ and $\yt$:
	\begin{align*}
		\dv[2]{\xt}{t} &= -\frac{e^2 B^2}{m^2 c^2} \xt, &
		\dv[2]{\yt}{t} &= -\frac{e^2 B^2}{m^2 c^2} \yt.
	\end{align*}
	These equations have the solutions
	\begin{align*}
		\xt(t) &= \Cq \cos(\frac{e B}{m c} t) + \Cw \sin(\frac{e B}{m c} t)
		\equiv \Cq \cos(\omg t) + \Cw \sin(\omg t), \\
		\yt(t) &= \Dq \cos(\frac{e B}{m c} t) + \Dw \sin(\frac{e B}{m c} t)
		\equiv \Dq \cos(\omg t) + \Dw \sin(\omg t),
	\end{align*}
	where $\Cq, \Cw, \Dq, \Dw$ are constants, and we have defined
	\beqn \label{omg2}
		\omg \equiv \frac{e B}{m c}.
	\eeqn
	Applying \refeq{xd2} and \refeq{yd2}, we have
	\begin{align*}
		\dv{\xt}{t} &= -\Cq \omg \sin(\omg t) + \Cw \omg \cos(\omg t) = \omg \yt, &
		\dv{\yt}{t} &= -\Dq \omg \sin(\omg t) + \Dw \omg \cos(\omg t) = -\omg \xt.
	\end{align*}
	This implies $\Cq = -\Dw$ and $\Cw = \Dq$.  We may fix $\Cq = \Dw = 0$ and $\Cw = \Dq = R$ without loss of generality, where $R$ is some constant.
	Transforming back to the original coordinates, we have
	\begin{align*}
		\xxt &= R \sin(\omg t) + \frac{c}{e B} a, &
		\yyt &= R \cos(\omg t) - \frac{c}{e B} b.
	\end{align*}
	These solutions show that the particle moves in a circle with angular frequency $\omg$ defined by \refeq{omg2}, as desired. \qed
\end{solution}

\begin{problem}
	Now consider the limit where the $B$ field can be made arbitrarily strong.  Compare the Poisson bracket $[x, \px]$ for the charged particle with the Poisson bracket relation
	\beqn \label{poiss7a}
		[\xi, \yj] = \frac{\delta_{ij}}{\gami}
	\eeqn
	for the system of line vortices described in problems~\ref{vort1} and \ref{vort2}.
\end{problem}

\begin{solution}
	In general, the Poisson bracket for the charged particle is given by
	\beq
		[f, g] = \pdv{f}{x} \pdv{g}{\px} - \pdv{f}{\px} \pdv{g}{x}.
	\eeq
	Note that
	\begin{align*}
		\pdv{x}{x} &= \pdv{\px}{\px} = 1, &
		\pdv{x}{\px} &= \pdv{\px}{x} = 0,
	\end{align*}
	so
	\beq
		[x, \px] = \pdv{x}{x} \pdv{\px}{\px} - \pdv{x}{\px} \pdv{\px}{x} = 1.
	\eeq
	In the limit $B \to \infty$, \refeq{b} becomes
	\beq
		\px \to -\frac{e B}{c} y.
	\eeq
	It follows that
	\begin{align*}
		\pdv{y}{\px} &= -\frac{c}{e B}, &
		\pdv{y}{x} &= 0.
	\end{align*}
	Thus,
	\beqn \label{poiss7b}
		[x, y] = \pdv{x}{x} \pdv{y}{\px} - \pdv{x}{\px} \pdv{y}{x} = -\frac{c}{e B}.
	\eeqn
	
	Consider a particular vortex $i$ from the line vortex problem.  Its position is $(\xi, \yi)$, and as we discussed in class, $\yi$ may also be treated as its generalized momentum.  From \refeq{poiss7a}, the Poisson bracket relating the position coordinates of the vortex in the two-dimensional plane is inversely proportional to the strength of the vortex itself.  This is similar to the charged particle Poisson bracket in \refeq{poiss7b}, which relates the position coordinates of the particle in the plane perpendicular to a strong magnetic field.  This bracket is inversely proportional to the strength of the field.  So, in both cases, we see an inverse proportionality between position and ``strength,'' which suggests a similarity between the two systems.
\end{solution}