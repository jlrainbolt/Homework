\documentclass[11pt]{article}
\usepackage{geometry, titlesec}
\usepackage[parfill]{parskip}
\usepackage{physics, amsfonts, amsthm}
\usepackage{fullpage}
\usepackage{fancyhdr}
\usepackage{enumitem}
\usepackage{xcolor, soul}
%\allowdisplaybreaks

\makeatletter
\renewcommand*\env@cases[1][1.2]{%
  \let\@ifnextchar\new@ifnextchar
  \left\lbrace
  \def\arraystretch{#1}%
  \array{@{}l@{\quad}l@{}}%
}
\makeatother
 
 
\setlist[enumerate]{leftmargin=*}
\pagestyle{fancy}
\fancyhf{}
\lhead{\textbf{Physics 316 Homework 2}}
\rhead{Lacey Rainbolt}
\setlength{\headheight}{14pt}
\setlength{\headsep}{12pt}
\cfoot{\thepage}

\titleformat{\subsection}[runin]{\normalfont\large\bfseries}{\thesubsection}{1em}{}
\newcommand{\refeq}[1]{(\ref{#1})}


\newenvironment{statement}
{
    \color{darkgray}
    \ignorespaces
}
{
    \bigskip
}

\newenvironment{problem}
{
    \color{darkgray}
    \subsection{}
    \ignorespaces
}


\newenvironment{solution}
{
    \paragraph{Solution.}
    \ignorespaces
}
{
    \bigskip
}



\begin{document}

\renewcommand{\vec}[1]{\mathbf{#1}}

\newcommand{\vR}{\vec{R}}
\newcommand{\vRd}{\vec{\dot{R}}}
\newcommand{\vRdd}{\vec{\ddot{R}}}
\newcommand{\vr}{\vec{r}}
\newcommand{\vrd}{\vec{\dot{r}}}

\section{Reduced three-body problem}
\begin{statement}
		The problem of three point particles interacting gravitationally has a particularly simple limit: let the third body $m_3 \ll m_2, m_1$ so that its effect on the motions of $m_1$ and $m_2$ is negligible.  Assume in addition that $m_3$ moves in the same orbital plane as $m_1$ and $m_2$.  For simplicity, consider only the case of $m_1$ and $m_2$ in circular orbit about their center of mass.
\end{statement}

\begin{problem}
	Switch into a reference frame rotating with angular velocity $\omega$ associated with the circular orbit for the two-body problem.  Choose the center of mass of the two-body problem to be the origin.  Choose the $x$ axis to go through $m_1$ and $m_2$.  Show that the (now stationary) $m_1$ and $m_2$ are located at $-r_c \mu / m_1$ and $r_c \mu / m_2$.
\end{problem}

\begin{solution}
	\hl{Call the stationary coordinate system} $(r, \theta, \phi)$.  We showed in Prob.~4 of Homework 1, that the motion for $m_1, m_2$ is confined to a plane, which we may choose to be $(r, \theta)$.  Then $m_1$ and $m_2$ are located at $\vR_1 = \vR_1(t, r, \theta)$ and $\vR_2 = \vR_2(t, r, \theta)$.  Let $\vR = \vR_1 - \vR_2$ denote the separation between the masses.  Using the method discussed in class, the motion of $m_1$ and $m_2$ is governed by the equation
	\begin{equation} \label{lagr1}
		\mu \vRdd = -\pdv{}{r} V_\text{eff}
	\end{equation}
	where $\mu = m_1 m_2 / (m_1 + m_2)$ is the reduced mass, $\vR_{12} = \vR_1 - \vR_2$ is the separation between the two masses, and the effective potential
	\begin{equation}
		V_\text{eff}(R) = V(R) + \frac{J^2}{2 \mu R^2} = -\frac{G m_1 m_2}{R} + \frac{J^2}{2 \mu R^2}
	\end{equation}
	where $J = \mu R^2 \dot{\theta}$ is the magnitude of the total angular momentum, which is conserved. 
	
	
\hl{???}

	Call the stationary coordinate system $(X, Y, Z)$, and choose $(X, Y)$ as the orbital plane.  Let the locations of $m_1$ and $m_2$ be given by $\vR_1 = \vR_1(t, X, Y)$ and $\vR_2 = \vR_2(t, X, Y)$.  Let $r_c$ be the radius of the orbit about the $Z$ axis.  We know that $m_1, m_2$ both have circular orbits about the $Z$ axis with frequency $\omega$, so we can write
	\begin{equation}
		\vR_1(t) = 
	\end{equation}
	
	Call the rotating coordinate system $(x, y, z)$, in which the locations of $m_1$ and $m_2$ are $\vr_1 = \vr_1(t, x, y, z)$ and $\vr_2 = \vr_2(x, y, z)$.
	
	Since we have chosen the $x$ axis to go through $m_1$ and $m_2$, we will choose our reference frame to be rotating about the $z$ axis.  We thus have the transformation
	\begin{align}
		x &= X \cos{\omega t} + Y \sin{\omega t}, \\
		y &= Y \cos{\omega t} - X \sin{\omega t}, \\
		z &= Z.
	\end{align}
\end{solution}

\newcommand{\xd}{\dot{x}}
\newcommand{\yd}{\dot{y}}

\begin{problem}
	Show that the Lagrangian governing the wquation of motion of $m_3$ at location $(x(t), y(t))$ is
	\begin{equation}
		L_3 = \frac{m_3}{2} \left[ (\xd - \omega y)^2 + (\yd + \omega x)^2 \right] - U_{13} - U_{23},
	\end{equation}
	where $U_13(x, y)$ is the gravitaional interaction of $m_3$ with $m_1$, while $U_{23}$ is associated with $m_3$ and $m_2$.
\end{problem}
    

\end{document}