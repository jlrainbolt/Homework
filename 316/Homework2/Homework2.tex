\documentclass[11pt]{article}
\usepackage{geometry, titlesec}
\usepackage[parfill]{parskip}
\usepackage{physics, amsfonts, amsthm}
\usepackage{fullpage}
\usepackage{fancyhdr}
\usepackage{enumitem}
\usepackage{xcolor, soul}
%\allowdisplaybreaks

\makeatletter
\renewcommand*\env@cases[1][1.2]{%
  \let\@ifnextchar\new@ifnextchar
  \left\lbrace
  \def\arraystretch{#1}%
  \array{@{}l@{\quad}l@{}}%
}
\makeatother
 
 
\setlist[enumerate]{leftmargin=*}
\pagestyle{fancy}
\fancyhf{}
\lhead{\textbf{Physics 316 Homework 2}}
\rhead{Lacey Rainbolt}
\setlength{\headheight}{14pt}
\setlength{\headsep}{12pt}
\cfoot{\thepage}

\titleformat{\subsection}[runin]{\normalfont\large\bfseries}{\thesubsection}{1em}{}
\newcommand{\refeq}[1]{(\ref{#1})}


\newenvironment{statement}
{
    \color{darkgray}
    \ignorespaces
}
{
    \bigskip
}

\newenvironment{problem}
{
    \subsection{}
    \color{darkgray}
    \ignorespaces
}


\newenvironment{solution}
{
    \paragraph{Solution.}
    \ignorespaces
}
{
    \bigskip
}



\begin{document}

\renewcommand{\vec}[1]{\mathbf{#1}}

\newcommand{\vR}{\vec{R}}
\newcommand{\vRd}{\vec{\dot{R}}}
\newcommand{\vRdd}{\vec{\ddot{R}}}
\newcommand{\vr}{\vec{r}}
\newcommand{\vrd}{\vec{\dot{r}}}

\section{Reduced three-body problem}
\begin{statement}
		The problem of three point particles interacting gravitationally has a particularly simple limit: let the third body $m_3 \ll m_2, m_1$ so that its effect on the motions of $m_1$ and $m_2$ is negligible.  Assume in addition that $m_3$ moves in the same orbital plane as $m_1$ and $m_2$.  For simplicity, consider only the case of $m_1$ and $m_2$ in circular orbit about their center of mass.
\end{statement}

\begin{problem}
	Switch into a reference frame rotating with angular velocity $\omega$ associated with the circular orbit for the two-body problem.  Choose the center of mass of the two-body problem to be the origin.  Choose the $x$ axis to go through $m_1$ and $m_2$.  Show that the (now stationary) $m_1$ and $m_2$ are located at $-r_c \mu / m_1$ and $r_c \mu / m_2$.
\end{problem}

\begin{solution}
	Call the stationary coordinate system $\vR = (X, Y, Z)$, and choose $(X, Y)$ as the orbital plane.  Call the rotating coordinate system $\vr = (x, y, z)$, which is rotated about the $Z$ axis by angle $\omega t$.  This gives us the transformation
	\begin{align}
		x &= X \cos{\omega t} + Y \sin{\omega t}, \label{oldx} \\
		y &= Y \cos{\omega t} - X \sin{\omega t},  \label{oldy} \\
		z &= Z.
	\end{align}
	From our choice of orbital plane, there is no motion in the $z$ direction.  Let the locations of $m_1$ and $m_2$ be given by $\vr_1 = (x_1, y_1)$ and $\vr_2 = (x_2, y_2)$ in the rotating frame.
	
	From our choice of $x$ axis, we know that $y_1 = y_2 = 0$.  From our choice of the origin as the center of mass, we have
	\begin{equation} \label{cm1}
		m_1 x_1 + m_2 x_2 = 0.
	\end{equation}
	By construction, $m_1$ and $m_2$ are stationary in the rotating frame, so $\dv*{x_1}{t} = \dv*{x_2}{t} = 0$.  In other words, $x_1$ and $x_2$ must both be constant.  Therefore, let
	\begin{equation} \label{cm2}
		r_c = x_2 - x_1
	\end{equation}
	be the constant distance between $m_1$ and $m_2$.  Now we have the system of two equations \refeq{cm1} and \refeq{cm2}, so we can solve for $x_1$ and $x_2$.  Substituting \refeq{cm2} as $x_2 = r_c + x_1$ into \refeq{cm1},
	\begin{equation} \label{x1}
		m_1 x_1 + m_2 (r_c + x_1) = 0 \implies x_1 = -\frac{m_2}{m_1 + m_2} r_c.
	\end{equation}
	Now substituting \refeq{x1} back into \refeq{cm1},
	\begin{equation} \label{x2}
		r_c = x_2 + \frac{m_2}{m_1 + m_2} r_c \implies x_2 = \frac{m_1}{m_1 + m_2} r_c.
	\end{equation}
	Note that the reduced mass $\mu = m_1 m_2 / (m_1 + m_2)$.  Substituting $\mu$ into \refeq{x1} and \refeq{x2} yields
	\begin{align} \label{result1a}
		\vr_1 &= x_1 = - r_c \mu / m_1, & \vr_2 &= x_2 = r_c \mu / m_2
	\end{align}
	as desired. \qed


\end{solution}

\newcommand{\xd}{\dot{x}}
\newcommand{\Xd}{\dot{X}}
\newcommand{\yd}{\dot{y}}
\newcommand{\Yd}{\dot{Y}}

\begin{problem}
	Show that the Lagrangian governing the equation of motion of $m_3$ at location $(x(t), y(t))$ is
	\begin{equation} \label{lagr3given}
		L_3 = \frac{m_3}{2} \left[ (\xd - \omega y)^2 + (\yd + \omega x)^2 \right] - U_{13} - U_{23},
	\end{equation}
	where $U_13(x, y)$ is the gravitational interaction of $m_3$ with $m_1$, while $U_{23}(x, y)$ is associated with $m_3$ and $m_2$.
\end{problem}

\begin{solution}
	In general, the Lagrangian for $m_3$ is given by
	\begin{equation} \label{lagr3}
		L_3 = T_3 - U_3,
	\end{equation}
	where $T_3$ is the kinetic energy of $m_3$ and $U_3$ is its potential energy.
	
	Beginning with $U_3$, the only forces acting upon $m_3$ are the gravitational interactions with $m_1$ and $m_2$.  We know from the problem statement that these interactions are independent of each other; $m_3$ has a negligible effect on the motions of each $m_1$ and $m_2$, so it cannot couple them in any way.  Thus, we can write
	\begin{equation} \label{potential}
		U_3 = -G \frac{m_1 m_3}{r_{13}} - G \frac{m_2 m_3}{r_{23}} \equiv U_{13} + U_{23},
	\end{equation}
	where $r_{13}$~($r_23$) is the separation between $m_3$ and $m_1$~($m_2$), and we have defined $U_{13}$ and $U_{23}$.
	
	Now we will find an expression for $T_3$.  Let $\vR_3 = (X(t), Y(t))$ be the position of $m_3$ in the stationary coordinate system.  Then
	\begin{equation} \label{kinetic1}
		T_3 = \frac{m_3}{2} \vRd_3^2 = \frac{m_3}{2} (\Xd + \Yd)^2.
	\end{equation}
	We want to find an expression for $T_3$ in the rotating coordinate system.  We can define an inverse transformation back to the stationary coordinate system by simply rotating the $(x, y)$ plane about the $z$ axis in the opposite direction; that is, by angle $-\omega t$.  This inverse transformation is
	\begin{align}
		X &= x \cos{\omega t} - y \sin{\omega t}, \label{newx} \\
		Y &= x \sin{\omega t} + y \cos{\omega t},  \label{newy} \\
		Z &= z.
	\end{align}
	It follows from \refeq{newx} and \refeq{newy} that
	\begin{align}
		\Xd &= \pdv{X}{t} + \pdv{X}{x} \dv{x}{t} + \pdv{X}{y} \dv{y}{t} = - \omega x \sin{\omega t} - \omega y \cos{\omega t} + \xd \cos{\omega t} - \yd \sin{\omega t} \label{xlong} \\
			&= (\xd - \omega y) \cos{\omega t} - (\yd + \omega x) \sin{\omega t}, \\
		\Yd &= \pdv{Y}{t} + \pdv{Y}{x} \dv{x}{t} + \pdv{Y}{y} \dv{y}{t} =  \omega x \cos{\omega t} - \omega y \sin{\omega t} + \xd \sin{\omega t} + \yd \cos{\omega t} \label{ylong} \\
		&= (\yd + \omega x) \cos{\omega t} + (\xd - \omega y) \sin{\omega t}.
	\end{align}
	Now using the forms of \refeq{xlong} and \refeq{ylong},
	\begin{align}
		\Xd^2 &= (\xd - \omega y)^2 \cos^2{\omega t} - 2 (\xd - \omega y) (\yd + \omega x) \cos{\omega t} \sin{\omega t} + (\yd + \omega x)^2 \sin^2{\omega t}, \\
		\Yd^2 &= (\yd + \omega x)^2 \cos^2{\omega t} + 2 (\xd - \omega y) (\yd + \omega x) \cos{\omega t} \sin{\omega t} + (\xd - \omega y)^2 \sin^2{\omega t},
	\end{align}
	which implies
	\begin{align}
		\Xd^2 + \Yd^2 &= (\xd - \omega y)^2 (\cos^2{\omega t} + \sin^2{\omega t}) + (\yd + \omega x)^2 (\cos^2{\omega t} + \sin^2{\omega t}) \\
		&= (\xd - \omega y)^2 + (\yd + \omega x)^2. \label{squared}
	\end{align}
	Substituting \refeq{squared} into \refeq{kinetic1}, we have
	\begin{equation} \label{kinetic2}
		T_3 = \frac{m_3}{2} \left[ (\xd - \omega y)^2 + (\yd + \omega x)^2 \right].
	\end{equation}
	
	Finally, substituting \refeq{potential} and \refeq{kinetic2} into \refeq{lagr3} yields \refeq{lagr3given} as desired. \qed
\end{solution}

\newcommand{\xdd}{\ddot{x}}
\newcommand{\ydd}{\ddot{y}}

\begin{problem}
	Show that the mechanical system described by $L_3$ has five locations in mechanical equilibrium.  These are known as \emph{Lagrange points}.  (Hint: the graphical method is perfectly good for demonstrating a real root exists in a particular instance.)
\end{problem}

\begin{solution}
	The Euler-Lagrange equations for $m_3$ are given by
	\begin{align}
		0 &= \pdv{L_3}{x} - \dv{}{t} \pdv{L_3}{\xd}, \label{elx} \\
		0 &= \pdv{L_3}{y} - \dv{}{t} \pdv{L_3}{\yd}. \label{ely}
	\end{align}
	We will attack each term of the Lagrangian in \refeq{lagr3} separately.  Beginning with $T_3$, note that
	\begin{align}
		\pdv{T_3}{x} &= m_3 (\omega \yd + \omega^2 x), &
		\pdv{T_3}{y} &= m_3 (-\omega \xd + \omega^2 y), \\
		\pdv{T_3}{\xd} &= m_3 (\xd - \omega y), &
		\pdv{T_3}{\yd} &= m_3 (\yd + \omega x). \label{partd}
	\end{align}
	In turn, \refeq{partd} implies
	\begin{align}
		\dv{}{t} \pdv{T_3}{\xd} &= m_3 (\xdd - \omega \yd), &
		\dv{}{t} \pdv{T_3}{\yd} &= m_3 (\ydd + \omega \xd).
	\end{align}
	
	Now for $U_3$, we can find explicitly the $r_{13}$ and $r_{23}$ appearing in \refeq{potential} using the positions of $m_1$ and $m_2$ on the $x$ axis given by \refeq{result1a}.  These are
	\begin{align} \label{rdefs}
		r_{13} &= \sqrt{\left( x + \frac{r_c \mu}{m_1} \right)^2 + y^2}, &
		r_{23} &= \sqrt{\left( x - \frac{r_c \mu}{m_2} \right)^2 + y^2}.
	\end{align}
	It follows from \refeq{rdefs} that
	\begin{align}
		\pdv{r_{13}}{x} &= \frac{1}{r_{13}} \left( x + \frac{r_c \mu}{m_1} \right), &
		\pdv{r_{23}}{x} &= \frac{1}{r_{23}} \left( x - \frac{r_c \mu}{m_2} \right), \\
		\pdv{r_{13}}{y} &= \frac{y}{r_{13}},
		& \pdv{r_{23}}{y} &= \frac{y}{r_{23}}, \\
		\pdv{r_{13}}{\xd} &= \pdv{r_{23}}{\xd} = \pdv{r_{13}}{\yd} = \pdv{r_{23}}{\yd} = 0.
	\end{align}
	Then
	\begin{align}
		\pdv{U_{13}}{x} &= \pdv{U_{13}}{r_{13}} \pdv{r_{13}}{x} = G \frac{m_1 m_3}{r_{13}^3} \left( x + \frac{r_c \mu}{m_1} \right), & 
		\pdv{U_{23}}{x} &= G \frac{m_2 m_3}{r_{23}^3} \left( x - \frac{r_c \mu}{m_2} \right), \\
		\pdv{U_{13}}{y} &= \pdv{U_{13}}{r_{13}} \pdv{r_{13}}{y} = G \frac{m_1 m_3}{r_{13}^3} y, &
		\pdv{U_{23}}{y} &= G \frac{m_2 m_3}{r_{23}^3} y, \\
		\pdv{U_{13}}{\xd} &= \pdv{U_{23}}{\xd} = \pdv{U_{13}}{\yd} = \pdv{U_{23}}{\yd} = 0.
	\end{align}
	Making the appropriate substitutions, \refeq{elx} becomes
	\begin{align}
		0 &= m_3 (\omega \yd + \omega^2 x) - G \frac{m_1 m_3}{r_{13}^3} \left( x + \frac{r_c \mu}{m_1} \right) - G \frac{m_2 m_3}{r_{23}^3} \left( x - \frac{r_c \mu}{m_2} \right) - m_3 (\xdd - \omega \yd) \\
		\implies \xdd &= 2 \omega \yd + \omega^2 x - G \frac{m_1}{r_{13}^3} \left( x + \frac{r_c \mu}{m_1} \right) - G \frac{m_2}{r_{23}^3} \left( x - \frac{r_c \mu}{m_2} \right), \label{motionx}
	\end{align}
	and \refeq{ely} becomes
	\begin{align}
		0 &= m_3 (-\omega \xd + \omega^2 y) - G \frac{m_1 m_3}{r_{13}^3} y - G \frac{m_2 m_3}{r_{23}^3} y - m_3 (\ydd + \omega \xd) \\
		\implies \ydd &= -2 \omega \xd + \omega^2 y - G y \left( \frac{m_1}{r_{13}^3} + \frac{m_2}{r_{23}^3} \right). \label{motiony}
	\end{align}

	The system is in mechanical equilibrium at points where $\xd = \yd = 0$.  The equilibrium behavior persists over time, implying $\xdd = \ydd = 0$.  With these restrictions, \refeq{motionx} and \refeq{motiony} become
	\begin{align}
		x &= \frac{G}{\omega^2} \frac{m_1}{r_{13}^3} \left( x + \frac{r_c \mu}{m_1} \right) + \frac{G}{\omega^2} \frac{m_2}{r_{23}^3} \left( x - \frac{r_c \mu}{m_2} \right), \label{simplex} \\
		y &= \frac{G}{\omega^2} y \left( \frac{m_1}{r_{13}^3} + \frac{m_2}{r_{23}^3} \right).\label{simpley}
	\end{align}
	The real roots of \refeq{simplex} and \refeq{simpley} are the Lagrange points.
	
	Inspection of \refeq{simpley} indicates that there is at least one solution where $y = 0$.  In this case \refeq{simpley} is eliminated.  Additionally, \refeq{rdefs} becomes
	\begin{align}
		r_{13} &= \qty|x + \frac{r_c \mu}{m_1}|, &
		r_{23} &= \qty|x - \frac{r_c \mu}{m_2}|,
	\end{align}
	and thus \refeq{simplex} reduces to
	\begin{equation}
		x = \frac{G}{\omega^2} \frac{m_1}{\qty|x + r_c \mu / m_1|^3} \left( x + \frac{r_c \mu}{m_1} \right) + \frac{G}{\omega^2} \frac{m_2}{\qty|x - r_c \mu / m_2|^3} \left( x - \frac{r_c \mu}{m_2} \right) \equiv f(x), \label{y0case}
	\end{equation}
	where we have defined $f(x)$ as the right-hand side of the equation.  Note the following observations about $f(x)$:
	\begin{itemize}
		\item $f(x)$ has singularities at $x = - r_c \mu / m_1$ and $x = r_c \mu / m_2$;
		\item $f(x) < 0$ in the regime $x < - r_c \mu / m_1$;
		\item $f(x) > 0$ in the regime $x > r_c \mu / m_2$;
		\item $f(x)$ crosses the $x$ axis somewhere in the regime $- r_c \mu / m_1 < x < r_c \mu / m_2$;
		\item $\dv*{f}{x} < 0$ for all defined values of $x$ because it is dominated by negative powers of $x$.
	\end{itemize}
	Based on these observations, we can sketch $f(x)$ and $x$ as shown in Fig.~\ref{fig1a}.  The three intersection points indicate that there are three real roots of \refeq{y0case}.  These are the first three Lagrange points.
	
	In the case $y \neq 0$, \refeq{simpley} may be written
	\begin{equation} \label{simplery}
		\frac{\omega^2}{G} = \frac{m_1}{r_{13}^3} + \frac{m_2}{r_{23}^3}.
	\end{equation}
	Substituting \refeq{simplery} into \refeq{simplex},
	\begin{equation}
		\left (\frac{m_1}{r_{13}^3} + \frac{m_2}{r_{23}^3} \right) x = \frac{m_1}{r_{13}^3} \left( x + \frac{r_c \mu}{m_1} \right) + \frac{m_2}{r_{23}^3} \left( x - \frac{r_c \mu}{m_2} \right) \implies \frac{m_1}{r_{13}^3} \frac{r_c \mu}{m_1} = \frac{m_2}{r_{23}^3} \frac{r_c \mu}{m_2} \implies r_{13} = r_{23}.
	\end{equation}
	\hl{Applying this condition to} \refeq{rdefs},
	\begin{equation} \label{yn0case}
		\left( x + \frac{r_c \mu}{m_1} \right)^2 + y^2 = \left( x - \frac{r_c \mu}{m_2} \right)^2 + y^2 \implies 
	\end{equation}
	\hl{But why does this imply they equal} $r$?.
	Geometrically, this is only possible at two locations in the $(x, y)$ plane as shown in Fig.~\ref{fig1b}.  These are the final two Lagrange points, for a total of five as desired. \qed
\end{solution}
	
\unitlength=.35in
\begin{figure}[p] \centering \label{fig1a}
	\begin{picture}(10.5,10.5)(-5,-5)
		{\color{lightgray}
		\thinlines
		\multiput(-5,-4)(0,1){9}{\line(1,0){10}}
		\multiput(-4,-5)(1,0){9}{\line(0,1){10}}
		}
		\thicklines
		\put(-5,0){\vector(1,0){10.2}}
		\put(0,-5){\vector(0,1){10.2}}
		\put(5.3,0){\makebox(1,0)[l]{$x$}}
		\put(0,5.3){\makebox(0,1)[b]{}}
	\end{picture}
	\caption{Three Lagrange points, indicated by roots of \refeq{y0case}.}
\end{figure}
	
\begin{figure} \centering \label{fig1b}
	\begin{picture}(10.5,10.5)(-5,-5)
		{\color{lightgray}
		\thinlines
		\multiput(-5,-4)(0,1){9}{\line(1,0){10}}
		\multiput(-4,-5)(1,0){9}{\line(0,1){10}}
		}
		\thicklines
		\put(-5,0){\vector(1,0){10.2}}
		\put(0,-5){\vector(0,1){10.2}}
		\put(5.3,0){\makebox(1,0)[l]{$x$}}
		\put(0,5.3){\makebox(0,1)[b]{$y$}}
	\end{picture}
	\caption{Two more Lagrange points, found by the geometrical argument implied by \refeq{yn0case}.}
\end{figure}

\section{Spherical pendulum}

\begin{statement}
	A point mass $m$ in three spatial dimensions is connected by a light inextensible string of length $\ell$ to a fixed pivot and experiences a uniform gravitational field.  Use spherical polar coordinates $(\rho, \theta, \phi)$, where $\rho$ is the radial distance, $\theta$ the relative inclination with respect to the downward vertical, and $\phi$ the azimuthal angle.
\end{statement}

\begin{problem}
	Is this mechanical system integrable?  In other words, does this problem have as many independent conserved quantities as there are unknown dynamical variables?
\end{problem}

\begin{solution}
	\hl{Yes, energy and angular momentum are conserved and there are two degrees of freedom }$\theta$ and $\phi$
\end{solution}

\begin{problem} \label{2b}
	When appropriately simplified, the motion of the spherical pendulum reduces to one-dimensional motion of a point mass in an effective potential.  Find the effective potential.
\end{problem}

\begin{solution}
	Fix $\phi \implies$ \hl{normal pendulum?}
\end{solution}

\section{Spherical pendulum, continued}

\begin{problem}
	Write down the Hamiltonian describing the one-dimensional motion in problem \ref{2b}.  Sketch some time-evolution trajectories in phase space.  Make sure you include all qualitatively different features and indicate the direction of time evolution.  If there are fixed points corresponding to states in mechanical equilibrium, identify them.  If there is a separatrix, a trajectory separating qualitatively different motion, write down the equation describing its shape and specify its energy.  (You do not need to solve the equation.)
\end{problem}

\begin{solution}
	\hl{If this is the normal pendulum, it's exactly what we did in class?}
\end{solution}

\begin{problem}
	Using the results derived earlier, give a simple qualitative description of the spherical pendulum motion in three-dimensional space.
\end{problem}

\begin{solution}
	\hl{Sinusoidal on the surface of a sphere}
\end{solution}

\section{Double-well potential}

\begin{statement}
	A particle with mass $m$ is confined to one dimension and placed in the potential
	\begin{equation}
		U(q) = U_0 - \frac{q^2}{2} + \frac{q^4}{4}.
	\end{equation}
\end{statement}

\begin{problem}
	Write down the Lagrangian and the Hamiltonian.
\end{problem}

\newcommand{\qd}{\dot{q}}

\begin{solution}
	The Lagrangian is
	\begin{equation} \label{lagr2well}
		L = T - U = m \frac{\qd^2}{2} - U_0 + \frac{q^2}{2} - \frac{q^4}{4},
	\end{equation}
	and the Hamiltonian is
	\begin{equation} \label{ham2well} 
		H = T + U = m \frac{\qd^2}{2} + U_0 - \frac{q^2}{2} + \frac{q^4}{4}.
	\end{equation}
\end{solution}

\begin{problem}
	Sketch some time evolution trajectories in phase space.  Make sure you include all qualitatively different features and indicate the direction of time evolution.  If there are fixed points corresponding to states in mechanical equilibrium, identify them.  If there is a separatrix, a trajectory separating qualitatively different motion, write down the equation describing its shape and specify its energy.
\end{problem}

\newcommand{\qdd}{\ddot{q}}
\newcommand{\pd}{\dot{p}}
\newcommand{\dlq}{\delta q}
\newcommand{\dlp}{\delta p}

\begin{solution}
%	The Euler-Lagrange equation corresponding to the Lagrangian \refeq{lagr2well} is
%	\begin{equation}
%		0 = \pdv{L}{q} - \dv{}{t} \pdv{L}{\qd} = q - q^3 - m \qdd
%	\end{equation}
	Using the Lagrangian of \refeq{lagr2well}, the generalized momentum corresponding to the Hamiltonian in \refeq{ham2well} is
	\begin{equation}
		p = \pdv{L}{\qd} = m\qd.
	\end{equation}
	Making this substitution in \refeq{ham2well} gives
	\begin{equation}
		H = \frac{p^2}{2m} + U_0 - \frac{q^2}{2} + \frac{q^4}{4}.
	\end{equation}
	Then Hamilton's equations for the system are
	\begin{align}
		\qd &= \pdv{H}{p} = \frac{p}{m} \equiv f(q, p) \implies p = m\qd, \label{qdot} \\
		\pd &= -\pdv{H}{q} = q - q^3 \equiv g(q, p), \label{pdot}
	\end{align}
	where we have defined $f(q, p)$ and $g(q, p)$.
	
	We are interested in the phase space $(q, p)$.  Fixed points $(q^*, p^*)$ occur where $\pd = \qd = 0$.  Making the appropriate equality from \refeq{pdot} and \refeq{qdot}, the fixed points are the roots of the equation
	\begin{equation} \label{fps}
		q^* - (q^*)^3 = p^*.
	\end{equation}
	By inspection of \refeq{fps}, the fixed points are located at
	\begin{equation}
		(q^*, p^*) = (0,0), (\pm 1, 0).
	\end{equation}
	We may linearize the system of two equations \refeq{pdot} and \refeq{qdot} to determine the staibility of the fixed points.  Consider some small perturbation $\dlq = q - q^*$, $\dlp = p - p^*$ from a fixed point.  Performing a Taylor series expansion about a fixed point $(q^*, p^*)$ gives the general expressions
	\begin{align}
		\dv{\,\dlq}{t} &= \qd = f(\dlq + q^*, \dlp + p^*) = f(q^*, p^*) + \left. \dlq \pdv{f}{q} \right|_{q^*} + \left. \dlp \pdv{f}{p} \right|_{p^*} + \order{\dlq^2, \dlp^2, \dlq \,\dlp}, \label{taylorq} \\
		\dv{\,\dlp}{t} &= g(q^*, p^*) + \left. \dlq \pdv{g}{q} \right|_{q^*} + \left. \dlp \pdv{g}{p} \right|_{p^*} + \order{\dlq^2, \dlp^2, \dlq \,\dlp}. \label{taylorp}
	\end{align}
%	Neglecting higher-order terms, we can write down the linearized system in the basis $\dlq, \dlp$ using a matrix notation:
%	\begin{equation}
%		\mqty[\dv*{\,\dlq}{t} \\ \dv*{\,\dlp}{t} ] = \mqty[\left. \pdv*{f}{q} \right|_{q^*} & \left. \pdv*{f}{p} \right|_{p^*} \\ \left. \pdv*{g}{q} \right|_{q^*} & \left. \pdv*{g}{p} \right|_{p^*} ] \mqty[\dlq \\ \dlp] \equiv J \mqty[\dlq \\ \dlp],
%	\end{equation}
%	where we have defined the Jacobian matrix $J$.
	Evaluating \refeq{taylorq} and \refeq{taylorp} for $f(q, p)$ and $g(q, p)$ as defined in \refeq{qdot} and \refeq{pdot} yields the linearized system
	\begin{align}
		\dv{\,\dlq}{t} &\approx f(q^*, p^*) + \dlp\, \frac{p^*}{m}, &
		\dv{\,\dlp}{t} &\approx g(q^*, p^*) + \dlq\, (1 - 3 (q^*)^2).
	\end{align}
	For the fixed point $(0, 0)$,
	\begin{align}
		\dv{\,\dlq}{t} &\approx 0, &
		\dv{\,\dlp}{t} &\approx \dlq,
	\end{align}
	so this point is \hl{unstable.}  The change in momentum of the particle scales as its distance from the equlibrium point.

	For the fixed point $(\pm 1, 0)$,
	\begin{align}
		\dv{\,\dlq}{t} &\approx 0, &
		\dv{\,\dlp}{t} &\approx -2 \,\dlq,
	\end{align}
	so these points are stable.  A perturbation 
\end{solution}
	
In writing these solutions, I consulted David Tong's lecture notes and Steven Strogatz'z \emph{Nonlinear Dynamics and Chaos}.

\end{document}