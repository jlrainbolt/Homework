\documentclass[11pt]{article}
\usepackage{geometry, titlesec}
\usepackage[parfill]{parskip}
\usepackage{physics, amsfonts, amsthm}
\usepackage{fullpage}
\usepackage{fancyhdr}
\usepackage{enumitem}
\usepackage{xcolor, soul}
%\allowdisplaybreaks

\renewcommand{\thesubsection}{\thesection.\alph{subsection}}

\makeatletter
\renewcommand*\env@cases[1][1.2]{%
  \let\@ifnextchar\new@ifnextchar
  \left\lbrace
  \def\arraystretch{#1}%
  \array{@{}l@{\quad}l@{}}%
}
\makeatother
 
 
\setlist[enumerate]{leftmargin=*}
\pagestyle{fancy}
\fancyhf{}
\lhead{\textbf{Physics 316 Homework 3}}
\rhead{Lacey Rainbolt}
\setlength{\headheight}{14pt}
\setlength{\headsep}{12pt}
\cfoot{\thepage}

\titleformat{\subsection}[runin]{\normalfont\large\bfseries}{\thesubsection}{1em}{}
\newcommand{\refeq}[1]{(\ref{#1})}


\newenvironment{statement}
{
    \color{darkgray}
    \ignorespaces
}
{
    \smallskip
}

\newenvironment{problem}
{
    \subsection{}
    \color{darkgray}
    \ignorespaces
}


\newenvironment{solution}
{
    \paragraph{Solution.}
    \ignorespaces
}
{
    \bigskip
}

\renewcommand{\vec}[1]{\mathbf{#1}}



\begin{document}

\newcommand{\omo}{\omega_0}
\newcommand{\omp}{\omega_+}
\newcommand{\omm}{\omega_-}
\newcommand{\ompm}{\omega_\pm}
\newcommand{\lmp}{\lambda_+}
\newcommand{\lmm}{\lambda_-}
\newcommand{\lmpm}{\lambda_\pm}

\newcommand{\vx}{\vec{x}}
\newcommand{\xq}{x_1}
\newcommand{\xw}{x_2}
\newcommand{\xddq}{\ddot{x}_1}
\newcommand{\xddw}{\ddot{x}_2}

\section{}
\begin{statement}
	Two identical harmonic oscillators with mass $M$ and natural frequency $\omo$ are coupled to each other, and to an extra mass $m$, such that the equations of motion have the form
	\begin{align}
		\xddq + \frac{m}{M} \xddw + \omo^2 \xq &= 0, \label{given1} \\
		\xddw + \frac{m}{M} \xddq + \omo^2 \xw &= 0. \label{given2}
	\end{align}
	What are the normal mode frequencies?
\end{statement}

\begin{solution}
	We will begin by rewriting \refeq{given1} and \refeq{given2} such that they have no cross terms.  Solving \refeq{given1} for $\xddw$ and \refeq{given2} for $\xddq$ gives us
	\begin{align}
		\xddw &= -\frac{M}{m} (\xddq + \omo^2 \xq), \label{solved2} \\
		\xddq &= -\frac{M}{m} (\xddw + \omo^2 \xw). \label{solved1}
	\end{align}
	Now substituting \refeq{solved2} into \refeq{given2} and \refeq{solved1} into \refeq{given1} gives us
	\begin{align}
		-\frac{M}{m} (\xddq + \omo^2 \xq) + \frac{m}{M} \xddq + \omo^2 \xw = 0 &\implies \xddq = \frac{\omo^2}{m^2 - M^2} (M^2 \xq - M m \xw), \label{new1} \\
		-\frac{M}{m} (\xddw + \omo^2 \xw) + \frac{m}{M} \xddw + \omo^2 \xq = 0 &\implies \xddw = \frac{\omo^2}{m^2 - M^2} (M^2 \xw - M m \xq). \label{new2}
	\end{align}
	Then \refeq{new1} and \refeq{new2} may be rewritten in a matrix form in the basis $\vx$:
	\begin{align}
		\dv{}{t} \mqty[ \xq \\ \xw ] = \frac{\omo^2}{m^2 - M^2} \mqty[ M^2 & -Mm \\ -Mm & M^2 ] \mqty[ \xq \\ \xw ] \equiv \frac{\omo^2}{m^2 - M^2} A \vx,
	\end{align}
	where we have defined the matrix $A$ and the vector $\vx$.
	
	Let $\lmpm$ be the two eigenvalues of $A$.  The eigenvalues are given by $\det(A - \lambda I) = 0$, where $I$ is the identity matrix.  That is,
	\begin{align}
		0 = \mdet{M^2 - \lambda & -Mm \\ -Mm & M^2 - \lambda} = (M^2 - \lambda)^2 - M^2 m^2 \implies M^2 - \lambda = \pm Mm \implies \lmpm = M^2 \pm Mm.
	\end{align}
	Let $\ompm$ be the normal mode frequencies, which are given by $\ompm^2 = \omo^2 \lmpm / (m^2 - M^2)$.  Explicitly,
	\begin{align}
		\omp &= \omo \sqrt{\frac{M^2 + Mm}{m^2 - M^2}}, &
		\omm &= \omo \sqrt{\frac{M^2 - Mm}{m^2 - M^2}}.
	\end{align}
\end{solution}

\newcommand{\mq}{m_1}
\newcommand{\mw}{m_2}
\newcommand{\elq}{\ell_1}
\newcommand{\elw}{\ell_2}

\newcommand{\thq}{\theta_1}
\newcommand{\thw}{\theta_2}
\newcommand{\thdq}{\dot{\theta}_1}
\newcommand{\thdw}{\dot{\theta}_2}
\newcommand{\thddq}{\ddot{\theta}_1}
\newcommand{\thddw}{\ddot{\theta}_2}

\newcommand{\vrq}{\vec{r}_1}
\newcommand{\vrw}{\vec{r}_2}
\newcommand{\xdq}{\dot{x}_1}
\newcommand{\xdw}{\dot{x}_2}
\newcommand{\yq}{y_1}
\newcommand{\yw}{y_2}
\newcommand{\ydq}{\dot{y}_1}
\newcommand{\ydw}{\dot{y}_2}

\newcommand{\Lh}{\hat{L}}
\newcommand{\vth}{\boldsymbol{\theta}}

\section{Designing a Double Pendulum} \label{prob2}
\begin{statement}
	Suppose you are asked to design a double pendulum whose lower frequency is half that of the higher frequency by changing the lengths of the strings and/or the masses.  What are the possible designs?
\end{statement}

\unitlength=.35in
\begin{figure}[b] \centering
	\begin{picture}(10.5,10.5)(-1.2,-5)
		{\color{lightgray}
		\thinlines
		\multiput(-1,-4)(0,1){9}{\line(1,0){9}}
		\multiput(0,-5)(1,0){8}{\line(0,1){10}}
		}
		\thicklines
		\put(-1,4){\vector(1,0){9.2}}
		\put(0,5){\vector(0,-1){10.2}}
		\put(8.3,4){\makebox(1,0)[l]{$x$}}
		\put(0,-5){\makebox(0,-1)[c]{$y$}}
	\end{picture}
	\medskip
	\caption{Double pendulum in problem~\ref{prob2} with labeled coordinates.}
	\label{fig:dp}
\end{figure}

\begin{solution}
	A labeled diagram of the system is shown in figure~\ref{fig:dp}.  Let the upper part of the pendulum have mass $\mq$ and string length $\elq$.  Let its position be $\vrq = (\xq, \yq)$ where the pivot is located at the origin, and the $y$ axis points downward.  Define $\mw$, $\elw$, and $\vrw$ similarly for the lower part.  Then the Lagrangian for the system is given by
	\begin{equation} \label{lagrdp}
		L = T_1 + T_2 - U_1 - U_2 = \frac{1}{2} \mq (\xdq^2 + \ydq^2) + \frac{1}{2} \mw (\xdw^2 + \ydw^2) - m g \yq - m g \yw.
	\end{equation}
	Define the generalized coordinates $\thq, \thw$ which represent the inclination of each mass with respect to the vertical.  Then the Cartesian coordinates representing the position of each mass are
	\begin{align}
		\xq &= \elq \sin{\thq}, &
		\yq &= \elq \cos{\thq}, \\
		\xw &= \elq \sin{\thq} + \elw \sin{\thw}, &
		\yw &= \elq \cos{\thq} + \elw \cos{\thw},
	\end{align}
	which have the time derivatives
	\begin{align}
		\xdq &= \pdv{\xq}{\thq} \dv{\thq}{t} = \elq \cos{\thq} \thdq, \label{x1d} \\
		\ydq &= \pdv{\yq}{\thq} \dv{\thq}{t} = -\elq \sin{\thq} \thdq, \label{y1d} \\
		\xdw &= \pdv{\xw}{\thq} \dv{\thq}{t} + \pdv{\xw}{\thw} \dv{\thw}{t} = \elq \cos{\thq} \thdq + \elw \cos{\thw} \thdw, \label{x2d} \\
		\ydw &= \pdv{\yw}{\thq} \dv{\thq}{t} + \pdv{\yw}{\thw} \dv{\thw}{t} = -\elq \sin{\thq} \thdq - \elw \sin{\thw} \thdw. \label{y2d}
	\end{align}
	From \refeq{x1d} and \refeq{y1d},
	\begin{equation}
		\xdq^2 + \ydq^2 = \elq^2 \thdq^2 (\cos^2{\thq} + \sin^2{\thq}) = \elq^2 \thdq^2.
	\end{equation}
	From \refeq{x2d} and \refeq{y2d},
	\begin{align}
		\xdw^2 &= \elq^2 \cos^2{\thq} \thdq^2 + 2 \elq \elw \cos{\thq} \cos{\thw} \thdq \thdw + \elw^2 \cos^2{\thw} \thdw^2, \\
		\ydw^2 &= \elq^2 \sin^2{\thq} \thdq^2 + 2 \elq \elw \sin{\thq} \sin{\thw} \thdq \thdw + \elw^2 \sin^2{\thw} \thdw^2,
	\end{align}
	so
	\begin{align}
		\xdw^2 + \ydw^2 &= \elq^2 \thdq^2 + \elw^2 \thdw^2 + 2 \elq \elw \thdq \thdw (\cos{\thq} \cos{\thw} + \sin{\thq} \sin{\thw}) \\
		&= \elq^2 \thdq^2 + \elw^2 \thdw^2 + 2 \elq \elw \cos(\thq - \thw) \thdq \thdw.
	\end{align}
	Writing \refeq{lagrdp} in terms of the generalized coordinates, we have
	\begin{align}
		L &= \frac{1}{2} \mq \elq^2 \thdq^2 + \frac{1}{2} \mw \left( \elq^2 \thdq^2 + \elw^2 \thdw^2 + 2 \elq \elw \cos(\thq - \thw) \thdq \thdw \right) + \mq g \elq \cos{\thq} \notag \\
		&\phantom{=\ } + \mw g \left( \elq \cos{\thq} + \elw \cos{\thw} \right) \\
		&= \frac{1}{2} (\mq + \mw) \elq^2 \thdq^2 + \frac{1}{2} \mw \elw^2 \thdw^2 + \mw \elq \elw \cos(\thq - \thw) \thdq \thdw + g (\mq + \mw) \elq \cos{\thq} \notag \\
		&\phantom{=\ } + g \elw \mw \cos{\thw}. \label{lagrdpnew}
	\end{align}
	
	The stable equilibrium solution is for the pendulum hanging straight down, which is at the point $(\thq^*, \thw^*) = (0, 0)$.  We will linearize the $L = L(\thq, \thw, \thdq, \thdw)$ given by \refeq{lagrdpnew} about this stable point in order to find general expressions for the normal modes.  Note that
	\begin{align}
		\pdv{L}{\thq} &= -\mw \elq \elw (\sin{\thq} \cos{\thw} - \cos{\thq} \sin{\thw}) \thdq \thdw - g (\mq + \mw) \elq \sin{\thq} \\
		&= -\mw \elq \elw \sin(\thq - \thw) \thdq \thdw - g (\mq + \mw) \elq \sin{\thq}, \\
		\pdv{L}{\thw} &= \mw \elq \elw (\sin{\thq} \cos{\thw} - \cos{\thq} \sin{\thw}) \thdq \thdw - g \mw \elw \sin{\thw} \\
		&= \mw \elq \elw \sin(\thq - \thw) \thdq \thdw - g \mw \elw \sin{\thw},
	\end{align}
	which implies
	\begin{equation}
		\left. \pdv{L}{\thq}\right|_{0, 0} \thq = \left. \pdv{L}{\thw}\right|_{0, 0} \thw = 0.
	\end{equation}
	Thus, we must expand to second order.  Note that
	\begin{align}
		\pdv[2]{L}{\thq} &= -\mw \elq \elw (\cos{\thq} \cos{\thw} + \sin{\thq} \sin{\thw}) \thdq \thdw - g (\mq + \mw) \elq \cos{\thq} \\
		&= -\mw \elq \elw \cos(\thq - \thw) \thdq \thdw - g (\mq + \mw) \elq \cos{\thq}, \\
%		\pdv[2]{L}{\thq}{\thw} &= \mw \elq \elw (\cos{\thq} \cos{\thw} + \sin{\thq} \sin{\thw}) \thdq \thdw = \mw \elq \elw \cos(\thq - \thw) \thdq \thdw \\
		\pdv[2]{L}{\thw} &= -\mw \elq \elw (\cos{\thq} \cos{\thw} + \sin{\thq} \sin{\thw}) \thdq \thdw - g \mw \elw \cos{\thw} \\
		&= -\mw \elq \elw \cos(\thq - \thw) \thdq \thdw - g \mw \elw \cos{\thw}.
	\end{align}
	Then, expanding to second order in $\thq$ and $\thw$, %FIXME
	\begin{align}
%		L(\thq, \thw, \thdq, \thdw) &\approx L(0, 0, \thdq, \thdw) + \frac{1}{2} \left. \pdv[2]{L}{\thq}\right|_{0, 0} \thq^2 + \left. \pdv[2]{L}{\thq}{\thw}\right|_{0, 0} \thq \thw + \frac{1}{2} \left. \pdv[2]{L}{\thw}\right|_{0, 0} \thw^2 \\
		L &\approx L(0, 0, \thdq, \thdw) + \frac{1}{2} \left. \pdv[2]{L}{\thq}\right|_{0, 0} \thq^2 + \frac{1}{2} \left. \pdv[2]{L}{\thw}\right|_{0, 0} \thw^2 \\
		&\approx \frac{1}{2} (\mq + \mw) \elq^2 \thdq^2 + \frac{1}{2} \mw \elw^2 \thdw^2 + \mw \elq \elw \thdq \thdw + g (\mq + \mw) \elq + g \elw \mw \notag \\
		&\phantom{\approx\ } + \frac{1}{2} \left( -\mw \elq \elw \thdq \thdw - g (\mq + \mw) \elq \right) \thq^2 + \frac{1}{2} \left( -\mw \elq \elw \thdq \thdw - g \mw \elw \right) \thw^2 \\
		&\approx \frac{1}{2} (\mq + \mw) \elq^2 \thdq^2 + \frac{1}{2} \mw \elw^2 \thdw^2 + \mw \elq \elw \thdq \thdw - \frac{1}{2} g (\mq + \mw) \elq \thq^2 - \frac{1}{2} g \mw \elw \thw^2, \label{lagrlin}
	\end{align}
	where in going to \refeq{lagrlin} we have omitted constant terms and terms proportional to $\thdq \thdw \thq^2$ and to $\thdq \thdw \thw^2$.
	
	Now we can obtain the equations of motion for the linearized Lagrangian \refeq{lagrlin}, which we will call $\Lh$:
	\begin{align}
		0 &= \pdv{\Lh}{\thq} - \dv{}{t} \pdv{\Lh}{\thdq} = (\mq + \mw) \elq^2 \thddq + \mw \elq \elw \thddw + g (\mq + \mw) \elq \thq, \label{el1} \\
		0 &= \pdv{\Lh}{\thw} - \dv{}{t} \pdv{\Lh}{\thdw} = \mw \elq \elw \thddq + \mw \elw^2 \thddw + g \mw \elw \thw. \label{el2}
	\end{align}
	Solving \refeq{el1} for $\thddw$ and \refeq{el2} for $\thddq$ gives us
	\begin{align}
		\thddw &= -\frac{1}{\mw \elq \elw} \left( g (\mq + \mw) \elq \thq + (\mq + \mw) \elq^2 \thddq \right), \label{2solved2} \\
		\thddq &= -\frac{1}{\mw \elq \elw} \left( \mw \elw^2 \thddw + g \mw \elw \thw \right) \label{2solved1}.
	\end{align}
	Substituting \refeq{2solved2} into \refeq{el2} gives us
	\begin{align}
		0 &= \mw \elq \elw \thddq - \frac{\elw}{\elq} \left( g (\mq + \mw) \elq \thq + (\mq + \mw) \elq^2 \thddq \right) + g \mw \elw \thw \\
		&\implies \mq \elq \elw \thddq = -g (\mq + \mw) \elw \thq + g \mw \elw \thw, \label{mot1}
	\end{align}
	and substituting \refeq{2solved1} into \refeq{el1} gives us
	\begin{align}
		0 &= -(\mq + \mw) \elq \left( \elw \thddw + g \thw \right) + \mw \elq \elw \thddw + g (\mq + \mw) \elq \thq \\
		&\implies \mq \elq \elw \thddw = g (\mq + \mw) \elq \thq - g (\mq + \mw) \elq \thw. \label{mot2}
	\end{align}
	Then \refeq{mot1} and \refeq{mot2} may be rewritten in a matrix form in the basis $\vth$:
	\begin{equation}
		\dv{}{t} \mqty[ \thq \\ \thw ] = -\frac{g (\mq + \mw)}{\mq \elq \elw} \mqty[ \elw & -\mw \elw / (\mq + \mw) \\ -\elq & \elq ] \mqty[ \thq \\ \thw ] \equiv -\frac{g (\mq + \mw)}{\mq \elq \elw} A \vth,
	\end{equation}
	where we have defined the matrix $A$ and the vector $\vth$.
	
	Let $\lmpm$ be the two eigenvalues of $A$.  Then
	\begin{align}
		0 &= \mdet{\elw - \lambda & -\mw \elw / (\mq + \mw) \\ -\elq & \elq - \lambda} = (\elq - \lambda) (\elw - \lambda) - \frac{\mw}{\mq + \mw} \elq \elw \\
		&= \lambda^2 - (\elq + \elw) \lambda + \frac{\mq}{\mq + \mw} \elq \elw
	\end{align}
	which implies
	\begin{equation} \label{eigenvalues}
		\lmpm = \frac{1}{2} \left( (\elq + \elw) \pm \sqrt{(\elq + \elw)^2 - 4 \elq \elw \frac{\mq}{\mq + \mw}} \right).
	\end{equation}
	Let $\ompm$ be the normal mode frequencies, which are given by
	\begin{equation}
		\ompm^2 = -\frac{g (\mq + \mw)}{\mq \elq \elw} \lmpm.
	\end{equation}
	The higher frequency is $\omm$.  In order for the lower frequency to be half of this, we need $\omm^2 = 4 \omp^2$, or equivalently $\lmm = 4 \lmp$.  Using \refeq{eigenvalues}, this gives us the condition
	\begin{align}
		\left( (\elq + \elw) - \sqrt{(\elq + \elw)^2 - 4 \elq \elw \frac{\mq}{\mq + \mw}} \right) &= 4 \left( (\elq + \elw) + \sqrt{(\elq + \elw)^2 - 4 \elq \elw \frac{\mq}{\mq + \mw}} \right) \\
		-3 (\elq + \elw) &= 5 \sqrt{(\elq + \elw)^2 - 4 \elq \elw \frac{\mq}{\mq + \mw}} \\
		9 (\elq + \elw)^2 &= 25 \left( (\elq + \elw)^2 - 4 \elq \elw \frac{\mq}{\mq + \mw} \right) \\
		100 \elq \elw \frac{\mq}{\mq + \mw} &= 16 (\elq + \elw)^2 \\
		\frac{\elq}{\elq + \elw} \frac{\elw}{\elq + \elw} \frac{\mq}{\mq + \mw} &= 0.16. \label{designs}
	\end{align}
	The possible designs are those that satisfy \refeq{designs}.
\end{solution}


\section{Beats and Double Pendulum}
\begin{statement}
	Given a double pendulum whose two strings are of equal length, how should the masses be chosen so that the two eigenfrequencies approach each other, i.e.~that the system approaches a degeneracy?  Show that the resultant motion proceeds in ``beats.''
\end{statement}

%\begin{solution}
%
%\end{solution}


\section{Triple Oscillator System}
\begin{statement}
	Consider three identical masses connected by identical springs in the shape of an equilateral triangle.  Suppose the three springs lie along the arcs of a circle that circumscribes the triangle.  Suppose also that the motion of the masses is constrained to move along the circle.  Find the normal modes and the eigenfrequencies about the equilibrium state.  If there is a zero mode, identify the associated continuous symmetry.
\end{statement}

%\begin{solution}
%
%\end{solution}

	
In writing these solutions, I consulted David Tong's lecture notes.

\end{document}