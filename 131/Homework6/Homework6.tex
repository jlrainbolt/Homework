\documentclass[11pt]{article}
\usepackage{geometry, titlesec}
\usepackage[parfill]{parskip}
\usepackage[italicdiff]{physics}
\usepackage{amsfonts, amsthm}
\usepackage[cm]{fullpage}
\usepackage{fancyhdr}
\usepackage{enumitem}
\usepackage{xcolor, soul}
\usepackage{siunitx}
%\allowdisplaybreaks

\renewcommand{\thesubsection}{\thesection.\alph{subsection}}
\setenumerate[1]{label={(\alph*)}}

\makeatletter
\renewcommand*\env@cases[1][1.2]{%
  \let\@ifnextchar\new@ifnextchar
  \left\lbrace
  \def\arraystretch{#1}%
  \array{@{}l@{\quad}l@{}}%
}
\makeatother
 
\renewcommand{\footrulewidth}{.2pt}
%\setlist[enumerate]{leftmargin=*}
\pagestyle{fancy}
\fancyhf{}
\lhead{Physics 131-B}
\chead{\textbf{Homework 6 Solutions}}
\rhead{Lacey Rainbolt}
\setlength{\headheight}{11pt}
\setlength{\headsep}{11pt}
\setlength{\footskip}{24pt}
\lfoot{\today}
\rfoot{\thepage}

\titleformat{\subsection}[runin]{\normalfont\large\bfseries}{\thesubsection}{1em}{}
\newcommand{\refeq}[1]{(\ref{#1})}

\newcommand{\beq}{\begin{equation*}}
\newcommand{\eeq}{\end{equation*}}

\newcommand{\beqn}{\begin{equation}}
\newcommand{\eeqn}{\end{equation}}

\newcommand{\blg}{\begin{align*}}
\newcommand{\elg}{\end{align*}}


\newenvironment{statement}
{
    \color{darkgray}
    \ignorespaces
}
{
%    \smallskip
}

\newenvironment{problem}
{
%    \color{darkgray}
    \ignorespaces
}

\newenvironment{solution}
{
    \paragraph{Solution.}
    \ignorespaces
}
{
    \bigskip
}

\renewcommand{\vec}[1]{\mathbf{#1}}


\begin{document}

\begin{figure}
	\vspace{1in}
	\caption{\textbf{E10.2}}
	\label{E10.2}
\end{figure}

\paragraph{Exercise 10.2}
\begin{problem}
	Calculate the net torque about point $O$ for the two forces applied as in Fig.~\ref{E10.2}.  The rod and both forces are in the plane of the page.
\end{problem}

%\begin{figure} \centering
	
%	\label{E10.2}
%\end{figure}

\newcommand{\ih}{\vec{\,\hat{i}}}
\newcommand{\jh}{\vec{\,\hat{j}}}
\newcommand{\kh}{\vec{\,\hat{k}}}

\newcommand{\vF}{\vec{F}}
\newcommand{\vt}{\boldsymbol{\tau}}
\newcommand{\vr}{\vec{r}}

\newcommand{\vFq}{\vF_1}
\newcommand{\Fq}{F_1}
\newcommand{\vFw}{\vF_2}
\newcommand{\Fw}{F_2}

\newcommand{\vtq}{\vt_1}
\newcommand{\vtw}{\vt_2}
\newcommand{\vrq}{\vr_1}
\newcommand{\rrq}{r_1}
\newcommand{\vrw}{\vr_2}
\newcommand{\rw}{r_2}

\newcommand{\vtnet}{\vt_\text{net}}
\newcommand{\tnet}{\tau_\text{net}}


\begin{solution}
	This problem is a simple plug and chug to help us practice calculating torques.  The net torque on the rod is the sum of the torques due to each of the forces:
	\beqn \label{nett}
		\vtnet = \vtq + \vtw.
	\eeqn
	Here, $\vtq$ is the~(vector) torque due to $\vFq$ relative to point $O$.  Let $\vrq$ be the vector from $O$ to where $\vFq$ acts on the rod.  Using the coordinate axes drawn in Fig.~\ref{E10.2}, we have
	\begin{align*}
		\vrq &= -\rrq \ih, \\
		\vFq &= -\Fq \jh,
	\end{align*}
	where $\rrq = \SI{5.00}{\meter}$ and $\Fw = \SI{8.00}{\newton}$.  Then
	\beq
		\vtq = \vrq \times \vFq = \rrq \Fq \kh.
	\eeq
	Now for $\vtq$, define $\vrw$ as the vector from $O$ to where $\vFw$ acts on the rod.  Then
		\begin{align*}
		\vrw &= -\rw \ih, \\
		\vFw &= -(\Fw \cos{30^\circ}) \ih + (\Fw \sin{30^\circ}) \jh = -\frac{\sqrt{3}}{2} \Fw \ih + \frac{1}{2} \Fw \jh,
	\end{align*}
	where $\rw = \SI{2.00}{\meter}$ and $\Fw = \SI{12.0}{\newton}$, and
	\beq
		\vtw = \vrw \times \vFw = -\frac{1}{2} \rw \Fw \kh.
	\eeq
	Feeding our results into \refeq{nett},
	\beq
		\vtnet = \left( \rrq \Fq - \frac{1}{2} \rw \Fw \right)\!\kh = \tnet \kh,
	\eeq
	where $\tnet$ is the magnitude of the net torque.  Plugging everything in,
	\beq
		\tnet = \rrq \Fq - \frac{1}{2} \rw \Fw = (\SI{5.00}{\meter}) (\SI{8.00}{\newton}) - \frac{1}{2} (\SI{2.00}{\meter}) (\SI{12.0}{\newton}) = {\color{blue} \SI{28.0}{\newton\meter}}.
	\eeq
\end{solution}

\newcommand{\mb}{m_b}
\newcommand{\mw}{m_w}
\newcommand{\mmp}{m_p}

\newcommand{\simb}{\SI{12.0}{\kg} }
\newcommand{\simw}{\SI{5.00}{\kg} }
\newcommand{\simp}{\SI{2.00}{\kg} }
\newcommand{\sIip}{\SI{0.0625}{\kg\square\meter} }

\newcommand{\sig}{\SI{9.81}{\meter\per\square\second} }

\newcommand{\Ww}{W_w}
\newcommand{\aq}{a_b}

\newcommand{\Tb}{T_b}
\newcommand{\Tw}{T_w}

\newcommand{\az}{\alpha_z}

\newcommand{\Nx}{N_x}
\newcommand{\Ny}{N_y}

\clearpage
\paragraph{Exercise 10.16}
\begin{problem}
	A \simb box resting on a horizontal, frictionless surface is attached to a \simp weight by a thin, light wire that passes over a frictionless pulley (Fig.~\ref{E10.16}).  The pulley has the shape of a uniform solid disk of mass \simp and diameter \SI{0.500}{\meter}.  After the system is released, find
	\begin{enumerate}
		\item the tension in the wire on both sides of the pulley,
		\item the acceleration of the box, and
		\item the horizontal and vertical components of the force that the axle exerts on the pulley.
	\end{enumerate}
\end{problem}

\begin{figure}
	\vspace{1in}
	\caption{\textbf{E10.16}}
	\label{E10.16}
\end{figure}

\begin{figure}
	\vspace{1.5in}
	\caption{Free-body diagrams for 10.16(a).}
	\label{E10.16a}
\end{figure}

\begin{solution}
	The box and the weight must have the same acceleration $a$ since they are connected by the wire.  It is safe to assume the pulley rolls without slipping against the wire (otherwise it would not be a very effective pulley!), so its tangential acceleration is also $a$.  This is the key we need to solve the problem.
	
	We can draw a free-body diagram for each object, as shown in Fig.~\ref{E10.16a}.  Using these diagrams, we can write down three equations using Newton's second law: one for the box of mass $\mb$,
	\beqn \label{box}
		\mb a = \Tb,
	\eeqn
	one for the weight of mass $\mw$,
	\beqn \label{weight}
		\mw a = \mw g - \Tw.
	\eeqn
	and one for the pulley (which we will write generally for now),
	\beqn \label{pulley0}
		\tnet = I \alpha = I \frac{a}{r}.
	\eeqn
	Here, $I$ is the moment of inertia about the pulley's center and $\alpha$ its angular acceleration.  Since the pulley rolls without slipping, $\alpha = a / r$.
			
	$\Tw$ and $\Tb$ each exert a torque on the outer edge of the pulley, but in opposite directions.  We know the weight must be moving downward, meaning $\Tw$ has a greater magnitude, and so the pulley is rotating in the direction due to $\tau_w$.  The lever arm for each torque is the radius of the pulley $r$.  Putting this all together,
	\beq
		\tnet = \tau_w + \tau_b = \Tw r - \Tb r.
	\eeq
	The pulley is a solid cylinder rotating about its $z$ axis with mass $\mmp$.  Then
	\beq
		I = \frac{1}{2} \mmp r^2,
	\eeq
	and \refeq{pulley0} can be rewritten as follows:
	\beqn \label{pulley}
		(\Tw - \Tb) r = \frac{1}{2} \mmp r^2 \frac{a}{r} \implies \Tw - \Tb = \frac{1}{2} \mmp a.
	\eeqn
	The system of three equations \refeq{box}, \refeq{weight}, and \refeq{pulley} has three unknowns.  The solutions are
	\begin{align}
		\Tb &= 2g \frac{\mb \mw}{2 \mb + 2 \mw + \mmp}, \label{Tb} \\
		\Tw &= g \mw \frac{2 \mb + \mmp}{2 \mb + 2 \mw + \mmp}, \label{Tw} \\
		a &= 2g \frac{\mw}{2 \mb + 2 \mw + \mmp}. \label{acc}
	\end{align}
			
	\begin{enumerate}
		\item Plugging numbers into \refeq{Tb} and \refeq{Tw} gives us
			\begin{align*}
				\Tb &= 2 (\sig) \frac{(\simb) (\simw)}{2 (\simb) + 2 (\simw) + (\simp)} = {\color{blue} \SI{32.7}{\newton}}, \\
				\Tw &= (\sig) (\simw) \frac{2 (\simb) + (\simp)}{2 (\simb) + 2 (\simw) + (\simp)} = {\color{blue} \SI{35.4}{\newton}}.
			\end{align*}
		
		\item Plugging numbers into \refeq{acc} gives us
			\beq
				a = 2 (\sig) \frac{(\simw)}{2 (\simb) + 2 (\simw) + (\simp)} = {\color{blue} \SI{2.73}{\meter\per\square\second}}.
			\eeq
			
		\item The pulley is not moving up or down.  This means the axle must be exerting a normal force $\vec{N}$ upon its center of mass which exactly cancels all other forces acting upon it.  The relevant free-body diagram is shown in Fig.~\ref{E10.16c}.  Balancing forces in the vertical direction, we have
		\beq
			\Ny = \Tw + \mmp g = \SI{35.4}{\newton} + (\simp) (\sig) = {\color{blue} \SI{55.0}{\newton}},
		\eeq
		and in the horizontal direction,
		\beq
			\Nx = \Tb = {\color{blue} \SI{32.7}{\newton}}.
		\eeq
	\end{enumerate}
\end{solution}

\begin{figure}
	\vspace{1.5in}
	\caption{Free-body diagram for 10.16(c).}
	\label{E10.16c}
\end{figure}


\clearpage
\begin{figure}
	\vspace{1in}
	\caption{\textbf{E10.22}}
	\label{E10.22}
\end{figure}

\newcommand{\Ug}{U_g}

\paragraph{Exercise 10.22}
\begin{problem}
	A string is wrapped several times around the rim of a small hoop with radius \SI{8.00}{\centi\meter} and mass \SI{0.180}{\kilo\gram}.  The free end of the string is held in place and the hoop is released from rest (Fig.~\ref{E10.22}).  After the hoop has descended \SI{75.0}{\centi\meter}, calculate
	\begin{enumerate}
		\item the angular speed of the rotating hoop, and
		\item the speed of its center.
	\end{enumerate}
\end{problem}

\begin{solution}
	We can solve this problem using conservation of energy.  The ring starts from rest and then experiences both translational and rotational motion.  The change in its kinetic energy is given by
	\beq
		\Delta K = \frac{1}{2} m v^2 + \frac{1}{2} I \omega^2,
	\eeq
	where $v$ is the (translational) speed of its center, $I$ the moment of inertia about its center of mass, and $\omega$ is its angular speed.  For a hoop,
	\beq
		I = m r^2.
	\eeq
	Once the ring has descended $\Delta y = \SI{75.0}{\centi\meter}$, its gravitational potential energy $\Ug$ has decreased:
	\beq
		\Delta \Ug = -mg \,\Delta y.
	\eeq
	Invoking conservation of energy,
	\beqn \label{conse}
		\Delta K + \Delta U = 0 \implies mg \,\Delta y = \frac{1}{2} m v^2 + \frac{1}{2} m r^2 \omega^2.
	\eeqn
	We need to invoke another condition in order to solve for the \emph{two} unknowns $v$ and $\omega$.  Once again, this is rolling without slipping: the edge of the hoop must be moving at the same rate that the string is unwinding.  This means
	\beqn \label{roll}
		v = r \omega,
	\eeqn
	which we can substitute into \refeq{conse} to obtain
	\beqn \label{conse2}
		mg \,\Delta y = \frac{1}{2} m r^2 \omega^2 + \frac{1}{2} m r^2 \omega^2 \implies g \,\Delta y = r^2 \omega^2.
	\eeqn
	
	\begin{enumerate}
		\item Rearranging \refeq{conse2} and plugging in numbers,
			\beq
				\omega = \frac{\sqrt{g \,\Delta y}}{r} = \frac{\sqrt{(\sig) (\SI{0.750}{\meter})}}{\SI{8.00e-2}{\meter}} = {\color{blue} \SI{33.9}{\radian\per\second}}.
			\eeq
			
		\item Now plugging into \refeq{roll},
			\beq
				v = (\SI{8.00e-2}{\meter}) (\SI{33.9}{\radian\per\second}) = {\color{blue} \SI{2.71}{\meter\per\second}}.
			\eeq
	\end{enumerate}
\end{solution}

\clearpage
\begin{figure}
	\vspace{1.5in}
	\caption{\textbf{10.19}}
	\label{10.19}
\end{figure}

\paragraph{Exercise 10.30 (A Ball Rolling Uphill)}
\begin{problem}
	A bowling ball rolls without slipping up a ramp that slopes upward at and angle $\beta$ to the horizontal (see Example 10.7 and Fig.~\ref{10.19}).  Treat the ball as a uniform solid sphere, ignoring the finger holes.
	\begin{enumerate}
		\item Draw the free-body diagram for the ball.  Explain why the friction force must be directed \emph{uphill}.
		\item What is the acceleration of the center of mass of the ball?
		\item What minimum coefficient of static friction is needed to prevent slipping?
	\end{enumerate}
\end{problem}

\begin{solution}
	Example 10.7 describes the same situation, except the ball is rolling downhill.  We can use that worked example to solve this problem.
	
	\begin{enumerate}
		\item The free-body diagram is the same as for Example 10.7, and is shown in Fig.~\ref{10.19}(b).  The friction force must be directed uphill because it is preventing the ball from sliding down the ramp.  This is true even when the ball is rolling uphill instead of downhill, because the force of gravity is \emph{always} trying to make the ball slide downhill.
		\item The free-body diagram is the same as in Example 10.7, so the acceleration of the center of mass is also the same:
			\beq
				a_{\text{cm-}x} = {\color{blue} \frac{5}{7} g \sin{\beta}}.
			\eeq
		Taking note of the $x$ axis as drawn in Fig.~\ref{10.19}(b), we see that the acceleration points \emph{down} the hill.  This means the ball is slowing down as it rolls uphill, which is easily confirmed by physical intuition.
		
		\item We are looking to saturate the inequality
			\beqn \label{stf}
				f \leq \mu_s n,
			\eeqn
			where $f$ is the magnitude of the static friction force, $\mu_s$ is the coefficient of static friction, and $n$ is the magnitude of the normal force acting on the ball.  From Fig.~\ref{10.19}(b),
			\beq
				n = M g \cos{\beta},
			\eeq
		and from Example 10.7,
			\beq
				f = \frac{2}{7} M g \sin{\beta}.
			\eeq
			Now we just need to solve for $\mu_s$ in \refeq{stf}:
			\beq
				\mu_s = \frac{f}{n} = \frac{2}{7} \frac{M g \sin{\beta}}{M g \cos{\beta}} = {\color{blue} \frac{2}{7} \tan{\beta}}.
			\eeq
	\end{enumerate}
\end{solution}


\clearpage
\paragraph{Exercise 10.48 (Asteroid Collision!)}
\begin{problem}
	Suppose that an asteroid traveling straight toward the center of the earth were to collide with our planet at the equator and bury itself just below the surface.  What would have to be the mass of this asteroid, in terms of the earth's mass $M$, for the day to become 25.0\% longer than it presently is as a result of the collision?  Assume that the asteroid is very small compared to the earth and that the earth is uniform throughout.
\end{problem}

\begin{solution}
	We can solve this problem using conservation of angular momentum:
	\beqn \label{consmom}
		I_i \omega_i = I_f \omega_f.
	\eeqn
	When the asteroid hits the earth, its moment of inertia will change from $I_i$ to $I_f$, which will cause its angular velocity to change from $\omega_i$ to $\omega_f$.
	
	According to the problem statement, we can model the earth before the collision as a solid sphere.  This means
	\beq
		I_i = \frac{2}{5} M R^2,
	\eeq
	where $R$ is the radius of the earth.
	
	After the collision, the asteroid is essentially a point mass $m$ stuck to the side of the earth at the equator.  This earth-asteroid system has a center of mass that is some distance $d$ away from the earth's center of mass.  If we fix the origin at the earth's center of mass, then
	\beq
		d = \frac{m R}{m + M}.
	\eeq
	This means the asteroid is orbiting the center of mass with orbital radius $R - d$, so its moment of inertia is
	\beq
		I_a = m (R - d)^2.
	\eeq
	We can use the parallel axis theorem to find the new moment of inertia of the earth:
	\beq
		I_e = I_i + M d^2,
	\eeq
	because $I_i$ is the moment of inertia of the earth rotating about its center of mass.  Finally, the moment of inertia of the entire system is just their sum,
	\beq
		I_f = I_e + I_a = \frac{2}{5} M R^2 + M d^2 + m (R - d) = \frac{2}{5} M R^2 + M \left( \frac{m R}{m + M} \right)^2 + m \left( R - \frac{m R}{m + M} \right)^2.
	\eeq
	
	An earth day is equivalent to the period of the earth's rotation about its axis:
	\beq
		T = \frac{2 \pi}{\omega}.
	\eeq
	For the day to become 25.0\% longer, we need
	\beq
		\frac{T_f}{T_i} = 1.25 \implies \frac{\omega_i}{\omega_f} = 1.25.
	\eeq
	This suggests that we rewrite \refeq{consmom} as
	\beq
		I_f = \frac{\omega_i}{\omega_f} I_i = \frac{5}{4} I_i.
	\eeq
	Feeding in our results for the moments of inertia and solving for $m$, we find
	\beq
		m = {\color{blue} \frac{M}{9}}.
	\eeq
\end{solution}


\clearpage
\begin{figure}
	\vspace{1in}
	\caption{\textbf{P10.72}}
	\label{P10.72}
\end{figure}

\newcommand{\ho}{h_0}

\paragraph{Problem 10.72}
\begin{problem}
	A thin-walled, hollow spherical shell of mass $m$ and radius $r$ starts from rest and rolls without slipping down a track (Fig.~\ref{P10.72}).  Points $A$ and $B$ are on a circular part of the track having radius $R$.  The diameter of the shell is very small compared to $\ho$ and $R$, and the work done by rolling friction is negligible.
	\begin{enumerate}
		\item What is the minimum height $\ho$ for which this shell will make a complete loop-the-loop on the circular part of the track?
		\item How hard does the track push on the shell at point $B$, which is at the same level as the center of the circle?
		\item Suppose that the track had no friction and the shell was released from the same height $\ho$ you found in part (a).  Would it make a complete loop-the-loop?  How do you know?
		\item In part (c), how hard does the track push on the shell at point $A$, the top of the circle?  How hard did it push in part (a)?
	\end{enumerate}
\end{problem}



\paragraph{Problem 10.76}
\begin{problem}
	You are designing a system for moving aluminum cylinders from the ground to a loading dock.  You use a sturdy wooden ramp that is \SI{6.00}{\meter} long and inclined at $37.0^\circ$ above the horizontal.  Each cylinder is fitted with a light, frictionless yoke through its center, and a light (but strong) rope is attached to the yoke.  Each cylinder is uniform and has mass \SI{460}{\kg} and radius \SI{0.300}{\meter}.  The cylinders are pulled up the ramp by applying a constant force $\vF$ to the free end of the rope.  $\vF$ is parallel to the surface of the ramp and exerts no torque on the cylinder.  The coefficient of static friction between the ramp surface and the cylinder is 0.120.
	\begin{enumerate}
		\item What is the largest magnitude $\vF$ can have so that the cylinder still rolls without slipping as it moves up the ramp?
		\item If the cylinder starts from rest at the bottom of the ramp and rolls without slipping as it moves up the ramp, what is the shortest time it can take the cylinder to reach the top of the ramp?
	\end{enumerate}
\end{problem}


\paragraph{Problem 10.80}
\begin{problem}
	A \SI{5.00}{\kg} ball is dropped from a height of \SI{12.0}{\meter} above one end of a uniform bar that pivots at its center.  The bar has mass \SI{8.00}{\kg} and is \SI{4.00}{\meter} in length.  At the other end of the bar sits another \SI{5.00}{\kg} ball, unattached to the bar.  The dropped ball sticks to the bar after the collision.  How high will the other ball go after the collision?
\end{problem}

\end{document}