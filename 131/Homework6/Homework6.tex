\documentclass[11pt]{article}
\usepackage{geometry, titlesec}
\usepackage[parfill]{parskip}
\usepackage[italicdiff]{physics}
\usepackage{amsfonts, amsthm}
\usepackage[cm]{fullpage}
\usepackage{fancyhdr}
\usepackage{enumitem}
\usepackage{xcolor, soul}
\usepackage{siunitx}
%\allowdisplaybreaks

\renewcommand{\thesubsection}{\thesection.\alph{subsection}}
\setenumerate[1]{label={(\alph*)}}

\makeatletter
\renewcommand*\env@cases[1][1.2]{%
  \let\@ifnextchar\new@ifnextchar
  \left\lbrace
  \def\arraystretch{#1}%
  \array{@{}l@{\quad}l@{}}%
}
\makeatother
 
\renewcommand{\footrulewidth}{.2pt}
%\setlist[enumerate]{leftmargin=*}
\pagestyle{fancy}
\fancyhf{}
\lhead{\textbf{Physics 131 Homework 6}}
\rhead{Lacey Rainbolt}
\setlength{\headheight}{11pt}
\setlength{\headsep}{11pt}
\setlength{\footskip}{24pt}
\lfoot{\today}
\rfoot{\thepage}

\titleformat{\subsection}[runin]{\normalfont\large\bfseries}{\thesubsection}{1em}{}
\newcommand{\refeq}[1]{(\ref{#1})}

\newcommand{\beq}{\begin{equation*}}
\newcommand{\eeq}{\end{equation*}}

\newcommand{\beqn}{\begin{equation}}
\newcommand{\eeqn}{\end{equation}}

\newcommand{\blg}{\begin{align*}}
\newcommand{\elg}{\end{align*}}


\newenvironment{statement}
{
    \color{darkgray}
    \ignorespaces
}
{
%    \smallskip
}

\newenvironment{problem}
{
%    \color{darkgray}
    \ignorespaces
}

\newenvironment{solution}
{
    \paragraph{Solution.}
    \ignorespaces
}
{
    \bigskip
}

\renewcommand{\vec}[1]{\mathbf{#1}}


\begin{document}

\paragraph{10.2}
\begin{problem}
	Calculate the net torque about point $O$ for the two forces applied as in Fig.~\ref{E10.2}.  The rod and both forces are in the plane of the page.
\end{problem}

%\begin{figure} \centering
	
%	\label{E10.2}
%\end{figure}

\newcommand{\ih}{\vec{\,\hat{i}}}
\newcommand{\jh}{\vec{\,\hat{j}}}
\newcommand{\kh}{\vec{\,\hat{k}}}

\newcommand{\vF}{\vec{F}}
\newcommand{\vt}{\boldsymbol{\tau}}
\newcommand{\vr}{\vec{r}}

\newcommand{\vFq}{\vF_1}
\newcommand{\Fq}{F_1}
\newcommand{\vFw}{\vF_2}
\newcommand{\Fw}{F_2}

\newcommand{\vtq}{\vt_1}
\newcommand{\vtw}{\vt_2}
\newcommand{\vrq}{\vr_1}
\newcommand{\rrq}{r_1}
\newcommand{\vrw}{\vr_2}
\newcommand{\rw}{r_2}

\newcommand{\vtnet}{\vt_\text{net}}
\newcommand{\tnet}{\tau_\text{net}}


\begin{solution}
	The net torque is the sum of the torques due to each of the forces.  Let $\vtq$ be the~(vector) torque due to $\vFq$ relative to point $O$, and $\vrq$ the vector from $O$ to where $\vFq$ acts.  Using the coordinate axes drawn in Fig.~\ref{E10.2}, we have
	\begin{align*}
		\vrq &= -\rrq \ih, \\
		\vFq &= -\Fq \jh,
	\end{align*}
	where $\rrq = \SI{5.00}{\meter}$ and $\Fw = \SI{8.00}{\newton}$.  Then
	\beq
		\vtq = \vrq \times \vFq = \rrq \Fq \kh.
	\eeq
	Now define $\vtq$, $\vtw$, and $\vrw$ similarly for $\vFw$.  Then
		\begin{align*}
		\vrw &= -\rw \ih, \\
		\vFw &= -(\cos{30^\circ} \Fw) \ih + (\sin{30^\circ} \Fw) \jh = -\frac{\sqrt{3}}{2} \Fw \ih + \frac{1}{2} \Fw \jh,
	\end{align*}
	where $\rw = \SI{2.00}{\meter}$ and $\Fw = \SI{12.0}{\newton}$, and
	\beq
		\vtw = \vrw \times \vFw = -\frac{1}{2} \rw \Fw \kh.
	\eeq
	The net torque $\vtnet$ is then
	\beq
		\vtnet = \vtq + \vtw = \left( \rrq \Fq - \frac{1}{2} \rw \Fw \right)\!\kh = \tnet \kh,
	\eeq
	where $\tnet$ is the magnitude of the net torque.  Plugging everything in,
	\beq
		\tnet = \rrq \Fq - \frac{1}{2} \rw \Fw = (\SI{5.00}{\meter}) (\SI{8.00}{\newton}) - \frac{1}{2} (\SI{2.00}{\meter}) (\SI{12.0}{\newton}) = {\color{blue} \SI{28.0}{\newton\meter}}.
	\eeq
\end{solution}

\newcommand{\mb}{m_b}
\newcommand{\mw}{m_w}
\newcommand{\mmp}{m_p}

\newcommand{\simb}{\SI{12.0}{\kg}}
\newcommand{\simw}{\SI{5.00}{\kg}}
\newcommand{\simp}{\SI{2.00}{\kg}}
\newcommand{\sIip}{\SI{0.0625}{\kg\square\meter}}

\newcommand{\Ww}{W_w}
\newcommand{\aq}{a_b}

\newcommand{\Tb}{T_b}
\newcommand{\Tw}{T_w}

\newcommand{\az}{\alpha_z}

\paragraph{10.16}
\begin{problem}
	A \simb box resting on a horizontal, frictionless surface is attached to a \simp weight by a thin, light wire that passes over a frictionless pulley (Fig.~\ref{E10.16}).  The pulley has the shape of a uniform solid disk of mass \simp and diameter \SI{0.500}{\meter}.  After the system is released, find
	\begin{enumerate}
		\item the tension in the wire on both sides of the pulley,
		\item the acceleration of the box, and
		\item the h\hl{orizontal and vertical components of the force that the axle exerts on the pulley.}
	\end{enumerate}
\end{problem}

\begin{solution}
	The box and the weight must have the same acceleration $a$ since they are connected by the wire.  We also assume the pulley rolls without slipping, so its tangential acceleration is also $a$.  Using the notation of the free-body diagrams in Fig.~\ref{E10.16}, let $\Tb$ be the tension in the side of the wire connected to the box and $\Tw$ the tension in the side connected to the weight.
	
	Let $\mb$ denote the mass of the box and $\mw$ that of the weight.  We can write down three equations using Newton's second law: one equation from the free-body diagram for the box in Fig.~\ref{E10.16},
			\beqn \label{box}
				\mb a = \Tb,
			\eeqn
			one from the diagram for the weight,
			\beqn \label{weight}
				\mw a = \mw g - \Tw.
			\eeqn
			and one from the diagram for the pulley,
			\beq
				\tnet = I \alpha \iff (\Tw - \Tb) r = I \frac{a}{r} \iff \Tw - \Tb = I \frac{a}{r^2},
			\eeq
			where $r$ is the pulley's radius, $I$ the moment of inertia about its center, and $\alpha$ its angular acceleration.  For rolling without slipping, $\alpha = a / r$.
			
			The pulley is a solid cylinder rotating about its $z$ axis.  Let $\mmp$ be its mass.  Then
			\beq
				I = \frac{1}{2} \mmp r^2,
			\eeq
			and \refeq{pulley0} can be rewritten as
			\beqn \label{pulley}
				\Tw - \Tb = \frac{1}{2} \mmp a.
			\eeqn
			The system of three equations \refeq{box}, \refeq{weight}, and \refeq{pulley} has three unknowns.  The solutions are
			\begin{align}
				\Tb &= 2g \frac{\mb \mw}{2 \mb + 2 \mw + \mmp}, \label{Tb} \\
				\Tw &= g \mw \frac{2 \mb + \mmp}{2 \mb + 2 \mw + \mmp}, \label{Tw} \\
				a &= 2g \frac{\mw}{2 \mb + 2 \mw + \mmp}. \label{acc}
			\end{align}
			
			
			
	\begin{enumerate}
		\item Plugging numbers into \refeq{Tb} and \refeq{Tw}, we have
			\begin{align*}
				\Tb &= \frac{\mb \mw}{I + \mb + \mw} g
			\end{align*}
		
		\item Torque due to $\Tb$ is $\Tb R$, also $\Tw R$
	\end{enumerate}
	
\end{solution}


\paragraph{10.22}
\begin{problem}
	A string is wrapped several times around the rim of a small hoop with radius \SI{8.00}{\centi\meter} and mass \SI{0.180}{\kilo\gram}.  The free end of the string is held in place and the hoop is released from rest (Fig.~\ref{E10.22}).  After the hoop has descended \SI{75.0}{\centi\meter}, calculate
	\begin{enumerate}
		\item the angular speed of the rotating hoop, and
		\item the speed of its center.
	\end{enumerate}
\end{problem}

\begin{solution}

\end{solution}

\end{document}