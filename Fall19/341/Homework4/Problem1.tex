\newcommand{\tht}{\theta}
\newcommand{\sint}{\sin{\tht}}
\newcommand{\cost}{\cos{\tht}}
\newcommand{\sinwt}{\sin^2{\tht}}
\newcommand{\coswt}{\cos^2{\tht}}
\newcommand{\sinp}{\sin{\phi}}
\newcommand{\cosp}{\cos{\phi}}
\newcommand{\sinwp}{\sin^2{\phi}}
\newcommand{\coswp}{\cos^2{\phi}}

\newcommand{\vL}{\vec{L}}
\newcommand{\vLw}{\vL^2}

\newcommand{\Ylm}{Y^l_m}
\newcommand{\Ylmtp}{\Ylm(\tp)}
\newcommand{\alm}{a^l_m}
\newcommand{\dOmega}{\sint \dd{\tht} \dd{\phi}}

\newcommand{\tp}{\tht, \phi}
\newcommand{\psitpo}{\psi(\tp, 0)}
\newcommand{\psitpt}{\psi(\tp, t)}

\section{Problem 1}
\begin{statement}
	A particle of mass $m$ is moving on a sphere of radius $a$.  Its wave function is given by $\psi(\tp)$ where $\tht$ and $\phi$ parameterize the sphere $(x, y, z) = a(\sint \, \cosp, \sint \, \sinp, \cost)$.  The Hamiltonian of the system is ${H = \vLw / 2 m a^2}$, where $\vLw$ is the square of the angular momentum operator, and is given by
	\beq
		\vLw = -\hbar^2 \left( \frac{1}{\sint} \pdv{}{\tht} \sint \pdv{}{\tht} + \frac{1}{\sinwt} \pdv[2]{}{\phi} \right).
	\eeq
	The eigenfunctions of $H$ are spherical harmonics $\Ylm$ with energies
	\beqn \label{eigene}
		E_l = \frac{\hbar^2 l (l + 1)}{2 m a^2}.
	\eeqn
\vfix
\end{statement}

\begin{problem} \label{1}
	The wave function of the system at $t = 0$ is given by
	\beq
		\psi(\tp, 0) = A \sinwt \, \coswp,
	\eeq
	where $A$ is a constant.  This wave function can be expanded in spherical harmonics:
	\beq
		\psitpo = \sum_{l, m} \alm \Ylmtp.
	\eeq
	Find all nonzero $\alm$.
\end{problem}

\begin{solution}
	We will look for nonzero $\alm$ by comparing the $\tht$ and $\phi$ dependence of $\Ylm$ and $\psitpo$.  From (3.6.36) and (3.6.37) in Sakurai, the spherical harmonic functions are given by
	\begin{align*}
		\Ylmtp &= \frac{(-1)^l}{2^l l!} \sqrt{\frac{(2l + 1)}{4\pi} \frac{(l + m)!}{(l - m)!}} \frac{e^{i m \phi}}{\sin^m{\tht}} \dv[l - m]{(\sint)^{2l}}{(\cost)}, &
		Y^l_{-m}(\tp) &= (-1)^m \, {\Ylm}^*(\tp)
	\end{align*}
	for $m \geq 0$.  Beginning with the $\phi$ dependence of $\psitpo$, note that
	\beqn \label{cos}
		\psitpo \propto \coswp = \left( \frac{e^{i \phi} + e^{-i \phi}}{2} \right)^2 = \frac{1}{2} + \frac{e^{i 2 \phi}}{4} + \frac{e^{-i 2 \phi}}{4},
	\eeqn
	which implies that the only nonzero $\alm$ correspond to $m \in \{ 0, \pm 2 \}$.
	
	For the $\tht$ dependence, we have $\psitpo \propto \sinwt$.  Looking at $\Ylm$, note that $(\sint)^{2l} = (1 - \coswt)^l$, so
	\beq
		\Ylm \propto \frac{1}{\sin^m{\tht}} \dv[l - m]{(1 - \coswt)^l}{(\cost)}.
	\eeq
	Plugging in $m = 0$ and the first few values of $l$,
	\begin{align*}
		Y^0_0 &\propto \dv[0]{(1 - \coswt)^0}{(\cost)} = 1, \\
		Y^1_0 &\propto \dv{(1 - \coswt)}{(\cost)} = -2\cost, \\
		Y^2_0 &\propto \dv[2]{(1 - 2\coswt + \cos^4{\tht})}{(\cost)} = \dv{(-4 \cost + 4 \cos^3{\tht})}{(\cost)} = -4 + 12 \coswt = 8 - 12 \sinwt,
	\end{align*}
	so we know $a^1_0 = 0$.  Inspecting the above, we deduce that $Y^l_0$ with $l > 2$ contain mixed terms of $\sint$ and $\cost$ and higher powers of $\sint$, so $a^l_0 = 0$ for $l > 2$.
	
	Plugging in $m = \pm 2$ and $l = 2$,
	\beq
		Y^2_{\pm 2} \propto \frac{1}{\sinwt} \dv[0]{(1 - \coswt)^2}{(\cost)} = \frac{\sin^4{\tht}}{\sinwt} = \sinwt.
	\eeq
	Again, by inspection $Y^l_{\pm2}$ with $l > 2$ contain terms that are not in $\psitpo$, so $a^l_{\pm2} = 0$ for $l > 2$ as well.
	
	Thus, only $a^0_0$, $a^2_0$, and $a^2_{\pm2}$ are nonzero; that is,
	\beq
		\psitpo = a^0_0 Y^0_0 + a^2_0 Y^2_0 + a^2_2 Y^2_2 + a^2_{-2} Y^2_{-2}.
	\eeq
	The relevant spherical harmonics are
	\begin{align} \label{shar}
		Y^0_0 &= \sqrt{\frac{1}{4 \pi}}, &
		Y^2_0 &= \sqrt{\frac{5}{16 \pi}} (2 - 3 \sinwt), &
		Y^2_{\pm2} &= \sqrt{\frac{15}{32 \pi}} \sinwt e^{\pm2 i \phi}.
	\end{align}
	Expanding out $\psitpo$ as in \refeq{cos},
	\beq
		\psitpo = \frac{A}{2} \sinwt + \frac{A}{4} \sinwt e^{i 2 \phi} + \frac{A}{4} \sinwt e^{-i 2 \phi}.
	\eeq
	Then we can deduce the nonzero $\alm$:
	\begin{align*}
		\frac{A}{4} \sinwt e^{\pm i 2 \phi} = a^2_{\pm2} \sqrt{\frac{15}{32 \pi}} \sinwt e^{\pm2 i \phi}
		& \implies a^2_{\pm2} = A \sqrt{\frac{2\pi}{15}}, \\
		\frac{A}{2} \sinwt = a^0_0 \sqrt{\frac{1}{4 \pi}} + a^2_0 \sqrt{\frac{5}{16 \pi}} (2 - 3 \sinwt)
		&\implies a^2_0 = -\frac{2}{3} A \sqrt{\frac{\pi}{5}},\ a^0_0 = \frac{2}{3} A \sqrt{\pi}.
	\end{align*}
\end{solution}

\newcommand{\Ut}{U(t)}

\begin{problem} \label{2}
	Now consider the wave function at nonzero time $t$.  Use your results from \ref{1} and the expressions for spherical harmonics to derive an explicit expression in terms of sines and cosines of $\tht$ and $\phi$ for $\psitpt$.
\end{problem}

\begin{solution}
	From \ref{1}, we have
	\beqn \label{expand}
		\psitpo = \frac{2}{3} A \sqrt{\pi} \, Y^0_0 - \frac{2}{3} A \sqrt{\frac{\pi}{5}} \, Y^2_0 + A \sqrt{\frac{2\pi}{15}} \, Y^2_2 + A \sqrt{\frac{2\pi}{15}} \, Y^2_{-2}.
	\eeqn
	We can evaluate the time evolution for each spherical harmonic term in \refeq{expand} individually, and sum them up to find $\psitpt$:
	\beq
		\psitpt = \Ut \psitpo = \frac{2}{3} A \sqrt{\pi} \, \Ut Y^0_0 - \frac{2}{3} A \sqrt{\frac{\pi}{5}} \, \Ut Y^2_0 + A \sqrt{\frac{2\pi}{15}} \, \Ut Y^2_2 + A \sqrt{\frac{2\pi}{15}} \, \Ut Y^2_{-2}
	\eeq
	  The time evolution operator is given by $\Ut = e^{-i H t / \hbar}$.  From \refeq{eigene}, the relevant eigenvalues are
	\begin{align*}
		E_0 &= 0, &
		E_2 &= 3 \frac{\hbar^2}{m a^2},
	\end{align*}
	so
	\begin{align*}
		\Ut Y^0_0 &= \exp(-\frac{i}{\hbar} E_0 t) Y^0_0 = Y^0_0, &
		\Ut Y^2_m &= \exp(-\frac{i}{\hbar} E_2 t) Y^2_m = \exp(-3 i \frac{\hbar}{m a^2} t) Y^2_m.
	\end{align*}
	Then, using the explicit $\Ylm$ from \refeq{shar},
	\begin{align}
		\psitpt &= \frac{2}{3} A \sqrt{\pi} \sqrt{\frac{1}{4 \pi}} - \frac{2}{3} A \sqrt{\frac{\pi}{5}} \exp(-3 i \frac{\hbar}{m a^2} t) \sqrt{\frac{5}{16 \pi}} (2 - 3 \sinwt) + A \sqrt{\frac{2\pi}{15}} \exp(-3 i \frac{\hbar}{m a^2} t) \sqrt{\frac{15}{32 \pi}} \sinwt e^{2 i \phi} \notag \\
		&\phantom{mmmmmmmmmmmmmmmmmmmmmmmmmmmmm} + A \sqrt{\frac{2\pi}{15}} \exp(-3 i \frac{\hbar}{m a^2} t) \sqrt{\frac{15}{32 \pi}} \sinwt e^{-2 i \phi} \notag \\
		&= \frac{A}{3} - \frac{A}{6} \exp(-3 i \frac{\hbar}{m a^2} t) (2 - 3 \sinwt) + \frac{A}{4} \exp(-3 i \frac{\hbar}{m a^2} t) \sinwt e^{2 i \phi} + \frac{A}{4} \exp(-3 i \frac{\hbar}{m a^2} t) \sinwt e^{-2 i \phi} \notag \\
		&= \frac{A}{3} - \frac{A}{3} \exp(-3 i \frac{\hbar}{m a^2} t) + \frac{A}{2} \exp(-3 i \frac{\hbar}{m a^2} t) \sinwt + \frac{A}{2} \exp(-3 i \frac{\hbar}{m a^2} t) \sinwt \cos{2\phi} \notag \\
		&= \frac{A}{3} \left[ 1 - \exp(-3 i \frac{\hbar}{m a^2} t) \right] + A \exp(-3 i \frac{\hbar}{m a^2} t) \sinwt \coswp. \label{psitpt}
	\end{align}
\end{solution}

\newcommand{\Li}{L_i}
\newcommand{\Lx}{L_x}
\newcommand{\Ly}{L_y}
\newcommand{\Lz}{L_z}

\newcommand{\Udt}{U^\dagger(t)}

\newcommand{\psit}{\psi(t)}

\begin{problem}
	Use your results from \ref{2} to derive expressions for the expected values of $\Lx$, $\Ly$, and $\Lz$ as functions of time.
\end{problem}

\begin{solution}
	From (3.6.23) in Sakurai, $\braket{\tp}{l, m} = \Ylmtp$ and therefore $\psitpt = \braket{\tp}{\psit}$.  Using the result of \ref{2}, this implies
	\beq
		\ket{\psit} = a^0_0 \ket{0, 0} + a^2_0 \exp(-3 i \frac{\hbar}{m a^2} t) \ket{2, 0} + a^2_2 \exp(-3 i \frac{\hbar}{m a^2} t) \ket{2, 2} + a^2_{-2} \exp(-3 i \frac{\hbar}{m a^2} t) \ket{2, -2}.
	\eeq
	Then the time-dependent expectation value of an operator $O$ is given by
	\begin{align*}
		\ev{O}{\psit} &= {a^0_0}^2 \ev{O}{0, 0} + a^0_0 a^2_0 \Ut \mel{0, 0}{O}{2, 0} + a^0_0 a^2_2 \Ut \mel{0, 0}{O}{2, 2} + a^0_0 a^2_{-2} \Ut \mel{0, 0}{O}{2, -2} \\
		&\phantom{=\ } + a^0_0 a^2_0 \Udt \mel{2, 0}{O}{0, 0} + {a^2_0}^2 \ev{O}{2, 0} + a^2_0 a^2_2 \mel{2, 0}{O}{2, 2} + a^2_0 a^2_{-2} \mel{2, 0}{O}{2, -2} \\
		&\phantom{=\ } + a^0_0 a^2_2 \Udt \mel{2, 2}{O}{0, 0} + a^2_0 a^2_2 \mel{2, 2}{O}{2, 0} + {a^2_2}^2 \ev{O}{2, 2} + a^2_2 a^2_{-2} \mel{2, 2}{O}{2, -2} \\
		&\phantom{=\ } + a^0_0 a^2_{-2} \Udt \mel{2, -2}{O}{0, 0} + a^2_0 a^2_{-2} \mel{2, -2}{O}{2, 0} + a^2_2 a^2_{-2} \mel{2, -2}{O}{2, 2} + {a^2_{-2}}^2 \ev{O}{2, -2},
	\end{align*}
	where $\Ut = e^{-3i \hbar t / m a^2}$ and $\Udt = e^{3i \hbar t / m a^2}$.
	
	From the results of 3.3 on the previous homework,
	\begin{align*}
		0 &= \mel{2, -2}{\Li}{2, -2} = \mel{2, -2}{\Li}{2, 0} = \mel{2, -2}{\Li}{2, 2} \\
		&= \mel{2, 0}{\Li}{2, -2} = \mel{2, 0}{\Li}{2, 0} = \mel{2, 0}{\Li}{2, 2} \\
		&= \mel{2, 2}{\Li}{2, -2} = \mel{2, 2}{\Li}{2, 0} = \mel{2, 2}{\Li}{2, 2}
	\end{align*}
	for $i \in \{ x, y, z \}$.  For $(l, m) = (0, 0)$, a similar procedure to the one used for 3.3 yields
	\begin{align*}
		\mel{l', m'}{\Lx}{0, 0} &= \mel{0, 0}{\Lx}{l', m'} = \frac{\hbar}{2} \delta_{0, l'} \, \delta_{1, m'} \sqrt{l^2 + l} = 0, \\
		\mel{l', m'}{\Ly}{0, 0} &= \mel{0, 0}{\Ly}{l', m'} = -\frac{i \hbar}{2} \delta_{0, l'} \, \delta_{1, m'} \sqrt{l^2 + l} = 0, \\
		\mel{l', m'}{\Lz}{0, 0} &= \mel{0, 0}{\Lz}{l', m'} = 0,
	\end{align*}
	where the last result comes from the eigenvalues of $\Lz$ being $\hbar m$.  Thus, we find
	\beq
		\ev{\Lx}{\psit} = \ev{\Ly}{\psit} = \ev{\Lz}{\psit} = 0.
	\eeq
\end{solution}