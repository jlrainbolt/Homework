%\newcommand{\gam}{\gamma}
\newcommand{\gami}{\gam^{(i)}}
\newcommand{\gamj}{\gam^{(j)}}
\newcommand{\vr}{\vec{r}}
\newcommand{\vri}{\vr_i}
\newcommand{\vrj}{\vr_j}
\renewcommand{\xi}{x_i}
\newcommand{\yi}{y_i}
\newcommand{\xj}{x_j}
\newcommand{\yj}{y_j}

\newcommand{\xd}{\dot{x}}
\newcommand{\yd}{\dot{y}}
\newcommand{\xdi}{\xd_i}
\newcommand{\ydi}{\yd_i}

\newcommand{\gamq}{\gam^{(1)}}
\newcommand{\gamw}{\gam^{(2)}}
\newcommand{\vrq}{\vr_1}
\newcommand{\vrw}{\vr_2}

%\newcommand{\xq}{x_q}
\newcommand{\yq}{y_1}
\newcommand{\xw}{x_2}
\newcommand{\yw}{y_2}

\newcommand{\xdq}{\xd_1}
\newcommand{\ydq}{\yd_1}
\newcommand{\xdw}{\xd_2}
\newcommand{\ydw}{\yd_2}

\newcommand{\vR}{\vec{R}}
\newcommand{\Xd}{\dot{X}}
\newcommand{\Xdd}{\ddot{X}}
\newcommand{\Yd}{\dot{Y}}
\newcommand{\Ydd}{\ddot{Y}}

\newcommand{\Cq}{C_1}
\newcommand{\Cw}{C_2}
\newcommand{\Dq}{D_1}
\newcommand{\Dw}{D_2}

\newcommand{\omg}{\omega}

\begin{statement}{Interacting line vortices} \label{vort1}
\subparagraph{}
	A system of $n$ vortices moving on a two-dimensional plane has the Hamiltonian
	\beqn \label{ham5}
		H = \sum_{j = 1}^n \sum_{i = 1}^{j - 1} -\gami \gamj \ln|\vri - \vrj|,
	\eeqn
	where $\gami$ is the strength of the $i$th line vortex, and $\vri = (\xi, \yi)$ its position in the plane.  Using the Poisson bracket structure
	\beqn \label{poiss}
		[f, g] = \sum_{i = 1}^n \frac{1}{\gami} \left( \pdv{f}{\xi} \pdv{g}{\yi} - \pdv{f}{\yi} \pdv{g}{\xi} \right),
	\eeqn
	Hamilton's equations for the vortex system simplify to
	\begin{align*}
		\xdi &= [\xi, H], &
		\ydi = [\yi, H].
	\end{align*}
	Consider two vortices.  Show that the equations of motion can be solved explicitly.  Most importantly, show that the solution tells us the two vortices orbit each other with a frequency that is inversely proportional to the square of their separation.
\end{statement}

\begin{solution}
	For two vortices, \refeq{ham5} reduces to
	\beq
		H = -\gamq \gamw \ln|\vrq - \vrw|.
	\eeq
	For $i, j \in \{ 1, 2 \}$, note that
	\begin{align*}
		\pdv{\xi}{\xj} &= \pdv{\yi}{\yj} = \delta_{ij}, &
		\pdv{\xi}{\yj} &= \pdv{\yi}{\xj} = 0.
	\end{align*}
	Note also that
	\beq
		|\vrq - \vrw| = \sqrt{(\xq - \xw)^2 + (\yq - \yw)^2}
		= \sqrt{\xq^2 - 2 \xq \xw + \xw^2 + \yq^2 - 2 \yq \yw + \yw^2},
	\eeq
	and define $R \equiv |\vrq - \vrw|$ as the separation of the vortices.  Define also
	\begin{align*}
		u &\equiv \ln{R}, &
		v &\equiv (\xq - \xw)^2 + (\yq - \yw)^2 = R^2.
	\end{align*}
	 Then
	\beq
		\pdv{H}{\xq} = \pdv{H}{u} \pdv{u}{R} \pdv{R}{v} \pdv{v}{\xq}
		= -\gamq \gamw \frac{1}{R} \frac{1}{2R} (2 \xq - 2 \xw)
		= -\gamq \gamw \frac{\xq - \xw}{(\xq - \xw)^2 + (\yq - \yw)^2}
		= -\gamq \gamw \frac{\xq - \xw}{R^2},
	\eeq
	and, likewise,
	\begin{align*}
		\pdv{H}{\xw} &= \gamq \gamw \frac{\xq - \xw}{R^2}, &
		\pdv{H}{\yq} &= -\gamq \gamw \frac{\yq - \yw}{R^2}, &
		\pdv{H}{\yw} &= \gamq \gamw \frac{\yq - \yw}{R^2}.
	\end{align*}
	Combining the above, we use \refeq{poiss} to find
	\begin{align}
		\xdq &= \frac{1}{\gamq} \left( \pdv{\xq}{\xq} \pdv{H}{\yq} - \pdv{\xq}{\yq} \pdv{H}{\xq} \right) + \frac{1}{\gamw} \left( \pdv{\xq}{\xw} \pdv{H}{\yw} - \pdv{\xq}{\yw} \pdv{H}{\xw} \right)
		= \frac{1}{\gamq} \pdv{H}{\yq}
		= -\gamw \frac{\yq - \yw}{R^2}, \label{4eqns1} \\
		\xdw &= \frac{1}{\gamq} \left( \pdv{\xw}{\xq} \pdv{H}{\yq} - \pdv{\xw}{\yq} \pdv{H}{\xq} \right) + \frac{1}{\gamw} \left( \pdv{\xw}{\xw} \pdv{H}{\yw} - \pdv{\xw}{\yw} \pdv{H}{\xw} \right)
		= \frac{1}{\gamw} \pdv{H}{\yw}
		= \gamq \frac{\yq - \yw}{R^2},  \label{4eqns2} \\
		\ydq &= \frac{1}{\gamq} \left( \pdv{\yq}{\xq} \pdv{H}{\yq} - \pdv{\yq}{\yq} \pdv{H}{\xq} \right) + \frac{1}{\gamw} \left( \pdv{\yq}{\xw} \pdv{H}{\yw} - \pdv{\yw}{\yw} \pdv{H}{\xw} \right)
		= -\frac{1}{\gamq} \pdv{H}{\xq}
		= \gamw \frac{\xq - \xw}{R^2},  \label{4eqns3} \\
		\ydw &= -\frac{1}{\gamq} \left( \pdv{\yw}{\xq} \pdv{H}{\yq} - \pdv{\yw}{\yq} \pdv{H}{\xq} \right) + \frac{1}{\gamw} \left( \pdv{\yw}{\xw} \pdv{H}{\yw} - \pdv{\yw}{\yw} \pdv{H}{\xw} \right)
		= -\frac{1}{\gamw} \pdv{H}{\xw}
		= -\gamw \frac{\xq - \xw}{R^2}. \label{4eqns4}
	\end{align}
	Note that
	\begin{align*}
		\pdv{R}{\xi} &= \pdv{R}{v} \pdv{v}{\xi}
		= \frac{\xi - \xj}{R^2}, &
		\pdv{R}{\yi} &= \frac{\yi - \yj}{R^2}
	\end{align*}
	so
	\beq
		[H, R] = \sum_{i = 1}^n \frac{1}{\gami} \left( \pdv{H}{\xi} \pdv{v}{\yi} - \pdv{H}{\yi} \pdv{v}{\xi} \right)
		= \sum_{i = 1}^n \frac{1}{\gami} \left( -\gami \gamj \frac{\xi - \xj}{R^2} \frac{\yi - \yj}{R^2} + \gami \gamj \frac{\yi - \yj}{R^2} \frac{\xi - \xj}{R^2} \right)
		= 0.
	\eeq
	This means $R$ is a conserved quantity and therefore constant.

	Define $\vR \equiv \vrq - \vrw = (X, Y)$, where $|\vR| = R$.  Now we have two generalized coordinates $X$ and $Y$, where $X = \xq - \xw$ and $Y = \yq - \yw$.  This gives us the two equations of motion
	\begin{align}
		\Xd &= \xdq - \xdw
		= -(\gamq + \gamw) \frac{\yq - \yw}{R^2}
		= -\frac{\gamq + \gamw}{R^2} Y, \label{eqnX} \\
		\Yd &= \ydq - \ydw
		= (\gamq + \gamw) \frac{\xq - \xw}{R^2}
		= \frac{\gamq + \gamw}{R^2} X. \label{eqnY}
	\end{align}
	We can differentiate these to obtain two uncoupled second-order equations:
	\begin{align*}
		\Xdd &= -\frac{\gamq + \gamw}{R^2} \Yd
		= -\frac{(\gamq + \gamw)^2}{R^4} X, &
		\Ydd &= \frac{\gamq + \gamw}{R^2} \Xd
		= -\frac{(\gamq + \gamw)^2}{R^4} Y.
	\end{align*}
	These equations have the solutions
	\begin{align*}
		X(t) &= \Cq \cos(\frac{\gamq + \gamw}{R^2} t) + \Cw \sin(\frac{\gamq + \gamw}{R^2} t), &
		Y(t) &= \Dq \cos(\frac{\gamq + \gamw}{R^2} t) + \Dw \sin(\frac{\gamq + \gamw}{R^2} t),
	\end{align*}
	where $\Cq, \Cw, \Dq, \Dw$ are constants.  Define
	\beqn \label{omg}
		\omg \equiv \frac{\gamq + \gamw}{R^2}
	\eeqn
	as the angular frequency.  To solve for the constants, we apply \refeq{eqnX} and \refeq{eqnY}:
	\begin{align*}
		\Xd(t) &= -\Cq \omg \sin(\omg t) + \Cw \omg \cos(\omg t) = -\omg Y, &
		\Yd(t) &= -\Dq \omg \sin(\omg t) + \Dw \omg \cos(\omg t) = \omg X.
	\end{align*}
	This implies $\Cq = \Dw$ and $\Cw = -\Dq$.  We can fix $\Cw = \Dq = 0$ without loss of generality, which implies $\Cq = \Dq = R$.  We now have
	\begin{align} \label{2eqns}
		X(t) &= R \cos(\omg t) = \xq(t) - \xw(t), &
		Y(t) &= R \sin(\omg t) = \yq(t) - \yw(t).
	\end{align}
	We can now find the solutions to the original four equations by integrating \refeq{4eqns1}--\refeq{4eqns4} with respect to $t$:
	\begin{align*}
		\xq(t) &= -\frac{\gamw}{R^2} \int Y(t) \dd{t}
		= \frac{\gamw}{\omg R} \cos(\omg t)
		= \frac{\gamw}{\gamq + \gamw} R \cos(\omg t), \\
		\xw(t) &= \frac{\gamq}{R^2} \int Y(t) \dd{t}
		= -\frac{\gamq}{\gamq + \gamw} R \cos(\omg t), \\
		\yq(t) &= \frac{\gamw}{R^2} \int X(t) \dd{t}
		= \frac{\gamw}{\gamq + \gamw} R \sin(\omg t), \\
		\yw(t) &= -\frac{\gamq}{R^2} \int X(t) \dd{t}
		= \frac{\gamq}{\gamq + \gamw} R \sin(\omg t),
	\end{align*}
	where we have taken the constants of integration to be zero without loss of generality.

	Thus, we have shown that the equations of motion can be solved explicitly.  Since $\vR = (X, Y)$ is the vector separating the vortices, and \refeq{2eqns} show that it rotates in a circle, we have also shown that the vortices orbit each other.  The orbital frequency $\omg$ given by \refeq{omg} is clearly inversely proportional to $R^2$, where $R = |\vR|$ is the magnitude of the vortices' separation.
\end{solution}