\documentclass[11pt]{article}
\usepackage{geometry, titlesec}
\usepackage[parfill]{parskip}
\usepackage{physics, amsfonts, amsthm}
\usepackage[cm]{fullpage}
\usepackage{fancyhdr}
\usepackage{enumitem}
\usepackage{xcolor, soul}
%\allowdisplaybreaks

\renewcommand{\thesubsection}{\thesection.\alph{subsection}}

\makeatletter
\renewcommand*\env@cases[1][1.2]{%
  \let\@ifnextchar\new@ifnextchar
  \left\lbrace
  \def\arraystretch{#1}%
  \array{@{}l@{\quad}l@{}}%
}
\makeatother
 
 
\renewcommand{\footrulewidth}{.2pt}
%\setlist[enumerate]{leftmargin=*}
\pagestyle{fancy}
\fancyhf{}
\lhead{\textbf{Physics 316 Homework 4}}
\rhead{Lacey Rainbolt}
\setlength{\headheight}{11pt}
\setlength{\headsep}{11pt}
\setlength{\footskip}{24pt}
\lfoot{\today}
\rfoot{\thepage}

\titleformat{\subsection}[runin]{\normalfont\large\bfseries}{\thesubsection}{1em}{}
\newcommand{\refeq}[1]{(\ref{#1})}


\newenvironment{statement}
{
    \color{darkgray}
    \ignorespaces
}
{
%    \smallskip
}

\newenvironment{problem}
{
%    \subsection{}
    \color{darkgray}
    \ignorespaces
}


\newenvironment{solution}
{
    \paragraph{Solution.}
    \ignorespaces
}
{
%    \bigskip
}

\renewcommand{\vec}[1]{\mathbf{#1}}



\begin{document}

\newcommand{\ko}{k_0}
\newcommand{\DS}{\Delta S}
\newcommand{\dS}{\delta S}
\newcommand{\eps}{\epsilon}

\newcommand{\dudtw}{\bigg( \pdv{u}{t} \bigg)^2}
\newcommand{\dudxw}{\bigg( \pdv{u}{x} \bigg)^2}

\newcommand{\uo}{u_0}
\newcommand{\uel}{u_\ell}
\newcommand{\tto}{t_0}
\newcommand{\tq}{t_1}
\newcommand{\psio}{\psi_0}
\newcommand{\psil}{\psi_\ell}

\newcommand{\intt}{\int_{\tto}^{\tq}}
\newcommand{\intx}{\int_0^\ell}
\newcommand{\ut}{u_t}
\newcommand{\ux}{u_x}

\newcommand{\Uq}{U_1}
\newcommand{\Uw}{U_2}

\newcommand{\Ld}{\mathcal{L}}

\section{Elastically fastened ends} \label{1}

\begin{problem}
	Consider an ideal stretched string in two dimensions with length $\ell$, density per unit length $\rho$, and effective elastic modulus $k$.  Suppose its two ends are fastened \emph{elastically} by two springs with spring constant $\ko$ so that a nonzero deflection $u(x, t)$ of the end location from either $(0, 0)$ or $\ell, 0)$ is penalized by a linear restrictive force $-ku$.  Adapt the derivation in class for a stretched spring with two fixed ends to this situation.  What is the Euler-Lagrange equation and the associated boundary conditions?
\end{problem}

\begin{solution}
	We will begin with the expression for the kinetic energy $T$ of the string.  Let $\dd{x}$ denote an infinitesimal length of string.  Its mass $\dd{m} = \rho \dd{x}$, so its kinetic energy $\dd{T}$ is
\begin{equation} \label{T}
	\dd{T} = \frac{\rho}{2} \dudtw \dd{x} \implies T = \frac{\rho}{2} \intx \dudtw \dd{x},
\end{equation}
	where we have integrated over the length of the string to obtain $T$.
	
	For the potential energy, let $\Uq$ be the work required to stretch the string, and let $\Uw$ be the work to stretch and compress the springs.  For $\Uq$, consider an infinitesimal length of string $\dd{x}$.  If this length is stretched by some amount $\dd{\ell}$ to a total length
	\begin{equation}
		\dd{x} + \dd{\ell} = \sqrt{(\dd{x})^2 + (\dd{u})^2},
	\end{equation}
	it has potential energy $\dd{\Uq} = k \,\dd{\ell}$.  The limit $\dd{x} \to \infty$ respresents the entire length of string, $\ell$, so we may perform a Taylor series expansion about $\infty$ to first order in $\dd{x}$:
	\begin{equation} \label{U1}
		\dd{\Uq} = k \,\dd{\ell} = k (\sqrt{(\dd{x})^2 + (\dd{u})^2} - \dd{x}) \approx \frac{k}{2} \bigg( \dv{u}{x} \bigg)^2 \dd{x} \implies \Uq = \frac{k}{2} \intx \dudxw \dd{x},
	\end{equation}
	where we have integrated to obtain $\Uq$.  For $\Uw$, the potential energy in the two springs is given by
	\begin{equation} \label{U2}
		\Uw = \frac{\ko}{2} \uo^2 + \frac{\ko}{2} \uel^2,
	\end{equation}
	where $\uo = \uo(t) = u(0, t)$ and $\uel = \uel(t) = u(\ell, t)$.  The total potential energy $U = \Uq + \Uw$.  (The addition of $\Uw$ is what differs from the derivation in class.)
	
	Using \refeq{T}, \refeq{U1}, and \refeq{U2}, we can write an expression for the Lagrangian density $\Ld$:
	\begin{equation} \label{ld1}
		\Ld = T - U = \frac{\rho}{2} \intx \dudtw \dd{x} - \frac{k}{2} \intx \dudxw \dd{x} - \frac{\ko}{2} \uo^2 - \frac{\ko}{2} \uel^2.
	\end{equation}
	The general expression for the Euler-Lagrange equation in the generalized coordinates $t$ and $x$ is
	\begin{equation} \label{el1}
		0 = \pdv{\Ld}{u} - \pdv{t} \pdv{\Ld}{\ut} - \pdv{x} \pdv{\Ld}{\ux}.
	\end{equation}
	From \refeq{ld1},
	\begin{align} \label{devs}
		\pdv{\Ld}{u} &= 0, &
		\pdv{\Ld}{\ut} &= \rho \pdv{u}{t}, &
		\pdv{\Ld}{\ux} &= \rho \pdv{u}{x},
	\end{align}
	so \refeq{el1} becomes
	\begin{equation} \label{el}
		\pdv[2]{u}{t} = \frac{k}{\rho} \pdv[2]{u}{x}
	\end{equation}
	(This is the same as we derived in class.)
	
	In order to evaluate the boundary conditions, we must invoke Hamilton's principle.  The action of the string is
	\begin{align} \label{action}
		S[u] &= \intt \intx \Ld \dd{x} \dd{t} = \intt \left[ \frac{\rho}{2} \intx \dudtw \dd{x} - \frac{k}{2} \intx \dudxw \dd{x} - \frac{\ko}{2} \uo^2 - \frac{\ko}{2} \uel^2 \right] \dd{t} \\
		&= \frac{\rho}{2} \intt \intx \dudtw \dd{x} \dd{t} - \frac{k}{2} \intt \intx \dudxw \dd{x} \dd{t} - \frac{\ko}{2} \intt \left( \uel^2 + \uo^2 \right) \dd{t}.
	\end{align}
	Consider some variation of the action $\DS = S[u + \eps \psi] - S[u]$, where $\psi = \psi(x, t)$ is a function representing a variation and $\eps \ll 1$.  The principle component of $\DS$, $\dS$, is given by
	\begin{equation}
		\dS = \eps \intt \intx \left( \pdv{\Ld}{u} - \pdv{}{t} \pdv{\Ld}{\ut} - \pdv{}{x} \pdv{\Ld}{\ux} \right) \psi \dd{x} \dd{t} + \eps \intt \intx \left[ \pdv{}{t} \left( \pdv{\Ld}{\ut} \psi \right) + \pdv{}{x} \left( \pdv{\Ld}{\ux} \psi \right) \right] \dd{x} \dd{t},
	\end{equation}
	where $\ut = \pdv*{u}{t}$ and $\ux = \pdv*{u}{x}$.  Once again invoking \refeq{devs},
	\begin{align} \label{dS}
		\frac{\dS}{\eps} = \intt \intx \left( k \pdv[2]{u}{x} - \rho \pdv[2]{u}{t} \right) \psi \dd{x} \dd{t} + \rho \intt \intx \pdv{}{t} \left( \pdv{u}{t} \psi \right) \dd{x} \dd{t} - k \intt &\intx \pdv{}{x} \left( \pdv{u}{x} \psi \right) \dd{x} \dd{t} \notag \\
		& - \ko \intt (\uel \psil + \uo \psio) \dd{t} ,
	\end{align}
	where $\psio = \psio(t) = \psi(0, t)$ and $\psil = \psil(t) = \psi(\ell, t)$.  We stipulate that $\psi(x, \tto) = \psi(x, \tq) = 0$ for $x \in [0, \ell]$, and that \refeq{el} is satisfied.  Then \refeq{dS} becomes
	\begin{align}
		\frac{\dS}{\eps} &= \intt \intx \left( k \pdv[2]{u}{x} - \rho \pdv[2]{u}{t} \right) \psi \dd{x} \dd{t} + k \intt \left( \psio \left. \pdv{u}{x} \right|_{x=0} - \psil \left. \pdv{u}{x} \right|_{x=\ell} \right) \dd{t} - \ko \intt (\uel \psil + \uo \psio) \dd{t} \label{psit} \\
		&= k \intt \left( \psio \left. \pdv{u}{x} \right|_{x=0} - \psil \left. \pdv{u}{x} \right|_{x=\ell} \right) \dd{t} - \ko \intt (\uel \psil + \uo \psio) \dd{t}.
	\end{align}
	By Hamilton's principle, $\dS = 0$ for the solution $u(x, t)$.  Then
	\begin{equation} \label{bc}
		\dS = 0 \implies k \left(\psio \left. \pdv{u}{x} \right|_{x=0} - \psil \left. \pdv{u}{x} \right|_{x=\ell} \right) = \ko (\uel \psil + \uo \psio) = 0.
	\end{equation}
	Rearranging the result of \refeq{bc}, we find
	\begin{align} \label{bcs}
		0 &= \frac{\ko}{k} u(0, t) - \left. \pdv{u}{x} \right|_{x=0}, &
		0 &= \frac{\ko}{k} u(\ell, t) + \left. \pdv{u}{x} \right|_{x=\ell}
	\end{align}
	as the boundary conditions for \refeq{el}.
\end{solution}


\clearpage
\section{}

\begin{problem}
	Check that the boundary conditions you derived in problem~\ref{1} are reasonable by considering two limiting scenarios: $\ko / k \to 0$ and $\ko / k \to \infty$.  Show that the boundary conditions simplify to the forms one would expect based on physical intuition.
\end{problem}

\begin{solution}
	For the case $\ko / k \to \infty$, $k / \ko \to 0$ and the boundary conditions in \label{bcs} reduce to
	\begin{align}
		0 &= u(0, t), &
		0 &= u(\ell, t).
	\end{align}
	These are the boundary conditions for a string with fixed endpoints, as we discussed in class.  In the limit $\ko / k \to \infty$, infinite force is required to deflect the ends of the string.  Hence, the ends are effectively fixed.  The vibrations of the string are standing waves in the $(x, u)$ plane, with the ends as nodes.
	
	For the case $\ko / k \to 0$, the boundary conditions in \label{bcs} reduce to
	\begin{align}
		0 &= \left. \pdv{u}{x} \right|_{x=0}, &
		0 &= \left. \pdv{u}{x} \right|_{x=\ell}.
	\end{align}
	In the limit $\ko / k \to 0$, the force that penalizes displacements at the ends of the string disappears.  Physically, this means the ends of the string are allowed to slide up and down freely.  In this situation, the vibrations in $(x, u)$ are standing waves with the ends as antinodes.  An antinode is a local maximum or minimum of a wave, so the slope of the wave at such a point must be zero, as represented in the boundary conditions.
\end{solution}


\newcommand{\inty}{\int_0^\ell}
\newcommand{\Dy}{\Delta y}
\newcommand{\dudyw}{\bigg( \pdv{u}{y} \bigg)^2}
\newcommand{\uy}{u_y}

\section{Stretched membrane}

\begin{problem}
	Consider the transverse motion of a stretched membrane.  Assume the membrane is homogeneous, with density per unit area $\rho$, and let $u(x, y)$ denote the displacement from the stable equilibrium position.  For simplicity you may assume the boundary of the membrane is fixed.  Adapt the derivation for a stretched string to a membrane and find the Euler-Lagrange equation.
\end{problem}

\begin{solution}
	Let the membrane cover a square in the $(x, y)$ plane with corners at $(0, 0),\ (0, \ell),\ (\ell, 0)$, and $(\ell, \ell)$.  Assume the boundary around the edge of the square is fixed.  Let $k$ be the effective elastic modulus of the membrane.
	
	Analogous to problem~\ref{1}, let $\dd{x} \dd{y}$ denote an infinitesimal area of the membrane with mass $\dd{m}$.  Then we can find the kinetic energy as in \refeq{T}:
	\begin{equation} \label{T2}
		\dd{T} = \frac{\rho}{2} \dudtw \dd{x} \dd{y} \implies T = \frac{\rho}{2} \intx \inty \dudtw \dd{x} \dd{y}.
	\end{equation}
	Because we are assuming a fixed boundary, we only have one potential energy term, which is the work required to stretch the membrane.  In two dimensions, \refeq{U1} becomes
	\begin{equation}
		\dd{\Uq} = k \,\dd{\ell} = k (\sqrt{(\dd{x})^2 + (\dd{y})^2 + (\dd{u})^2} - \sqrt{(\dd{x})^2 + (\dd{y})^2} \approx \frac{k}{2} \bigg( \dv{u}{x} \bigg)^2 \dd{x} + \frac{k}{2} \bigg( \dv{u}{x} \bigg)^2 \dd{y},
	\end{equation}
	where we note that the cross terms in the two-dimensional Taylor series expansion have vanished.  Then
	\begin{equation} \label{U3}
		\Uq = \frac{k}{2} \intx \dudxw \dd{x} + \frac{k}{2} \int \dudyw \dd{y}.
	\end{equation}
	Using \refeq{T2} and \refeq{U3}, we have as the action
	\begin{equation}
		S[u] = \frac{\rho}{2} \intt \intx \inty \dudtw \dd{x} \dd{y} \dd{t} - \frac{k}{2} \intt \intx \dudxw \dd{x} \dd{t} - \frac{k}{2} \inty \inty \dudyw \dd{y} \dd{t}.
	\end{equation}
	In general, the Euler-Lagrange equation is given by
	\begin{equation} \label{el3}
		0 = \pdv{\Ld}{u} - \pdv{t} \pdv{\Ld}{\ut} - \pdv{x} \pdv{\Ld}{\ux} - \pdv{y} \pdv{\Ld}{\uy}.
	\end{equation}
	Here,
	\begin{align}
		\pdv{\Ld}{u} &= 0, &
		\pdv{\Ld}{\ut} &= \rho \pdv{u}{t}, &
		\pdv{\Ld}{\ux} &= -k \pdv{u}{x}, &
		\pdv{\Ld}{\uy} &= -k \pdv{u}{y},
	\end{align}
	so \refeq{el3} becomes
	\begin{equation}
		\rho \pdv[2]{u}{t} = k \left( \pdv[2]{u}{x} + \pdv[2]{u}{y} \right).
	\end{equation}
\end{solution}


\newcommand{\pio}{\pi_0}
\newcommand{\intR}{\int_R}
\newcommand{\phid}{\dot{\phi}}
\newcommand{\phidd}{\ddot{\phi}}
\newcommand{\phix}{\phi_x}
\newcommand{\phiy}{\phi_y}
\newcommand{\phiz}{\phi_z}

\section{$\pio$ mesons}

\begin{problem}
	A model for uncharged particles of mass $M$ and spin 0 involves a scalar field $\phi(t, x, y, z)$ with kinetic and potential energy
	\begin{align}
		T&= \intR \frac{\phid^2}{2} \dd{x} \dd{y} \dd{z}, &
		U &= \intR \left( \frac{\phix^2 + \phiy^2 + \phiz^2}{2} + \frac{M^2 \phi^2}{2} \right) \dd{x} \dd{y} \dd{z},
	\end{align}
	satisfying Hamilton's principle.  The region $R$ is simply connected and $u$ has fixed values and the boundaries of $R$.  What is the associated Euler-Lagrange equation?
\end{problem}

\begin{solution}
	The Lagrangian density is
	\begin{equation} \label{Lpi}
		\Ld = T - U = \frac{1}{2} \intR \left( \phid^2 - \phix^2 - \phiy^2 - \phiz^2 - M^2 \phi^2 \right) \dd{x} \dd{y} \dd{z}
	\end{equation}
	and the Euler-Lagrange equation is given by
	\begin{equation}
		0 = \pdv{\Ld}{\phi} - \pdv{}{t} \pdv{\Ld}{\phid} - \pdv{}{x} \pdv{\Ld}{\phix} - \pdv{}{y} \pdv{\Ld}{\phiy} - \pdv{}{z} \pdv{\Ld}{\phiz},
	\end{equation}
	where
	\begin{align}
		\pdv{\Ld}{\phi} &= -M^2 \phi, &
		\pdv{\Ld}{\phid} &= \phid, &
		\pdv{\Ld}{\phix} &= -\phix, &
		\pdv{\Ld}{\phiy} &= -\phiy, &
		\pdv{\Ld}{\phiz} &= -\phiz.
	\end{align}
	Then \refeq{Lpi} becomes
	\begin{align}
		\phidd + M^2 \phi &= \pdv[2]{\phi}{x} + \pdv[2]{\phi}{y} + \pdv[2]{\phi}{z}.
	\end{align}
\end{solution}


In writing these solutions, I consulted Gelfand's \emph{Calculus of Variations}.

\end{document}