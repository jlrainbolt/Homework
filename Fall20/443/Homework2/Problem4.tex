\state{(Peskin \& Schroeder 3.7)}{
	This problem concerns the discrete symmetries $P$, $C$, and $T$.
}



\prob{}{
	Compute the transformation properties under $P$, $C$, and $T$ of the antisymmetric tensor fermion bilinears, $\psib \sigmn \psi$, with $\sigmn = \frac{i}{2} [\gamm, \gamn]$.  This completes the table of the transformation properties of bilinears at the end of the chapter.
}

\sol{
	For the transformation under $P$,
	\al{
		P \psib \sigmn \psi P &= P \psib P P \sigmn P P \psi P
		= \etaas \psibtmx \gamo P \sigmn P \etaa \gamo \psitmx \\
		&= \frac{i}{2} \absetaa^2 \psibtmx \gamo [\gamm, \gamn] \gamo \psitmx
		= \frac{i}{2} \psibtmx \gamo [\gamm, \gamn] \gamo \psitmx,
	}
	where we have used Peskin \& Schroeder~(3.125), $\eta \etab = -\etaa \etas = -1$ and $\etabs = -\etaa$~\cite[p.~66]{Peskin}.  We have also used their Eqs.~(3.126) and (3.128),
	\al{
		P \psitx P &= \etaa \gamo \psitmx, &
		P \psibtx P &= \etaas \psibtmx \gamo.
	}
	From Eqs.~(3.26) and (3.27),
	\al{
		\sigoi &= \frac{i}{2} [\gamo, \gami] = -i \mqty( \sigi & 0 \\ 0 & -\sigi ), &
		\sigij &= \frac{i}{2} [\gami, \gamj] = \epsijk \mqty( \sigk & 0 \\ 0 & \sigk ).
	}
	Note also that
	\al{
		\gamo \mqty( \sigi & 0 \\ 0 & \pm\sigi ) \gamo = \mqty( 0 & \sigo \\ \sigo & 0 ) \mqty( \sigi & 0 \\ 0 & \pm\sigi ) \mqty( 0 & \sigo \\ \sigo & 0 )
		= \mqty( 0 & \sigo \\ \sigo & 0 ) \mqty( 0 & \sigi \\ \pm\sigi & 0 )
		= \mqty( \pm\sigi & 0 \\ 0 & \sigi ).
	}
	for any $\sigi$.  Then
	\al{
		P \psib \sigoo \psi P &= 0, \\
		P \psib \sigoi \psi P &= i \psibtmx \mqty( \sigi & 0 \\ 0 & -\sigi ) \psitmx
		= -\psibtmx \sigoi \psitmx, \\
		P \psib \sigio \psi P &= -i \psibtmx \mqty( \sigi & 0 \\ 0 & -\sigi ) \psitmx
		= -\psibtmx \sigio \psitmx, \\
		P \psib \sigij \psi P &= \epsijk \psibtmx \mqty( \sigk & 0 \\ 0 & \sigk ) \psitmx
		= \psibtmx \psib \sigij \psi \psitmx,
	}
	or
	\eq{\ans{
		P \psib \sigmn \psi P = \begin{cases}
			0 & \mu = \nu = 0; \\
			-\psibtmx  \sigmn \psitmx & \mu = 0, \nu \neq 0; \\
			-\psibtmx \sigmn \psitmx & \mu \neq 0, \nu = 0; \\
			\psibtmx \sigmn \psitmx & \mu \neq 0, \nu \neq 0.
		\end{cases}
	}}%
	
	For the transformation under $T$,
	\eq{
		T \psib \sigmn \psi T = T \psib T T \sigmn T T \psi T
		= -\psibmtx \gamq \game T \sigmn T \gamq \game \psimtx
		= -\psibmtx \gamq \game \sigsmn \gamq \game \psimtx,
	}
	where we have used Peskin \& Schroeder~(3.139) and (3.140),
	\al{
		T \psitx T &= -\gamq \game \psitmx, &
		T \psibtx T &= \psibmtx \gamq \game,
	}
	and Eq.~(3.133), $T (\text{c-number}) = (\text{c-number})^* T$.  Note that
	\eq{
		\gamq \game = \mqty( 0 & \sigq \\ -\sigq & 0 ) \mqty( 0 & \sige \\ -\sige & 0 )
		= -\mqty( \sigq \sige & 0 \\ 0 & \sigq \sige ),
	}
	and that
	\eq{
		\gamq \game \mqty( \sigi & 0 \\ 0 & \pm\sigi ) \gamq \game = \mqty( \sigq \sige & 0 \\ 0 & \sigq \sige ) \mqty( \sigi & 0 \\ 0 & \pm\sigi ) \mqty( \sigq \sige & 0 \\ 0 & \sigq \sige )
		= \mqty( \sigq \sige \sigi \sigq \sige & 0 \\ 0 & \pm \sigq \sige \sigi \sigq \sige ),
	}
	Taking each case of the matrix elements,
	\al{
		\sigq \sige \sigq \sigq \sige &= \sigq \sige \sige
		= \sigq, \\
		\sigq \sige \sigw \sigq \sige &= -\sigq \sige \sigq \sigw \sige
		= \sigq \sige \sigq \sige \sigw
		= -\sigq \sigq \sige \sige \sigw
		= -\sigw, \\
		\sigq \sige \sige \sigq \sige &= \sigq \sigq \sige
		= \sige.
	}
	The case of $T \psib \sigmn \psi T$ where $\mu = \nu = 0$ is trivially 0.  Now we consider the case where either $\mu$ or $\nu$ is 0.  Since $\sigow = \sigsow$ and $\sigoi = -\sigsoi$ for $i \in \{1, 3\}$,
	\al{
		T \psib \sigow \psi T &= -\psibmtx \gamq \game \sigow \gamq \game \psimtx
		= \psibmtx \sigow \psimtx, \\
		T \psib \sigoi \psi T &= \psibmtx \gamq \game \sigoi \gamq \game \psimtx
		= \psibmtx \sigoi \psimtx.
	}
	Finally we consider the case where $\mu \neq 0$ and $\nu \neq 0$.  If $i, j \in \{1, 3\}$, $\sigsij = -\sigij$ and $\sigiw = \sigsiw = \sigwi = \sigswi$.  Then
	\al{
		T \psib \sigij \psi T &= \psibmtx \gamq \game \sigij \gamq \game \psimtx
		= -\psibmtx \sigij \psimtx, \\
		T \psib \sigiw \psi T &= -\psibmtx \gamq \game \sigiw \gamq \game \psimtx
		= \psibmtx \sigiw \psimtx, \\
		T \psib \sigwi \psi T &= -\psibmtx \gamq \game \sigwi \gamq \game \psimtx
		= \psibmtx \sigwi \psimtx.
	}
	In summary,
	\eq{\ans{
		T \psib \sigmn \psi T = \begin{cases}
			0 & \mu = \nu = 0; \\
			\psibmtx  \sigmn \psimtx & \mu = 0, \nu \neq 0; \\
			\psibmtx \sigmn \psimtx & \mu \neq 0, \nu = 0; \\
			-\psibmtx \sigmn \psimtx & \mu \neq 0, \nu \neq 0.
		\end{cases}
	}}%
	
	For the transformation under $C$,
	\eq{
		C \psib \sigmn \psi C = C \psib C C \sigmn C C \psi C
		= -(\gamo \gamw \psi)^T C \sigmn C (\psib \gamo \gamw)^T
		= -(\gamo \gamw \psi)^T \sigmn (\psib \gamo \gamw)^T,
	}
	where we have used Peskin \& Schroeder~(3.145) and (3.146),
	\al{
		C \psix C &= -i (\psib \gamo \gamw)^T, &
		C \psibx C &= -i (\gamo \gamw \psi)^T.
	}
	Since $\gamo{}^T = \gamo$ and $\gami{}^T = -\gami$,
	\eq{
		C \psib \sigmn \psi C = -\psiT \gamw \gamo \sigmn \gamw \gamo \psibT.
	}
	Note that
	\eq{
		\gamw \gamo = \mqty( 0 & \sigw \\ -\sigw & 0 ) \mqty( 0 & \sigo \\ \sigo & 0 )
		= \mqty( \sigw & 0 \\ 0 & -\sigw ),
	}
	and that
	\eq{
		\gamw \gamo \mqty( \sigi & 0 \\ 0 & \pm\sigi ) \gamw \gamo = \mqty( \sigw & 0 \\ 0 & -\sigw ) \mqty( \sigi & 0 \\ 0 & \pm\sigi ) \mqty( \sigw & 0 \\ 0 & -\sigw )
		= \mqty( \sigw & 0 \\ 0 & -\sigw ) \mqty( 0 & \mp \sigi \sigw \\ \sigi \sigw & 0 )
		= -\mqty( \sigi & 0 \\ 0 & \pm \sigi ),
	}
	so
	\al{
		C \psib \sigoi \psi C &= \psiT \sigoi \psibT
		= -\psib \sigoi \psi, \\
		C \psib \sigio \psi C &= \psiT \sigio \psibT
		= -\psib \sigio \psi, \\
		C \psib \sigij \psi C &= \psiT \sigij \psibT
		= -\psib \sigij \psi,
	}
	where we have used anticommutation of fermions~\cite[p.~70]{Peskin}.  In summary,
	\eq{ \ans{
		C \psib \sigmn \psi C = \begin{cases}
			0 & \mu = \nu = 0; \\
			-\psib \sigij \psi & \text{otherwise}.
		\end{cases}
	}}%
	\vfix
}



\prob{}{ \label{3.7b}
	Let $\phix$ be a complex-valued Klein-Gordon field, such as we considered in Problem~2.2.  Find unitary operators $P$, $C$, and an antiunitary operator $T$ (all defined in terms of their action on the annihilation operators $\avp$ and $\bvp$ for the Klein-Gordon particles and antiparticles) that give the following transformations of the Klein-Gordon field:
	\al{
		P \phitx P &= \phitmx, &
		T \phitx T &= \phimtx, &
		C \phitx C &= \phistx.
	}
	Find the transformation properties of the components of the current
	\eq{
		\Jm = i (\phis \ptm \phi - \ptm \phis \phi)
	}
	under $P$, $C$, and $T$.
}

\sol{
	Referring to Peskin \& Schroeder~(2.47) and 2(b) of Homework~1, the complex-valued Klein-Gordon field can be written
	\al{
		\phitx &= \left. \int \ddcpf \frac{1}{\sqrt{2 \Ep}} \paren{ \avp e^{i p \cdot x} + \bvpdag e^{-i p \cdot x} } \right|_{\po = \Ep}, \\
		\phistx &= \left. \int \ddcpf \frac{1}{\sqrt{2 \Ep}} \paren{ \bvp e^{i p \cdot x} + \avpdag e^{-i p \cdot x} } \right|_{\po = \Ep}.
	}
	From here the evaluation at $\po = \Ep$ will be understood.
	
	Beginning with $P$, adapting Eq.~(3.123) to the spin-0 Klein-Gordon field,
	\ans{\al{
		P \avp P &= \anvp, &
		P \bvp P &= \bnvp,
	}}%
	where we choose the phase to be 1 without loss of generality~\cite[p.~66]{Peskin}.  These imply
	\al{
		P \avpdag P &= \anvpdag, &
		P \bvpdag P &= \bnvpdag.
	}
	Then
	\al{
		P \phitx P &= \int \ddcpf \frac{1}{\sqrt{2 \Ep}} P \paren{ \avp e^{i p \cdot x} + \bvpdag e^{-i p \cdot x} } P
		= \int \ddcpf \frac{1}{\sqrt{2 \Ep}} \paren{ P \avp P e^{i p \cdot x} + P \bvpdag P e^{-i p \cdot x} } \\
		&= \int \ddcpf \frac{1}{\sqrt{2 \Ep}} \paren{ \anvp e^{i p \cdot x} + \bnvpdag e^{-i p \cdot x} }.
	}
	Changing variables to $\pt = (\po, -\vp)$~\cite[p.~65]{Peskin},
	\eq{
		P \phitx P = \int \ddcptf \frac{1}{\sqrt{2 \Ept}} \paren{ \avpt e^{i \pt \cdot x} + \bvptdag e^{-i \pt \cdot x} }
		= \ans{ \phitmx }
	}
	as required.
	
	For $T$, we adapt Eq.~(3.138) as follows:
	\ans{\al{
		T \avp T &= \anvp, &
		T \bvp T &= \bnvp,
	}}%
	and
	\al{
		T \avpdag T &= \anvpdag, &
		T \bvpdag T &= \bnvpdag.
	}%
	Then
	\al{
		T \psitx T &= \int \ddcpf \frac{1}{\sqrt{2 \Ep}} T \paren{ \avp e^{i p \cdot x} + \bvpdag e^{-i p \cdot x} } T
		= \int \ddcpf \frac{1}{\sqrt{2 \Ep}} \paren{ T \avp T e^{-i p \cdot x} + T \bvpdag T e^{i p \cdot x} } \\
		&= \int \ddcpf \frac{1}{\sqrt{2 \Ep}} \paren{ \anvp e^{-i p \cdot x} + \anvp e^{i p \cdot x} }
		= \int \ddcptf \frac{1}{\sqrt{2 \Ept}} \paren{ \avpt e^{-i \pt \cdot x} + \avpt e^{i \pt \cdot x} } \\
		&= \ans{ \phimtx, }
	}
	where we have used the antiunitarity of $T$ and the fact that $\pt \cdot (t, -\vx) = -\pt \cdot (-t, \vx)$~\cite[p.~69]{Peskin}.
	
	For $C$, we adapt Eq.~(3.143) as follows:
	\ans{\al{
		C \avp C &= \bvp, &
		C \bvp C &= \avp,
	}}%
	which imply
	\al{
		C \avpdag C &= \bvpdag, &
		C \bvpdag C &= \avpdag.
	}
	So
	\al{
		C \psitx C &= \int \ddcpf \frac{1}{\sqrt{2 \Ep}} C \paren{ \avp e^{i p \cdot x} + \bvpdag e^{-i p \cdot x} } C
		= \int \ddcpf \frac{1}{\sqrt{2 \Ep}} \paren{ C \avp C e^{i p \cdot x} + C \bvpdag C e^{-i p \cdot x} } \\
		&= \int \ddcpf \frac{1}{\sqrt{2 \Ep}} \paren{ \bvp e^{i p \cdot x} + \avpdag e^{-i p \cdot x} } \\
		&= \ans{ \phistx }
	}
	as required.
	
	For the current, note that
	\al{
		P \phistx P &= \phistmx, &
		T \phistx T &= \phismtx, &
		C \phistx C &= \phitx,
	}
	and that
	\al{
		P \ptsm P &= \mqm \ptsm, &
		T \ptsm T &= -\mqm \ptsm
		C \ptsm C &= \ptsm,
	}
	where $\mqo = 1$ and $\mqi = -1$ for $i \in \{ 1, 2, 3 \}$~\cite[p.~71]{Peskin}.  Then we have
	\al{
		P \Jmtx P &= i P [ \phistx \ptm \phitx - \ptm \phistx \phitx ] P \\
		&= i [ P \phistx P P \ptm P P \phitx P - P \ptm P P \phistx P \phitx P ] \\
		&= i \mqm [ \phistmx \ptm \phitmx - \ptm \phistmx \phitmx ] \\
		&= \ans{ \mqm \Jmtmx, }
	}
	\al{
		T \Jmtx T &= -i T [ \phistx \ptm \phitx - \ptm \phistx \phitx ] T \\
		&= -i [ T \phistx T T \ptm T T \phitx T - T \ptm T T \phistx T T \phitx T ] \\
		&= i \mqm [ \phismtx \ptm \phimtx - \ptm \phismtx \phimtx ] \\
		&= \ans{ \mqm \Jmmtx, } \\[1.5ex]
		C \Jmtx C &= i C [ \phistx \ptm \phitx - \ptm \phistx \phitx ] C \\
		&= i [ C \phistx C C \ptm C C \phitx C - C \ptm C C \phistx C C \phitx C ] \\
		&= i [ \phitx \ptm \phistx - \ptm \phitx \phistx ]
		= -i [ \phistx \ptm \phitx - \ptm \phistx \phitx ] \\
		&= \ans{ -\Jmtx, }
	}
	where we have used $[\phi(\vx), \phis(\vy)] = 0$ from 2(b) of Homework~1.
}



\prob{}{
	Show that any Hermitian Lorentz-scalar local operator built from $\psix$, $\phix$, and their conjugates has $CPT = +1$.
}

\sol{
	The only Hermitian Lorentz-scalar local operator that can be built from these $\psix$ and its conjugate is $\psib \psi$.  From $\psix$ and its conjugate, we have $\phis \phi = \phi \phis$ since  $[\phi(\vx), \phis(\vy)] = 0$.  We can combine these to build the Hermitian Lorentz-scalar local operator
	\eq{
		(\psib \psi)^N (\phis \phi)^M,
	}
	where $N$ and $M$ are arbitrary integers.  Since
	\eq{
		[\psi(\vx), \phi(\vy)] = [\psib(\vx), \phi(\vy)]
		= [\psi(\vx), \phis(\vy)]
		= [\psib(\vx), \phis(\vy)]
		= 0,
	}
	other orderings of the fields are equivalent as long as $\psib$ and $\psi$ are in the correct order (for if they are not, the object is not a Lorentz scalar).
	
	We know from the table in Peskin \& Schroder that $\psib \psi$ has $CPT = +1$~\cite[p.~71]{Peskin}.  Using the results of \ref{3.7b},
	\al{
		C P T \phistx \phitx T P C &= C P \phismtx \phimtx P C
		= C \phismtmx \phimtmx C
		= \phimtmx \phismtmx \\
		&= \phismtmx \phimtmx,
	}
	so $\phis \phi$ has $CPT = +1$ also.  Finally, $(\psib \psi)^N (\phis \phi)^M$ has
	\eq{
		CPT = (+1)^N (+1)^M
		= (+1) (+1)
		= \ans{ +1 }
	}
	as we wanted to show. \qed
}