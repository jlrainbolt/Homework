\state{Electron-phonon interaction}{
	Write short notes explaining the physical effects that may be produced by the electron-phonon interaction in metals.
}

\sol{
	Phonons cause the crystal lattice to distort on a local scale, which moves the ions from their equilibrium positions.  Since the ions carry charge, this disturbance creates an electric potential that is screened by nearby conduction electrons.  The potential scatters electrons from state $\vk$ to state $\vk'$, which alters the density distribution of the electron gas.  The disturbance in the electron density caused by the scattering may in turn create a new phonon or lattice distortion, the degree of which is determined by the phonon susceptibility of the crystal~[lecture notes, p.~129--130]\cite[pp.~671--672]{Kittel}\cite[p.~512]{Ashcroft}.

	The lattice distortion created by an electron density fluctuation lasts longer than the fluctuation itself, and creates more local electron density fluctuations over its lifetime.  This retarded interaction creates an effective ``attraction'' between conduction electrons in the metal, which can lead to superconductivity and the creation of Cooper pairs.  In addition, the interaction between phonons and electrons causes electrons to effectively carry polarized lattice distortions with them as they move.  This decreases their effective velocity and increases their effective mass~[lecture notes, p.~130--134]\cite[pp.~672]{Kittel}.
}