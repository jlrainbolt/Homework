\state{Electronic mass enhancement}{
	The integral in Eq.~(7.10) can be approximated by neglecting the momentum dependence of the coupling constant $g$, and replacing the phonon frequency by the characteristic scale $\omgD$.  Show that in this case the integral becomes
	\eq{
		g^2 \intnimu \ddepsp \frac{\Nepsp}{(\eps' - \epsk)^2 - \omgD^2}
	}
	where $\Neps$ is the density of states in energy.
}

\sol{
	Equation~(7.10) is
	\eqn{7.10}{
		\epsk - \mu = \epsko - \mu - \int \frac{\ddkp}{(2\pi)^3} \frac{\abs{\gkmkp}^2 \nkp}{(\epsk - \epskp)^2 - \omg (\vk - \vk')^2}.
	}
	Applying $\omg \to \omgD$ and neglecting the momentum dependence of $g$,
	\eq{
		\int \frac{\ddkp}{(2\pi)^3} \frac{\abs{\gkmkp}^2 \nkp}{(\epsk - \epskp)^2 - \omg (\vk - \vk')^2}
		\approx g^2 \int \frac{\ddkp}{(2\pi)^3} \frac{\nkp}{(\epsk - \epskp)^2 - \omgD (\vk - \vk')^2}.
	}
	Since the mass enhancement only exists for states whose energies are the same within $\hbar \omgD$~(lecture notes p.~132), $(\vk - \vk')^2 \approx \omgD$.  Thus
	\aln{
		g^2 \int \frac{\ddkp}{(2\pi)^3} \frac{\nkp}{(\epsk - \epskp)^2 - \omgD (\vk - \vk')^2}
		&\approx g^2 \int \frac{\ddkp}{(2\pi)^3} \frac{\nkp}{(\epsk - \epskp)^2 - \omgD^2} \notag \\
		&= g^2 \iiint \frac{{k'}^2 \ddkkp \ddcost \ddphi}{(2\pi)^3} \frac{\nkp}{(\epsk - \epskp)^2 - \omgD^2} \notag \\
		&= 4\pi g^2 \int \frac{{k'}^2 \ddkkp}{(2\pi)^3} \frac{\nkp}{(\epsk - \epskp)^2 - \omgD^2} \label{thing2a}
	}
	The definition of $\nk$ is given on p.~91 of the lecture notes:
	\eq{
		n(k) = \begin{cases}
			1 & \abs{k} < \kF, \\
			0 & \text{otherwise}.
		\end{cases}
	}
	Feeding this into Eq.~\refeq{thing2a}, we can change the limits of integration to $(0, \kF)$:
	\eqn{thing2a2}{
		4\pi g^2 \int \frac{{k'}^2 \ddkkp}{(2\pi)^3} \frac{\nkp}{(\epsk - \epskp)^2 - \omgD^2} = 4\pi g^2 \intokF \frac{{k'}^2 \ddkkp}{(2\pi)^3} \frac{1}{(\epsk - \epskp)^2 - \omgD^2}.
	}
	We can transform variables by (2.11) in the lecture notes, which gives the density of states for an electron gas.  Assuming this is an appropriate model for our metal, and that $\Neps$ represents the density of states per unit volume, we use
	\eq{
		\ddkk = \frac{1}{2} \frac{(2\pi)^3}{4\pi k^2} \Neps \ddeps.
	}
	Then Eq.~\refeq{thing2a2} becomes
	\eq{
		4\pi g^2 \intokF \frac{{k'}^2 \ddkkp}{(2\pi)^3} \frac{1}{(\epsk - \epskp)^2 - \omgD^2} = g^2 \intnpepsF \ddepsp \frac{\Nepsp}{(\eps' - \epskp)^2 - \omgD^2}.
	}
	(The factors of 2 and the bounds of integration do not seem to match up here...)
	
	In the approximation $\epsF \approx \mu$, we can replace the limits of integration by $(-\mu, \mu)$.  For the lower bound, we assume that large negative do not contribute much: that is, $-\mu \to -\infty$.  Then
	\eqn{ans2a}{
		\ans{ \int \frac{\ddkp}{(2\pi)^3} \frac{\abs{\gkmkp}^2 \nkp}{(\epsk - \epskp)^2 - \omg (\vk - \vk')^2} \approx g^2 \intnimu \ddepsp \frac{\Nepsp}{(\eps' - \epsk)^2 - \omgD^2} }
	}
	as desired. \qed
}



\prob{
	Since the dominant part of the integral comes from energies near the Fermi energy, we can usually replace $\Neps$ by $\Nmu$.  Making this approximation, show that for energies $\abs{\epsk - \mu} \ll \omgD$
	\eq{
		\epsk - \mu = \frac{\epsko - \mu}{1 + \lam}
	}
	where
	\eq{
		\lam = \frac{g^2 \Nmu}{\omgD^2}.
	}
}

\sol{
	Applying Eq.~\refeq{ans2a} and this approximation, Eq.~\refeq{7.10} becomes
	\eqn{thing2b}{
		\epsk - \mu = \epsko - \mu - g^2 \Nmu \intnimu \frac{\ddepsp}{(\eps' - \epsk)^2 - \omgD^2}.
	}
	Since we assume that only energies very close to the Fermi energy $\epsF \approx \mu$ contribute, we approximate the integral by the indefinite integral at $\eps' = \mu$.  Thus (using Mathematica)
	\eq{
		\intnimu \frac{\ddepsp}{(\eps' - \epsk)^2 - \omgD^2} \approx \int \frac{\ddmu}{(\mu - \epsk)^2 - \omgD^2}
		= -\frac{1}{\omgD} \tanh[-1](\frac{\mu - \epsk}{\omgD}).
	}
	Since $\abs{\epsk - \mu} \ll \omgD$, we can perform a Taylor expansion (again using Mathematica):
	\eq{
		\tanh[-1](\frac{\mu - \epsk}{\omgD}) \approx \frac{\mu - \epsk}{\omgD}.
	}
	So we have for Eq.~\refeq{thing2b}
	\eq{
		\epsk - \mu = \epsko - \mu + g^2 \Nmu \frac{\mu - \epsk}{\omgD^2}
		= \epsko - \mu - \lam (\epsk - \mu).
	}
	We assume that replacing $\epsk$ by $\epsko$ on the right side, thereby  ignoring the ionic correction to the screening in that term~\cite[p.~520]{Ashcroft}, is a valid approximation in this regime.  Then we have
	\eq{
		\epsk - \mu = (\epsko - \mu) (1 - \lam)
		\approx \ans{ \frac{\epsko - \mu}{1 + \lam}, }
	}
	where we have simply used the Taylor expansion $1 / (1 + x) \approx 1 - x$ for small $x$.  In doing so we have assumed $\lam$, and therefore the mass enhancement, is small.  Nevertheless, we have achieved the desired result. \qed
}



\prob{
	Making the approximation $\Neps \approx \Nmu$, show that for energies $\abs{\epsk - \mu}$ several times $\omgD$ the correction to $\epsk$ is of order
	\eq{
		\lam \frac{\omgD^2}{(\epsk - \mu)^2} (\epsk - \mu).
	}
}

\sol{
	In this limit, we again approximate the integral in Eq.~\refeq{thing2b} as an antiderivative at $\eps' = \mu$.  In this regime we also approximate the denominator of the integrand by $(\eps' - \epsk)^2$, since $(\eps' - \epsk)^2 \gg \omgD^2$.  Then
	\eq{
		\intnimu \frac{\ddepsp}{(\eps' - \epsk)^2 - \omgD^2} \approx \int \frac{\ddmu}{(\mu - \epsk)^2}
		= -\frac{1}{\mu - \epsk},
	}
	so the correction term in Eq.~\refeq{thing2b} becomes
	\eq{
		\frac{g^2 \Nmu}{\mu - \epsk} = -\frac{\lam \omgD^2}{\epsk - \mu}
		\propto \ans{ \lam \frac{\omgD^2}{(\epsk - \mu)^2} (\epsk - \mu) }
	}
	as we wanted to show. \qed
}