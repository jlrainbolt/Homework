\state{Bloch's theorem}{	\label{1}
	Prove Bloch's theorem by operating with the translation operator on $H \psi$ and using the periodic symmetry of the potential.  Show furthermore that $\TR \TRp = \TRp \TR = \TRpRp$; the translation operators commute with themselves.
}

\sol{
	Bloch's theorem is given by (8.6) of Ashcroft \& Mermin:
	\eqn{bloch}{
		\psirpR = e^{i \vk \vdot \vR} \psir.
	}
	Additionally, $H \psir$ is given by (4.1):
	\eq{
		H \psir = \paren{ -\frac{\hbar^2}{2 m} \laplacian + \Ur } \psir.
	}
	
	According to (4.5) in the lecture notes, the action of the translation operator is $\TR \fr = \frpR$.  Operating on $H \psi$ yields~\cite[p.~134]{Ashcroft}
	\al{
		\TR (H \psi) &= \HrpR \psirpR \\
		&= \paren{ -\frac{\hbar^2}{2 m} \laplacian + \UrpR } \psirpR \\
		&= \paren{ -\frac{\hbar^2}{2 m} \laplacian + \Ur } \psirpR \\
		&= \Hr \psirpR \\
		&= H \TR \psi,
	}
	since $\Ur = \UrpR$ for all $\vR$ in a Bravais lattice according to p.~35 of the lecture notes.  Thus we have shown
	\eq{
		\TR H = H \TR;
	}
	that is, $H$ and $\TR$ commute.
	
	Note also that
	\eq{
		\TR \TRp \fr = \TR f(\vr + \vR')
		= f(\vr + \vR' + \vR)
		= f(\vr + \vR + \vR')
		= \TRp f(\vr + \vR)
		= \TRp \TR \fr
	}
	and that
	\eq{
		\TRpRp \fr = f(\vr + \vR + \vR'),
	}
	so we have shown that
	\eq{ \eqqed{
		\ans{ \TR \TRp = \TRp \TR = \TRpRp. } \qedhere
	}}
	
	Since $H$ and $\TR$ commute and the $\TR$s commute with each other, they can be simultaneously diagonalized.  We can apply (4.8) of the lecture notes:
	\aln{ \label{eigenvals}
		H \psi &= E \psi, &
		\TR \psi &= \cR \psi.
	}
	Now applying what we have just proven about the $\TR$s,
	\eqn{cRs}{
		\cR \cRp \psi = \TR \TRp \psi
		= \TRpRp \psi
		= \cRpRp \psi
		\qimplies
		\cR \cRp = \cRpRp.
	}
	From Ashcroft \& Mermin~(4.1), any position vector $\vR$ on the Bravais lattice can be written
	\eq{
		\vR = \nq \vaq + \nw \vaw + \nee \vae,
	}
	where the $\nii$ are integers.  So using Eq.~\refeq{cRs}, we can write
	\eqn{thing1}{
		\cR = c(\nq \vaq + \nw \vaw + \nee \vae)
		= c(\nq \vaq) c(\nw \vaw) c(\nee \vae)
		= c(\vaq)^{\nq} c(\vaw)^{\nw} c(\vae)^{\nee}.
	}
	We define the numbers $\ki$ such that (4.10) of the lecture notes is satisfied:
	\eq{
		\cai = e^{2\pi i \ki},
	}
	where $\vai$ are the primitive vectors of the lattice.  Applying this to Eq.~\refeq{thing1},
	\eqn{thing1.b}{
		\cR = (e^{2\pi i \kq})^{\nq} (e^{2\pi i \kw})^{\nw} (e^{2\pi i \ke})^{\nee}
		= e^{2\pi i (\kq \nq + \kw \nw + \ke \nee)}
		= e^{i \vk \vdot \vR},
	}
	where
	\eq{
		\vk = \kq \vbq + \kw \vbw + \ke \vbe
	}
	from Ashcroft \& Mermin~(5.6) is any vector on the reciprocal lattice (with primitive vectors $\vbi$), and we have applied their~(4.8),
	\eq{
		\vk \vdot \vR = 2\pi ( (\kq \nq + \kw \nw + \ke \nee).
	}
	Putting Eq.~\refeq{thing1.b} together with Eq.~\refeq{eigenvals} and the definition of $\TR$, we have
	\eq{
		\ans{ \psirpR = \TR \psir
		= \cR \psir
		= e^{i \vk \vdot \vR} \psir,}
	}
	which is Bloch's theorem~\cite[pp.~134--135]{Ashcroft}. \qed
}