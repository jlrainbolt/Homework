\documentclass[11pt]{article}
\usepackage{geometry, titlesec}
\usepackage[parfill]{parskip}
\usepackage[italicdiff]{physics}
\usepackage{amsfonts, amsthm}
\usepackage[cm]{fullpage}
\usepackage{fancyhdr}
\usepackage{xcolor}
\usepackage{siunitx}
%\allowdisplaybreaks
 
%\renewcommand{\footrulewidth}{.2pt}
%\setlist[enumerate]{leftmargin=*}
\pagestyle{fancy}
\fancyhf{}
\lhead{Homework 2}
\rhead{Physics 133-B}
\setlength{\headheight}{11pt}
\setlength{\headsep}{11pt}
%\setlength{\footskip}{24pt}
%\cfoot{\today}
%\rfoot{\thepage}

\titleformat{\section}[runin]{\normalfont\large\bfseries}{Problem \thesection.}{1em}{}
\titleformat{\subsection}[runin]{\normalfont\large\bfseries}{\thesubsection}{1em}{}
\titleformat{\subparagraph}[leftmargin]{\normalfont\large\bfseries}{}{0pt}{}
\newcommand{\refeq}[1]{(\ref{#1})}

\newcommand{\beq}{\begin{equation*}}
\newcommand{\eeq}{\end{equation*}}

\newcommand{\beqn}{\begin{equation}}
\newcommand{\eeqn}{\end{equation}}

\newcommand{\qimplies}{\quad \implies \quad}


\newenvironment{statement}[1]
{
	\section{#1}
	\ignorespaces
}

\newenvironment{problem}
{
    \paragraph{Problem.}
    \color{darkgray}
    \ignorespaces
}

\newenvironment{solution}
{
    \paragraph{Solution.}
    \ignorespaces
}

\DeclareSIUnit{\mph}{mph}

\begin{document}


\newcommand{\ofreq}{\SI{10.5}{\giga\Hz}}
\newcommand{\ifreq}{\SI{12.5}{\giga\Hz}}
\newcommand{\pspeed}{\SI{35}{\meter\per\second}}
\newcommand{\sspeed}{\SI{30.6}{\meter\per\second}}
\newcommand{\vsound}{\SI{344}{\meter\per\second}}

\setcounter{section}{1}
\begin{statement}{}
	% Y&F 16.51
	A police car is waiting on the shoulder of Lake Shore Drive.  In order to catch speeding commuters, the officer uses a radar gun that emits sound of frequency {\ofreq}.  There is a single speeding car on the road, which is directly in front of or behind her.  When she fires the gun, the frequency that returns from the speeder's car is {\ifreq}.  What is the speed of the car?  Is it moving toward the officer, or away from her?

	A few moments later, the officers takes off at {\pspeed} in pursuit of the speeder, who does not change his speed.  If the officer fires the gun now, what frequency will she receive?
\end{statement}


\newcommand{\vL}{v_L}
\newcommand{\vS}{v_S}
\newcommand{\fL}{f_L}
\newcommand{\fS}{f_S}
\newcommand{\vcar}{v_\text{car}}
\newcommand{\voff}{v_\text{off}}

\begin{solution}
	This is a problem about the Doppler effect.  In general, we have
	\beqn \label{dopp}
		\fL = \frac{v \pm \vL}{v \pm \vS} \fS,
	\eeqn
	where $\fL$ and $\fS$ are the frequenceis in the frames of the listener and source, respectively, $\vL$ and $\vS$ are the velocities of the listener and source, and $v$ is the speed of sound.  (The speed of sound in air is $v = \vsound$.)  Written in this form, we choose a $+$ sign in the numerator if the listener is moving toward the source, and a $-$ sign if it is moving away.  The convention is flipped for the denominator: we choose a $-$ sign if the \emph{source} is moving toward the \emph{listener}, and a $+$ sign if it is moving away.
	
	It is easiest to answer the question about the direction of the car first.  Since the frequency that the officer hears is greater than the frequency she emitted, we know {\color{blue} the speeding car is moving toward her}.

	In order to find the speed of the car, we need to apply \refeq{dopp} twice.  Let $f$ be the {\ofreq} frequency emitted by the radar gun, and $f'$ the frequency that is ``heard'' by the speeding car.  Equation~\refeq{dopp} becomes
	\beqn \label{f}
		f' = \frac{v + \vcar}{v} f,
	\eeqn
	where $\vcar$ is the velocity of the speeding car, which is the ``listener'' in this scenario, and is moving toward the officer.  The officer, who is the ``source,'' is stationary.  Now we can use $f'$ to find the {\ifreq} frequency that returns to the officer, which we will call $f''$.  This is found by writing Eq.~\refeq{dopp} as
	\beqn \label{ff}
		f'' = \frac{v}{v - \vcar} f',
	\eeqn
	where the stationary officer is now the ``listener,'' and the speeding car is now the ``source,'' which again is moving toward the officer.
	
	Substituting Eq.~\refeq{f} into Eq.~\refeq{ff}, we find
	\beq
		f'' = \frac{v}{v - \vcar} \frac{v + \vcar}{v} f
		= \frac{v + \vcar}{v - \vcar} f.
	\eeq
	Now we can solve for $\vcar$:
	\beq
		f'' (v - \vcar) = f (v + \vcar)
		\qimplies
%		\frac{f''}{f} v - \frac{f''}{f} \vcar = v + \vcar
%		\qimplies
%		\frac{f''}{f} v - v = \frac{f''}{f} \vcar + \vcar
%		\qimplies
		(f'' - f) v = (f'' + f) \vcar
		\qimplies
		\vcar = \frac{f'' - f}{f'' + f} v.
	\eeq
	Finally, we can plug in known quantities to obtain
	\beq
		\vcar = \frac{(\ifreq) - (\ofreq)}{(\ifreq) + (\ofreq)} (\vsound)
		= \frac{2}{22.5} (\vsound)
		= {\color{blue} \sspeed},
	\eeq
	which is over \SI{68}{\mph}!  (The speed limit on Lake Shore Drive is 40--\SI{45}{\mph}.)
	
	When the officer pursues the car, the equivalent of Eq.~\refeq{f} is
	\beq
		f' = \frac{v - \vcar}{v - \voff} f,
	\eeq
	where $\voff$ is the speed of the officer's police car.  Here we use a $-$ sign in the numerator since the speeder is moving away from the officer, and a $-$ sign in the denominator because the officer is moving toward the speeder.  Likewise, the equivalent of Eq.~\refeq{ff} is
	\beq
		f'' = \frac{v + \voff}{v + \vcar} f'.
	\eeq
	Once again, we may substitute in $f'$ to find
	\beq
		f'' = \frac{v + \voff}{v + \vcar} \frac{v - \vcar}{v - \voff} f.
	\eeq
	Finally, plugging in quantities,
	\begin{align*}
		f'' &= \frac{(\vsound) + (\pspeed)}{(\vsound) + (\sspeed)} \frac{(\vsound) - (\sspeed)}{(\vsound) - (\pspeed)} (\ofreq)
		= \frac{379}{374.6} \frac{313.4}{309} (\ofreq) \\
		&= 1.03 (\ofreq)
		= {\color{blue} \SI{10.8}{\giga\Hz}}.
	\end{align*}
\end{solution}

\end{document}