\documentclass[12pt]{article}
\usepackage{geometry}
\usepackage{graphicx}
\usepackage[parfill]{parskip}
\pagenumbering{gobble}


\title{Physics 335 project proposal: \\
	Design and construction of a system \\ to test ATLAS TileCal electronics}
\author{\textbf{Student:} Lacey Rainbolt, \texttt{jlrainbolt@uchicago.edu} \\ \textbf{Advisor:} Mark Oreglia, \texttt{m-oreglia@uchicago.edu}}
\date{\today}

\begin{document}
\maketitle

\section*{Project description}

The Tile Calorimeter~(TileCal) is the hadronic calorimeter in the ATLAS experiment measuring proton-proton collisions at CERN's Large Hadron Collider.  The TileCal covers the barrel region of the ATLAS detector and provides energy and direction measurements for physics objects such as jets, taus, hadrons, and missing energy.  The electronics used to record calorimeter data is being redesigned to handle higher rates and ambient radiation at the high luminosity upgrade of the LHC.  The current prototype of the new TileCal electronics must be tested in order to assess its functionality, performance, and lifetime under these detector conditions.

The 335 project will consist of designing and building a system to run a prototype of the new electronics during a burn-in period.  The system will record the data stream from the new electronics and exercise the communication paths for configuring its parameters, such as gains.  In addition, the system will provide a user interface to view and analyze the data.  Data analysis will include measurement of noise levels and the linearity of energy response.   The user interface will eventually be used to monitor the electronics during their use in the ATLAS detector.

\section*{Signatures}


\end{document}
