\state{Thermodynamics of solids}{
	Compute the following thermodynamic quantities for the harmonic photonic modes in a 1D and a 2D crystal at low temperatures (a.k.a.~phonons) and compare with the textbook example of a 3D crystal.
}

%
%	6.1
%

\prob{}{
	Free energy.
}

\sol{
	A crystal of $N$ molecules, each containing $\nu$ atoms, is composed of quantum harmonic oscillators that are free to oscillate in all spatial dimensions.  We can count the number of states in the interval $\ddk$, where $k$ is the wave number.  For a crystal, it is related to the frequency of vibration by $k = d \omg / \ub$, where $\ub$ is the averaged velocity of sound for the particular crystal structure and $d$ the number of spatial dimensions.  The number of states in the interval is, for each case,
	\al{
		\frac{L}{2\pi} \ddk
		&= \frac{L}{2\pi \ub} \ddomg
		\quad (d = 1), &
		\frac{2 \pi A}{(2\pi)^2} k \ddk
		&= \frac{A}{\pi \ub^2} \omg \ddomg,
	}
	where we have taken into account that there are $d$ independent polarization directions~\cite[pp.172--173]{Landau}.
	
	The free energy is $F = N \epso - T \ln Z$, where $\epso$ is the energy per molecule when the system is at equilibrium, which depends on $N$ and the volume $V$~\cite[pp.~87, 172]{Landau}.  The single-particle vibrational partition function is~\cite[p.~136]{Landau}
	\eq{
		Z_1 = \frac{1}{1 - e^{-\hbar \omg / T}}.
	}
	The entire crystal can be modeled as $d \,N \nu$ independent oscillators with total free energy~\cite[p.~172]{Landau}
	\eq{
		F = N \epso - T \sum_{\alp = 1}^{d \,N \nu} \ln(1 - e^{-\hbar \omg_\alp / T}).
	}
	For the entire crystal, the sum can be transformed to an integral over $\omg \in (0, \infty)$~\cite[p.~173]{Landau}.  Referring to the similar integrals in Prob.~{5.2}, we have
	\al{
		(d = 1) \quad
		F &= N \epso - \frac{L T}{2\pi \ub} \int \ln(1 - e^{-\hbar \omg / T}) \ddomg
		= N \epso - \frac{L T^2}{2\pi \hbar \ub} \frac{\pi}{6}
		= \ans{ N \epso - \frac{L T^2}{12 \hbar \ub}, } \\[2ex]
		(d = 2) \quad
		F &= N \epso - \frac{A T}{\pi \ub^2} \int \omg \ln(1 - e^{-\hbar \omg / T}) \ddomg
		= N \epso - \frac{A T^3}{\pi \hbar^2 \ub^2} \Gam(3) \zeta(3)
		= \ans{ N \epso - \frac{1.202 \,A T^3}{\pi \hbar^2 \ub^2}. }
	}
	The 3D expression is~\cite[p.~173]{Landau}
	\eq{
		F = N \epso - \frac{\pi^2 V T^3}{30 \hbar^3 \ub^3},
	}
	suggesting
	\eq{
		\ans{ F = N \epso - j(d) \,\frac{L^d T^{d + 1}}{\hbar^d \ub^d}, }
	}
	where $j(d)$ is a constant that depends on the number of dimensions, and we note that both $\epso$ and $\ub$ depend on the crystal structure and therefore $d$.
}

%
%	6.2
%

\prob{}{
	Entropy.
}

\sol{
	As in Prob.~{5.3}, $S = -\pdv{F}{T}$:
	\al{
		S &= -\pdv{T}(N \epso - \frac{L T^2}{12 \hbar \ub})
		= \ans{ \frac{L T}{6 \hbar \ub} }
		\quad (d = 1), &
		S &= -\pdv{T}(N \epso - \frac{1.202 \,A T^3}{\pi \hbar^2 \ub^2})
		= \ans{ \frac{3.606 \,A T^2}{\pi \hbar^2 \ub^2} }
		\quad (d = 2).
	}
	In 3D, the entropy is~\cite[p.~173]{Landau}
	\eq{
		S = \frac{2 \pi^2 V T^3}{15 \hbar^3 \ub^3},
	}
	which suggests
	\eq{
		\ans{ S \propto \frac{L^d T^d}{\hbar^d \ub^d}. }
	}
	\vfix
}

%
%	6.3
%

\prob{}{
	Energy.
}

\sol{
	As in Prob.~{5.3}, $E = F + T S$:
	\al{
		(d = 1) \quad
		E &= N \epso - \frac{L T^2}{12 \hbar \ub} + T \frac{L T}{6 \hbar \ub}
		= \ans{ N \epso + \frac{L T^2}{12 \hbar \ub}, } \\[2ex]
		(d = 2) \quad
		E &= N \epso - \frac{1.202 \,A T^3}{\pi \hbar^2 \ub^2} + T \frac{3.606 \,A T^2}{\pi \hbar^2 \ub^2}
		= \ans{ N \epso + \frac{2.404 \,A T^3}{\pi \hbar^2 \ub^2}. }
	}
	The 3D equivalent is~\cite[p.~173]{Landau}
	\eq{
		E = N \epso + \frac{\pi^2 T^4}{10 \hbar^3 \ub^3},
	}
	which suggests
	\eq{
		\ans{ E = N \epso + j(d) \frac{d T^{d + 1}}{\hbar^d \ub^d}
		= N - d \,F, }
	}
	where $j(d)$ is a constant that depends on the number of dimensions, and is not necessarily the same as that in Prob.~{6.1}.
}

%
%	6.4
%

\prob{}{
	Specific heat.
}

\sol{
	As in Prob.~{5.4}, $\Cv = (\pdv*{E}{T})_V$.  Then
	\al{
		C &= \pdv{T}(N \epso + \frac{L T^2}{12 \hbar \ub})
		= \ans{ \frac{L T}{6 \hbar \ub} }
		\quad (d = 1), &
		C &= \pdv{T}(N \epso + \frac{2.404 \,A T^3}{\pi \hbar^2 \ub^2})
		= \ans{ \frac{7.212 \,A T^2}{\pi \hbar^2 \ub^2} }
		\quad (d = 2).
	}
	\clearpage
	In 3D~\cite[p.~173]{Landau}
	\eq{
		C = \frac{2 \pi^2 V T^3}{5 \hbar^3 \ub^3},
	}
	which suggests
	\eq{
		\ans{ C \propto \frac{L^d T^d}{\hbar^d \ub^d}. }
	}
	\vfix
}