\state{Fluctuations of thermodynamics}{\ }

%
%	4.1
%

\prob{}{
	Find the energy fluctuation $\evDEs = \ev{(E - \evE)^2}$ and the number fluctuation $\evDNs = \ev{(N - \evN)^2}$ for photons in the black body radiation.
}

\sol{
	Planck's distribution, which gives the occupation number for state $k$ of a blackbody, is~\cite[p.~163]{Landau}
	\eq{
		\evnk = \frac{1}{e^{\hbar \omgk / T} - 1}.
	}
	This is a special case of the Bose distribution with $\mu = 0$ and $\epsk = \hbar \omgk$.  The Bose distribution is~\cite[p.~146]{Landau}
	\eq{
		\evnk = \frac{1}{e^{(\epsk - \mu)/T} - 1}.
	}
	Applying $\evDNs = T \pdv*{\!\evN}{\mu}$, which is derived in Prob.~{4.2}, we find~\cite[p.~355]{Landau}
	\al{
		\ev{(\Del\nk)^2} &= T \pdv{\!\evnk}{\mu}
		= T \pdv{\mu}(\frac{1}{e^{(\epsk - \mu)/T} - 1})
		= T \frac{e^{(\epsk - \mu)/T}}{T (e^{(\epsk - \mu)/T} - 1)^2}
		= \frac{e^{(\epsk - \mu)/T} - 1 + 1}{(e^{(\epsk - \mu)/T} - 1)^2} \\
		&= \frac{1}{e^{(\epsk - \mu)/T} - 1} + \frac{1}{(e^{(\epsk - \mu)/T} - 1)^2}
		= \evnk (1 + \evnk).
	}
	
	The number of photons in the frequency interval $\ddomg$ is~\cite[p.~163]{Landau}
	\eq{
		\ddNomg = \frac{V}{\pi^2 c^3} \frac{\omg^2}{e^{\hbar \omg / T} - 1} \ddomg
		= \frac{V}{\pi^2 c^3} \omg^2 \evnk \ddomg,
	}
	where $\evnk = 1 / (e^{\hbar \omg / T} - 1)$ is the Planck distribution.  By analogy,
	\eq{
		\ev{(\Del \ddNomg)^2} = \frac{V}{\pi^2 c^3} \omg^2 \ev{(\Del\nk)^2} \ddomg
		= \frac{V}{\pi^2 c^3} \omg^2 \evnk (1 + \evnk) \ddomg
		= \ddNomg + \evnk \ddNomg.
	}
	For the total number of particles, we integrate over $\omg \in (0, \infty)$~\cite[p.~165]{Landau}:
	\al{
		\evDNs &= \intoi \ev{(\Del \ddNomg)^2}
		= \intoi (\ddNomg + \evnk \ddNomg)
		= N + \frac{V}{\pi^2 c^3} \intoi \frac{\omg^2}{(e^{\hbar \omg / T} - 1)^2} \ddomg \\
		&= N + \frac{V T^3}{\pi^2 c^3 \hbar^3} \intoi \frac{x^2}{(e^x - 1)^2} \ddx
		= \frac{V T^3}{\hbar^3 c^3} \frac{2 \,\zeta(3)}{\pi^2} + \frac{V T^3}{\pi^2 c^3 \hbar^3} \paren{ \frac{\pi^2}{3} - 2 \,\zeta(3) }
		= \ans{ \frac{V T^3}{3 c^3 \hbar^3}, }
	}
	where $N$ is given in the book, and we have evaluated the second integral using Mathematica.
	
	Likewise, the radiation energy in the interval $\ddomg$ is $\ddEomg = \hbar\omg \ddNomg$.  So we need to multiply $\ev{(\Del \ddNomg)^2}$ by $(\hbar \omg)^2$~\cite[p.~346]{Landau}:
	\al{
		\evDEs &= \hbar^2 \intoi \omg^2 \ev{(\Del \ddNomg)^2}
		= \hbar^2 \intoi \omg^2 (\ddNomg + \evnk \ddNomg) \\
		&= \frac{\hbar^2 V}{\pi^2 c^3} \paren{ \intoi \frac{\omg^4}{e^{\hbar \omg / T} - 1} \ddomg + \intoi \frac{\omg^4}{(e^{\hbar \omg / T} - 1)^2} \ddomg }
		= \frac{\hbar^2 V}{\pi^2 c^3} \brac{ \frac{24 \,\zeta(5) \,T^5}{h^5} + \paren{ \frac{4\pi^4 T^5}{15 \hbar^5} - \frac{24 \,\zeta(5) \,T^5}{\hbar^5} } } \\
		&= \frac{\hbar^2 V}{\pi^2 c^3} \frac{4\pi^4 T^5}{15 \hbar^5}
		= \ans{ \frac{4\pi^2 V T^5}{15 c^3 \hbar^3}. }
	}
	\vfix
}

%
%	4.2
%

\prob{}{
	Show that the number of particles in a sub-volume of a gas fluctuates according the formula $\evDNs = T \pdv*{\!\evN}{\mu}$.  Furthermore, apply this formula to the Boltzmann, Fermi, and Bose ideal gases.
}

\sol{
	Let $p(x)$ denote the probability of a fluctuation in $x$.  Then $p(x) \propto e^{S(x)}$, where $S(x)$ is the entropy of a closed system representing a sub-volume of a gas~\cite[pp.~343, 348]{Landau}.  It follows that $p(x) \propto e^{\Del S(x)}$, where $\Del S(x)$ is the change in the entropy due to the fluctuation~\cite[p.~348]{Landau}.  This change is equal to the difference between $S(x)$ and its equilibrium value, which is given by
	\eq{
		\Del S(x) = -\frac{\Del E - T \,\Del S + P \,\Del V}{T},
	}
	where $T$ and $P$ are the equilibrium values~\cite[pp.~60, 349]{Landau}.  Assuming small fluctuations and thus small $\Del E$, we can expand $\Del E$ as
	\al{
		\Del E &= \pdv{E}{S} \Del S + \pdv{E}{V} \Del V + \frac{1}{2} \brac{ \pdv[2]{E}{S} (\Del S)^2 + 2 \pdv{E}{S}{V} \Del S \,\Del V + \pdv{E}{V} (\Del V)^2 } \\
		&= T \,\Del S - P \,\Del V + \frac{1}{2} \brac{ \paren{ \Del\pdv{E}{S} }_V \,\Del S + \paren{ \Del\pdv{E}{V} }_S \,\Del V }
		= T \,\Del S - P \,\Del V + \frac{\Del S \,\Del T - \Del P \,\Del V}{2},
	}
	where we have used $\pdv*{E}{S} = T$ and $\pdv*{E}{V} = -P$~\cite[pp.~60, 349]{Landau}.  Then the fluctuation probability has the proportionality
	\eq{
		p \propto e^{\Del S(x)}
		= \exp(\frac{\Del P \,\Del V - \Del S \,\Del T}{2 T}).
	}
	Expanding $\Del S$ and $\Del P$ in terms of $V$ and $T$, we find
	\al{
		\Del P &= \paren{ \pdv{P}{T} }_V \,\Del T + \paren{ \pdv{P}{V} }_t \,\Del V, &
		\Del S &= \paren{ \pdv{S}{T} }_V \,\Del T + \paren{ \pdv{S}{V} }_T \,\Del V
		= \frac{\Cv}{T} \,\Del T + \paren{ \pdv{P}{T} }_V \,\Del V,
	}
	where we have used $(\pdv*{S}{V})_T = (\pdv*{P}{T})_V$ and $\Cv = T (\pdv*{S}{T})_V$~\cite[pp.~45, 50, 349]{Landau}.  Making these substitutions,
	\aln{
		p &\propto \exp\!\curly{ \frac{1}{2 T} \brac{ \paren{ \pdv{P}{T} }_V \,\Del T \,\Del V + \paren{ \pdv{P}{V} }_t (\Del V)^2 - \pdv{\Cv}{T} (\Del T)^2 - \paren{ \pdv{P}{T} }_V \,\Del V \,\Del T } } \notag \\
		&= \exp[ \paren{ \frac{1}{2T} \pdv{P}{V} }_t (\Del V)^2 - \frac{\Cv}{2 T^2} (\Del T) ]
		= \exp[ \paren{ \frac{1}{2T} \pdv{P}{V} }_t (\Del V)^2 ] \exp[ -\frac{\Cv}{2 T^2} (\Del T) ]. \label{thing}
	}
	Thus, the expression is separable and fluctuations in $V$ and in $T$ can be regarded as independent~\cite[p.~349]{Landau}.
	
	We will focus on fluctuations in volume, and assume their probability to be Gaussian distributed.  The Gaussian distribution is given by~\cite[p.~345]{Landau}
	\eq{
		p(x) \ddx = \frac{1}{\sqrt{2\pi \ev*{x^2}}} \exp(-\frac{x^2}{2 \ev*{x^2}}) \ddx.
	}
	Comparing Eq.~\refeq{thing}, we find that~\cite[p.~350]{Landau}
	\eq{
		\ev*{(\Del V)^2} = -T \paren{ \pdv{V}{P} }_T.
	}
	Dividing both sides by $N^2$~\cite[p.~351]{Landau},
	\eq{
		\ev{[ \Del(V / N) ]^2} = -\frac{T}{N^2} \paren{ \pdv{V}{P} }_T.
	}
	Now we fix $V$ and consider fluctuations in $N$.  Note that
	\eq{
		\Del(V / N) = V \,\Del(1 / N)
		= -\frac{V}{N^2} \,\Del N,
	}
	so we have
	\eq{
		\evDNs = -\frac{T N^2}{V^2} \paren{ \pdv{V}{P} }_T.
	}
	Since $N = V \,f(P, T)$, we can write
	\eq{
		 -\frac{N^2}{V^2} \paren{ \pdv{V}{P} }_T = N \brac{ \pdv{P}(\frac{N}{V}) }_{T, N}
		 = N \brac{ \pdv{P}(\frac{N}{V}) }_{T, v}
		 = \frac{N}{V} \paren{ \pdv{N}{P} }_{T, v}
		 = \paren{ \pdv{P}{\mu} }_{T, V} \paren{\pdv{N}{P} }_{T, V}
		 = \paren{ \pdv{N}{\mu} }_{T, V},
	}
	where we have used $N / V = (\pdv*{P}{\mu})_T$~\cite[pp.~351--352]{Landau}.  Since we associated all quantities with those at equilibrium, we have shown that
	\eqn{dvmu}{
		\ans{ \evDNs = T \pdv{\!\evN}{\mu} }
	}
	as desired. \qed
	
	For a classical Boltzmann gas, the number of particles in a interval $\ddcp$ is~\cite[pp.~108--109]{Landau}
	\eq{
		\ddNp = \frac{V}{(2\pi m T)^{3/2}} \exp(\frac{\mu}{T} - \frac{\vp^2}{2 m T}) \ddcp,
	}
	so the total number of particles is
	\eq{
		N = \frac{V}{(2\pi m T)^{3/2}} \int \exp(\frac{\mu}{T} - \frac{\vp^2}{2 m T}) \ddcp.
	}
	To apply Eq.~\refeq{dvmu}, note that
	\al{
		T \pdv{\!\evN}{\mu} &= T \pdv{\mu}(\frac{V}{(2\pi m T)^{3/2}} \int \exp(\frac{\mu}{T} - \frac{\vp^2}{2 m T}) \ddcp)
		= T \frac{V}{(2\pi m T)^{3/2}} \int \dv{T}(e^{\mu / T} e^{-\vp^2 / (2 m T)} \ddcp) \\
		&= \frac{T}{T} \pdv{\mu}(\frac{V}{(2\pi m T)^{3/2}} \int \exp(\frac{\mu}{T} - \frac{\vp^2}{2 m T}) \ddcp)
		= N.
	}
	Thus, for the Boltzmann gas,
	\eq{
		\ans{ \evDNs = \ev{N}\!. }
	}
	
	For the Fermi and Bose gases, the number of particles is given by
	\eq{
		N = \frac{g V}{\pi^2 \hbar^2} \sqrt{\frac{m^3 T^3}{2}} \intoi \frac{\sqrt{z}}{e^{z - \mu / T} \pm 1} \ddz
		\begin{cases}
			\text{Fermi}, \\
			\text{Bose},
		\end{cases}
	}
	where $z = \eps / T$~\cite[pp.~149, 354]{Landau}.  Evaluating the integrals using
	\eq{
		\intoi \frac{k^s}{e^{k - \mu} \pm 1} \dd{k} = \mp \Gam(s + 1) \,\Li_{1 + s}(\mp e^{\mu}),
	}
	where $\Li$ is the polylogarithm~\cite{Polylog}, we have
	\eq{
		N = \mp \frac{g V}{\pi^2 \hbar^2} \sqrt{\frac{m^3 T^3}{2}} \Gam(3/2) \,\Li_{3/2}(\mp e^{\mu / T})
		= \mp \frac{g V}{\pi^2 \hbar^2} \paren{ \frac{m T}{2} }^{3/2} \Li_{3/2}(\mp e^{\mu / T}).
	}
	Using the formula $\dv*{\Li_n(x)}{x} = \Li_{n - 1}(x) / x$~\cite{Polylog}, we find
	\eq{
		T \pdv{\mu}[ \Li_{3/2}(\mp e^{\mu / T}) ] = \mp T \pdv{\mu}(\mp e^{\mu / T}) \frac{\Li_{1/2}(\mp e^{\mu / T})}{e^{\mu / T}}
		= T \frac{\Li_{1/2}(\mp e^{\mu / T})}{T}
		= \Li_{1/2}(\mp e^{\mu / T}).
	}
	So the fluctuations are
	\al{
		\evDNs &= \ans{ \mp \frac{g V}{\pi^2 \hbar^2} \paren{ \frac{m T}{2} }^{3/2} \,\Li_{1/2}(\mp e^{\mu / T}) } \\
		&= \mp \frac{g V}{\pi^2 \hbar^2} \paren{ \frac{m T}{2} }^{3/2} \frac{1}{\mp \Gam(1/2)} \intoi \frac{\ddz}{\sqrt{z} (e^{z - \mu / T} \pm 1)}
		= \ans{ \frac{g V T}{\hbar^2} \sqrt{\frac{m^3}{2^3 \pi^5}} \intoi \frac{\ddeps}{\sqrt{\eps} (e^{(\eps - \mu) / T} \pm 1)}
		\begin{cases}
			\text{Fermi}, \\
			\text{Bose}.
		\end{cases} }
	}
	\vfix
}