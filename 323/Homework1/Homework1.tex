\documentclass[11pt]{article}
\usepackage{geometry, titlesec}
\usepackage[parfill]{parskip}
\usepackage[italicdiff]{physics}
\usepackage{amsfonts, amsthm, mathrsfs}
\usepackage[cm]{fullpage}
\usepackage{fancyhdr}
\usepackage{enumitem}
\usepackage{xcolor, soul}
\usepackage{siunitx, graphicx}
%\allowdisplaybreaks

\renewcommand{\thesubsection}{\thesection.\alph{subsection}}
\newcommand{\vfix}{\vspace{-\baselineskip}}
\newcommand{\qimplies}{\quad\implies\quad}

\makeatletter
\renewcommand*\env@cases[1][1.2]{%
  \let\@ifnextchar\new@ifnextchar
  \left\lbrace
  \def\arraystretch{#1}%
  \array{@{}l@{\quad}l@{}}%
}
\makeatother
 
\renewcommand{\footrulewidth}{.2pt}
%\setlist[enumerate]{leftmargin=*}
\pagestyle{fancy}
\fancyhf{}
\lhead{\textbf{Physics 323 Homework 1}}
\rhead{Lacey Rainbolt}
\setlength{\headheight}{11pt}
\setlength{\headsep}{11pt}
\setlength{\footskip}{24pt}
\lfoot{\today}
\rfoot{\thepage}

\titleformat{\section}[runin]{\normalfont\large\bfseries}{Problem \thesection}{1em}{}
\titleformat{\subsection}[runin]{\normalfont\large\bfseries}{\thesubsection}{1em}{}
\titleformat{\subparagraph}[leftmargin]{\normalfont\normalsize\bfseries}{}{0pt}{}

\newcommand{\refeq}[1]{(\ref{#1})}

\newcommand{\beq}{\begin{equation*}}
\newcommand{\eeq}{\end{equation*}}

\newcommand{\beqn}{\begin{equation}}
\newcommand{\eeqn}{\end{equation}}

\newcommand{\blg}{\begin{align*}}
\newcommand{\elg}{\end{align*}}


\newenvironment{statement}[1]
{
	\section{#1.}
	\color{darkgray}
	\ignorespaces
}

\newenvironment{problem}
{
	\subsection{}
	\color{darkgray}
	\ignorespaces
}

\newenvironment{solution}
{
	\paragraph{Solution.}
	\ignorespaces
}
{
    \bigskip
}

\DeclareSIUnit{\MeV}{\mega\electronvolt}
\DeclareSIUnit{\GeV}{\giga\electronvolt}

\begin{document}

\newcommand{\alp}{\alpha}
\newcommand{\bet}{\beta}
\newcommand{\gam}{\gamma}
\newcommand{\del}{\delta}
\newcommand{\lam}{\lambda}
\newcommand{\tht}{\theta}

\newcommand{\Del}{\Delta}
\newcommand{\Lam}{\Lambda}
\newcommand{\Tht}{\Theta}

\newcommand{\etamn}{\eta^{\mu \nu}}


\state{(Jackson 9.8)}{\ 
	%\emph{Hint:} The electromagnetic angular momentum density comes from more than the transverse (radiation zone) components of the fields.
}

%
%	Jackson 9.8(a)
%

\prob{}{
	Show that a classical oscillating electric dipole $\vp$ with fields given by
	\aln{ \label{fields1}
		\vH &= \frac{c k^2}{4\pi} (\nh \cross \vp) \frac{e^{i k r}}{r} \paren{ 1 - \frac{1}{i k r} }, &
		\vE &= \frac{1}{4\pi \epso} \curly{ k^2 (\nh \cross \vp) \cross \nh \frac{e^{i k r}}{r} + [ 3 \nh (\nh \vdot \vp) - \vp ] \paren{ \frac{1}{r^3} - \frac{i k}{r^2} } e^{i k r} },
	}
	radiates electromagnetic angular momentum to infinity at the rate
	\eq{
		\dv{\vL}{t} = \frac{k^3}{12 \pi \epso} \Im[ \vp^* \cross \vp ].
	}
	\vfix
}

\sol{
	According to Jackson~(9.20), the time-averaged angular momentum density is
	\eq{
		\vl = \frac{\Re[ \vx \cross (\vE \cross \vHs)}{2 c^2}.
	}
	One of the vector identities on the inside cover of Jackson is $\vaa \cross (\vbb \cross \vcc) = (\vaa \vdot \vcc) \vbb - (\vaa \vdot \vbb) \vcc$, so
	\eqn{l1}{
		\vl = \frac{(\vx \vdot \vHs) \vE - (\vx \vdot \vE) \vHs}{2 c^2}.
	}
	From Eq.~\refeq{fields1}, note that
	\eq{
		\vx \vdot \vHs \propto \vx \vdot (\nh \cross \vps)
		= \vps \vdot (\vx \cross \nh)
		= \vO,
	}
	where we have used the identity $\vaa \vdot (\vbb \cross \vcc) = \vcc \vdot (\vaa \cross \vbb)$ and the fact that $\nh$ points in the $\vx$ direction.  For $\vx \vdot \vE$, note that
	\al{
		\vx \vdot [ (\nh \cross \vp) \cross \nh ] &= -\vx \vdot [ \nh \cross (\nh \cross \vp) ]
		= -\vx \vdot [ (\nh \vdot \vp) \nh - (\nh \vdot \nh) \vp ]
		= -(\nh \vdot \vp) (\vx \vdot \nh) + \vx \vdot \vp \\
		&= -r (\nh \vdot \vp) + \vx \vdot \vp
		= \vx \vdot \vp - \vx \vdot \vp
		= 0, \\[1.5ex]
		\vx \vdot [ 3 \nh (\nh \vdot \vp) - \vp ] &= 3 (\vx \vdot \nh) (\nh \vdot \vp) - \vx \vdot \vp
		= 3r (\nh \vdot \vp) - \vx \vdot \vp
		= 3(\vx \vdot \vp) - \vx \vdot \vp
		= 2(\vx \vdot \vp),
	}
	since $\abs{\vx} = r$ and $\vx = r \,\nh$.  Then
	\eq{
		\vx \vdot \vE = \frac{1}{2\pi \epso} (\vx \vdot \vp) \paren{ \frac{1}{r^3} - \frac{i k}{r^2} } e^{i k r}
		= \frac{1}{2\pi \epso} (\nh \vdot \vp) \paren{ \frac{1}{r^2} - \frac{i k}{r} } e^{i k r}.
	}
	
	With these substitutions, Eq.~\refeq{l1} becomes
	\al{
		\vl &= -\frac{(\vx \vdot \vE) \vHs}{c^2}
		= -\frac{1}{4\pi \epso c^2} (\nh \vdot \vp) \paren{ \frac{1}{r^2} - \frac{i k}{r} } e^{i k r} \frac{c k^2}{4\pi} (\nh \cross \vps) \frac{e^{-i k r}}{r} \paren{ 1 + \frac{1}{i k r} } \\
		&= -\frac{k^2}{16\pi^2 \epso c r} (\nh \vdot \vp) (\nh \cross \vps) \paren{ \frac{1}{r^2} - \frac{i k}{r} } \paren{ 1 - \frac{i}{k r} }
		= -\frac{k^2}{16\pi^2 \epso c} (\nh \vdot \vp) (\nh \cross \vps) \paren{ \frac{1}{r^2} - \frac{i}{k r^3} - \frac{i k}{r} - \frac{1}{r^2} } \\
		&= -\frac{i k^2}{16\pi^2 \epso c r} (\nh \vdot \vp) (\nh \cross \vps) \paren{ \frac{1}{k r^3} + \frac{k}{r^2} }
		= \frac{i k^3}{16\pi^2 \epso c r^2} (\nh \vdot \vp) (\nh \cross \vps) \paren{ \frac{1}{k^2 r^2} + 1 }.
	}
	
	Let $\vL$ be the angular momentum radiated to a distance $R$.  Then
	\eq{
		\vL = \int_R \vl(r) \ddcx
		= \intopi \intotp \intoR \vl(r) \,r^2 \sin\tht \ddr \ddphi \dd\tht,
	}
	and the time derivative is
	\aln{
		\dv{\vL}{t} &= \dv{t}(\intopi \intotp \intoR \vl(r) \,r^2 \sin\tht \ddr \ddphi \dd\tht)
		= \dv{r}{t} \dv{r}(\intopi \intotp \intoR \vl(r) \,r^2 \sin\tht \ddr \ddphi \dd\tht) \notag \\
		&= c \intopi \intotp \vl(r) \,r^2 \sin\tht \ddphi \dd\tht
		= \frac{i k^3}{16\pi^2 \epso} \paren{ \frac{1}{k^2 r^2} + 1 } \intopi \intotp (\nh \vdot \vp) (\nh \cross \vps) \sin\tht \ddphi \dd\tht. \label{dLdt}
	}
	Note that
	\eq{
		[ (\nh \vdot \vp) (\nh \cross \vps) ]_i = \sumje n_j p_j (\nh \cross \vps)_i
		= \sumje \sumke \sumle \epsikl n_j p_j n_k p_l^*,
	}
	so
	\eq{
		\dv{L_i}{t} \propto \sumje \sumke \sumle \epsikl p_j p_l^* \int n_j p_k \ddOmg
		= \sumje \sumke \sumle \epsikl p_j p_l^* \frac{4\pi}{3} \del_{jk}
		= \frac{4\pi}{3} \epsikl p_k p_l^*
		= \frac{4\pi}{3} (\vp \cross \vps)_i,
	}
	where we have used Jackson~(9.47), $\int n_\bet n_\gam \ddOmg = 4\pi \del_{\bet \gam} / 3$.  Making this substitution into Eq.~\refeq{dLdt},
	\eq{
		\dv{\vL}{t} = \frac{i k^3}{6\pi \epso} \paren{ \frac{1}{k^2 r^2} + 1 } (\vp \cross \vps).
	}
	Taking the limit as $r \to \infty$, we find
	\eqn{ans1a}{
		\dv{\vL}{t} = \Re\!\brac{ \frac{i k^3}{12\pi \epso} (\vp \cross \vps) }
		= \Re\!\brac{ -\frac{i k^3}{12\pi \epso} (\vps \cross \vp) }
		= \ans{ \frac{k^3}{12\pi \epso} \Im[ \vps \cross \vp ], }
	}
	as desired. \qed
}

%
%	Jackson 9.8(b)
%

\prob{}{
	What is the ratio of angular momentum radiated to energy radiated?  Interpret.
}

\sol{
	According to Jackson~(9.24), the total power radiated by an oscillating electric dipole $\vp$ is
	\eq{
		P = \dv{E}{t}
		= \frac{c^2 \Zo k^4}{12 \pi} \abs{\vp}^2.
	}
	Then the ratio of angular momentum radiated to energy radiated is
	\eq{
		\frac{\dv*{\vL}{t}}{\dv*{E}{t}} = \frac{k^3}{12\pi \epso} \Im[ \vps \cross \vp ] \frac{12 \pi}{c^2 \Zo k^4 \abs{\vp}^2}
		= \frac{1}{\epso} \Im[ \vps \cross \vp ] \frac{1}{c^2 \Zo k \abs{\vp}^2}
		= \ans{ \frac{\Im[ \vps \cross \vp ]}{\omg \abs{\vp}^2}, }
	}
	where we have used $\Zo = \sqrt{\muo / \epso} = 1 / \sqrt{\epso^2 c^2} = 1 / \epso c$, $c^2 = 1 / (\epso \muo)$, and $\omg = k c$.
	
	In the limit of high frequency, $(\dv*{\vL}{t}) / (\dv*{E}{t}) \to 0$.  In this scenario, the energy radiated dominates over the angular momentum radiated.  Likewise, in the limit of low frequency, $(\dv*{\vL}{t}) / (\dv*{E}{t}) \to \infty$, meaning that angular momentum radiation dominates.  This is sensible because rotational kinetic energy $E \propto \omg^2$, while angular momentum $L \propto \omg$.
}

%
%	Jackson 9.8(c)
%

\prob{}{
	For a charge $e$ rotating in the $xy$ plane at radius $a$ and angular speed $\omg$, show that there is only a $z$ component of radiated angular momentum with magnitude $\dv*{\Lz}{t} = e^2 k^3 a^2 / 6 \pi \epso$.  What about a charge oscillating along the $z$ axis?
}

\sol{
	We know from Homework~5 that the position of a point charge rotating counterclockwise in the $xy$ plane is
	\eq{
		\vx(t) = a \cos(\omg t) \,\vx + a \sin(\omg t) \,\yh.
	}
	\clearpage
	Then the charge distribution is
	\eq{
		\rho(\vx, t) = e \del[ x - a \cos(\omg t) ] \,\del[ y - a \sin(\omg t) ] \,\del(z).
	}
	
	According to Jackson~(4.8), the dipole moment is defined
	\eq{
		\vp = \int \vx' \,\rho(\vx') \ddcxp.
	}
	The components of $\vp$ for the point charge are then
	\al{
		\px &= e \iiint x \,\del[ x - a \cos(\omg t) ] \,\del[ y - a \sin(\omg t) ] \,\del(z) \ddx \ddy \ddz
		= e a \cos(\omg t), \\
		\py &= e \iiint y \,\del[ x - a \cos(\omg t) ] \,\del[ y - a \sin(\omg t) ] \,\del(z) \ddx \ddy \ddz
		= e a \sin(\omg t), \\
		\pz &= e \iiint z \,\del[ x - a \cos(\omg t) ] \,\del[ y - a \sin(\omg t) ] \,\del(z) \ddx \ddy \ddz
		= 0,
	}
	so we can write $\vp = e a \,e^{-i \omg t} (\xh + i\,\yh).$  Substituting into Eq.~\refeq{ans1a},
	\al{
		\dv{\vL}{t} &= \Re\!\brac{ \frac{i k^3}{12\pi \epso} e^2 a^2 e^{-i \omg t} e^{i \omg t} [ (\xh + i\,\yh) \cross (\xh - i\,\yh) ] }
		= \Re\!\brac{ \frac{i e^2 k^3 a^2}{12\pi \epso} (-2i \,\xh \cross \yh) }
		= \Re\!\brac{ \frac{e^2 k^3 a^2}{6\pi \epso} \,\zh } \\
		&= \ans{ \frac{e^2 k^3 a^2}{6\pi \epso} \cos(\omg t) \,\zh, }
	}
	as desired. \qed
	
	A charge oscillating along the $z$ axis with amplitude $a$ has the charge density
	\eq{
		\rho(\vx, t) = e a \,\del(x) \,\del(y) \,\del[ z - \cos(\omg t) ],
	}
	which gives the dipole moment
	\al{
		\px &= e a \iiint x \,\del(x) \,\del(y) \,\del[ z - \cos(\omg t) ] \ddx \ddy \ddz
		= 0, \\
		\py &= e a \iiint y \,\del(x) \,\del(y) \,\del[ z - \cos(\omg t) ] \ddx \ddy \ddz
		= 0, \\
		\pz &= e a \iiint z \,\del(x) \,\del(y) \,\del[ z - \cos(\omg t) ] \ddx \ddy \ddz
		= e a \cos(\omg t).
	}
	In complex notation, $\vp = e a \,e^{-i\omg t} \,\zh$.  Substituting into Eq.~\refeq{ans1a}, we find
	\eq{
		\dv{\vL}{t} = \Re\!\brac{ \frac{i k^3}{12\pi \epso} e^2 a^2 e^{-i \omg t} e^{i \omg t} (\zh \cross \zh) }
		= \ans{ \vO. }
	}
	So we see that a charge undergoing linear motion does not lead to a radiated angular momentum, which is sensible.
}

%
%	Jackson 9.8(d)
%

\prob{}{
	What are the results corresponding to Probs.~{1(a)} and {1(b)} for magnetic dipole radiation?
}

\sol{
	The radiation fields for a magnetic dipole are given by Jackson~(19.35--36),
	\al{
		\vH &= \frac{1}{4\pi} \curly{ k^2 (\nh \cross \vm) \cross \nh \frac{e^{i k r}}{r} + [ 3 \nh (\nh \vdot \vm) - \vm ] \paren{ \frac{1}{r^3} - \frac{i k}{r^2} } e^{i k r} }, &
		\vE &= -\frac{\Zo}{4\pi} k^2 (\nh \cross \vm) \frac{e^{i k r}}{r} \paren{ 1 - \frac{1}{i k r} }.
	}
	\clearpage
	Comparing with Eq.~\refeq{fields1}, we see that $\vH \to -\vE / \Zo$, $\vE \to \Zo \vH$, and $\vp \to \vm / c$ as stated in the book~\cite[p.~413]{Jackson}.  Making these substitutions, the results of Probs.~{1.1(a)} and {(b)} become
	\al{
		\ans{ \dv{\vL}{t}\ }&\ans{= \frac{\muo k^3}{12\pi} \Im[ \vms \cross \vm ], } &
		\ans{ \frac{\dv*{\vL}{t}}{\dv*{E}{t}}\ }&\ans{= \frac{\Im[ \vms \cross \vm ]}{\omg \abs{\vm}^2} }
	}
	where we have used $\mu = 1 / \epso c^2$.
}

\state{Beta function of the Gross-Neveu model~(P\&S~12.2)}{
	Compute $\bet(g)$ in the two-dimensional Gross-Neveu model studied in Problem~11.3,
	\eq{
		\cL = \psibsi i \ptsl \psisi + \frac{1}{2} g^2 (\psibsi \psisi)^2,
	}
	with $i = 1, \ldots, N$.  You should find that this model is asymptotically free.  How was that fact reflected in the solution to Problem~11.3?
}

\sol{
	We saw in Problem~2 of Homework~4 that this Lagrangian can be written as
	\eq{
		\cL = \psibsi i \ptsl \psisi - \sig \psibsi \psisi - \frac{1}{2 g^2} \sig^2,
	}
	where $\sig$ is a new scalar field with no kinetic energy terms.  In the modified minimal subtraction scheme, we found the effective potential was
	\eqn{Veff}{
		\Veff = \sig^2 \curly{ \frac{1}{2 g^2} + \frac{N}{4\pi} \brac{ \ln(\frac{\sig^2}{M^2}) - 1 } }.
	}
	Since $\Gam[ \phicl ] = -(V T) \Veff(\phi)$ by P\&S~(11.50), we have
	\eqn{Gam}{
		\Gam[ \sigcl ] = -(V T)  \sig^2 \curly{ \frac{1}{2 g^2} + \frac{N}{4\pi} \brac{ \ln(\frac{\sig^2}{M^2}) - 1 } }.
	}
	Referring to p.~3 of Lecture~11, we can apply the Callan-Symanzik equation to $\Gam$.   The Callan-Symanzik equation is P\&S~(12.41),
	\eq{
		\brac{ M \pdv{M} + \bet(\lam) \pdv{\lam} + n \gam(\lam) } G^{(n)}(\{ x_i \}; M, \lam) = 0.
	}
	For our problem, $\gam$ is 0 because there are no field insertions.  That is, we have
	\eq{
		\brac{ M \pdv{M} + \bet(g) \pdv{g} } \Gam[ \phicl ] = 0.
	}
	Using Eq.~\refeq{Gam}, note that
	\al{
		\pdv{\Gam}{M} &= (V T) \frac{N \sig^2}{2 \pi M}, &
		\pdv{\Gam}{g} &= (V T) \frac{\sig^2}{g^3}.
	}
	Then
	\eq{
		0 = (V T) \paren{ \frac{N \sig^2}{2 \pi} + \bet(g) \frac{\sig^2}{g^3} }
		\qimplies
		\ans{ \betg = -\frac{N g^3}{2\pi}. }
	}
	This model is asymptotically free because the $\bet$ function is proportional to $-g^3$~\cite[pp.~424--425]{Peskin}.
	
	In 2(e) of Homework~4, we found that the vacuum expectation value of $\sig$ was
	\eq{
		\sig = \pm M e^{-\pi / N g^2} = \pm v.
	}
	We showed that the vacuum expectation value does not depend on the renormalization condition chosen.  This means that we can increase $M \to 0$ while holding $\sig$ constant, and see that $g \to 0$ logarithmically.  This is indicative of an asymptotically-free theory~\cite[p.~425]{Peskin}. \qed
}





\newcommand{\vr}{\vb{r}}
\newcommand{\vrperp}{\vr_\perp}

\begin{statement}{(Jackson 11.18)}
	The electric and magnetic fields of a particle of charge $q$ moving in a straight line with speed $v = \bet c$, given by \refeq{fields}, become more and more concentrated as $\bet \to 1$.  Choose axes so that the charge moves along the $z$ axis in the positive direction, passing the origin at $t = 0$.  Let the spatial coordinates of the observation point be $(x, y, z)$ and define the transverse vector $\vrperp$, with components $x$ and $y$.  Consider the fields and the source in the limit of $\bet = 1$.
\end{statement}

\newcommand{\rperp}{r_\perp}
\newcommand{\vE}{\vb{E}}
\newcommand{\vB}{\vb{B}}
\newcommand{\vh}{\vb{\hat{v}}}
\newcommand{\delctz}{\del(ct - z)}

\begin{problem} \label{3.a}
	Show that the fields can be written as
	\begin{align*}
		\vE &= 2q \frac{\vrperp}{\rperp^2} \delctz, &
		\vB &= 2q \frac{\vh \cross \vrperp}{\rperp^2} \delctz,
	\end{align*}
	where $\vh$ is a unit vector in the direction of the particle's velocity.
\end{problem}

\newcommand{\nh}{\vb{\hat{n}}}
\newcommand{\zh}{\vb{\hat{z}}}
\newcommand{\diffr}{\abs{\vr - \vr'}}
\newcommand{\limbet}{\lim_{\bet \to 1}}
\newcommand{\tif}{\text{if }}

\begin{solution}
	According to Jackson~(11.154), the electric field can be written
	\beq
		\vE = \frac{q \vr}{r^3 \gam^2 (1 - \bet^2 \sin^2\psi)^{3/2}}.
	\eeq
	\clearpage
	Here $\psi = \cos[-1](\nh \vdot \vh)$ is shown in Fig.~\ref{11.8}, and $\nh$ is a unit vector pointing from the current position of the charge to the observation point $\vr$.  Note that
	\beq
		\limbet \vE \sim \limbet \frac{1}{\gam^2 (1 - \bet^2 \sin^2\psi)^{3/2}}
		= \limbet \frac{1 - \bet^2}{(1 - \bet^2 \sin^2\psi)^{3/2}}
		= \begin{cases}
			\infty & \tif \sin\psi = 1, \\
			0 & \tif \sin\psi \neq 1.
		\end{cases}
	\eeq
	Obviously $\sin\psi = 1$ if and only if $\cos\psi = 0$.
	
	Let $\vr'(t)$ be the position of the charge, so $\nh = (\vr - \vr') / \diffr$.  We know $\vh = \zh$, and $\vr' = \vv t \to ct \,\zh$ as $\bet \to 1$.  Then
	\beq
		\cos\psi = \nh \vdot \vh
		= \frac{(\vr - \vr') \vdot \zh}{\diffr}
		= \frac{z - ct}{\diffr},
	\eeq
	so $\cos\psi = 0$ if and only if $z = ct$.  Thus, we have shown that
	\beq
		\limbet \frac{1}{\gam^2 (1 - \bet^2 \sin^2\psi)^{3/2}} = \delctz.
	\eeq
	
	Now we have
	\beq
		\limbet \vE = \frac{q \vr}{r^3} \delctz
	\eeq
\end{solution}

\newcommand{\Ja}{J^\alp}
\newcommand{\Jb}{J^\bet}
\newcommand{\dela}{\partial_\alp}
\newcommand{\valp}{v^\alp}

\newcommand{\rperpq}{{\rperp}_1}
\newcommand{\rperpw}{{\rperp}_2}

\newcommand{\delt}{\partial_t}
\newcommand{\delx}{\partial_x}
\newcommand{\dely}{\partial_y}
\newcommand{\delz}{\partial_z}

\begin{problem}
	Show that by substitution into the Maxwell equations that these fields are consistent with a 4-vector source density,
	\beq
		\Ja = q c \valp \del^2(\vrperp) \delctz,
	\eeq
	where the 4-vector $\valp = (1, \vh)$.
\end{problem}

\begin{solution}
	According to Jackson~(11.141), the inhomogeneous Maxwell equations can be written
	\beq
		\dela \Fab = \frac{4\pi}{c} \Jb.
	\eeq
	Here,
	\beq
		\Fab %= \frac{2q}{{\rperp}^2} \delctz
%			\mqty[ 	0 & -\rperpq & -\rperpw & 0 \\
%					\rperpq & 0 & 0 & \rperpq \\
%					\rperpw & 0 & 0 & \rperpw \\
%					0 & -\rperpq & -\rperpw & 0 ]
		= \frac{2q}{{\rperp}^2} \delctz
			\mqty[ 	0 & -x & -y & 0 \\
					x & 0 & 0 & x \\
					y & 0 & 0 & y \\
					0 & -x & -y & 0 ]	
	\eeq
	and so
	\beq
		\dela \Fab = \mqty[ \delt & -\delx & -\dely & -\delz] \frac{2q}{{\rperp}^2} \delctz
			\mqty[ 	0 & -x & -y & 0 \\
					x & 0 & 0 & x \\
					y & 0 & 0 & y \\
					0 & -x & -y & 0 ]	
		= \mqty[ & & & ]
	\eeq
\end{solution}




\newcommand{\Ao}{A^0}
\newcommand{\Az}{A^z}
\newcommand{\vA}{\vb{A}}
\newcommand{\vAperp}{\vA_\perp}
\newcommand{\Thtctz}{\Tht(ct - z)}
\newcommand{\grperp}{\grad_\perp}

\begin{problem}
	Show that the fields of \refeq{3.a} are derivable from either of the following 4-vector potentials,
	\begin{align*}
		\Ao &= \Az = -2q \delctz \ln(\lam \rperp), &
		\vAperp &= 0,
	\end{align*}
	or
	\begin{align*}
		\Ao &= 0 = \Az, &
		\vAperp &= -2q \Thtctz \grperp \ln(\lam \rperp),
	\end{align*}
	where $\lam$ is an irrelevant parameter setting the scale of the logarithm.
	
	Show that the two potentials differ by a gauge transformation and find the gauge function, $\chi$.
\end{problem}




\clearpage

\newcommand{\mq}{m_1}
\newcommand{\mw}{m_2}
\newcommand{\pq}{p_1}
\newcommand{\pw}{p_2}

\newcommand{\mLam}{\SI{1115}{\MeV}}
\newcommand{\tauLam}{\SI{2.9e-10}{\second}}
\newcommand{\mnuc}{\SI{939}{\MeV}}
\newcommand{\mpi}{\SI{140}{\MeV}}

\begin{statement}{(Jackson 11.20)}
	The lambda particle ($\Lam$) is a neutral baryon of mass $M = \mLam$ that decays with a lifetime of $\tau = \tauLam$ into a nucleon of mass $\mq \approx \mnuc$ and a pi-meson of mass $\mw \approx \mpi$.  It was first observed in flight by its charged decay mode $\Lam \to p + \pi^-$ in cloud chambers.  The charged tracks originate from a single point and have the appearance of an inverted vee or lambda.  The particles' identities and momenta can be inferred from their ranges and curvature in the magnetic field of the chamber.
\end{statement}

\newcommand{\Pq}{P_1}
\newcommand{\Pw}{P_2}

\newcommand{\Pmu}{P^\mu}
\newcommand{\Psmu}{P_\mu}
\newcommand{\Pqmu}{{\Pq}^\mu}
\newcommand{\Pqsmu}{{\Pq}_\mu}
\newcommand{\Pwmu}{{\Pq}^\mu}
\newcommand{\Pwsmu}{{\Pq}_\mu}

\newcommand{\vp}{\vb{p}}
\newcommand{\vpq}{\vp_1}
\newcommand{\vpw}{\vp_2}

\begin{problem} \label{4.a}
	Using conservation of momentum and energy and the invariance of scalar products of 4-vectors show that, if the opening angle $\tht$ between the two tracks is measured, the mass of the decaying particle can be found from the formula
	\beq
		M^2 = \mq^2 + \mw^2 + 2 \Eq \Ew - 2 \pq \pw \cos\tht,
	\eeq
	where here $\pq$ and $\pw$ are the magnitudes of the 3-momenta.
\end{problem}

\begin{solution}
	The general momentum 4-vector for a particle is
	\beq
		\Pmu = (E / c, \vp),
	\eeq
	where $E$ is the energy of the particle and $\vp$ its three-dimensional momentum.  Let $\Pqmu$ and $\Pwmu$ be the momentum 4-vectors for the two particles, and define
	\beq
		\Pmu = \Pqmu + \Pwmu.
	\eeq
	Firstly, note that
	\beq
		\Pmu \Psmu = \frac{E^2}{c^2} - p^2.
	\eeq
	According to Jackson~(11.55),
	\beqn \label{E}
		E = \sqrt{c^2 p^2 + m^2 c^4},
	\eeqn
	so we have
	\beqn \label{thing4.1}
		\Pmu \Psmu = \frac{c^2 p^2 + M^2 c^4}{c^2} - p^2
		= M^2 c^2.
	\eeqn
	Note also that
	\beq
		\Pmu \Psmu = (\Pqmu + \Pwmu) (\Pqsmu + \Pwsmu)
		= \frac{(\Eq + \Ew)^2}{c^2} - (\vpq + \vpw)^2
		= \frac{\Eq^2 + 2 \Eq \Ew + \Ew^2}{c^2} - \pq^2 - 2 \vpq \vdot \vpw - \pw^2,
	\eeq
	and once again making use of \refeq{E},
	\beqn \label{thing4.2}
		\Pmu \Psmu = \frac{c^2 \pq^2 + \mq^2 c^4 + 2 \Eq \Ew + c^2 \pw^2 + \mw^2 c^4}{c^2} - \pq^2 - 2 \vpq \vdot \vpw - \pw^2
		= \mq^2 c^2 + \mw^2 c^2 + 2 \Eq \Ew / c^2 - 2 \pq \pw \cos\tht.
	\eeqn
	Equating \refeq{thing4.1} and \refeq{thing4.2}, we have
	\beq
		M^2 c^2 = \mq^2 c^2 + \mw^2 c^2 + 2 \Eq \Ew / c^2 - 2 \pq \pw \cos\tht.
	\eeq
	Taking $c = 1$, this becomes
	\beq
		M^2 = \mq^2 + \mw^2 + 2 \Eq \Ew - 2 \pq \pw \cos\tht
	\eeq
	as desired. \qed
\end{solution}



\newcommand{\ELam}{\SI{10}{\GeV}}

\begin{problem}
	A lambda particle is created with a total energy of {\ELam} in a collision in the top plate of a cloud chamber.  How far on the average will it travel in the chamber before decaying?  What range of opening angles will occur for a \SI{10}{\GeV} lambda if the decay is more or less isotropic in the lambda's rest frame?
\end{problem}

\newcommand{\Delt}{\Del t}
\newcommand{\lspeed}{\SI{3.00e8}{\meter\per\second}}

\newcommand{\vvq}{\vv_1}
\newcommand{\vvw}{\vv_2}

\newcommand{\Enucp}{\SI{944}{\MeV}}
\newcommand{\Epip}{\SI{171}{\MeV}}
\newcommand{\Enuc}{\SI{8422}{\MeV}}
\newcommand{\Epi}{\SI{1256}{\MeV}}

\newcommand{\gamnuc}{1.005}
\newcommand{\gampi}{1.221}
\newcommand{\gamLam}{8.969}

\newcommand{\pnucp}{\SI{98}{\MeV}}
\newcommand{\ppip}{\pnucp}
\newcommand{\pnuc}{\SI{8369}{\MeV}}
\newcommand{\ppi}{\SI{1248}{\MeV}}

%\newcommand{\Lammatt}{\mqty[	\gam & 0 & 0 & -\gam \bet \\
%							0 & 1 & 0 & 0 \\
%							0 & 0 & 1 & 0 \\
%							-\gam \bet & 0 & 0 & \gam ]}
%\newcommand{\pqq}{\pq^1}
%\newcommand{\pqw}{\pq^2}
%\newcommand{\pqe}{\pq^3}
%\newcommand{\pwq}{\pw^1}
%\newcommand{\pww}{\pw^2}
%\newcommand{\pwe}{\pw^3}
%
%\newcommand{\ph}{\vb{\hat{p}}}
%\newcommand{\pqh}{\ph_1}

\begin{solution}
	According to Jackson~(11.51),
	\beq
		E = \gam m c^2.
	\eeq
	For the $\Lam$, this gives us $\gam = E / M$ in natural units.  Substituting this into the expression for time dilation, we find
	\beq
		\Delt = \gam \tau
		= \frac{E}{M} \tau
	\eeq
	as the average lifetime in the lab frame.  We also find
	\beq
		\bet = \sqrt{1 - \frac{1}{\gam^2}}
		= \sqrt{1 - \frac{M^2}{E^2}}.
	\eeq
	Then the average distance traveled before decaying is
	\beq
		d = v \,\Delt
		= c \tau \sqrt{1 - \frac{M^2}{E^2}} \frac{E}{M}
		= (\lspeed) \sqrt{1 - \frac{(\mLam)^2}{(\ELam)^2}} \frac{\ELam}{\mLam} (\tauLam)
		= {\color{blue} \SI{77}{\cm}}.
	\eeq
	
	Let $K'$ denote the rest frame of the $\Lam$.  Using the result of \ref{4.a} and the fact that $M$, $\mq$, and $\mw$ are Lorentz invariant, we can write
	\beqn \label{thing4}
		\Eq' \Ew' - \pq' \pw' \cos\tht' = \Eq \Ew - \pq \pw \cos\tht,
	\eeqn
	where $\tht'$ is the angle between the daughter particles in $K'$.  We know from conservation of momentum that they must be ``back to back'' in this frame, meaning $\tht' = \pi$.  Taking this into account and rearranging, we find
	\beqn \label{thing4.2}
		\cos\tht = \frac{\Eq \Ew - \Eq' \Ew' - \pq' \pw'}{\pq \pw}.
	\eeqn
	Say that the $\Lam$ is moving in the $z$ direction in the lab frame $K$.  Then the opening angle $\tht$ will be maximized when the daughter particles are emitted transverse to the $z'$ axis in $K'$.
	
	Using natural units, in the $K'$ frame we have
	\begin{align*}
		{P'}^\mu &= (M, \vo), &
		{\Pq'}^\mu &= (\Eq', \vpq'), &
		{\Pw'}^\mu &= (\Ew', \vpw').
	\end{align*}
	Conservation of momentum and energy stipulates that
	\begin{align*}
		M &= \Eq' + \Eq'
		= \gamq' \mq + \gamw' \mw, &
		\vo &= \vpq' + \vpw'
		= \gamq' \vpq' + \gamw' \vpw'
		= \gamq' \mq \vvq' + \gamw' \mw \vvw',
	\end{align*}
	where $\gamq'$ and $\gamw'$ are associated with boosting to the rest frames of $\mq$ and $\mw$, respectively, from $K'$.  Note that
	\beq
		\gamq' \mq \vvq' = -\gamw' \mw \vvw'
		\qimplies
		{\gamq'}^2 \mq^2 {\vvq'}^2 = {\gamw'}^2 \mw^2 {\vvw'}^2,
	\eeq
	and that, in natural units,
	\beq
		\gam^2 v^2 = \gam^2 \bet^2
		= \gam^2 \left( 1 - \frac{1}{\gam^2} \right)
		= \gam^2 - 1.
	\eeq
	Then
	\beq
		({\gamq'}^2 - 1) \mq^2 = ({\gamw'}^2 - 1) \mw^2
		\qimplies
		{\Eq'}^2 - \mq^2 = {\Ew'}^2 - \mw^2
		\qimplies
		\mq^2 - \mw^2 = {\Eq'}^2 - {\Ew'}^2,
	\eeq
	which implies
	\beq
		\mq^2 - \mw^2 = {\Eq'}^2 - (M - \Eq')^2 = -M^2 - 2 M \Eq'
		\qimplies
		\Eq' = \frac{M^2 + \mq^2 - \mw^2}{2 M},
	\eeq
	and similarly for $\Ew$.  Let $\mq$ denote the nucleon and $\mw$ the pion.  Then we have
	\begin{align*}
		\Eq' &= \frac{1}{2} \frac{(\mLam)^2 + (\mnuc)^2 - (\mpi)^2}{\mLam} 
		\approx \Enucp, \\
		\Ew' &= \frac{1}{2} \frac{(\mLam)^2 - (\mnuc)^2 + (\mpi)^2}{\mLam} 
		\approx \Epip.
	\end{align*}
	For the momenta, note that
	\beq
		\pq' = \mq \sqrt{{\gamq'}^2 - 1}
		= \mw \sqrt{{\gamw'}^2 - 1}
		= \pw',
	\eeq
	where
	\begin{align*}
		\gamq' &= \frac{\Eq'}{\mq}
		= \frac{\Enucp}{\mnuc}
		\approx \gamnuc, &
		\gamw' &= \frac{\Ew'}{\mw}
		= \frac{\Epip}{\mpi}
		\approx \gampi.
	\end{align*}
	Then
	\beq
		\pq' = \pw' = (\mpi) \sqrt{{\gampi}^2 - 1}
		\approx \pnucp.
	\eeq
	
	For the left side of \refeq{thing4}, note that
	\beq
		\gam = \frac{E}{M}
		= \frac{\ELam}{\mLam}
		\approx \gamLam
		\qimplies
		\bet = \sqrt{1 - \frac{1}{\gam^2}}
		\approx 0.994,
	\eeq
	whereas
	\begin{align*}
		\betq' &= \sqrt{1 - \frac{1}{{\gamq'}^2}}
		\approx 0.104, &
		\betw' &= \sqrt{1 - \frac{1}{{\gamw'}^2}}
		\approx 0.573. &
	\end{align*}
	Since these are not nearly as relativistic as $\bet$, and we have stipulated that $\vbet'_1$ and $\vbet'_1$ are transverse to the $z'$ axis, it may be sufficient to approximate $\gamq, \gamw \approx \gam$.  Then we have
	\begin{align*}
		\Eq &\approx \gam \mq
		= \gamLam (\mnuc)
		\approx \Enuc, &
		\Ew &\approx \gam \mw
		= \gamLam (\mpi)
		\approx \Epi, \\
		\pq &\approx \mq \sqrt{\gam^2 - 1}
		= (\mnuc) \sqrt{\gamLam^2 - 1}
		\approx \pnuc, &
		\pw &\approx \mw \sqrt{\gam^2 - 1}
%		= (\mpi) \sqrt{\gamLam^2 - 1}
		\approx \ppi.
	\end{align*}
	
	Making these substitutions into \refeq{thing4.2}, we have
	\beq
		\cos\tht = \frac{(\Enuc) (\Epi) - (\Enucp) (\Epip) - (\pnucp)^2}{(\pnuc) (\ppi)}
		\approx 0.996
	\eeq
	which implies
	\beq
		\tht \approx \cos[-1](0.996)
		\approx {\color{blue} 5.13^\circ}.
	\eeq
\end{solution}




\clearpage

\state{Acoustic and optic phonons in the diatomic chain}{
	In the diatomic chain, we take the unit cell to be of length $a$, and take $\xA$ and $\xB$ to be the coordinates of the A and B atoms within the unit cell.  Hence, in the $n$th cell,
	\al{
		\rnA &= n a + \xA; &
		\rnB &= n a + \xB
	}
	\vfix
}

\prob{}{
	In the equations of motion Eq.~(2.30), look for solutions of the form
	\eqn{5a}{
		\unalp = \ealpq \exp( i [ q \rnalp - \omgq t ] ) + \ealpsq \exp( i [-q \rnalp + \omgq t] )
	}
	where $\alp = A$ or $B$, and $\ealp$ are complex numbers that give the amplitude and phase of the oscillation of the two atoms.
	
	Separating out the terms that have the same time dependence, show that (for equal masses, ${\mA = \mB = m}$)
	\al{
		m \omgsq \eAq &= \DAAq \eAq + \DABq \eBq, \\
		m \omgsq \eBq &= \DBAq \eAq + \DBBq \eBq,
	}
	where
	\al{
		\DAAq &= \DBBq = K + K', \\
		-\DABq &= K \exp( i q [ \rnB - \rnA ] ) + K' \exp( i q [ \rnmqB - \rnA ] ), \\
		-\DBAq &= K \exp( i q [ \rnA - \rnB ] ) + K' \exp( i q [ \rnpqA - \rnB ] ).
	}
	Check that $\DAB = \DsBA$.
}

\sol{
	Equation~(2.30) is
	\aln{ \label{thing5a}
		\mA \pdv[2]{\unA}{t} &= K (\unB - \unA) + K' (\unmqB - \unA), &
		\mB \pdv[2]{\unB}{t} &= K' (\unpqA - \unB) + K (\unA - \unB).
	}
	Note that
	\al{
		\pdv[2]{\unalp}{t} &= \pdv{t}\{ -i \omgq \ealpq \exp( i [ q \rnalp - \omgq t ] ) + i \omgq \ealpsq \exp( i [-q \rnalp + \omgq t] ) \} \\
		&= -\omgsq \{ \ealpq \exp( i [ q \rnalp - \omgq t ] ) + \ealpsq \exp( i [-q \rnalp + \omgq t] ) \} \\
		&= -\omgsq \unalp,
	}
	so the first of Eq.~\refeq{5a} can be written
	\al{
		[ K + K' - \mA \omgsq ] \unA &= K \unB + K' \unmqB, \\[1ex]
		&= K \eBq e^{i q \rnB} e^{-i \omgq t} + K \eBsq e^{-i q \rnB} e^{i \omgq t} + K' \eBq e^{i q \rnmqB} e^{-i \omgq t} \\
		&\hspace{5em} \phantom{=\ } + K' \eBsq e^{-i q \rnmqB} e^{i \omgq t} \\[1ex]
		&= K \eBq e^{i q [ na + \xB ]} e^{-i \omgq t} + K' \eBq e^{i q [ (n - 1) a + \xB ]} e^{-i \omgq t} + K \eBsq e^{-i q [ na + \xB ]} e^{i \omgq t} \\
		&\hspace{5em} \phantom{=\ } + K' \eBsq e^{-i q [ (n - 1) a + \xB ]} e^{i \omgq t} \\[1ex]
		&= (K + e^{-i q a} K') \eBq e^{i q (na + \xB)} e^{-i \omgq t} + (K + e^{-i q a} K') \eBsq e^{-i q (na + \xB)} e^{i \omgq t} \\[1ex]
		&= (K + e^{-i q a} K') \unB.
	}
	Generalizing this, we have
	\al{
		[ K + K' - \mA \omgsq ] \unA &= (K + e^{-i q a} K') \unB, &
		[ K + K' - \mB \omgsq ] \unB &= (K + e^{i q a} K') \unA.
	}
	
	Collecting terms of like time dependence yields
	\aln{
		[ K + K' - m \omgsq ] \eAq e^{i q \rnA} &= (K + e^{-i q a} K') \eBq e^{i q \rnB}, \label{thing5.a1} \\
		[ K + K' - m \omgsq ] \eBq e^{i q \rnB} &= (K + e^{i q a} K') \eAq e^{i q \rnA}, \label{thing5.a2}
	}
	for $e^{-i \omg t}$, and
	\al{
		[ K + K' - m \omgsq ] \eAsq e^{-i q \rnA} &= (K + e^{-i q a} K') \eBsq e^{-i q \rnB}, \\
		[ K + K' - m \omgsq ] \eBsq e^{-i q \rnB} &= (K + e^{-i q a} K') \eAsq e^{-i q \rnA}.
	}
	for $e^{i \omg t}$.
	
	Rearranging Eqs.~\refeq{thing5.a1} and \refeq{thing5.a2}, we have
	\al{
		m \omgsq \eAq &= (K + K') \eAq - (K + e^{-i q a} K') e^{i q (\rnB - \rnA)} \eBq \\
		&= (K + K') \eAq - (e^{i q (\rnB - \rnA)} K + e^{i q (\rnmqB - \rnA)} K') \eBq, \\[1ex]
		m \omgsq \eBq &= -(K + e^{i q a} K') e^{i q (\rnA - \rnB)} \eAq - (K + K') \eBq \\
		&= -(e^{i q (\rnA - \rnB)} K + e^{i q (\rnpqA - \rnB)} K') \eAq - (K + K') \eBq,
	}
	which gives us
	\ans{ \al{
		\DAAq &= \DBBq = K + K', \\
		\DABq &= -e^{i q (\rnB - \rnA)} K - e^{i q (\rnmqB - \rnA)} K', \\
		\DBAq &= -e^{i q (\rnA - \rnB)} K - e^{i q (\rnpqA - \rnB)} K',
	}}%
	as we wanted to show. \qed
	
	Finally, note that
	\al{
		\DsBA &= [ -e^{i q (\rnA - \rnB)} K - e^{i q (\rnpqA - \rnB)} K' ]^*
		= -e^{i q (\rnB - \rnA)} K - e^{i q (\rnB - \rnpqA)} K' \\
		&= -e^{i q (\rnB - \rnA)} K - e^{i q (\rnB - \rnA)} e^{-i q a} K'
		= -e^{i q (\rnB - \rnA)} K - e^{i q (\rnmqB - \rnA)} K' \\
		&= \ans{ \DAB }
	}
	as desired. \qed
}



\prob{}{
	The $2 \times 2$ matrix equation can have a nontrivial solution if the determinant vanishes:
	\eq{
		\mqty| 	\DAAq - m \omgsq & \DABq \\
				\DBAq & \DBBq - m \omgsq |
		= 0.
	}
	Hence show that the frequencies of the modes are given by
	\eq{
		m \omgsq = K + K' \pm \sqrt{ (K + K')^2 - 4 K K' \sin[2]( \frac{q a}{2} ) }.
	}
	\vfix
}

\sol{
	The determinant is
	\eq{
		0 = [ \DAAq - m \omgsq ] [ \DBBq - m \omgsq ] - \DABq \DBAq
		= [ \DAAq - m \omgsq ]^2 - \DABq \DBAq,
	}
	which implies
	{\allowdisplaybreaks
	\aln{
		m \omgsq &= \DAAq \pm \sqrt{\DABq \DBAq} \notag \\
		&= K + K' \pm \sqrt{(e^{i q (\rnB - \rnA)} K + e^{i q (\rnmqB - \rnA)} K') (e^{i q (\rnA - \rnB)} K + e^{i q (\rnpqA - \rnB)} K')} \notag \\
		&= K + K' \pm \sqrt{(K + e^{-i q a} K') e^{i q (\rnB - \rnA)} (K + e^{i q a} K') e^{i q (\rnA - \rnB)}} \notag \\
		&= K + K' \pm \sqrt{(K + e^{-i q a} K') (K + e^{i q a} K')}
		= K + K' \pm \sqrt{K^2 + (e^{i q a} + e^{-i q a}) K K' + {K'}^2} \notag \\
		&= K + K' \pm \sqrt{K^2 + 2\cos(q a) K K' + {K'}^2} \label{5bo} \\
		&= K + K' \pm \sqrt{K^2 + \brac{ 2 - 4 \sin[2](\frac{q a}{2}) } K K' + {K'}^2} \notag \\
		&= K + K' \pm \sqrt{K^2 + 2 K K' + {K'}^2 - 4 \sin[2](\frac{q a}{2}) K K'} \notag \\
		&= \ans{ K + K' \pm \sqrt{ (K + K')^2 - 4 K K' \sin[2]( \frac{q a}{2} ) }, } \label{5b}
	}}
	where we have used the double-angle formula $\cos(2x) = 1 - 2 \sin[2](x)$~\cite{DoubleAngle}. \qed
}



\prob{}{
	Sketch the dispersion relations when $K / K' =  2$.
}

\sol{
	There are two dispersion curves since there are two solutions in Eq.~\refeq{5b}.  The expressions for the branches are
	\eqn{5ceq}{
		\omgq = \frac{1}{\sqrt{m}} \sqrt{ K + K' \pm \sqrt{ (K + K')^2 - 4 K K' \sin[2]( \frac{q a}{2} ) } }
		\begin{cases}
			\text{optical}, \\
			\text{acoustic},
		\end{cases}
	}
	where the acoustic~(optical) branch corresponds to the upper~(lower) sign.  Both branches are shown in Fig.~\ref{5c}, with the $K / K' =  2$ case on the left and the $K = K'$ case on the right.
	
	\fig{5c}{
		\includegraphics[width=0.5\textwidth,trim=1.5cm 0 0 0,clip]{5c}
		\caption{Dispersion curves for $K / K' =  2$.  The optical branch~(blue) corresponds to the upper sign in Eq.~\refeq{5ceq}, and the acoustic branch~(gold) to the lower sign.}
	}
}



\prob{}{
	  Discuss what happens if $K = K'$.
}

\sol{
	If $K = K'$, then not only are the masses of the two atoms identical, but so are their restorative forces.  Thus, the system is essentially reduced to a monatomic chain~\cite[p.~437]{Ashcroft}.  Picking up from Eq.~\refeq{5bo},
	\eq{
		m \omgsq = 2 K \pm \sqrt{2 K^2 + 2 \cos(q a) K^2}
		= 2 K \pm K \sqrt{4 \cos[2](\frac{q a}{2})}
		= 2 K \brac{ 1 - \cos(\frac{q a}{2}) }
		= 4 K \sin[2](\frac{q a}{4}),
	}
	where we have used the double-angle formula $\cos(2 x) = 2 \cos[2](x) - 1$~\cite{DoubleAngle}.  This is Eq.~\refeq{2.25} with $q a \to q a / 2$.  So in this limit, the diatomic chain is reduced to a monatomic chain with lattice constant $a / 2$~\cite[p.~437]{Ashcroft}.
}



\vfill
In addition to the course lecture notes, I consulted Jackson's \emph{Classical Electrodynamics} while writing these solutions.
\end{document}