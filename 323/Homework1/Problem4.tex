\newcommand{\mq}{m_1}
\newcommand{\mw}{m_2}
\newcommand{\pq}{p_1}
\newcommand{\pw}{p_2}

\newcommand{\mLam}{\SI{1115}{\MeV}}
\newcommand{\tauLam}{\SI{2.9e-10}{\second}}
\newcommand{\mnuc}{\SI{939}{\MeV}}
\newcommand{\mpi}{\SI{140}{\MeV}}

\begin{statement}{(Jackson 11.20)}
	The lambda particle ($\Lam$) is a neutral baryon of mass $M = \mLam$ that decays with a lifetime of $\tau = \tauLam$ into a nucleon of mass $\mq \approx \mnuc$ and a pi-meson of mass $\mw \approx \mpi$.  It was first observed in flight by its charged decay mode $\Lam \to p + \pi^-$ in cloud chambers.  The charged tracks originate from a single point and have the appearance of an inverted vee or lambda.  The particles' identities and momenta can be inferred from their ranges and curvature in the magnetic field of the chamber.
\end{statement}

\newcommand{\Pq}{P_1}
\newcommand{\Pw}{P_2}

\newcommand{\Pmu}{P^\mu}
\newcommand{\Psmu}{P_\mu}
\newcommand{\Pqmu}{{\Pq}^\mu}
\newcommand{\Pqsmu}{{\Pq}_\mu}
\newcommand{\Pwmu}{{\Pq}^\mu}
\newcommand{\Pwsmu}{{\Pq}_\mu}

\newcommand{\vp}{\vb{p}}
\newcommand{\vpq}{\vp_1}
\newcommand{\vpw}{\vp_2}

\begin{problem} \label{4.a}
	Using conservation of momentum and energy and the invariance of scalar products of 4-vectors show that, if the opening angle $\tht$ between the two tracks is measured, the mass of the decaying particle can be found from the formula
	\beq
		M^2 = \mq^2 + \mw^2 + 2 \Eq \Ew - 2 \pq \pw \cos\tht,
	\eeq
	where here $\pq$ and $\pw$ are the magnitudes of the 3-momenta.
\end{problem}

\begin{solution}
	The general momentum 4-vector for a particle is
	\beq
		\Pmu = (E / c, \vp),
	\eeq
	where $E$ is the energy of the particle and $\vp$ its three-dimensional momentum.  Let $\Pqmu$ and $\Pwmu$ be the momentum 4-vectors for the two particles, and define
	\beq
		\Pmu = \Pqmu + \Pwmu.
	\eeq
	Firstly, note that
	\beq
		\Pmu \Psmu = \frac{E^2}{c^2} - p^2.
	\eeq
	According to Jackson~(11.55),
	\beqn \label{E}
		E = \sqrt{c^2 p^2 + m^2 c^4},
	\eeqn
	so we have
	\beqn \label{thing4.1}
		\Pmu \Psmu = \frac{c^2 p^2 + M^2 c^4}{c^2} - p^2
		= M^2 c^2.
	\eeqn
	Note also that
	\beq
		\Pmu \Psmu = (\Pqmu + \Pwmu) (\Pqsmu + \Pwsmu)
		= \frac{(\Eq + \Ew)^2}{c^2} - (\vpq + \vpw)^2
		= \frac{\Eq^2 + 2 \Eq \Ew + \Ew^2}{c^2} - \pq^2 - 2 \vpq \vdot \vpw - \pw^2,
	\eeq
	and once again making use of \refeq{E},
	\beqn \label{thing4.2}
		\Pmu \Psmu = \frac{c^2 \pq^2 + \mq^2 c^4 + 2 \Eq \Ew + c^2 \pw^2 + \mw^2 c^4}{c^2} - \pq^2 - 2 \vpq \vdot \vpw - \pw^2
		= \mq^2 c^2 + \mw^2 c^2 + 2 \Eq \Ew / c^2 - 2 \pq \pw \cos\tht.
	\eeqn
	Equating \refeq{thing4.1} and \refeq{thing4.2}, we have
	\beq
		M^2 c^2 = \mq^2 c^2 + \mw^2 c^2 + 2 \Eq \Ew / c^2 - 2 \pq \pw \cos\tht.
	\eeq
	Taking $c = 1$, this becomes
	\beq
		M^2 = \mq^2 + \mw^2 + 2 \Eq \Ew - 2 \pq \pw \cos\tht
	\eeq
	as desired. \qed
\end{solution}



\newcommand{\ELam}{\SI{10}{\GeV}}

\begin{problem}
	A lambda particle is created with a total energy of {\ELam} in a collision in the top plate of a cloud chamber.  How far on the average will it travel in the chamber before decaying?  What range of opening angles will occur for a \SI{10}{\GeV} lambda if the decay is more or less isotropic in the lambda's rest frame?
\end{problem}

\newcommand{\Delt}{\Del t}
\newcommand{\lspeed}{\SI{3.00e8}{\meter\per\second}}

\newcommand{\vvq}{\vv_1}
\newcommand{\vvw}{\vv_2}

\newcommand{\Enucp}{\SI{944}{\MeV}}
\newcommand{\Epip}{\SI{171}{\MeV}}
\newcommand{\Enuc}{\SI{8422}{\MeV}}
\newcommand{\Epi}{\SI{1256}{\MeV}}

\newcommand{\gamnuc}{1.005}
\newcommand{\gampi}{1.221}
\newcommand{\gamLam}{8.969}

\newcommand{\pnucp}{\SI{98}{\MeV}}
\newcommand{\ppip}{\pnucp}
\newcommand{\pnuc}{\SI{8369}{\MeV}}
\newcommand{\ppi}{\SI{1248}{\MeV}}


\begin{solution}
	According to Jackson~(11.51),
	\beq
		E = \gam m c^2.
	\eeq
	For the $\Lam$, this gives us $\gam = E / M$ in natural units.  Substituting this into the expression for time dilation, we find
	\beq
		\Delt = \gam \tau
		= \frac{E}{M} \tau
	\eeq
	as the average lifetime in the lab frame.  We also find
	\beq
		\bet = \sqrt{1 - \frac{1}{\gam^2}}
		= \sqrt{1 - \frac{M^2}{E^2}}.
	\eeq
	Then the average distance traveled before decaying is
	\beq
		d = v \,\Delt
		= c \tau \sqrt{1 - \frac{M^2}{E^2}} \frac{E}{M}
		= (\lspeed) \sqrt{1 - \frac{(\mLam)^2}{(\ELam)^2}} \frac{\ELam}{\mLam} (\tauLam)
		= {\color{blue} \SI{77}{\cm}}.
	\eeq
	
	Let $K'$ denote the rest frame of the $\Lam$.  Using the result of \ref{4.a} and the fact that $M$, $\mq$, and $\mw$ are Lorentz invariant, we can write
	\beqn \label{thing4}
		\Eq' \Ew' - \pq' \pw' \cos\tht' = \Eq \Ew - \pq \pw \cos\tht,
	\eeqn
	where $\tht'$ is the angle between the daughter particles in $K'$.  We know from conservation of momentum that they must be ``back to back'' in this frame, meaning $\tht' = \pi$.  Taking this into account and rearranging, we find
	\beqn \label{thing4.2}
		\cos\tht = \frac{\Eq \Ew - \Eq' \Ew' - \pq' \pw'}{\pq \pw}.
	\eeqn

	Say that the $\Lam$ is moving in the $z$ direction in the lab frame $K$.  The opening angle $\tht$ is minimized when the particles are emitted along the $\pm z'$ axis in $K'$, because the boost of the $\Lam$ in $K$ is much greater than that of either daughter particle in $K'$.  In this scenario, both of the daughter particles travel along the $z$ axis in the $+z$ direction in $K$, so the minimum possible opening angle $\tht = 0$.
	
	By a similar argument, $\tht$ is maximized when the daughter particles are emitted transverse to the $z'$ axis in $K'$.  For this scenario, and using natural units, in the $K'$ frame we have
	\begin{align*}
		{P'}^\mu &= (M, \vo), &
		{\Pq'}^\mu &= (\Eq', \vpq'), &
		{\Pw'}^\mu &= (\Ew', \vpw').
	\end{align*}
	Conservation of momentum and energy stipulates that
	\begin{align*}
		M &= \Eq' + \Eq'
		= \gamq' \mq + \gamw' \mw, &
		\vo &= \vpq' + \vpw'
		= \gamq' \vpq' + \gamw' \vpw'
		= \gamq' \mq \vvq' + \gamw' \mw \vvw',
	\end{align*}
	where $\gamq'$ and $\gamw'$ are associated with boosting to the rest frames of $\mq$ and $\mw$, respectively, from $K'$.  Note that
	\beq
		\gamq' \mq \vvq' = -\gamw' \mw \vvw'
		\qimplies
		{\gamq'}^2 \mq^2 {\vvq'}^2 = {\gamw'}^2 \mw^2 {\vvw'}^2,
	\eeq
	and that, in natural units,
	\beq
		\gam^2 v^2 = \gam^2 \bet^2
		= \gam^2 \left( 1 - \frac{1}{\gam^2} \right)
		= \gam^2 - 1.
	\eeq
	Then
	\beq
		({\gamq'}^2 - 1) \mq^2 = ({\gamw'}^2 - 1) \mw^2
		\qimplies
		{\Eq'}^2 - \mq^2 = {\Ew'}^2 - \mw^2
		\qimplies
		\mq^2 - \mw^2 = {\Eq'}^2 - {\Ew'}^2,
	\eeq
	which implies
	\beq
		\mq^2 - \mw^2 = {\Eq'}^2 - (M - \Eq')^2 = -M^2 - 2 M \Eq'
		\qimplies
		\Eq' = \frac{M^2 + \mq^2 - \mw^2}{2 M},
	\eeq
	and similarly for $\Ew$.  Let $\mq$ denote the nucleon and $\mw$ the pion.  Then we have
	\begin{align*}
		\Eq' &= \frac{1}{2} \frac{(\mLam)^2 + (\mnuc)^2 - (\mpi)^2}{\mLam} 
		\approx \Enucp, \\
		\Ew' &= \frac{1}{2} \frac{(\mLam)^2 - (\mnuc)^2 + (\mpi)^2}{\mLam} 
		\approx \Epip.
	\end{align*}
	For the momenta, note that
	\beq
		\pq' = \mq \sqrt{{\gamq'}^2 - 1}
		= \mw \sqrt{{\gamw'}^2 - 1}
		= \pw',
	\eeq
	where
	\begin{align*}
		\gamq' &= \frac{\Eq'}{\mq}
		= \frac{\Enucp}{\mnuc}
		\approx \gamnuc, &
		\gamw' &= \frac{\Ew'}{\mw}
		= \frac{\Epip}{\mpi}
		\approx \gampi.
	\end{align*}
	Then
	\beq
		\pq' = \pw' = (\mpi) \sqrt{{\gampi}^2 - 1}
		\approx \pnucp.
	\eeq
	
	For the left side of \refeq{thing4}, note that
	\beq
		\gam = \frac{E}{M}
		= \frac{\ELam}{\mLam}
		\approx \gamLam
		\qimplies
		\bet = \sqrt{1 - \frac{1}{\gam^2}}
		\approx 0.994,
	\eeq
	whereas
	\begin{align*}
		\betq' &= \sqrt{1 - \frac{1}{{\gamq'}^2}}
		\approx 0.104, &
		\betw' &= \sqrt{1 - \frac{1}{{\gamw'}^2}}
		\approx 0.573. &
	\end{align*}
	Since these are not nearly as relativistic as $\bet$, and we have stipulated that $\vbet'_1$ and $\vbet'_1$ are transverse to the $z'$ axis, it may be sufficient to approximate $\gamq, \gamw \approx \gam$.  Then we have
	\begin{align*}
		\Eq &\approx \gam \mq
		\approx \gamLam (\mnuc)
		\approx \Enuc, &
		\pq &\approx \mq \sqrt{\gam^2 - 1}
		\approx (\mnuc) \sqrt{\gamLam^2 - 1}
		\approx \pnuc, \\
		\Ew &\approx \gam \mw
		\approx \gamLam (\mpi)
		\approx \Epi, &
		\pw &\approx \mw \sqrt{\gam^2 - 1}
		\approx (\mpi) \sqrt{\gamLam^2 - 1}
		\approx \ppi.
	\end{align*}
	Making these substitutions into \refeq{thing4.2}, we have
	\beq
		\cos\tht = \frac{(\Enuc) (\Epi) - (\Enucp) (\Epip) - (\pnucp)^2}{(\pnuc) (\ppi)}
		\approx 0.996
	\eeq
	which implies
	\beq
		\max\tht \approx \cos[-1](0.996)
		\approx 5.1^\circ.
	\eeq
	
	This gives the possible range of $\tht$ as
	\beq
		{\color{blue} 0 \leq \tht \lessapprox 5.1^\circ}.
	\eeq
	\vfix
\end{solution}
