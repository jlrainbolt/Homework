\begin{statement}{\hspace{-1em}}
	Show that $\vE \vdot \vB$ is Lorentz invariant.  You can do this either by using the Lorentz transformation laws for $\vE$ and $\vB$ derived in class, or by writing $\vE \vdot \vB$ in a manifestly Lorentz invariant (and gauge invariant) form.
\end{statement}

\newcommand{\Ft}{\tilde{F}}
\newcommand{\Fsab}{F_{\alp \bet}}
\newcommand{\Ftab}{\Ft^{\alp \bet}}
\newcommand{\Ftsmn}{\Ft_{\alp \bet}}
\newcommand{\sumab}{\sum_{\alp, \bet}}

\newcommand{\Fsmat}{\mqty[	0 & \Eq & \Ew & \Ee \\
						-\Eq & 0 & -\Be & \Bw \\
						-\Ew & \Be & 0 & -\Bq \\
						-\Ee & -\Bw & \Bq & 0 ]}
\newcommand{\Ftmat}{\mqty[	0 & -\Bq & -\Bw & -\Be \\
						\Bq & 0 & \Ee & -\Ew \\
						\Bw & -\Ee & 0 & \Eq \\
						\Be & \Ew & -\Eq & 0 ]}

\begin{solution}
	From Jackson~(11.140),
	\beq
		\Ftab = \Ftmat.
	\eeq
	Then, applying \refeq{F} as well,
	\begin{align*}
		\Fsab \Ftab &= \sumab \Fsmat \Ftmat
		= \sumab \mqty[\vE \vdot \vB & 0 & 0 & 0 \\
					0 & \vE \vdot \vB & 0 & 0 \\
					0 & 0 & \vE \vdot \vB & 0 \\
					0 & 0 & 0 & \vE \vdot \vB ] \\
		&= 4 \vE \vdot \vB.
	\end{align*}
	Here we have shown that $\vE \vdot \vB$ is directly proportional to the inner product of two 4-tensors.  Thus, it is Lorentz invariant. \qed
\end{solution}