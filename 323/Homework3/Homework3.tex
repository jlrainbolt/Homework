\documentclass[11pt]{article}
\usepackage{homework}

\classname{322}
\homeworknum{3}



\begin{document}

% Environments

\newcommand{\state}[2]{\begin{statement}{#1} #2 \end{statement}}
\newcommand{\prob}[2]{\begin{problem}{#1} #2 \end{problem}}
\newcommand{\subprob}[1]{\begin{subproblem} #1 \end{subproblem}}
\newcommand{\sol}[1]{\begin{solution} #1 \end{solution}}
\newcommand{\fig}[2]{\begin{figure} \centering #2  \label{#1} \end{figure}}

\newcommand{\makebib}{
	\vfill
	\color{black}
	\bibliography{references}{}
	\bibliographystyle{lucas_unsrt}
}
	

% Implication

\newcommand{\qwhere}{\quad \text{where} \quad}
\newcommand{\qimplies}{\quad \implies \quad}
\newcommand{\impliesq}{\implies \quad}



% Brackets

\newcommand{\paren}[1]{\left( #1 \right)}
\newcommand{\brac}[1]{\left[ #1 \right]}


% Greek

\newcommand{\alp}{\alpha}
\newcommand{\bet}{\beta}
\newcommand{\gam}{\gamma}
\newcommand{\del}{\delta}
\newcommand{\eps}{\epsilon}
\newcommand{\zet}{\zeta}
\newcommand{\tht}{\theta}
\newcommand{\kap}{\kappa}
\newcommand{\lam}{\lambda}
\newcommand{\sig}{\sigma}
\newcommand{\ups}{\upsilon}
\newcommand{\omg}{\omega}

\newcommand{\Gam}{\Gamma}
\newcommand{\Del}{\Delta}
\newcommand{\Tht}{\Theta}
\newcommand{\Lam}{\Lambda}
\newcommand{\Sig}{\Sigma}
\newcommand{\Omg}{\Omega}
% Problem 1

\newcommand{\Psii}{\Psi^i}
\newcommand{\Psiix}{\Psii(x)}

\newcommand{\Pii}{\Pi^i}

\newcommand{\Phii}{\Phi^i}
\newcommand{\Phiix}{\Phii(x)}
\newcommand{\PhiN}{\Phi^N}
\newcommand{\PhiNx}{\PhiN(x)}
\newcommand{\Phiq}{\Phi^1}
\newcommand{\Phiw}{\Phi^2}

\newcommand{\ddcx}{\dd[3]{x}}

\newcommand{\delij}{\del^{i j}}
\newcommand{\delkl}{\del^{k l}}
\newcommand{\delil}{\del^{i l}}
\newcommand{\deljk}{\del^{j k}}
\newcommand{\delik}{\del^{i k}}
\newcommand{\deljl}{\del^{j l}}

\newcommand{\DF}{D_F}

\newcommand{\sigx}{\sig(x)}

\newcommand{\pii}{\pi^i}
\newcommand{\pij}{\pi^j}
\newcommand{\pik}{\pi^k}
\newcommand{\pil}{\pi^l}
\newcommand{\piix}{\pi(x)}

\newcommand{\pq}{p_1}
\newcommand{\pw}{p_2}
\newcommand{\pe}{p_3}
\newcommand{\pr}{p_4}

\newcommand{\vp}{\vb{p}}
\newcommand{\vpsi}{\vp_i}

\newcommand{\mpi}{m_\pi}

\state{(Jackson 12.3)}{
	A particle with mass $m$ and charge $e$ moves in a uniform, static, electric field $\vEo$.
}

%
%	12.3(a)
%

\prob{}{
	Solve for the velocity and position of the particle as explicit functions of time, assuming that the initial velocity $\vvo$ was perpendicular to the electric field.
}

\sol{

}

\clearpage
%
%	12.3(b)
%

\prob{}{
	Eliminate the time to obtain the trajectory of the particle in space.  Discuss the shape of the path for short and long times (define ``short'' and ``long'' times).
}



\clearpage
\state{(Jackson 12.5)}{
	A particle of mass $m$ and charge $e$ moves in the laboratory in crossed, static, uniform, electric and magnetic fields.  $\vE$ is parallel to the $x$ axis; $\vB$ is parallel to the $y$ axis.
}

\prob{}{
	For $\abs{\vE} < \abs{\vB}$ make the necessary Lorentz transformation described in Section~12.3 to obtain explicitly parametric equations for the particle's trajectory.
}

\prob{}{
	Repeat the calculation of part~(a) for $\abs{\vE} > \abs{\vB}$.
}



\state{(Jackson 12.19)}{
	Source-free electromagnetic fields exist in a localized region of space.  Consider the various conservation laws that are contained in the integral of $\ptsa \Mabg = 0$ over all space, where
	\eq{
		\Mabg = \Thab \xg - \Thag \xb.
	}
	\vfix
}

\prob{}{
	Show that when $\bet$ and $\gam$ are both space indices conservation of the total field angular momentum follows.
}

\prob{}{
	Show that when $\bet = 0$ the conservation law is
	\eq{
		\dv{\vX}{t} = \frac{c^2 \vPem}{\Eem},
	}
	where $\vX$ is the coordinate of the center of mass of the electromagnetic fields, defined by
	\eq{
		\vX \int u \dcx = \int \vx u \dcx,
	}
	where $u$ is the electromagnetic energy density and $\Eem$ and $\vPem$ are the total energy and momentum of the fields.
}



\state{}{
	We discussed in class the construction of linearly polarized electromagnetic waves.
}

\prob{}{
	Generalize the discussion to circularly polarized waves (see also Wald Sec.~{5.5}).  Discuss both right-handed and left-handed polarizations.
}

\prob{}{
	Compute the angular momentum of the circularly polarized waves of part~(a) using the formula for angular momentum derived in class.
}



\state{}{
	We wrote in class the Lagrangian of a charged particle coupled to the electromagnetic field (see pp.~159--160) in the lecture notes).
}

\prob{}{
	Show that the Euler-Lagrange equations that follow from this Lagrangian give rise to the Lorentz force law %on p.~161.
	\eq{
		\dv{\pii}{t} = q \brac{ \Ei + \frac{1}{c} (\vv \cross \vB)^i }.
	}
	\vfix
}

\prob{}{
	Show that the Lorentz force law can be written covariantly in the form %given on p.~163 of the lecture notes (and Eq.~(11.163) in Jackson).
	\eq{
		\dv{\Um}{\tau} = \frac{q}{m c} \Fmn \Usn.
	}
	\vfix
}


\makebib

\end{document}