\state{(Jackson 12.3)}{
	A particle with mass $m$ and charge $e$ moves in a uniform, static, electric field $\vEo$.
}

%
%	12.3(a)
%

\prob{}{
	Solve for the velocity and position of the particle as explicit functions of time, assuming that the initial velocity $\vvo$ was perpendicular to the electric field.
}

\sol{
	Jackson~(12.1) gives the force exerted on a charged particle in an external electromagnetic field:
	\al{
		\dv{\vp}{t} &= e \paren{ \vE + \frac{\vuu}{c} \cross \vB },
	}
	where $\vuu$ is the velocity of the particle.  In this problem $\vE = \vEo$ and $\vB = \vo$, so
	\eq{
		\dv{\vp}{t} = e \vEo.
	}
		Since $\vEo$ is constant, we can easily find $\vp$ as a function of time by solving this expression as a differential equation.  This gives us
	\eq{
		\intoi \dd{\vp} = e \vEo \intoi \ddt
		\qimplies
		\vp(t) = e \vEo t + \vpo.
	}
	
	In order to use this result to find the velocity of the particle, we need to write the particle's velocity in terms of its momentum.  According to Jackson~(11.46), (11.51), and (11.55), $\vp = \gam m v$, $\cE = m\gam c^2$, and $\cE = \sqrt{c^2 p^2 + m^2 c^4}$, where $\cE$ is the total energy of the particle.  Combining these gives us
	\eqn{velocity1}{
		\vv = \frac{\vp}{\gam m}
		= \frac{c^2 \vp}{\cE}
		= \frac{c \vp}{\sqrt{m^2 c^2 + \vp^2}}.
	}
	Then, substituting into Eq.~\refeq{velocity1}, we have
	\eq{
		\vv(t) = c \frac{e \vEo t + \vpo}{\sqrt{m^2 c^2 + (e \vEo t + \vpo)^2}}
		= c \frac{e \vEo t + \vpo}{\sqrt{m^2 c^2 + e^2 \vEo^2 t^2 + 2 e \vpo \vdot \vEo t + \vpo^2}}
		= c \frac{e \vEo t + \vpo}{\sqrt{m^2 c^2 + e^2 \vEo^2 t^2 + \vpo^2}},
	}
	where in going to the final equality we have used the fact that $\vpo = \gam m \vvo$ is perpendicular to $\vEo$.
	
	Finally, we can solve this as a differential equation to find the position of the particle as a function of time.  Let $\vv(t) = \dv*{\vr}{t}$, where $\vr(t)$ is the position of the particle.  Then
	\aln{
		\intoi \dd{\vr} &= c \intoi \frac{e \vEo t + \vpo}{\sqrt{m^2 c^2 + e^2 \vEo^2 t^2 + \vpo^2}} \ddt \notag \\
		&= c e \vEo \intoi \frac{t}{\sqrt{m^2 c^2 + e^2 \vEo^2 t^2 + \vpo^2}} \ddt + c \vpo \intoi \frac{\ddt}{\sqrt{m^2 c^2 + e^2 \vEo^2 t^2 + \vpo^2}}. \label{thing1.2}
	}
	For the first integral on the right side,
	\eq{
		\intoi \frac{t}{\sqrt{e^2 \vEo^2 t^2 + m^2 c^2 + \vpo^2}} \ddt
%		= \intoi \frac{t}{\sqrt{u}} \frac{\ddu}{2 e^2 \vEo^2 t}
		= \frac{1}{2 e^2 \vEo^2} \int_{\uo}^\infty \frac{1}{\sqrt{u}} \ddu
		= \frac{\sqrt{u} - \sqrt{\uo}}{e^2 \vEo^2},
%		= \frac{\sqrt{e^2 \vEo^2 t^2 + m^2 c^2 + \vpo^2}}{e^2 \vEo^2},
	}
	where we have used the substitution $u = e^2 \vEo^2 t^2 + m^2 c^2 + \vpo^2$, with $\uo = m^2 c^2 + \vpo^2$.
	
	For the second integral on the right side of Eq.~\refeq{thing1.2},
	\eq{
		\intoi \frac{\ddt}{\sqrt{e^2 \vEo^2 t^2 + m^2 c^2 + \vpo^2}} = \frac{1}{\sqrt{m^2 c^2 + \vpo^2}} \intoi \frac{\ddt}{\sqrt{e^2 \vEo^2 t^2 / (m^2 c^2 + \vpo^2) + 1}}
		= \frac{1}{e \absvEo} \intoi \frac{\ddu}{\sqrt{u^2 + 1}}
		= \frac{\sinh^{-1} u}{e \absvEo},
	}
	where we have used the substitution $u = e \absvEo t / \sqrt{m^2 c^2 + \vpo^2}$, the fact that $\dv*{\sinh^{-1} z}{z} = 1 / \sqrt{1 + z^2}$, and the fact that $\sinh \uo = \sinh(0) = 0$~\cite{Arcsinh}.
	
	With these solutions, Eq.~\refeq{thing1.2} becomes
	\eq{
		\ans{ \vr(t) = \frac{c}{e \absvEo} \brac{ \frac{\vEo}{\absvEo} \paren{ \sqrt{e^2 \vEo^2 t^2 + m^2 c^2 + \vpo^2} - \sqrt{m^2 c^2 + \vpo^2} } + \vpo \sinh[-1](\frac{e \absvEo t}{\sqrt{m^2 c^2 + \vpo^2}}) } + \vro, }
	}
	where $\vro$ is the initial position of the particle.
	
	To make the equation a little neater, we can once again apply Jackson~(11.55) to define the particle's initial energy as $\Eo = \sqrt{m^2 c^4 + c^2 \vpo^2}$.  This gives us
	\eq{
		\ans{ \vr(t) = \frac{1}{e \absvEo} \brac{ \frac{\vEo}{\absvEo} \paren{ \sqrt{c^2 e^2 \vEo^2 t^2 + \Eo^2} - \Eo } + c \vpo \sinh[-1](\frac{c e \absvEo t}{\Eo}) } + \vro. }
	}
	For the velocity, we have
	\eq{
		\ans{ \vv(t) = \frac{c e \vEo t + c \vpo}{\sqrt{e^2 \vEo^2 t^2 + m^2 c^2 + \vpo^2}}
		= \frac{c^2 e \vEo t + c^2 \vpo}{\sqrt{c^2 e^2 \vEo^2 t^2 + \Eo^2}}. }
	}
	\vfix
}

%
%	12.3(b)
%

\prob{}{
	Eliminate the time to obtain the trajectory of the particle in space.  Discuss the shape of the path for short and long times (define ``short'' and ``long'' times).
}

\sol{
	Let $\vro = \vo$, and let $\rperp(t)$ and $\rpar(t)$ denote the components of the particle's position that are, respectively, parallel to and perpendicular to its original velocity.  Then
	\aln{ \label{r2}
		\rperp(t) &= \frac{\sqrt{c^2 e^2 \vEo^2 t^2 + \Eo^2} - \Eo}{e \absvEo}, &
		\rpar(t) &= \frac{c \po}{e \absvEo} \sinh[-1](\frac{c e \absvEo t}{\Eo}).
	}
	It is easiest to solve $\rpar(t)$ for $t$, which gives us
	\eq{
		\frac{e \absvEo \rpar}{c \po} = \sinh[-1](\frac{c e \absvEo t}{\Eo})
		\qimplies
%		\sinh(\frac{e \absvEo \rpar}{c \po}) = \frac{c e \absvEo t}{\Eo}
%		\qimplies
		t = \frac{\Eo}{c e \absvEo} \sinh(\frac{e \absvEo \rpar}{c \po}).
	}
	Substituting into the expression for $\rperp$, we find
	\al{
		\rperp &= \frac{1}{e \absvEo} \paren{ \sqrt{c^2 e^2 \vEo^2 \brac{ \frac{\Eo}{c e \absvEo} \sinh(\frac{e \absvEo \rpar}{c \po}) }^2 + \Eo^2} - \Eo }
		= \frac{\Eo}{e \absvEo} \brac{ \sqrt{\sinh[2](\frac{e \absvEo \rpar}{c \po}) + 1} - 1 } \\
		&= \frac{\Eo}{c \absvEo} \brac{ \cosh(\frac{e \absvEo \rpar}{c \po}) - 1 },
	}
	where we have used $\cosh^2 x - \sinh^2 x = 1$~\cite{Hyperbolic}.  Then the trajectory of the particle is given by
	\eq{
		\ans{ \rperp = \frac{\Eo}{c \absvEo} \brac{ \cosh(\frac{e \absvEo \rpar}{c \po}) - 1 }
		= \frac{\sqrt{m c^2 + \vpo^2}}{\absvEo} \brac{ \cosh(\frac{e \absvEo \rpar}{c \po}) - 1 }. }
	}
	
	For short times, the argument of $\sinh^{-1}$ in Eq.~\refeq{r2} must be small.  Note also that $\rperp(t)$ can be written as
	\eqn{traj}{
		\rperp(t) = \Eo \frac{\sqrt{c^2 e^2 \vEo^2 t^2 / \Eo^2 + 1} - 1}{e \absvEo},
	}
	so we can conclude that \ans{ $t \ll \Eo / c e \absvEo$ for short times. }  Likewise, \ans{ $t \gg \Eo / c e \absvEo$ for long times. }
	
	To obtain the trajectory for short times, we note that $u = \Eo / c e \absvEo \ll 1$ implies that $\rpar \ll 1$.  Thus, we can Taylor expand Eq.~\refeq{traj} around $\rpar = 0$.  The Taylor series for $\cosh$ is~\cite{Cosh}
	\eq{
		\cosh z = 1 + \frac{z^2}{2} + \cdots,
	}
	so
	\eq{
		\ans{ \limuo \rperp = \frac{\Eo}{c \absvEo} \frac{\rpar^2}{2}
		= \frac{\sqrt{m^2 c^2 + \vpo^2}}{\absvEo} \frac{\rpar^2}{2}, }
	}
	\ans{ indicating that the trajectory of the particle is parabolic for short times.}  As soon as the field is turned on, it will start pulling the particle in a direction that is perpendicular to its original velocity.  This is just like projectile motion under the influence of gravity.
	
	To obtain the trajectory for long times, we note that $\limui \sinh^{-1} u = \infty$~\cite{Arcsinh}, so taking $u$ to be large is the same as taking $\rpar$ to be large.  Note that
	\eq{
		\limzi \cosh z = \limzi \frac{e^z + e^{-z}}{2}
		= \frac{e^z}{2},
	}
	so
	\eq{
		\ans{ \limui \rperp = \frac{\Eo}{c \absvEo} \frac{e^{\rpar}}{2}
		= \frac{\sqrt{m^2 c^2 + \vpo^2}}{\absvEo} \frac{e^{\rpar}}{2}, }
	}
	\ans{ indicating that the trajectory of the particle is exponential for long times.}  When the field has been turned on for a long time, the particle has been accelerating parallel to the field for a long time.  At infinite time, the particle's original direction of velocity has been completely washed out by the force of the electric field.
	
%	To obtain the trajectory for short times, we Taylor expand Eq.~\refeq{r2} about $u = \Eo / c e \absvEo = 0$.  Let
%	\aln{ \label{ru}
%		\rperp(u) &= \Eo \frac{\sqrt{u^2 + 1} - 1}{e \absvEo}, &
%		\rpar(u) &= \frac{c \po}{e \absvEo} \sinh^{-1} u.
%	}
%	Note also that
%	\al{
%		\dv{(\sinh^{-1} u)}{u} &= \dv{u} \ln(u + \sqrt{u^2 + 1})
%		= \frac{1}{u + \sqrt{u^2 + 1}} \paren{ 1 + \frac{u}{\sqrt{u^2 + 1}} }
%		= \frac{1}{\sqrt{u^2 + 1}}, \\[2ex]
%		\dv[2]{(\sinh^{-1} u)}{u} &= \dv{u} \paren{ \frac{1}{\sqrt{u^2 + 1}} }
%		= -\frac{u}{(u^2 + 1)^{3/2}}, % \\[2ex]
%%		\dv[3]{(\sinh^{-1} u)}{u} &= \dv{u} \paren{ -\frac{u}{(u^2 + 1)^{3/2}} }
%%		= \frac{3 u^2}{(u^2 + 1)^{5/2}} - \frac{u}{(u^2 + 1)^{3/2}},
%	}
%	where we have applied the definition of $\sinh^{-1}$~\cite{Arcsinh}
%	Then
%	\al{
%		\rperp(u) &= \rperp(u = 0) + u \bigg[ \dv{\rperp}{u} \bigg]_{u = 0} + \frac{u^2}{2} \bigg[ \dv[2]{\rperp}{u} \bigg]_{u = 0} + \order{u^3} \\
%		&\approx u \bigg[ \frac{\Eo}{e \absvEo} \frac{u}{\sqrt{u^2 + 1}} \bigg]_{u = 0} + \frac{u^2}{2} \bigg[ \frac{\Eo}{e \absvEo} \paren{ \frac{1}{\sqrt{u^2 + 1}} - \frac{u}{(u^2 + 1)^{3/2}} } \bigg]_{u = 0}
%		= \frac{\Eo}{e \absvEo} \frac{u^2}{2}, \\[2ex]
%		\rpar(u) &= \rpar(u = 0) + u \bigg[ \dv{\rpar}{u} \bigg]_{u = 0} + \frac{u^2}{2} \bigg[ \dv[2]{\rpar}{u} \bigg]_{u = 0} + \order{u^3} \\
%		&\approx u \bigg[ \frac{\Eo}{e \absvEo} \frac{1}{\sqrt{u^2 + 1}} \bigg]_{u = 0} + \frac{u^2}{2} \bigg[ -\frac{\Eo}{e \absvEo} \frac{u}{(u^2 + 1)^{3/2}} \bigg]_{u = 0}
%		= \frac{\Eo}{e \absvEo} u.
%	}
%	Solving $\rpar$ for $u$ and substituting into $\rperp$, we find
%	\eq{
%		u = \frac{e \absvEo}{\Eo} \rpar
%		\qimplies
%		\ans{ \rperp = \frac{e \absvEo}{2 \Eo} \rpar^2 = \frac{e \absvEo}{2 c \sqrt{m^2 c^2 \vpo^2}} \rpar^2, }
%	}
%	\ans{ indicating that the trajectory of the particle is parabolic for short times. }  As soon as the field is turned on, it will start pushing the particle in a direction that is perpendicular to its original velocity.  However, 
%	
%	For long times, $u \to \infty$.  In this limit, Eq.~\refeq{ru} becomes
%	\al{
%		\limui \rperp &= \limui \Eo \frac{u - 1}{e \absvEo}
%		= \limui \frac{\Eo}{e \absvEo} u, \\[2ex]
%		\limui \rpar &= \limui \frac{c \po}{e \absvEo} \ln(u + \sqrt{u^2 + 1})
%		= \limui \frac{c \po}{e \absvEo} \ln(u + u)
%		= \limui \frac{c \po}{e \absvEo} \ln(2u).
%	}
%	Once again solving $\rpar$ for $u$, we find
%	\eq{
%		\frac{e \absvEo}{c \po} \rpar = \ln 2u
%		\qimplies
%		u = \frac{1}{2} \exp(\frac{e \absvEo}{c \po})
%		\qimplies
%		\ans{ \rperp = \frac{\Eo}{2 e \absvEo} e^{e \absvEo / c \po}
%		= \frac{c \sqrt{m^2 c^2 + \vpo^2}}{2 e \absvEo} e^{e \absvEo / c \po}, }
%	}
%	\ans{ indicating that the trajectory of the particle is exponential for long times. }
}