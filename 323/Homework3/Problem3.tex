\state{(Jackson 12.19)}{
	Source-free electromagnetic fields exist in a localized region of space.  Consider the various conservation laws that are contained in the integral of $\ptsa \Mabg = 0$ over all space, where %$\Mabg$ is defined by (12.117).
	\eqn{Mabg}{
		\Mabg = \Thab \xg - \Thag \xb.
	}
	\vfix
}

%
%	12.19(a)
%

\prob{}{
	Show that when $\bet$ and $\gam$ are both space indices conservation of the total field angular momentum follows.
}

\sol{
Conservation of angular momentum means that $\dv*{\vL}{t} = \vo$, where $\vL$ is defined by the equation just before Jackson~(12.109):
	\eqn{angmom}{
		\vL = \frac{1}{4\pi c} \int \vx \cross (\vE \cross \vB) \dcx.
	}
	\clearpage
	The field invariants corresponding to the electromagnetic field being invariant under orthochronous Lorentz transformations are summarized by $\Mabg$.  These field invariants are~\cite[pp.184--185]{Gelfand}
	\al{
		\int M^{0 \bet \gam} \dcx, &
		\qq{where} \bet < \gam.
	}
	By definition, field invariants do not change with time~\cite[p.~182]{Gelfand}.  That is,
	\aln{ \label{invariants}
		\dv{(ct)}(\int M^{0 \bet \gam} \dcx) \dcx &= 0, &
		\qq{where} \bet &< \gam.
	}
	Thus, we need to show that the components of $\vL$ are given by Eq.~\refeq{invariants}. 
	
	Let $\bet = i$ and $\gam = j$, which are both spatial indices.  From Jackson~(12.114),
	\aln{ \label{Theta}
		\Tht^{0 0} &= \frac{\vE^2 + \vB^2}{8\pi}, &
		\Tht^{0 i} &= \frac{(\vE \cross \vB)_i}{4\pi}, &
		\Tht^{i j} &= -\frac{\Eii \Ejj + \Bii \Bjj - \delij (\vE^2 + \vB^2) / 2}{4\pi}.
	}
	Applying this to Eq.~\refeq{Mabg} for $\alp = 0$, we have
	\eq{
		\Moij = \Tht^{0 i} \xjj - \Tht^{0 j} \xii
		= \frac{(\vE \cross \vB)_i}{4\pi} \xjj - \frac{(\vE \cross \vB)_j}{4\pi} \xii
		= \frac{1}{4\pi} \begin{cases}
			\!\, [ \vx \cross (\vE \cross \vB) ]_k & i \neq j, \\
			0 & i = j,
			\end{cases}
	}
	where $k$ is orthogonal to both $i$ and $j$.
	
	Applying Eq.~\refeq{invariants}, we find
	\eq{
		0 = \dv{(ct)}(\int \Moij \dcx)
		= \dv{t}(\frac{1}{4\pi c} \int [ \vx \cross (\vE \cross \vB) ]_k \dcx)
		= \dv{\Lk}{t}.
	}
	In vector notation, we have
	\eq{
		\ans{ \dv{\vL}{t} = 0 }
	}
	as desired. \qed
}

%
%	12.19(b)
%

\prob{}{
	Show that when $\bet = 0$ the conservation law is
	\eqn{show3b}{
		\dv{\vX}{t} = \frac{c^2 \vPem}{\Eem},
	}
	where $\vX$ is the coordinate of the center of mass of the electromagnetic fields, defined by
	\eqn{thing3b}{
		\vX \int u \dcx = \int \vx u \dcx,
	}
	where $u$ is the electromagnetic energy density and $\Eem$ and $\vPem$ are the total energy and momentum of the fields.
}

\sol{
	From Wald~(5.9--10), the energy density and momentum density of the electromagnetic field are, respectively,
	\aln{ \label{EP1}
		u &= \frac{\vE^2 + \vB^2}{8\pi}, &
		\vcP &= \frac{\vE \cross \vB}{4\pi c}.
	}
	Then the total energy and momentum of the fields are, respectively,
	\aln{ \label{EP2}
		\Eem &= \int u \dcx
		= \frac{1}{8\pi} \int (\vE^2 + \vB^2) \dcx, &
		\vPem &= \int \vcP \dcx
		= \frac{1}{4\pi c} \int \vE \cross \vB \dcx.
	}
	
	Proceeding similarly as in Prob.~{3(b)}, we need to show that Eq.~\refeq{show3b} follows from Eq.~\refeq{invariants}.  Applying Eq.~\refeq{Mabg} with $\alp = \bet = 0$,
	\eq{
		\Moog = \Tht^{0 0} \xg - \Tht^{0 \gam} \xo.
	}
	Clearly $M^{0 0 0} = 0$, so we need only concern ourselves with the case in which $\gam = i$ is a space index.  Applying Eqs.~\refeq{Theta} and \refeq{EP1},
	\eq{
		\Mooi = \frac{\vE^2 + \vB^2}{8\pi} \xii - c t \frac{(\vE \cross \vB)_i}{4\pi}
		= u \xii - c^2 t \cPi.
	}
	Applying Eq.~\refeq{invariants}, we find
	\al{
		0 &= \dv{(ct)}(\int M^{0 0 i} \dcx)
		= \frac{1}{c} \dv{t}(\int (u \xii - c^2 t \cPi) \dcx)
		= \frac{1}{c} \dv{t}(\int u \xii \dcx) - \frac{1}{c} \int \dv{t} (c^2 t \cPi) \dcx \\
		&= \frac{1}{c} \dv{t}(\int u \xii \dcx) - c \int \cPi \dcx,
	}
	where in going to the third equality we have moved the derivative inside the integral.  In vector notation,
	\al{
		0 &= \frac{1}{c} \dv{t} \paren{ \int u \vx \dcx } - c \int \vP \dcx
		= \frac{1}{c} \dv{t} \paren{ \vX \int u \dcx } - c \vPem
		= \frac{1}{c} \dv{t} (\vX \Eem) - c \vPem \\
		&= \frac{1}{c} \paren{ \Eem \dv{\vX}{t} + \vX \dv{\Eem}{t} } - c \vPem
		= \frac{\Eem}{c} \dv{\vX}{t} - c \vPem,
	}
	where we have applied Eqs.~\refeq{thing3b} and \refeq{EP2}, and the fact that $\dv*{\Eem}{t} = 0$, since the total energy is also a field invariant.  Rearranging, we have
	\eq{
		\ans{ \dv{\vX}{t} = \frac{c^2 \vPem}{\Eem} }
	}
	as desired. \qed
}