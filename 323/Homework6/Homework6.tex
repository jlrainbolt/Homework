\documentclass[11pt]{article}
\usepackage{homework}

\classname{323}
\homeworknum{6}



\begin{document}

% Environments

\newcommand{\state}[2]{\begin{statement}{#1} #2 \end{statement}}
\newcommand{\prob}[2]{\begin{problem}{#1} #2 \end{problem}}
\newcommand{\subprob}[1]{\begin{subproblem} #1 \end{subproblem}}
\newcommand{\sol}[1]{\begin{solution} #1 \end{solution}}
\newcommand{\fig}[2]{\begin{figure} \centering #2  \label{#1} \end{figure}}

\newcommand{\makebib}{
	\vfill
	\color{black}
	\bibliography{references}{}
	\bibliographystyle{lucas_unsrt}
}
	

% Implication

\newcommand{\qwhere}{\quad \text{where} \quad}
\newcommand{\qimplies}{\quad \implies \quad}
\newcommand{\impliesq}{\implies \quad}



% Brackets

\newcommand{\paren}[1]{\left( #1 \right)}
\newcommand{\brac}[1]{\left[ #1 \right]}


% Greek

\newcommand{\alp}{\alpha}
\newcommand{\bet}{\beta}
\newcommand{\gam}{\gamma}
\newcommand{\del}{\delta}
\newcommand{\eps}{\epsilon}
\newcommand{\zet}{\zeta}
\newcommand{\tht}{\theta}
\newcommand{\kap}{\kappa}
\newcommand{\lam}{\lambda}
\newcommand{\sig}{\sigma}
\newcommand{\ups}{\upsilon}
\newcommand{\omg}{\omega}

\newcommand{\Gam}{\Gamma}
\newcommand{\Del}{\Delta}
\newcommand{\Tht}{\Theta}
\newcommand{\Lam}{\Lambda}
\newcommand{\Sig}{\Sigma}
\newcommand{\Omg}{\Omega}
% Problem 1

\newcommand{\Psii}{\Psi^i}
\newcommand{\Psiix}{\Psii(x)}

\newcommand{\Pii}{\Pi^i}

\newcommand{\Phii}{\Phi^i}
\newcommand{\Phiix}{\Phii(x)}
\newcommand{\PhiN}{\Phi^N}
\newcommand{\PhiNx}{\PhiN(x)}
\newcommand{\Phiq}{\Phi^1}
\newcommand{\Phiw}{\Phi^2}

\newcommand{\ddcx}{\dd[3]{x}}

\newcommand{\delij}{\del^{i j}}
\newcommand{\delkl}{\del^{k l}}
\newcommand{\delil}{\del^{i l}}
\newcommand{\deljk}{\del^{j k}}
\newcommand{\delik}{\del^{i k}}
\newcommand{\deljl}{\del^{j l}}

\newcommand{\DF}{D_F}

\newcommand{\sigx}{\sig(x)}

\newcommand{\pii}{\pi^i}
\newcommand{\pij}{\pi^j}
\newcommand{\pik}{\pi^k}
\newcommand{\pil}{\pi^l}
\newcommand{\piix}{\pi(x)}

\newcommand{\pq}{p_1}
\newcommand{\pw}{p_2}
\newcommand{\pe}{p_3}
\newcommand{\pr}{p_4}

\newcommand{\vp}{\vb{p}}
\newcommand{\vpsi}{\vp_i}

\newcommand{\mpi}{m_\pi}

\state{(Jackson 9.8)}{\ 
	%\emph{Hint:} The electromagnetic angular momentum density comes from more than the transverse (radiation zone) components of the fields.
}

%
%	Jackson 9.8(a)
%

\prob{}{
	Show that a classical oscillating electric dipole $\vp$ with fields given by
	\aln{ \label{fields1}
		\vH &= \frac{c k^2}{4\pi} (\nh \cross \vp) \frac{e^{i k r}}{r} \paren{ 1 - \frac{1}{i k r} }, &
		\vE &= \frac{1}{4\pi \epso} \curly{ k^2 (\nh \cross \vp) \cross \nh \frac{e^{i k r}}{r} + [ 3 \nh (\nh \vdot \vp) - \vp ] \paren{ \frac{1}{r^3} - \frac{i k}{r^2} } e^{i k r} },
	}
	radiates electromagnetic angular momentum to infinity at the rate
	\eq{
		\dv{\vL}{t} = \frac{k^3}{12 \pi \epso} \Im[ \vp^* \cross \vp ].
	}
	\vfix
}

\sol{
	According to Jackson~(9.20), the time-averaged angular momentum density is
	\eq{
		\vl = \frac{\Re[ \vx \cross (\vE \cross \vHs)}{2 c^2}.
	}
	One of the vector identities on the inside cover of Jackson is $\vaa \cross (\vbb \cross \vcc) = (\vaa \vdot \vcc) \vbb - (\vaa \vdot \vbb) \vcc$, so
	\eqn{l1}{
		\vl = \frac{(\vx \vdot \vHs) \vE - (\vx \vdot \vE) \vHs}{2 c^2}.
	}
	From Eq.~\refeq{fields1}, note that
	\eq{
		\vx \vdot \vHs \propto \vx \vdot (\nh \cross \vps)
		= \vps \vdot (\vx \cross \nh)
		= \vO,
	}
	where we have used the identity $\vaa \vdot (\vbb \cross \vcc) = \vcc \vdot (\vaa \cross \vbb)$ and the fact that $\nh$ points in the $\vx$ direction.  For $\vx \vdot \vE$, note that
	\al{
		\vx \vdot [ (\nh \cross \vp) \cross \nh ] &= -\vx \vdot [ \nh \cross (\nh \cross \vp) ]
		= -\vx \vdot [ (\nh \vdot \vp) \nh - (\nh \vdot \nh) \vp ]
		= -(\nh \vdot \vp) (\vx \vdot \nh) + \vx \vdot \vp \\
		&= -r (\nh \vdot \vp) + \vx \vdot \vp
		= \vx \vdot \vp - \vx \vdot \vp
		= 0, \\[1.5ex]
		\vx \vdot [ 3 \nh (\nh \vdot \vp) - \vp ] &= 3 (\vx \vdot \nh) (\nh \vdot \vp) - \vx \vdot \vp
		= 3r (\nh \vdot \vp) - \vx \vdot \vp
		= 3(\vx \vdot \vp) - \vx \vdot \vp
		= 2(\vx \vdot \vp),
	}
	since $\abs{\vx} = r$ and $\vx = r \,\nh$.  Then
	\eq{
		\vx \vdot \vE = \frac{1}{2\pi \epso} (\vx \vdot \vp) \paren{ \frac{1}{r^3} - \frac{i k}{r^2} } e^{i k r}
		= \frac{1}{2\pi \epso} (\nh \vdot \vp) \paren{ \frac{1}{r^2} - \frac{i k}{r} } e^{i k r}.
	}
	
	With these substitutions, Eq.~\refeq{l1} becomes
	\al{
		\vl &= -\frac{(\vx \vdot \vE) \vHs}{c^2}
		= -\frac{1}{4\pi \epso c^2} (\nh \vdot \vp) \paren{ \frac{1}{r^2} - \frac{i k}{r} } e^{i k r} \frac{c k^2}{4\pi} (\nh \cross \vps) \frac{e^{-i k r}}{r} \paren{ 1 + \frac{1}{i k r} } \\
		&= -\frac{k^2}{16\pi^2 \epso c r} (\nh \vdot \vp) (\nh \cross \vps) \paren{ \frac{1}{r^2} - \frac{i k}{r} } \paren{ 1 - \frac{i}{k r} }
		= -\frac{k^2}{16\pi^2 \epso c} (\nh \vdot \vp) (\nh \cross \vps) \paren{ \frac{1}{r^2} - \frac{i}{k r^3} - \frac{i k}{r} - \frac{1}{r^2} } \\
		&= -\frac{i k^2}{16\pi^2 \epso c r} (\nh \vdot \vp) (\nh \cross \vps) \paren{ \frac{1}{k r^3} + \frac{k}{r^2} }
		= \frac{i k^3}{16\pi^2 \epso c r^2} (\nh \vdot \vp) (\nh \cross \vps) \paren{ \frac{1}{k^2 r^2} + 1 }.
	}
	
	Let $\vL$ be the angular momentum radiated to a distance $R$.  Then
	\eq{
		\vL = \int_R \vl(r) \ddcx
		= \intopi \intotp \intoR \vl(r) \,r^2 \sin\tht \ddr \ddphi \dd\tht,
	}
	and the time derivative is
	\aln{
		\dv{\vL}{t} &= \dv{t}(\intopi \intotp \intoR \vl(r) \,r^2 \sin\tht \ddr \ddphi \dd\tht)
		= \dv{r}{t} \dv{r}(\intopi \intotp \intoR \vl(r) \,r^2 \sin\tht \ddr \ddphi \dd\tht) \notag \\
		&= c \intopi \intotp \vl(r) \,r^2 \sin\tht \ddphi \dd\tht
		= \frac{i k^3}{16\pi^2 \epso} \paren{ \frac{1}{k^2 r^2} + 1 } \intopi \intotp (\nh \vdot \vp) (\nh \cross \vps) \sin\tht \ddphi \dd\tht. \label{dLdt}
	}
	Note that
	\eq{
		[ (\nh \vdot \vp) (\nh \cross \vps) ]_i = \sumje n_j p_j (\nh \cross \vps)_i
		= \sumje \sumke \sumle \epsikl n_j p_j n_k p_l^*,
	}
	so
	\eq{
		\dv{L_i}{t} \propto \sumje \sumke \sumle \epsikl p_j p_l^* \int n_j p_k \ddOmg
		= \sumje \sumke \sumle \epsikl p_j p_l^* \frac{4\pi}{3} \del_{jk}
		= \frac{4\pi}{3} \epsikl p_k p_l^*
		= \frac{4\pi}{3} (\vp \cross \vps)_i,
	}
	where we have used Jackson~(9.47), $\int n_\bet n_\gam \ddOmg = 4\pi \del_{\bet \gam} / 3$.  Making this substitution into Eq.~\refeq{dLdt},
	\eq{
		\dv{\vL}{t} = \frac{i k^3}{6\pi \epso} \paren{ \frac{1}{k^2 r^2} + 1 } (\vp \cross \vps).
	}
	Taking the limit as $r \to \infty$, we find
	\eqn{ans1a}{
		\dv{\vL}{t} = \Re\!\brac{ \frac{i k^3}{12\pi \epso} (\vp \cross \vps) }
		= \Re\!\brac{ -\frac{i k^3}{12\pi \epso} (\vps \cross \vp) }
		= \ans{ \frac{k^3}{12\pi \epso} \Im[ \vps \cross \vp ], }
	}
	as desired. \qed
}

%
%	Jackson 9.8(b)
%

\prob{}{
	What is the ratio of angular momentum radiated to energy radiated?  Interpret.
}

\sol{
	According to Jackson~(9.24), the total power radiated by an oscillating electric dipole $\vp$ is
	\eq{
		P = \dv{E}{t}
		= \frac{c^2 \Zo k^4}{12 \pi} \abs{\vp}^2.
	}
	Then the ratio of angular momentum radiated to energy radiated is
	\eq{
		\frac{\dv*{\vL}{t}}{\dv*{E}{t}} = \frac{k^3}{12\pi \epso} \Im[ \vps \cross \vp ] \frac{12 \pi}{c^2 \Zo k^4 \abs{\vp}^2}
		= \frac{1}{\epso} \Im[ \vps \cross \vp ] \frac{1}{c^2 \Zo k \abs{\vp}^2}
		= \ans{ \frac{\Im[ \vps \cross \vp ]}{\omg \abs{\vp}^2}, }
	}
	where we have used $\Zo = \sqrt{\muo / \epso} = 1 / \sqrt{\epso^2 c^2} = 1 / \epso c$, $c^2 = 1 / (\epso \muo)$, and $\omg = k c$.
	
	In the limit of high frequency, $(\dv*{\vL}{t}) / (\dv*{E}{t}) \to 0$.  In this scenario, the energy radiated dominates over the angular momentum radiated.  Likewise, in the limit of low frequency, $(\dv*{\vL}{t}) / (\dv*{E}{t}) \to \infty$, meaning that angular momentum radiation dominates.  This is sensible because rotational kinetic energy $E \propto \omg^2$, while angular momentum $L \propto \omg$.
}

%
%	Jackson 9.8(c)
%

\prob{}{
	For a charge $e$ rotating in the $xy$ plane at radius $a$ and angular speed $\omg$, show that there is only a $z$ component of radiated angular momentum with magnitude $\dv*{\Lz}{t} = e^2 k^3 a^2 / 6 \pi \epso$.  What about a charge oscillating along the $z$ axis?
}

\sol{
	We know from Homework~5 that the position of a point charge rotating counterclockwise in the $xy$ plane is
	\eq{
		\vx(t) = a \cos(\omg t) \,\vx + a \sin(\omg t) \,\yh.
	}
	\clearpage
	Then the charge distribution is
	\eq{
		\rho(\vx, t) = e \del[ x - a \cos(\omg t) ] \,\del[ y - a \sin(\omg t) ] \,\del(z).
	}
	
	According to Jackson~(4.8), the dipole moment is defined
	\eq{
		\vp = \int \vx' \,\rho(\vx') \ddcxp.
	}
	The components of $\vp$ for the point charge are then
	\al{
		\px &= e \iiint x \,\del[ x - a \cos(\omg t) ] \,\del[ y - a \sin(\omg t) ] \,\del(z) \ddx \ddy \ddz
		= e a \cos(\omg t), \\
		\py &= e \iiint y \,\del[ x - a \cos(\omg t) ] \,\del[ y - a \sin(\omg t) ] \,\del(z) \ddx \ddy \ddz
		= e a \sin(\omg t), \\
		\pz &= e \iiint z \,\del[ x - a \cos(\omg t) ] \,\del[ y - a \sin(\omg t) ] \,\del(z) \ddx \ddy \ddz
		= 0,
	}
	so we can write $\vp = e a \,e^{-i \omg t} (\xh + i\,\yh).$  Substituting into Eq.~\refeq{ans1a},
	\al{
		\dv{\vL}{t} &= \Re\!\brac{ \frac{i k^3}{12\pi \epso} e^2 a^2 e^{-i \omg t} e^{i \omg t} [ (\xh + i\,\yh) \cross (\xh - i\,\yh) ] }
		= \Re\!\brac{ \frac{i e^2 k^3 a^2}{12\pi \epso} (-2i \,\xh \cross \yh) }
		= \Re\!\brac{ \frac{e^2 k^3 a^2}{6\pi \epso} \,\zh } \\
		&= \ans{ \frac{e^2 k^3 a^2}{6\pi \epso} \cos(\omg t) \,\zh, }
	}
	as desired. \qed
	
	A charge oscillating along the $z$ axis with amplitude $a$ has the charge density
	\eq{
		\rho(\vx, t) = e a \,\del(x) \,\del(y) \,\del[ z - \cos(\omg t) ],
	}
	which gives the dipole moment
	\al{
		\px &= e a \iiint x \,\del(x) \,\del(y) \,\del[ z - \cos(\omg t) ] \ddx \ddy \ddz
		= 0, \\
		\py &= e a \iiint y \,\del(x) \,\del(y) \,\del[ z - \cos(\omg t) ] \ddx \ddy \ddz
		= 0, \\
		\pz &= e a \iiint z \,\del(x) \,\del(y) \,\del[ z - \cos(\omg t) ] \ddx \ddy \ddz
		= e a \cos(\omg t).
	}
	In complex notation, $\vp = e a \,e^{-i\omg t} \,\zh$.  Substituting into Eq.~\refeq{ans1a}, we find
	\eq{
		\dv{\vL}{t} = \Re\!\brac{ \frac{i k^3}{12\pi \epso} e^2 a^2 e^{-i \omg t} e^{i \omg t} (\zh \cross \zh) }
		= \ans{ \vO. }
	}
	So we see that a charge undergoing linear motion does not lead to a radiated angular momentum, which is sensible.
}

%
%	Jackson 9.8(d)
%

\prob{}{
	What are the results corresponding to Probs.~{1(a)} and {1(b)} for magnetic dipole radiation?
}

\sol{
	The radiation fields for a magnetic dipole are given by Jackson~(19.35--36),
	\al{
		\vH &= \frac{1}{4\pi} \curly{ k^2 (\nh \cross \vm) \cross \nh \frac{e^{i k r}}{r} + [ 3 \nh (\nh \vdot \vm) - \vm ] \paren{ \frac{1}{r^3} - \frac{i k}{r^2} } e^{i k r} }, &
		\vE &= -\frac{\Zo}{4\pi} k^2 (\nh \cross \vm) \frac{e^{i k r}}{r} \paren{ 1 - \frac{1}{i k r} }.
	}
	\clearpage
	Comparing with Eq.~\refeq{fields1}, we see that $\vH \to -\vE / \Zo$, $\vE \to \Zo \vH$, and $\vp \to \vm / c$ as stated in the book~\cite[p.~413]{Jackson}.  Making these substitutions, the results of Probs.~{1.1(a)} and {(b)} become
	\al{
		\ans{ \dv{\vL}{t}\ }&\ans{= \frac{\muo k^3}{12\pi} \Im[ \vms \cross \vm ], } &
		\ans{ \frac{\dv*{\vL}{t}}{\dv*{E}{t}}\ }&\ans{= \frac{\Im[ \vms \cross \vm ]}{\omg \abs{\vm}^2} }
	}
	where we have used $\mu = 1 / \epso c^2$.
}



\state{(Jackson 10.1)}{\ }

%
%	Jackson 10.1(a)
%

\prob{}{
	Show that for arbitrary initial polarization, the scattering cross section of a perfectly conducting sphere of radius $a$, summed over outgoing polarizations, is given in the long-wavelength limit by
	\eq{
		\dv{\sig}{\Omg}(\vepso, \nho, \nh) = k^4 a^6 \brac{ \frac{5}{4} - \abs{\vepso \vdot \nh}^2 - \frac{1}{4} \abs{\nh \vdot (\nho \cross \vepso)}^2 - \nho \vdot \nh },
	}
	where $\nho$ and $\nh$ are the directions of the incident and scattered radiations, respectively, while $\vepso$ is the (perhaps complex) unit polarization vector of the incident radiation ($\vepso^* \vdot \vepso = 1$; $\nho \vdot \vepso = 0$).
}

\sol{
	Jackson~(10.14) gives the differential cross section for scattering off a small, perfectly conducting sphere with initial polarization $\vepso$ and outgoing polarization $\veps$:
	\eqn{thing2}{
		\dv{\sig}{\Omg} {\nh, \veps; \nho, \vepso)} = k^4 a^6 \abs{ (\vepss \vdot \vepso - \frac{1}{2} (\nh \cross \vepss) \vdot (\nho \cross \vepso) }^2.
	}
	We will use the polarization vectors $\vepsq$ and $\vepsw$, which are defined in Fig.~\refeq{pol}~\cite[p.~458]{Jackson}.  According to the figure,
	\al{
		\vepsw &= \frac{\nh \cross \nho}{\abs{\nh \cross \nho}}
		= \frac{\nh \cross \nho}{\sqrt{1 - (\nh \vdot \nho)^2}}, \\
		\vepsq &= \vepsw \cross \nh
		= \frac{-\nh \cross (\nh \cross \nho)}{\sin\tht}
		= \frac{(\nh \vdot \nh) \nho - (\nh \vdot \nho) \nh}{\sin\tht}
		= \frac{\nho - (\nh \vdot \nho) \nh}{\sqrt{1 - (\nh \vdot \nho)^2}},
	}
	which are both real.  In the denominator, we have used $\sin^2\tht = 1 + \cos^2\tht = 1 + (\nh \vdot \nho)^2$.  We also note that $\nho$, $\nh$, and $\vepsq$ are in the same plane, and that $\nh \perp \vepsq$.

	The cross section summed over outgoing polarizations is then found by plugging $\veps = \vepsq$ and $\veps = \vepsw$ into Eq.~\refeq{thing2}, and taking the sum.  For the first term,
	\al{
		\dv{\sig}{\Omg} {(\nh, \vepsq; \nho, \vepso)} &= k^4 a^6 \abs{ {\vepsq}^* \vdot \vepso - \frac{1}{2} (\nh \cross {\vepsq}^*) \vdot (\nho \cross \vepso) }^2 \\
		&= k^4 a^6 \abs{ \frac{\nho - (\nh \vdot \nho) \nh}{\sqrt{1 - (\nh \vdot \nho)^2}} \vdot \vepso - \frac{1}{2} \paren{ \nh \cross \frac{\nho - (\nh \vdot \nho) \nh}{\sqrt{1 - (\nh \vdot \nho)^2}} } \vdot (\nho \cross \vepso) }^2 \\
		&= \frac{k^4 a^6}{1 - (\nh \vdot \nho)^2} \abs{ -(\nh \vdot \nho) (\nh \vdot \vepso) - \frac{1}{2} (\nh \cross \nho) \vdot (\nho \cross \vepso) }^2.
	}
	One of the vector identities on the inside cover of Jackson is $(\vaa \cross \vbb) \vdot (\vcc \cross \vdd) = (\vaa \vdot \vcc) (\vbb \vdot \vdd) - (\vaa \vdot \vdd) (\vbb \vdot \vcc)$.  Applying this, we have
	\al{
		\dv{\sig}{\Omg} {(\nh, \vepsq; \nho, \vepso)} &= \frac{k^4 a^6}{1 - (\nh \vdot \nho)^2} \abs{ (\nh \vdot \nho) (\nh \vdot \vepso) + \frac{1}{2} (\nh \vdot \nho) (\nho \vdot \vepso) - \frac{1}{2} (\nh \vdot \vepso) (\nho \vdot \nho) }^2 \\
		&= \frac{k^4 a^6}{1 - (\nh \vdot \nho)^2} \abs{ (\nh \vdot \nho) (\nh \vdot \vepso) - \frac{1}{2} (\nh \vdot \vepso) }^2
		= \frac{k^4 a^6}{1 - (\nh \vdot \nho)^2} \abs{ \nh \vdot \vepso }^2 \brac{ (\nh \vdot \nho) - \frac{1}{2} }^2 \\
		&= \frac{k^4 a^6}{1 - (\nh \vdot \nho)^2} \abs{ \nh \vdot \vepso }^2 \brac{ (\nh \vdot \nho)^2 - (\nh \vdot \nho) + \frac{1}{4} }.
	}
	
	For the second term,
	\al{
		\dv{\sig}{\Omg} {(\nh, \vepsw; \nho, \vepso)} &= k^4 a^6 \abs{ {\vepsw}^* \vdot \vepso - \frac{1}{2} (\nh \cross {\vepsw}^*) \vdot (\nho \cross \vepso) }^2 \\
		&= k^4 a^6 \abs{ \frac{\nh \cross \nho}{\sqrt{1 - (\nh \vdot \nho)^2}} \vdot \vepso - \frac{1}{2} \paren{ \nh \cross \frac{\nh \cross \nho}{\sqrt{1 - (\nh \vdot \nho)^2}} } \vdot (\nho \cross \vepso) }^2 \\
		&= \frac{k^4 a^6}{1 - (\nh \vdot \nho)^2} \abs{ \nh \vdot (\nho \cross \vepso) - \frac{1}{2} [ (\nh \vdot \nho) \nh - (\nh \vdot \nh) \nho ] \vdot (\nho \cross \vepso) }^2 \\
		&= \frac{k^4 a^6}{1 - (\nh \vdot \nho)^2} \abs{ \nh \vdot (\nho \cross \vepso) - \frac{1}{2} [ (\nh \vdot \nho) \nh - \nho ] \vdot (\nho \cross \vepso) }^2 \\
		&= \frac{k^4 a^6}{1 - (\nh \vdot \nho)^2} \abs{ \nh \vdot (\nho \cross \vepso) - \frac{1}{2} (\nh \vdot \nho) \nh \vdot (\nho \cross \vepso) + \frac{1}{2} \epso \vdot (\nho \cross \nho) }^2 \\
		&= \frac{k^4 a^6}{1 - (\nh \vdot \nho)^2} \abs{ \paren{ 1 - \frac{1}{2} (\nh \vdot \nho) } \nh \vdot (\nho \cross \vepso) }^2 \\
		&= \frac{k^4 a^6}{1 - (\nh \vdot \nho)^2} \brac{ 1 - (\nh \vdot \nho) + \frac{1}{4} (\nh \vdot \nho)^2 } \abs{ \nh \vdot (\nho \cross \vepso) }^2.
	}
	
	Summing the two terms, we find
	\aln{
		\dv{\sig}{\Omg} &= \dv{\sig}{\Omg} {(\nh, \vepsq; \nho, \vepso)} + \dv{\sig}{\Omg} {(\nh, \vepsw; \nho, \vepso)} \notag \\
		&= \frac{k^4 a^6}{1 - (\nh \vdot \nho)^2} \curly{ \abs{ \nh \vdot \vepso }^2 \brac{ (\nh \vdot \nho)^2 - (\nh \vdot \nho) + \frac{1}{4} } + \brac{ 1 - (\nh \vdot \nho) + \frac{1}{4} (\nh \vdot \nho)^2 } \abs{ \nh \vdot (\nho \cross \vepso) }^2 } \notag \\
		&= \frac{k^4 a^6}{1 - (\nh \vdot \nho)^2} \bigg\{ \abs{ \nh \vdot \vepso }^2 (\nh \vdot \nho)^2 - \abs{ \nh \vdot \vepso }^2 (\nh \vdot \nho) + \frac{\abs{ \nh \vdot \vepso }^2}{4} + \abs{ \nh \vdot (\nho \cross \vepso) }^2 \notag \\
		&\phantom{mmmmmmmmmmmmmmmmmmmmmmmmmm} - (\nh \vdot \nho) \abs{ \nh \vdot (\nho \cross \vepso) }^2 + \frac{(\nh \vdot \nho)^2 \abs{ \nh \vdot (\nho \cross \vepso) }^2}{4} \bigg\} \notag \\
		&= \frac{k^4 a^6}{1 - (\nh \vdot \nho)^2} \bigg\{ \frac{5 \abs{ \nh \vdot (\nho \cross \vepso) }^2}{4} + \frac{5 \abs{ \nh \vdot \vepso }^2}{4} - (\nh \vdot \nho) \abs{ \nh \vdot (\nho \cross \vepso) }^2 - (\nh \vdot \nho) \abs{ \nh \vdot \vepso }^2 \notag \\
		&\phantom{mmmmmmmmmmmmmmm} - \frac{\abs{ \nh \vdot (\nho \cross \vepso) }^2}{4} - \frac{\abs{ \nh \vdot \vepso }^2}{4} + \frac{(\nh \vdot \nho)^2 \abs{ \nh \vdot (\nho \cross \vepso) }^2}{4} + (\nh \vdot \nho)^2 \abs{ \nh \vdot \vepso }^2 \bigg\} \notag \\
		&= \frac{k^4 a^6}{1 - (\nh \vdot \nho)^2} \curly{ \brac{ \frac{5}{4} - \nh \vdot \nho } \brac{ \abs{ \nh \vdot (\nho \cross \vepso) }^2 + \abs{ \nh \vdot \vepso }^2 } - \brac{ 1 - (\nh \vdot \nho)^2 } \brac{ \frac{\abs{ \nh \vdot (\nho \cross \vepso) }^2}{4} + \abs{ \nh \vdot \vepso }^2 } } \notag \\
		&= \frac{k^4 a^6}{1 - (\nh \vdot \nho)^2} \brac{ \frac{5}{4} - \nh \vdot \nho } \brac{ \abs{ \nh \vdot (\nho \cross \vepso) }^2 + \abs{ \nh \vdot \vepso }^2 } - k^4 a^6 \brac{ \frac{\abs{ \nh \vdot (\nho \cross \vepso) }^2}{4} + \abs{ \nh \vdot \vepso }^2 }. \label{thing2.1}
	}
	
	Since $\nho \vdot \vepso = 0$, we note that
	\eq{
		\nh = (\nh \vdot \nho) \,\nho + (\nh \vdot \epso) \,\vepso + [ \nh \vdot (\nho \cross \vepso) ] \,(\nho \cross \vepso)
		\qimplies
		1 = (\nh \vdot \nho)^2 + \abs{ \nh \vdot \epso }^2 + \abs{ \nh \vdot (\nho \vdot \vepso) }^2.
	}
	Substituting into Eq.~\refeq{thing2.1},
	\al{
		\dv{\sig}{\Omg} &= \frac{k^4 a^6}{1 - (\nh \vdot \nho)^2} \brac{ \frac{5}{4} - \nh \vdot \nho } \brac{ 1 - (\nh \vdot \nho)^2 } - k^4 a^6 \brac{ \frac{1}{4} \abs{ \nh \vdot (\nho \cross \vepso) }^2 + \abs{ \nh \vdot \vepso }^2 } \\
		&= \ans{ k^4 a^6 \brac{ \frac{5}{4} - \abs{ \nh \vdot \vepso }^2 - \frac{1}{4} \abs{ \nh \vdot (\nho \cross \vepso) }^2 - \nh \vdot \nho }, }
	}
	as we sought to prove. \qed
}

%
%	Jackson 10.1(b)

\prob{}{
	If the incident radiation is linearly polarized, show that the cross section is
	\eq{
		\dv{\sig}{\Omg}(\vepso, \nho, \nh) = k^4 a^6 \brac{ \frac{5}{8} (1 + \cos^2\tht) - \cos\tht - \frac{3}{8} \sin^2\tht \cos(2\phi) },
	}
	where $\nh \vdot \nho = \cos\tht$ and the azimuthal angle $\phi$ is measured from the direction of linear polarization.
}

\sol{
	We choose coordinates as in Fig.~\ref{pol}, such that the direction of linear polarization $\vepso$ points along the $x$ axis and $\nho$ points along the $z$ axis.  Then $\nho \cross \vepso$ points along the $y$ axis.
}

%
%	Jackson 10.1(c)
%
\clearpage
\prob{}{
	What is the ratio of scattered intensities at $\tht = \pi / 2$, $\phi = 0$ and $\tht = \pi / 2$, $\phi = \pi / 2$?  Explain physically in terms of the induced multipoles and their radiation patterns.
}




\state{(Jackson 12.15)}{
	Consider the Proca equation for a localized steady-state distribution of current that has only a static magnetic moment.  This model can be used to study the observable effects of a finite photon mass on the earth's magnetic field.  Note that if the magnetization is $\vcM(\vx)$ the current density can be written as $\vJ = c (\grad \cross \vcM)$.
}

%
%	Jackson 12.15(a)
%

\prob{}{
	Show that if $\vcM = \vm f(\vx)$, where $\vm$ is a fixed vector and $f(\vx)$ is a localized scalar function, the vector potential is
	\eq{
		\vA(\vx) = -\vm \cross \grad \int f(\vx') \frac{e^{-\mu \abs{\vx - \vx'}}}{\abs{\vx - \vx'}} \ddcxp.
	}
	\vfix
}

%
%	Jackson 12.15(b)
%

\prob{}{
	If the magnetic dipole is a point dipole at the origin [$f(\vx) = \del(\vx)$], show that the magnetic field away from the origin is
	\eq{
		\vB(\vx) = [ 3 \,\rh (\rh \vdot \vm) - \vm ] \paren{ 1 + \mu r + \frac{\mu^2 r^2}{3} }\frac{e^{-\mu r}}{r^3} - \frac{2}{3} \mu^2 \vm \frac{e^{-\mu r}}{r}.
	}
	\vfix
}

%
%	Jackson 12.15(c)
%

\prob{}{
	The result of Prob.~{3(b)} shows that at fixed $r = R$ (on the surface of the earth), the earth's magnetic field will appear as a dipole angular distribution, plus an added constant magnetic field (an apparently external field) antiparallel to $\vm$.  Satellite and surface observations lead to the conclusion that the ``external'' field is less than $\num{4e-3}$ times the dipole field at the magnetic equator.  Estimate a lower limit on $\mu^{-1}$ in earth radii and an upper limit on the photon mass in grams from this datum.
}


\makebib

\end{document}