\documentclass[11pt]{article}
\usepackage{homework}

\classname{323}
\homeworknum{6}



\begin{document}

% Environments

\newcommand{\state}[2]{\begin{statement}{#1} #2 \end{statement}}
\newcommand{\prob}[2]{\begin{problem}{#1} #2 \end{problem}}
\newcommand{\subprob}[1]{\begin{subproblem} #1 \end{subproblem}}
\newcommand{\sol}[1]{\begin{solution} #1 \end{solution}}
\newcommand{\fig}[2]{\begin{figure} \centering #2  \label{#1} \end{figure}}

\newcommand{\makebib}{
	\vfill
	\color{black}
	\bibliography{references}{}
	\bibliographystyle{lucas_unsrt}
}
	

% Implication

\newcommand{\qwhere}{\quad \text{where} \quad}
\newcommand{\qimplies}{\quad \implies \quad}
\newcommand{\impliesq}{\implies \quad}



% Brackets

\newcommand{\paren}[1]{\left( #1 \right)}
\newcommand{\brac}[1]{\left[ #1 \right]}


% Greek

\newcommand{\alp}{\alpha}
\newcommand{\bet}{\beta}
\newcommand{\gam}{\gamma}
\newcommand{\del}{\delta}
\newcommand{\eps}{\epsilon}
\newcommand{\zet}{\zeta}
\newcommand{\tht}{\theta}
\newcommand{\kap}{\kappa}
\newcommand{\lam}{\lambda}
\newcommand{\sig}{\sigma}
\newcommand{\ups}{\upsilon}
\newcommand{\omg}{\omega}

\newcommand{\Gam}{\Gamma}
\newcommand{\Del}{\Delta}
\newcommand{\Tht}{\Theta}
\newcommand{\Lam}{\Lambda}
\newcommand{\Sig}{\Sigma}
\newcommand{\Omg}{\Omega}
% Problem 1

\newcommand{\Psii}{\Psi^i}
\newcommand{\Psiix}{\Psii(x)}

\newcommand{\Pii}{\Pi^i}

\newcommand{\Phii}{\Phi^i}
\newcommand{\Phiix}{\Phii(x)}
\newcommand{\PhiN}{\Phi^N}
\newcommand{\PhiNx}{\PhiN(x)}
\newcommand{\Phiq}{\Phi^1}
\newcommand{\Phiw}{\Phi^2}

\newcommand{\ddcx}{\dd[3]{x}}

\newcommand{\delij}{\del^{i j}}
\newcommand{\delkl}{\del^{k l}}
\newcommand{\delil}{\del^{i l}}
\newcommand{\deljk}{\del^{j k}}
\newcommand{\delik}{\del^{i k}}
\newcommand{\deljl}{\del^{j l}}

\newcommand{\DF}{D_F}

\newcommand{\sigx}{\sig(x)}

\newcommand{\pii}{\pi^i}
\newcommand{\pij}{\pi^j}
\newcommand{\pik}{\pi^k}
\newcommand{\pil}{\pi^l}
\newcommand{\piix}{\pi(x)}

\newcommand{\pq}{p_1}
\newcommand{\pw}{p_2}
\newcommand{\pe}{p_3}
\newcommand{\pr}{p_4}

\newcommand{\vp}{\vb{p}}
\newcommand{\vpsi}{\vp_i}

\newcommand{\mpi}{m_\pi}

\state{(Jackson 9.8)}{\ 
	%\emph{Hint:} The electromagnetic angular momentum density comes from more than the transverse (radiation zone) components of the fields.
}

%
%	Jackson 9.8(a)
%

\prob{}{
	Show that a classical oscillating electric dipole $\vp$ with fields given by
	\aln{ \label{fields1}
		\vH &= \frac{c k^2}{4\pi} (\nh \cross \vp) \frac{e^{i k r}}{r} \paren{ 1 - \frac{1}{i k r} }, &
		\vE &= \frac{1}{4\pi \epso} \curly{ k^2 (\nh \cross \vp) \cross \nh \frac{e^{i k r}}{r} + [ 3 \nh (\nh \vdot \vp) - \vp ] \paren{ \frac{1}{r^3} - \frac{i k}{r^2} } e^{i k r} },
	}
	radiates electromagnetic angular momentum to infinity at the rate
	\eq{
		\dv{\vL}{t} = \frac{k^3}{12 \pi \epso} \Im[ \vp^* \cross \vp ].
	}
	\vfix
}

\sol{
	According to Jackson~(9.20), the time-averaged angular momentum density is
	\eq{
		\vl = \frac{\Re[ \vx \cross (\vE \cross \vHs)}{2 c^2}.
	}
	One of the vector identities on the inside cover of Jackson is $\vaa \cross (\vbb \cross \vcc) = (\vaa \vdot \vcc) \vbb - (\vaa \vdot \vbb) \vcc$, so
	\eqn{l1}{
		\vl = \frac{(\vx \vdot \vHs) \vE - (\vx \vdot \vE) \vHs}{2 c^2}.
	}
	From Eq.~\refeq{fields1}, note that
	\eq{
		\vx \vdot \vHs \propto \vx \vdot (\nh \cross \vps)
		= \vps \vdot (\vx \cross \nh)
		= \vO,
	}
	where we have used the identity $\vaa \vdot (\vbb \cross \vcc) = \vcc \vdot (\vaa \cross \vbb)$ and the fact that $\nh$ points in the $\vx$ direction.  For $\vx \vdot \vE$, note that
	\al{
		\vx \vdot [ (\nh \cross \vp) \cross \nh ] &= -\vx \vdot [ \nh \cross (\nh \cross \vp) ]
		= -\vx \vdot [ (\nh \vdot \vp) \nh - (\nh \vdot \nh) \vp ]
		= -(\nh \vdot \vp) (\vx \vdot \nh) + \vx \vdot \vp \\
		&= -r (\nh \vdot \vp) + \vx \vdot \vp
		= \vx \vdot \vp - \vx \vdot \vp
		= 0, \\[1.5ex]
		\vx \vdot [ 3 \nh (\nh \vdot \vp) - \vp ] &= 3 (\vx \vdot \nh) (\nh \vdot \vp) - \vx \vdot \vp
		= 3r (\nh \vdot \vp) - \vx \vdot \vp
		= 3(\vx \vdot \vp) - \vx \vdot \vp
		= 2(\vx \vdot \vp),
	}
	since $\abs{\vx} = r$ and $\vx = r \,\nh$.  Then
	\eq{
		\vx \vdot \vE = \frac{1}{2\pi \epso} (\vx \vdot \vp) \paren{ \frac{1}{r^3} - \frac{i k}{r^2} } e^{i k r}
		= \frac{1}{2\pi \epso} (\nh \vdot \vp) \paren{ \frac{1}{r^2} - \frac{i k}{r} } e^{i k r}.
	}
	
	With these substitutions, Eq.~\refeq{l1} becomes
	\al{
		\vl &= -\frac{(\vx \vdot \vE) \vHs}{c^2}
		= -\frac{1}{4\pi \epso c^2} (\nh \vdot \vp) \paren{ \frac{1}{r^2} - \frac{i k}{r} } e^{i k r} \frac{c k^2}{4\pi} (\nh \cross \vps) \frac{e^{-i k r}}{r} \paren{ 1 + \frac{1}{i k r} } \\
		&= -\frac{k^2}{16\pi^2 \epso c r} (\nh \vdot \vp) (\nh \cross \vps) \paren{ \frac{1}{r^2} - \frac{i k}{r} } \paren{ 1 - \frac{i}{k r} }
		= -\frac{k^2}{16\pi^2 \epso c} (\nh \vdot \vp) (\nh \cross \vps) \paren{ \frac{1}{r^2} - \frac{i}{k r^3} - \frac{i k}{r} - \frac{1}{r^2} } \\
		&= -\frac{i k^2}{16\pi^2 \epso c r} (\nh \vdot \vp) (\nh \cross \vps) \paren{ \frac{1}{k r^3} + \frac{k}{r^2} }
		= \frac{i k^3}{16\pi^2 \epso c r^2} (\nh \vdot \vp) (\nh \cross \vps) \paren{ \frac{1}{k^2 r^2} + 1 }.
	}
	
	Let $\vL$ be the angular momentum radiated to a distance $R$.  Then
	\eq{
		\vL = \int_R \vl(r) \ddcx
		= \intopi \intotp \intoR \vl(r) \,r^2 \sin\tht \ddr \ddphi \dd\tht,
	}
	and the time derivative is
	\aln{
		\dv{\vL}{t} &= \dv{t}(\intopi \intotp \intoR \vl(r) \,r^2 \sin\tht \ddr \ddphi \dd\tht)
		= \dv{r}{t} \dv{r}(\intopi \intotp \intoR \vl(r) \,r^2 \sin\tht \ddr \ddphi \dd\tht) \notag \\
		&= c \intopi \intotp \vl(r) \,r^2 \sin\tht \ddphi \dd\tht
		= \frac{i k^3}{16\pi^2 \epso} \paren{ \frac{1}{k^2 r^2} + 1 } \intopi \intotp (\nh \vdot \vp) (\nh \cross \vps) \sin\tht \ddphi \dd\tht. \label{dLdt}
	}
	Note that
	\eq{
		[ (\nh \vdot \vp) (\nh \cross \vps) ]_i = \sumje n_j p_j (\nh \cross \vps)_i
		= \sumje \sumke \sumle \epsikl n_j p_j n_k p_l^*,
	}
	so
	\eq{
		\dv{L_i}{t} \propto \sumje \sumke \sumle \epsikl p_j p_l^* \int n_j p_k \ddOmg
		= \sumje \sumke \sumle \epsikl p_j p_l^* \frac{4\pi}{3} \del_{jk}
		= \frac{4\pi}{3} \epsikl p_k p_l^*
		= \frac{4\pi}{3} (\vp \cross \vps)_i,
	}
	where we have used Jackson~(9.47), $\int n_\bet n_\gam \ddOmg = 4\pi \del_{\bet \gam} / 3$.  Making this substitution into Eq.~\refeq{dLdt},
	\eq{
		\dv{\vL}{t} = \frac{i k^3}{6\pi \epso} \paren{ \frac{1}{k^2 r^2} + 1 } (\vp \cross \vps).
	}
	Taking the limit as $r \to \infty$, we find
	\eqn{ans1a}{
		\dv{\vL}{t} = \Re\!\brac{ \frac{i k^3}{12\pi \epso} (\vp \cross \vps) }
		= \Re\!\brac{ -\frac{i k^3}{12\pi \epso} (\vps \cross \vp) }
		= \ans{ \frac{k^3}{12\pi \epso} \Im[ \vps \cross \vp ], }
	}
	as desired. \qed
}

%
%	Jackson 9.8(b)
%

\prob{}{
	What is the ratio of angular momentum radiated to energy radiated?  Interpret.
}

\sol{
	According to Jackson~(9.24), the total power radiated by an oscillating electric dipole $\vp$ is
	\eq{
		P = \dv{E}{t}
		= \frac{c^2 \Zo k^4}{12 \pi} \abs{\vp}^2.
	}
	Then the ratio of angular momentum radiated to energy radiated is
	\eq{
		\frac{\dv*{\vL}{t}}{\dv*{E}{t}} = \frac{k^3}{12\pi \epso} \Im[ \vps \cross \vp ] \frac{12 \pi}{c^2 \Zo k^4 \abs{\vp}^2}
		= \frac{1}{\epso} \Im[ \vps \cross \vp ] \frac{1}{c^2 \Zo k \abs{\vp}^2}
		= \ans{ \frac{\Im[ \vps \cross \vp ]}{\omg \abs{\vp}^2}, }
	}
	where we have used $\Zo = \sqrt{\muo / \epso} = 1 / \sqrt{\epso^2 c^2} = 1 / \epso c$, $c^2 = 1 / (\epso \muo)$, and $\omg = k c$.
	
	In the limit of high frequency, $(\dv*{\vL}{t}) / (\dv*{E}{t}) \to 0$.  In this scenario, the energy radiated dominates over the angular momentum radiated.  Likewise, in the limit of low frequency, $(\dv*{\vL}{t}) / (\dv*{E}{t}) \to \infty$, meaning that angular momentum radiation dominates.  This is sensible because rotational kinetic energy $E \propto \omg^2$, while angular momentum $L \propto \omg$.
}

%
%	Jackson 9.8(c)
%

\prob{}{
	For a charge $e$ rotating in the $xy$ plane at radius $a$ and angular speed $\omg$, show that there is only a $z$ component of radiated angular momentum with magnitude $\dv*{\Lz}{t} = e^2 k^3 a^2 / 6 \pi \epso$.  What about a charge oscillating along the $z$ axis?
}

\sol{
	We know from Homework~5 that the position of a point charge rotating counterclockwise in the $xy$ plane is
	\eq{
		\vx(t) = a \cos(\omg t) \,\vx + a \sin(\omg t) \,\yh.
	}
	\clearpage
	Then the charge distribution is
	\eq{
		\rho(\vx, t) = e \del[ x - a \cos(\omg t) ] \,\del[ y - a \sin(\omg t) ] \,\del(z).
	}
	
	According to Jackson~(4.8), the dipole moment is defined
	\eq{
		\vp = \int \vx' \,\rho(\vx') \ddcxp.
	}
	The components of $\vp$ for the point charge are then
	\al{
		\px &= e \iiint x \,\del[ x - a \cos(\omg t) ] \,\del[ y - a \sin(\omg t) ] \,\del(z) \ddx \ddy \ddz
		= e a \cos(\omg t), \\
		\py &= e \iiint y \,\del[ x - a \cos(\omg t) ] \,\del[ y - a \sin(\omg t) ] \,\del(z) \ddx \ddy \ddz
		= e a \sin(\omg t), \\
		\pz &= e \iiint z \,\del[ x - a \cos(\omg t) ] \,\del[ y - a \sin(\omg t) ] \,\del(z) \ddx \ddy \ddz
		= 0,
	}
	so we can write $\vp = e a \,e^{-i \omg t} (\xh + i\,\yh).$  Substituting into Eq.~\refeq{ans1a},
	\al{
		\dv{\vL}{t} &= \Re\!\brac{ \frac{i k^3}{12\pi \epso} e^2 a^2 e^{-i \omg t} e^{i \omg t} [ (\xh + i\,\yh) \cross (\xh - i\,\yh) ] }
		= \Re\!\brac{ \frac{i e^2 k^3 a^2}{12\pi \epso} (-2i \,\xh \cross \yh) }
		= \Re\!\brac{ \frac{e^2 k^3 a^2}{6\pi \epso} \,\zh } \\
		&= \ans{ \frac{e^2 k^3 a^2}{6\pi \epso} \cos(\omg t) \,\zh, }
	}
	as desired. \qed
	
	A charge oscillating along the $z$ axis with amplitude $a$ has the charge density
	\eq{
		\rho(\vx, t) = e a \,\del(x) \,\del(y) \,\del[ z - \cos(\omg t) ],
	}
	which gives the dipole moment
	\al{
		\px &= e a \iiint x \,\del(x) \,\del(y) \,\del[ z - \cos(\omg t) ] \ddx \ddy \ddz
		= 0, \\
		\py &= e a \iiint y \,\del(x) \,\del(y) \,\del[ z - \cos(\omg t) ] \ddx \ddy \ddz
		= 0, \\
		\pz &= e a \iiint z \,\del(x) \,\del(y) \,\del[ z - \cos(\omg t) ] \ddx \ddy \ddz
		= e a \cos(\omg t).
	}
	In complex notation, $\vp = e a \,e^{-i\omg t} \,\zh$.  Substituting into Eq.~\refeq{ans1a}, we find
	\eq{
		\dv{\vL}{t} = \Re\!\brac{ \frac{i k^3}{12\pi \epso} e^2 a^2 e^{-i \omg t} e^{i \omg t} (\zh \cross \zh) }
		= \ans{ \vO. }
	}
	So we see that a charge undergoing linear motion does not lead to a radiated angular momentum, which is sensible.
}

%
%	Jackson 9.8(d)
%

\prob{}{
	What are the results corresponding to Probs.~{1(a)} and {1(b)} for magnetic dipole radiation?
}

\sol{
	The radiation fields for a magnetic dipole are given by Jackson~(19.35--36),
	\al{
		\vH &= \frac{1}{4\pi} \curly{ k^2 (\nh \cross \vm) \cross \nh \frac{e^{i k r}}{r} + [ 3 \nh (\nh \vdot \vm) - \vm ] \paren{ \frac{1}{r^3} - \frac{i k}{r^2} } e^{i k r} }, &
		\vE &= -\frac{\Zo}{4\pi} k^2 (\nh \cross \vm) \frac{e^{i k r}}{r} \paren{ 1 - \frac{1}{i k r} }.
	}
	\clearpage
	Comparing with Eq.~\refeq{fields1}, we see that $\vH \to -\vE / \Zo$, $\vE \to \Zo \vH$, and $\vp \to \vm / c$ as stated in the book~\cite[p.~413]{Jackson}.  Making these substitutions, the results of Probs.~{1.1(a)} and {(b)} become
	\al{
		\ans{ \dv{\vL}{t}\ }&\ans{= \frac{\muo k^3}{12\pi} \Im[ \vms \cross \vm ], } &
		\ans{ \frac{\dv*{\vL}{t}}{\dv*{E}{t}}\ }&\ans{= \frac{\Im[ \vms \cross \vm ]}{\omg \abs{\vm}^2} }
	}
	where we have used $\mu = 1 / \epso c^2$.
}

\state{Beta function of the Gross-Neveu model~(P\&S~12.2)}{
	Compute $\bet(g)$ in the two-dimensional Gross-Neveu model studied in Problem~11.3,
	\eq{
		\cL = \psibsi i \ptsl \psisi + \frac{1}{2} g^2 (\psibsi \psisi)^2,
	}
	with $i = 1, \ldots, N$.  You should find that this model is asymptotically free.  How was that fact reflected in the solution to Problem~11.3?
}

\sol{
	We saw in Problem~2 of Homework~4 that this Lagrangian can be written as
	\eq{
		\cL = \psibsi i \ptsl \psisi - \sig \psibsi \psisi - \frac{1}{2 g^2} \sig^2,
	}
	where $\sig$ is a new scalar field with no kinetic energy terms.  In the modified minimal subtraction scheme, we found the effective potential was
	\eqn{Veff}{
		\Veff = \sig^2 \curly{ \frac{1}{2 g^2} + \frac{N}{4\pi} \brac{ \ln(\frac{\sig^2}{M^2}) - 1 } }.
	}
	Since $\Gam[ \phicl ] = -(V T) \Veff(\phi)$ by P\&S~(11.50), we have
	\eqn{Gam}{
		\Gam[ \sigcl ] = -(V T)  \sig^2 \curly{ \frac{1}{2 g^2} + \frac{N}{4\pi} \brac{ \ln(\frac{\sig^2}{M^2}) - 1 } }.
	}
	Referring to p.~3 of Lecture~11, we can apply the Callan-Symanzik equation to $\Gam$.   The Callan-Symanzik equation is P\&S~(12.41),
	\eq{
		\brac{ M \pdv{M} + \bet(\lam) \pdv{\lam} + n \gam(\lam) } G^{(n)}(\{ x_i \}; M, \lam) = 0.
	}
	For our problem, $\gam$ is 0 because there are no field insertions.  That is, we have
	\eq{
		\brac{ M \pdv{M} + \bet(g) \pdv{g} } \Gam[ \phicl ] = 0.
	}
	Using Eq.~\refeq{Gam}, note that
	\al{
		\pdv{\Gam}{M} &= (V T) \frac{N \sig^2}{2 \pi M}, &
		\pdv{\Gam}{g} &= (V T) \frac{\sig^2}{g^3}.
	}
	Then
	\eq{
		0 = (V T) \paren{ \frac{N \sig^2}{2 \pi} + \bet(g) \frac{\sig^2}{g^3} }
		\qimplies
		\ans{ \betg = -\frac{N g^3}{2\pi}. }
	}
	This model is asymptotically free because the $\bet$ function is proportional to $-g^3$~\cite[pp.~424--425]{Peskin}.
	
	In 2(e) of Homework~4, we found that the vacuum expectation value of $\sig$ was
	\eq{
		\sig = \pm M e^{-\pi / N g^2} = \pm v.
	}
	We showed that the vacuum expectation value does not depend on the renormalization condition chosen.  This means that we can increase $M \to 0$ while holding $\sig$ constant, and see that $g \to 0$ logarithmically.  This is indicative of an asymptotically-free theory~\cite[p.~425]{Peskin}. \qed
}



\clearpage
\state{(Jackson 12.15)}{
	Consider the Proca equation for a localized steady-state distribution of current that has only a static magnetic moment.  This model can be used to study the observable effects of a finite photon mass on the earth's magnetic field.  Note that if the magnetization is $\vcM(\vx)$ the current density can be written as $\vJ = c \,(\grad \cross \vcM)$.
}

%
%	Jackson 12.15(a)
%

\prob{}{
	Show that if $\vcM = \vm \,f(\vx)$, where $\vm$ is a fixed vector and $f(\vx)$ is a localized scalar function, the vector potential is
	\eq{
		\vA(\vx) = -\vm \cross \grad \int f(\vx') \frac{e^{-\mu \abs{\vx - \vx'}}}{\abs{\vx - \vx'}} \ddcxp.
	}
	\vfix
}

\sol{
	The Proca equations of motion in the static limit are given by the equation immediately following Jackson~(12.93),
	\eq{
		\lap \Asa - \mu^2 \Asa = -\frac{4\pi}{c} \Jsa,
	}
	which implies
	\eqn{thing3}{	
		\lap \vA - \mu^2 \vA = -\frac{4\pi}{c} \vJ.
	}
	We will proceed by finding the Green's function for this equation, which satisfies
	\eqn{green}{
		(\lap - \mu^2) \,G(\vx) = \del^3(\vx).
	}
	We can Fourier transform $G(\vx)$ so long as $\vA$ and its derivatives vanish at infinity~\cite{Poisson}.  The Fourier transform expression and its inverse in one dimension are, according to Jackson~(2.45--46),
	\al{
		A(k) &= \frac{1}{\sqrt{2\pi}} \intnii e^{-i k x} \,f(x) \ddx, &
		f(x) &= \frac{1}{\sqrt{2\pi}} \intnii A(k) \,e^{i k x} \ddk.
	}
	Generalizing to three dimensions, the Green's function transforms as~\cite{Poisson}
	\eq{
		G(\vk) = \frac{1}{(2\pi)^{3/2}} \int G(\vx) \,e^{i \vk \vdot \vx} \ddcx.
	}
	Transforming both sides of Eq.~\refeq{green} in the same way, we have
	\al{
		\frac{1}{(2\pi)^{3/2}} \int (\lap - \mu^2) \,G(\vx) \,e^{i \vk \vdot \vx} \ddcx &= \frac{1}{(2\pi)^{3/2}} \int \lap \,G(\vx) \,e^{i \vk \vdot \vx} \ddcx - \frac{\mu^2}{(2\pi)^{3/2}} \int G(\vx) \,e^{i \vk \vdot \vx} \ddcx \\
		&= \frac{k^2 - \mu^2}{(2\pi)^{3/2}}  \int G(\vx) \,e^{i \vk \vdot \vx} \ddcx, \\
		\frac{1}{(2\pi)^{3/2}} \int \del^3(\vx) \,e^{i \vk \vdot \vx} \ddcx &= \frac{1}{(2\pi)^{3/2}},
	}
	so
	\eq{
		(k^2 - \mu^2) \,G(\vk) = \frac{1}{(2\pi)^{3/2}}
		\qimplies
		G(\vk) = \frac{1}{(2\pi)^{3/2} (k^2 - \mu^2)}.
	}
	Then we can find $G(\vx)$ using the inverse Fourier transform~\cite{Poisson}:
	\al{
		G(\vx) &= \frac{1}{(2\pi)^{3/2}} \int G(\vk) \,e^{i \vk \vdot \vx} \ddck
		= \frac{1}{(2\pi)^3} \int \frac{e^{i \vk \vdot \vx}}{k^2 - \mu^2} \ddck
		= \frac{1}{(2\pi)^3} \int_0^{2\pi} \int_0^\pi \intoi \frac{e^{i k \absvx}}{k^2 - \mu^2} k^2 \sin\tht \ddk \dd{\tht} \dd{\phi} \\
		&= \frac{4\pi}{(2\pi)^3 \absvx} \intoi \frac{k \sin(k \absvx)}{k^2 - \mu^2} k^2 \ddk
		= \frac{e^{-\mu \absvx}}{4\pi \absvx}.
	}
	
	Then the solution to Eq.~\refeq{thing3} is
	\eq{
		\vA(\vx) = \frac{4\pi}{c} \int \vJ(\vx') \,G(\vx - \vx') \ddcxp
		= \frac{1}{c} \int \vJ(\vx') \frac{e^{-\mu \abs{\vx - \vx'}}}{\abs{\vx - \vx'}} \ddcxp.
	}
	Note that
	\eq{
		\frac{\vJ}{c} = \grad \cross [ \vm \,f(\vx) ]
		= \grad f(\vx) \cross \vm + f(\vx) \,\grad \cross \vm
		= \grad f(\vx) \cross \vm
		= -\vm \cross \grad f(\vx),
	}
	where we have used the identity $\grad \cross (\psi \vaa) = \grad \psi \cross \vaa + \psi \grad \cross \vaa$.  Making this substitution,
	\eqn{thing3.2}{
		\vA(\vx) = -\int \vm \cross \grad' f(\vx') \,\frac{e^{-\mu \abs{\vx - \vx'}}}{\abs{\vx - \vx'}} \ddcxp
		= -\vm \cross \int \grad' f(\vx') \,\frac{e^{-\mu \abs{\vx - \vx'}}}{\abs{\vx - \vx'}} \ddcxp.
	}
	Integrating by parts,
	\eq{
		\int \grad' f(\vx') \,\frac{e^{-\mu \abs{\vx - \vx'}}}{\abs{\vx - \vx'}} \ddcxp = \brac{ f(\vx') \grad' \frac{e^{-\mu \abs{\vx - \vx'}}}{\abs{\vx - \vx'}} }_{-\infty}^\infty - \int f(\vx') \,\grad' \frac{e^{-\mu \abs{\vx - \vx'}}}{\abs{\vx - \vx'}} \ddcxp
		= -\int f(\vx') \,\grad' \frac{e^{-\mu \abs{\vx - \vx'}}}{\abs{\vx - \vx'}} \ddcxp,
	}
	since $f(\vx)$ is a localized scalar function and therefore vanishes at infinity.  Since the Green's function depends only on $\abs{\vx - \vx'}$, we can replace $\grad'$ by $-\grad$:
	\eq{
		-\int f(\vx') \,\grad' \frac{e^{-\mu \abs{\vx - \vx'}}}{\abs{\vx - \vx'}} \ddcxp
		= \int f(\vx') \,\grad \frac{e^{-\mu \abs{\vx - \vx'}}}{\abs{\vx - \vx'}} \ddcxp
		= \grad \int f(\vx') \frac{e^{-\mu \abs{\vx - \vx'}}}{\abs{\vx - \vx'}} \ddcxp.
	}
	Making this substitution in Eq.~\refeq{thing3.2},
	\eqn{ans3a}{
		\ans{ \vA(\vx) = -\vm \cross \grad \int f(\vx') \frac{e^{-\mu \abs{\vx - \vx'}}}{\abs{\vx - \vx'}} \ddcxp, }
	}
	as desired. \qed
}

%
%	Jackson 12.15(b)
%

\prob{}{
	If the magnetic dipole is a point dipole at the origin [$f(\vx) = \del(\vx)$], show that the magnetic field away from the origin is
	\eq{
		\vB(\vx) = [ 3 \,\rh (\rh \vdot \vm) - \vm ] \paren{ 1 + \mu r + \frac{\mu^2 r^2}{3} }\frac{e^{-\mu r}}{r^3} - \frac{2}{3} \mu^2 \vm \frac{e^{-\mu r}}{r}.
	}
	\vfix
}

\sol{
	Setting $f(\vx) = \del(\vx)$ in Eq.~\refeq{ans3a},
	\eq{
		\vA(\vx) = -\vm \cross \grad \int \del(\vx') \frac{e^{-\mu \abs{\vx - \vx'}}}{\abs{\vx - \vx'}} \ddcxp
		= -\vm \cross \grad \frac{e^{-\mu \absvx}}{\absvx}.
	}
	Note that
	\eq{
		\grad \frac{e^{-\mu \absvx}}{\absvx} = \paren{ -\frac{e^{-\mu \absvx}}{\absvx^2} - \frac{\mu e^{-\mu \absvx}}{\absvx} } \vx
		= -(1 + \mu \absvx) \frac{e^{-\mu \absvx}}{\absvx^2} \xh
		= -(1 + \mu \absvx) \frac{e^{-\mu \absvx}}{\absvx^3} \vx,
	}
	so
	\eq{
		\vA(\vx) = (\vm \cross \vx) (1 + \mu \absvx) \frac{e^{-\mu \absvx}}{\absvx^2}.
	}
	
	The magnetic field is given by $\vB = \grad \cross \vA$:
	\eq{
		\vB(\vx) = \grad \cross \brac{ (\vm \cross \vx) (1 + \mu \absvx) \frac{e^{-\mu \absvx}}{\absvx^3} }
		= \grad \brac{ (1 + \mu \absvx) \frac{e^{-\mu \absvx}}{\absvx^3}} \cross (\vm \cross \vx) + (1 + \mu \absvx) \frac{e^{-\mu \absvx}}{\absvx^3} \grad \cross (\vm \cross \vx).
	}
	For the first term,
	\eq{
		\grad \brac{ (1 + \mu \absvx) \frac{e^{-\mu \absvx}}{\absvx^3}}
		= \brac{ -\mu \frac{e^{-\mu \absvx}}{\absvx^3} + (1 + \mu \absvx) \paren{ -\frac{\mu e^{- \mu \absvx}}{\absvx^3} - \frac{3 e^{-\mu \absvx}}{\absvx^4} } } \xh \\
		= -(3 + 3\mu \absvx + \mu^2 \absvx^2) \frac{e^{-\mu \absvx}}{\absvx^4} \,\xh.
	}
	Then, using the identity $\grad \cross (\vaa \cross \vbb) = \vaa (\grad \vdot \vbb) - \vbb (\grad \vdot \vaa) + (\vbb \vdot \grad) \vaa - (\vaa \vdot \grad) \vbb$,
	\eq{
		\grad \cross (\vm \cross \vx) = \vm (\grad \vdot \vx) - \vx (\grad \vdot \vm) + (\vx \vdot \grad) \vm - (\vm \vdot \grad) \vx
		= \vm (\grad \vdot \vx) - (\vm \vdot \grad) \vx.
	}
	Making these substitutions,
	\al{
		\vB(\vx) &= -(3 + 3\mu \absvx + \mu^2 \absvx^2) \frac{e ^{-\mu \absvx}}{\absvx^3} \,\xh \cross (\vm \cross \xh) + (1 + \mu \absvx) \frac{e^{-\mu \absvx}}{\absvx^3} [ \vm (\grad \vdot \xh) - (\vm \vdot \grad) \xh ] \\
		&= -(3 + 3\mu \absvx + \mu^2 \absvx^2) \frac{e^{-\mu \absvx}}{\absvx^3} [ (\xh \vdot \xh) \vm - (\xh \vdot \vm) \xh ] + 2 (1 + \mu \absvx) \frac{e^{-\mu \absvx}}{\absvx^2} \vm \\
		&= -3 \paren{ 1 + \mu \absvx + \frac{\mu^2 \absvx^2}{3} } \frac{e^{-\mu \absvx}}{\absvx^3} [ \vm - (\xh \vdot \vm) \xh ] + 2 (1 + \mu \absvx) \frac{e^{-\mu \absvx}}{\absvx^2} \vm \\
		&= [ 3 \,\xh (\xh \vdot \vm) - \vm ] \paren{ 1 + \mu \absvx + \frac{\mu^2 \absvx^2}{3} }\frac{e^{-\mu \absvx}}{\absvx^3} - \frac{2}{3} \mu^2 \vm \frac{e^{-\mu \absvx}}{\absvx}.
	}
	Letting $\xh \to \rh$ and $\absvx \to r$, we have
	\eqn{ans3b}{
		\ans{ \vB(\vx) = [ 3 \,\rh (\rh \vdot \vm) - \vm ] \paren{ 1 + \mu r + \frac{\mu^2 r^2}{3} } \frac{e^{-\mu r}}{r^3} - \frac{2}{3} \mu^2 \vm \frac{e^{-\mu r}}{r} }
	}
	as we sought to show. \qed
}

%
%	Jackson 12.15(c)
%

\prob{}{
	The result of Prob.~{3(b)} shows that at fixed $r = R$ (on the surface of the earth), the earth's magnetic field will appear as a dipole angular distribution, plus an added constant magnetic field (an apparently external field) antiparallel to $\vm$.  Satellite and surface observations lead to the conclusion that the ``external'' field is less than $\num{4e-3}$ times the dipole field at the magnetic equator.  Estimate a lower limit on $\mu^{-1}$ in earth radii and an upper limit on the photon mass in grams from this datum.
}

\sol{
	The earth's magnetic field is parallel to its surface at the magnetic equator~\cite{Magnetic}.  Figure~\ref{mag} shows the field of a magnetic dipole $\vm \propto \zh$~\cite[pp.~245--246]{Griffiths}.  Thus, at the magnetic equator, $\vm$ is also parallel to the field, and so the spherical unit vector $\rh$ is perpendicular to $\vm$.  So Eq.~\refeq{ans3b} becomes
	\eq{
		\vB(\vx) = -\vm \paren{ 1 + \mu R + \frac{\mu^2 R^2}{3} }\frac{e^{-\mu R}}{R^3} - \frac{2}{3} \mu^2 \vm \frac{e^{-\mu R}}{R}
		\equiv \vBdip + \vBext,
	}
	where we have defined $\vBdip$ as the dipole field and $\vBext$ as the ``external' field.  The identification is made by comparing with Jackson~(5.56), which gives the magnetic field due to a dipole (in SI units):
	\eq{
		\vB(\vx) = \frac{\muo}{4\pi} \frac{3 \nh (\nh \vdot \vm)}{\absvx^3}.
	}
	As stated in the problem, $\abs{\vBext} < (\num{4e-3}) \abs{\vBdip}$.  This gives us
	\eq{
		\frac{2}{3} \mu^2 \frac{e^{-\mu R}}{R} < (\num{4e-3}) \paren{ 1 + \mu R + \frac{\mu^2 R^2}{3} } \frac{e^{-\mu R}}{R^3}
		\qimplies
		\frac{2}{3} \mu^2 R^2 < (\num{4e-3}) \paren{ 1 + \mu R + \frac{\mu^2 R^2}{3} }.
	}
	Solving the second expression with Mathematica gives the positive $\mu = 0.0806 / R$.  Thus, the lower limit on $\mu^{-1}$ is \ans{$\mu^{-1} > 12 R$.}
	
	The parameter $\mu$ is defined by $\mu = \mgam c / \hbar$, where $\mgam$ is the photon mass~\cite[p.~600]{Jackson}.  The earth's radius is $R = \SI{6.37e6}{\meter}$~\cite{YF}, so $\mu < \SI{1.267e-8}{\per\meter}$.  Since $c = \SI{2.998e8}{\meter\per\second}$ and $\hbar = h / (2\pi) = (\SI{6.63e-34}{\joule\second}) / (2\pi) = \SI{1.055e-34}{\joule\second}$~\cite{YF}, we have
	\eq{
		\mgam = \frac{\hbar \mu}{c}
		< \frac{(\SI{1.055e-34}{\joule\second}) (\SI{1.267e-8}{\per\meter})}{\SI{2.998e8}{\meter\per\second}}
		\approx \SI{4e-51}{\kg}
		= \SI{4e-48}{\gram}.
	}
	So the upper bound on the photon mass is \ans{$\mgam < \SI{4e-48}{\gram}$.}
}


\makebib

\end{document}