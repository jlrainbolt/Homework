\documentclass[11pt]{article}
\usepackage{homework}

\classname{323}
\homeworknum{6}



\begin{document}

% Environments

\newcommand{\state}[2]{\begin{statement}{#1} #2 \end{statement}}
\newcommand{\prob}[2]{\begin{problem}{#1} #2 \end{problem}}
\newcommand{\subprob}[1]{\begin{subproblem} #1 \end{subproblem}}
\newcommand{\sol}[1]{\begin{solution} #1 \end{solution}}
\newcommand{\fig}[2]{\begin{figure} \centering #2  \label{#1} \end{figure}}

\newcommand{\makebib}{
	\vfill
	\color{black}
	\bibliography{references}{}
	\bibliographystyle{lucas_unsrt}
}
	

% Implication

\newcommand{\qwhere}{\quad \text{where} \quad}
\newcommand{\qimplies}{\quad \implies \quad}
\newcommand{\impliesq}{\implies \quad}



% Brackets

\newcommand{\paren}[1]{\left( #1 \right)}
\newcommand{\brac}[1]{\left[ #1 \right]}


% Greek

\newcommand{\alp}{\alpha}
\newcommand{\bet}{\beta}
\newcommand{\gam}{\gamma}
\newcommand{\del}{\delta}
\newcommand{\eps}{\epsilon}
\newcommand{\zet}{\zeta}
\newcommand{\tht}{\theta}
\newcommand{\kap}{\kappa}
\newcommand{\lam}{\lambda}
\newcommand{\sig}{\sigma}
\newcommand{\ups}{\upsilon}
\newcommand{\omg}{\omega}

\newcommand{\Gam}{\Gamma}
\newcommand{\Del}{\Delta}
\newcommand{\Tht}{\Theta}
\newcommand{\Lam}{\Lambda}
\newcommand{\Sig}{\Sigma}
\newcommand{\Omg}{\Omega}
% Problem 1

\newcommand{\Psii}{\Psi^i}
\newcommand{\Psiix}{\Psii(x)}

\newcommand{\Pii}{\Pi^i}

\newcommand{\Phii}{\Phi^i}
\newcommand{\Phiix}{\Phii(x)}
\newcommand{\PhiN}{\Phi^N}
\newcommand{\PhiNx}{\PhiN(x)}
\newcommand{\Phiq}{\Phi^1}
\newcommand{\Phiw}{\Phi^2}

\newcommand{\ddcx}{\dd[3]{x}}

\newcommand{\delij}{\del^{i j}}
\newcommand{\delkl}{\del^{k l}}
\newcommand{\delil}{\del^{i l}}
\newcommand{\deljk}{\del^{j k}}
\newcommand{\delik}{\del^{i k}}
\newcommand{\deljl}{\del^{j l}}

\newcommand{\DF}{D_F}

\newcommand{\sigx}{\sig(x)}

\newcommand{\pii}{\pi^i}
\newcommand{\pij}{\pi^j}
\newcommand{\pik}{\pi^k}
\newcommand{\pil}{\pi^l}
\newcommand{\piix}{\pi(x)}

\newcommand{\pq}{p_1}
\newcommand{\pw}{p_2}
\newcommand{\pe}{p_3}
\newcommand{\pr}{p_4}

\newcommand{\vp}{\vb{p}}
\newcommand{\vpsi}{\vp_i}

\newcommand{\mpi}{m_\pi}


\state{(Jackson 9.8)}{
	\emph{Hint:} The electromagnetic angular momentum density comes from more than the transverse (radiation zone) components of the fields.
}

%
%	Jackson 9.8(a)
%

\prob{}{
	Show that a classical oscillating electric dipole $\vp$ with fields given by
	\aln{ \label{fields1}
		\vH &= \frac{c k^2}{4\pi} (\nh \cross \vp) \frac{e^{i k r}}{r} \paren{ 1 - \frac{1}{i k r} }, &
		\vE &= \frac{1}{4\pi \epso} \curly{ k^2 (\nh \cross \vp) \cross \nh \frac{e^{i k r}}{r} + [ 3 \nh (\nh \vdot \vp) - \vp ] \paren{ \frac{1}{r^3} - \frac{i k}{r^2} } e^{i k r} },
	}
	radiates electromagnetic angular momentum to infinity at the rate
	\eq{
		\dv{\vL}{t} = \frac{k^3}{12 \pi \epso} \Im[ \vp^* \cross \vp ].
	}
	\vfix
}

\sol{
	From Prob.~3 of Homework~{4} for Physics 322, the angular momentum density is $\vl = \vx \cross \vg$, where $\vg$ is the linear momentum density in SI units.  According to Jackson~{6.118}, $\vg = (\vE \cross \vH) / c^2$.  However, according to Jackson~(9.20), the time-averaged power radiated per unit solid angle by $\vp$ is given by
	\eq{
		\dv{P}{\Omg} = \frac{1}{2} \Re[ r^2 \,\nh \vdot \vE \cross \vHs ],
	}
	which indicates that we should send $\vH \to \vHs$ for a sensible result.  Thus,
	\eq{
		\vl = \frac{\vE \cross \vHs}{2 c^2}.
	}
	
	One of the vector identities on the inside cover of Jackson is $\vaa \cross (\vbb \cross \vcc) = (\vaa \vdot \vcc) \vbb - (\vaa \vdot \vbb) \vcc$, so
	\eqn{l1}{
		\vl = \frac{(\vx \vdot \vHs) \vE - (\vx \vdot \vE) \vHs}{2 c^2}.
	}
	From Eq.~\refeq{fields1}, note that
	\eq{
		\vx \vdot \vHs \propto \vx \vdot (\nh \cross \vps)
		= \vps \vdot (\vx \cross \nh)
		= \vO,
	}
	where we have used the identity $\vaa \vdot (\vbb \cross \vcc) = \vcc \vdot (\vaa \cross \vbb)$ and the fact that $\nh$ points in the $\vx$ direction.  For $\vx \vdot \vE$, note that
	\al{
		\vx \vdot [ (\nh \cross \vp) \cross \nh ] &= -\vx \vdot [ \nh \cross (\nh \cross \vp) ]
		= -\vx \vdot [ (\nh \vdot \vp) \nh - (\nh \vdot \nh) \vp ]
		= -(\nh \vdot \vp) (\vx \vdot \nh) + \vx \vdot \vp \\
		&= -r (\nh \vdot \vp) + \vx \vdot \vp
		= \vx \vdot \vp - \vx \vdot \vp
		= 0, \\[1.5ex]
		\vx \vdot [ 3 \nh (\nh \vdot \vp) - \vp ] &= 3 (\vx \vdot \nh) (\nh \vdot \vp) - \vx \vdot \vp
		= 3r (\nh \vdot \vp) - \vx \vdot \vp
		= 3(\vx \vdot \vp) - \vx \vdot \vp
		= 2(\vx \vdot \vp),
	}
	since $\abs{\vx} = r$ and $\vx = r \,\nh$.  Then
	\eq{
		\vx \vdot \vE = \frac{1}{2\pi \epso} (\vx \vdot \vp) \paren{ \frac{1}{r^3} - \frac{i k}{r^2} } e^{i k r}
		= \frac{1}{2\pi \epso} (\nh \vdot \vp) \paren{ \frac{1}{r^2} - \frac{i k}{r} } e^{i k r}.
	}
	
	With these substitutions, Eq.~\refeq{l1} becomes
	\al{
		\vl &= -\frac{(\vx \vdot \vE) \vHs}{c^2}
		= -\frac{1}{4\pi \epso c^2} (\nh \vdot \vp) \paren{ \frac{1}{r^2} - \frac{i k}{r} } e^{i k r} \frac{c k^2}{4\pi} (\nh \cross \vps) \frac{e^{-i k r}}{r} \paren{ 1 + \frac{1}{i k r} } \\
		&= -\frac{k^2}{16\pi^2 \epso c r} (\nh \vdot \vp) (\nh \cross \vps) \paren{ \frac{1}{r^2} - \frac{i k}{r} } \paren{ 1 - \frac{i}{k r} }
		= -\frac{k^2}{16\pi^2 \epso c} (\nh \vdot \vp) (\nh \cross \vps) \paren{ \frac{1}{r^2} - \frac{i}{k r^3} - \frac{i k}{r} - \frac{1}{r^2} } \\
		&= -\frac{i k^2}{16\pi^2 \epso c r} (\nh \vdot \vp) (\nh \cross \vps) \paren{ \frac{1}{k r^3} + \frac{k}{r^2} }
		= \frac{i k^3}{16\pi^2 \epso c r^2} (\nh \vdot \vp) (\nh \cross \vps) \paren{ \frac{1}{k^2 r^2} + 1 }.
	}
	
	Let $\vL$ be the angular momentum radiated to a distance $R$.  Then
	\eq{
		\vL = \int_R \vl(r) \ddcx
		= \intopi \intotp \intoR \vl(r) \,r^2 \sin\tht \ddr \ddphi \dd\tht,
	}
	and the time derivative is
	\aln{
		\dv{\vL}{t} &= \dv{t}(\intopi \intotp \intoR \vl(r) \,r^2 \sin\tht \ddr \ddphi \dd\tht)
		= \dv{r}{t} \dv{r}(\intopi \intotp \intoR \vl(r) \,r^2 \sin\tht \ddr \ddphi \dd\tht) \notag \\
		&= c \intopi \intotp \vl(r) \,r^2 \sin\tht \ddphi \dd\tht
		= \frac{i k^3}{16\pi^2 \epso c} \paren{ \frac{1}{k^2 r^2} + 1 } \intopi \intotp (\nh \vdot \vp) (\nh \cross \vps) \sin\tht \ddphi \dd\tht. \label{dLdt}
	}
	Note that
	\eq{
		[ (\nh \vdot \vp) (\nh \cross \vps) ]_i = \sumje n_j p_j (\nh \cross \vps)_i
		= \sumje \sumke \sumle \epsikl n_j p_j n_k p_l^*,
	}
	so
	\eq{
		\dv{L_i}{t} \propto \sumje \sumke \sumle \epsikl p_j p_l^* \int n_j p_k \ddOmg
		= \sumje \sumke \sumle \epsikl p_j p_l^* \frac{4\pi}{3} \del_{jk}
		= \frac{4\pi}{3} \epsikl p_k p_l^*
		= \frac{4\pi}{3} (\vp \cross \vps)_i,
	}
	where we have used Jackson~(9.47), $\int n_\bet n_\gam \ddOmg = 4\pi \del_{\bet \gam} / 3$.  Making this substitution into Eq.~\refeq{dLdt},
	\eq{
		\dv{\vL}{t} = \frac{i k^3}{6\pi \epso c} \paren{ \frac{1}{k^2 r^2} + 1 } (\vp \cross \vps).
	}
	Taking the limit as $r \to \infty$, we find
	\eq{
		\dv{\vL}{t} = \Re\!\brac{ \frac{i k^3}{12\pi \epso c} (\vp \cross \vps) }
		= \Re\!\brac{ -\frac{i k^3}{12\pi \epso c} (\vps \cross \vp) }
		= \ans{ \frac{k^3}{12\pi \epso c} \Im[ \vps \cross \vp ], }
	}
	as desired. \qed
}

%
%	Jackson 9.8(b)
%
\clearpage
\prob{}{
	What is the ratio of angular momentum radiated to energy radiated?  Interpret.
}

%
%	Jackson 9.8(c)
%

\prob{}{
	For a charge $e$ rotating in the $xy$ plane at radius $a$ and angular speed $\omg$, show that there is only a $z$ component of radiated angular momentum with magnitude $\dv*{\Lz}{t} = e^2 k^3 a^2 / 6 \pi \epso$.  What about a charge oscillating along the $z$ axis?
}

%
%	Jackson 9.8(d)
%

\prob{}{
	What are the results corresponding to Probs.~{1(a)} and {1(b)} for magnetic dipole radiation?
}




\state{(Jackson 10.1)}{\ }

%
%	Jackson 10.1(a)
%

\prob{}{
	Show that for arbitrary initial polarization, the scattering cross section of a perfectly conducting sphere of radius $a$, summed over outgoing polarizations, is given in the long-wavelength limit by
	\eq{
		\dv{\sig}{\Omg}(\vepso, \nho, \nh) = k^4 a^6 \brac{ \frac{5}{4} - \abs{\vepso \vdot \nh}^2 - \frac{1}{4} \abs{\nh \vdot (\nho \cross \vepso)}^2 - \nho \vdot \nh },
	}
	where $\nho$ and $\nh$ are the directions of the incident and scattered radiations, respectively, while $\vepso$ is the (perhaps complex) unit polarization vector of the incident radiation ($\vepso^* \vdot \vepso = 1$; $\nho \vdot \vepso = 0$).
}

\prob{}{
	If the incident radiation is linearly polarized, show that the cross section is
	\eq{
		\dv{\sig}{\Omg}(\vepso, \nho, \nh) = k^4 a^6 \brac{ \frac{5}{8} (1 + \cos^2\tht) - \cos\tht - \frac{3}{8} \sin^2\tht \cos(2\phi) },
	}
	where $\nh \vdot \nho = \cos\tht$ and the azimuthal angle $\phi$ is measured from the direction of linear polarization.
}

\prob{}{
	What is the ratio of scattered intensities at $\tht = \pi / 2$, $\phi = 0$ and $\tht = \pi / 2$, $\phi = \pi / 2$?  Explain physically in terms of the induced multipoles and their radiation patterns.
}




\state{(Jackson 12.15)}{
	Consider the Proca equation for a localized steady-state distribution of current that has only a static magnetic moment.  This model can be used to study the observable effects of a finite photon mass on the earth's magnetic field.  Note that if the magnetization is $\vcM(\vx)$ the current density can be written as $\vJ = c (\grad \cross \vcM)$.
}

%
%	Jackson 12.15(a)
%

\prob{}{
	Show that if $\vcM = \vm f(\vx)$, where $\vm$ is a fixed vector and $f(\vx)$ is a localized scalar function, the vector potential is
	\eq{
		\vA(\vx) = -\vm \cross \grad \int f(\vx') \frac{e^{-\mu \abs{\vx - \vx'}}}{\abs{\vx - \vx'}} \ddcxp.
	}
	\vfix
}

%
%	Jackson 12.15(b)
%

\prob{}{
	If the magnetic dipole is a point dipole at the origin [$f(\vx) = \del(\vx)$], show that the magnetic field away from the origin is
	\eq{
		\vB(\vx) = [ 3 \,\rh (\rh \vdot \vm) - \vm ] \paren{ 1 + \mu r + \frac{\mu^2 r^2}{3} }\frac{e^{-\mu r}}{r^3} - \frac{2}{3} \mu^2 \vm \frac{e^{-\mu r}}{r}.
	}
	\vfix
}

%
%	Jackson 12.15(c)
%

\prob{}{
	The result of Prob.~{3(b)} shows that at fixed $r = R$ (on the surface of the earth), the earth's magnetic field will appear as a dipole angular distribution, plus an added constant magnetic field (an apparently external field) antiparallel to $\vm$.  Satellite and surface observations lead to the conclusion that the ``external'' field is less than $\num{4e-3}$ times the dipole field at the magnetic equator.  Estimate a lower limit on $\mu^{-1}$ in earth radii and an upper limit on the photon mass in grams from this datum.
}


\makebib

\end{document}