\state{}{\ \vfix}

%
%	3(a)
%

\prob{}{
	Apply the Noether procedure for constructing the energy-momentum tensor to the source-free electromagnetic field and show that the resulting tensor $\Tmn$ satisfies the conservation equation $\partsm \Tmn = 0$.
}

\sol{
	Adapting Eq.~\refeq{Lsource}, the Lagrangian for the source-free electromagnetic field is
	\eq{
		\cL = -\frac{1}{16\pi} \Fsmn \Fmn.
	}
	We want to evaluate
	\eqn{Tmn3}{
		\Tmn = \pdv{\cL}{(\partsm \Al)} \partn \Al - \etamn \cL.
	}
	In order to evaluate the derivatives, we can use the variational method to calculate $\pdv*{\cL}{(\partsa \Asb)}$ by letting $\partsa \Asb \to \partsa \Asb + \del \partsa \Asb$~\cite[p.~86]{Landau}.  Let
	\eq{
		\delL = \cL(\partsa \Asb) - \cL(\partsa \Asb + \del \partsa \Asb).
	}
	Note that
	\eq{
		\cL(\partsa \Asb + \del \partsa \Asb) = -\frac{1}{16} (\Fsab + \del\Fsab) (\Fab + \del\Fab)
		\approx -\frac{1}{16\pi} (\Fsab \Fab + \Fsab \,\del\Fab + \del\Fsab \,\Fab),
	}
	so
	\al{
		\delL &= -\frac{1}{16\pi} (\Fsab \,\del\Fab + \del\Fsab \,\Fab)
		= -\frac{1}{8\pi} \del\Fsab \,\Fab
		= -\frac{1}{8\pi} (\partsa \,\del\Asb - \partsb \,\del\Asa) \Fab \\
		&= -\frac{1}{8\pi} (\partsa \,\del\Asb + \partsa \,\del\Asb) \Fab
		= -\frac{1}{4\pi} \partsa \,\del\Asb \,\Fab,
	}
	where we have used the antisymmetry of $\Fab$.  This gives us
	\eq{
		\pdv{\cL}{(\partsa \Asb)} = -\frac{1}{4\pi} \Fab
		\qimplies
		\pdv{\cL}{(\partsa \Ab)} = -\frac{1}{4\pi} \Fasb,
	}
	and then we find
	\eqn{TmnA}{
		\Tmn = -\frac{1}{4\pi} \Fmsl \,\partn \Al + \frac{1}{16\pi} \etamn \Fsab \Fab
		= \ans{ \frac{1}{4\pi} \paren{\frac{1}{4} \etamn \Fsab \Fab - \Fmsl \,\partn \Al }. }
	}
	
	To prove conservation, firstly we note that
	\eq{
		\partsm \Tmn = \frac{1}{4\pi} \paren{\frac{1}{4} \partsm (\etamn \Fsab \Fab) - \partsm (\Fmsl \,\partn \Al) },
	}
	which implies
	\al{
		4\pi \Tmn &= \frac{1}{4} \partn \Fsab \,\Fab + \frac{1}{4} \Fsab \,\partn \Fab - \partm \Fsml \,\partn \Al - \Fmsl \,\partsm \partn \Al \\
		&= \frac{1}{2} \Fasb \,\partn \Fsasb - \parta \Fsab \,\partn \Ab - \Fasb \,\partsa \partn \Ab.
	}
	For a source-free field, the inhomogeneous Maxwell equations become $\partsm \Fmn = 0$.  This means the second term disappears.  Then
	\al{
		4\pi \partsm \Tmn &= \frac{1}{2} \Fasb \,\partn \Fsasb - \Fasb \,\partsa \partn \Ab
		= \frac{1}{2} \Fasb \,\partn (\partsa \Ab - \partb \Asa) - \Fasb \,\partsa \partn \Ab \\
		&= \frac{1}{2} \Fasb \,\partsa \partn \Ab - \frac{1}{2} \Fasb \,\partn \partb \Asa - \Fasb \,\partsa \partn \Ab
%		&= \frac{1}{2} \Fasb \,\partn \partsa \Ab - \frac{1}{2} \Fsba \,\parta \partn \Asb - \Fasb \,\partsa \partn \Ab \\
		= \frac{1}{2} \Fasb \,\partsa \partn \Ab - \frac{1}{2} \Fsba \,\partsa \partn \Ab - \Fasb \,\partsa \partn \Ab \\
		&= \frac{1}{2} \Fasb \,\partn \partsa \Ab + \frac{1}{2} \Fasb \,\partsa \partn \Ab - \Fasb \,\partsa \partn \Ab
		= \ans{ 0, }
	}
	where we have used the antisymmetry of $\Fmn$.  Thus, we have shown that $\Tmn$ is conserved. \qed
}

%
%	3(b)
%

\prob{}{
	Show that the ``improvement'' of this tensor discussed in class, that leads to
	\eqn{Tmn2}{
		\Tmn = \frac{1}{4\pi} \paren{ \Fml \Fsln + \frac{1}{4} \etamn \Fsab \Fab },
	}
	does not spoil conservation.
}

\sol{
	The derivative of $\Tmn$ in this case can be written
	\eq{
		\partsm \Tmn = \frac{1}{4\pi} \paren{ \partsm (\Fml \Fsln) + \frac{1}{4} \partsm (\etamn \Fsab \Fab) }.
	}
	Rearranging and applying $\partsm \Fmn = 0$ as in Prob.~3(a),
	\al{
		4\pi \partsm \Tmn &= \partsm \Fml \, \Fsln + \Fml \,\partsm \Fsln + \frac{1}{4} \partsm (\etamn \Fsab \Fab)
		= \Fsasb \,\parta \Fsbn + \frac{1}{2} \Fasb \,\partn \Fsasb \\
		&= \Fsasb \,\parta (\partsb \An - \partn \Asb) + \frac{1}{2} \Fasb \,\partn (\partsa \Ab - \partb \Asa) \\
		&= \Fsasb \,\parta \partsb \An - \Fsasb \,\partn \parta \Asb + \frac{1}{2} \Fasb \,\partn \partsa \Ab - \frac{1}{2} \Fasb \,\partn \partb \Asa \\
%		&= \Fsasb \,\parta \partsb \An - \Fsasb \,\partn \parta \Asb + \frac{1}{2} \Fasb \,\partn \partsa \Ab + \frac{1}{2} F_\bet{}^\alp \,\partn \partb \Asa \\
		&= \Fsasb \,\parta \partsb \An - \Fsasb \,\partn \parta \Asb + \frac{1}{2} \Fsasb \,\partn \parta \Asb + \frac{1}{2} \Fsasb \,\partn \parta \Asb
		= \Fab \,\partsa \partsb \An \\
		&= (\parta \Ab - \partb \Aa) \,\partsa \partsb \An
		= \parta \Ab \, \partsa \partsb \An - \partb \Aa \,\partsa \partsb \An
		= \parta \Ab \, \partsa \partsb \An - \parta \Ab \,\partsa \partsb \An
		= \ans{ 0, }
	}
	and so we have shown that this version of $\Tmn$ is also conserved. \qed
}

%
%	3(c)
%

\prob{}{
	Evaluate $\Too$ and $\Toi$ in terms of electric and magnetic fields.  What is the physical interpretation of these quantities?
}

\sol{
	From Eq.~\refeq{Tmn2},
	\aln{
		\Too &= \frac{1}{4\pi} \paren{ F^{0 \lam} F_\lam{}^0 + \frac{1}{4} \eta^{00} \Fsab \Fab }
		= \frac{1}{4\pi} \paren{ F^{0 \lam} F_\lam{}^0 + \frac{1}{4} \Fsab \Fab }, \label{Too} \\
		T^{0i} &= \frac{1}{4\pi} \paren{ F^{0\lam} F_\lam{}^i + \frac{1}{4} \eta^{0i} \Fsab \Fab }
		= \frac{1}{4\pi} F^{0\lam} F_\lam{}^i. \label{Toi}
	}
	According to Jackson~(11.137--138),
	\al{
		\Fmn &= \mqty[ 0 & -\Ex & -\Ey & -\Ez \\
					\Ex & 0 & -\Bz & \By \\
					\Ey & \Bz & 0 & -\Bx \\
					\Ez & -\By & \Bx & 0 ], &
		\Fsmn &= \mqty[ 0 & \Ex & \Ey & \Ez \\
					-\Ex & 0 & -\Bz & \By \\
					-\Ey & \Bz & 0 & -\Bx \\
					-\Ez & -\By & \Bx & 0 ].
	}
	Then
	\al{
		\Fsmn \Fmn &= -\Ex^2 - \Ey^2 - \Ez^2 - \Ex^2 + \Bz^2 + \By^2 - \Ey^2 + \Bz^2 + \Bx^2 - \Ez^2 + \By^2 + \Bx^2
		= 2 (\vB^2 - \vE^2).
	}
	
	Note also that
	\al{
		\Fsln = \eta_{\lam \mu} \Fmn
		= \mqty[ 0 & -\Ex & -\Ey & -\Ez \\
				-\Ex & 0 & -\Bz & \By \\
				-\Ey & \Bz & 0 & -\Bx \\
				-\Ez & -\By & \Bx & 0 ],
	}
	so
	\al{
		F^{0 \lam} F_\lam{}^0 &= (-\vE) \vdot (-\vE)
		= \vE^2, &
		F^{0\lam} F_\lam{}^i &= B_j E_k - E_k B_j
		= (\vE \cross \vB)_i.
	}
	Equations~(\ref{Too}--\ref{Toi}) are then
	\al{
		\ans{ \Too\ }&\ans{= \frac{1}{8\pi} (\vE^2 + \vB^2), }&
		\ans{\Toi\ }&\ans{= \frac{1}{4\pi} (\vE \cross \vB)_i. }
	}
	
	According to Wald~(5.9--10),
	\al{
		\mathcal{E} &= \frac{1}{8\pi} (\vE^2 + \vB^2), &
		\boldsymbol{\mathcal{P}} &= \frac{1}{4\pi} (\vE \cross \vB),
	}
	are the energy density and momentum density, respectively, of the electromagnetic field.  Obviously, then, $\Too$ is the energy density of the free field, and $\Toi$ is a component of its momentum density.
}

%
%	3(d)
%

\prob{}{
	Calculate the correction to the conservation equation $\partsm\Tmn = 0$ in the presence of a nonzero current~$\Jm$.
}

\sol{
	When the current is nonzero, the only difference from Probs.~(a--b) is that $\partsm \Tmn = (4\pi / c) \Jn \neq 0$.  For the ``unimproved'' tensor given by Eq.~\refeq{TmnA},
	\eq{
		4\pi \partsm\Tmn = -\parta \Fsab \,\partn \Ab
		= -\frac{4\pi}{c} \Jsb \,\partn \Ab,
	}
	so the corrected equation is
	\eq{
		\ans{ \partsm\Tmn = -\frac{1}{c} \Jsm \,\partn \Am. }
	}
	\clearpage
	For the ``improved'' tensor given by Eq.~\refeq{Tmn2},
	\eq{
		4\pi \partsm\Tmn = \partsm \Fml \, \Fsln
		= \frac{4\pi}{c} \Jl \Fsln
		= -\frac{4\pi}{c} F^\nu{}_\lam \Jl
		= -\frac{4\pi}{c} F^{\nu\lam} J_\lam,
	}
	so the corrected equation is
	\eq{
		\ans{ \partsm\Tmn = -\frac{1}{c} F^{\nu\mu} \Jsm. }
	}
}