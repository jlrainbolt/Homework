\documentclass[11pt]{article}
\usepackage{homework}

\classname{352}
\homeworknum{1}



\begin{document}

% Environments

\newcommand{\state}[2]{\begin{statement}{#1} #2 \end{statement}}
\newcommand{\prob}[2]{\begin{problem}{#1} #2 \end{problem}}
\newcommand{\subprob}[1]{\begin{subproblem} #1 \end{subproblem}}
\newcommand{\sol}[1]{\begin{solution} #1 \end{solution}}
\newcommand{\fig}[2]{\begin{figure} \centering #2  \label{#1} \end{figure}}

\newcommand{\makebib}{
	\vfill
	\color{black}
	\bibliography{references}{}
	\bibliographystyle{lucas_unsrt}
}
	

% Implication

\newcommand{\qwhere}{\quad \text{where} \quad}
\newcommand{\qimplies}{\quad \implies \quad}
\newcommand{\impliesq}{\implies \quad}



% Brackets

\newcommand{\paren}[1]{\left( #1 \right)}
\newcommand{\brac}[1]{\left[ #1 \right]}


% Greek

\newcommand{\alp}{\alpha}
\newcommand{\bet}{\beta}
\newcommand{\gam}{\gamma}
\newcommand{\del}{\delta}
\newcommand{\eps}{\epsilon}
\newcommand{\zet}{\zeta}
\newcommand{\tht}{\theta}
\newcommand{\kap}{\kappa}
\newcommand{\lam}{\lambda}
\newcommand{\sig}{\sigma}
\newcommand{\ups}{\upsilon}
\newcommand{\omg}{\omega}

\newcommand{\Gam}{\Gamma}
\newcommand{\Del}{\Delta}
\newcommand{\Tht}{\Theta}
\newcommand{\Lam}{\Lambda}
\newcommand{\Sig}{\Sigma}
\newcommand{\Omg}{\Omega}
% Problem 1

\newcommand{\Psii}{\Psi^i}
\newcommand{\Psiix}{\Psii(x)}

\newcommand{\Pii}{\Pi^i}

\newcommand{\Phii}{\Phi^i}
\newcommand{\Phiix}{\Phii(x)}
\newcommand{\PhiN}{\Phi^N}
\newcommand{\PhiNx}{\PhiN(x)}
\newcommand{\Phiq}{\Phi^1}
\newcommand{\Phiw}{\Phi^2}

\newcommand{\ddcx}{\dd[3]{x}}

\newcommand{\delij}{\del^{i j}}
\newcommand{\delkl}{\del^{k l}}
\newcommand{\delil}{\del^{i l}}
\newcommand{\deljk}{\del^{j k}}
\newcommand{\delik}{\del^{i k}}
\newcommand{\deljl}{\del^{j l}}

\newcommand{\DF}{D_F}

\newcommand{\sigx}{\sig(x)}

\newcommand{\pii}{\pi^i}
\newcommand{\pij}{\pi^j}
\newcommand{\pik}{\pi^k}
\newcommand{\pil}{\pi^l}
\newcommand{\piix}{\pi(x)}

\newcommand{\pq}{p_1}
\newcommand{\pw}{p_2}
\newcommand{\pe}{p_3}
\newcommand{\pr}{p_4}

\newcommand{\vp}{\vb{p}}
\newcommand{\vpsi}{\vp_i}

\newcommand{\mpi}{m_\pi}

\state{(Jackson 9.8)}{\ 
	%\emph{Hint:} The electromagnetic angular momentum density comes from more than the transverse (radiation zone) components of the fields.
}

%
%	Jackson 9.8(a)
%

\prob{}{
	Show that a classical oscillating electric dipole $\vp$ with fields given by
	\aln{ \label{fields1}
		\vH &= \frac{c k^2}{4\pi} (\nh \cross \vp) \frac{e^{i k r}}{r} \paren{ 1 - \frac{1}{i k r} }, &
		\vE &= \frac{1}{4\pi \epso} \curly{ k^2 (\nh \cross \vp) \cross \nh \frac{e^{i k r}}{r} + [ 3 \nh (\nh \vdot \vp) - \vp ] \paren{ \frac{1}{r^3} - \frac{i k}{r^2} } e^{i k r} },
	}
	radiates electromagnetic angular momentum to infinity at the rate
	\eq{
		\dv{\vL}{t} = \frac{k^3}{12 \pi \epso} \Im[ \vp^* \cross \vp ].
	}
	\vfix
}

\sol{
	According to Jackson~(9.20), the time-averaged angular momentum density is
	\eq{
		\vl = \frac{\Re[ \vx \cross (\vE \cross \vHs)}{2 c^2}.
	}
	One of the vector identities on the inside cover of Jackson is $\vaa \cross (\vbb \cross \vcc) = (\vaa \vdot \vcc) \vbb - (\vaa \vdot \vbb) \vcc$, so
	\eqn{l1}{
		\vl = \frac{(\vx \vdot \vHs) \vE - (\vx \vdot \vE) \vHs}{2 c^2}.
	}
	From Eq.~\refeq{fields1}, note that
	\eq{
		\vx \vdot \vHs \propto \vx \vdot (\nh \cross \vps)
		= \vps \vdot (\vx \cross \nh)
		= \vO,
	}
	where we have used the identity $\vaa \vdot (\vbb \cross \vcc) = \vcc \vdot (\vaa \cross \vbb)$ and the fact that $\nh$ points in the $\vx$ direction.  For $\vx \vdot \vE$, note that
	\al{
		\vx \vdot [ (\nh \cross \vp) \cross \nh ] &= -\vx \vdot [ \nh \cross (\nh \cross \vp) ]
		= -\vx \vdot [ (\nh \vdot \vp) \nh - (\nh \vdot \nh) \vp ]
		= -(\nh \vdot \vp) (\vx \vdot \nh) + \vx \vdot \vp \\
		&= -r (\nh \vdot \vp) + \vx \vdot \vp
		= \vx \vdot \vp - \vx \vdot \vp
		= 0, \\[1.5ex]
		\vx \vdot [ 3 \nh (\nh \vdot \vp) - \vp ] &= 3 (\vx \vdot \nh) (\nh \vdot \vp) - \vx \vdot \vp
		= 3r (\nh \vdot \vp) - \vx \vdot \vp
		= 3(\vx \vdot \vp) - \vx \vdot \vp
		= 2(\vx \vdot \vp),
	}
	since $\abs{\vx} = r$ and $\vx = r \,\nh$.  Then
	\eq{
		\vx \vdot \vE = \frac{1}{2\pi \epso} (\vx \vdot \vp) \paren{ \frac{1}{r^3} - \frac{i k}{r^2} } e^{i k r}
		= \frac{1}{2\pi \epso} (\nh \vdot \vp) \paren{ \frac{1}{r^2} - \frac{i k}{r} } e^{i k r}.
	}
	
	With these substitutions, Eq.~\refeq{l1} becomes
	\al{
		\vl &= -\frac{(\vx \vdot \vE) \vHs}{c^2}
		= -\frac{1}{4\pi \epso c^2} (\nh \vdot \vp) \paren{ \frac{1}{r^2} - \frac{i k}{r} } e^{i k r} \frac{c k^2}{4\pi} (\nh \cross \vps) \frac{e^{-i k r}}{r} \paren{ 1 + \frac{1}{i k r} } \\
		&= -\frac{k^2}{16\pi^2 \epso c r} (\nh \vdot \vp) (\nh \cross \vps) \paren{ \frac{1}{r^2} - \frac{i k}{r} } \paren{ 1 - \frac{i}{k r} }
		= -\frac{k^2}{16\pi^2 \epso c} (\nh \vdot \vp) (\nh \cross \vps) \paren{ \frac{1}{r^2} - \frac{i}{k r^3} - \frac{i k}{r} - \frac{1}{r^2} } \\
		&= -\frac{i k^2}{16\pi^2 \epso c r} (\nh \vdot \vp) (\nh \cross \vps) \paren{ \frac{1}{k r^3} + \frac{k}{r^2} }
		= \frac{i k^3}{16\pi^2 \epso c r^2} (\nh \vdot \vp) (\nh \cross \vps) \paren{ \frac{1}{k^2 r^2} + 1 }.
	}
	
	Let $\vL$ be the angular momentum radiated to a distance $R$.  Then
	\eq{
		\vL = \int_R \vl(r) \ddcx
		= \intopi \intotp \intoR \vl(r) \,r^2 \sin\tht \ddr \ddphi \dd\tht,
	}
	and the time derivative is
	\aln{
		\dv{\vL}{t} &= \dv{t}(\intopi \intotp \intoR \vl(r) \,r^2 \sin\tht \ddr \ddphi \dd\tht)
		= \dv{r}{t} \dv{r}(\intopi \intotp \intoR \vl(r) \,r^2 \sin\tht \ddr \ddphi \dd\tht) \notag \\
		&= c \intopi \intotp \vl(r) \,r^2 \sin\tht \ddphi \dd\tht
		= \frac{i k^3}{16\pi^2 \epso} \paren{ \frac{1}{k^2 r^2} + 1 } \intopi \intotp (\nh \vdot \vp) (\nh \cross \vps) \sin\tht \ddphi \dd\tht. \label{dLdt}
	}
	Note that
	\eq{
		[ (\nh \vdot \vp) (\nh \cross \vps) ]_i = \sumje n_j p_j (\nh \cross \vps)_i
		= \sumje \sumke \sumle \epsikl n_j p_j n_k p_l^*,
	}
	so
	\eq{
		\dv{L_i}{t} \propto \sumje \sumke \sumle \epsikl p_j p_l^* \int n_j p_k \ddOmg
		= \sumje \sumke \sumle \epsikl p_j p_l^* \frac{4\pi}{3} \del_{jk}
		= \frac{4\pi}{3} \epsikl p_k p_l^*
		= \frac{4\pi}{3} (\vp \cross \vps)_i,
	}
	where we have used Jackson~(9.47), $\int n_\bet n_\gam \ddOmg = 4\pi \del_{\bet \gam} / 3$.  Making this substitution into Eq.~\refeq{dLdt},
	\eq{
		\dv{\vL}{t} = \frac{i k^3}{6\pi \epso} \paren{ \frac{1}{k^2 r^2} + 1 } (\vp \cross \vps).
	}
	Taking the limit as $r \to \infty$, we find
	\eqn{ans1a}{
		\dv{\vL}{t} = \Re\!\brac{ \frac{i k^3}{12\pi \epso} (\vp \cross \vps) }
		= \Re\!\brac{ -\frac{i k^3}{12\pi \epso} (\vps \cross \vp) }
		= \ans{ \frac{k^3}{12\pi \epso} \Im[ \vps \cross \vp ], }
	}
	as desired. \qed
}

%
%	Jackson 9.8(b)
%

\prob{}{
	What is the ratio of angular momentum radiated to energy radiated?  Interpret.
}

\sol{
	According to Jackson~(9.24), the total power radiated by an oscillating electric dipole $\vp$ is
	\eq{
		P = \dv{E}{t}
		= \frac{c^2 \Zo k^4}{12 \pi} \abs{\vp}^2.
	}
	Then the ratio of angular momentum radiated to energy radiated is
	\eq{
		\frac{\dv*{\vL}{t}}{\dv*{E}{t}} = \frac{k^3}{12\pi \epso} \Im[ \vps \cross \vp ] \frac{12 \pi}{c^2 \Zo k^4 \abs{\vp}^2}
		= \frac{1}{\epso} \Im[ \vps \cross \vp ] \frac{1}{c^2 \Zo k \abs{\vp}^2}
		= \ans{ \frac{\Im[ \vps \cross \vp ]}{\omg \abs{\vp}^2}, }
	}
	where we have used $\Zo = \sqrt{\muo / \epso} = 1 / \sqrt{\epso^2 c^2} = 1 / \epso c$, $c^2 = 1 / (\epso \muo)$, and $\omg = k c$.
	
	In the limit of high frequency, $(\dv*{\vL}{t}) / (\dv*{E}{t}) \to 0$.  In this scenario, the energy radiated dominates over the angular momentum radiated.  Likewise, in the limit of low frequency, $(\dv*{\vL}{t}) / (\dv*{E}{t}) \to \infty$, meaning that angular momentum radiation dominates.  This is sensible because rotational kinetic energy $E \propto \omg^2$, while angular momentum $L \propto \omg$.
}

%
%	Jackson 9.8(c)
%

\prob{}{
	For a charge $e$ rotating in the $xy$ plane at radius $a$ and angular speed $\omg$, show that there is only a $z$ component of radiated angular momentum with magnitude $\dv*{\Lz}{t} = e^2 k^3 a^2 / 6 \pi \epso$.  What about a charge oscillating along the $z$ axis?
}

\sol{
	We know from Homework~5 that the position of a point charge rotating counterclockwise in the $xy$ plane is
	\eq{
		\vx(t) = a \cos(\omg t) \,\vx + a \sin(\omg t) \,\yh.
	}
	\clearpage
	Then the charge distribution is
	\eq{
		\rho(\vx, t) = e \del[ x - a \cos(\omg t) ] \,\del[ y - a \sin(\omg t) ] \,\del(z).
	}
	
	According to Jackson~(4.8), the dipole moment is defined
	\eq{
		\vp = \int \vx' \,\rho(\vx') \ddcxp.
	}
	The components of $\vp$ for the point charge are then
	\al{
		\px &= e \iiint x \,\del[ x - a \cos(\omg t) ] \,\del[ y - a \sin(\omg t) ] \,\del(z) \ddx \ddy \ddz
		= e a \cos(\omg t), \\
		\py &= e \iiint y \,\del[ x - a \cos(\omg t) ] \,\del[ y - a \sin(\omg t) ] \,\del(z) \ddx \ddy \ddz
		= e a \sin(\omg t), \\
		\pz &= e \iiint z \,\del[ x - a \cos(\omg t) ] \,\del[ y - a \sin(\omg t) ] \,\del(z) \ddx \ddy \ddz
		= 0,
	}
	so we can write $\vp = e a \,e^{-i \omg t} (\xh + i\,\yh).$  Substituting into Eq.~\refeq{ans1a},
	\al{
		\dv{\vL}{t} &= \Re\!\brac{ \frac{i k^3}{12\pi \epso} e^2 a^2 e^{-i \omg t} e^{i \omg t} [ (\xh + i\,\yh) \cross (\xh - i\,\yh) ] }
		= \Re\!\brac{ \frac{i e^2 k^3 a^2}{12\pi \epso} (-2i \,\xh \cross \yh) }
		= \Re\!\brac{ \frac{e^2 k^3 a^2}{6\pi \epso} \,\zh } \\
		&= \ans{ \frac{e^2 k^3 a^2}{6\pi \epso} \cos(\omg t) \,\zh, }
	}
	as desired. \qed
	
	A charge oscillating along the $z$ axis with amplitude $a$ has the charge density
	\eq{
		\rho(\vx, t) = e a \,\del(x) \,\del(y) \,\del[ z - \cos(\omg t) ],
	}
	which gives the dipole moment
	\al{
		\px &= e a \iiint x \,\del(x) \,\del(y) \,\del[ z - \cos(\omg t) ] \ddx \ddy \ddz
		= 0, \\
		\py &= e a \iiint y \,\del(x) \,\del(y) \,\del[ z - \cos(\omg t) ] \ddx \ddy \ddz
		= 0, \\
		\pz &= e a \iiint z \,\del(x) \,\del(y) \,\del[ z - \cos(\omg t) ] \ddx \ddy \ddz
		= e a \cos(\omg t).
	}
	In complex notation, $\vp = e a \,e^{-i\omg t} \,\zh$.  Substituting into Eq.~\refeq{ans1a}, we find
	\eq{
		\dv{\vL}{t} = \Re\!\brac{ \frac{i k^3}{12\pi \epso} e^2 a^2 e^{-i \omg t} e^{i \omg t} (\zh \cross \zh) }
		= \ans{ \vO. }
	}
	So we see that a charge undergoing linear motion does not lead to a radiated angular momentum, which is sensible.
}

%
%	Jackson 9.8(d)
%

\prob{}{
	What are the results corresponding to Probs.~{1(a)} and {1(b)} for magnetic dipole radiation?
}

\sol{
	The radiation fields for a magnetic dipole are given by Jackson~(19.35--36),
	\al{
		\vH &= \frac{1}{4\pi} \curly{ k^2 (\nh \cross \vm) \cross \nh \frac{e^{i k r}}{r} + [ 3 \nh (\nh \vdot \vm) - \vm ] \paren{ \frac{1}{r^3} - \frac{i k}{r^2} } e^{i k r} }, &
		\vE &= -\frac{\Zo}{4\pi} k^2 (\nh \cross \vm) \frac{e^{i k r}}{r} \paren{ 1 - \frac{1}{i k r} }.
	}
	\clearpage
	Comparing with Eq.~\refeq{fields1}, we see that $\vH \to -\vE / \Zo$, $\vE \to \Zo \vH$, and $\vp \to \vm / c$ as stated in the book~\cite[p.~413]{Jackson}.  Making these substitutions, the results of Probs.~{1.1(a)} and {(b)} become
	\al{
		\ans{ \dv{\vL}{t}\ }&\ans{= \frac{\muo k^3}{12\pi} \Im[ \vms \cross \vm ], } &
		\ans{ \frac{\dv*{\vL}{t}}{\dv*{E}{t}}\ }&\ans{= \frac{\Im[ \vms \cross \vm ]}{\omg \abs{\vm}^2} }
	}
	where we have used $\mu = 1 / \epso c^2$.
}

\clearpage
\state{Beta function of the Gross-Neveu model~(P\&S~12.2)}{
	Compute $\bet(g)$ in the two-dimensional Gross-Neveu model studied in Problem~11.3,
	\eq{
		\cL = \psibsi i \ptsl \psisi + \frac{1}{2} g^2 (\psibsi \psisi)^2,
	}
	with $i = 1, \ldots, N$.  You should find that this model is asymptotically free.  How was that fact reflected in the solution to Problem~11.3?
}

\sol{
	We saw in Problem~2 of Homework~4 that this Lagrangian can be written as
	\eq{
		\cL = \psibsi i \ptsl \psisi - \sig \psibsi \psisi - \frac{1}{2 g^2} \sig^2,
	}
	where $\sig$ is a new scalar field with no kinetic energy terms.  In the modified minimal subtraction scheme, we found the effective potential was
	\eqn{Veff}{
		\Veff = \sig^2 \curly{ \frac{1}{2 g^2} + \frac{N}{4\pi} \brac{ \ln(\frac{\sig^2}{M^2}) - 1 } }.
	}
	Since $\Gam[ \phicl ] = -(V T) \Veff(\phi)$ by P\&S~(11.50), we have
	\eqn{Gam}{
		\Gam[ \sigcl ] = -(V T)  \sig^2 \curly{ \frac{1}{2 g^2} + \frac{N}{4\pi} \brac{ \ln(\frac{\sig^2}{M^2}) - 1 } }.
	}
	Referring to p.~3 of Lecture~11, we can apply the Callan-Symanzik equation to $\Gam$.   The Callan-Symanzik equation is P\&S~(12.41),
	\eq{
		\brac{ M \pdv{M} + \bet(\lam) \pdv{\lam} + n \gam(\lam) } G^{(n)}(\{ x_i \}; M, \lam) = 0.
	}
	For our problem, $\gam$ is 0 because there are no field insertions.  That is, we have
	\eq{
		\brac{ M \pdv{M} + \bet(g) \pdv{g} } \Gam[ \phicl ] = 0.
	}
	Using Eq.~\refeq{Gam}, note that
	\al{
		\pdv{\Gam}{M} &= (V T) \frac{N \sig^2}{2 \pi M}, &
		\pdv{\Gam}{g} &= (V T) \frac{\sig^2}{g^3}.
	}
	Then
	\eq{
		0 = (V T) \paren{ \frac{N \sig^2}{2 \pi} + \bet(g) \frac{\sig^2}{g^3} }
		\qimplies
		\ans{ \betg = -\frac{N g^3}{2\pi}. }
	}
	This model is asymptotically free because the $\bet$ function is proportional to $-g^3$~\cite[pp.~424--425]{Peskin}.
	
	In 2(e) of Homework~4, we found that the vacuum expectation value of $\sig$ was
	\eq{
		\sig = \pm M e^{-\pi / N g^2} = \pm v.
	}
	We showed that the vacuum expectation value does not depend on the renormalization condition chosen.  This means that we can increase $M \to 0$ while holding $\sig$ constant, and see that $g \to 0$ logarithmically.  This is indicative of an asymptotically-free theory~\cite[p.~425]{Peskin}. \qed
}



\state{Quantum diatomic ideal gas}{}

\prob{Classical system}{
	An ideal diatomic gas consists of non-interacting identical molecules $H = \sumiN \hii$ which have three independent degrees of freedom $h = \hK + \hV + \hR$. The first one is the kinetic energy of translational motion $\hK = \vp^2 / 2m$. The second is vibrational, i.e.~each molecule is an oscillator with ${\hV = \pi^2/2 + \omg^2 q^2 / 2}$. The third is rotational $\hR = \vL^2 / 2 I$, where $\vL$ is the angular momentum.  These three d.o.f.~can be treated independently.  Treat them as independent subsystems.
}

%
%	3.1
%

\sol{
	Let $(\pi, q) \to (\ppV, \qV)$ to avoid confusion.  For the rotational degrees of freedom, we will use the coordinates $\tht, \phi$ and the corresponding momenta $\ptht, \pphi$.  The coordinates are distinct since they concern different subsystems.  The Hamiltonians for each of the subsystems are
	\al{
		\HK &= \sumiN \frac{\vpii^2}{2m}, &
		\HV &= \sumiN \frac{{\ppV}_i^2}{2} + \frac{\omg^2 {\qV}_i^2}{2}, &
		\HR &= \sumiN \frac{1}{2I} \paren{ {\ptht}_i^2 + \frac{{\pphi}_i^2}{\sin^2\tht_i} }
	}
	\clearpage
	All three subsystems have per-particle energies that are not quantized.  The general expression for the partition function in this case is~\cite[p.~55-56]{Pathria}
	\eq{
		Z = \frac{1}{N!} \prodiN \Zii
		= \frac{1}{N!} \paren{ \frac{1}{(2\pi\hbar)^s} \iint e^{-\bet h} \dd[s]{p} \dd[s]{q} }^N,
	}
	where $s$ is the number of degrees of freedom of the particle.  The kinetic subsystem has three d.o.f., the vibrational subsystem has one d.o.f., and the rotational subsystem has two d.o.f.~(since a diatomic molecule is azimuthally symmetric).
	
	For each of the three subsystems, we find the single-particle partition functions~\cite[p.~55-56, 65]{Pathria}~\cite[p.~160]{Kardar}
	\al{
		\ZKii &= \frac{1}{(2\pi\hbar)^3} \iint e^{-p^2 / 2 m T} \dcp \dcq
		= \frac{4\pi V}{(2\pi\hbar)^3} \intoi p^2 e^{-p^2 / 2 m T} \dcp
		= \frac{V}{(2\pi\hbar)^3} \paren{ \frac{2\pi m}{\bet} }^{3/2}
		= \frac{V}{\hbar^3} \paren{ \frac{m T}{2\pi} }^{3/2}, \\[2ex]
		\ZVii &= \frac{1}{2\pi\hbar} \iint \exp(-\bet \frac{\ppV^2 + \omg^2 \qV}{2}) \dd{\ppV} \dd{\qV}
		= \frac{1}{2\pi\hbar} \sqrt{\frac{2\pi}{\bet \omg^2}} \sqrt{\frac{2\pi}{\bet}}
		= \frac{T}{\hbar \omg}, \\[2ex]
		\ZRii &= \frac{1}{(2\pi\hbar)^2} \int_0^\pi \dd{\tht} \int_0^{2\pi} \dd{\phi} \iint \exp[-\frac{\bet}{2I} \paren{ \ptht^2 + \frac{\pphi^2}{\sin^2\tht} }] \dd{\ptht} \dd{\pphi}
		= \frac{2\pi I}{\bet} \frac{4\pi}{(2\pi\hbar)^2}
		= \frac{2 I T}{\hbar^2},
	}
	so the partition functions for each system are
	\al{
		\ZK &= \frac{1}{N!} \brac{ \frac{V}{\hbar^3} \paren{ \frac{m T}{2\pi} }^{3/2} }^N, &
		\ZV &= \frac{1}{N!} \paren{ \frac{T}{\hbar \omg} }^N, &
		\ZR &= \frac{1}{N!} \paren{ \frac{2 I T}{\hbar^2} }^N.
	}
	
	Let $\EK$, $\EV$ and $\ER$ be the total energies of the respective subsystems.  The energy at equilibrium can be found by Eq.~\refeq{EfromZ}.  This yields
	\aln{
		\EK &= -\pdv{\bet} \ln \ZK
		= -\pdv{\bet} \brac{ N \ln(\frac{V}{\hbar^3}) + \frac{3 N}{2} \ln(\frac{m}{2\pi}) - \frac{3 N}{2} \ln \bet - \ln N! }
		= -\frac{3}{2} \frac{N}{\beta}
		= \frac{3}{2} N T, \label{EK3.1} \\[2ex]
		\EV &= -\pdv{\bet} \ln \ZV
		= -\pdv{\bet} \paren{ -N \ln(\hbar \omg) - N \ln \bet - \ln N! }
		= \frac{N}{\bet}
		= N T, \label{EV3.1} \\[2ex]
		\ER &= -\pdv{\bet} \ln \ZR
		= -\pdv{\bet} \paren{ N \ln(\frac{2 I}{\hbar^2}) - N \ln \bet - \ln N!}
		= \frac{N}{\bet}
		= NT. \label{ER3.1}
	}
	\vfix
}

%
%	3.1.1
%

\subprob{}{
	Compute for each subsystem the equilibrium value of entropy as a function of energy.
}

\sol{
	The entropy can be found by Eq.~\refeq{SfromZ}.  For each subsystem, we find
	\aln{
		\SK &= \pdv{T} (T \ln \ZK)
		= \ln \ZK + T \pdv{T}(\ln \ZK) \notag \\
		&= N \ln(\frac{V}{\hbar^3} \paren{ \frac{mT}{2\pi}}^{3/2}) - \ln N! + T \pdv{T} \brac{ N \ln(\frac{V}{\hbar^3}) + \frac{3 N}{2} \ln(\frac{m}{2\pi}) + \frac{3 N}{2} \ln T - \ln N! } \notag \\
		&\approx N \ln(\frac{V}{\hbar^3} \paren{ \frac{mT}{2\pi}}^{3/2}) - N \ln N + N + T \frac{3}{2} \frac{N}{T}
		= \ans{ N \ln(\frac{V}{N \hbar^3} \paren{ \frac{m \EK}{3N\pi}}^{3/2}) + \frac{5N}{2}, } \label{SK3.1}
	}
	\aln{
		\SV &= \pdv{T} (T \ln \ZV)
		= \ln \ZV + T \pdv{T}(\ln \ZV)
		= N \ln(\frac{T}{\hbar\omg}) - \ln N! + T \pdv{T} \brac{ N \ln(T) - N \ln(\hbar \omg) - \ln N! } \notag \\
		&\approx N \ln(\frac{T}{\hbar\omg}) - N \ln N + N + T \frac{N}{T}
		= \ans{ N \ln(\frac{\EV}{N^2 \hbar\omg}) + 2N, } \label{SV3.1} \\[2ex]
		\SR &= \pdv{T} (T \ln \ZR)
		= \ln \ZR + T \pdv{T}(\ln \ZR)
		= N \ln(\frac{2 I T}{\hbar^2}) - \ln N! + T \pdv{T} \paren{ N \ln(\frac{2 I}{\hbar^2}) + N \ln T - \ln N!} \notag \\
		&\approx N \ln(\frac{2 I T}{\hbar^2}) - N \ln N + N + T \frac{N}{T}
		= \ans{ N \ln(\frac{2 I \ER}{N^2 \hbar^2}) + 2N, } \label{SR3.1}
	}
	where we have eliminated $T$ using Eqs.~(\ref{EK3.1}--\ref{ER3.1}).
}

%
%	3.1.2
%

\subprob{}{
	Compute for each subsystem the equilibrium value of energy as a function of entropy.
}

\sol{
	We need to solve Eqs.~(\ref{SK3.1}--\ref{SR3.1}) for $\EK$, $\EV$, and $\ER$:
	\al{
%		\frac{\SK}{N} - \frac{5}{2} &= \ln(\frac{V}{N \hbar^3} \paren{ \frac{m \EK}{3N\pi}}^{3/2})
%		\qimplies
		e^{\SK / N - 5/2} = \frac{V}{N \hbar^3} \paren{ \frac{m \EK}{3N\pi}}^{3/2}
		&\qimplies
		\ans{ \EK = \frac{3N \pi}{m} \paren{ \frac{N \hbar^3}{V} e^{\SK / N - 5/2} }^{2/3}, } \\[2ex]
%		\frac{\SV}{N} - 2 = \ln(\frac{\EV}{N^2 \hbar\omg})
%		\qimplies
		e^{\SV / N - 2} = \frac{\EV}{N^2 \hbar\omg}
		&\qimplies
		\ans{ \EV = N^2 \hbar \omg e^{\SV / N - 2}, } \\[2ex]
%		\frac{\SR}{N} - 2 = \ln(\frac{2 I \ER}{N^2 \hbar^2})
%		\qimplies
		e^{\SR / N - 2} = \frac{2 I \ER}{N^2 \hbar^2}
		&\qimplies
		\ans{ \ER = \frac{N^2 \hbar^2}{2I} e^{\SR / N - 2}. }
	}
	\vfix
}

%
%	3.1.3
%

\subprob{}{
	Compute for each subsystem the equilibrium value of entropy as a function of temperature.
}

\sol{
	We already did this work in Prob.~{3.1.1}, where we found
	\al{
			\ans{ \SK\ } &\ans{\approx N \ln(\frac{V}{N \hbar^2} \paren{ \frac{mT}{2\pi} }^{3/2}) + \frac{5N}{2}, } &
			\ans{ \SV\ } &\ans{\approx N \ln(\frac{T}{N \hbar \omg}) + 2N, }&
			\ans{ \SR\ } &\ans{\approx N \ln(\frac{2 I T}{N \hbar^2}) + 2N. }
	}
	\vfix
}

\subprob{}{
	Compute for each subsystem the equilibrium value of free energy as a function of temperature.
}

\sol{
	Recall from Prob.{2.1} that $F = -T \ln Z$.  Then we have
	\al{
		\FK &= -T \brac{ N \ln(\frac{V}{\hbar^3} \paren{ \frac{m T}{2\pi} }^{3/2}) - \ln N! }
		\approx \ans{ NT \brac{ \ln(\frac{N \hbar^3}{V} \paren{ \frac{m T}{2\pi} }^{3/2}) - 1 } }, \\[2ex]
		\FV &= -T \brac{ N \ln(\frac{T}{\hbar \omg}) - \ln N! }
		\approx \ans{ NT \brac{ \ln(\frac{N \hbar \omg}{T}) - 1 }, } \\[2ex]
		\FR &= -T \brac{ N \ln(\frac{2 I T}{\hbar^2}) - \ln N! }
		\approx \ans{ NT \brac{ \ln(\frac{N \hbar^2}{2 I T}) - 1 }. }
	}
	\vfix
}

\prob{Quantum system}{
	Now consider all systems as quantum and repeat the calculations.  This means that the momentum $\vp$ is quantized, each component of momentum taking the values $\pk = (2 \pi \hbar / L) k$, where $k$ is an arbitrary integer and $L$ is the linear size of the box.  Similarly, the
energy of the vibrational modes is quantized as $\En = \hbar \omg (n + 1 / 2)$, and the square of the angular momentum as $L^2 = \hbar^2 l (l + 1)$, where $l$ is a non-negative integer.
}

\subprob{}{
	Compute for each subsystem the equilibrium value of entropy as a function of energy.
}

\subprob{}{
	Compute for each subsystem the equilibrium value of energy as a function of entropy.
}

\subprob{}{
	Compute for each subsystem the equilibrium value of entropy as a function of temperature.
}

\subprob{}{
	Compute for each subsystem the equilibrium value of free energy as a function of temperature.
}

\subprob{}{
	Discuss the quantum (low temperature) and the classical (high temperature) limits.
}

\makebib

\end{document}