\state{Landau diamagnetism}{\ }

%
%	3.1
%

\prob{}{
	Compute the Landau diamagnetic susceptibility for an ultra-relativistic Fermi gas in a weak field.
}

\sol{
	We will assume an electron gas.  The energy levels $\eps$ for a relativistic electron in magnetic field $\vB = B \,\zh$ are given by
	\eq{
		\eps^2 - m^2 - \pz^2 = e B (2n + 1) + e B \sig,
	}
	where $\sig = \pm 1$ are the eigenvalues of $\sig_z$ and $n = 0, 1, 2, \ldots$ ~\cite[p.~101]{Relativistic}.  Then
	\eq{
		\eps^2 = e B (2n + 1 + \sig) + m^2 + \pz^2.
	}
	In order to take $\sig$ into account, we renumber the states such that $\eps^2 = 2n e B + \pz^2$, where $n$th state has degeneracy 2 for $n > 0$, and the 0th state is not degenerate.  Then
	\eq{
		\eps = \pm c \sqrt{2n e B + m^2 c^2 + \pz^2},
	}
	which has no degeneracy, since the $\pm$ sign splits the degenerate levels, and we have inserted factors of $c$.  In practice, however, we need only consider the positive energy levels, since the negative ones have exponentially suppressed contributions at small $T$.
	
	The number of states in the interval $\ddpz$ for a given $\eps$ is the same as in the nonrelativistic case~\cite[p.~173]{Landau2}:
	\eqn{deg}{
		\frac{2 e V B}{(2\pi\hbar)^2 c} \ddpz,
	}
	where $2 = g$ for the electron.
	
	The relevant thermodynamic potential for a nonrelativistic electron gas, for which $\eps = \pz^2 / 2m + (2n + 1) \muB B$, is~\cite[p.~173]{Landau2}
	\aln{ \label{thing3}
		\Omg &= 2\muB B \sumni f[ \mu - (2n + 1) \muB B ], &
		\qq{where}
		f(\mu) &= -\frac{T m V}{2\pi^2 \hbar^3} \intnii \ln\!\brac{ 1 + \exp(\frac{\mu}{T} - \frac{\pz^2}{2m T}) } \ddpz.
	}
	For the ultra-relativistic gas, then,
	\aln{
		\Omg &= -\frac{\muB B T m V}{\pi^2 \hbar^3} \sumni \intnii \ln\!\brac{ 1 + \exp(\frac{\mu - c \sqrt{2n e B + m^2 c^2 + \pz^2}}{T}) } \ddpz \notag \\
		&= -\frac{\muB B T m V}{\pi^2 \hbar^3} \sumni \intoi \ln\!\brac{ 1 + \exp(\frac{\mu - c \sqrt{2n e B + m^2 c^2 + \pz^2}}{T}) } \ddpz, \label{Omg}
	}
	where we have used that fact that $\pz^2 = (-\pz)^2$.
	
	The Euler-Maclaurin formula is~\cite[p.~173]{Landau2}
	\eq{
		\frac{1}{2} F(a) + \sum_{n = 1}^\infty F(a + n) \approx \int_a^\infty F(x) \ddx - \frac{1}{12} F'(a).
	}
	In our case $a = 0$, so
	\eqn{sum}{
		\sum_{n = 0}^\infty F(n) \approx \intoi F(x) \ddx + \frac{1}{2} F(0) - \frac{1}{12} F'(0),
	}
	where we have added $F(0) / 2$ to both sides.  From Eq.~\refeq{Omg}, note that
	\al{
		F(0) &= -\frac{2 \muB B T m V}{\pi^2 \hbar^3} \intoi \ln\!\brac{ 1 + \exp(\frac{\mu - c \sqrt{m^2 c^2 + \pz^2}}{T}) } \ddpz, \\[2ex]
		F'(0) &= -\frac{2 \muB B T m V}{\pi^2 \hbar^3}  \brac{ \dv{n} \intoi \ln\!\curly{ 1 + \exp(\frac{\mu - c \sqrt{2n e B + m^2 c^2 + \pz^2}}{T}) } \ddpz }_{n = 0} \\
		&= -\frac{2 \muB B T m V}{\pi^2 \hbar^3} \brac{ -\intoi \frac{c e B}{T \sqrt{2 n e B + m^2 c^2 + \pz^2}} \curly{ \exp(\frac{c \sqrt{2 n e B + m^2 c^2 + \pz^2} - \mu}{T}) + 1}^{-1} \ddpz }_{n = 0} \\
		&= \frac{2 \muB B T m V}{\pi^2 \hbar^3} \frac{c e B}{T} \intoi \frac{\ddpz}{\sqrt{m^2 c^2 + \pz^2} (e^{(c \sqrt{m^2 c^2 + \pz^2} - \mu) / T} + 1)} \\
		&= \frac{4 m^2 c^2 \muB^2 B^2 V}{\pi^2 \hbar^4} \intoi \frac{\ddpz}{\sqrt{m^2 c^2 + \pz^2} (e^{(c \sqrt{m^2 c^2 + \pz^2} - \mu) / T} + 1)}.
	}
	Then
	\al{
		\Omg &= -\frac{2\muB B T m V}{\pi^2 \hbar^3} \intoi \intoi \ln\!\brac{ 1 + \exp(\frac{\mu - c \sqrt{2 e B x + m^2 c^2 + \pz^2}}{T}) } \ddpz \ddx \\
		&\phantom{mmmmmmmmmmmmm} - \frac{\muB B T m V}{\pi^2 \hbar^3} \intoi \ln\!\brac{ 1 + \exp(\frac{\mu - c \sqrt{m^2 c^2 + \pz^2}}{T}) } \ddpz \\
		&\phantom{mmmmmmmmmmmmmmmmmmmmmm} + \frac{m^2 c^2 \muB^2 B^2 V}{3 \pi^2 \hbar^4} \intoi \frac{\ddpz}{\sqrt{m^2 c^2 + \pz^2} (e^{(c \sqrt{m^2 c^2 + \pz^2} - \mu) / T} + 1)}.
	}
	
	The magnetic moment of the gas is $M = -(\pdv*{\Omg}{B})_{T, V, \mu}$~\cite[p.~172]{Landau}, so
	\al{
		M &= \frac{2\muB T m V}{\pi^2 \hbar^3} \intoi \intoi \ln\!\brac{ 1 + \exp(\frac{\mu - c \sqrt{2 e B x + m^2 c^2 + \pz^2}}{T}) } \ddpz \ddx \\
		&\phantom{mmmmmmmmmmmmm} + \frac{\muB T m V}{\pi^2 \hbar^3} \intoi \ln\!\brac{ 1 + \exp(\frac{\mu - c \sqrt{m^2 c^2 + \pz^2}}{T}) } \ddpz \\
		&\phantom{mmmmmmmmmmmmmmmmmmmmmm} - \frac{2 m^2 c^2 \muB^2 B V}{3 \pi^2 \hbar^4} \intoi \frac{\ddpz}{\sqrt{m^2 c^2 + \pz^2} (e^{(c \sqrt{m^2 c^2 + \pz^2} - \mu) / T} + 1)},
	}
	and the diamagnetic susceptibility is $\chidia = (\pdv*{M}{B}) / V$, so
	\eq{
		\chidia = -\frac{2 m^2 c^2 \muB^2}{3 \pi^2 \hbar^4} \intoi \frac{\ddpz}{\sqrt{m^2 c^2 + \pz^2} (e^{(c \sqrt{m^2 c^2 + \pz^2} - \mu) / T} + 1)}.
	}
	At the limit $T = 0$, the gas is completely degenerate and the occupation number $\evn = 1$ for $\pz < \po$, $\evn = 0$ for $\pz > \po$ where $\po$ is the Fermi momentum~\cite[p.~357]{Landau2}.  Transforming to a Fermi-Dirac integral and taking this limit,
	\al{
		\chidia &= -\frac{2 m^2 c^2 \muB^2}{3 \pi^2 \hbar^4} \frac{c}{T} \int_a^\infty \frac{\ddz}{z (e^{z - \mu / T} + 1)}
		\approx -\frac{2 m^2 c^3 \muB^2}{3 \pi^2 \hbar^4 T} \int_a^b \frac{\ddz}{z} \\
		&= -\frac{2 m^2 c^3 \muB^2}{3 \pi^2 \hbar^4 T} \brac{ \ln(\frac{c \sqrt{m^2 c^2 + \po^2}}{T}) - \ln(\frac{m c^2}{T}) }
		= -\frac{2 m^2 c^3 \muB^2}{3 \pi^2 \hbar^4 T} \ln(\frac{\sqrt{m^2 c^2 + \po^2}}{m c^2}),
	}
	where $z = c \sqrt{m^2 c^2 + p^2} / T$, $a = m c^2 / T$, and $b = c \sqrt{m^2 c^2 + \po^2} / T$.

	In the ultra-relativistic limit, $m \to 0$.  Taking the Taylor series expansion of the argument for small $m$ using Mathematica, we have
	\eq{
		\frac{\sqrt{m^2 c^2 + \po^2}}{m c^2} \approx \frac{\po}{m c} + \frac{m c}{2 \po}.
	}
	Then
	\eq{
		\ans{ \chidia \approx -\frac{2 m^2 c^3 \muB^2}{3 \pi^2 \hbar^4 T} \ln(\frac{\po}{m c} + \frac{m c}{2 \po})
		= -\frac{c e^2}{6 \pi^2 \hbar^2 T} \ln(\frac{\po}{m c} + \frac{m c}{2 \po}). }
	}
	\vfix
}


%
%	3.2
%

\prob{\hspace{-1em}*}{
	Compute the Landau diamagnetic susceptibility for a Fermi gas confined to a box whose linear size in the $z$ direction is $\Lz \ll \Lx, \Ly$.  The magnetic field is directed along the $z$ direction.  Consider two cases when the energy spacing $(2\pi \hbar / \Lz)^2 / 2m$ is much larger/smaller than the cyclotron energy $\muB B$.
}

\sol{
	Once again, we will consider an electron gas.  For this geometry, the energy levels are
	\eq{
		\eps = \hbar \omg (n + 1/2) + \frac{2 \pi^2 \hbar^2 {n'}^2}{m \Lz^2},
	}
	where $\omg = e B / m c$ is the cyclotron frequency, $n = 0, 1, 2, \ldots$, and $n' = 0, \pm 1, \pm 2, \ldots$~\cite[pp.~3--4]{Lopez}.  In terms of the quantities used in Probs.~{4} and {5.1},
	\eqn{eps3.2}{
		\eps = \muB B (2n + 1) + \paren{ \frac{2 \pi \hbar}{\Lz} }^2 \frac{{n'}^2}{2 m}.
	}
	The degeneracy of a state with a given $n$ is $\Lx \Ly e B / 2 \pi \hbar c$, which is the same as in the textbook case~\cite[p.~243]{Pathria}.  In notation similar to that of Eq.~\refeq{thing3}, we have
	\eq{
		\Lx \Ly \frac{e B}{2\pi \hbar c} = \Lx \Ly \frac{2 m c \muB B}{2\pi \hbar^2 c}
		= \Lz \Ly \frac{m T}{\pi \hbar^2}.
	}
	So the analogue of Eq.~\refeq{thing3} in this case is
	\aln{
		\Omg &= 2\muB B \sumni f[ \mu - (2n + 1) \muB B ], \label{thing3.2} \\
		\qq{where}
		f(\mu) &= -2 \Lx \Ly \frac{m T}{\pi \hbar^2} \sum_{n' = 0}^\infty \ln\!\brac{ 1 + \exp(\frac{\mu}{T} - \paren{ \frac{2 \pi \hbar}{\Lz} }^2 \frac{{n'}^2}{2m T}) } \notag,
	}
	where in the second expression we written the sum over $n' \in (-\infty, \infty)$ as twice the sum over $n' \in (0, \infty)$, since the expression depends only on ${n'}^2$.
	
	In the case where the energy spacing is much smaller than the cyclotron energy, it is sufficient to replace the sum in $f(\mu)$ by an integral.  In this limit,
	\eq{
		f(\mu) \approx -\Lx \Ly \frac{m T}{\pi \hbar^2} \intnii \ln\!\brac{ 1 + \exp(\frac{\mu}{T} - \paren{ \frac{2 \pi \hbar}{\Lz} }^2 \frac{{n'}^2}{2m T}) } \dd{n'} \\
		= -\frac{T m V}{ 2\pi^2 \hbar^3} \intnii \ln\!\brac{ 1 + \exp(\frac{\mu}{T} - \frac{\pz^2}{2m T}) } \ddpz,
	}
	where we defined $\pz = 2\pi \hbar n' / \Lz$ and $V = \Lx \Ly \Lz$.  This is exactly the same as the expression for the textbook case in which does not impose the restriction $\Lz \ll \Lx, \Ly$, so the diamagnetic susceptibility is~\cite[pp.~173--174]{Landau2}
	\eq{
		\ans{ \chidia = -\frac{\muB^2 \po m}{3 \pi^2 \hbar^3}
		= -\frac{\chipara}{3} }
	}
	when the energy spacing is much smaller than the cyclotron energy.
	
	In the case where the energy spacing is much larger than the cyclotron energy, we can replace the expression of $\Omg$ in Eq.~\refeq{thing3.2} with an integral and use the Euler-Maclaurin formula to handle the sum in $F(\mu)$.  Applying Eq.~\refeq{sum} to $f(\mu)$, note that
	\al{
		F'(0) &= \brac{ \pdv{n'} \ln\!\curly{ 1 + \exp(\frac{\mu}{T} - \paren{ \frac{2 \pi \hbar}{\Lz} }^2 \frac{{n'}^2}{2m T}) } }_{n' = 0} \\
		&= \brac{ \paren{ \frac{2 \pi \hbar}{\Lz} }^2 \frac{n'}{2m T} \curly{ 1 + \exp(\frac{4 \pi^2 \hbar^2}{\Lz^2} \frac{{n'}^2}{2m T} - \frac{\mu}{T}) }^{-1} }_{n' = 0}
		= 0, \\[1.1ex]
		F(0) &= \ln(1 + e^{\mu / T}).
	}
	Then we have
	\al{
		\Omg &= -4 \Lx \Ly \frac{m T}{\pi \hbar^2} \muB B \intoi \Bigg\{ \intoi \ln\!\brac{ 1 + \exp(\frac{\mu}{T} - (2x + 1) \frac{\muB B}{T} - \paren{ \frac{2 \pi \hbar}{\Lz} }^2 \frac{{n'}^2}{2m T}) } \dd{n'} \\
		&\phantom{mmmmmmmmmmmmmmmmmmmmmmmmmmmmmmmmmmm} + \ln(1 + e^{\mu / T - (2x + 1) \muB B}) \Bigg\} \ddx,
	}
	so
	\eq{
		\ans{ \chidia = \frac{1}{V} \pdv{M}{B}
		= -\frac{1}{V} \pdv[2]{\Omg}{B}
		= 0 }
	}
	when the energy spacing is much larger than the cyclotron energy.
}