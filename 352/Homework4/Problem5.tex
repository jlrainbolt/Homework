\state{Pair correlation function}{\ }

%
%	5.1
%

\prob{}{
	Compute the  pair correlation of density $C(r)= \ev{\ev{n(r) \,n(0)}}$ and the fluctuation of the occupation number $\ev*{\absnk^2}$ of the degenerate Fermi gas ($T \ll \EF$) in 2D.  Discuss various distance regimes.
}

\sol{
	The spatial correlation of the density fluctuations in a 2D Fermi gas is given by
	\eq{
		\ev{\Del\nq \,\Del\nw} = \frac{1}{A^2} {\sum_{\sig, \vp, \vp'}}\!' (1 - \ev*{\npps}) \ev{\nps} e^{i (\vp - \vp') (\vrw - \vrq) / \hbar},
	}
	where $n$ is an occupation number, $\vp$ and $\vp'$ are momenta, and $\sig$ is a spin component~\cite[p.~356]{Landau2}.  We can approximate the sum by an integral using the momentum elements $A \ddsp / (2\pi\hbar)^2$ and $A \ddspp / (2\pi\hbar)^2$:
	\eqn{thing5}{
		\ev{\Del\nq \,\Del\nw} = \frac{1}{(2\pi\hbar)^4} \sumsig \iint (1 - \ev*{\npps}) \ev{\nps} e^{i (\vp - \vp') (\vrw - \vrq) / \hbar} \ddsp \ddspp.
	}
	For the first term~\cite[p.~356]{Landau2},
	\al{
		\iint \ev{\nps} e^{i (\vp - \vp') (\vrw - \vrq) / \hbar} \ddsp \ddspp &= \sumsig \int \ev{\nps} e^{i \vp (\vrw - \vrq) / \hbar} \ddcp \int e^{-i \vp' (\vrw - \vrq) / \hbar} \ddspp \\
		&= \sumsig \int \ev{\nps} \del(\vrw - \vrq) \,e^{i \vp (\vrw - \vrq) / \hbar} \ddsp \\
		&= \del(\vrw - \vrq) \sumsig \int \ev{\nps} \ddsp
		= \evn \del(\vrw - \vrq).
	}
	This is the first term in the definition of the spatial correlation, which is
	\eqn{C}{
		C(\rqq, \rw) = \ev{\Del\nq \,\Del\nw} = \evn \del(\vrw - \vrq) + \evn \nu(r),
	}
	meaning we can associate $\nu(r)$ with the second term in Eq.~\refeq{thing5}~\cite[pp.~351, 356]{Landau2}.  So the correlation function is
	\eq{
		\nu(r) = -\frac{1}{(2\pi\hbar)^4 \evn} \sumsig \abs{ \int e^{i \vp \vdot \vr / \hbar} \ev{\nps} \ddsp }^2.
	}
	For a Fermi gas, $\ev{\nps} = \ev{n_{\vp}} = 1 / (e^{(\eps - \mu) / T} + 1)$, which does not depend on $\sigma$.  Then~\cite[p.~356]{Landau2}
	\aln{
		\nu(r) &= -\frac{g}{\evn (2\pi\hbar)^4} \abs{ \int \frac{e^{i \vp \vdot \vr / \hbar}}{e^{(\eps - \mu) / T} + 1} \ddsp }^2
		= -\frac{g}{\evn (2\pi\hbar)^2} \abs{ \intotp \intoi \frac{p \,e^{i p r \cos\tht / \hbar}}{e^{(\eps - \mu) / T} + 1} \ddp \ddtht }^2 \label{exp5} \\
		&= \frac{(2\pi)^2 g}{\evn (2\pi\hbar)^4} \paren{ \intoi \frac{p\, \Jo(p r / \hbar)}{e^{(\eps - \mu) / T} + 1} \ddp }^2
		\equiv \frac{g}{4 \pi^2 \hbar^4 \evn} I^2, \label{thing5}
	}
	where $\Jo(x)$ is a Bessel function of the first kind, and we have defined $I$.
	
	For $T \ll \EF = \po^2 / 2m$ where $\po$ is the Fermi momentum, the integral has contributions from only $p \in (0, \po)$.  Thus $\evn$ is approximately a step function in the $T = 0$ limit, with $\evn = 1$ for $p < \po$~\cite[p.~357]{Landau}.  This gives us
	\eq{
		I \approx \int_0^{\po} p\, \Jo\!\paren{ \frac{p r}{\hbar} } \ddp
		= \brac{ \frac{\hbar p}{r} \Jq\!\paren{ \frac{p r}{\hbar} } }_0^{\po}
		= \frac{\hbar \po}{r} \Jq\!\paren{ \frac{\po r}{\hbar} },
	}
	where we have used the identity $\dv*{[ x^m J_m(x) ]}{x} = x^m J_{m-1}(x)$~\cite{BesselJ}.  Substituting into Eq.~\refeq{thing5},
	\eq{
		\nu(r) = \frac{g \po^2}{4 \pi^2 \hbar^2 r^2 \evn} \Jq^2\!\paren{ \frac{\po r}{\hbar} },
	}
	and feeding this into Eq.~\refeq{C}, we find
	\eq{
		\ans{ C(r) = \evn \del(r) + \frac{g \po^2}{4 \pi^2 \hbar^2 r^2} \Jq^2\!\paren{ \frac{\po r}{\hbar} }. }
	}
	
	For $r \ll \hbar / \po$, the argument of $\Jo$ is small everywhere.  This follows because for a degenerate gas, only the range $p \in (0, \po)$ contributes to the integral.  Then we can use the asymptotic approximation $J_m(z) \approx z^m / [ 2^m \,\Gam(m + 1) ]$~\cite{BesselJ}.  In this limit,
%	\eq{
%		I \approx \frac{1}{2} \intoi \frac{p}{e^{(p^2 - \po^2) / 2m T} + 1} \ddp
%		= m T \ln(1 + e^{\po^2 / 2m T}),
%	}
%	where we have evaluated the integral using Mathematica.  So Eqs.~\refeq{thing5} and \refeq{C} give us
%	\eq{
%		\ans{ C(r) \approx \evn \del(r) + \frac{g m^2 T^2}{4 \pi^2 \hbar^4 \evn} \ln^2(1 + e^{\po^2 / 2m T}) }
%	}
	\eq{
		C(r) \approx \evn \del(r) + \frac{g \po^2}{4 \pi^2 \hbar^2 r^2} \paren{ \frac{1}{2} \frac{\po r}{\hbar} }^2
		= \ans{ \evn \del(r) + \frac{g \po^4}{16 \pi^2 \hbar^4}, }
	}
	which is independent of $r$.
	
	For $r \gg \hbar / \po$, the argument of $\Jq$ is large everywhere.  This means we can use the asymptotic approximation~\cite{Bessel}
	\eq{
		J_m(z) \approx \sqrt{\frac{2}{\pi z}} \cos(z - \frac{m \pi}{2} -\frac{\pi}{4}).
	}
	The integral then becomes
	\al{
		I &\approx \sqrt{\frac{2 \hbar}{\pi r}} \int_0^{\po} \sqrt{p} \cos(\frac{p r}{\hbar} - \frac{\pi}{4}) \ddp
		= \sqrt{\frac{2 \hbar^3}{\pi r^3}} \pdv{r} \int_0^{\po} \frac{\sin(p r / \hbar - \pi / 4)}{\sqrt{p}} \ddp \\
		&= \sqrt{\frac{2 \hbar^3}{\pi r^3}} \pdv{r}\!\curly{ \sqrt{\frac{\pi \hbar}{r}} \brac{ S\!\paren{ \sqrt{\frac{2 \po r}{\pi \hbar}} } - C\!\paren{ \sqrt{\frac{2 \po r}{\pi \hbar}} } } }
		\approx -\sqrt{\frac{2 \hbar^3}{\pi r^3}} \pdv{r}\!\curly{ \sqrt{\frac{\pi \hbar}{r}} \sqrt{\frac{\hbar}{2 \pi \po r}} \brac{ \sin(\frac{\po r}{\hbar}) + \cos(\frac{\po r}{\hbar}) } } \\
%		&\approx -\sqrt{\frac{\hbar^5}{\pi \po r^3}} \pdv{r}\!\paren{ \frac{\sin(\po r / \hbar) + \cos(\po r / \hbar)}{r} }
		&= -\sqrt{\frac{2 \hbar^5}{\pi \po r^3}} \pdv{r}\!\paren{ \frac{\sin(\po r / \hbar + \pi / 4)}{r} }
		= -\sqrt{\frac{4 \hbar^5}{\pi \po r^3}} \paren{ \frac{\po \cos(\po r / \hbar + \pi / 4)}{\hbar r} - \frac{\sin(\po r / \hbar + \pi / 4)}{r^2} } \\
		&\approx -\sqrt{\frac{4 \hbar^3 \po}{\pi r^5}} \cos(\frac{\po r}{\hbar} + \frac{\pi}{4})
		\approx -\sqrt{\frac{4 \hbar^3 \po}{\pi r^5}} \cos(\frac{\po r}{\hbar}),
	}
	where $S$ and $C$ are the Fresnel integrals, and we have used the asymptotic expansions
	\al{
		C(u) &\approx \frac{1}{2} + \frac{\sin(\pi u^2 / 2)}{\pi u}, &
		S(u) &\approx \frac{1}{2} - \frac{\cos(\pi u^2 / 2)}{\pi u},
	}
	which are valid for $u \gg 1$~\cite{Fresnel}.  In going to the final line, we have retained only the term in the lowest power of $1 / r$~\cite[p.~357]{Landau}.  So we have
	\eq{
		\ans{ C(r) \approx %\evn \del(r) + \frac{g}{4 \pi^2 \hbar^4 \evn} \frac{4 \hbar^3 \po}{\pi r^5} \cos[2](\frac{\po r}{\hbar})
		\evn \del(r) + \frac{g \po}{\pi^3 \hbar r^5} \cos[2](\frac{\po r}{\hbar}). }
	}
	
%	For small distances $r \ll \hbar / \sqrt{m T}$, let $r \sqrt{m T} / \hbar = u$.  Then $e^{\po^2 / mT - u^2} = e^{\po^2 / mT} (1 - u^2) + \order{u^4}$.  So for small distances,
%	\eq{
%		\ans{ C(r) \approx \evn \del(r) + \frac{g m^2 T^2}{4 \pi^2 \hbar^4} e^{\po^2 / m T}. }
%	}
%	
%	For large distances $r \gg \hbar / \sqrt{m T}$, $e^{\po^2 / mT - u^2} \approx e^{-u^2}$.  So in this limit,
%	\eq{
%		\ans{ C(r) \approx \evn \del(r) + \frac{g m^2 T^2}{4 \pi^2 \hbar^4} e^{-m T r^2 / \hbar^2}. }
%	}

	The general expression for the fluctuation of the occupation number for a Fermi gas is, in three dimensions~\cite[p.~356]{Landau2},
	\eq{
		\ev*{\abs{\Del\nk}^2} = \frac{g}{(2\pi\hbar)^3 V} \int \ev{n_{\vp}} (1 - \ev{n_{\vp + \hbar \vk}}) \dd[3]{p}.
	}
	The 2D analogue is then
	\eq{
		\ev*{\abs{\Del\nk}^2} = \frac{g}{(2\pi\hbar)^2 V} \int \ev{n_{\vp}} (1 - \ev{n_{\vp + \hbar \vk}}) \ddsp.
	}
	We will assume that we are interested in small wave numbers, $k \ll \po / \hbar$.  In the $T =  0$ limit, the integral above is nonzero only for points that are in a circle of radius $\po$, with center at $p = 0$, but not in the circle of radius $\po$ with center at $p = \hbar k$~\cite[p.~357]{Landau}.  The area of the parts of these circles that \emph{intersect} is~\cite{Intersection}
	\eq{
		A' = 2 \po^2 \cos[-1](\frac{\hbar k}{2 R}) - \frac{\hbar k}{2} \sqrt{4 R^2 - (\hbar k)^2}
		= 2 \po^2 \cos[-1](\frac{\hbar k}{2 R}) - \frac{\hbar k \po}{2} \sqrt{4 - \paren{ \frac{\hbar k}{R} }^2}
		\approx \pi \po^2 - \hbar k \po,
	}
	where we have approximated each term to zeroeth order in $\hbar k / \po$.  So the area we are interested in is $A = \pi \po^2 - A' \approx \hbar k \po$.  This gives us
	\eq{
		\ev*{\abs{\Del\nk}^2} = \frac{g}{(2\pi\hbar)^2 V} \hbar k \po
		= \frac{g}{(2\pi\hbar)^2 V} \hbar k 2\hbar \sqrt{\frac{\pi \evn}{g}}
		= \ans{ \frac{k}{2 \hbar V} \sqrt{\frac{g \evn}{\pi^3}}, }
	}
	where we have used $\po = 2\hbar \sqrt{\pi \evn / g}$ from the result of Homework 3, Prob.~{3.1}.
}


%
%	5.2
%

\prob{}{
	Repeat the above for the Bose gas slightly above the condensation temperature.
}

\sol{
	For a Bose gas, $\ev{\nps} = \ev{n_{\vp}} = 1 / (e^{(\eps - \mu) / T} - 1)$, which does not depend on $\sigma$.  Then the analogue of Eq.~\refeq{exp5}~\cite[p.~356]{Landau2}
	\eqn{thing5.2}{
		\nu(r) = \frac{(2\pi)^2 g}{\evn (2\pi\hbar)^4} \paren{ \intoi \frac{p\, \Jo(p r / \hbar)}{e^{(\eps - \mu) / T} - 1} \ddp }^2
		\equiv \frac{g}{4 \pi^2 \hbar^4 \evn} I^2,
	}
	where we have defined $I$.  Just above the condensation temperature $\To$, the integral is dominated by small $p$, so $p^2 / mT \sim \abs{\mu} / T \ll 1$~\cite[p.~358]{Landau2}.  In this limit,
	\eq{
		e^{(\eps - \mu) / T} = \exp(\frac{p^2}{2m T} - \frac{\mu}{T})
		\approx 1 + \frac{p^2}{2m T} - \frac{\mu}{T},
	}
	where we have used $e^x \approx 1 + x$ for small $x$.  Then
	\eq{
		I \approx T \intoi \frac{p\, \Jo(p r / \hbar)}{p^2 / 2m + \abs{\mu}} \ddp
		= \frac{2 T}{m} \Ko\!\paren{ \sqrt{\frac{2 \abs{\mu}}{m}} \frac{r}{\hbar} },
	}
	where we have evaluated the integral using Mathematica, and $\Ko$ is the modified Bessel function of the second kind.  From Eq.~\refeq{thing5.2},
	\eq{
		\nu(r) = \frac{g}{4 \pi^2 \hbar^4 \evn} \brac{ \frac{2 T}{m} \Ko\!\paren{ \sqrt{\frac{2 \abs{\mu}}{m}} \frac{r}{\hbar} } }^2
		= \frac{g T^2}{\pi^2 \hbar^4 m^2 \evn} \Ko^2\!\paren{ \sqrt{\frac{2 \abs{\mu}}{m}} \frac{r}{\hbar} },
	}
	so from Eq.~\refeq{C},
	\eq{
		\ans{ C(r) = \evn \del(r) + \frac{g T^2}{\pi^2 \hbar^4 m^2 \evn} \Ko^2\!\paren{ \sqrt{\frac{2 \abs{\mu}}{m}} \frac{r}{\hbar} }, }
	}
	which is valid for all distance regimes.

	For $r \ll \hbar \sqrt{m / 2 \abs{\mu}}$, we Taylor expand $\Ko(z)$ about $z = 0$.  Using Mathematica, $K(z) = \ln(2) - \gam - \ln z + \order{z^2}$, where $\gam$ is Euler's constant.  Feeding this into Eq.~\refeq{thing5.2}, we find
	\eq{
		\nu(r) \approx \frac{g}{4 \pi^2 \hbar^4 \evn} \brac{ \ln(2) - \gam - \ln(\sqrt{\frac{2 \abs{\mu}}{m}} \frac{r}{\hbar}) }^2
		= \frac{g}{4 \pi^2 \hbar^4 \evn} \brac{ \ln(\frac{2}{\hbar}) - \gam - \frac{1}{2} \ln(\frac{2 \abs{\mu}}{m}) - \ln r }^2,
	}
	so, from Eq.~\refeq{C},
	\eq{
		\ans{ C(r) = \evn \del(r) + \frac{g}{4 \pi^2 \hbar^4} \brac{ \ln(\frac{2}{\hbar}) - \gam - \frac{1}{2} \ln(\frac{2 \abs{\mu}}{m}) - \ln r }^2 }
	}
	for $r \ll \hbar \sqrt{m / 2 \abs{\mu}}$.
	
	For $r \gg \hbar \sqrt{m / 2 \abs{\mu}}$, we use the series expansion about $z \to \infty$ $K_\nu(z) \propto e^{-z} \sqrt{\pi / 2 z} + \order{1/z}$, also evaluated with Mathematica.  Equation~\refeq{thing5.2} becomes
	\eq{
		\nu(r) \approx \frac{g}{4 \pi^2 \hbar^4 \evn} \brac{ \sqrt{\frac{\pi}{2}} \paren{ \sqrt{\frac{m}{2 \abs{\mu}}} \frac{\hbar}{r} }^{1/2} \exp(\sqrt{\frac{2 \abs{\mu}}{m}} \frac{r}{\hbar}) }^2
		= \frac{g}{8 \pi \hbar^3 \evn r} \sqrt{\frac{m}{2 \abs{\mu}}} \exp(2 \sqrt{\frac{2 \abs{\mu}}{m}} \frac{r}{\hbar}),
	}
	so, from Eq.~\refeq{C},
	\eq{
		\ans{ C(r) = \evn \del(r) + \frac{g}{8 \pi \hbar^3 r} \sqrt{\frac{m}{2 \abs{\mu}}} \exp(\sqrt{\frac{8 \abs{\mu}}{m}} \frac{r}{\hbar}) }
	}
	for $r \gg \hbar \sqrt{m / 2 \abs{\mu}}$.
	
		The general expression for the fluctuation of the occupation number for a Bose gas is, in three dimensions~\cite[p.~356]{Landau2},
	\eq{
		\ev*{\abs{\Del\nk}^2} = \frac{g}{(2\pi\hbar)^3 V} \int \ev{n_{\vp}} (1 + \ev{n_{\vp + \hbar \vk}}) \dd[3]{p}.
	}
	The 2D analogue is then
	\eq{
		\ev*{\abs{\Del\nk}^2} = \frac{g}{(2\pi\hbar)^2 V} \int \ev{n_{\vp}} (1 + \ev{n_{\vp + \hbar \vk}}) \ddsp.
	}
}