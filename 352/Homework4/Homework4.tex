\documentclass[11pt]{article}
\usepackage{homework}

\classname{352}
\homeworknum{4}



\begin{document}

% Environments

\newcommand{\state}[2]{\begin{statement}{#1} #2 \end{statement}}
\newcommand{\prob}[2]{\begin{problem}{#1} #2 \end{problem}}
\newcommand{\subprob}[1]{\begin{subproblem} #1 \end{subproblem}}
\newcommand{\sol}[1]{\begin{solution} #1 \end{solution}}
\newcommand{\fig}[2]{\begin{figure} \centering #2  \label{#1} \end{figure}}

\newcommand{\makebib}{
	\vfill
	\color{black}
	\bibliography{references}{}
	\bibliographystyle{lucas_unsrt}
}
	

% Implication

\newcommand{\qwhere}{\quad \text{where} \quad}
\newcommand{\qimplies}{\quad \implies \quad}
\newcommand{\impliesq}{\implies \quad}



% Brackets

\newcommand{\paren}[1]{\left( #1 \right)}
\newcommand{\brac}[1]{\left[ #1 \right]}


% Greek

\newcommand{\alp}{\alpha}
\newcommand{\bet}{\beta}
\newcommand{\gam}{\gamma}
\newcommand{\del}{\delta}
\newcommand{\eps}{\epsilon}
\newcommand{\zet}{\zeta}
\newcommand{\tht}{\theta}
\newcommand{\kap}{\kappa}
\newcommand{\lam}{\lambda}
\newcommand{\sig}{\sigma}
\newcommand{\ups}{\upsilon}
\newcommand{\omg}{\omega}

\newcommand{\Gam}{\Gamma}
\newcommand{\Del}{\Delta}
\newcommand{\Tht}{\Theta}
\newcommand{\Lam}{\Lambda}
\newcommand{\Sig}{\Sigma}
\newcommand{\Omg}{\Omega}
% Problem 1

\newcommand{\Psii}{\Psi^i}
\newcommand{\Psiix}{\Psii(x)}

\newcommand{\Pii}{\Pi^i}

\newcommand{\Phii}{\Phi^i}
\newcommand{\Phiix}{\Phii(x)}
\newcommand{\PhiN}{\Phi^N}
\newcommand{\PhiNx}{\PhiN(x)}
\newcommand{\Phiq}{\Phi^1}
\newcommand{\Phiw}{\Phi^2}

\newcommand{\ddcx}{\dd[3]{x}}

\newcommand{\delij}{\del^{i j}}
\newcommand{\delkl}{\del^{k l}}
\newcommand{\delil}{\del^{i l}}
\newcommand{\deljk}{\del^{j k}}
\newcommand{\delik}{\del^{i k}}
\newcommand{\deljl}{\del^{j l}}

\newcommand{\DF}{D_F}

\newcommand{\sigx}{\sig(x)}

\newcommand{\pii}{\pi^i}
\newcommand{\pij}{\pi^j}
\newcommand{\pik}{\pi^k}
\newcommand{\pil}{\pi^l}
\newcommand{\piix}{\pi(x)}

\newcommand{\pq}{p_1}
\newcommand{\pw}{p_2}
\newcommand{\pe}{p_3}
\newcommand{\pr}{p_4}

\newcommand{\vp}{\vb{p}}
\newcommand{\vpsi}{\vp_i}

\newcommand{\mpi}{m_\pi}

\state{(Jackson 9.8)}{\ 
	%\emph{Hint:} The electromagnetic angular momentum density comes from more than the transverse (radiation zone) components of the fields.
}

%
%	Jackson 9.8(a)
%

\prob{}{
	Show that a classical oscillating electric dipole $\vp$ with fields given by
	\aln{ \label{fields1}
		\vH &= \frac{c k^2}{4\pi} (\nh \cross \vp) \frac{e^{i k r}}{r} \paren{ 1 - \frac{1}{i k r} }, &
		\vE &= \frac{1}{4\pi \epso} \curly{ k^2 (\nh \cross \vp) \cross \nh \frac{e^{i k r}}{r} + [ 3 \nh (\nh \vdot \vp) - \vp ] \paren{ \frac{1}{r^3} - \frac{i k}{r^2} } e^{i k r} },
	}
	radiates electromagnetic angular momentum to infinity at the rate
	\eq{
		\dv{\vL}{t} = \frac{k^3}{12 \pi \epso} \Im[ \vp^* \cross \vp ].
	}
	\vfix
}

\sol{
	According to Jackson~(9.20), the time-averaged angular momentum density is
	\eq{
		\vl = \frac{\Re[ \vx \cross (\vE \cross \vHs)}{2 c^2}.
	}
	One of the vector identities on the inside cover of Jackson is $\vaa \cross (\vbb \cross \vcc) = (\vaa \vdot \vcc) \vbb - (\vaa \vdot \vbb) \vcc$, so
	\eqn{l1}{
		\vl = \frac{(\vx \vdot \vHs) \vE - (\vx \vdot \vE) \vHs}{2 c^2}.
	}
	From Eq.~\refeq{fields1}, note that
	\eq{
		\vx \vdot \vHs \propto \vx \vdot (\nh \cross \vps)
		= \vps \vdot (\vx \cross \nh)
		= \vO,
	}
	where we have used the identity $\vaa \vdot (\vbb \cross \vcc) = \vcc \vdot (\vaa \cross \vbb)$ and the fact that $\nh$ points in the $\vx$ direction.  For $\vx \vdot \vE$, note that
	\al{
		\vx \vdot [ (\nh \cross \vp) \cross \nh ] &= -\vx \vdot [ \nh \cross (\nh \cross \vp) ]
		= -\vx \vdot [ (\nh \vdot \vp) \nh - (\nh \vdot \nh) \vp ]
		= -(\nh \vdot \vp) (\vx \vdot \nh) + \vx \vdot \vp \\
		&= -r (\nh \vdot \vp) + \vx \vdot \vp
		= \vx \vdot \vp - \vx \vdot \vp
		= 0, \\[1.5ex]
		\vx \vdot [ 3 \nh (\nh \vdot \vp) - \vp ] &= 3 (\vx \vdot \nh) (\nh \vdot \vp) - \vx \vdot \vp
		= 3r (\nh \vdot \vp) - \vx \vdot \vp
		= 3(\vx \vdot \vp) - \vx \vdot \vp
		= 2(\vx \vdot \vp),
	}
	since $\abs{\vx} = r$ and $\vx = r \,\nh$.  Then
	\eq{
		\vx \vdot \vE = \frac{1}{2\pi \epso} (\vx \vdot \vp) \paren{ \frac{1}{r^3} - \frac{i k}{r^2} } e^{i k r}
		= \frac{1}{2\pi \epso} (\nh \vdot \vp) \paren{ \frac{1}{r^2} - \frac{i k}{r} } e^{i k r}.
	}
	
	With these substitutions, Eq.~\refeq{l1} becomes
	\al{
		\vl &= -\frac{(\vx \vdot \vE) \vHs}{c^2}
		= -\frac{1}{4\pi \epso c^2} (\nh \vdot \vp) \paren{ \frac{1}{r^2} - \frac{i k}{r} } e^{i k r} \frac{c k^2}{4\pi} (\nh \cross \vps) \frac{e^{-i k r}}{r} \paren{ 1 + \frac{1}{i k r} } \\
		&= -\frac{k^2}{16\pi^2 \epso c r} (\nh \vdot \vp) (\nh \cross \vps) \paren{ \frac{1}{r^2} - \frac{i k}{r} } \paren{ 1 - \frac{i}{k r} }
		= -\frac{k^2}{16\pi^2 \epso c} (\nh \vdot \vp) (\nh \cross \vps) \paren{ \frac{1}{r^2} - \frac{i}{k r^3} - \frac{i k}{r} - \frac{1}{r^2} } \\
		&= -\frac{i k^2}{16\pi^2 \epso c r} (\nh \vdot \vp) (\nh \cross \vps) \paren{ \frac{1}{k r^3} + \frac{k}{r^2} }
		= \frac{i k^3}{16\pi^2 \epso c r^2} (\nh \vdot \vp) (\nh \cross \vps) \paren{ \frac{1}{k^2 r^2} + 1 }.
	}
	
	Let $\vL$ be the angular momentum radiated to a distance $R$.  Then
	\eq{
		\vL = \int_R \vl(r) \ddcx
		= \intopi \intotp \intoR \vl(r) \,r^2 \sin\tht \ddr \ddphi \dd\tht,
	}
	and the time derivative is
	\aln{
		\dv{\vL}{t} &= \dv{t}(\intopi \intotp \intoR \vl(r) \,r^2 \sin\tht \ddr \ddphi \dd\tht)
		= \dv{r}{t} \dv{r}(\intopi \intotp \intoR \vl(r) \,r^2 \sin\tht \ddr \ddphi \dd\tht) \notag \\
		&= c \intopi \intotp \vl(r) \,r^2 \sin\tht \ddphi \dd\tht
		= \frac{i k^3}{16\pi^2 \epso} \paren{ \frac{1}{k^2 r^2} + 1 } \intopi \intotp (\nh \vdot \vp) (\nh \cross \vps) \sin\tht \ddphi \dd\tht. \label{dLdt}
	}
	Note that
	\eq{
		[ (\nh \vdot \vp) (\nh \cross \vps) ]_i = \sumje n_j p_j (\nh \cross \vps)_i
		= \sumje \sumke \sumle \epsikl n_j p_j n_k p_l^*,
	}
	so
	\eq{
		\dv{L_i}{t} \propto \sumje \sumke \sumle \epsikl p_j p_l^* \int n_j p_k \ddOmg
		= \sumje \sumke \sumle \epsikl p_j p_l^* \frac{4\pi}{3} \del_{jk}
		= \frac{4\pi}{3} \epsikl p_k p_l^*
		= \frac{4\pi}{3} (\vp \cross \vps)_i,
	}
	where we have used Jackson~(9.47), $\int n_\bet n_\gam \ddOmg = 4\pi \del_{\bet \gam} / 3$.  Making this substitution into Eq.~\refeq{dLdt},
	\eq{
		\dv{\vL}{t} = \frac{i k^3}{6\pi \epso} \paren{ \frac{1}{k^2 r^2} + 1 } (\vp \cross \vps).
	}
	Taking the limit as $r \to \infty$, we find
	\eqn{ans1a}{
		\dv{\vL}{t} = \Re\!\brac{ \frac{i k^3}{12\pi \epso} (\vp \cross \vps) }
		= \Re\!\brac{ -\frac{i k^3}{12\pi \epso} (\vps \cross \vp) }
		= \ans{ \frac{k^3}{12\pi \epso} \Im[ \vps \cross \vp ], }
	}
	as desired. \qed
}

%
%	Jackson 9.8(b)
%

\prob{}{
	What is the ratio of angular momentum radiated to energy radiated?  Interpret.
}

\sol{
	According to Jackson~(9.24), the total power radiated by an oscillating electric dipole $\vp$ is
	\eq{
		P = \dv{E}{t}
		= \frac{c^2 \Zo k^4}{12 \pi} \abs{\vp}^2.
	}
	Then the ratio of angular momentum radiated to energy radiated is
	\eq{
		\frac{\dv*{\vL}{t}}{\dv*{E}{t}} = \frac{k^3}{12\pi \epso} \Im[ \vps \cross \vp ] \frac{12 \pi}{c^2 \Zo k^4 \abs{\vp}^2}
		= \frac{1}{\epso} \Im[ \vps \cross \vp ] \frac{1}{c^2 \Zo k \abs{\vp}^2}
		= \ans{ \frac{\Im[ \vps \cross \vp ]}{\omg \abs{\vp}^2}, }
	}
	where we have used $\Zo = \sqrt{\muo / \epso} = 1 / \sqrt{\epso^2 c^2} = 1 / \epso c$, $c^2 = 1 / (\epso \muo)$, and $\omg = k c$.
	
	In the limit of high frequency, $(\dv*{\vL}{t}) / (\dv*{E}{t}) \to 0$.  In this scenario, the energy radiated dominates over the angular momentum radiated.  Likewise, in the limit of low frequency, $(\dv*{\vL}{t}) / (\dv*{E}{t}) \to \infty$, meaning that angular momentum radiation dominates.  This is sensible because rotational kinetic energy $E \propto \omg^2$, while angular momentum $L \propto \omg$.
}

%
%	Jackson 9.8(c)
%

\prob{}{
	For a charge $e$ rotating in the $xy$ plane at radius $a$ and angular speed $\omg$, show that there is only a $z$ component of radiated angular momentum with magnitude $\dv*{\Lz}{t} = e^2 k^3 a^2 / 6 \pi \epso$.  What about a charge oscillating along the $z$ axis?
}

\sol{
	We know from Homework~5 that the position of a point charge rotating counterclockwise in the $xy$ plane is
	\eq{
		\vx(t) = a \cos(\omg t) \,\vx + a \sin(\omg t) \,\yh.
	}
	\clearpage
	Then the charge distribution is
	\eq{
		\rho(\vx, t) = e \del[ x - a \cos(\omg t) ] \,\del[ y - a \sin(\omg t) ] \,\del(z).
	}
	
	According to Jackson~(4.8), the dipole moment is defined
	\eq{
		\vp = \int \vx' \,\rho(\vx') \ddcxp.
	}
	The components of $\vp$ for the point charge are then
	\al{
		\px &= e \iiint x \,\del[ x - a \cos(\omg t) ] \,\del[ y - a \sin(\omg t) ] \,\del(z) \ddx \ddy \ddz
		= e a \cos(\omg t), \\
		\py &= e \iiint y \,\del[ x - a \cos(\omg t) ] \,\del[ y - a \sin(\omg t) ] \,\del(z) \ddx \ddy \ddz
		= e a \sin(\omg t), \\
		\pz &= e \iiint z \,\del[ x - a \cos(\omg t) ] \,\del[ y - a \sin(\omg t) ] \,\del(z) \ddx \ddy \ddz
		= 0,
	}
	so we can write $\vp = e a \,e^{-i \omg t} (\xh + i\,\yh).$  Substituting into Eq.~\refeq{ans1a},
	\al{
		\dv{\vL}{t} &= \Re\!\brac{ \frac{i k^3}{12\pi \epso} e^2 a^2 e^{-i \omg t} e^{i \omg t} [ (\xh + i\,\yh) \cross (\xh - i\,\yh) ] }
		= \Re\!\brac{ \frac{i e^2 k^3 a^2}{12\pi \epso} (-2i \,\xh \cross \yh) }
		= \Re\!\brac{ \frac{e^2 k^3 a^2}{6\pi \epso} \,\zh } \\
		&= \ans{ \frac{e^2 k^3 a^2}{6\pi \epso} \cos(\omg t) \,\zh, }
	}
	as desired. \qed
	
	A charge oscillating along the $z$ axis with amplitude $a$ has the charge density
	\eq{
		\rho(\vx, t) = e a \,\del(x) \,\del(y) \,\del[ z - \cos(\omg t) ],
	}
	which gives the dipole moment
	\al{
		\px &= e a \iiint x \,\del(x) \,\del(y) \,\del[ z - \cos(\omg t) ] \ddx \ddy \ddz
		= 0, \\
		\py &= e a \iiint y \,\del(x) \,\del(y) \,\del[ z - \cos(\omg t) ] \ddx \ddy \ddz
		= 0, \\
		\pz &= e a \iiint z \,\del(x) \,\del(y) \,\del[ z - \cos(\omg t) ] \ddx \ddy \ddz
		= e a \cos(\omg t).
	}
	In complex notation, $\vp = e a \,e^{-i\omg t} \,\zh$.  Substituting into Eq.~\refeq{ans1a}, we find
	\eq{
		\dv{\vL}{t} = \Re\!\brac{ \frac{i k^3}{12\pi \epso} e^2 a^2 e^{-i \omg t} e^{i \omg t} (\zh \cross \zh) }
		= \ans{ \vO. }
	}
	So we see that a charge undergoing linear motion does not lead to a radiated angular momentum, which is sensible.
}

%
%	Jackson 9.8(d)
%

\prob{}{
	What are the results corresponding to Probs.~{1(a)} and {1(b)} for magnetic dipole radiation?
}

\sol{
	The radiation fields for a magnetic dipole are given by Jackson~(19.35--36),
	\al{
		\vH &= \frac{1}{4\pi} \curly{ k^2 (\nh \cross \vm) \cross \nh \frac{e^{i k r}}{r} + [ 3 \nh (\nh \vdot \vm) - \vm ] \paren{ \frac{1}{r^3} - \frac{i k}{r^2} } e^{i k r} }, &
		\vE &= -\frac{\Zo}{4\pi} k^2 (\nh \cross \vm) \frac{e^{i k r}}{r} \paren{ 1 - \frac{1}{i k r} }.
	}
	\clearpage
	Comparing with Eq.~\refeq{fields1}, we see that $\vH \to -\vE / \Zo$, $\vE \to \Zo \vH$, and $\vp \to \vm / c$ as stated in the book~\cite[p.~413]{Jackson}.  Making these substitutions, the results of Probs.~{1.1(a)} and {(b)} become
	\al{
		\ans{ \dv{\vL}{t}\ }&\ans{= \frac{\muo k^3}{12\pi} \Im[ \vms \cross \vm ], } &
		\ans{ \frac{\dv*{\vL}{t}}{\dv*{E}{t}}\ }&\ans{= \frac{\Im[ \vms \cross \vm ]}{\omg \abs{\vm}^2} }
	}
	where we have used $\mu = 1 / \epso c^2$.
}



\state{Pauli paramagnetism}{
	Cold atomic gases could be realized by atomic isotopes which are fermions ($^6$Li, $^{40}$K, etc.).  Such isotopes may have a large atomic spin.  Assuming that the Fermi gas is degenerate and its constituents have a spin $s> 1 / 2$, compute the Pauli magnetic susceptibility.
}

\sol{
	According to p.~2 of Lecture 12, the magnetic susceptibility is defined
	\eq{
		\chi = \frac{1}{V} \pdv{N}{\mu},
	}
	where $N = \pdv*{\Omg}{\mu}$, and
	\eq{
		\Omg(\mu, B) = \frac{1}{2} \Omgo(\mu + B) + \frac{1}{2} \Omgo(\mu - B)
		\approx \Omgo(\mu) + \frac{B^2}{2} \pdv[2]{\Omgo}{\mu},
	}
	where $B$ is the strength of the magnetic field and $\Omgo$ is the thermodynamic potential with no field present.  \hl{but I think this doesn't work because it is only for spin 1/2}
	
	For a Fermi gas, the thermodynamic potential is~\cite[p.~145]{Landau}
	\eq{
		\Omgo = -T \sumk \ln(1 + e^{(\mu - \epsk) / T}).
	}
	Note that
	\al{
		\pdv{\Omgo}{\mu} &= 
	}
	Then the thermodynamic potential in the magnetic field is
	\eq{
		\Omg = 
	}
}




\clearpage
\state{Landau diamagnetism}{\ }

%
%	3.1
%

\prob{}{
	Compute the Landau diamagnetic susceptibility for ultra-relativistic Fermi gas.
}

\sol{
	
}

%
%	3.2
%

\prob{(*)}{
	Compute the Landau diamagnetic susceptibility for a Fermi gas confined to a box whose linear size in the $z$ direction is $\Lz \ll \Lx, \Ly$.  The magnetic field is directed along the $z$ direction.  Consider two cases when the energy spacing $(2\pi \hbar / \Lz)^2 / 2m$ is much larger/smaller than the cyclotron energy $\muB B$.
}




\clearpage
\state{Fluctuations of thermodynamics}{\ }

%
%	4.1
%

\prob{}{
	Find the energy fluctuation $\evDEs = \ev{(E - \evE)^2}$ and the number fluctuation $\evDNs = \ev{(N - \evN)^2}$ for photons in the black body radiation.
}

%
%	4.2
%

\prob{}{
	Show that the number of particles in a sub-volume of a gas fluctuates according the formula $\evDNs = T \pdv*{\!\evN}{\mu}$.  Furthermore, apply this formula to the Boltzmann, Fermi, and Bose ideal gases.
}

\sol{
	Let $p(x)$ denote the probability of a fluctuation in $x$.  Then $p(x) \propto e^{S(x)}$, where $S(x)$ is the entropy of a closed system representing a sub-volume of a gas~\cite[pp.~343, 348]{Landau}.  It follows that $p(x) \propto e^{\Del S(x)}$, where $\Del S(x)$ is the change in the entropy due to the fluctuation~\cite[p.~348]{Landau}.  This change is equal to the difference between $S(x)$ and its equilibrium value, which is given by
	\eq{
		\Del S(x) = -\frac{\Del E - T \,\Del S + P \,\Del V}{T},
	}
	where $T$ and $P$ are the equilibrium values~\cite[pp.~60, 349]{Landau}.  Assuming small fluctuations and thus small $\Del E$, we can expand $\Del E$ as
	\al{
		\Del E &= \pdv{E}{S} \Del S + \pdv{E}{V} \Del V + \frac{1}{2} \brac{ \pdv[2]{E}{S} (\Del S)^2 + 2 \pdv{E}{S}{V} \Del S \,\Del V + \pdv{E}{V} (\Del V)^2 } \\
		&= T \,\Del S - P \,\Del V + \frac{1}{2} \brac{ \paren{ \Del\pdv{E}{S} }_V \,\Del S + \paren{ \Del\pdv{E}{V} }_S \,\Del V }
		= T \,\Del S - P \,\Del V + \frac{\Del S \,\Del T - \Del P \,\Del V}{2},
	}
	where we have used $\pdv*{E}{S} = T$ and $\pdv*{E}{V} = -P$~\cite[pp.~60, 349]{Landau}.  Then the fluctuation probability has the proportionality
	\eq{
		p \propto e^{\Del S(x)}
		= \exp(\frac{\Del P \,\Del V - \Del S \,\Del T}{2 T}).
	}
	Expanding $\Del S$ and $\Del P$ in terms of $V$ and $T$, we find
	\al{
		\Del P &= \paren{ \pdv{P}{T} }_V \,\Del T + \paren{ \pdv{P}{V} }_t \,\Del V, &
		\Del S &= \paren{ \pdv{S}{T} }_V \,\Del T + \paren{ \pdv{S}{V} }_T \,\Del V
		= \frac{\Cv}{T} \,\Del T + \paren{ \pdv{P}{T} }_V \,\Del V,
	}
	where we have used $(\pdv*{S}{V})_T = (\pdv*{P}{T})_V$ and $\Cv = T (\pdv*{S}{T})_V$~\cite[pp.~45, 50, 349]{Landau}.  Making these substitutions,
	\aln{
		p &\propto \exp\!\curly{ \frac{1}{2 T} \brac{ \paren{ \pdv{P}{T} }_V \,\Del T \,\Del V + \paren{ \pdv{P}{V} }_t (\Del V)^2 - \pdv{\Cv}{T} (\Del T)^2 - \paren{ \pdv{P}{T} }_V \,\Del V \,\Del T } } \notag \\
		&= \exp[ \paren{ \frac{1}{2T} \pdv{P}{V} }_t (\Del V)^2 - \frac{\Cv}{2 T^2} (\Del T) ]
		= \exp[ \paren{ \frac{1}{2T} \pdv{P}{V} }_t (\Del V)^2 ] \exp[ -\frac{\Cv}{2 T^2} (\Del T) ]. \label{thing}
	}
	Thus, the expression is separable and fluctuations in $V$ and in $T$ can be regarded as independent~\cite[p.~349]{Landau}.
	
	We will focus on fluctuations in volume, and assume their probability to be Gaussian distributed.  The Gaussian distribution is given by~\cite[p.~345]{Landau}
	\eq{
		p(x) \ddx = \frac{1}{\sqrt{2\pi \ev*{x^2}}} \exp(-\frac{x^2}{2 \ev*{x^2}}) \ddx.
	}
	Comparing Eq.~\refeq{thing}, we find that~\cite[p.~350]{Landau}
	\eq{
		\ev*{(\Del V)^2} = -T \paren{ \pdv{V}{P} }_T.
	}
	Dividing both sides by $N^2$~\cite[p.~351]{Landau},
	\eq{
		\ev{[ \Del(V / N) ]^2} = -\frac{T}{N^2} \paren{ \pdv{V}{P} }_T.
	}
	Now we fix $V$ and consider fluctuations in $N$.  Note that
	\eq{
		\Del(V / N) = V \,\Del(1 / N)
		= -\frac{V}{N^2} \,\Del N,
	}
	so we have
	\eq{
		\evDNs = -\frac{T N^2}{V^2} \paren{ \pdv{V}{P} }_T.
	}
	Since $N = V \,f(P, T)$, we can write
	\eq{
		 -\frac{N^2}{V^2} \paren{ \pdv{V}{P} }_T = N \brac{ \pdv{P}(\frac{N}{V}) }_{T, N}
		 = N \brac{ \pdv{P}(\frac{N}{V}) }_{T, v}
		 = \frac{N}{V} \paren{ \pdv{N}{P} }_{T, v}
		 = \paren{ \pdv{P}{\mu} }_{T, V} \paren{\pdv{N}{P} }_{T, V}
		 = \paren{ \pdv{N}{\mu} }_{T, V},
	}
	where we have used $N / V = (\pdv*{P}{\mu})_T$~\cite[pp.~351--352]{Landau}.  Since we associated all quantities with those at equilibrium, we have shown that
	\eq{
		\ans{ \evDNs = T \pdv{\!\evN}{\mu} }
	}
	as desired. \qed
	
	\hl{Boltzmann} $\evDNs = N$
	
	For the Fermi and Bose gases, the number of particles is given by
	\eq{
		N = \frac{g V}{\pi^2 \hbar^2} \sqrt{\frac{m^3 T^3}{2}} \intoi \frac{\sqrt{z}}{e^{z - \mu / T} \pm 1} \ddz
		\begin{cases}
			\text{Fermi}, \\
			\text{Bose},
		\end{cases}
	}
	where $z = \eps / T$~\cite[pp.~149, 354]{Landau}.  Evaluating the integrals using
	\eq{
		\intoi \frac{k^s}{e^{k - \mu} \pm 1} \dd{k} = -\Gam(s + 1) \,\Li_{1 + s}(\mp e^{\mu}),
	}
	where $\Li$ is the polylogarithm~\cite{Polylog}, we have
	\eq{
		N = \mp \frac{g V}{\pi^2 \hbar^2} \sqrt{\frac{m^3 T^3}{2}} \Gam(3/2) \,\Li_{3/2}(\mp e^{\mu / T})
		= \mp \frac{g V}{\pi^2 \hbar^2} \paren{ \frac{m T}{2} }^{3/2} \Li_{3/2}(\mp e^{\mu / T}).
	}
	Using the formula $\dv*{\Li_n(x)}{x} = \Li_{n - 1}(x) / x$~\cite{Polylog}, we find
	\eq{
		\pdv{\mu}[ \Li_{3/2}(\mp e^{\mu / T}) ] = \mp \pdv{\mu}(\mp e^{\mu / T}) \frac{\Li_{3/2}(\mp e^{\mu / T})}{e^{\mu / T}}
		= \frac{\Li_{3/2}(\mp e^{\mu / T})}{T}.
	}
	So the fluctuations are
	\eq{
		\ans{ \evDNs = \mp \frac{g V}{\pi^2 \hbar^2} \paren{ \frac{m T}{2} }^{3/2} \frac{\Li_{3/2}(\mp e^{\mu / T})}{T}
		\begin{cases}
			\text{Fermi}, \\
			\text{Bose}.
		\end{cases} }
	}
}




\clearpage
\state{Pair correlation function}{\ }

%
%	5.1
%

\prob{}{
	Compute the  pair correlation of density $C(r)= \ev{\ev{n(r) n(0)}}$ and the fluctuation of the occupation number $\ev*{\absnk^2}$ of the degenerate Fermi gas ($T \ll \EF$) in dimensions $d = 1, 2, 3$. Discuss various distance regimes.
}

%
%	5.2
%

\prob{}{
	Repeat the above for the Bose gas above the condensation temperature.
}


%\makebib

\end{document}