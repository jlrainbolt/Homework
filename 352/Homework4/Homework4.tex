\documentclass[11pt]{article}
\usepackage{homework}

\classname{352}
\homeworknum{4}



\begin{document}

% Environments

\newcommand{\state}[2]{\begin{statement}{#1} #2 \end{statement}}
\newcommand{\prob}[2]{\begin{problem}{#1} #2 \end{problem}}
\newcommand{\subprob}[1]{\begin{subproblem} #1 \end{subproblem}}
\newcommand{\sol}[1]{\begin{solution} #1 \end{solution}}
\newcommand{\fig}[2]{\begin{figure} \centering #2  \label{#1} \end{figure}}

\newcommand{\makebib}{
	\vfill
	\color{black}
	\bibliography{references}{}
	\bibliographystyle{lucas_unsrt}
}
	

% Implication

\newcommand{\qwhere}{\quad \text{where} \quad}
\newcommand{\qimplies}{\quad \implies \quad}
\newcommand{\impliesq}{\implies \quad}



% Brackets

\newcommand{\paren}[1]{\left( #1 \right)}
\newcommand{\brac}[1]{\left[ #1 \right]}


% Greek

\newcommand{\alp}{\alpha}
\newcommand{\bet}{\beta}
\newcommand{\gam}{\gamma}
\newcommand{\del}{\delta}
\newcommand{\eps}{\epsilon}
\newcommand{\zet}{\zeta}
\newcommand{\tht}{\theta}
\newcommand{\kap}{\kappa}
\newcommand{\lam}{\lambda}
\newcommand{\sig}{\sigma}
\newcommand{\ups}{\upsilon}
\newcommand{\omg}{\omega}

\newcommand{\Gam}{\Gamma}
\newcommand{\Del}{\Delta}
\newcommand{\Tht}{\Theta}
\newcommand{\Lam}{\Lambda}
\newcommand{\Sig}{\Sigma}
\newcommand{\Omg}{\Omega}
% Problem 1

\newcommand{\Psii}{\Psi^i}
\newcommand{\Psiix}{\Psii(x)}

\newcommand{\Pii}{\Pi^i}

\newcommand{\Phii}{\Phi^i}
\newcommand{\Phiix}{\Phii(x)}
\newcommand{\PhiN}{\Phi^N}
\newcommand{\PhiNx}{\PhiN(x)}
\newcommand{\Phiq}{\Phi^1}
\newcommand{\Phiw}{\Phi^2}

\newcommand{\ddcx}{\dd[3]{x}}

\newcommand{\delij}{\del^{i j}}
\newcommand{\delkl}{\del^{k l}}
\newcommand{\delil}{\del^{i l}}
\newcommand{\deljk}{\del^{j k}}
\newcommand{\delik}{\del^{i k}}
\newcommand{\deljl}{\del^{j l}}

\newcommand{\DF}{D_F}

\newcommand{\sigx}{\sig(x)}

\newcommand{\pii}{\pi^i}
\newcommand{\pij}{\pi^j}
\newcommand{\pik}{\pi^k}
\newcommand{\pil}{\pi^l}
\newcommand{\piix}{\pi(x)}

\newcommand{\pq}{p_1}
\newcommand{\pw}{p_2}
\newcommand{\pe}{p_3}
\newcommand{\pr}{p_4}

\newcommand{\vp}{\vb{p}}
\newcommand{\vpsi}{\vp_i}

\newcommand{\mpi}{m_\pi}

\state{(Jackson 9.8)}{\ 
	%\emph{Hint:} The electromagnetic angular momentum density comes from more than the transverse (radiation zone) components of the fields.
}

%
%	Jackson 9.8(a)
%

\prob{}{
	Show that a classical oscillating electric dipole $\vp$ with fields given by
	\aln{ \label{fields1}
		\vH &= \frac{c k^2}{4\pi} (\nh \cross \vp) \frac{e^{i k r}}{r} \paren{ 1 - \frac{1}{i k r} }, &
		\vE &= \frac{1}{4\pi \epso} \curly{ k^2 (\nh \cross \vp) \cross \nh \frac{e^{i k r}}{r} + [ 3 \nh (\nh \vdot \vp) - \vp ] \paren{ \frac{1}{r^3} - \frac{i k}{r^2} } e^{i k r} },
	}
	radiates electromagnetic angular momentum to infinity at the rate
	\eq{
		\dv{\vL}{t} = \frac{k^3}{12 \pi \epso} \Im[ \vp^* \cross \vp ].
	}
	\vfix
}

\sol{
	According to Jackson~(9.20), the time-averaged angular momentum density is
	\eq{
		\vl = \frac{\Re[ \vx \cross (\vE \cross \vHs)}{2 c^2}.
	}
	One of the vector identities on the inside cover of Jackson is $\vaa \cross (\vbb \cross \vcc) = (\vaa \vdot \vcc) \vbb - (\vaa \vdot \vbb) \vcc$, so
	\eqn{l1}{
		\vl = \frac{(\vx \vdot \vHs) \vE - (\vx \vdot \vE) \vHs}{2 c^2}.
	}
	From Eq.~\refeq{fields1}, note that
	\eq{
		\vx \vdot \vHs \propto \vx \vdot (\nh \cross \vps)
		= \vps \vdot (\vx \cross \nh)
		= \vO,
	}
	where we have used the identity $\vaa \vdot (\vbb \cross \vcc) = \vcc \vdot (\vaa \cross \vbb)$ and the fact that $\nh$ points in the $\vx$ direction.  For $\vx \vdot \vE$, note that
	\al{
		\vx \vdot [ (\nh \cross \vp) \cross \nh ] &= -\vx \vdot [ \nh \cross (\nh \cross \vp) ]
		= -\vx \vdot [ (\nh \vdot \vp) \nh - (\nh \vdot \nh) \vp ]
		= -(\nh \vdot \vp) (\vx \vdot \nh) + \vx \vdot \vp \\
		&= -r (\nh \vdot \vp) + \vx \vdot \vp
		= \vx \vdot \vp - \vx \vdot \vp
		= 0, \\[1.5ex]
		\vx \vdot [ 3 \nh (\nh \vdot \vp) - \vp ] &= 3 (\vx \vdot \nh) (\nh \vdot \vp) - \vx \vdot \vp
		= 3r (\nh \vdot \vp) - \vx \vdot \vp
		= 3(\vx \vdot \vp) - \vx \vdot \vp
		= 2(\vx \vdot \vp),
	}
	since $\abs{\vx} = r$ and $\vx = r \,\nh$.  Then
	\eq{
		\vx \vdot \vE = \frac{1}{2\pi \epso} (\vx \vdot \vp) \paren{ \frac{1}{r^3} - \frac{i k}{r^2} } e^{i k r}
		= \frac{1}{2\pi \epso} (\nh \vdot \vp) \paren{ \frac{1}{r^2} - \frac{i k}{r} } e^{i k r}.
	}
	
	With these substitutions, Eq.~\refeq{l1} becomes
	\al{
		\vl &= -\frac{(\vx \vdot \vE) \vHs}{c^2}
		= -\frac{1}{4\pi \epso c^2} (\nh \vdot \vp) \paren{ \frac{1}{r^2} - \frac{i k}{r} } e^{i k r} \frac{c k^2}{4\pi} (\nh \cross \vps) \frac{e^{-i k r}}{r} \paren{ 1 + \frac{1}{i k r} } \\
		&= -\frac{k^2}{16\pi^2 \epso c r} (\nh \vdot \vp) (\nh \cross \vps) \paren{ \frac{1}{r^2} - \frac{i k}{r} } \paren{ 1 - \frac{i}{k r} }
		= -\frac{k^2}{16\pi^2 \epso c} (\nh \vdot \vp) (\nh \cross \vps) \paren{ \frac{1}{r^2} - \frac{i}{k r^3} - \frac{i k}{r} - \frac{1}{r^2} } \\
		&= -\frac{i k^2}{16\pi^2 \epso c r} (\nh \vdot \vp) (\nh \cross \vps) \paren{ \frac{1}{k r^3} + \frac{k}{r^2} }
		= \frac{i k^3}{16\pi^2 \epso c r^2} (\nh \vdot \vp) (\nh \cross \vps) \paren{ \frac{1}{k^2 r^2} + 1 }.
	}
	
	Let $\vL$ be the angular momentum radiated to a distance $R$.  Then
	\eq{
		\vL = \int_R \vl(r) \ddcx
		= \intopi \intotp \intoR \vl(r) \,r^2 \sin\tht \ddr \ddphi \dd\tht,
	}
	and the time derivative is
	\aln{
		\dv{\vL}{t} &= \dv{t}(\intopi \intotp \intoR \vl(r) \,r^2 \sin\tht \ddr \ddphi \dd\tht)
		= \dv{r}{t} \dv{r}(\intopi \intotp \intoR \vl(r) \,r^2 \sin\tht \ddr \ddphi \dd\tht) \notag \\
		&= c \intopi \intotp \vl(r) \,r^2 \sin\tht \ddphi \dd\tht
		= \frac{i k^3}{16\pi^2 \epso} \paren{ \frac{1}{k^2 r^2} + 1 } \intopi \intotp (\nh \vdot \vp) (\nh \cross \vps) \sin\tht \ddphi \dd\tht. \label{dLdt}
	}
	Note that
	\eq{
		[ (\nh \vdot \vp) (\nh \cross \vps) ]_i = \sumje n_j p_j (\nh \cross \vps)_i
		= \sumje \sumke \sumle \epsikl n_j p_j n_k p_l^*,
	}
	so
	\eq{
		\dv{L_i}{t} \propto \sumje \sumke \sumle \epsikl p_j p_l^* \int n_j p_k \ddOmg
		= \sumje \sumke \sumle \epsikl p_j p_l^* \frac{4\pi}{3} \del_{jk}
		= \frac{4\pi}{3} \epsikl p_k p_l^*
		= \frac{4\pi}{3} (\vp \cross \vps)_i,
	}
	where we have used Jackson~(9.47), $\int n_\bet n_\gam \ddOmg = 4\pi \del_{\bet \gam} / 3$.  Making this substitution into Eq.~\refeq{dLdt},
	\eq{
		\dv{\vL}{t} = \frac{i k^3}{6\pi \epso} \paren{ \frac{1}{k^2 r^2} + 1 } (\vp \cross \vps).
	}
	Taking the limit as $r \to \infty$, we find
	\eqn{ans1a}{
		\dv{\vL}{t} = \Re\!\brac{ \frac{i k^3}{12\pi \epso} (\vp \cross \vps) }
		= \Re\!\brac{ -\frac{i k^3}{12\pi \epso} (\vps \cross \vp) }
		= \ans{ \frac{k^3}{12\pi \epso} \Im[ \vps \cross \vp ], }
	}
	as desired. \qed
}

%
%	Jackson 9.8(b)
%

\prob{}{
	What is the ratio of angular momentum radiated to energy radiated?  Interpret.
}

\sol{
	According to Jackson~(9.24), the total power radiated by an oscillating electric dipole $\vp$ is
	\eq{
		P = \dv{E}{t}
		= \frac{c^2 \Zo k^4}{12 \pi} \abs{\vp}^2.
	}
	Then the ratio of angular momentum radiated to energy radiated is
	\eq{
		\frac{\dv*{\vL}{t}}{\dv*{E}{t}} = \frac{k^3}{12\pi \epso} \Im[ \vps \cross \vp ] \frac{12 \pi}{c^2 \Zo k^4 \abs{\vp}^2}
		= \frac{1}{\epso} \Im[ \vps \cross \vp ] \frac{1}{c^2 \Zo k \abs{\vp}^2}
		= \ans{ \frac{\Im[ \vps \cross \vp ]}{\omg \abs{\vp}^2}, }
	}
	where we have used $\Zo = \sqrt{\muo / \epso} = 1 / \sqrt{\epso^2 c^2} = 1 / \epso c$, $c^2 = 1 / (\epso \muo)$, and $\omg = k c$.
	
	In the limit of high frequency, $(\dv*{\vL}{t}) / (\dv*{E}{t}) \to 0$.  In this scenario, the energy radiated dominates over the angular momentum radiated.  Likewise, in the limit of low frequency, $(\dv*{\vL}{t}) / (\dv*{E}{t}) \to \infty$, meaning that angular momentum radiation dominates.  This is sensible because rotational kinetic energy $E \propto \omg^2$, while angular momentum $L \propto \omg$.
}

%
%	Jackson 9.8(c)
%

\prob{}{
	For a charge $e$ rotating in the $xy$ plane at radius $a$ and angular speed $\omg$, show that there is only a $z$ component of radiated angular momentum with magnitude $\dv*{\Lz}{t} = e^2 k^3 a^2 / 6 \pi \epso$.  What about a charge oscillating along the $z$ axis?
}

\sol{
	We know from Homework~5 that the position of a point charge rotating counterclockwise in the $xy$ plane is
	\eq{
		\vx(t) = a \cos(\omg t) \,\vx + a \sin(\omg t) \,\yh.
	}
	\clearpage
	Then the charge distribution is
	\eq{
		\rho(\vx, t) = e \del[ x - a \cos(\omg t) ] \,\del[ y - a \sin(\omg t) ] \,\del(z).
	}
	
	According to Jackson~(4.8), the dipole moment is defined
	\eq{
		\vp = \int \vx' \,\rho(\vx') \ddcxp.
	}
	The components of $\vp$ for the point charge are then
	\al{
		\px &= e \iiint x \,\del[ x - a \cos(\omg t) ] \,\del[ y - a \sin(\omg t) ] \,\del(z) \ddx \ddy \ddz
		= e a \cos(\omg t), \\
		\py &= e \iiint y \,\del[ x - a \cos(\omg t) ] \,\del[ y - a \sin(\omg t) ] \,\del(z) \ddx \ddy \ddz
		= e a \sin(\omg t), \\
		\pz &= e \iiint z \,\del[ x - a \cos(\omg t) ] \,\del[ y - a \sin(\omg t) ] \,\del(z) \ddx \ddy \ddz
		= 0,
	}
	so we can write $\vp = e a \,e^{-i \omg t} (\xh + i\,\yh).$  Substituting into Eq.~\refeq{ans1a},
	\al{
		\dv{\vL}{t} &= \Re\!\brac{ \frac{i k^3}{12\pi \epso} e^2 a^2 e^{-i \omg t} e^{i \omg t} [ (\xh + i\,\yh) \cross (\xh - i\,\yh) ] }
		= \Re\!\brac{ \frac{i e^2 k^3 a^2}{12\pi \epso} (-2i \,\xh \cross \yh) }
		= \Re\!\brac{ \frac{e^2 k^3 a^2}{6\pi \epso} \,\zh } \\
		&= \ans{ \frac{e^2 k^3 a^2}{6\pi \epso} \cos(\omg t) \,\zh, }
	}
	as desired. \qed
	
	A charge oscillating along the $z$ axis with amplitude $a$ has the charge density
	\eq{
		\rho(\vx, t) = e a \,\del(x) \,\del(y) \,\del[ z - \cos(\omg t) ],
	}
	which gives the dipole moment
	\al{
		\px &= e a \iiint x \,\del(x) \,\del(y) \,\del[ z - \cos(\omg t) ] \ddx \ddy \ddz
		= 0, \\
		\py &= e a \iiint y \,\del(x) \,\del(y) \,\del[ z - \cos(\omg t) ] \ddx \ddy \ddz
		= 0, \\
		\pz &= e a \iiint z \,\del(x) \,\del(y) \,\del[ z - \cos(\omg t) ] \ddx \ddy \ddz
		= e a \cos(\omg t).
	}
	In complex notation, $\vp = e a \,e^{-i\omg t} \,\zh$.  Substituting into Eq.~\refeq{ans1a}, we find
	\eq{
		\dv{\vL}{t} = \Re\!\brac{ \frac{i k^3}{12\pi \epso} e^2 a^2 e^{-i \omg t} e^{i \omg t} (\zh \cross \zh) }
		= \ans{ \vO. }
	}
	So we see that a charge undergoing linear motion does not lead to a radiated angular momentum, which is sensible.
}

%
%	Jackson 9.8(d)
%

\prob{}{
	What are the results corresponding to Probs.~{1(a)} and {1(b)} for magnetic dipole radiation?
}

\sol{
	The radiation fields for a magnetic dipole are given by Jackson~(19.35--36),
	\al{
		\vH &= \frac{1}{4\pi} \curly{ k^2 (\nh \cross \vm) \cross \nh \frac{e^{i k r}}{r} + [ 3 \nh (\nh \vdot \vm) - \vm ] \paren{ \frac{1}{r^3} - \frac{i k}{r^2} } e^{i k r} }, &
		\vE &= -\frac{\Zo}{4\pi} k^2 (\nh \cross \vm) \frac{e^{i k r}}{r} \paren{ 1 - \frac{1}{i k r} }.
	}
	\clearpage
	Comparing with Eq.~\refeq{fields1}, we see that $\vH \to -\vE / \Zo$, $\vE \to \Zo \vH$, and $\vp \to \vm / c$ as stated in the book~\cite[p.~413]{Jackson}.  Making these substitutions, the results of Probs.~{1.1(a)} and {(b)} become
	\al{
		\ans{ \dv{\vL}{t}\ }&\ans{= \frac{\muo k^3}{12\pi} \Im[ \vms \cross \vm ], } &
		\ans{ \frac{\dv*{\vL}{t}}{\dv*{E}{t}}\ }&\ans{= \frac{\Im[ \vms \cross \vm ]}{\omg \abs{\vm}^2} }
	}
	where we have used $\mu = 1 / \epso c^2$.
}

\state{Beta function of the Gross-Neveu model~(P\&S~12.2)}{
	Compute $\bet(g)$ in the two-dimensional Gross-Neveu model studied in Problem~11.3,
	\eq{
		\cL = \psibsi i \ptsl \psisi + \frac{1}{2} g^2 (\psibsi \psisi)^2,
	}
	with $i = 1, \ldots, N$.  You should find that this model is asymptotically free.  How was that fact reflected in the solution to Problem~11.3?
}

\sol{
	We saw in Problem~2 of Homework~4 that this Lagrangian can be written as
	\eq{
		\cL = \psibsi i \ptsl \psisi - \sig \psibsi \psisi - \frac{1}{2 g^2} \sig^2,
	}
	where $\sig$ is a new scalar field with no kinetic energy terms.  In the modified minimal subtraction scheme, we found the effective potential was
	\eqn{Veff}{
		\Veff = \sig^2 \curly{ \frac{1}{2 g^2} + \frac{N}{4\pi} \brac{ \ln(\frac{\sig^2}{M^2}) - 1 } }.
	}
	Since $\Gam[ \phicl ] = -(V T) \Veff(\phi)$ by P\&S~(11.50), we have
	\eqn{Gam}{
		\Gam[ \sigcl ] = -(V T)  \sig^2 \curly{ \frac{1}{2 g^2} + \frac{N}{4\pi} \brac{ \ln(\frac{\sig^2}{M^2}) - 1 } }.
	}
	Referring to p.~3 of Lecture~11, we can apply the Callan-Symanzik equation to $\Gam$.   The Callan-Symanzik equation is P\&S~(12.41),
	\eq{
		\brac{ M \pdv{M} + \bet(\lam) \pdv{\lam} + n \gam(\lam) } G^{(n)}(\{ x_i \}; M, \lam) = 0.
	}
	For our problem, $\gam$ is 0 because there are no field insertions.  That is, we have
	\eq{
		\brac{ M \pdv{M} + \bet(g) \pdv{g} } \Gam[ \phicl ] = 0.
	}
	Using Eq.~\refeq{Gam}, note that
	\al{
		\pdv{\Gam}{M} &= (V T) \frac{N \sig^2}{2 \pi M}, &
		\pdv{\Gam}{g} &= (V T) \frac{\sig^2}{g^3}.
	}
	Then
	\eq{
		0 = (V T) \paren{ \frac{N \sig^2}{2 \pi} + \bet(g) \frac{\sig^2}{g^3} }
		\qimplies
		\ans{ \betg = -\frac{N g^3}{2\pi}. }
	}
	This model is asymptotically free because the $\bet$ function is proportional to $-g^3$~\cite[pp.~424--425]{Peskin}.
	
	In 2(e) of Homework~4, we found that the vacuum expectation value of $\sig$ was
	\eq{
		\sig = \pm M e^{-\pi / N g^2} = \pm v.
	}
	We showed that the vacuum expectation value does not depend on the renormalization condition chosen.  This means that we can increase $M \to 0$ while holding $\sig$ constant, and see that $g \to 0$ logarithmically.  This is indicative of an asymptotically-free theory~\cite[p.~425]{Peskin}. \qed
}




\state{Landau diamagnetism}{\ }

%
%	3.1
%

\prob{}{
	Compute the Landau diamagnetic susceptibility for an ultra-relativistic Fermi gas in a weak field.
}

\sol{
	We will assume an electron gas.  The energy levels $\eps$ for a relativistic electron in magnetic field $\vB = B \,\zh$ are given by
	\eq{
		\eps^2 - m^2 - \pz^2 = e B (2n + 1) + e B \sig,
	}
	where $\sig = \pm 1$ are the eigenvalues of $\sig_z$ and $n = 0, 1, 2, \ldots$ ~\cite[p.~101]{Relativistic}.  For an ultra-relativistic electron, $m = 0$.  Then
	\eq{
		\eps^2 = e B (2n + 1 + \sig) + \pz^2.
	}
	In order to take $\sig$ into account, we renumber the states such that $\eps^2 = 2n e B + \pz^2$, where $n$th state has degeneracy 2 for $n > 0$, and the 0th state is not degenerate.  Then
	\eq{
		\eps = \pm c \sqrt{2n e B + \pz^2},
	}
	which has no degeneracy, since the $\pm$ sign splits the degenerate levels, and we have inserted a factor of $c$.  The number of states in the interval $\ddpz$ for a given $\eps$ is the same as in the nonrelativistic case~\cite[p.~173]{Landau2}:
	\eq{
		\frac{2 e V B}{(2\pi\hbar)^2 c} \ddpz,
	}
	where $2 = g$ for the electron.
	
	The relevant thermodynamic potential for a nonrelativistic electron gas, for which $\eps = \pz^2 / 2m + (2n + 1) \muB B$, is~\cite[p.~173]{Landau2}
	\aln{ \label{thing3}
		\Omg &= 2\muB B \sumni f[ \mu - (2n + 1) \muB B ], &
		\qq{where}
		f(\mu) &= -\frac{T m V}{2\pi^2 \hbar^3} \intnii \ln\!\brac{ 1 + \exp(\frac{\mu}{T} - \frac{\pz^2}{2m T}) } \ddpz.
	}
	For the relativistic case, note that
	\eq{
		\frac{\sqrt{2n e B + \pz^2}}{T} = \sqrt{\frac{2n e B}{T^2} + \frac{\pz^2}{T^2}}
		= \sqrt{\frac{4n m c \muB B}{\hbar T^2} + \frac{\pz^2}{T^2}}
	}
	We will ignore the negative energy states for now.  The condition for a weak field is $\muB B \ll T$~\cite[p.~174]{Landau2}.  Let $\muB B / T = u$ and $k = 4n m c / \hbar$.  Then, Taylor expanding for small $u$,
	\al{
		\sqrt{\frac{k u}{T} + \frac{\pz^2}{T^2}} &\approx \sqrt{\frac{\pz^2}{T^2}} + u \brac{ \dv{u}(\sqrt{\frac{k u}{T} + \frac{\pz^2}{T^2}}) }_{u = 0}
		= \frac{\pz}{T} + u \brac{ \frac{k}{2 T \sqrt{k u + \pz^2 / T^2}} }_{u = 0}
		= \frac{\pz}{T} + \frac{k}{2 T} \sqrt{\frac{T^2}{\pz^2}} u \\
		&= \frac{\pz}{T} + \frac{k u}{2 \pz}
		= \frac{\pz}{T} + \frac{4n m c \muB B}{2 \hbar T \pz},
	}
	so the analogue of Eq.~\refeq{thing3} is
	\aln{ \label{Omg2}
		\Omg &\approx 2\muB B \sumni f\!\paren{ \mu - \frac{2n m c \muB B}{\hbar \pz} }, &
		\qq{where}
		f(\mu) &= -\frac{T m V}{2\pi^2 \hbar^3} \intoi \ln\!\brac{ 1 + \exp(\frac{\mu}{T} - \frac{\pz}{2m T}) } \ddpz,
	}
	where we integrate over $\pz \in (0, \infty)$ since we consider only positive energies.
	
	The Euler-Maclaurin formula is~\cite[p.~173]{Landau2}
	\eq{
		\frac{1}{2} F(a) + \sum_{n = 1}^\infty F(a + n) \approx \int_a^\infty F(x) \ddx - \frac{1}{12} F'(a).
	}
	In our case $a = 0$, so
	\eq{
		\sum_{n = 0}^\infty F(n) \approx \intoi F(x) \ddx + \frac{1}{2} F(0) - \frac{1}{12} F'(0),
	}
	where we have added $F(0) / 2$ to both sides.  Feeding this into Eq.~\refeq{Omg2}~\cite[p.~174]{Landau2},
	\al{
		\Omg &\approx 2 \muB B \intoi f\!\paren{ \mu - \frac{2 m c \muB B}{\hbar \pz} x } \ddx + \muB B f(\mu) - \frac{\muB B}{6} \brac{ \dv{n}f\!\paren{ \mu - \frac{2n m c \muB B}{\hbar \pz} }  }_{x = 0} \\[1.1ex]
		&= 2 \muB B \frac{\hbar}{2 m c \muB B} \frac{T m V}{2\pi^2 \hbar^3} \int_\mu^\infty \intoi \pz \ln\!\brac{ 1 + \exp(\frac{x}{T} - \frac{\pz}{2m T}) } \ddpz \ddx + \muB B \,f(\mu) \\
		&\phantom{mmmmmmmmmmmmmmmmmmmmmmmmmmm} - \frac{\muB B}{6} \frac{2 m c \muB B}{\hbar} \frac{T m V}{2\pi^2 \hbar^3} \intoi \frac{1}{\pz} \frac{e^{\mu / T - \pz / 2m T}}{1 + e^{\mu / T - \pz / 2m T}} \ddpz \\[1.1ex]
		&= \frac{T V \pz}{4\pi^2 \hbar^2 c} \int_\mu^\infty \intoi \pz \ln\!\brac{ 1 + \exp(\frac{x}{T} - \frac{\pz}{2m T}) } \ddpz \ddx + \muB B \,f(\mu) - \frac{m^2 c \muB^2 B^2 T V}{6\pi^2 \hbar^4} \intoi \frac{\ddpz}{\pz (1 + e^{\pz / 2m T - \mu / T })}.
	}
	The magnetic moment of the gas is $M = -(\pdv*{\Omg}{B})_{T, V, \mu}$~\cite[p.~172]{Landau}, so
	\eq{
		M = -\muB \,f(\mu) - \frac{m^2 c \muB^2 B TV}{3\pi^2 \hbar^4} \int_0^{\po} \frac{\ddpz}{\pz (1 + e^{\pz / 2m T - \mu / T })},
	}
	and the diamagnetic susceptibility is $\chi = (\pdv*{M}{B}) / V$, so
	\eq{
		\chi = -\frac{m^2 c \muB^2 T}{3\pi^2 \hbar^4} \intoi \frac{\ddpz}{\pz (1 + e^{\pz / 2m T - \mu / T })}.
	}
	
%	We assume the gas is completely degenerate so $T \ll \mu$~\cite[p.~173]{Landau}.  In this case, the integral has nonzero contributions only from $p \approx \po$, where $\po$ is the Fermi momentum.  Then
%	
%	We note that $1 / (1 + e^{\pz / 2m T - \mu / T }) \approx 1 - e^{\pz / 2m T - \mu / T }$ in the limit $\mu / T \gg 1$.  Then
%	\eq{
%		\chi \approx -\frac{m^2 c \muB^2 T}{3\pi^2 \hbar^4} \int_0^{\po} \paren{ \frac{1}{\pz} - \frac{e^{\pz / 2m T - \mu / T }}{\pz} } \ddpz
%		= -\frac{m^2 c \muB^2 T}{3\pi^2 \hbar^4} \bigg[ \ln\pz \bigg]_0^{\po}
%		= -\frac{m^2 c \muB^2 T}{3\pi^2 \hbar^4} \ln\po
%	}
	
	Note that
	\eq{
		\intoi \frac{k^s}{e^{k - \mu} \pm 1} \dd{k} = \mp \Gam(s + 1) \,\Li_{1 + s}(\mp e^{\mu}),
	}
	where $\Li$ is the polylogarithm~\cite{Polylog}.  So we have
	\eq{
		\chi = -\frac{m^2 c \muB^2}{3\pi^2 \hbar^4 V} \intoi \frac{z^{-1}}{1 + e^{z - \mu / T }} \ddz
		= \frac{m^2 c \muB^2 T}{3\pi^2 \hbar^4} \Gam(0) \,\Li_0(-e^\mu),
	}
	which 
	
%	\al{
%		\chi \approx -\frac{m^2 c \muB^2 T}{3\pi^2 \hbar^4} \intoi \frac{\ddpz}{\po (1 + e^{\pz / 2m T - \mu / T })}
%		= -\frac{2 m^3 c \muB^2 T^2}{3\pi^2 \hbar^4 \po} \int_{-\mu/T}^\infty \frac{\ddz}{1 + e^z}
%		= -\frac{2 m^3 c \muB^2 T^2}{3\pi^2 \hbar^4 \po} \paren{ \ln(1 + e^{-\mu / T}) + \frac{\mu}{T} },
%	}
%	so the diamagnetic susceptibility is
%	\eq{
%		\chi = -\frac{2m c \muB}{3 \hbar \pz} \frac{2 m^3 c \muB T V}{3 \pi^2 \hbar^3} \brac{ \frac{\mu}{T} + \ln(1 + e^{-\mu / T}) }
%	}
}


%
%	3.2
%
\clearpage
\prob{(*)}{
	Compute the Landau diamagnetic susceptibility for a Fermi gas confined to a box whose linear size in the $z$ direction is $\Lz \ll \Lx, \Ly$.  The magnetic field is directed along the $z$ direction.  Consider two cases when the energy spacing $(2\pi \hbar / \Lz)^2 / 2m$ is much larger/smaller than the cyclotron energy $\muB B$.
}




\clearpage
\newcommand{\lap}{\nabla^2}
\newcommand{\vF}{\vec{F}}
\newcommand{\nabx}{\nabla_{\!x}}
\newcommand{\absxp}{\abs{\vx'}}
\newcommand{\nh}{\vec{\hat{n}}}
\newcommand{\rh}{\vec{\hat{r}}}
\newcommand{\Gd}{G_D}
\newcommand{\Gdxxp}{\Gd(\vx,\vx')}

\begin{statement}{}
	A point charge of charge $q$ is placed at point $\vx'$ inside a conducting spherical shell of radius $R$.  There is no net charge on the conductor.  The potential inside the sphere is thus given by $q \, \Gdxxp$, where the explicit formula for $\Gdxxp$ for a spherical cavity is given in the lecture notes.
\end{statement}

\begin{problem}
	Find the surface charge density $\sigtv$ on the conducting shell.
\end{problem}

\begin{solution}
	The Green's function for a spherical cavity is given by Eq.~(2.91),
	\beq
		\Gdxxp = \frac{1}{\abs{\vx - \vx'}} + \frac{\alp}{\abs{\vx - \vx''}} \qq{where} \vx'' = \vx' \frac{R^2}{\absxp^2} \qand \alp = - \frac{R}{\absxp}.
	\eeq
	The surface charge density can be found from Eq.~(2.86),
	\beqn \label{scdeq}
		\vE \cdot \nh = 4\pi \sig,
	\eeqn
	where $\vE = -\nabla \phi$ in electrostatics.
	
	We will begin by finding $\vE$.  We will orient our coordinate system such that $\vx'$ (and consequently $\vx''$) points along the $z$ axis.  Note that
	\beq
		\Gdxxp = \frac{1}{\abs{\vx - \vx'}} - \frac{R}{\absxp \abs{\vx - \dfrac{R^2}{\absxp^2} \vx'}}
		= \frac{1}{\sqrt{\vx^2 - 2 \vx \cdot \vx' + {\vx'}^2}} - \frac{R}{\absxp \sqrt{\vx^2 - 2 \dfrac{R^2}{{\vx'}^2} \vx \cdot \vx' + \dfrac{R^4}{{\vx'}^4} {\vx'}^2}}.
	\eeq
	In spherical coordinates, we have
	\beq
		\Gdxxp = \frac{1}{\sqrt{r^2 - 2 r r' \cost + {r'}^2}} - \frac{R}{r'} \frac{1}{\sqrt{r^2 - 2 R^2 r \cost / r' + R^4 / {r'}^2}},
	\eeq
	where we note that $\tht$ is the angle between $\vx$ and the $z$ axis.  The gradient in spherical coordinates is given by
	\beq
		\nabla = \pdv{}{r} \,\rh + \frac{1}{r} \pdv{}{\tht} \,\thh + \frac{1}{r \sint} \pdv{}{\vph} \, \phh.
	\eeq
	The $r$ component of the electric field inside the conductor is then
	\beq
		\Er(\vx) = -q \pdv{\Gdxxp}{r}
		= q \left( \frac{r - r' \cost}{(r^2 - 2 r r' \cost + {r'}^2)^{3/2}} - \frac{R}{r'} \frac{r - R^2 \cost / r'}{(r^2 - 2 R^2 r \cost / r' + R^4 / {r'}^2)^{3/2}} \right).
	\eeq
	Since $\nh = -\rh$ for the inner surface of a sphere, we are interested in only the $r$ component of the field.  On the surface of the sphere, the field is $\Er(r=R) \,\rh$.  So we have
	\begin{align*}
		\Er(r=R) &= q \left( \frac{R - r' \cost}{(R^2 - 2 R r' \cost + {r'}^2)^{3/2}} - \frac{R}{r'} \frac{R - R^2 \cost / r'}{(R^2 - 2 R^3 \cost / r' + R^4 / {r'}^2)^{3/2}} \right) \\
		&= q \left( \frac{R - r' \cost}{{r'}^3 (R^2 / {r'}^2 - 2 R \cost / r' + 1)^{3/2}} - \frac{R}{r'} \frac{R - R^2 \cost / r'}{R^3 (1 - 2 R \cost / r' + R^2 / {r'}^2)^{3/2}} \right) \\
		&= \frac{q}{r'} \frac{R^3 - R^2 r' \cost - R {r'}^2 + R^2 r' \cost}{R^2 {r'}^2 (R^2 / {r'}^2 - 2 R \cost / r' + 1)^{3/2}}
		= \frac{q}{R {r'}^3} \frac{R^2 - {r'}^2}{(R^2 / {r'}^2 - 2 R \cost / r' + 1)^{3/2}}.
	\end{align*}
	Finally, feeding this into \refeq{scdeq},
	\beq
		\sig = -\frac{\vE \cdot \rh}{4\pi}
		= \frac{q}{4\pi R {r'}^3} \frac{{r'}^2 - R^2}{(R^2 / {r'}^2 - 2 R \cost / r' + 1)^{3/2}}
		= \frac{q}{4\pi R \absxp^3} \frac{\absxp^2 - R^2}{(R^2 / \absxp^2 - 2 R \cost / \absxp + 1)^{3/2}}.
	\eeq
\end{solution}
\vfix


\newcommand{\vEo}{\vE_0}
\newcommand{\del}{\delta}
\newcommand{\Etht}{E_\tht}
\newcommand{\Fr}{F_r}

\begin{problem}
	Find the force $\vF$ that must be exerted on the point charge in order to hold it in place.
\end{problem}

\begin{solution}
	The total force on a charge distribution arises only from the external electric field $\vEo$, and is given by Eq.~(2.42) in the lecture notes:
	\beq
		\vF = \int \rhox \, \vEo(\vx) \dcx.
	\eeq
	The force required to keep the point charge in place is equal and opposite to this force, so we need to insert a minus sign.  We also need the $\tht$ component of the field inside the conductor, which is
	\beq
		\Etht(\vx) = -\frac{q}{r} \pdv{\Gdxxp}{\tht}
		= -q \left( \frac{r' \sint}{(r^2 - 2 r r' \cost + {r'}^2)^{3/2}} - \frac{R^3 \sint}{{r'}^2 (r^2 - 2 R^2 r \cost / r' + R^4 / {r'}^2)^{3/2}} \right).
	\eeq
	The charge density for a point charge located at $\vx'$ is given by $\rhox = q \, \delta(\vx - \vx')$.  Evaluating the integral, we have
	\beq
		\vF = -\int q \, \delta(\vx - \vx') \, \vE(\vx) \dcx
		= -q \vE(\vx').
	\eeq
	Recall that we chose $\vx'$ to point along the $z$ axis, so $\tht' = 0$.  The $\tht$ component of $\vF$ is then $0$, and the $r$ component is
	\begin{align*}
		\Fr &= -q^2 \left( \frac{r' - r'}{({r'}^2 - 2 {r'}^2 + {r'}^2)^{3/2}} - \frac{R}{r'} \frac{r' - R^2 / r'}{({r'}^2 - 2 R^2 + R^4 / {r'}^2)^{3/2}} \right)
		= q^2 R {r'}^2 \frac{r' - R^2 / r'}{({r'}^4 - 2 R^2 {r'}^2 + R^4)^{3/2}} \\
		&= -q^2 R {r'}^2 \frac{({r'}^2 - R^2) / r'}{({r'}^2 - R^2)^3}
		= -q^2 \frac{R r'}{({r'}^2 - R^2)^2}.
	\end{align*}
	Since only the $r$ component of $\vF$ is nonzero, it points in the $z$ direction, which we chose to be equivalent to the unit vector $\vx' / \absxp$.  Therefore,
	\beq
		\vF = -q^2 \frac{R \absxp}{(R^2 - \absxp^2)^2} \frac{\vx'}{\absxp}
		= -q^2 \frac{R}{(R^2 - \absxp^2)^2} \vx'.
	\eeq
\end{solution}






\state{Pair correlation function}{\ }

%
%	5.1
%

\prob{}{
	Compute the  pair correlation of density $C(r)= \ev{\ev{n(r) \,n(0)}}$ and the fluctuation of the occupation number $\ev*{\absnk^2}$ of the degenerate Fermi gas ($T \ll \EF$) in 2D.  Discuss various distance regimes.
}

\sol{
	The spatial correlation of the density fluctuations in a 2D Fermi gas is given by
	\eq{
		\ev{\Del\nq \,\Del\nw} = \frac{1}{A^2} {\sum_{\sig, \vp, \vp'}}\!' (1 - \ev*{\npps}) \ev{\nps} e^{i (\vp - \vp') (\vrw - \vrq) / \hbar},
	}
	where $n$ is an occupation number, $\vp$ and $\vp'$ are momenta, and $\sig$ is a spin component~\cite[p.~356]{Landau2}.  We can approximate the sum by an integral using the momentum elements $A \ddsp / (2\pi\hbar)^2$ and $A \ddspp / (2\pi\hbar)^2$:
	\eqn{thing5}{
		\ev{\Del\nq \,\Del\nw} = \frac{1}{(2\pi\hbar)^4} \sumsig \iint (1 - \ev*{\npps}) \ev{\nps} e^{i (\vp - \vp') (\vrw - \vrq) / \hbar} \ddsp \ddspp.
	}
	For the first term~\cite[p.~356]{Landau2},
	\al{
		\iint \ev{\nps} e^{i (\vp - \vp') (\vrw - \vrq) / \hbar} \ddsp \ddspp &= \sumsig \int \ev{\nps} e^{i \vp (\vrw - \vrq) / \hbar} \ddcp \int e^{-i \vp' (\vrw - \vrq) / \hbar} \ddspp \\
		&= \sumsig \int \ev{\nps} \del(\vrw - \vrq) \,e^{i \vp (\vrw - \vrq) / \hbar} \ddsp \\
		&= \del(\vrw - \vrq) \sumsig \int \ev{\nps} \ddsp
		= \evn \del(\vrw - \vrq).
	}
	This is the first term in the definition of the spatial correlation, which is
	\eqn{C}{
		C(\rqq, \rw) = \ev{\Del\nq \,\Del\nw} = \evn \del(\vrw - \vrq) + \evn \nu(r),
	}
	meaning we can associate $\nu(r)$ with the second term in Eq.~\refeq{thing5}~\cite[pp.~351, 356]{Landau2}.  So the correlation function is
	\eq{
		\nu(r) = -\frac{1}{(2\pi\hbar)^4 \evn} \sumsig \abs{ \int e^{i \vp \vdot \vr / \hbar} \ev{\nps} \ddsp }^2.
	}
	For a Fermi gas, $\ev{\nps} = \ev{n_{\vp}} = 1 / (e^{(\eps - \mu) / T} + 1)$, which does not depend on $\sigma$.  Then~\cite[p.~356]{Landau2}
	\aln{
		\nu(r) &= -\frac{g}{\evn (2\pi\hbar)^4} \abs{ \int \frac{e^{i \vp \vdot \vr / \hbar}}{e^{(\eps - \mu) / T} + 1} \ddsp }^2
		= -\frac{g}{\evn (2\pi\hbar)^2} \abs{ \intotp \intoi \frac{p \,e^{i p r \cos\tht / \hbar}}{e^{(\eps - \mu) / T} + 1} \ddp \ddtht }^2 \label{exp5} \\
		&= \frac{(2\pi)^2 g}{\evn (2\pi\hbar)^4} \paren{ \intoi \frac{p\, \Jo(p r / \hbar)}{e^{(\eps - \mu) / T} + 1} \ddp }^2
		\equiv \frac{g}{4 \pi^2 \hbar^4 \evn} I^2, \label{thing5}
	}
	where $\Jo(x)$ is a Bessel function of the first kind, and we have defined $I$.
	 
	For a degenerate gas, $\mu \approx \epso = \po^2 / 2m$, where $\epso$ and $\po$ are the Fermi energy and momentum, respectively.  
	 
%	No particle will have momentum greater than $\po$, so we can integrate only over $p \in (0, \po)$.  Let $x = (\eps - \epso) / T = (p^2 - \po^2) / 2mT$.  Then
%	 \eq{
%	 	\intoi \frac{p\, \Jo(p r / \hbar)}{e^{(\eps - \mu) / T} + 1} \ddp = \intopo \frac{p\, \Jo(p r / \hbar)}{e^x + 1} \ddp.
%	 }
%	 We will consider the distance regimes $r \ll \hbar / \po$ and $r \gg \hbar / \po$.  For $r \ll \hbar / \po$, the argument of $\Jo$ is small everywhere, and we can use the asymptotic approximation $J_m(z) \approx z^m / [ 2^m \,\Gam(m + 1) ]$.  For $m = 0$, $J_0(z) \approx 1$.  This gives us
%	 \eq{
%	 	\intoi \frac{p\, \Jo(p r / \hbar)}{e^{(\eps - \mu) / T} + 1} \ddp \to \intopo \frac{p}{e^x + 1} \ddp.
%	 }
%	 Note that 
	 
	Making this substitution and integrating by parts~\cite[p.~358]{Landau2},
	\al{
		I &= \intoi \frac{p\, \Jo(p r / \hbar)}{e^{(p^2 - \po^2) / 2m T} + 1} \ddp
		= \brac{ \frac{\hbar p \,\Jq(p r / \hbar)}{r (e^{(p^2 - \po^2) / 2m T} + 1)} }_0^\infty + \frac{\hbar \po}{m T r} \intoi \frac{p \,\Jq(p r / \hbar) \,e^{(p^2 - \po^2) / 2m T}}{(e^{(p^2 - \po^2) / 2m T} + 1)^2} \ddp \\
		&= \frac{\hbar \po}{m T r} \intoi \frac{p \,\Jq(p r / \hbar)}{(e^{(p^2 - \po^2) / 2m T} + 1) (e^{-(p^2 - \po^2) / 2m T} + 1)} \ddp,
	}
	where we have used $\dv*{[ x^m J_m(x) ]}{x} = x^m J_{m - 1}(x)$ and the fact that $\Jq(0) = 0$~\cite{Bessel}.  Since $T$ is small, $\exp[ (p^2 - \po^2) / 2mT ]$ is dominated by contributions from $p \approx \po$.  Letting $x = \po (p - \po) / m T$ and $\lam = m T / \hbar \po$~\cite[p.~358]{Landau2}, we can write
	\eq{
		I \approx \frac{\hbar \po}{m T r} \intoi \frac{p \,\Jq(p r / \hbar)}{(e^x + 1) (e^{-x} + 1)} \ddp
		= \frac{\hbar \po}{m T r} \intoi (\hbar \lam x + \po) \frac{\Jq(\po r / \hbar + \lam r x)}{(e^x + 1) (e^{-x} + 1)} \ddp.
	}
	Since $T$ is small, the size of the argument of $\Jq$ depends on the size of $\po r / \hbar$.  We will consider the distance regimes $r \ll \hbar / \po$ and $r \gg \hbar / \po$.
	
	For $r \ll \hbar / \po$, the argument of $\Jo$ is small everywhere, and we can use the asymptotic approximation $J_m(z) \approx z^m / [ 2^m \,\Gam(m + 1) ]$.  For $m = 0$, $J_1(z) \approx z / 2$.  This gives us
	\al{
		I &\approx \frac{\hbar \po}{2 m T r} \intoi (\hbar \lam x + \po) \frac{\po r / \hbar + \lam r x}{(e^x + 1) (e^{-x} + 1)} \ddp \\
		&= \frac{\hbar \po}{2 m T} \paren{ \frac{\po^2}{\hbar} \intoi \frac{\ddp}{(e^x + 1) (e^{-x} + 1)} + 2 \lam \po \intoi \frac{x}{(e^x + 1) (e^{-x} + 1)} \ddp + \hbar \lam^2 \intoi \frac{x^2}{(e^x + 1) (e^{-x} + 1)} \ddp } \\
		&= \frac{\hbar \po}{2 m T} \paren{ \frac{\po^2}{\hbar} \frac{1}{2} + 2 \lam \po \ln(2) + \hbar \lam^2 \frac{\pi^2}{6} }
		= \frac{\hbar \po}{2 m T} \paren{ \frac{\po^2}{\hbar} \frac{1}{2} + \frac{2 \ln(2) m T}{\hbar} + \frac{\pi^2 m^2 T^2}{6 \hbar \po^2} }
		= \frac{\po^3}{4 m T} + \ln(2) + \frac{\pi^2 m T}{12 \po^2},
	}
	where we have used Mathematica to evaluate the integrals.  Substituting into Eq.~\refeq{thing5},
	\eq{
		\nu(r) \approx \frac{g}{4 \pi^2 \hbar^4 \evn} \paren{ \frac{\po^3}{4 m T} + \ln(2) + \frac{\pi^2 m T}{12 \po^2} }^2,
	}
	and feeding this into Eq.~\refeq{C},
	\eq{
		\ans{ C(r) \approx \evn \del(r) + \frac{g}{4 \pi^2 \hbar^4} \paren{ \frac{\po^3}{4 m T} + \ln(2) + \frac{\pi^2 m T}{12 \po^2} }^2 }
	}
	for $r \ll \po / \hbar$.
	
	For $r \gg \po / \hbar$, the argument of $\Jq$ is large.  This means we can use the asymptotic approximation~\cite{Bessel}
	\eq{
		J_m(z) \approx \sqrt{\frac{2}{\pi z}} \cos(z - \frac{m \pi}{2} -\frac{\pi}{4}).
	}
	Let $z = \po r / \hbar + \lam r x$.  Then
	\al{
		I &\approx \frac{\hbar \po}{m T r} \sqrt{\frac{2}{\pi}} \intoi (\hbar \lam x + \po) \frac{\cos(z - 3\pi / 4)}{\sqrt{z} (e^x + 1) (e^{-x} + 1)} \ddp
		= \frac{\hbar \po}{m T r} \sqrt{\frac{2}{\pi}} \intoi \frac{\hbar z}{r} \frac{\cos(z - 3\pi / 4)}{\sqrt{z} (e^x + 1) (e^{-x} + 1)} \ddp \\
		&= \frac{\hbar^2 \po}{m T r^2} \sqrt{\frac{2}{\pi}} \intoi \frac{\sqrt{z} \cos(z - 3\pi / 4)}{(e^x + 1) (e^{-x} + 1)} \ddp
	}
	\hl{idek}
	 
	The general expression for the fluctuation of the occupation number is, in three dimensions,
	\eq{
		\ev*{\abs{\Del\nk}^2} = \frac{g}{(2\pi\hbar)^3 V} \int \ev{n_{\vp}} (1 \mp \ev{n_{\vp + \hbar \vk}}) \dd[3]{p}.
	}
	The 2D analogue is then
	\eq{
		\ev*{\abs{\Del\nk}^2} = \frac{g}{(2\pi\hbar)^2 V} \int \ev{n_{\vp}} (1 \mp \ev{n_{\vp + \hbar \vk}}) \ddsp
	}
	
%	\al{
%		\ev*{\abs{\Del\nk}^2} &= \evnk \frac{1 + \nu(k)}{V}, &
%		\qq{where}
%		\nu(k) &= \int v(r) \,e^{-i \vk \vdot \vr} \ddA.
%	}
%	Substituting Eq.~\refeq{exp5} into the expression for $\nu(k)$,
%	\eq{
%		\nu(k) = -\frac{g}{\evn (2\pi\hbar)^2} \abs{ \intotp \intoi \frac{p \,e^{i p r \cos\tht / \hbar}}{e^{(\eps - \mu) / T} + 1} \ddp \ddtht }^2
%	}
}


%
%	5.2
%
\clearpage
\prob{}{
	Repeat the above for the Bose gas slightly above the condensation temperature.
}

\sol{
	For a Bose gas, $\ev{\nps} = \ev{n_{\vp}} = 1 / (e^{(\eps - \mu) / T} - 1)$, which does not depend on $\sigma$.  Then the analogue of Eq.~\refeq{exp5}~\cite[p.~356]{Landau2}
	\eqn{thing5.2}{
		\nu(r) = \frac{(2\pi)^2 g}{\evn (2\pi\hbar)^4} \paren{ \intoi \frac{p\, \Jo(p r / \hbar)}{e^{(\eps - \mu) / T} - 1} \ddp }^2
		\equiv \frac{g}{4 \pi^2 \hbar^4 \evn} I^2,
	}
	where we have defined $I$.  Just above the condensation temperature $\To$, the integral is dominated by small $p$, so $p^2 / mT \sim \abs{\mu} / T \ll 1$~\cite[p.~358]{Landau2}.  In this limit,
	\eq{
		e^{(\eps - \mu) / T} = \exp(\frac{p^2}{2m T} - \frac{\mu}{T})
		\approx 1 + \frac{p^2}{2m T} - \frac{\mu}{T},
	}
	where we have used $e^x \approx 1 + x$ for small $x$.  Then
	\eq{
		I \approx T \intoi \frac{p\, \Jo(p r / \hbar)}{p^2 / 2m + \abs{\mu}} \ddp
		= \frac{2 T}{m} \Ko\!\paren{ \sqrt{\frac{2 \abs{\mu}}{m}} \frac{r}{\hbar} },
	}
	where we have evaluated the integral using Mathematica, and $\Ko$ is the modified Bessel function of the second kind.  From Eq.~\refeq{thing5.2},
	\eq{
		\nu(r) = \frac{g}{4 \pi^2 \hbar^4 \evn} \brac{ \frac{2 T}{m} \Ko\!\paren{ \sqrt{\frac{2 \abs{\mu}}{m}} \frac{r}{\hbar} } }^2
		= \frac{g T^2}{\pi^2 \hbar^4 m^2 \evn} \Ko^2\!\paren{ \sqrt{\frac{2 \abs{\mu}}{m}} \frac{r}{\hbar} },
	}
	so from Eq.~\refeq{C},
	\eq{
		\ans{ C(r) = \evn \del(r) + \frac{g T^2}{\pi^2 \hbar^4 m^2 \evn} \Ko^2\!\paren{ \sqrt{\frac{2 \abs{\mu}}{m}} \frac{r}{\hbar} }, }
	}
	which is valid for all distance regimes.

	For $r \ll \hbar \sqrt{m / 2 \abs{\mu}}$, we Taylor expand $\Ko(z)$ about $z = 0$.  Using Mathematica, $K(z) = \ln(2) - \gam - \ln z + \order{z^2}$, where $\gam$ is Euler's constant.  Feeding this into Eq.~\refeq{thing5.2}, we find
	\eq{
		\nu(r) \approx \frac{g}{4 \pi^2 \hbar^4 \evn} \brac{ \ln(2) - \gam - \ln(\sqrt{\frac{2 \abs{\mu}}{m}} \frac{r}{\hbar}) }^2
		= \frac{g}{4 \pi^2 \hbar^4 \evn} \brac{ \ln(\frac{2}{\hbar}) - \gam - \frac{1}{2} \ln(\frac{2 \abs{\mu}}{m}) - \ln r }^2,
	}
	so, from Eq.~\refeq{C},
	\eq{
		\ans{ C(r) = \evn \del(r) + \frac{g}{4 \pi^2 \hbar^4} \brac{ \ln(\frac{2}{\hbar}) - \gam - \frac{1}{2} \ln(\frac{2 \abs{\mu}}{m}) - \ln r }^2 }
	}
	for $r \ll \hbar \sqrt{m / 2 \abs{\mu}}$.
	
	For $r \gg \hbar \sqrt{m / 2 \abs{\mu}}$, we use the series expansion about $z \to \infty$ $K_\nu(z) \propto e^{-z} \sqrt{\pi / 2 z} + \order{1/z}$, also evaluated with Mathematica.  Equation~\refeq{thing5.2} becomes
	\eq{
		\nu(r) \approx \frac{g}{4 \pi^2 \hbar^4 \evn} \brac{ \sqrt{\frac{\pi}{2}} \paren{ \sqrt{\frac{m}{2 \abs{\mu}}} \frac{\hbar}{r} }^{1/2} \exp(\sqrt{\frac{2 \abs{\mu}}{m}} \frac{r}{\hbar}) }^2
		= \frac{g}{8 \pi \hbar^3 \evn r} \sqrt{\frac{m}{2 \abs{\mu}}} \exp(2 \sqrt{\frac{2 \abs{\mu}}{m}} \frac{r}{\hbar}),
	}
	so, from Eq.~\refeq{C},
	\eq{
		\ans{ C(r) = \evn \del(r) + \frac{g}{8 \pi \hbar^3 r} \sqrt{\frac{m}{2 \abs{\mu}}} \exp(\sqrt{\frac{8 \abs{\mu}}{m}} \frac{r}{\hbar}) }
	}
	for $r \gg \hbar \sqrt{m / 2 \abs{\mu}}$.
}


%\makebib

\end{document}