\documentclass[11pt]{article}
\usepackage{homework}

\classname{352}
\homeworknum{4}



\begin{document}

% Environments

\newcommand{\state}[2]{\begin{statement}{#1} #2 \end{statement}}
\newcommand{\prob}[2]{\begin{problem}{#1} #2 \end{problem}}
\newcommand{\subprob}[1]{\begin{subproblem} #1 \end{subproblem}}
\newcommand{\sol}[1]{\begin{solution} #1 \end{solution}}
\newcommand{\fig}[2]{\begin{figure} \centering #2  \label{#1} \end{figure}}

\newcommand{\makebib}{
	\vfill
	\color{black}
	\nocite{*}
	\bibliography{references}{}
	\bibliographystyle{lucas_unsrt}
}
	

% Implication

\newcommand{\qwhere}{\quad \text{where} \quad}
\newcommand{\qimplies}{\quad \implies \quad}
\newcommand{\impliesq}{\implies \quad}



% Brackets

\newcommand{\paren}[1]{\left( #1 \right)}
\newcommand{\brac}[1]{\left[ #1 \right]}
\newcommand{\curly}[1]{\left\{ #1 \right\}}


% Greek

\newcommand{\alp}{\alpha}
\newcommand{\bet}{\beta}
\newcommand{\gam}{\gamma}
\newcommand{\del}{\delta}
\newcommand{\eps}{\epsilon}
\newcommand{\zet}{\zeta}
\newcommand{\tht}{\theta}
\newcommand{\kap}{\kappa}
\newcommand{\lam}{\lambda}
\newcommand{\sig}{\sigma}
\newcommand{\ups}{\upsilon}
\newcommand{\omg}{\omega}

\newcommand{\Gam}{\Gamma}
\newcommand{\Del}{\Delta}
\newcommand{\Tht}{\Theta}
\newcommand{\Lam}{\Lambda}
\newcommand{\Sig}{\Sigma}
\newcommand{\Omg}{\Omega}


% Text

\newcommand{\where}{\text{where }}

% Problem 1

\newcommand{\Hint}{H_\text{int}}
\newcommand{\ddcx}{\dd[3]{x}}
\newcommand{\psib}{\bar{\psi}}

\newcommand{\mh}{m_h}
\newcommand{\mmu}{m_\mu}
\newcommand{\me}{m_e}
\newcommand{\ma}{m_a}

\newcommand{\aexpt}{a_\text{expt.}}
\newcommand{\aQED}{a_\text{QED}}
\renewcommand{\GeV}{\giga\electronvolt}

\newcommand{\gamt}{\gam^5}

\state{Spin-wave theory~(P\&S 11.1)}{\hfix}

\prob{ \label{1a}
	Prove the following wonderful formula: Let $\phix$ be a free scalar field with propagator $\ev{T \phix \phio} = \Dx$.  Then
	\eqn{show1}{
		\ev{ T e^{i \phix} e^{-i \phio} } = e^{[ \Dx - \Do ]}.
	}
	(The  factor $\Do$ gives a formally divergent adjustment of the overall normalization.)
}

\sol{
	According to P\&S~(9.18),
	\eq{
		\ev*{T \phi(\xq) \phi(\xw)}{\Omg} = \frac{\int \DDphi \phi(\xq) \phi(\xw) \exp[ i \int \ddqx \cL ]}{\int \DDphi \exp[ i \int \ddqx \cL ]}.
	}
	We use this expression to write the left-hand side of Eq.~\refeq{show1}:
	\eqn{thing1}{
		\ev{ T e^{i \phix} e^{-i \phio} } = \frac{\int \DDphi e^{i \phix} e^{-i \phio} \exp[ i \int \ddqy \cL ]}{\int \DDphi \exp[ i \int \ddqy \cL ]}
		= \frac{\int \DDphi \exp[i \phix - i \phio + i \int \ddqy \cL ]}{\int \DDphi \exp[ i \int \ddqy \cL ]}.
	}
	For a free Klein-Gordon~(i.e., scalar) field, Eq.~(9.39) tells us that the generating functional $\ZJ$ is given by
	\eq{
		\ZJ = \Zo \exp[ -\frac{1}{2} \int \ddqx \ddqy \Jx \DF(x - y) \Jy ],
	}
	where $\Zo = Z[0]$.  Thus, we want to find some $\Jy$ such that
	\eqn{thing1b}{
		\ev{ T e^{i \phix} e^{-i \phio} } = \frac{\ZJ}{\Zo}
	}
	where in general
	\eq{
		\ZJ = \int \DDphi \exp[ i \int \ddqx [ \cL + \Jx \phi(x) ] ]
	}
	by (9.34).  Inspecting Eq.~\refeq{thing1}, we recognize the denominator as $\Zo$ and see that if
	\eq{
		\Jy = \delq(y - x) - \delq(y)
	}
	we have an expression like Eq.~\refeq{thing1b}.  Collecting these findings, we have
	\al{
		\ans{ \ev{ T e^{i \phix} e^{-i \phio} } }&= \frac{\ZJ}{\Zo} \\
		&= \exp[ -\frac{1}{2} \int \ddqy \ddqz \Jy \DF(y - z) \Jz ] \\
		&= \exp[ -\frac{1}{2} \int \ddqy \ddqz \Jy \DF(y - z) [ \delq(z - x) - \delq(z) ] ] \\
		&= \exp[ -\frac{1}{2} \int \ddqy [ \delq(y - x) - \delq(y) ] [ \DF(y - x) - \DF(y) ] ] \\
		&= \exp[ -\frac{1}{2} [ \DF(0) - \DF(x) - \DF(-x) + \DF(0) ] ] \\
		&= \exp[ \DF(x) - \DF(0) ] \\
		&\ans{\; = e^{[ \Dx - \Do ]}, }
	}
	as we wanted to show. \qed
}



\prob{ \label{1b}
	We can use this formula in Euclidean field theory to discuss correlation functions in a theory with spontaneously broken symmetry for $T < \TC$.  Let us consider only the simplest case of a broken $O(2)$ or $U(1)$ symmetry.  We can write the local spin density as a complex variable
	\eq{
		\sx = \sqx + i \swx.
	}
	The global symmetry is the transformation
	\eq{
		\sx \to e^{-i \alp} \sx.
	}
	If we assume that the physics freezes the modulus of $\sx$, we can parameterize
	\eqn{sx}{
		\sx = A e^{i \phix}
	}
	and write an effective Lagrangian for the field $\phix$.  The symmetry of the theory becomes the translation symmetry
	\eqn{symmetry}{
		\phix \to \phix - \alp.
	}
	Show that (for $d > 0$) the most general renormalizable Lagrangian consistent with this symmetry is the free field theory
	\eqn{show1b}{
		\cL = \frac{1}{2} \rho(\vgrad \phi)^2.
	}
	In statistical mechanics, the constant $\rho$ is called the \emph{spin wave modulus}.  A reasonable hypothesis for $\rho$ is that it is finite for $T < \TC$ and tends to 0 as $T \to \TC$ from below.
}

\sol{
	In accordance with the Klein-Gordon Lagrangian in P\&S~(2.6),
	\eqn{KGL}{
		\cL_\text{K-G} = \frac{1}{2} (\pt \phi)^2 - \frac{1}{2} m^2 \phi^2,
	}
	we interpret $(\vgrad \phi)^2$ as $(\pt \phi)^2$.
	
	The Lagrangian cannot have terms of $\order{\phi^n}$ for any $n \neq 0$ since $\phi(x)$ is not invariant under Eq.~\refeq{symmetry}.  Any combination of derivatives of $\phi$ is invariant, however, since $\alp$ is a constant and does not contribute to any derivative.  Thus, only terms like $\pt^n \phi^m$ (where $n$ denotes a power of $\pt$) for $n, m > 0$ and $n \geq m$ are consistent with the symmetry of Eq.~\refeq{symmetry} for $d$ an integer.
	
	Now we must determine which of these terms are renormalizable.  We know that the Lagrangian must have dimension $d$, and that $\phi$ has dimension $(d - 2) / 2$.  Taking a derivative adds a mass dimension.  The theory is renormalizable if the coupling constant $\rho$ has dimension greater than or equal to 0~\cite[p.~322]{Peskin}.  Let $p$ be the dimension of $\rho$.  The dimension of our allowed term is then
	\eq{
		[ \rho \pt^n \phi^m ] = p + n + m \frac{d - 2}{2},
	}
	which we require to be equal to $d$.  Thus we seek solutions to the system of equations
	\al{
		d &= p + n + m \frac{d - 2}{2}, &
		n &\geq m, &
		p &\geq 0.
	}
	Solving with Mathematica, we find that this system has two solutions: $n = m = 2$ and $p = 0$; and $n = m = 1$ and $p = d / 2$.  However, the term $\pt \phi$ for $n = m = 1$ does not contribute to the action because it is a total derivative and does not contribute when the integral over $\cL$ is evaluated:
	\eq{
		\int \dd[d]{x} \pt\phi = \phi \bigg|_{-\infty}^\infty
		= 0.
	}
	Thus the only possibility is $n = m = 2$.  Note that
	\eq{
		\pt^2 \phi^2 = \pt(\pt \phi^2)
		= 2 \pt( \phi \pt \phi)
		= \pt \phi \pt \phi + \phi \pt^2 \phi
		= (\pt \phi)^2,
	}
	since $\phi \pt^2 \phi$ is not invariant under Eq.~\refeq{sx}.  This means that $\rho$ must be dimensionless and that the only allowed terms in the Lagrangian are proportional to $(\pt \phi)^2$, which is consistent with Eq.~\refeq{show1b}. \qed
}



\prob{
	Compute the correlation function $\ev{ \sx \sao }$.  Adjust $A$ to give a physically sensible normalization (assuming that the system has a physical cutoff at the scale of one atomic spacing) and display the dependence of this correlation function on $x$ for $d = 1, 2, 3, 4$.  Explain the significance of your results.
}

\sol{
	Applying Eq.~\refeq{sx},
	\eq{
		\ev{ \sx \sao } = \ev*{ A e^{i \phix} \As e^{-i \phio} }
		= \ev*{ \abs{A}^2 } \ev*{ e^{i \phix} e^{-i \phio} }.
	}
	Now we can apply Eq.~\refeq{show1} to find
	\eqn{thing1c}{
		\ans{ \ev{ \sx \sao } = \abs{A}^2 \exp[ D(x) - D(0) ], }
	}
	where $D(x - y)$ is a Green's function.  Since our Lagrangian is similar to the Klein-Gordon Lagrangian Eq.~\refeq{2.6}, our Green's function is similar to that of the Klein-Gordon operator, which is given by P\&S~(2.56):
	\eq{
		(\pt^2 + m^2) D(x - y) = -i \delq(x - y).
	}
	The Feynman prescription for this Green's function is given by (2.59),
	\eqn{DF}{
		\DF(x - y) = \int \ddqpf \frac{i}{p^2 - m^2 + i \eps} e^{-i p \cdot (x - y)}.
	}
	For the Lagrangian in Eq.~\refeq{show1b}, we set $m = 0$ and insert a factor of $\rho$:
	\eq{
		\rho \pt^2 D(x - y) = -i \deld(x - y),
	}
	so adapting Eq.~\refeq{DF} for this situation yields
	\eqn{DF}{
		\DF(x - y) = \frac{1}{\rho} \int \dddpf \frac{i}{p^2 + i \eps} e^{-i p \cdot (x - y)}.
	}
	We see that $\DF(0)$ diverges, so we absorb it into the constant to make the normalization physically sensible.  We can do this because, as we showed in \ref{1b}, the theory is renormalizable.  Define $A'$ such that
	\eq{
		{A'}^2 = \abs{A}^2 e^{-D(0)}.
	}
	Then Eq.~\refeq{thing1c} can be written
	\eq{
		\ans{ \ev{ \sx \sao } =  {A'}^2 e^{D(x)}. }
	}
	
	To evaluate the divergent integral $D(x)$, we look to the Feynman parameter method we have been using to solve divergent integrals.  Apparently, the Schwinger parametrization is useful in deriving the Feynman parametrization, and it is given by~\cite{Feynman}
	\eq{
		\frac{1}{A} = \intoi \dds e^{-s A}.
	}
	Using this equation, we can write Eq.~\refeq{DF} as
	\eq{
		\DF(x) = \frac{1}{\rho} \int \dddpf \frac{i}{p^2} e^{-i p \cdot x}
		= \frac{i}{\rho} \int \dddpf \intoi \dds e^{-s p^2} e^{-i p \cdot x}.
	}
	Now we can complete the square in the exponential to get a Gaussian integral:
	\al{
		\DF(x) &= \frac{i}{\rho} \int \dddpf \intoi \dds \exp[ -s p^2 - i p \cdot x + \frac{x^2}{4 s} - \frac{x^2}{4 s} ] \\
		&= \frac{i}{\rho} \int \dddpf \intoi \dds \exp[ -s \paren{ p + \frac{i x}{2 s} }^2 - \frac{x^2}{4 s} ] \\
		&= \frac{i}{\rho (2 \pi)^d} \intoi \dds e^{-x^2 / 4 s} \int \dd[d]{u} e^{-s u^2} \\
		&= \frac{i}{\rho (2 \pi)^{d}} \intoi \dds e^{-x^2 / 4 s} \sqrt{ \frac{(2\pi)^d}{(2s)^d} } \\
		&= \frac{i}{\rho (4 \pi)^{d / 2}} \intoi \dds \frac{e^{-x^2 / 4 s}}{s^{d / 2}}
	}
	where we have used~\cite{QFT}
	\eq{
		\int \exp( -\frac{1}{2} x \cdot A \cdot x ) \dd[n]{x} = \sqrt{\frac{(2\pi)^n}{\det A}},
	}
	with $A$ a $d \times d$ diagonal matrix $2s$.  Using Mathematica to integrate with respect to $s$, we find
	\eq{
		\DF(x) = \frac{i}{\rho (4 \pi)^{d / 2}} \frac{2^{d - 2}}{x^{d - 2}} \Gam(d / 2 - 1)
		= \frac{i}{4 \pi^d \rho} \Gam(d / 2 - 1) x^{2 - d}.
	}
	The gamma function diverges as $d \to 2$, so as we have done in previous problems, we expand about $\eps = 2 - d$.  Evaluating the series expansion using Mathematica, we obtain
	\eq{
		\DF(x) = \frac{i}{4 \pi^{1 - \eps} \rho} \Gam(\eps / 2) x^\eps
		\approx \frac{i}{4 \pi \rho} \paren{ \frac{2}{\eps} - \gam + 2 \ln(\pi x) }
		\sim \frac{i}{2 \pi \rho} \ln(x)
		= i \ln(\frac{1}{x^{2 \pi \rho}}).
	}
	We Wick rotate $x \to i x$.  Then the dependence of the correlation function on $x$ for $d = 1, 2, 3, 4$ is
	\ans{\al{
		(d = 1) &\qquad \ev{ \sx \sao } \sim e^{-x / 2 \sqrt{\pi} \rho}, &
		(d = 2) &\qquad \ev{ \sx \sao } \sim x^{2 \pi \rho}, \\
		(d = 3) &\qquad \ev{ \sx \sao } \sim \frac{1}{x}, &
		(d = 4) &\qquad \ev{ \sx \sao } \sim \frac{1}{x^2}.
	}}%
	In $d > 2$ dimensions, the expectation value of the correlation function tends to 0 at large distances $x$.  For $d > 2$, it drops off more quickly as $d$ increases.  The $d \leq 2$ cases depend on $\rho$, which we assume is positive.  The $d = 1$ case drops off with increasing distance, and more quickly with smaller $\rho$.  For $d = 2$, the expectation value of the correlation function increases with increasing distance, and it blows up more quickly with larger $\rho$.
	
	These results are consistent with the Mermin--Wagner theorem, which states that a continuous symmetry cannot be broken in $d \leq 2$ dimensions~\cite{CMW}.  That is, in $d \leq 2$ dimensions, a symmetry-breaking field cannot have a nonzero vacuum expectation value~\cite[p.~460]{Peskin}.  A physical explanation is that each spin has more nearest neighbors in higher dimensions.  Since the spins are inclined to align with their neighbors, there is a higher degree of correlation in higher dimensions at the same distance.  In two dimensions, the correlations are weak enough that they are overpowered by the field fluctuations.
}



\state{Pauli paramagnetism}{
	Cold atomic gases could be realized by atomic isotopes which are fermions ($^6$Li, $^{40}$K, etc.).  Such isotopes may have a large atomic spin.  Assuming that the Fermi gas is degenerate and its constituents have a spin $s> 1 / 2$, compute the Pauli magnetic susceptibility.
}

\sol{
	According to p.~2 of Lecture 12, the magnetic susceptibility is defined
	\eq{
		\chi = \frac{1}{V} \pdv{N}{\mu},
	}
	where $N = \pdv*{\Omg}{\mu}$, and
	\eq{
		\Omg(\mu, B) = \frac{1}{2} \Omgo(\mu + B) + \frac{1}{2} \Omgo(\mu - B)
		\approx \Omgo(\mu) + \frac{B^2}{2} \pdv[2]{\Omgo}{\mu},
	}
	where $B$ is the strength of the magnetic field and $\Omgo$ is the thermodynamic potential with no field present.  \hl{but I think this doesn't work because it is only for spin 1/2}
	
	For a Fermi gas, the thermodynamic potential is~\cite[p.~145]{Landau}
	\eq{
		\Omgo = -T \sumk \ln(1 + e^{(\mu - \epsk) / T}).
	}
	Note that
	\al{
		\pdv{\Omgo}{\mu} &= 
	}
	Then the thermodynamic potential in the magnetic field is
	\eq{
		\Omg = 
	}
}




\clearpage
\state{Landau diamagnetism}{\ }

%
%	3.1
%

\prob{}{
	Compute the Landau diamagnetic susceptibility for ultra-relativistic Fermi gas.
}

\sol{
	
}

%
%	3.2
%

\prob{(*)}{
	Compute the Landau diamagnetic susceptibility for a Fermi gas confined to a box whose linear size in the $z$ direction is $\Lz \ll \Lx, \Ly$.  The magnetic field is directed along the $z$ direction.  Consider two cases when the energy spacing $(2\pi \hbar / \Lz)^2 / 2m$ is much larger/smaller than the cyclotron energy $\muB B$.
}




\clearpage
\state{Degenerate Bose gas}{\hfix}

%
%	4.1
%

\prob{}{
	The chemical potential of the degenerate Bose gas vanishes below $\Ts$ (the critical temperature of the BEC).  Find its temperature dependence at temperatures slightly above $\Ts$.
}

\sol{
	In three dimensions, the energy distribution of a Bose gas is~\cite[p.~149]{Landau}
	\eqn{ddNeps}{
		\ddNeps = \frac{g V}{\pi^2 \hbar^3} \sqrt{\frac{m^3}{2}} \frac{\sqrt{\eps}}{e^{(\eps - \mu) / T} - 1} \ddeps.
	}
	Integrating over all energies, we find the total number of molecules~\cite[p.~149]{Landau}.  This gives an expression relating the chemical potential $\mu$ and the density $\nb$~\cite[p.~159]{Landau}:
	\eqn{nb}{
		\nb = \frac{g}{\pi^2 \hbar^3} \sqrt{\frac{m^3}{2}} \intoi \frac{\sqrt{\eps}}{e^{(\eps - \mu) / T} - 1} \ddeps.
	}
	The critical temperature $\Ts$ satisfies this relation for $\mu = 0$, and can be found by making the substitution $z = \eps / T$:
	\eq{
		\nb = \frac{g}{\pi^2 \hbar^3} \sqrt{\frac{m^3}{2}} \intoi \frac{\sqrt{\eps}}{e^{\eps / T} - 1} \ddeps
		= \frac{g}{\pi^2 \hbar^3} \sqrt{\frac{m^3 T^3}{2}} \intoi \frac{\sqrt{z}}{e^z - 1} \ddeps.
	}
	The integral may be evaluated using the formula~\cite[p.~156]{Landau}
	\eqn{formula}{
		\intoi \frac{z^{x - 1}}{e^z - 1} \ddz = \Gam(x) \zeta(x),
	}
	with $x > 1$.  The relevant values are $\Gam(3/2) = \sqrt{\pi} / 2$, and $\zeta(3/2) = 2.612$~\cite[p.~156]{Landau}.  Thus,
	\eq{
		\nb = \frac{g}{\pi^2 \hbar^3} \sqrt{\frac{m^3 T^3}{2}} (2.612)\frac{\sqrt{\pi}}{2}
		= \frac{g}{\pi^2 \hbar^3} \sqrt{\frac{m^3 T^3}{2}} (2.612)\frac{\sqrt{\pi}}{2}
		= \frac{0.9235 \,g}{\hbar^3} \paren{ \frac{m T}{\pi} }^{3/2},
	}
	and
	\eq{
		\paren{ \frac{m \Ts}{\pi} }^{3/2} = \frac{\nb \hbar^3}{0.9235 \,g}
		\qimplies
		\Ts = \frac{\pi}{m} \paren{ \frac{\nb \hbar^3}{0.9235 \,g} }^{2/3}
		= \frac{1.054\, \pi}{m \hbar^2} \paren{ \frac{\nb}{g} }^{2/3}.
	}
	
	Define the function
	\eq{
		\nbs(T) = \frac{g}{\pi^2 \hbar^3} \sqrt{\frac{m^3}{2}} \intoi \frac{\sqrt{\eps}}{e^{\eps / T} - 1} \ddeps
		= \frac{0.9235 \,g}{\hbar^3} \paren{ \frac{m T}{\pi} }^{3/2},
	}
	and note that $\nbs(\Ts) = \nb$.  Then we can rewrite Eq.~\refeq{nb} as
	\eq{
		\nb = \nbs(T) + \frac{g}{\pi^2 \hbar^3} \sqrt{\frac{m^3}{2}} \intoi \frac{\sqrt{\eps}}{e^{(\eps - \mu) / T} - 1} \ddeps - \nbs(T)
		= \nbs(T) + \frac{g}{\pi^2 \hbar^3} \sqrt{\frac{m^3}{2}} \intoi \paren{ \frac{\sqrt{\eps}}{e^{(\eps - \mu) / T} - 1} - \frac{\sqrt{\eps}}{e^{\eps / T} - 1} } \ddeps.
	}
	Expanding the integrand for small exponential powers using $e^x \approx 1 + x$, which is vaild for $T - \Ts \ll 1$, we find
	\eq{
		\frac{\sqrt{\eps}}{e^{(\eps - \mu) / T} - 1} - \frac{\sqrt{\eps}}{e^{\eps / T} - 1}
		\approx \frac{\sqrt{\eps}}{1 + (\eps - \mu) / T - 1} - \frac{\sqrt{\eps}}{1 + \eps / T - 1}
		= \frac{T \sqrt{\eps}}{\eps - \mu} - \frac{T}{\sqrt{\eps}}
		= \frac{T \eps - T (\eps - \mu)}{\sqrt{\eps} (\eps - \mu)}
		= \frac{T \mu}{\sqrt{\eps} (\eps - \mu)}.
	}
	Then the integral is
	\eq{
		T \mu \intoi \frac{\ddeps}{\sqrt{\eps} (\eps - \mu)} = T \mu \frac{\pi}{\sqrt{-\mu}}
		= \pi T \sqrt{-\mu},
	}
	so long as $\mu < 0$, which is true for the Bose distribution~\cite[p.~145]{Landau}.  Making this substitution and solving for $\mu$, we find
	\eqn{density}{
		\nb = \nbs(T) - \frac{g T}{\pi \hbar^3} \sqrt{\frac{-\mu m^3}{2}}
		\qimplies
		\mu = -\frac{2}{m^3} \paren{ \frac{\pi \hbar^3 [ \nbs(T) - \nb ]}{g T} }^2
		= -\frac{2 \pi^2 \hbar^6 [ \nbs(T) - \nb ]^2}{m^3 g^2 T^2}.
	}
	Note that
	\eq{
		\nbs(T) - \nb = \nb \paren{ \frac{\nbs(T)}{\nb} - 1 }
		= \nb \paren{ \frac{\nbs(T)}{\nbs(\Ts)} - 1 }
		= \nb \paren{ \frac{T^{3/2}}{{\Ts}^{3/2}} - 1 },
	}
	since $\nbs(\Ts) = \nb$.  Then the relationship between chemical potential and temperature is
	\eqn{mu}{
		\mu = -\frac{2 \pi^2 \hbar^6 \nb^2}{m^3 g^2 T^2} \paren{ \frac{T^{3/2}}{{\Ts}^{3/2}} - 1 }^2
		= \ans{ -\frac{2 \pi^2 \hbar^6 \nb^2}{m^3 g^2} \paren{ \frac{T^{1/2}}{{\Ts}^{3/2}} - \frac{1}{T} }^2. }
	}
	Since $T / \Ts \approx 1$, the leading behavior is \ans{$\mu \sim -1 / T^2$.}
}

%
%	4.2
%

\prob{}{
	Find the discontinuities in the derivatives of thermodynamic quantities (energy, particle density, entropy, thermodynamic potential, and specific heat) at the BEC transition.  Which order is this phase transition?
}

\sol{
	Using Eq.~\refeq{ddNeps}, the energy of the Bose gas is
	\eq{
		E = \intoi \eps \ddNeps
		= \frac{g V}{\pi^2 \hbar^3} \sqrt{\frac{m^3}{2}} \intoi \frac{\eps^{3/2}}{e^{(\eps - \mu) / T} - 1} \ddeps.
	}
	\clearpage
	The thermodynamic potential for a Bose gas is~\cite[p.~146]{Landau}
	\eq{
		\Omg = T \sumk \ln(1 - e^{(\mu - \epsk) / T}).
	}
	Transforming the sum to an integral as in Prob.~{3.2}, we have~\cite[p.~149]{Landau}
	\al{
		\Omg &= \frac{g V T}{\pi^2 \hbar^3} \sqrt{\frac{m^3}{2}} \intoi \sqrt{\eps} \ln(1 - e^{(\mu - \eps) / T}) \ddeps \\
		&= \frac{g V T}{\pi^2 \hbar^3} \sqrt{\frac{m^3}{2}} \paren{ \brac{ \frac{2}{3} \eps^{3/2} \ln(1 - e^{(\mu - \epsk) / T}) }\oi - \frac{2}{3 T} \intoi \frac{\eps^{3/2}}{e^{(\eps - \mu) / T} - 1} \ddeps } \\
	&= -\frac{3 g V T}{\pi^2 \hbar^3} \paren{ \frac{m}{2} }^{3/2} \intoi \frac{\eps^{3/2}}{e^{(\eps - \mu) / T} - 1} \ddeps
	= -\frac{2}{3} E.
	}
	
	Note that $N = -(\pdv*{\Omg}{\mu})_{T, V}$~\cite[p.~24]{Landau}.  Then~\cite[p.~161]{Landau}
	\eq{
		\nb = -\frac{1}{V} \pdv{\Omg}{\mu}
		= \frac{2}{3 V} \pdv{E}{\mu}
		\approx \nbs,
	}
	since the contribution to $\nb$ is small for $\mu \ll 1$.  This gives us
	\al{
		\Omg &= \Omgs - \nbs V \mu, &
		E &= \Es + \frac{3}{2} \nbs V \mu,
	}
	where $\Omgs$ and $\Es$ are the thermodynamic potential and the energy at $\mu = 0$.  Using Eq.~\refeq{formula},
	\al{
		\Es &= \frac{g V}{\pi^2 \hbar^3} \sqrt{\frac{m^3}{2}} \intoi \frac{\eps^{3/2}}{e^{\eps / T} - 1} \ddeps
		= \frac{g V}{\pi^2 \hbar^3} \sqrt{\frac{m^3 T^5}{2}} \intoi \frac{z^{3/2}}{e^z - 1} \ddz
		= \frac{g V}{\pi^2 \hbar^3} \sqrt{\frac{m^3 T^5}{2}} \Gam(5/2) \zeta(5/2) \\
		&= \frac{0.711 \,g V}{\hbar^3} \sqrt{\frac{m^3 T^5}{\pi^3}}, \\[2ex]
		\Omgs &= -\frac{0.474 \,g V}{\hbar^3} \sqrt{\frac{m^3 T^5}{\pi^3}},
	}
	both of which are continuously differentiable in $T$.  So the discontinuities in the $T$ derivatives of $\Omg$ and $E$ stem from $\mu$, given by Eq.~\refeq{mu}.  Since
	\eq{
		\pdv{\mu}{T} \sim -\pdv{T}(\frac{1}{T^2})
		\propto -\frac{1}{T^3},
	}
	where $T$ is, by definition, slightly larger than $\Ts$, we conclude that
	\ans{
	\al{
		\pdv{\Omg}{T} &\sim \frac{1}{(T - \Ts)^3}, &
		\pdv{E}{T} &\sim -\frac{1}{(T - \Ts)^3},
	}
	which both have infinite discontinuities at $T = \Ts$.
	}
	
	The particle density is given in Eq.~\refeq{density}.  Differentiating with respect to chemical potential, we see that
	\eq{
		\pdv{\nb}{\mu} = \pdv{\mu}(\nbs(T) - \frac{g T}{\pi \hbar^3} \sqrt{\frac{-\mu m^3}{2}})
		\propto \ans{ \frac{1}{\sqrt{-\mu}}, }
	}
	\ans{which diverges as $\mu \to 0$ from the left} (and is negative for real $\mu$).
	
	Entropy can be found by $S = -(\pdv*{\Omg}{T})_{V, \mu}$~\cite[p.~150]{Landau}, and the specific heat by $\Cv = (\pdv*{E}{T})_V$~\cite[p.~165]{Landau}.  Since
	\al{
		S &\sim -\frac{1}{T^3}, &
		\Cv &\sim -\frac{1}{T^3},
	}
	again for small $T - \Ts$,
	\ans{
	\al{
		\pdv{S}{T} &\sim -\pdv{T}(\frac{1}{T^3})
		\sim -\frac{1}{(T - \Ts)^4}, &
		\pdv{\Cv}{T} &\sim -\frac{1}{(T - \Ts)^4},
	}
	which both have infinite discontinuities at $T = \Ts$.
	}
	
	The order of a phase transition is determined by whether the first or the second derivative of the free energy with respect to some thermodynamic quantity is discontinuous~\cite{Wikipedia}.  The free energy can be found by $F - \mu N = \Omg$~\cite[p.~69]{Landau}.  Since $\pdv*{\mu}{T}$ and $\pdv*{\Omg}{T}$ are discontinuous, so is $\pdv*{F}{T}$.  Thus, this is a \ans{ first-order phase transition. }
}

%
%	4.3
%

\prob{}{
	Can the ideal Bose gas condense in spatial dimensions 1 and 2?  Discuss what happens in these cases. 
}

\sol{
	The ideal Bose gas can condense if the equivalent of Eq.~\refeq{nb} can be solved with $\mu = 0$ to obtain an expression for $\Ts$.  The number of quantum states in the interval $\ddp$ is the same as for a Fermi gas, and so is given by Eq.~\refeq{Nstates}~\cite[p.~148]{Landau}.  Transforming this to the number of states in the interval $\ddeps$ by Eq.~\refeq{transform}, we obtain
	\aln{ \label{thing4}
		\frac{g L}{2\pi \hbar} \sqrt{\frac{m}{2}} \frac{1}{\sqrt{\eps}} \ddeps
		&\quad (d = 1), &
		\frac{m g A}{2\pi \hbar^2} \ddeps
		&\quad (d = 2), &
		\frac{g V}{\pi^2 \hbar^3} \sqrt{\frac{m^3}{2}} \eps^{3/2} \ddeps
		&\quad (d = 3).
	}
	Applying the expression for the total number of particles in a Bose gas~\cite[p.~146]{Landau},
	\eq{
		N = \sumk \frac{1}{e^{(\epsk - \mu) / T} - 1},
	}
	replacing the sum by an integral over $p \in (0, \infty)$, and transforming coordinates to $z = \eps / \Ts$ as in Prob.~{4.1}, we obtain
	\al{
		(d = 1) \quad
		\nb &= \frac{g L}{2\pi \hbar} \sqrt{\frac{m}{2}} \intoi \frac{\ddeps}{\sqrt{\eps} (e^{\eps / \Ts} - 1)}
		= \frac{g L}{2\pi \hbar} \sqrt{\frac{m \Ts}{2}} \intoi \frac{\ddz}{\sqrt{z} (e^z - 1)}
		\to \infty, \\[2ex]
		(d = 2) \quad
		\nb &= \frac{m g A}{2\pi \hbar^2} \intoi \frac{\ddeps}{e^{\eps / \Ts} - 1}
		= \frac{m g A \Ts}{2\pi \hbar^2} \intoi \frac{\ddz}{e^z - 1}
		\to \infty.
	}
	Both integrals diverge, making it impossible to solve for $\Ts$ in either case.
	
	However, these integrals will converge in the limit that $z \to \infty$, which is equivalent to $T \to 0$.  In this limit,
		\al{
		(d = 1) \quad
		\limTo \nb &= \frac{g L}{2\pi \hbar} \sqrt{\frac{m \Ts}{2}} \intoi \frac{\ddz}{e^z \sqrt{z}}
		= \frac{g L}{2 \hbar} \sqrt{\frac{m \Ts}{2 \pi}}, \\[2ex]
		(d = 2) \quad
		\limTo \nb &= \frac{m g A \Ts}{2 \pi \hbar^3} \intoi \frac{\ddz}{e^z}
		= \frac{m g A \Ts}{2 \pi \hbar^3}.
	}
	Thus, we conclude that, \ans{it is not possible for the 1D and 2D ideal Bose gases to condense above $T = 0$.}
	
	Referring back to Eq.~\refeq{thing4}, for $d = 1$ the number of states in the interval $\ddeps$ diverges as $\eps \to 0$.  For $d = 2$, the number of states is independent of $\eps$.  For $d = 3$, the number of states approaches 0 as $\eps \to 0$.  It would seem that, in 1D and in 2D, there are many states with very low energy that may be occupied instead of $\eps = 0$, while this is not the case in 3D.  Since the particles are therefore not ``forced'' into the ground state at nonzero temperature, the gas will not condense.
}





\state{Pair correlation function}{\ }

%
%	5.1
%

\prob{}{
	Compute the  pair correlation of density $C(r)= \ev{\ev{n(r) \,n(0)}}$ and the fluctuation of the occupation number $\ev*{\absnk^2}$ of the degenerate Fermi gas ($T \ll \EF$) in dimensions $d = 1, 2, 3$. Discuss various distance regimes.
}

%
%	5.2
%

\prob{}{
	Repeat the above for the Bose gas above the condensation temperature.
}


%\makebib

\end{document}