\state{\textit{H}-theorem and Pauli kinetic balance equation}{
	The Pauli balance equation (a version of the Boltzmann kinetic equation more suitable for a quantum setting) reads
	\eqn{Pauli}{
		\dwi = \sumj (\Pij \wj - \Pji \wi),
	}
	where $\wi$ is the probability of a system to be in the state $\ki$ and $\Pij$ is a transition probability rate (i.e.~the probability of a state $\ki$ to transition to $\kj$ during unit time).  In addition, a detailed balance condition is imposed: $\Pij = \Pji$.
}

%
%	1.1
%

\setcounter{subsection}{1}

%
%	1.1(a)
%

\subprob{
	Show that the Pauli balance equation respects the normalization condition $\sumi \wi = 1$.
}

\sol{
	Since $\Pij = \Pji$,
	\eq{
		\sumi \sumj \Pij \wj = \sumi \sumj \Pji \wj.
	}
	Swapping indices on the right side,
	\eq{
		\sumi \sumj \Pij \wj = \sumi \sumj \Pij \wi
		= \sumi \sumj \Pji \wi,
	}
	where we have once again applied $\Pij = \Pji$.  Then, by Eq.~\refeq{Pauli},
	\eqn{d0}{
		\sumi \dwi = \sumi \sumj (\Pij \wj - \Pij \wi) = 0.
	}
	This implies $\sumi \wi = k$, where $k$ is some constant.  If $k \neq 1$, we may redefine $\wi \to \wi / k$ without affecting the validity of the proof.  Thus, we have shown that Eq.~\refeq{Pauli} respects the normalization condition. \qed
}

%
%	1.1(b)
%

\subprob{
	Show that the Pauli balance equation is time irreversible.
}

\sol{
	We will provide a counterexample that shows the equation is \emph{not} time \emph{reversible}.
	
	Assume the probabilities are properly normalized, so $\sumi \Pij = \sumj \Pij = 1$.  Consider a two-state system with states $\kq$ and $\kw$, which has
	\eq{
		P = \mqty[ 1 - \mu & \mu \\ \mu & 1 - \mu ],
	}
	where $0 \leq \mu \leq 1$.  Applying Eq.~\refeq{Pauli}, we obtain the system of differential equations
	\al{
		\dwq &= (\Pqq \wq - \Pqq \wq) + (\Pqw \ww - \Pwq \wq)
		= \mu (\ww - \wq), \\
		\dww &= (\Pwq \wq - \Pqw \ww) + (\Pww \ww - \Pww \ww)
		= \mu (\wq - \ww).
	}
	This system can be written as the matrix equation
	\eq{
		\dv{t} \mqty[ \wq \\ \ww ] = \mu \mqty[ -1 & 1 \\ 1 & -1 ] \mqty[ \wq \\ \ww ]
		\equiv M \mqty[ \wq \\ \ww ],
	}
	where we have defined the matrix $M$.  $M$ has eigenvalues $\lam$ given by
	\eq{
		0 = \mqty| -(\mu + \lam) & \mu \\ \mu & -(\mu + \lam) |
		= (\mu + \lam)^2 - \mu^2
		\qimplies
		(\mu + \lam)^2 = \mu^2
		\qimplies
		\lam = -2\mu, 0.
	}
	The respective eigenvectors $u, v$ can be found by
	\al{
		\mu \mqty[ -1 & 1 \\ 1 & -1 ] \mqty[ \uq \\ \uw ] &= -2\mu \mqty[ \uq \\ \uw ]
		\qimplies -\uq + \uw = -2\mu \uq
		\qimplies \uq = 1, \uw = -1, \\
		\mu \mqty[ -1 & 1 \\ 1 & -1 ] \mqty[ \vq \\ \vw ] &= 0 \mqty[ \vq \\ \vw ]
		\qimplies -\vq + \vw = 0
		\qimplies \vq = \vw = 1.
	}
	
	We can analyze the behavior of the system using stability analysis methods which are well known in applied mathematics, but which we will not prove here~\cite[pp.~127--130]{Strogatz}.  Since one of the eigenvalues is 0, there is a line of fixed points along the direction of the corresponding eigenvector, $\vb{v} = (1, 1)$.  Since the other eigenvalue is negative, these fixed points are stable; all trajectories are along $\vb{u} = (-1, 1)$ and point toward the fixed points.
	
	In practice, however, the normalization condition $\sumi \wi = 1$ restricts the system to a line.  Figure~\refeq{pplane} shows trajectories in the $(\wq, \ww)$ phase plane.  The green line indicates the line of stable fixed points.  The orange line indicates the allowed values of $\wq, \ww$.  For any initial condition along the line, the system will tend toward the point $\wq = \ww = 1/2$.  The system can never return to its initial condition (unless it starts at the equilibrium point, in which cases it remains there for all time).  Hence, \ans{the system described by Eq.~\refeq{Pauli} is time irreversible.} \qed
	
	\begin{figure} \centering
		\includegraphics[width=0.5\textwidth]{1-1(b)}
		\caption{Plot of the $(\wq, \ww)$ phase plane indicating trajectories.  The normalized system is confined to the orange line.  The green line represents stable equilibirum points.}
		\label{pplane}
	\end{figure}

	These arguments and the phase space shown in Fig.~\refeq{pplane} are easily generalized to higher dimensions, using the fact that the signs of the trace and determinant of $M$ are sufficient to determine the type of fixed points and their stability~\cite[pp.~136--137]{Strogatz}.  However, one example is sufficient to show that Eq.~\refeq{Pauli} is \emph{not} time \emph{reversible}.
}

%
%	1.1(c)
%
\clearpage
\subprob{
	Show that the entropy $S = -\sumi \wi \ln \wi$ is non-decreasing: $\dS \geq 0$. 
}

\sol{
	Note that
	\eq{
		\dS = -\sumi \dv{t}(\wi \ln \wi)
		= -\sumi \dv{\wi}{t} \dv{\wi}(\wi \ln \wi)
		= -\sumi \dwi (\ln \wi + 1)
		= -\sumi \dwi \ln \wi,
	}
	where we have applied Eq.~\refeq{d0}.  We now apply Eq.~\refeq{Pauli}:
	\eq{
		\dS = -\sumi \sumj (\Pij \wj - \Pji \wi) \ln \wi
		= -\frac{1}{2} \paren{ \sumi \sumj (\Pij \wj - \Pji \wi) \ln \wi + \sumj \sumi (\Pji \wi - \Pij \wj) \ln \wj },
	}
	where we have split the sum in half and swapped indices for the second half.  Then, using the symmetry of $P$,
	\al{
		\dS &= -\frac{1}{2} \sumi \sumj \Pij [ (\wj - \wi) \ln \wi + (\wi - \wj) \ln \wj ]
		= \frac{1}{2} \sumi \sumj \Pij [ (\wi - \wj) \ln \wi - (\wi - \wj) \ln \wj ] \\
		&= \frac{1}{2} \sumi \sumj \Pij (\wi - \wj) (\ln \wi - \ln \wj).
	}
	Since $w_i$ represent probabilities, $0 \leq \wi \leq 1$ for all $i$, which implies $\ln \wi \leq 0$.  If $\wi > \wj$, $\ln \wj$ is more negative than $\ln \wi$.  That is,
	\al{
		\wi \geq \wj &\qimplies \ln \wi - \ln \wj \geq 0 \qq{and} \wi - \wj \geq 0, \\
		\wi \leq \wj &\qimplies \ln \wi - \ln \wj \leq 0 \qq{and} \wi - \wj \leq 0.
	}
	Thus, \ans{$\dS \geq 0$} as desired. \qed
}

%
%	1.2
%

\prob{}{
	{\Renyi} entropy of the order $\alp$ is defined by the formula $\Salp = 1 / (1 -\alp) \ln \sumi \wi^\alp$.
}

%
%	1.2(a)
%

\subprob{
	Show that {\Renyi} entropy of the order 1 is the Boltzmann entropy (in the context of information theory, Boltzmann entropy is called Shannon entropy).
}

\sol{
	Firstly,
	\eq{
		\Salp = \limaq \frac{1}{1 - \alp} \ln \sumi \wi^\alp.
	}
	Note that
	\al{
		\limaq \ln \sumi \wi^\alp &= \ln \sumi \wi
		= \ln(1)
		= 0, &
		\limaq (1 - \alp) &= 0,
	}
	where we have used the result of Prob.~{1.1(a)}.  Applying L'H\^{o}pital's rule, we find
	\eq{
		\limaq \Salp = \limaq \frac{\dv*{(\ln \sumi \wi^\alp)}{\alp}}{\dv*{(1 - \alp)}{\alp}}
		= \limaq -\frac{\dv*{(\sumi \wi^\alp)}{\alp}}{\sumi \wi^\alp}
		= \limaq -\sumi \wi^\alp \ln \wi
		= \ans{ -\sumi \wi \ln \wi, }
	}
	where we have used $\dv*{(a^x)}{x} = (\ln a) a^x$~\cite{Derivative}.  This is the Shannon entropy, as desired.  \qed
}

%
%	1.2(b)
%
\clearpage
\subprob{
	Show that {\Renyi} entropy doesn't decrease: $\dSalp \geq 0$.
}

\sol{
	We note that
	\al{
		\dSalp &= \dv{t}(\frac{1}{1 - \alp} \ln \sumi \wi^\alp)
		= \frac{1}{1 - \alp} \dv{t}(\ln \sumi \wi^\alp)
		= \frac{1}{1 - \alp} \frac{1}{\sumi \wi^\alp} \dv{t}(\sumi \wi^\alp)
		= \frac{1}{1 - \alp} \frac{1}{\sumi \wi^\alp} \alp \sumi \dwi \wi^{\alp - 1} \\
		&= \frac{\alp}{1 - \alp} \frac{\sumi \dwi \wi^{\alp - 1}}{\sumi \wi^\alp}.
	}
	Applying Eq.~\refeq{Pauli} and the same trick as in Prob.~{1.1(c)},
	\al{
		\dSalp &= \frac{\alp}{1 - \alp} \frac{1}{\sumi \wi^\alp} \sumi \wi^{\alp - 1} \sumj (\Pij \wj - \Pji \wi) \\
		&= \frac{\alp}{1 - \alp} \frac{1}{2 \sumi \wi^\alp} \paren{ \sumi \wi^{\alp - 1} \sumj (\Pij \wj - \Pji \wi) + \sumj \wj^{\alp - 1} \sumi (\Pji \wi - \Pij \wj) } \\
		&= \frac{\alp}{1 - \alp} \frac{1}{2 \sumi \wi^\alp} \sumi \sumj \Pij [ \wi^{\alp - 1} (\wj - \wi) + \wj^{\alp - 1} (\wi - \wj) ] \\
%		&= \frac{\alp}{1 - \alp} \frac{1}{2 \sumi \wi^\alp} \sumi \sumj \Pij [ \wi^{\alp - 1} (\wj - \wi) - \wj^{\alp - 1} (\wj - \wi) ] \\
		&= \frac{\alp}{1 - \alp} \frac{1}{2 \sumi \wi^\alp} \sumi \sumj \Pij [ (\wi^{\alp - 1} - \wj^{\alp - 1}) (\wj - \wi) ].
	}
	Keeping in mind that $0 \leq \wi \leq 1$, this result is non-negative in all possible regimes:
	\al{
		\wj \geq \wi \qq{and} \alp < 1
		&\qimplies
		\frac{\alp}{1 - \alp} > 0 \qq{and} \wi^{\alp - 1} - \wj^{\alp - 1} \geq 0 \qq{and} \wj - \wi \geq 0, \\
		\wj \geq \wi \qq{and} \alp > 1
		&\qimplies
		\frac{\alp}{1 - \alp} < 0 \qq{and} \wi^{\alp - 1} - \wj^{\alp - 1} \leq 0 \qq{and} \wj - \wi \geq 0, \\
		\wj \leq \wi \qq{and} \alp < 1
		&\qimplies
		\frac{\alp}{1 - \alp} > 0 \qq{and} \wi^{\alp - 1} - \wj^{\alp - 1} \leq 0 \qq{and} \wj - \wi \leq 0, \\
		\wj \leq \wi \qq{and} \alp > 1
		&\qimplies
		\frac{\alp}{1 - \alp} < 0 \qq{and} \wi^{\alp - 1} - \wj^{\alp - 1} \geq 0 \qq{and} \wj - \wi \leq 0.
	}
	Of course $\sumi \wi^\alp > 0$ in any case.  Thus, \ans{$\dSalp \geq 0$} as desired. \qed
}