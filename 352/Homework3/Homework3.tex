\documentclass[11pt]{article}
\usepackage{homework}

\classname{352}
\homeworknum{3}



\begin{document}

% Environments

\newcommand{\state}[2]{\begin{statement}{#1} #2 \end{statement}}
\newcommand{\prob}[2]{\begin{problem}{#1} #2 \end{problem}}
\newcommand{\subprob}[1]{\begin{subproblem} #1 \end{subproblem}}
\newcommand{\sol}[1]{\begin{solution} #1 \end{solution}}
\newcommand{\fig}[2]{\begin{figure} \centering #2  \label{#1} \end{figure}}

\newcommand{\makebib}{
	\vfill
	\color{black}
	\nocite{*}
	\bibliography{references}{}
	\bibliographystyle{lucas_unsrt}
}
	

% Implication

\newcommand{\qwhere}{\quad \text{where} \quad}
\newcommand{\qimplies}{\quad \implies \quad}
\newcommand{\impliesq}{\implies \quad}



% Brackets

\newcommand{\paren}[1]{\left( #1 \right)}
\newcommand{\brac}[1]{\left[ #1 \right]}
\newcommand{\curly}[1]{\left\{ #1 \right\}}


% Greek

\newcommand{\alp}{\alpha}
\newcommand{\bet}{\beta}
\newcommand{\gam}{\gamma}
\newcommand{\del}{\delta}
\newcommand{\eps}{\epsilon}
\newcommand{\zet}{\zeta}
\newcommand{\tht}{\theta}
\newcommand{\kap}{\kappa}
\newcommand{\lam}{\lambda}
\newcommand{\sig}{\sigma}
\newcommand{\ups}{\upsilon}
\newcommand{\omg}{\omega}

\newcommand{\Gam}{\Gamma}
\newcommand{\Del}{\Delta}
\newcommand{\Tht}{\Theta}
\newcommand{\Lam}{\Lambda}
\newcommand{\Sig}{\Sigma}
\newcommand{\Omg}{\Omega}


% Text

\newcommand{\where}{\text{where }}

% Problem 1

\newcommand{\Hint}{H_\text{int}}
\newcommand{\ddcx}{\dd[3]{x}}
\newcommand{\psib}{\bar{\psi}}

\newcommand{\mh}{m_h}
\newcommand{\mmu}{m_\mu}
\newcommand{\me}{m_e}
\newcommand{\ma}{m_a}

\newcommand{\aexpt}{a_\text{expt.}}
\newcommand{\aQED}{a_\text{QED}}
\renewcommand{\GeV}{\giga\electronvolt}

\newcommand{\gamt}{\gam^5}


\state{Non-equilibrium entropies of Fermi, Bose, and Boltzmann distributions}{
	Consider a gas out of equilibrium with a slightly non-uniform density in $n(x)$ and mean density $\nb = V^{-1} \int n(x) \ddcx$.  We know that if the gas obeys Boltzmann statistics, its entropy is $S = -\int n \log n \ddV$.
}

%
%	1.1
%

\prob{}{
	Argue that this formula is valid only if the gradients are small: $\abs{\gradx n} \ll \nb^{4/3}$ (``coarse-graining condition'') and that $\abs{n(x) - \nb} \ll \nb$.
}

%
%	1.2
%

\prob{}{
	Remove the second condition in {1.1} and obtain the general formula for the entropy for both Fermi and Bose gases.
}



\state{Quantum correction to the Boltzmann thermodynamics}{
	Find the the quantum correction to the free energy of the Boltzmann gas (the leading $\hbar$-dependent term in the expansion of the free energy at small $\hbar$) for Bose and Fermi gases.  From there, find the correction to the pressure.  Does the quantum correction increase or decrease the pressure (and why is the answer predictable)?
}



\clearpage
\newcommand{\tht}{\theta}
\newcommand{\thtxx}{\tht_{xx}}
\newcommand{\thtyy}{\tht_{yy}}
\newcommand{\thttt}{\tht_{tt}}
\newcommand{\thtt}{\tht_t}
\newcommand{\thtx}{\tht_x}
\newcommand{\thty}{\tht_y}
\newcommand{\sint}{\sin{\tht}}
\newcommand{\dxdydt}{\dxdy \dd{t}}

\begin{statement}{}
	The nondimensionalized, multidimensional Sine-Gordon equation,
	\beq
		\thtxx + \thtyy - \thttt = \sint
	\eeq
	for $\tht(x, y, t)$, is the Euler-Lagrange equation for the action integral
	\beq
		S[\tht] = \intR \left\{ \frac{1}{2} \left[ \thtt^2 - (\nabla\tht)^2 \right] - \sint \right\} \dxdydt
	\eeq
	with $\nabla\tht = (\pdv*{\tht}{x}, \pdv*{\tht}{y})$.  The functional $S[\tht]$ is invariant under translation of $x$, $y$, and $t$.  Find the associated energy-momentum tensor and energy-momentum vector.
\end{statement}

\begin{solution}
	Expanding out $(\nabla\tht)^2$, the Lagrangian density is
	\beqn \label{lagr3}
		\Ld = \frac{1}{2} (\thtt^2 - \thtx^2 - \thty^2) - \sint.
	\eeqn
	The energy-momentum tensor is defined by
	\beq
		T_{ij} = \pdv{\Ld}{\tht_{x_i}} \pdv{\tht}{x_j} - \Ld \, \delta_{ij},
	\eeq
	where $x_i \in \{ x_0, x_1, x_2 \} = \{ t, x, y \}$.  The diagonal elements of $T$ are then
	\begin{align*}
		T_{00} &= \pdv{\Ld}{\thtt} \pdv{\tht}{t} - \Ld = \thtt^2 - \frac{1}{2} (\thtt^2 - \thtx^2 - \thty^2) + \sint = \frac{1}{2} (\thtt^2 + \thtx^2 + \thty^2) + \sint, \\
		T_{11} &= \pdv{\Ld}{\thtx} \pdv{\tht}{x} - \Ld = -\thtx^2 - \frac{1}{2} (\thtt^2 - \thtx^2 - \thty^2) + \sint = -\frac{1}{2} (\thtt^2 + \thtx^2 - \thty^2) + \sint, \\
		T_{22} &= \pdv{\Ld}{\thty} \pdv{\tht}{y} - \Ld = -\thty^2 - \frac{1}{2} (\thtt^2 - \thtx^2 - \thty^2) + \sint = -\frac{1}{2} (\thtt^2 - \thtx^2 + \thty^2) + \sint,
	\end{align*}
	and the nondiagonal elements are
	\begin{align*}
		T_{01} &= \pdv{\Ld}{\thtt} \pdv{\tht}{x} = \thtt \thtx, &
		T_{02} &= \pdv{\Ld}{\thtt} \pdv{\tht}{y} = \thtt \thty, &
		T_{12} &= \pdv{\Ld}{\thtx} \pdv{\tht}{y} = -\thtx \thty, \\
		T_{10} &= \pdv{\Ld}{\thtx} \pdv{\tht}{t} = -\thtt \thtx, &
		T_{20} &= \pdv{\Ld}{\thty} \pdv{\tht}{t} = -\thtt \thty, &
		T_{21} &= \pdv{\Ld}{\thty} \pdv{\tht}{x} = -\thtx \thty.
	\end{align*}
	In matrix form, we have
	\beq
		T = \mqty[(\thtt^2 + \thtx^2 + \thty^2) / 2 + \sint & \thtt \thtx & \thtt \thty \\
				-\thtt \thtx & -(\thtt^2 + \thtx^2 - \thty^2) / 2 + \sint & -\thtx \thty \\
				-\thtt \thty & -\thtx \thty & -(\thtt^2 - \thtx^2 + \thty^2) / 2 + \sint ].
	\eeq
	The energy-momentum vector is defined by
	\beq
		P_j = \int T_{0j} \dd{x_1} \dd{x_2}.
	\eeq
	Its components are then
	\begin{align*}
		P_0 &= \int \left[ \frac{1}{2} (\thtt^2 + \thtx^2 + \thty^2) + \sint \right] \dxdy, &
		P_1 &= \int \thtt \thtx \dxdy, &
		P_2 &= \int \thtt \thty \dxdy.
	\end{align*}
\vfix
\end{solution}



\state{Degenerate Bose gas}{\hfix}

%
%	4.1
%

\prob{}{
	The chemical potential of the degenerate Bose gas vanishes below $\Ts$ (the critical temperature of the BEC).  Find its temperature dependence at temperatures slightly above $\Ts$.
}

\sol{
	In three dimensions, the energy distribution of a Bose gas is~\cite[p.~149]{Landau}
	\eq{
		\ddNeps = \frac{g V}{\pi^2 \hbar^3} \sqrt{\frac{m^3}{2}} \frac{\sqrt{\eps}}{e^{(\eps - \mu) / T} - 1} \ddeps.
	}
	Integrating over all energies, we find the total number of molecules~\cite[p.~149]{Landau}.  This gives an expression relating the chemical potential $\mu$ and the density $\nb$~\cite[p.~159]{Landau}:
	\eqn{nb}{
		\nb = \frac{g}{\pi^2 \hbar^3} \sqrt{\frac{m^3}{2}} \intoi \frac{\sqrt{\eps}}{e^{(\eps - \mu) / T} - 1} \ddeps.
	}
	The critical temperature $\Ts$ satisfies this relation for $\mu = 0$, and can be found by making the substitution $z = \eps \Ts$:
	\eq{
		\nb = \frac{g}{\pi^2 \hbar^3} \sqrt{\frac{m^3}{2}} \intoi \frac{\sqrt{\eps}}{e^{\eps / \Ts} - 1} \ddeps
		= \frac{g}{\pi^2 \hbar^3} \sqrt{\frac{m^3 {\Ts}^3}{2}} \intoi \frac{\sqrt{z}}{e^z - 1} \ddeps.
	}
	The integral may be evaluated using the formula~\cite[p.~156]{Landau}
	\eqn{formula}{
		\intoi \frac{z^{x - 1}}{e^z - 1} \ddz = \Gam(x) \zeta(x),
	}
	with $x > 1$.  The relevant values are $\Gam(3/2) = \sqrt{\pi} / 2$, and $\zeta(3/2) = 2.612$~\cite[p.~156]{Landau}.  Thus,
	\eq{
		\nb = \frac{g}{\pi^2 \hbar^3} \sqrt{\frac{m^3 {\Ts}^3}{2}} (2.612)\frac{\sqrt{\pi}}{2}
		= \frac{g}{\pi^2 \hbar^3} \sqrt{\frac{m^3 {\Ts}^3}{2}} (2.612)\frac{\sqrt{\pi}}{2}
		= \frac{0.9235 \,g}{\hbar^3} \paren{ \frac{m \Ts}{\pi} }^{3/2}.
	}
	
	Define the function
	\eq{
		\nbs(T) = \frac{g}{\pi^2 \hbar^3} \sqrt{\frac{m^3}{2}} \intoi \frac{\sqrt{\eps}}{e^{\eps / T} - 1} \ddeps
		= \frac{0.9235 \,g}{\hbar^3} \paren{ \frac{m T}{\pi} }^{3/2},
	}
	and note that $\nbs(\Ts) = \nb$.  Then we can rewrite Eq.~\refeq{nb} as
	\eq{
		\nb = \nbs(T) + \frac{g}{\pi^2 \hbar^3} \sqrt{\frac{m^3}{2}} \intoi \frac{\sqrt{\eps}}{e^{(\eps - \mu) / T} - 1} \ddeps - \nbs(T)
		= \nbs(T) + \frac{g}{\pi^2 \hbar^3} \sqrt{\frac{m^3}{2}} \intoi \paren{ \frac{\sqrt{\eps}}{e^{(\eps - \mu) / T} - 1} - \frac{\sqrt{\eps}}{e^{\eps / T} - 1} } \ddeps.
	}
	Expanding the integrand for small exponential powers using $e^x \approx 1 + x$, we find
	\eq{
		\frac{\sqrt{\eps}}{e^{(\eps - \mu) / T} - 1} - \frac{\sqrt{\eps}}{e^{\eps / T} - 1}
		\approx \frac{\sqrt{\eps}}{1 + (\eps - \mu) / T - 1} - \frac{\sqrt{\eps}}{1 + \eps / T - 1}
		= \frac{T \sqrt{\eps}}{\eps - \mu} - \frac{T}{\sqrt{\eps}}
		= \frac{T \eps - T (\eps - \mu)}{\sqrt{\eps} (\eps - \mu)}
		= \frac{T \mu}{\sqrt{\eps} (\eps - \mu)}.
	}
	Then the integral is
	\eq{
		T \mu \intoi \frac{\ddeps}{\sqrt{\eps} (\eps - \mu)} = T \mu \frac{\pi}{\sqrt{-\mu}}
		= \pi T \sqrt{-\mu},
	}
	so long as $\mu < 0$, which is true for the Bose distribution~\cite[p.~145]{Landau}.  Making this substitution and solving for $\mu$, we find
	\eq{
		\nb = \nbs(T) - \frac{g T}{\pi \hbar^3} \sqrt{\frac{-\mu m^3}{2}}
%		\qimplies
%		\sqrt{\frac{\abs{\mu} m^3}{2}} = \frac{\pi \hbar^3 (\nbs - \nb)}{g T}
		\qimplies
		\mu = -\frac{2}{m^3} \paren{ \frac{\pi \hbar^3 [ \nbs(T) - \nb ]}{g T} }^2
		= -\frac{2 \pi^2 \hbar^6 [ \nbs(T) - \nb ]^2}{m^3 g^2 T^2}.
	}
	Note that
	\eq{
		\nbs(T) - \nb = \nb \paren{ \frac{\nbs(T)}{\nb} - 1 }
		= \nb \paren{ \frac{\nbs(T)}{\nbs(\Ts)} - 1 }
		= \nb \paren{ \frac{T^{3/2}}{{\Ts}^{3/2}} - 1 },
	}
	since $\nbs(\Ts) = \nb$.  Then the relationship between chemical potential and temperature is
	\eq{
		\mu = -\frac{2 \pi^2 \hbar^6 \nb^2}{m^3 g^2 T^2} \paren{ \frac{T^{3/2}}{{\Ts}^{3/2}} - 1 }^2
		= \ans{ -\frac{2 \pi^2 \hbar^6 \nb^2}{m^3 g^2} \paren{ \frac{T^{1/2}}{{\Ts}^{3/2}} - \frac{1}{T} }^2. }
	}
	Since $T / \Ts \approx 1$, the leading behavior is \ans{$\mu \sim -1 / T^2$.}
}

%
%	4.2
%

\prob{}{
	Find the discontinuities in the derivatives of thermodynamic quantities at the BEC transition.  Which order is this phase transition?
}

%
%	4.3
%

\prob{(*)}{
	Can the ideal Bose gas condense in spatial dimensions 1 and 2?  Discuss what happens in these cases. 
}



\clearpage
\newcommand{\phiq}{\phi_1}
\newcommand{\phiw}{\phi_2}
\newcommand{\rhoq}{\rho_1}
\newcommand{\rhow}{\rho_2}
\newcommand{\intS}{\int_S}
\newcommand{\dS}{\dd{S}}
\newcommand{\vv}{\vec{v}}
\newcommand{\phixp}{\phi(\vx')}

\begin{statement}{}
	The ``mean value theorem'' is stated as follows: For any solution $\phi$ to $\lap \phi = 0$, the value of $\phi$ at $\vx$ is equal to the average value of $\phi$ on a sphere of radius $R$ (for any $R$) centered at $\vx$.
\end{statement}

\begin{problem} \label{5a}
	Prove the mean value theorem.  Hint: Apply Green's theorem to $\phi$ and $1/\abs{\vx - \vx'}$ for a suitable choice of region and a suitable choice of $\vx'$.
\end{problem}

\begin{solution}
	Green's theorem is given by Eq.~(2.96),
	\beq
		\intS \nh \cdot (\phiq \nabla\phiw - \phiw \nabla\phiq) \dS = -4\pi \intV (\phiq \rhow - \phiw \rhoq) \dcx.
	\eeq
	We will choose our volume as a sphere centered at $\vx$ with radius $r$, so $\nh = \rh$.  Suppose $\phiq = \phix$ is a solution to Laplace's equation as stated.  Let $\vx'$ point radially from $\vx$, located at the center of the sphere, to a point a distance $r$ away; that is, $\vx' = \vx + r \, \rh$.  Then
	\beq
		\phiw = \frac{1}{\abs{\vx - \vx'}} = \frac{1}{\abs{-r \, \rh}} = \frac{1}{r}.
	\eeq
	From Poisson's equation, $\lap\phi = -4\pi\rho$ in general.  This means $\rhoq = 0$.  For the Green's function, $\rhow = \delta(\vx - \vx') = \delta(r)$.
	
	Applying Green's theorem,
	\beqn \label{gt}
		\intS \rh \cdot \left( \phix \nabla\frac{1}{r} - \frac{1}{r} \nabla\phix \right) \dS = -4\pi \intV \phi \, \delta(r) \dcx
	\eeqn
	For the first term on the left side, note that
	\beq
		\rh \cdot \nabla\frac{1}{r} = \pdv{}{r} \frac{1}{r} \rh \cdot \rh = -\frac{1}{r^2}.
	\eeq
	Gauss's theorem is given by Eq.~(2.6),
	\beq
		\intV \nabla \cdot \vv \dcx = \intS \vv \cdot \nh \dS.
	\eeq
	Applying this to the second term on the left side of \refeq{gt},
	\beq
		-\intS \nh \cdot \frac{1}{r} \nabla\phix \dS = -\intV \nabla \cdot \frac{1}{r} \nabla\phix \dcx
		= -\frac{1}{r} \lap\phix
		= 0.
	\eeq
	For the right side of \refeq{gt},
	\beq
		-4\pi \intV \phix \, \delta(r) \dcx = -4\pi \phi(0).
	\eeq
	
	Putting this together, \refeq{gt} becomes
	\beq
		-\intS \frac{\phix}{r^2} \dS = -4\pi \phi(0) \dS.
	\eeq
	We can choose $\vx = 0$ without loss of generality and switch $\vx$ with $\vx'$, which gives us
	\beqn \label{mvt}
		\phix = \frac{1}{4\pi r^2} \intS \phixp \dd{S'}.
	\eeqn
	This equation demonstrates that the value of $\phi$ at $\vx$ is equal to its average value on a sphere of arbitrary radius $r$.  Thus, we have proven the mean value theorem. \qed
\end{solution}


\begin{problem}
	Use this result to show that a point charge can never be in stable equilibrium if placed in an electric field $\vE$ that is source free in a neighborhood of the charge.
\end{problem}

\begin{solution}
	Let $\cV$ denote the neighborhood of the point charge, which can be described as a sphere of radius $r$ centerd at the location of the point charge.  We will choose this point as the origin.  Let $S$ denote the boundary of $\cV$.
	
	Suppose, contrary to the problem statement, that the point charge is in stable equilibrium.  This means that the electrostatic potential $\phi$ has a local minimum at the origin, and so $\phi(0) < \phix$ for all other $\vx \neq 0$ within $\cV$.  In particular, $\phi(0) < \phi|_S$ at all points on the boundary, and so
	\beqn \label{fake}
		\phi(0) < \frac{1}{4\pi r^2} \intS \phix \dS.
	\eeqn
	However, $\phi$ must satisfy $\lap\phi = 0$, since $\cV$ is source free.  As proven in \ref{5a}, $\phi$ therefore obeys \refeq{mvt}, which contradicts \refeq{fake} and therefore our assumption that the point charge is in stable equilibrium.  So we have shown that stable equilibrium is impossible in this situation. \qed
\end{solution}



\state{Thermodynamics of solids}{
	Compute the following thermodynamic quantities for the harmonic photonic modes in a 1D and a 2D crystal at low temperatures (a.k.a.~phonons) and compare with the textbook example of a 3D crystal.
}

%
%	6.1
%

\prob{}{
	Free energy.
}

\sol{
	A crystal of $N$ molecules is comprised of quantum harmonic oscillators that are free to oscillate in all spatial dimensions.  We can count the number of states in the interval $\ddk$, where $k$ is the wave number.  For a crystal, it is related to the frequency of vibration by $k = d \omg / \ub$, where $\ub$ is the averaged velocity of sound for the particular crystal structure and $d$ the number of spatial dimensions.  The number of states in the interval is, for each case,
	\al{
		\frac{L}{2\pi} \ddk
		&= \frac{L}{2\pi \ub} \ddomg
		\quad (d = 1), &
		\frac{2 \pi A}{(2\pi)^2} k \ddk
		&= \frac{A}{\pi \ub^2} \omg \ddomg,
	}
	where we have taken into account that there are $d$ independent polarization directions in each case~\cite[pp.172--173]{Landau}.
	
	The free energy is $F = N \epso - T \ln Z$, where $\epso$ is the energy per molecule when the system is at equilibrium, which depends on $N$ and the volume $V$~\cite[pp.~87, 172]{Landau}.  The single-particle vibrational partition function is~\cite[p.~136]{Landau}
	\eq{
		Z_1 = \frac{1}{1 - e^{-\hbar \omg / T}}.
	}
	The entire crystal can be modeled as $d \,N \nu$ independent oscillators with total free energy~\cite[p.~172]{Landau}
	\eq{
		F = N \epso - T \sum_{\alp = 1}^{d \,N \nu} \ln(1 - e^{-\hbar \omg_\alp / T}).
	}
	For the entire crystal, the sum can be transformed to an integral over $\omg \in (0, \infty)$~\cite[p.~173]{Landau}.  Referring to the similar integrals in Prob.~{5.2}, we have
	\al{
		(d = 1) \quad
		F &= N \epso - \frac{L T}{2\pi \ub} \int \ln(1 - e^{-\hbar \omg / T}) \ddomg
		= N \epso - \frac{L T^2}{2\pi \hbar \ub} \frac{\pi}{6}
		= \ans{ N \epso - \frac{L T^2}{12 \hbar \ub}, } \\[2ex]
		(d = 2) \quad
		F &= N \epso - \frac{A T}{\pi \ub^2} \int \omg \ln(1 - e^{-\hbar \omg / T}) \ddomg
		= N \epso - \frac{A T^3}{\pi \hbar^2 \ub^2} \Gam(3) \zeta(3)
		= \ans{ N \epso - \frac{1.202 \,A T^3}{\pi \hbar^2 \ub^2}. }
	}
	The 3D expression is~\cite[p.~173]{Landau}
	\eq{
		F = N \epso - \frac{\pi^2 V T^3}{30 \hbar^3 \ub^3},
	}
	suggesting
	\eq{
		\ans{ F = N \epso - j(d) \,\frac{L^d T^{d + 1}}{\hbar^d \ub^d}, }
	}
	where $j(d)$ is a constant that depends on the number of dimensions, and we note that both $\epso$ and $\ub$ depend on the crystal structure and therefore $d$.
}

%
%	6.2
%

\prob{}{
	Entropy.
}

\sol{
	As in Prob.~{5.3}, $S = -\pdv{F}{T}$:
	\al{
		S &= -\pdv{T}(N \epso - \frac{L T^2}{12 \hbar \ub})
		= \ans{ \frac{L T}{6 \hbar \ub} }
		\quad (d = 1), &
		S &= -\pdv{T}(N \epso - \frac{1.202 \,A T^3}{\pi \hbar^2 \ub^2})
		= \ans{ \frac{3.606 \,A T^2}{\pi \hbar^2 \ub^2} }
		\quad (d = 2).
	}
	In 3D, the entropy is~\cite[p.~173]{Landau}
	\eq{
		S = \frac{2 \pi^2 V T^3}{15 \hbar^3 \ub^3},
	}
	which suggests
	\eq{
		\ans{ S \propto \frac{L^d T^d}{\hbar^d \ub^d}. }
	}
	\vfix
}

%
%	6.3
%

\prob{}{
	Energy.
}

\sol{
	As in Prob.~{5.3}, $E = F + T S$:
	\al{
		(d = 1) \quad
		E &= N \epso - \frac{L T^2}{12 \hbar \ub} + T \frac{L T}{6 \hbar \ub}
		= \ans{ N \epso + \frac{L T^2}{12 \hbar \ub}, } \\[2ex]
		(d = 2) \quad
		E &= N \epso - \frac{1.202 \,A T^3}{\pi \hbar^2 \ub^2} + T \frac{3.606 \,A T^2}{\pi \hbar^2 \ub^2}
		= \ans{ N \epso + \frac{2.404 \,A T^3}{\pi \hbar^2 \ub^2}. }
	}
	The 3D equivalent is~\cite[p.~173]{Landau}
	\eq{
		E = N \epso + \frac{\pi^2 T^4}{10 \hbar^3 \ub^3},
	}
	which suggests
	\eq{
		\ans{ E = N \epso + j(d) \frac{d T^{d + 1}}{\hbar^d \ub^d}
		= N - d \,F, }
	}
	where $j(d)$ is a constant that depends on the number of dimensions, and is not necessarily the same as that in Prob.~{6.1}.
}

%
%	6.4
%

\prob{}{
	Specific heat.
}

\sol{
	As in Prob.~{5.4}, $\Cv = (\pdv*{E}{T})_V$.  Then
	\al{
		C &= \pdv{T}(N \epso + \frac{L T^2}{12 \hbar \ub})
		= \ans{ \frac{L T}{6 \hbar \ub} }
		\quad (d = 1), &
		C &= \pdv{T}(N \epso + \frac{2.404 \,A T^3}{\pi \hbar^2 \ub^2})
		= \ans{ \frac{7.212 \,A T^2}{\pi \hbar^2 \ub^2} }
		\quad (d = 2).
	}
	In 3D~\cite[p.~173]{Landau}
	\eq{
		C = \frac{2 \pi^2 V T^3}{5 \hbar^3 \ub^3},
	}
	which suggests
	\eq{
		\ans{ C \propto \frac{L^d T^d}{\hbar^d \ub^d}. }
	}
	\vfix
}


\makebib

\end{document}