\documentclass[11pt]{article}
\usepackage{homework}

\classname{352}
\homeworknum{3}



\begin{document}

% Environments

\newcommand{\state}[2]{\begin{statement}{#1} #2 \end{statement}}
\newcommand{\prob}[2]{\begin{problem}{#1} #2 \end{problem}}
\newcommand{\subprob}[1]{\begin{subproblem} #1 \end{subproblem}}
\newcommand{\sol}[1]{\begin{solution} #1 \end{solution}}
\newcommand{\fig}[2]{\begin{figure} \centering #2  \label{#1} \end{figure}}

\newcommand{\makebib}{
	\vfill
	\color{black}
	\bibliography{references}{}
	\bibliographystyle{lucas_unsrt}
}
	

% Implication

\newcommand{\qwhere}{\quad \text{where} \quad}
\newcommand{\qimplies}{\quad \implies \quad}
\newcommand{\impliesq}{\implies \quad}



% Brackets

\newcommand{\paren}[1]{\left( #1 \right)}
\newcommand{\brac}[1]{\left[ #1 \right]}


% Greek

\newcommand{\alp}{\alpha}
\newcommand{\bet}{\beta}
\newcommand{\gam}{\gamma}
\newcommand{\del}{\delta}
\newcommand{\eps}{\epsilon}
\newcommand{\zet}{\zeta}
\newcommand{\tht}{\theta}
\newcommand{\kap}{\kappa}
\newcommand{\lam}{\lambda}
\newcommand{\sig}{\sigma}
\newcommand{\ups}{\upsilon}
\newcommand{\omg}{\omega}

\newcommand{\Gam}{\Gamma}
\newcommand{\Del}{\Delta}
\newcommand{\Tht}{\Theta}
\newcommand{\Lam}{\Lambda}
\newcommand{\Sig}{\Sigma}
\newcommand{\Omg}{\Omega}
% Problem 1

\newcommand{\Psii}{\Psi^i}
\newcommand{\Psiix}{\Psii(x)}

\newcommand{\Pii}{\Pi^i}

\newcommand{\Phii}{\Phi^i}
\newcommand{\Phiix}{\Phii(x)}
\newcommand{\PhiN}{\Phi^N}
\newcommand{\PhiNx}{\PhiN(x)}
\newcommand{\Phiq}{\Phi^1}
\newcommand{\Phiw}{\Phi^2}

\newcommand{\ddcx}{\dd[3]{x}}

\newcommand{\delij}{\del^{i j}}
\newcommand{\delkl}{\del^{k l}}
\newcommand{\delil}{\del^{i l}}
\newcommand{\deljk}{\del^{j k}}
\newcommand{\delik}{\del^{i k}}
\newcommand{\deljl}{\del^{j l}}

\newcommand{\DF}{D_F}

\newcommand{\sigx}{\sig(x)}

\newcommand{\pii}{\pi^i}
\newcommand{\pij}{\pi^j}
\newcommand{\pik}{\pi^k}
\newcommand{\pil}{\pi^l}
\newcommand{\piix}{\pi(x)}

\newcommand{\pq}{p_1}
\newcommand{\pw}{p_2}
\newcommand{\pe}{p_3}
\newcommand{\pr}{p_4}

\newcommand{\vp}{\vb{p}}
\newcommand{\vpsi}{\vp_i}

\newcommand{\mpi}{m_\pi}


\state{Non-equilibrium entropies of Fermi, Bose, and Boltzmann distributions}{
	Consider a gas out of equilibrium with a slightly non-uniform density in $n(x)$ and mean density $\nb = V^{-1} \int n(x) \ddcx$.  We know that if the gas obeys Boltzmann statistics, its entropy is $S = -\int n \log n \ddV$.
}

%
%	1.1
%

\prob{}{
	Argue that this formula is valid only if the gradients are small: $\abs{\gradx n} \ll \nb^{4/3}$ (``coarse-graining condition'') and that $\abs{n(x) - \nb} \ll \nb$.
}

%
%	1.2
%

\prob{}{
	Remove the second condition in {1.1} and obtain the general formula for the entropy for both Fermi and Bose gases.
}



\state{Quantum correction to the Boltzmann thermodynamics}{
	Find the the quantum correction to the free energy of the Boltzmann gas (the leading $\hbar$-dependent term in the expansion of the free energy at small $\hbar$) for Bose and Fermi gases.  From there, find the correction to the pressure.  Does the quantum correction increase or decrease the pressure (and why is the answer predictable)?
}



\clearpage
\newcommand{\kq}{\ket{1}}
\newcommand{\kw}{\ket{2}}
\newcommand{\ke}{\ket{3}}

\newcommand{\vq}{v_1}
\newcommand{\vw}{v_2}
\newcommand{\ve}{v_3}

\newcommand{\vqs}{\vq^*}
\newcommand{\vws}{\vw^*}

\newcommand{\Heff}{H_\text{eff}}
\newcommand{\Eo}{E\suo}
\newcommand{\Eod}{\Eo_D}

\newcommand{\Pq}{P_1}

%\clearpage
\begin{statement}{}
	Consider the Hamiltonian $\Ho$ acting on a three-dimensional Hilbert space spanned by the orthonormal basis $\{\kq, \kw, \ke\}$.  $\Ho = \sum_{i = 3}^3 E_i \ketbra{i}$, with energy eigenvalues $\Eoq, \Eow, \Eoe$.  Assume $\Eoq = \Eow = \Eod$.  To $\Ho$, we add a perturbation
	\beq
		V = \vq \ketbra{1}{3} + \vqs \ketbra{3}{1} + \vw \ketbra{2}{3} + \vws \ketbra{3}{2}.
	\eeq
	Here, $\vq$ and $\vw$ are complex constants and small compared to $\Ee$.
\end{statement}

\begin{problem}
	To second order in $V$, write down the explicit form of the effective Hamiltonian acting on the subspace spanned by $\{\kq, \kw\}$.
\end{problem}

\begin{solution}
	We have
	\begin{align*}
		\Ho &= \mqty[ \Eod & 0 & 0 \\ 0 & \Eod & 0 \\ 0 & 0 & \Eoe ], &
		V &= \mqty[ 0 & 0 & \vq \\ 0 & 0 & \vw \\ \vqs & \vws & 0 ], &
		H &= \Ho + \lam V = \mqty[ \Eod & 0 & \lam \vq \\ 0 & \Eod & \lam \vw \\ \vqs & \vws & \Eoe].
	\end{align*}
	From the lecture notes and (5.2.12) in Sakurai, the effective Hamiltonian is given to second order in $\lam$ by
	\beq
		\Heff = \Eod + \lam \Po V \Po + \lam^2 \Po V \Pq (\Eod - \Ho)^{-1} \Pq V \Po,
	\eeq
	where $\Po$ is the projection onto the degenerate subspace, $\Pq$ is the projection onto the nondegenerate subspace, and $\Eod$ is the degenerate energy.  Here, $\Po$ projects onto the subspace spanned by $\{ \kq, \kw \}$ and $\Pq$ onto that spanned by $\{ \ke \}$.
	
	Note that
	\beq
		Po V \Po = \mqty[ 1 & 0 & 0 \\ 0 & 1 & 0 \\ 0 & 0 & 0 ] \mqty[ 0 & 0 & \vq \\ 0 & 0 & \vw \\ \vqs & \vws & 0 ] \mqty[ 1 & 0 & 0 \\ 0 & 1 & 0 \\ 0 & 0 & 0 ]
		= \mqty[ 0 & 0 & 0 \\ 0 & 0 & 0 \\ 0 & 0 & 0 ],
	\eeq
	and
	\begin{align*}
		\Po V \Pq (\Eod - \Ho)^{-1} &\Pq V \Po = \frac{1}{\Eod - \Eoe} \Po \mqty[ 0 & 0 & \vq \\ 0 & 0 & \vw \\ \vqs & \vws & 0 ] \mqty[ 0 & 0 & 0 \\ 0 & 0 & 0 \\ 0 & 0 & 1] \mqty[ 0 & 0 & \vq \\ 0 & 0 & \vw \\ \vqs & \vws & 0 ] \Po \\
		&= \frac{1}{\Eod - \Eoe} \mqty[ 1 & 0 & 0 \\ 0 & 1 & 0 \\ 0 & 0 & 0 ] \mqty[|\vq|^2 & \vq \vws & 0 \\ \vqs \vw & |\vq|^2 & 0 \\ 0 & 0 & 0] \mqty[ 1 & 0 & 0 \\ 0 & 1 & 0 \\ 0 & 0 & 0 ]
		= \frac{1}{\Eod - \Eoe} \mqty[|\vq|^2 & \vq \vws & 0 \\ \vqs \vw & |\vq|^2 & 0 \\ 0 & 0 & 0].
	\end{align*}
	
	In the degenerate subspace, we have
	\begin{align*}
		\Heff = \mqty[ \Eod + |\vq|^2/(\Eod - \Eoe) & \vq \vws / (\Eod - \Eoe) \\ \vqs \vw / (\Eod - \Eoe) & \Eod + |\vw|^2/(\Eod - \Eoe)].
	\end{align*}
\end{solution}


%\clearpage
\newcommand{\uq}{u_1}
\newcommand{\uw}{u_2}
\newcommand{\wq}{w_1}
\newcommand{\ww}{w_2}

\begin{problem}
	By solving the effective Hamiltonian, construct the approximate solution for the eigenvalues and eigenfunctions of $\Ho + V$.  (The eigenkets only need to be constructed within the degenerate subspace.)
\end{problem}

\begin{solution}
	Let $E$ be the eigenvalues of $\Heff$.  We need to solve the characteristic equation
	\begin{align*}
		0 &= \det(\Heff - E I)
		= \vmqty{ \Eod + |\vq|^2/(\Eod - \Eoe) - E & \vq \vws / (\Eod - \Eoe) \\ \vqs \vw / (\Eod - \Eoe) & \Eod + |\vw|^2/(\Eod - \Eoe) - E } \\
		&= \left( \Eod + \frac{|\vq|^2}{\Eod - \Eoe} - E \right) \left( \Eod + \frac{|\vw|^2}{\Eod - \Eoe} - E \right) - \frac{|\vq|^2 |\vw|^2}{(\Eod - \Eoe)^2} \\
		&= {\Eod}^2 + \Eod \frac{|\vw|^2}{\Eod - \Eoe} - \Eod E + \Eod \frac{|\vq|^2}{\Eod - \Eoe} - E \frac{|\vq|^2}{\Eod - \Eoe} - \Eod E - E \frac{|\vw|^2}{\Eod - \Eoe} + E^2 \\
		&= E^2 - \Eod E - E \frac{|\vq|^2 + |\vw|^2}{\Eod - \Eoe} - \Eod E + {\Eod}^2 + \Eod \frac{|\vq|^2 + |\vw|^2}{\Eod - \Eoe} \\
		&= (E - \Eod) \left( E - \Eod - \frac{|\vq|^2 + |\vw|^2}{(\Eod - \Eoe)^2} \right),
	\end{align*}
	so the eigenvalues are
	\begin{align*}
		\Eq &= \Eod, &
		\Ew &= \Eod + \frac{|\vq|^2 + |\vw|^2}{(\Eod - \Eoe)^2}.
	\end{align*}
	
	The eigenvector corresponding to $\Eq$ can be found by
	\beq
		\mqty[ \Eod + |\vq|^2 / (\Eod - \Eoe) & \vq \vws / (\Eod - \Eoe) \\ \vqs \vw / (\Eod - \Eoe) & \Eod + |\vw|^2 / (\Eod - \Eoe) ] \mqty[ \uq \\ \uw ] = \Eod \mqty[ \uq \\ \uw ]
	\eeq
	which is equivalent to the system of equations
	\begin{align*}
		\left( \Eod + \frac{|\vq|^2}{\Eod - \Eoe} \right) \uq + \frac{\vq \vws}{\Eod - \Eoe} \uw &= \Eod \uq, &
		\frac{\vqs \vw}{\Eod - \Eoe} \uq + \left( \Eod + \frac{|\vw|^2}{\Eod - \Eoe} \right) \uw &= \Eod \uw.
	\end{align*}
	By inspection, these are satisfied when $\uq = -\vws$ and $\uw = \vqs$.
	
	For the eigenvector corresponding to $\Ew$, we have
	\beq
		\mqty[ \Eod + |\vq|^2 / (\Eod - \Eoe) & \vq \vws / (\Eod - \Eoe) \\ \vqs \vw / (\Eod - \Eoe) & \Eod + |\vw|^2 / (\Eod - \Eoe) ] \mqty[ \wq \\ \ww ] = \left( \Eod + \frac{|\vq|^2 + |\vw|^2}{\Eod - \Eoe} \right)\mqty[ \wq \\ \ww ]
	\eeq
	which is equivalent to the system of equations
	\begin{align*}
		\left( \Eod + \frac{|\vq|^2}{\Eod - \Eoe} \right) \wq + \frac{\vq \vws}{\Eod - \Eoe} \ww &= \left( \Eod + \frac{|\vq|^2 + |\vw|^2}{\Eod - \Eoe} \right) \wq, \\
		\frac{\vqs \vw}{\Eod - \Eoe} \wq + \left( \Eod + \frac{|\vw|^2}{\Eod - \Eoe} \right) \ww &= \left( \Eod + \frac{|\vq|^2 + |\vw|^2}{\Eod - \Eoe} \right) \ww.
	\end{align*}
	By inspection, these are satisfied when $\wq = \vq$ and $\ww = \ww$.  So we have the eigenvectors
	\begin{align*}
		\ket*{\Eq} &= \mqty[ -\vws \\ \vqs ], &
		\ket*{\Ew} &= \mqty[ \vq \\ \vw ].
	\end{align*}



\state{Degenerate Bose gas}{\hfix}

%
%	4.1
%

\prob{}{
	The chemical potential of the degenerate Bose gas vanishes below $\Ts$ (the critical temperature of the BEC).  Find its temperature dependence at temperatures slightly above $\Ts$.
}

\sol{
	In three dimensions, the energy distribution of a Bose gas is~\cite[p.~149]{Landau}
	\eq{
		\ddNeps = \frac{g V}{\pi^2 \hbar^3} \sqrt{\frac{m^3}{2}} \frac{\sqrt{\eps}}{e^{(\eps - \mu) / T} - 1} \ddeps.
	}
	Integrating over all energies, we find the total number of molecules~\cite[p.~149]{Landau}.  This gives an expression relating the chemical potential $\mu$ and the density $\nb$~\cite[p.~159]{Landau}:
	\eqn{nb}{
		\nb = \frac{g}{\pi^2 \hbar^3} \sqrt{\frac{m^3}{2}} \intoi \frac{\sqrt{\eps}}{e^{(\eps - \mu) / T} - 1} \ddeps.
	}
	The critical temperature $\Ts$ satisfies this relation for $\mu = 0$, and can be found by making the substitution $z = \eps \Ts$:
	\eq{
		\nb = \frac{g}{\pi^2 \hbar^3} \sqrt{\frac{m^3}{2}} \intoi \frac{\sqrt{\eps}}{e^{\eps / \Ts} - 1} \ddeps
		= \frac{g}{\pi^2 \hbar^3} \sqrt{\frac{m^3 {\Ts}^3}{2}} \intoi \frac{\sqrt{z}}{e^z - 1} \ddeps.
	}
	The integral may be evaluated using the formula~\cite[p.~156]{Landau}
	\eqn{formula}{
		\intoi \frac{z^{x - 1}}{e^z - 1} \ddz = \Gam(x) \zeta(x),
	}
	with $x > 1$.  The relevant values are $\Gam(3/2) = \sqrt{\pi} / 2$, and $\zeta(3/2) = 2.612$~\cite[p.~156]{Landau}.  Thus,
	\eq{
		\nb = \frac{g}{\pi^2 \hbar^3} \sqrt{\frac{m^3 {\Ts}^3}{2}} (2.612)\frac{\sqrt{\pi}}{2}
		= \frac{g}{\pi^2 \hbar^3} \sqrt{\frac{m^3 {\Ts}^3}{2}} (2.612)\frac{\sqrt{\pi}}{2}
		= \frac{0.9235 \,g}{\hbar^3} \paren{ \frac{m \Ts}{\pi} }^{3/2}.
	}
	
	Define the function
	\eq{
		\nbs(T) = \frac{g}{\pi^2 \hbar^3} \sqrt{\frac{m^3}{2}} \intoi \frac{\sqrt{\eps}}{e^{\eps / T} - 1} \ddeps
		= \frac{0.9235 \,g}{\hbar^3} \paren{ \frac{m T}{\pi} }^{3/2},
	}
	and note that $\nbs(\Ts) = \nb$.  Then we can rewrite Eq.~\refeq{nb} as
	\eq{
		\nb = \nbs(T) + \frac{g}{\pi^2 \hbar^3} \sqrt{\frac{m^3}{2}} \intoi \frac{\sqrt{\eps}}{e^{(\eps - \mu) / T} - 1} \ddeps - \nbs(T)
		= \nbs(T) + \frac{g}{\pi^2 \hbar^3} \sqrt{\frac{m^3}{2}} \intoi \paren{ \frac{\sqrt{\eps}}{e^{(\eps - \mu) / T} - 1} - \frac{\sqrt{\eps}}{e^{\eps / T} - 1} } \ddeps.
	}
	Expanding the integrand for small exponential powers using $e^x \approx 1 + x$, we find
	\eq{
		\frac{\sqrt{\eps}}{e^{(\eps - \mu) / T} - 1} - \frac{\sqrt{\eps}}{e^{\eps / T} - 1}
		\approx \frac{\sqrt{\eps}}{1 + (\eps - \mu) / T - 1} - \frac{\sqrt{\eps}}{1 + \eps / T - 1}
		= \frac{T \sqrt{\eps}}{\eps - \mu} - \frac{T}{\sqrt{\eps}}
		= \frac{T \eps - T (\eps - \mu)}{\sqrt{\eps} (\eps - \mu)}
		= \frac{T \mu}{\sqrt{\eps} (\eps - \mu)}.
	}
	Then the integral is
	\eq{
		T \mu \intoi \frac{\ddeps}{\sqrt{\eps} (\eps - \mu)} = T \mu \frac{\pi}{\sqrt{-\mu}}
		= \pi T \sqrt{-\mu},
	}
	so long as $\mu < 0$, which is true for the Bose distribution~\cite[p.~145]{Landau}.  Making this substitution and solving for $\mu$, we find
	\eq{
		\nb = \nbs(T) - \frac{g T}{\pi \hbar^3} \sqrt{\frac{-\mu m^3}{2}}
%		\qimplies
%		\sqrt{\frac{\abs{\mu} m^3}{2}} = \frac{\pi \hbar^3 (\nbs - \nb)}{g T}
		\qimplies
		\mu = -\frac{2}{m^3} \paren{ \frac{\pi \hbar^3 [ \nbs(T) - \nb ]}{g T} }^2
		= -\frac{2 \pi^2 \hbar^6 [ \nbs(T) - \nb ]^2}{m^3 g^2 T^2}.
	}
	Note that
	\eq{
		\nbs(T) - \nb = \nb \paren{ \frac{\nbs(T)}{\nb} - 1 }
		= \nb \paren{ \frac{\nbs(T)}{\nbs(\Ts)} - 1 }
		= \nb \paren{ \frac{T^{3/2}}{{\Ts}^{3/2}} - 1 },
	}
	since $\nbs(\Ts) = \nb$.  Then the relationship between chemical potential and temperature is
	\eq{
		\mu = -\frac{2 \pi^2 \hbar^6 \nb^2}{m^3 g^2 T^2} \paren{ \frac{T^{3/2}}{{\Ts}^{3/2}} - 1 }^2
		= \ans{ -\frac{2 \pi^2 \hbar^6 \nb^2}{m^3 g^2} \paren{ \frac{T^{1/2}}{{\Ts}^{3/2}} - \frac{1}{T} }^2. }
	}
	Since $T / \Ts \approx 1$, the leading behavior is \ans{$\mu \sim -1 / T^2$.}
}

%
%	4.2
%

\prob{}{
	Find the discontinuities in the derivatives of thermodynamic quantities at the BEC transition.  Which order is this phase transition?
}

%
%	4.3
%

\prob{(*)}{
	Can the ideal Bose gas condense in spatial dimensions 1 and 2?  Discuss what happens in these cases. 
}



\clearpage
\state{Acoustic and optic phonons in the diatomic chain}{
	In the diatomic chain, we take the unit cell to be of length $a$, and take $\xA$ and $\xB$ to be the coordinates of the A and B atoms within the unit cell.  Hence, in the $n$th cell,
	\al{
		\rnA &= n a + \xA; &
		\rnB &= n a + \xB
	}
	\vfix
}

\prob{}{
	In the equations of motion Eq.~(2.30), look for solutions of the form
	\eqn{5a}{
		\unalp = \ealpq \exp( i [ q \rnalp - \omgq t ] ) + \ealpsq \exp( i [-q \rnalp + \omgq t] )
	}
	where $\alp = A$ or $B$, and $\ealp$ are complex numbers that give the amplitude and phase of the oscillation of the two atoms.
	
	Separating out the terms that have the same time dependence, show that (for equal masses, ${\mA = \mB = m}$)
	\al{
		m \omgsq \eAq &= \DAAq \eAq + \DABq \eBq, \\
		m \omgsq \eBq &= \DBAq \eAq + \DBBq \eBq,
	}
	where
	\al{
		\DAAq &= \DBBq = K + K', \\
		-\DABq &= K \exp( i q [ \rnB - \rnA ] ) + K' \exp( i q [ \rnmqB - \rnA ] ), \\
		-\DBAq &= K \exp( i q [ \rnA - \rnB ] ) + K' \exp( i q [ \rnpqA - \rnB ] ).
	}
	Check that $\DAB = \DsBA$.
}

\sol{
	Equation~(2.30) is
	\aln{ \label{thing5a}
		\mA \pdv[2]{\unA}{t} &= K (\unB - \unA) + K' (\unmqB - \unA), &
		\mB \pdv[2]{\unB}{t} &= K' (\unpqA - \unB) + K (\unA - \unB).
	}
	Note that
	\al{
		\pdv[2]{\unalp}{t} &= \pdv{t}\{ -i \omgq \ealpq \exp( i [ q \rnalp - \omgq t ] ) + i \omgq \ealpsq \exp( i [-q \rnalp + \omgq t] ) \} \\
		&= -\omgsq \{ \ealpq \exp( i [ q \rnalp - \omgq t ] ) + \ealpsq \exp( i [-q \rnalp + \omgq t] ) \} \\
		&= -\omgsq \unalp,
	}
	so the first of Eq.~\refeq{5a} can be written
	\al{
		[ K + K' - \mA \omgsq ] \unA &= K \unB + K' \unmqB, \\[1ex]
		&= K \eBq e^{i q \rnB} e^{-i \omgq t} + K \eBsq e^{-i q \rnB} e^{i \omgq t} + K' \eBq e^{i q \rnmqB} e^{-i \omgq t} \\
		&\hspace{5em} \phantom{=\ } + K' \eBsq e^{-i q \rnmqB} e^{i \omgq t} \\[1ex]
		&= K \eBq e^{i q [ na + \xB ]} e^{-i \omgq t} + K' \eBq e^{i q [ (n - 1) a + \xB ]} e^{-i \omgq t} + K \eBsq e^{-i q [ na + \xB ]} e^{i \omgq t} \\
		&\hspace{5em} \phantom{=\ } + K' \eBsq e^{-i q [ (n - 1) a + \xB ]} e^{i \omgq t} \\[1ex]
		&= (K + e^{-i q a} K') \eBq e^{i q (na + \xB)} e^{-i \omgq t} + (K + e^{-i q a} K') \eBsq e^{-i q (na + \xB)} e^{i \omgq t} \\[1ex]
		&= (K + e^{-i q a} K') \unB.
	}
	Generalizing this, we have
	\al{
		[ K + K' - \mA \omgsq ] \unA &= (K + e^{-i q a} K') \unB, &
		[ K + K' - \mB \omgsq ] \unB &= (K + e^{i q a} K') \unA.
	}
	
	Collecting terms of like time dependence yields
	\aln{
		[ K + K' - m \omgsq ] \eAq e^{i q \rnA} &= (K + e^{-i q a} K') \eBq e^{i q \rnB}, \label{thing5.a1} \\
		[ K + K' - m \omgsq ] \eBq e^{i q \rnB} &= (K + e^{i q a} K') \eAq e^{i q \rnA}, \label{thing5.a2}
	}
	for $e^{-i \omg t}$, and
	\al{
		[ K + K' - m \omgsq ] \eAsq e^{-i q \rnA} &= (K + e^{-i q a} K') \eBsq e^{-i q \rnB}, \\
		[ K + K' - m \omgsq ] \eBsq e^{-i q \rnB} &= (K + e^{-i q a} K') \eAsq e^{-i q \rnA}.
	}
	for $e^{i \omg t}$.
	
	Rearranging Eqs.~\refeq{thing5.a1} and \refeq{thing5.a2}, we have
	\al{
		m \omgsq \eAq &= (K + K') \eAq - (K + e^{-i q a} K') e^{i q (\rnB - \rnA)} \eBq \\
		&= (K + K') \eAq - (e^{i q (\rnB - \rnA)} K + e^{i q (\rnmqB - \rnA)} K') \eBq, \\[1ex]
		m \omgsq \eBq &= -(K + e^{i q a} K') e^{i q (\rnA - \rnB)} \eAq - (K + K') \eBq \\
		&= -(e^{i q (\rnA - \rnB)} K + e^{i q (\rnpqA - \rnB)} K') \eAq - (K + K') \eBq,
	}
	which gives us
	\ans{ \al{
		\DAAq &= \DBBq = K + K', \\
		\DABq &= -e^{i q (\rnB - \rnA)} K - e^{i q (\rnmqB - \rnA)} K', \\
		\DBAq &= -e^{i q (\rnA - \rnB)} K - e^{i q (\rnpqA - \rnB)} K',
	}}%
	as we wanted to show. \qed
	
	Finally, note that
	\al{
		\DsBA &= [ -e^{i q (\rnA - \rnB)} K - e^{i q (\rnpqA - \rnB)} K' ]^*
		= -e^{i q (\rnB - \rnA)} K - e^{i q (\rnB - \rnpqA)} K' \\
		&= -e^{i q (\rnB - \rnA)} K - e^{i q (\rnB - \rnA)} e^{-i q a} K'
		= -e^{i q (\rnB - \rnA)} K - e^{i q (\rnmqB - \rnA)} K' \\
		&= \ans{ \DAB }
	}
	as desired. \qed
}



\prob{}{
	The $2 \times 2$ matrix equation can have a nontrivial solution if the determinant vanishes:
	\eq{
		\mqty| 	\DAAq - m \omgsq & \DABq \\
				\DBAq & \DBBq - m \omgsq |
		= 0.
	}
	Hence show that the frequencies of the modes are given by
	\eq{
		m \omgsq = K + K' \pm \sqrt{ (K + K')^2 - 4 K K' \sin[2]( \frac{q a}{2} ) }.
	}
	\vfix
}

\sol{
	The determinant is
	\eq{
		0 = [ \DAAq - m \omgsq ] [ \DBBq - m \omgsq ] - \DABq \DBAq
		= [ \DAAq - m \omgsq ]^2 - \DABq \DBAq,
	}
	which implies
	{\allowdisplaybreaks
	\aln{
		m \omgsq &= \DAAq \pm \sqrt{\DABq \DBAq} \notag \\
		&= K + K' \pm \sqrt{(e^{i q (\rnB - \rnA)} K + e^{i q (\rnmqB - \rnA)} K') (e^{i q (\rnA - \rnB)} K + e^{i q (\rnpqA - \rnB)} K')} \notag \\
		&= K + K' \pm \sqrt{(K + e^{-i q a} K') e^{i q (\rnB - \rnA)} (K + e^{i q a} K') e^{i q (\rnA - \rnB)}} \notag \\
		&= K + K' \pm \sqrt{(K + e^{-i q a} K') (K + e^{i q a} K')}
		= K + K' \pm \sqrt{K^2 + (e^{i q a} + e^{-i q a}) K K' + {K'}^2} \notag \\
		&= K + K' \pm \sqrt{K^2 + 2\cos(q a) K K' + {K'}^2} \label{5bo} \\
		&= K + K' \pm \sqrt{K^2 + \brac{ 2 - 4 \sin[2](\frac{q a}{2}) } K K' + {K'}^2} \notag \\
		&= K + K' \pm \sqrt{K^2 + 2 K K' + {K'}^2 - 4 \sin[2](\frac{q a}{2}) K K'} \notag \\
		&= \ans{ K + K' \pm \sqrt{ (K + K')^2 - 4 K K' \sin[2]( \frac{q a}{2} ) }, } \label{5b}
	}}
	where we have used the double-angle formula $\cos(2x) = 1 - 2 \sin[2](x)$~\cite{DoubleAngle}. \qed
}



\prob{}{
	Sketch the dispersion relations when $K / K' =  2$.
}

\sol{
	There are two dispersion curves since there are two solutions in Eq.~\refeq{5b}.  The expressions for the branches are
	\eqn{5ceq}{
		\omgq = \frac{1}{\sqrt{m}} \sqrt{ K + K' \pm \sqrt{ (K + K')^2 - 4 K K' \sin[2]( \frac{q a}{2} ) } }
		\begin{cases}
			\text{optical}, \\
			\text{acoustic},
		\end{cases}
	}
	where the acoustic~(optical) branch corresponds to the upper~(lower) sign.  Both branches are shown in Fig.~\ref{5c}, with the $K / K' =  2$ case on the left and the $K = K'$ case on the right.
	
	\fig{5c}{
		\includegraphics[width=0.5\textwidth,trim=1.5cm 0 0 0,clip]{5c}
		\caption{Dispersion curves for $K / K' =  2$.  The optical branch~(blue) corresponds to the upper sign in Eq.~\refeq{5ceq}, and the acoustic branch~(gold) to the lower sign.}
	}
}



\prob{}{
	  Discuss what happens if $K = K'$.
}

\sol{
	If $K = K'$, then not only are the masses of the two atoms identical, but so are their restorative forces.  Thus, the system is essentially reduced to a monatomic chain~\cite[p.~437]{Ashcroft}.  Picking up from Eq.~\refeq{5bo},
	\eq{
		m \omgsq = 2 K \pm \sqrt{2 K^2 + 2 \cos(q a) K^2}
		= 2 K \pm K \sqrt{4 \cos[2](\frac{q a}{2})}
		= 2 K \brac{ 1 - \cos(\frac{q a}{2}) }
		= 4 K \sin[2](\frac{q a}{4}),
	}
	where we have used the double-angle formula $\cos(2 x) = 2 \cos[2](x) - 1$~\cite{DoubleAngle}.  This is Eq.~\refeq{2.25} with $q a \to q a / 2$.  So in this limit, the diatomic chain is reduced to a monatomic chain with lattice constant $a / 2$~\cite[p.~437]{Ashcroft}.
}



\state{Thermodynamics of solids}{
	Compute the following thermodynamic quantities for the harmonic photonic modes in a 1D and a 2D crystal at low temperatures (a.k.a.~phonons) and compare with the textbook example of a 3D crystal.
}

%
%	6.1
%

\prob{}{
	Free energy.
}

\sol{
	A crystal of $N$ molecules is comprised of quantum harmonic oscillators that are free to oscillate in all spatial dimensions.  We can count the number of states in the interval $\ddk$, where $k$ is the wave number.  For a crystal, it is related to the frequency of vibration by $k = d \omg / \ub$, where $\ub$ is the averaged velocity of sound for the particular crystal structure and $d$ the number of spatial dimensions.  The number of states in the interval is, for each case,
	\al{
		\frac{L}{2\pi} \ddk
		&= \frac{L}{2\pi \ub} \ddomg
		\quad (d = 1), &
		\frac{2 \pi A}{(2\pi)^2} k \ddk
		&= \frac{A}{\pi \ub^2} \omg \ddomg,
	}
	where we have taken into account that there are $d$ independent polarization directions in each case~\cite[pp.172--173]{Landau}.
	
	The free energy is $F = N \epso - T \ln Z$, where $\epso$ is the energy per molecule when the system is at equilibrium, which depends on $N$ and the volume $V$~\cite[pp.~87, 172]{Landau}.  The single-particle vibrational partition function is~\cite[p.~136]{Landau}
	\eq{
		Z_1 = \frac{1}{1 - e^{-\hbar \omg / T}}.
	}
	The entire crystal can be modeled as $d \,N \nu$ independent oscillators with total free energy~\cite[p.~172]{Landau}
	\eq{
		F = N \epso - T \sum_{\alp = 1}^{d \,N \nu} \ln(1 - e^{-\hbar \omg_\alp / T}).
	}
	For the entire crystal, the sum can be transformed to an integral over $\omg \in (0, \infty)$~\cite[p.~173]{Landau}.  Referring to the similar integrals in Prob.~{5.2}, we have
	\al{
		(d = 1) \quad
		F &= N \epso - \frac{L T}{2\pi \ub} \int \ln(1 - e^{-\hbar \omg / T}) \ddomg
		= N \epso - \frac{L T^2}{2\pi \hbar \ub} \frac{\pi}{6}
		= \ans{ N \epso - \frac{L T^2}{12 \hbar \ub}, } \\[2ex]
		(d = 2) \quad
		F &= N \epso - \frac{A T}{\pi \ub^2} \int \omg \ln(1 - e^{-\hbar \omg / T}) \ddomg
		= N \epso - \frac{A T^3}{\pi \hbar^2 \ub^2} \Gam(3) \zeta(3)
		= \ans{ N \epso - \frac{1.202 \,A T^3}{\pi \hbar^2 \ub^2}. }
	}
	The 3D expression is~\cite[p.~173]{Landau}
	\eq{
		F = N \epso - \frac{\pi^2 V T^3}{30 \hbar^3 \ub^3},
	}
	suggesting
	\eq{
		\ans{ F = N \epso - j(d) \,\frac{L^d T^{d + 1}}{\hbar^d \ub^d}, }
	}
	where $j(d)$ is a constant that depends on the number of dimensions, and we note that both $\epso$ and $\ub$ depend on the crystal structure and therefore $d$.
}

%
%	6.2
%

\prob{}{
	Entropy.
}

\sol{
	As in Prob.~{5.3}, $S = -\pdv{F}{T}$:
	\al{
		S &= -\pdv{T}(N \epso - \frac{L T^2}{12 \hbar \ub})
		= \ans{ \frac{L T}{6 \hbar \ub} }
		\quad (d = 1), &
		S &= -\pdv{T}(N \epso - \frac{1.202 \,A T^3}{\pi \hbar^2 \ub^2})
		= \ans{ \frac{3.606 \,A T^2}{\pi \hbar^2 \ub^2} }
		\quad (d = 2).
	}
	In 3D, the entropy is~\cite[p.~173]{Landau}
	\eq{
		S = \frac{2 \pi^2 V T^3}{15 \hbar^3 \ub^3},
	}
	which suggests
	\eq{
		\ans{ S \propto \frac{L^d T^d}{\hbar^d \ub^d}. }
	}
	\vfix
}

%
%	6.3
%

\prob{}{
	Energy.
}

\sol{
	As in Prob.~{5.3}, $E = F + T S$:
	\al{
		(d = 1) \quad
		E &= N \epso - \frac{L T^2}{12 \hbar \ub} + T \frac{L T}{6 \hbar \ub}
		= \ans{ N \epso + \frac{L T^2}{12 \hbar \ub}, } \\[2ex]
		(d = 2) \quad
		E &= N \epso - \frac{1.202 \,A T^3}{\pi \hbar^2 \ub^2} + T \frac{3.606 \,A T^2}{\pi \hbar^2 \ub^2}
		= \ans{ N \epso + \frac{2.404 \,A T^3}{\pi \hbar^2 \ub^2}. }
	}
	The 3D equivalent is~\cite[p.~173]{Landau}
	\eq{
		E = N \epso + \frac{\pi^2 T^4}{10 \hbar^3 \ub^3},
	}
	which suggests
	\eq{
		\ans{ E = N \epso + j(d) \frac{d T^{d + 1}}{\hbar^d \ub^d}
		= N - d \,F, }
	}
	where $j(d)$ is a constant that depends on the number of dimensions, and is not necessarily the same as that in Prob.~{6.1}.
}

%
%	6.4
%

\prob{}{
	Specific heat.
}

\sol{
	As in Prob.~{5.4}, $\Cv = (\pdv*{E}{T})_V$.  Then
	\al{
		C &= \pdv{T}(N \epso + \frac{L T^2}{12 \hbar \ub})
		= \ans{ \frac{L T}{6 \hbar \ub} }
		\quad (d = 1), &
		C &= \pdv{T}(N \epso + \frac{2.404 \,A T^3}{\pi \hbar^2 \ub^2})
		= \ans{ \frac{7.212 \,A T^2}{\pi \hbar^2 \ub^2} }
		\quad (d = 2).
	}
	In 3D~\cite[p.~173]{Landau}
	\eq{
		C = \frac{2 \pi^2 V T^3}{5 \hbar^3 \ub^3},
	}
	which suggests
	\eq{
		\ans{ C \propto \frac{L^d T^d}{\hbar^d \ub^d}. }
	}
	\vfix
}


\makebib

\end{document}