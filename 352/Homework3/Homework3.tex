\documentclass[11pt]{article}
\usepackage{homework}

\classname{352}
\homeworknum{3}



\begin{document}

% Environments

\newcommand{\state}[2]{\begin{statement}{#1} #2 \end{statement}}
\newcommand{\prob}[2]{\begin{problem}{#1} #2 \end{problem}}
\newcommand{\subprob}[1]{\begin{subproblem} #1 \end{subproblem}}
\newcommand{\sol}[1]{\begin{solution} #1 \end{solution}}
\newcommand{\fig}[2]{\begin{figure} \centering #2  \label{#1} \end{figure}}

\newcommand{\makebib}{
	\vfill
	\color{black}
	\nocite{*}
	\bibliography{references}{}
	\bibliographystyle{lucas_unsrt}
}
	

% Implication

\newcommand{\qwhere}{\quad \text{where} \quad}
\newcommand{\qimplies}{\quad \implies \quad}
\newcommand{\impliesq}{\implies \quad}



% Brackets

\newcommand{\paren}[1]{\left( #1 \right)}
\newcommand{\brac}[1]{\left[ #1 \right]}
\newcommand{\curly}[1]{\left\{ #1 \right\}}


% Greek

\newcommand{\alp}{\alpha}
\newcommand{\bet}{\beta}
\newcommand{\gam}{\gamma}
\newcommand{\del}{\delta}
\newcommand{\eps}{\epsilon}
\newcommand{\zet}{\zeta}
\newcommand{\tht}{\theta}
\newcommand{\kap}{\kappa}
\newcommand{\lam}{\lambda}
\newcommand{\sig}{\sigma}
\newcommand{\ups}{\upsilon}
\newcommand{\omg}{\omega}

\newcommand{\Gam}{\Gamma}
\newcommand{\Del}{\Delta}
\newcommand{\Tht}{\Theta}
\newcommand{\Lam}{\Lambda}
\newcommand{\Sig}{\Sigma}
\newcommand{\Omg}{\Omega}


% Text

\newcommand{\where}{\text{where }}

% Problem 1

\newcommand{\Hint}{H_\text{int}}
\newcommand{\ddcx}{\dd[3]{x}}
\newcommand{\psib}{\bar{\psi}}

\newcommand{\mh}{m_h}
\newcommand{\mmu}{m_\mu}
\newcommand{\me}{m_e}
\newcommand{\ma}{m_a}

\newcommand{\aexpt}{a_\text{expt.}}
\newcommand{\aQED}{a_\text{QED}}
\renewcommand{\GeV}{\giga\electronvolt}

\newcommand{\gamt}{\gam^5}


\state{Non-equilibrium entropies of Fermi, Bose, and Boltzmann distributions}{
	Consider a gas out of equilibrium with a slightly non-uniform density in $n(x)$ and mean density $\nb = V^{-1} \int n(x) \ddcx$.  We know that if the gas obeys Boltzmann statistics, its entropy is $S = -\int n \log n \ddV$.
}

%
%	1.1
%

\prob{}{
	Argue that this formula is valid only if the gradients are small: $\abs{\gradx n} \ll \nb^{4/3}$ (``coarse-graining condition'') and that $\abs{n(x) - \nb} \ll \nb$.
}

%
%	1.2
%

\prob{}{
	Remove the second condition in {1.1} and obtain the general formula for the entropy for both Fermi and Bose gases.
}



\state{Quantum correction to the Boltzmann thermodynamics}{
	Find the the quantum correction to the free energy of the Boltzmann gas (the leading $\hbar$-dependent term in the expansion of the free energy at small $\hbar$) for Bose and Fermi gases.  From there, find the correction to the pressure.  Does the quantum correction increase or decrease the pressure (and why is the answer predictable)?
}



\clearpage
\newcommand{\tht}{\theta}
\newcommand{\thtxx}{\tht_{xx}}
\newcommand{\thtyy}{\tht_{yy}}
\newcommand{\thttt}{\tht_{tt}}
\newcommand{\thtt}{\tht_t}
\newcommand{\thtx}{\tht_x}
\newcommand{\thty}{\tht_y}
\newcommand{\sint}{\sin{\tht}}
\newcommand{\dxdydt}{\dxdy \dd{t}}

\begin{statement}{}
	The nondimensionalized, multidimensional Sine-Gordon equation,
	\beq
		\thtxx + \thtyy - \thttt = \sint
	\eeq
	for $\tht(x, y, t)$, is the Euler-Lagrange equation for the action integral
	\beq
		S[\tht] = \intR \left\{ \frac{1}{2} \left[ \thtt^2 - (\nabla\tht)^2 \right] - \sint \right\} \dxdydt
	\eeq
	with $\nabla\tht = (\pdv*{\tht}{x}, \pdv*{\tht}{y})$.  The functional $S[\tht]$ is invariant under translation of $x$, $y$, and $t$.  Find the associated energy-momentum tensor and energy-momentum vector.
\end{statement}

\begin{solution}
	Expanding out $(\nabla\tht)^2$, the Lagrangian density is
	\beqn \label{lagr3}
		\Ld = \frac{1}{2} (\thtt^2 - \thtx^2 - \thty^2) - \sint.
	\eeqn
	The energy-momentum tensor is defined by
	\beq
		T_{ij} = \pdv{\Ld}{\tht_{x_i}} \pdv{\tht}{x_j} - \Ld \, \delta_{ij},
	\eeq
	where $x_i \in \{ x_0, x_1, x_2 \} = \{ t, x, y \}$.  The diagonal elements of $T$ are then
	\begin{align*}
		T_{00} &= \pdv{\Ld}{\thtt} \pdv{\tht}{t} - \Ld = \thtt^2 - \frac{1}{2} (\thtt^2 - \thtx^2 - \thty^2) + \sint = \frac{1}{2} (\thtt^2 + \thtx^2 + \thty^2) + \sint, \\
		T_{11} &= \pdv{\Ld}{\thtx} \pdv{\tht}{x} - \Ld = -\thtx^2 - \frac{1}{2} (\thtt^2 - \thtx^2 - \thty^2) + \sint = -\frac{1}{2} (\thtt^2 + \thtx^2 - \thty^2) + \sint, \\
		T_{22} &= \pdv{\Ld}{\thty} \pdv{\tht}{y} - \Ld = -\thty^2 - \frac{1}{2} (\thtt^2 - \thtx^2 - \thty^2) + \sint = -\frac{1}{2} (\thtt^2 - \thtx^2 + \thty^2) + \sint,
	\end{align*}
	and the nondiagonal elements are
	\begin{align*}
		T_{01} &= \pdv{\Ld}{\thtt} \pdv{\tht}{x} = \thtt \thtx, &
		T_{02} &= \pdv{\Ld}{\thtt} \pdv{\tht}{y} = \thtt \thty, &
		T_{12} &= \pdv{\Ld}{\thtx} \pdv{\tht}{y} = -\thtx \thty, \\
		T_{10} &= \pdv{\Ld}{\thtx} \pdv{\tht}{t} = -\thtt \thtx, &
		T_{20} &= \pdv{\Ld}{\thty} \pdv{\tht}{t} = -\thtt \thty, &
		T_{21} &= \pdv{\Ld}{\thty} \pdv{\tht}{x} = -\thtx \thty.
	\end{align*}
	In matrix form, we have
	\beq
		T = \mqty[(\thtt^2 + \thtx^2 + \thty^2) / 2 + \sint & \thtt \thtx & \thtt \thty \\
				-\thtt \thtx & -(\thtt^2 + \thtx^2 - \thty^2) / 2 + \sint & -\thtx \thty \\
				-\thtt \thty & -\thtx \thty & -(\thtt^2 - \thtx^2 + \thty^2) / 2 + \sint ].
	\eeq
	The energy-momentum vector is defined by
	\beq
		P_j = \int T_{0j} \dd{x_1} \dd{x_2}.
	\eeq
	Its components are then
	\begin{align*}
		P_0 &= \int \left[ \frac{1}{2} (\thtt^2 + \thtx^2 + \thty^2) + \sint \right] \dxdy, &
		P_1 &= \int \thtt \thtx \dxdy, &
		P_2 &= \int \thtt \thty \dxdy.
	\end{align*}
\vfix
\end{solution}



\state{Degenerate Bose gas}{\hfix}

%
%	4.1
%

\prob{}{
	The chemical potential of the degenerate Bose gas vanishes below $\Ts$ (the critical temperature of the BEC).  Find its temperature dependence at temperatures slightly above $\Ts$.
}

\sol{
	In three dimensions, the energy distribution of a Bose gas is~\cite[p.~149]{Landau}
	\eq{
		\ddNeps = \frac{g V}{\pi^2 \hbar^3} \sqrt{\frac{m^3}{2}} \frac{\sqrt{\eps}}{e^{(\eps - \mu) / T} - 1} \ddeps.
	}
	Integrating over all energies, we find the total number of molecules~\cite[p.~149]{Landau}.  This gives an expression relating the chemical potential $\mu$ and the density $\nb$~\cite[p.~159]{Landau}:
	\eqn{nb}{
		\nb = \frac{g}{\pi^2 \hbar^3} \sqrt{\frac{m^3}{2}} \intoi \frac{\sqrt{\eps}}{e^{(\eps - \mu) / T} - 1} \ddeps.
	}
	The critical temperature $\Ts$ satisfies this relation for $\mu = 0$.  Let $\nbs$ be the density at $\Ts = 0$, which can be found by making the substitution $z = \eps \Ts$:
	\eq{
		\nbs = \frac{g}{\pi^2 \hbar^3} \sqrt{\frac{m^3}{2}} \intoi \frac{\sqrt{\eps}}{e^{\eps / \Ts} - 1} \ddeps
		= \frac{g}{\pi^2 \hbar^3} \sqrt{\frac{m^3 {\Ts}^3}{2}} \intoi \frac{\sqrt{z}}{e^z - 1} \ddeps.
	}
	The integral may be evaluated using the formula~\cite[p.~156]{Landau}
	\eq{
		\intoi \frac{z^{x - 1}}{e^z - 1} \ddz = \Gam(x) \zeta(x),
	}
	with $x > 1$.  The relevant values are $\Gam(3/2) = \sqrt{\pi} / 2$, and $\zeta(3/2) = 2.612$~\cite[p.~156]{Landau}.  Thus,
	\eq{
		\nbs = \frac{g}{\pi^2 \hbar^3} \sqrt{\frac{m^3 {\Ts}^3}{2}} (2.612)\frac{\sqrt{\pi}}{2}
		= \frac{g}{\pi^2 \hbar^3} \sqrt{\frac{m^3 {\Ts}^3}{2}} (2.612)\frac{\sqrt{\pi}}{2}
		= \frac{0.9235 \,g}{\hbar^3} \paren{ \frac{m \Ts}{\pi} }^{3/2}.
	}
	
	Using this result, we can rewrite Eq.~\refeq{nb} as
	\eq{
		\nb = \nbs + \frac{g}{\pi^2 \hbar^3} \sqrt{\frac{m^3}{2}} \intoi \frac{\sqrt{\eps}}{e^{(\eps - \mu) / T} - 1} \ddeps - \nbs
		= \nbs + \frac{g}{\pi^2 \hbar^3} \sqrt{\frac{m^3}{2}} \intoi \paren{ \frac{\sqrt{\eps}}{e^{(\eps - \mu) / T} - 1} - \frac{\sqrt{\eps}}{e^{\eps / T} - 1} } \ddeps.
	}
	It follows from $\mu(\Ts) = 0$ that $\mu \ll 1$ for temperatures such that $T - \Ts \ll 1$.  \hl{Then, somehow,}~\cite[p.~161]{Landau}
	\eq{
		\intoi \paren{ \frac{\sqrt{\eps}}{e^{(\eps - \mu) / T} - 1} - \frac{\sqrt{\eps}}{e^{\eps / T} - 1} } \ddeps
		= T \mu \intoi \frac{\ddeps}{\sqrt{\eps} (\eps + \abs{\mu})}
%		= T \mu \brac{  }\oi
		= -\pi T \sqrt{\abs{\mu}}.
	}
	Making this substitution and solving for $\mu$, we find
	\eq{
		\nb = \nbs - \frac{g T}{\pi \hbar^3} \sqrt{\frac{\abs{\mu} m^3}{2}}
%		\qimplies
%		\sqrt{\frac{\abs{\mu} m^3}{2}} = \frac{\pi \hbar^3 (\nbs - \nb)}{g T}
		\qimplies
		\abs{\mu} = \frac{2}{m^3} \paren{ \frac{\pi \hbar^3 (\nbs - \nb)}{g T} }^2
		= \frac{2 \pi^2 \hbar^6 (\nbs - \nb)^2}{m^3 g^2 T^2}.
	}
	For the Bose distribution, we know that $\mu < 0$~\cite[p.~145]{Landau}.  This gives us
	\eq{
		\mu = -\frac{2 \pi^2 \hbar^6 (\nbs - \nb)^2}{m^3 g^2 T^2}
		\qimplies
		\ans{ \mu \propto -\frac{1}{T^2} }
	}
	where $T - \Ts \ll 1$.
}

%
%	4.2
%

\prob{}{
	Find the discontinuities in the derivatives of thermodynamic quantities at the BEC transition.  Which order is this phase transition?
}

\sol{
	
}

%
%	4.3
%

\prob{(*)}{
	Can the ideal Bose gas condense in spatial dimensions 1 and 2?  Discuss what happens in these cases. 
}



\clearpage
\state{Thermodynamics of radiation}{
	Compute the following thermodynamic quantities of a radiation field in a 1D and a 2D cavity and compare it with the textbook example of a 3D cavity.
}

%
%	5.1
%

\prob{}{
	Planck formula and the Rayleigh-Jeans and Wien limits of the distribution over frequencies.
}

\sol{
	Planck's formula gives the spectral energy distribution of blackbody radiation.  We start with Planck's distribution, which gives the mean number of photons in quantum state $k$:
	\eq{
		\nkb = \frac{1}{e^{\hbar \omgk / T} - 1},
	}
	where $\omgk$ is the eigenfrequency for state $k$ in the cavity of volume $V$~\cite[p.~163]{Landau}.
	
	The number of states in the interval $(f, f + \ddf)$, where $f = \omg / c$ is the wave number, is in each case~\cite[p.~163]{Landau}
	\al{
		\frac{L}{2\pi} \ddf &= \frac{L}{2\pi c} \ddomg
		\quad (d = 1), &
		\frac{2\pi A}{(2\pi)^2} f \ddf &= \frac{A}{2 \pi c^2} \omg \ddomg
		\quad (d = 2).
	}
	(In both 1D and 2D, there is only one polarization direction for photons, so we do not need to multiply these expressions by a constant.)
	
	In each case, the number of photons in each interval is~\cite[p.~163]{Landau}
	\al{
		\ddNomg &= \frac{L}{2\pi c} \frac{\ddomg}{e^{\hbar \omg / T} - 1}
		\quad (d = 1), &
		\ddNomg &= \frac{A}{2 \pi c^2} \frac{\omg}{e^{\hbar \omg / T} - 1} \ddomg
		\quad (d = 2).
	}
	Transforming to total energy $\eps = \hbar \omg$, Planck's distribution is
	\al{
		\ans{ \ddEomg\ }&\ans{= \frac{\hbar L}{2\pi c} \frac{\omg}{e^{\hbar \omg / T} - 1} \ddomg }
		\quad (d = 1), &
		\ans{ \ddEomg\ }&\ans{= \frac{\hbar A}{2 \pi c^2} \frac{\omg^2}{e^{\hbar \omg / T} - 1} \ddomg }
		\quad (d = 2).
	}
	The 3D equivalent is~\cite[p.~163]{Landau}
	\eq{
		\ddEomg = \frac{\hbar V}{\pi^2 c^3} \frac{\omg^3}{e^{\hbar \omg / T} - 1} \ddomg
		\quad (d = 3).
	}
	Comparing the formulae, it appears that
	\eq{
		\ddEomg = \frac{\hbar L^d}{\pi^{\min(d - 1, 1)} c^d} \frac{\omg^d}{e^{\hbar \omg / T} - 1} \ddomg
	}
	where $d$ is the number of spatial dimensions and $A \equiv L^2$, $V \equiv L^3$.
	
	The Rayleigh-Jeans limit is $\hbar \omg \ll T$.  Letting $u = \hbar \omg / T$ and expanding about $u = 0$, we obtain
	\al{
		(d = 1) \quad
		\ddEomg &= \frac{L T}{2\pi c} \frac{u}{e^u - 1} \ddomg
		\approx \frac{L T}{2\pi c} \curly{ \limuo\paren{ \frac{u}{e^u - 1} } + u \brac{ \pdv{u}(\frac{u}{e^u - 1}) }\uo + \frac{u^2}{2} \brac{ \pdv[2]{u}(\frac{u}{e^u - 1}) }\uo } \ddomg \\
		&= \frac{L T}{2\pi c} \curly{ 1 + u \brac{ \frac{1}{e^u - 1} - \frac{e^u u}{(e^u - 1)^2} }\uo + \frac{u^2}{2} \brac{ \frac{2 e^u u}{(e^u - 1)^3} - \frac{(2 + u) e^u}{(e^u - 1)^2} }\uo } \ddomg \\
		&= \frac{L T}{2\pi c} \paren{ 1 - \frac{u}{2} + \frac{u^2}{12} } \ddomg
		= \ans{ \frac{L}{2\pi c} \paren{ T - \frac{\hbar \omg}{2} + \frac{\hbar^2 \omg^2}{12 T} } \ddomg, }
		\\[2ex]
		(d = 2) \quad
		\ddEomg &= \frac{A T^2}{2 \pi \hbar c^2} \frac{u^2}{e^u - 1} \ddomg
	}
}

%
%	5.2
%

\prob{}{
	Free energy and the Stefan-Boltzmann constant.
}

%
%	5.3
%

\prob{}{
	The relation between the free energy and energy (Boltzmann law).
}

%
%	5.4
%

\prob{}{
	Specific heat.
}

%
%	5.6
%

\prob{}{
	Pressure.
}

%
%	5.7
%

\prob{}{
	The total number of photons in the cavity.
}



\state{Thermodynamics of solids}{
	Compute the following thermodynamic quantities for the harmonic photonic modes in a 1D and a 2D crystal at low temperatures (a.k.a.~phonons) and compare with the textbook example of a 3D crystal.
}

%
%	6.1
%

\prob{}{
	Free energy.
}

%
%	6.2
%

\prob{}{
	Entropy.
}

%
%	6.3
%

\prob{}{
	Energy.
}

%
%	6.4
%

\prob{}{
	Specific heat.
}


\makebib

\end{document}