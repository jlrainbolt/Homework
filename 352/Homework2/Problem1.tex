\state{Thermodynamics of a relativistic gas}{\hfix}

%
%	1.1
%

\prob{}{
	Find the statistical distribution of a relativistic gas in momentum space, and in energies.  Discuss the relativistic corrections compared to the Maxwell distribution.
}

\sol{
	We will use the Boltzmann distribution for an ideal gas in the classical limit.  The distribution of the density of states in phase space is
	\eq{
		\npq = a \exp(-\frac{\epspq}{T}),
	}
	where $\npq$ is the mean number of molecules of energy $\epspq$ in a phase space volume element $\ddp \ddq$.  Here $a$ is a normalization constant, determined by normalizing to $N / V$ where $N$ is the total number of gas molecules and $V$ is the total volume.  The mean number of molecules contained in a single volume element is
	\eq{
		\ddN = \frac{n(p, q)}{(2\pi\hbar)^r} \ddp \ddq,
	}
	where $r$ is the number of translational degrees of freedom~\cite[pp.~107--108]{Landau}.  We assume $r = 3$.
	
	The energy of a single relativistic particle is $\eps = c \sqrt{m^2 c^2 + \vp^2}$, where $m$ is its mass, $\vp$ its three-dimensional momentum, and $c$ the speed of light~\cite[p.~110]{Landau}.  This gives us
	\eqn{dNp}{
		\ddNvp = \frac{a}{(2\pi\hbar)^3} \exp(-\frac{c \sqrt{m^2 c^2 + \vp^2}}{T}) \ddcp,
	}
	where we are ignoring the coordinate-space volume $\ddq$, because it would disappear anyway upon normalization.
	
	Now we must find $a$ by integrating over all of momentum space, which we will carry out  using spherical coordinates with $\ddcp = p^2 \sin\tht \ddp \ddtht \ddphi$.  We find
	\eqn{thingy2}{
		\frac{N}{V} = \int \ddNvp
		= \frac{4\pi a}{(2\pi\hbar)^3} \intoi p^2 \exp(-\frac{c \sqrt{m^2 c^2 + p^2}}{T}) \ddp.
	}
	Let $u = \sqrt{m^2 c^2 + p^2}$.  Then the lower bound of integration for $u$ is $m c$, and
	\eq{
		\dv{u}{p} = \frac{p}{\sqrt{m^2 c^2 + p^2}} 
		 = \frac{\sqrt{u^2 - m^2 c^2}}{u}
		 \qimplies
		 \ddp = \frac{u}{\sqrt{u^2 - m^2 c^2}} \ddu.
	}
	Then we have
	\eqn{thingy}{
		\frac{N}{V} = \frac{4\pi a}{(2\pi\hbar)^3} \intmci u \sqrt{u^2 - m^2 c^2} e^{-c u / T} \ddu.
	}
	Note that~\cite[p.~351]{Integrals}
	\eqn{integral}{
		\intui x (x^2 - u^2)^{\nu - 1} e^{-\mu x} \ddx = \frac{2^{\nu - 1/2}}{\sqrt{\pi}} \mu^{1/2 - \nu} u^{\nu + 1/2} \,\Gam(\nu) \,K_{\nu + 1/2}(u\mu)
	}
	for $\Re(u \mu) > 0$, where $\Gam(z)$ is the Gamma function and $K_n(z)$ is a modified Bessel function of the second kind~\cite[p.~175]{Pathria}.  Comparing with Eq.~\refeq{thingy}, we have $x \to u$, $u \to mc$, $\nu \to 3/2$, and $\mu \to c/T$.  Note also that $\Gam(3/2) = \sqrt{\pi} / 2$.  Then, evaluating Eq.~\refeq{thingy},
	\eq{
		\frac{N}{V} = \frac{4\pi a}{(2\pi\hbar)^3} T m^2 c \,K_2(\bet mc^2)
		\qimplies
		a = \frac{N}{V} \frac{(2\pi\hbar)^3}{4\pi} \frac{1}{T m^2 c \,K_2(\bet mc^2)}.
	}
	Substituting into Eq.~\refeq{dNp}, we obtain
	\eqn{ddNvp}{
		\ddNvp = \frac{N}{V} \frac{\exp(-\bet c \sqrt{m^2 c^2 + \vp^2})}{4\pi T m^2 c \,K_2(\bet mc^2)} \ddcp
	}
	as the occupation number distribution in momentum space.  Multiplying by $V / N$, we find the momentum distribution, which is normalized to unity:
	\eqn{1.1ans}{
		\ans{ \ddP = \frac{\exp(-\bet c \sqrt{m^2 c^2 + \vp^2})}{4\pi T m^2 c \,K_2(\bet mc^2)} \ddcp. }
	}
	
	To find the distribution in energy space, we will change variables in Eq.~\refeq{dNp} to $\eps = c \sqrt{m^2 c^2 + \vp^2}$.  Noting that
	\eq{
		\dv{p}{\eps} = \frac{c p}{\sqrt{m^2 c^2 + p^2}}
		\qimplies
		\ddp = \frac{\eps}{c^2} \sqrt{\eps^2 / c^2 - m^2 c^2} \dde
		= \frac{\eps}{c^3} \sqrt{\eps^2 - m^2 c^4} \dde,
	}
	we have
	\eq{
		\ddNeps = \frac{4\pi b}{(2\pi\hbar)^3} \frac{1}{c^3} \eps \sqrt{\eps^2 - m^2 c^4} e^{-\eps / T} \dde,
	}
	where $b$ is a normalization constant, which we will find by integration:
	\eq{
		\frac{N}{V} = \frac{4\pi b}{(2\pi\hbar)^3} \frac{1}{c^3} \intmcsi \eps \sqrt{\eps^2 - m^2 c^4} e^{-\eps / T} \dde
	}
	Again comparing to Eq.~\refeq{integral}, we have $x \to \eps$, $u \to mc^2$, $\nu \to 3/2$, and $\mu \to \bet$.  This gives us
	\eq{
		\frac{N}{V} = \frac{4\pi b}{(2\pi\hbar)^3} T m^2 c \,K_{2}(\bet m c^2)
		\qimplies b = a,
	}
	so the statistical distribution in energy space is found by
	\eqn{1.1ans2}{
		\ddNeps = \frac{N}{V} \frac{e^{-\bet \eps} \eps \sqrt{\eps^2 - m^2 c^4}}{T m^2 c^4 \,K_2(\bet mc^2)} \dde
		\qimplies
		\ans{ \ddcE = \frac{e^{-\bet \eps} \eps \sqrt{\eps^2 - m^2 c^4}}{T m^2 c^4 \,K_2(\bet mc^2)} \dde. }
	}

	
	The Maxwell distribution in momentum space is given by~\cite[p.~109]{Landau}
	\eqn{Maxwellp}{
		\ddNvp = \frac{N}{V} \frac{1}{(2\pi m T)^{3/2}} \exp(-\frac{\px^2 + \py^2 + \pz^2}{2m T}) \ddpxyz
		\qimplies
		\ddP = \frac{1}{(2\pi m T)^{3/2}} \exp(-\frac{\vp^2}{2m T}) \ddcp.
	}
	From p.~2 of Lecture 4, the Maxwell distribution in energy space is
	\eqn{Maxwelle}{
		\ddcE = \frac{2}{\sqrt{\pi T^3}} e^{-\eps / T} \sqrt{\eps} \dde.
	}
	
	Both distributions are similar to the relativistic ones in Eqs.~(\ref{1.1ans}--\ref{1.1ans2}).  The Maxwell distributions have the kinetic energy $\eps = \vp^2 / 2m$ in the exponent, whereas Eqs.~(\ref{1.1ans}--\ref{1.1ans2}) have the relativistic energy $\eps = c \sqrt{m^2 c^2 + \vp^2}$.  The factor of $\bet$ in the exponent is the same in both cases.  	However, Eq.~\refeq{1.1ans2} goes as $e^{-\bet \eps} \eps^2$ while Eq.~\refeq{Maxwelle} goes as $e^{-\bet \eps} \sqrt{\eps}$.
	
	The normalization of Eqs.~(\ref{1.1ans}--\ref{1.1ans2}) is different than that of Eqs.~(\ref{Maxwellp}--\ref{Maxwelle}) in order to account for the relativistic energy.  The factor of $1/K_2{\bet mc^2}$ means that the relativistic distributions fall off much more rapidly with $T$ than the nonrelativistic ones.  This is sensible because the relativistic particles are able to access a much larger range of momenta at high temperatures, which spreads them out over a larger range of energies.
}

%
%	1.2
%

\clearpage
\prob{}{
	Now take the ultra-relativistic limit.  Find the mean energy $\evE$ and the second moment of energy $\evEs$.  Find the free energy and the entropy in the limits of high and low temperature.
}

\sol{
	The ultra-relativistic limit is $T \gg m c^2$~\cite[p.~175]{Pathria}.  Let $u = m c^2 / T$.  Then Eq.~\refeq{1.1ans2} becomes
	\eq{
		\limuo \ddcE = \limuo \frac{1}{T^2} \frac{e^{-\bet \eps} \eps \sqrt{\eps^2 / T^2 - u^2}}{u^2 \,K_2(u)} \dde
		= \frac{1}{2 T^3} e^{-\bet \eps} \eps^2 \dde,
	}
	where we have used Mathematica to evaluate the limit of the denominator.
	
	The mean energy can be found by $\evE = N \eveps$, where $\eveps$ is the mean energy per molecule:
	\eq{
		\evE = N \eveps
		= N \limuo \int \eps \ddcE
		= \frac{N}{2 T^3} \intoi \eps^3 e^{-\bet \eps} \dde
		= \frac{N}{2 T^3} 3!\, T^4
		= \ans{ 3N T, }
	}
	where we integrate from $\eps = 0$ since $mc^2 \to 0$ in this limit, and we have used $\intoi x^n e^{-\mu x} \ddx = n!\, \mu^{-n - 1}$~\cite[p.~340]{Integrals}.

	The second moment of energy is not an additive quantity, so we cannot simply compute $N \ev{\eps^2}$.  Let $E = \sumiN \epsi$, where $\epsi$ is the energy of a given molecule.  Then
	\eq{
		E^2 = \paren{ \sumiN \epsi } \paren{ \sumjN \epsj }
		= \sumiN \epsi^2 + \sumiN \sumji \epsi \epsj,
	}
	and the second moment of energy can be found by
	\aln{
		\evEs &= \int \sumiN \paren{ \epsi^2 + \sumji \epsi \epsj } \prodkN \ddcEk
		= \sumiN \paren{ \int \epsi^2 \prodkN \ddcEk + \sumji \int \epsi \epsj \prodkN \ddcEk } \notag \\
		&= \sumiN \paren{ \int \epsi^2 \ddcEi + \sumji \int \epsi \ddcEi \int \epsj \ddcEj }, \label{evEs}
	}
	where in going to the final equality we have used the fact that $\int \ddcEk = 1$.  For the first term,
	\eq{
		\int \epsi^2 \ddcEi = \limuo \int \epsi^2 \ddcEi
		= \frac{1}{2 T^3} \intoi \epsi^4 e^{-\bet \epsi} \dde_i
		= \frac{1}{2 T^3} 4!\, T^5
		= 12 T^2.
	}
	For the second term,
	\eq{
		\int \epsi \ddcEi \int \epsj \ddcEj = \ev{\epsi} \ev{\epsj}
		= 9 T^2.
	}
	Then Eq.~\refeq{evEs} becomes
	\eq{
		\evEs =  N (12 T^2) + N (N - 1) (9 T^2)
		= \ans{ 3 N (3N + 1) T^2. }
	}

	The Helmholtz free energy is $F = -T \ln Z$, where $Z$ is the partition function~\cite[p.~87]{Landau}.  The single-particle partition function of the Maxwell distribution is simply the denominator of $\ddP$:
	\eqn{partfunc}{
		\ddP = \frac{e^{-\bet \vp^2 / 2m}}{\Zii} \ddcp
		\qimplies
		\Zii = (2\pi m T)^{3/2}.
	}
	Applying this procedure to Eq.~\refeq{1.1ans}, and assuming the gas molecules are indistinguishable, we find
	\eq{
		\Zii = 4\pi T m^2 c \,K_2(\bet mc^2)
		\qimplies
		Z = \frac{1}{N!} \brac{ 4\pi T m^2 c \,K_2(\bet mc^2) }^N.
	}
	
%	Then the free energy is
%	\eq{
%		F = -T \ln Z
%		\approx -T \brac{ N \ln(4\pi T m^2 c \,K_2(\bet mc^2)) - N \ln N + N }
%		= -N T \brac{ \ln(\frac{4\pi}{N} T m^2 c \,K_2(\bet mc^2)) + 1 },
%	}
%	where we have used Stirling's approximation $\ln N! \approx N \ln N - N$.  The entropy is
%	\al{
%		S &= -\paren{ \pdv{F}{T} }_V
%		= N \brac{ \ln(\frac{4\pi}{N} T m^2 c \,K_2(\bet mc^2)) + 1 } + N T \pdv{T} \brac{ \ln(\frac{4\pi}{N} m^2 c) + \ln T + \ln K_2(mc^2 / T) + 1 } \\
%		&= N \brac{ \ln(\frac{4\pi}{N} T m^2 c \,K_2(\bet mc^2)) + 1 } + N T \paren{ \frac{1}{T} + m c^2 \frac{K_1(\bet mc^2) + K_3(\bet mc^2)}{2 T^2} } \\
%		&= N \brac{ \ln(\frac{4\pi}{N} T m^2 c \,K_2(\bet mc^2)) + 1 + N + m c^2 \frac{K_1(\bet mc^2) + K_3(\bet mc^2)}{2 T^2}},
%	}
%	where we have used $\pdv*{K_\nu(z)}{z} = -[ K_{\nu - 1}(z) + K_{\nu + 1}(z) ]/2$~\cite{Derivative}.
%	
%	In the high-temperature limit,
%	\al{
%		\limTi F &= \limTi -N T \brac{ \ln(\frac{4\pi}{N} T m^2 c \,K_2(mc^2 / T)) + 1 }
%		= \limTi -N T \paren{ \ln T + \ln K_2(m c^2 / T) } \\
%		&= \ans{ \infty, } \\[2ex]
%		\limTi S &= \limTi N \brac{ \ln(\frac{4\pi}{N} T m^2 c \,K_2(\bet mc^2)) + 1 + N + m c^2 \frac{K_1(\bet mc^2) + K_3(\bet mc^2)}{2 T^2}} \\
%		&= \limTi N \brac{ \ln T + \ln K_2(mc^2 / T) + m c^2 \frac{K_1(mc^2 / T) + K_3(mc^2 / T)}{2 T^2}},
%	}
%	where we have used Mathematica to evaluate the limits~(graphically when needed).
	
	For the ultra-relativistic case,
	\eq{
		\limuo \Zii = 4\pi \frac{T^3}{c^3} \limuo u^2 \,K_2(u)
		= 8\pi \frac{T^3}{c^3}
		\qimplies
		Z = \frac{1}{N!} \paren{ 8\pi \frac{T^3}{c^3} }^N.
	}
	Then the free energy is
	\eq{
		F = -T \ln Z
		= -T \paren{ N \ln(8\pi \frac{T^3}{c^3}) - \ln N! }
		\approx \ans{ -N T \paren{ \ln(\frac{8\pi}{N} \frac{T^3}{c^3}) + 1 }, }
	}
	where we have used Stirling's approximation $\ln N! \approx N \ln N - N$.  The entropy can be found by $S = -(\pdv*{F}{T})_V$~\cite[p.~47]{Pathria}, which gives us
	\al{
		S &= -\paren{ \pdv{F}{T} }_V
		= \pdv{T} \brac{ N T \paren{ \ln(\frac{8\pi}{N} \frac{T^3}{c^3}) + 1 } }
		= N \paren{ \ln(\frac{8\pi}{N} \frac{T^3}{c^3}) + 1 } + N T \pdv{T} \brac{ \ln(\frac{8\pi}{N c^3}) + 3 \ln T + 1 } \\
		&= \ans{ N \paren{ \ln(\frac{8\pi}{N} \frac{T^3}{c^3}) + 4 }. }
	}	
	In the high-temperature limit,
	\al{
		\limTi F &= \limTi -N T \paren{ \ln(\frac{8\pi}{N} \frac{T^3}{c^3}) + 1 }
		= \limTi -3 N T \ln T
		= \ans{ -\infty, } \\
		\limTi S &= \limTi N \paren{ \ln(\frac{8\pi}{N} \frac{T^3}{c^3}) + 4 }
		= \limTi 3 N \ln T
		= \ans{ \infty. }
	}
	In the low-temperature limit,
	\al{
		\limTo F &= \limTo -N T \paren{ \ln(\frac{8\pi}{N} \frac{T^3}{c^3}) + 1 }
		= \limTo -3N T \ln T
		= \ans{ 0, } \\
		\limTo S &= \limTo N \paren{ \ln(\frac{8\pi}{N} \frac{T^3}{c^3}) + 4 }
		= \limTo 3 N \ln T
		= \ans{ -\infty. }
	}
	\vfix
}

%
%	1.3
%

\prob{}{
	In the non-relativistic Maxwell distribution, the different translational degrees of freedom are independent as the kinetic energy is the sum of three independent terms $K = \sum_{i=1}^3 \pii^2 / 2m$.  This is not so in the relativistic case.  For the ultra-relativistic gas compute the quantities
	\al{
		\aij &= \frac{\evpijs}{3 \evpis \evpjs}, &
		\rij &= \frac{\evpijs}{\sqrt{\evpiq \evpjq}},
	}
	in spatial dimensions $d = 2, 3$ (here $i, j$ enumerate spatial dimensions).  [$\rij$ is the uncentered ``correlation coefficient''.  $\aij = 1$ in the classical (Gaussian) case by Wick's theorem.]  Compare them to the non-relativistic case.  Discuss their meaning and dependence on $d$ (at least based on $d = 2, 3$).
}

\sol{
	In the ultra-relativistic case, Eq.~\refeq{1.1ans} becomes
	\eqn{ultraddP}{
		\limuo \ddP = \limuo \frac{c^3}{T^3} \frac{\exp(-\sqrt{u^2 + c^2 \vp^2 / T^2})}{4\pi u^2 \,K_2(u)} \ddcp
		= \frac{c^3}{8\pi T^3} \exp(-\bet c \abs{\vp}) \ddcp.
%		= \frac{c^3}{T^3} \frac{\exp[-\bet c (\px + \py + \pz)]}{8 \pi} \ddpxyz.
	}
	Clearly this represents the three-dimensional case.  For this case,
		\eq{
		\evpis = \evpzs
		= \int \pz^2 \ddP
		= \frac{c^3}{8\pi T^3} \intotp \ddphi \intqq \cos^2\tht \ddcost \intoi p^4 e^{-\bet c p} \ddp
		= \frac{c^3}{8\pi T^3} 2\pi \frac{2}{3} \frac{4!}{(\bet c)^5}
		= 4 \frac{T^2}{c^2},
	}
	\al{
		\evpiq &= \evpiis
		= \int \pz^4 \ddP
		= \frac{c^3}{8\pi T^3} \intotp \ddphi \intqq \cos^4\tht \ddcost \intoi p^6 e^{-\bet c p} \ddp
		= \frac{c^3}{8\pi T^3} 2\pi \frac{2}{5} \frac{6!}{(\bet c)^7}
		= 72 \frac{T^4}{c^4}, \\[2ex]
		\evpijs &= \evpxys
		= \int \px^2 \py^2
		= \frac{c^3}{8\pi T^3} \intotp \cos^2\phi \sin^2\phi \ddphi \intopi \sin^5\tht \ddtht \intoi p^6 e^{-\bet c p} \ddp
		= \frac{c^3}{8\pi T^3} \frac{\pi}{4} \frac{16}{15} \frac{6!}{(\bet c)^7}
		= 24 \frac{T^4}{c^4},
	}
	where we have used $\px = p \cos\phi \sin\tht$, $\py = p \sin\phi \sin\tht$, and $\pz = p \cos\tht$.  So we find
	\aln{
		\ans{\aij\ }&{\color{blue} = \begin{cases}
			3/2 & i = j, \\
			1/2 & i \neq j,
			\end{cases}} &
		\ans{\rij\ }&{\color{blue} = \begin{cases}
			1 & i = j, \\
			1/3 & i \neq j,
			\end{cases}} \label{ur3d}
	}
	for the three-dimensional ultra-relativistic gas.
	
	In the two-dimensional case, we need to return to Eq.~\refeq{dNp}, which becomes
	\eq{
		\ddNvp = \frac{a}{(2\pi\hbar)^2} \exp(-\frac{c \sqrt{m^2 c^2 + \vp^2}}{T}) \ddsp.
	}
	To integrate over all of momentum space and find $a$, we use the plane polar coordinates $\ddsp = p \ddp \ddtht$.  We find
	\al{
		\frac{N}{V} &= \int \ddNvp
		= \frac{2\pi a}{(2\pi\hbar)^2} \intoi p \exp(-\frac{c \sqrt{m^2 c^2 + p^2}}{T}) \ddp
		= \frac{2\pi a}{(2\pi\hbar)^2} \int_{mc}^\infty u e^{-\bet c u} \ddu \\
		&= \frac{2\pi a}{(2\pi\hbar)^2}  \paren{ \brac{ -\frac{T}{c} u e^{-\bet c u} }_{mc}^\infty + \frac{T}{c} \int_{mc}^\infty e^{-\bet c u} \ddu }
		= \frac{2\pi a}{(2\pi\hbar)^2} \paren{ mT e^{-\bet m c^2} - \frac{T}{c} \brac{ \frac{T}{c} e^{-\bet c u} }_{mc}^\infty } \\
		&= \frac{2\pi a}{(2\pi\hbar)^2} e^{-\bet m c^2} \paren{ mT + \frac{T^2}{c^2} },
	}
	so
	\eq{
		a = \frac{N}{V} \frac{(2\pi\hbar)^2}{2\pi} \frac{e^{\bet m c^2}}{m T + T^2 / c^2}
		\qimplies
		\ddNvp = \frac{N}{V} \frac{1}{2\pi} \frac{e^{\bet m c^2}}{m T + T^2 / c^2} \exp(-\frac{c \sqrt{m^2 c^2 + \vp^2}}{T}) \ddsp.
	}
	Then we have
	\eq{
		\ddP = \frac{1}{2\pi} \frac{e^{\bet m c^2}}{m T + T^2 / c^2} \exp(-\frac{c \sqrt{m^2 c^2 + \vp^2}}{T}) \ddsp.
	}
	Taking the ultra-relativistic limit,
	\eq{
		\limuo \ddP = \limuo \frac{c^2}{2\pi T^2} \frac{e^{u}}{u + 1} \exp(-\sqrt{u^2 + c^2 \vp^2 / T^2}) \ddsp
		= \frac{c^2}{2\pi T^2} \exp(-\bet c \abs{\vp}) \ddsp.
	}
	For this case,
	\al{
		\evpis &= \evpxs
%		= \frac{c^2}{2\pi T^2} \iint \px^2 \exp(-\bet c \sqrt{\px^2 + \py^2}) \ddpxy
%		= \frac{c^2}{2\pi T^2} \intotp \intoi p^2 \cos^2\tht \exp(-\bet c p) p \ddp \ddtht \\
		= \frac{c^2}{2\pi T^2} \intotp \cos^2\tht \ddtht \intoi p^3 e^{-\bet c p} \ddp
		= \frac{c^2}{2\pi T^2} \frac{3!\,\pi}{(\bet c)^4}
		= 3 \frac{T^2}{c^2}, \\[2ex]
		\evpiq &= \evpiis
%		= \ev{\px^4}
		= \frac{c^2}{2\pi T^2} \intotp \cos^4\tht \ddtht \intoi p^5 e^{-\bet c p} \ddp
		= \frac{c^2}{2\pi T^2} \frac{3\pi}{4} \frac{5!}{(\bet c)^6}
		= 45 \frac{T^4}{c^4}, \\[2ex]
		\evpijs &= \evpxys
%		= \frac{c^2}{2\pi T^2} \iint \px^2 \py^2 \exp(-\bet c \sqrt{\px^2 + \py^2}) \ddpxy
		= \frac{c^2}{2\pi T^2} \intotp \cos^2\tht \sin^2\tht \ddtht \intoi p^5 e^{-\bet c p} \ddp
		= \frac{c^2}{2\pi T^2} \frac{\pi}{4} \frac{5!}{(\bet c)^6}
		= 15 \frac{T^4}{c^4},
	}
	where we have used $\px = p \cos\tht$ and $\py = p \sin\tht$.  So we find
	\aln{
		\ans{\aij\ }&{\color{blue} =  \begin{cases}
			5/3 & i = j, \\
			5/9 & i \neq j,
			\end{cases}} &
		\ans{\rij\ }&{\color{blue} =  \begin{cases}
			1 & i = j, \\
			1/3 & i \neq j,
			\end{cases}} \label{ur2d}
	}
	for the two-dimensional ultra-relativistic gas.
	
	For the non-relativistic case, the three-dimensional momentum distribution is given by Eq.~\refeq{Maxwellp}.  This gives us
	\al{
		\evpis &= \evpzs
%		= \frac{1}{(2\pi m T)^{3/2}} \int \pz^2 \exp(-\frac{\px^2 + \py^2 + \pz^2}{2m T}) \ddpxyz
		= \frac{1}{(2\pi m T)^{3/2}} \intotp \ddphi \intqq \cos^2\tht \ddcost \intoi p^4 e^{-\bet p^2 / 2m} \ddp
		= \frac{1}{(2\pi m T)^{3/2}} 2\pi \frac{2}{3} \frac{\Gam(5/2)}{2 (2mT)^{-5/2}} \\
		&= \frac{1}{(2\pi m T)^{3/2}} \frac{\pi^{3/2} (2mT)^{5/2}}{2}
		= mT, \\[2ex]
		\evpiq &= \evpiis
		= \frac{1}{(2\pi m T)^{3/2}} \intotp \ddphi \intqq \cos^4\tht \ddcost \intoi p^6 e^{-\bet p^2 / 2m} \ddp
		= \frac{1}{(2\pi m T)^{3/2}} 2\pi \frac{2}{5} \frac{\Gam(7/2)}{2 (2mT)^{-7/2}} \\
		&= \frac{1}{(2\pi m T)^{3/2}} \frac{3 \pi^{3/2} (2mT)^{7/2}}{4}
		= 3 m^2 T^2, \\[2ex]
		\evpijs &= \evpxys
		= \frac{1}{(2\pi m T)^{3/2}} \intotp \cos^2\phi \sin^2\phi \ddphi \intopi \sin^5\tht \intoi p^6 e^{-\bet p^2 / 2m} \ddp
		= \frac{1}{(2\pi m T)^{3/2}} \frac{\pi}{4} \frac{16}{15} \frac{\Gam(7/2)}{2 (2mT)^{-7/2}} \\
		&= \frac{1}{(2\pi m T)^{3/2}} \frac{\pi^{3/2} (2mT)^{7/2}}{4}
		= m^2 T^2,
	}
	where we have used
	\aln{ \label{gaussian}
		\intoi x^m \exp(-\bet x^n) \ddx &= \frac{\Gam(\gam)}{n \bet^\gam}, &
		\gam &= \frac{m + 1}{n},
	}
	for $\Re(\bet), \Re(m), \Re(n) > 0$~\cite[p.~337]{Integrals}.  So we find
	\aln{
		\ans{\aij\ }&{\color{blue} = \begin{cases}
			1 & i = j, \\
			1/3 & i \neq j,
			\end{cases}} &
		\ans{\rij\ }&{\color{blue} = \begin{cases}
			1 & i = j, \\
			1/3 & i \neq j,
			\end{cases}} \label{nr3d}
	}
	for the three-dimensional non-relativistic gas.
	
	For the two-dimensional non-relativistic case, we return to Eq.~\refeq{dNp} with $r=2$ and $\eps = \vp^2/2m$:
	\eq{
		\ddNvp = \frac{a}{(2\pi\hbar)^2} \exp(-\frac{\vp^2}{2mT}) \ddsp.
	}
	Integrating to find $a$,
	\eq{
		\frac{N}{V} = \frac{2\pi a}{(2\pi\hbar)^2} \int p e^{-p^2 / 2mT} \ddp
		= \frac{2\pi a}{(2\pi\hbar)^2} \frac{\Gam(1)}{2 (2mT)^{-1}}
		= \frac{2\pi a}{(2\pi\hbar)^2} mT
		\qimplies
		a = \frac{N}{V} \frac{(2\pi\hbar)^2}{2\pi mT},
	}
	which gives us
	\eq{
		\ddNvp = \frac{N}{V} \frac{e^{-\vp^2 / 2mT}}{2\pi mT} \ddsp
		\qimplies
		\ddP = \frac{e^{-\vp^2 / 2mT}}{2\pi mT} \ddsp.
	}
	Then we find
	\eq{
		\evpis = \evpxs
		= \frac{1}{2\pi mT} \intotp \cos^2\tht \ddtht \intoi p^3 e^{-\vp^2 / 2mT} \ddp
		= \frac{\pi}{2\pi mT} \frac{\Gam(2)}{2 (2mT)^{-2}}
		= m T,
	}
	\al{
		\evpiq &= \evpiis
		= \frac{1}{2\pi mT} \intotp \cos^4\tht \ddtht \intoi p^5 e^{-\vp^2 / 2mT} \ddp
		= \frac{1}{2\pi mT} \frac{3\pi}{4} \frac{\Gam(3)}{2 (2mT)^{-3}}
		= 3 m^2 T^2, \\[2ex]
		\evpijs &= \evpxys
		= \frac{1}{2\pi mT} \intotp \cos^2\tht \sin^2\tht \ddtht \intoi p^5 e^{-\vp^2 / 2mT} \ddp
		= \frac{1}{2\pi mT} \frac{\pi}{4} \frac{\Gam(3)}{2 (2mT)^{-3}}
		= m^2 T^2,
	}
	which give us
	\aln{
		\ans{\aij\ }&{\color{blue} = \begin{cases}
			1 & i = j, \\
			1 / 3 & i \neq j,
			\end{cases}} &
		\ans{\rij\ }&{\color{blue} = \begin{cases}
			1 & i = j, \\
			1 / 3 & i \neq j,
			\end{cases}} \label{nr2d}
	}
	for the two-dimensional non-relativistic gas.
	
	Clearly $\rii = 1$ and $\rij = 1/3$ ($i \neq j$) in four cases.  Thus, we see that $\rij$ has no dependence upon dimension or upon whether the particles are non- or ultra-relativistic.  It is also identically 1 when $i = j$, and seems to be related to kurtosis, which is 3 for a Gaussian distribution.  This interpretation would imply that the kurtosis is the same for the ultra-relativistic and non-relativistic distributions.
	
	In both the $d = 2$ and $d = 3$ classical cases, $\aii = 1$ and $\aij = 1/3$ ($i \neq j$) as well.  In the ultra-relativistic cases, however, this is not so; $\aii > 1$ and $\aij > 1/3$ ($i \neq j$) for both $d = 2$ and $d = 3$.  Additionally, $\aij$ (in general) is greater for $d = 2$ that for $d = 3$ in the ultra-relativistic case.  These results show that $\aij$ is related to correlations among the components of momentum.  They are uncorrelated in the non-relativistic case, but they are correlated in the ultra-relativistic case, and the correlation is greater when $d = 2$ than when $d = 3$.
}