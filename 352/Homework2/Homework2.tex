\documentclass[11pt]{article}
\usepackage{homework}

\classname{352}
\homeworknum{2}



\begin{document}

% Environments

\newcommand{\state}[2]{\begin{statement}{#1} #2 \end{statement}}
\newcommand{\prob}[2]{\begin{problem}{#1} #2 \end{problem}}
\newcommand{\subprob}[1]{\begin{subproblem} #1 \end{subproblem}}
\newcommand{\sol}[1]{\begin{solution} #1 \end{solution}}
\newcommand{\fig}[2]{\begin{figure} \centering #2  \label{#1} \end{figure}}

\newcommand{\makebib}{
	\vfill
	\color{black}
	\bibliography{references}{}
	\bibliographystyle{lucas_unsrt}
}
	

% Implication

\newcommand{\qwhere}{\quad \text{where} \quad}
\newcommand{\qimplies}{\quad \implies \quad}
\newcommand{\impliesq}{\implies \quad}



% Brackets

\newcommand{\paren}[1]{\left( #1 \right)}
\newcommand{\brac}[1]{\left[ #1 \right]}


% Greek

\newcommand{\alp}{\alpha}
\newcommand{\bet}{\beta}
\newcommand{\gam}{\gamma}
\newcommand{\del}{\delta}
\newcommand{\eps}{\epsilon}
\newcommand{\zet}{\zeta}
\newcommand{\tht}{\theta}
\newcommand{\kap}{\kappa}
\newcommand{\lam}{\lambda}
\newcommand{\sig}{\sigma}
\newcommand{\ups}{\upsilon}
\newcommand{\omg}{\omega}

\newcommand{\Gam}{\Gamma}
\newcommand{\Del}{\Delta}
\newcommand{\Tht}{\Theta}
\newcommand{\Lam}{\Lambda}
\newcommand{\Sig}{\Sigma}
\newcommand{\Omg}{\Omega}
% Problem 1

\newcommand{\Psii}{\Psi^i}
\newcommand{\Psiix}{\Psii(x)}

\newcommand{\Pii}{\Pi^i}

\newcommand{\Phii}{\Phi^i}
\newcommand{\Phiix}{\Phii(x)}
\newcommand{\PhiN}{\Phi^N}
\newcommand{\PhiNx}{\PhiN(x)}
\newcommand{\Phiq}{\Phi^1}
\newcommand{\Phiw}{\Phi^2}

\newcommand{\ddcx}{\dd[3]{x}}

\newcommand{\delij}{\del^{i j}}
\newcommand{\delkl}{\del^{k l}}
\newcommand{\delil}{\del^{i l}}
\newcommand{\deljk}{\del^{j k}}
\newcommand{\delik}{\del^{i k}}
\newcommand{\deljl}{\del^{j l}}

\newcommand{\DF}{D_F}

\newcommand{\sigx}{\sig(x)}

\newcommand{\pii}{\pi^i}
\newcommand{\pij}{\pi^j}
\newcommand{\pik}{\pi^k}
\newcommand{\pil}{\pi^l}
\newcommand{\piix}{\pi(x)}

\newcommand{\pq}{p_1}
\newcommand{\pw}{p_2}
\newcommand{\pe}{p_3}
\newcommand{\pr}{p_4}

\newcommand{\vp}{\vb{p}}
\newcommand{\vpsi}{\vp_i}

\newcommand{\mpi}{m_\pi}

\state{Thermodynamics of a relativistic gas}{\hfix}

\prob{}{
	Find the statistical distribution of a relativistic gas in momentum space, and in energies.  Discuss the relativistic corrections compared to the Maxwell distribution.
}

\prob{}{
	Now take the ultra-relativistic limit.  Find the mean energy $\evE$ and the second moment of energy $\evEs$.  Find the free energy and the entropy in the limits of high and low temperature.
}

\prob{}{
	In the non-relativistic Maxwell distribution, the different translational degrees of freedom are independent as the kinetic energy is the sum of three independent terms $K = \sum_{i=1}^3 \pii^2 / 2m$.  This is not so in the relativistic case.  For the ultra-relativistic gas compute the quantities
	\al{
		\aij &= \frac{\evpijs}{3 \evpis \evpjs}, &
		\rij &= \frac{\evpijs}{\sqrt{\evpiq \evpjq}},
	}
	in spatial dimensions $d = 2, 3$ (here $i, j$ enumerate spatial dimensions).  Compare them to the non-relativistic case.  Discuss their meaning and dependence on $d$ (at least based on $d = 2,3$).
}



\state{Collision frequency and pressure}{
	Consider an ideal relativistic gas in a container.  Given the rate of the collisions of molecules with the wall of the container per unit area per unit time, find the pressure of the gas in the relativistic, non-relativistic, and ultra-relativistic cases, and compare the results.
}




\state{Boltzmann distribution}{
	Consider an ideal gas consisting of $N$ identical one-dimensional quantum harmonic oscillators with Hamiltonian $H(p, q) = p^2 / 2m + m\omg q^2 / 2$.  Determine the total number of oscillators in states with energies $\eps \geq \epsq = \hbar\omg (\nq + 1/2)$.
}



\state{Boltzmann \texorpdfstring{$H$}{H}-function}{
	The equilibrium distribution function $f(p, q)$ of a non-interacting gas is a Maxwell-Boltzmann distribution.  Show that the entropy of such a system satisfies $S = -\kB H + \const$, where $H = \int f \ln f \dG$ is the Boltzmann $H$-function.
}



\state{BBGKY}{
	Consider for simplicity a 1D system (a system on a circle) of $N$ particles with an arbitrary two-body interaction:
	\eq{
		H = \sumiN \frac{\pii^2}{2m} + \sumi U(\xii) + \sumij V(\xii - \xjj).
	}
	Give a derivation of the first equation of the BBGKY hierarchy at equilibrium for this system, which is a relation between the 1-point and 2-point distribution (correlation) functions.
}


\state{Partition function as a generating functional}{
	Consider the Gibbs distribution of the system described in Problem~5.  For simplicity neglect the kinetic energy. Let $n(x) = \sumi \del(x - \xii)$ be the density, and $\evnx$ its expectation value. Let $C(x, y) = \ev{\del n(x) \,\del n(y)}$, where $\del n(x) = n(x) - \evn$, be the two-point correlation function.
}

\prob{}{
	Show that $\evnx = -T \,\ddvs{\ln Z}{U(x)}$, where $Z[U(x)]$ is the partition function of the Gibbs distribution treated as a functional of the potential $U$.
}

\prob{}{
	Show that
	\eq{
		C(x, y) = T^2 \ddvm{\ln Z}{U(x)}{U(y)}
		= -T \ddv{\evnx}{U(y)}
		= -T \ddv{\evny}{U(x)}.
	}
}




%\makebib

\end{document}