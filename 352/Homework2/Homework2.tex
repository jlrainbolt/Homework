\documentclass[11pt]{article}
\usepackage{homework}

\classname{352}
\homeworknum{2}



\begin{document}

% Environments

\newcommand{\state}[2]{\begin{statement}{#1} #2 \end{statement}}
\newcommand{\prob}[2]{\begin{problem}{#1} #2 \end{problem}}
\newcommand{\subprob}[1]{\begin{subproblem} #1 \end{subproblem}}
\newcommand{\sol}[1]{\begin{solution} #1 \end{solution}}
\newcommand{\fig}[2]{\begin{figure} \centering #2  \label{#1} \end{figure}}

\newcommand{\makebib}{
	\vfill
	\color{black}
	\nocite{*}
	\bibliography{references}{}
	\bibliographystyle{lucas_unsrt}
}
	

% Implication

\newcommand{\qwhere}{\quad \text{where} \quad}
\newcommand{\qimplies}{\quad \implies \quad}
\newcommand{\impliesq}{\implies \quad}



% Brackets

\newcommand{\paren}[1]{\left( #1 \right)}
\newcommand{\brac}[1]{\left[ #1 \right]}
\newcommand{\curly}[1]{\left\{ #1 \right\}}


% Greek

\newcommand{\alp}{\alpha}
\newcommand{\bet}{\beta}
\newcommand{\gam}{\gamma}
\newcommand{\del}{\delta}
\newcommand{\eps}{\epsilon}
\newcommand{\zet}{\zeta}
\newcommand{\tht}{\theta}
\newcommand{\kap}{\kappa}
\newcommand{\lam}{\lambda}
\newcommand{\sig}{\sigma}
\newcommand{\ups}{\upsilon}
\newcommand{\omg}{\omega}

\newcommand{\Gam}{\Gamma}
\newcommand{\Del}{\Delta}
\newcommand{\Tht}{\Theta}
\newcommand{\Lam}{\Lambda}
\newcommand{\Sig}{\Sigma}
\newcommand{\Omg}{\Omega}


% Text

\newcommand{\where}{\text{where }}

% Problem 1

\newcommand{\Hint}{H_\text{int}}
\newcommand{\ddcx}{\dd[3]{x}}
\newcommand{\psib}{\bar{\psi}}

\newcommand{\mh}{m_h}
\newcommand{\mmu}{m_\mu}
\newcommand{\me}{m_e}
\newcommand{\ma}{m_a}

\newcommand{\aexpt}{a_\text{expt.}}
\newcommand{\aQED}{a_\text{QED}}
\renewcommand{\GeV}{\giga\electronvolt}

\newcommand{\gamt}{\gam^5}

\state{Spin-wave theory~(P\&S 11.1)}{\hfix}

\prob{ \label{1a}
	Prove the following wonderful formula: Let $\phix$ be a free scalar field with propagator $\ev{T \phix \phio} = \Dx$.  Then
	\eqn{show1}{
		\ev{ T e^{i \phix} e^{-i \phio} } = e^{[ \Dx - \Do ]}.
	}
	(The  factor $\Do$ gives a formally divergent adjustment of the overall normalization.)
}

\sol{
	According to P\&S~(9.18),
	\eq{
		\ev*{T \phi(\xq) \phi(\xw)}{\Omg} = \frac{\int \DDphi \phi(\xq) \phi(\xw) \exp[ i \int \ddqx \cL ]}{\int \DDphi \exp[ i \int \ddqx \cL ]}.
	}
	We use this expression to write the left-hand side of Eq.~\refeq{show1}:
	\eqn{thing1}{
		\ev{ T e^{i \phix} e^{-i \phio} } = \frac{\int \DDphi e^{i \phix} e^{-i \phio} \exp[ i \int \ddqy \cL ]}{\int \DDphi \exp[ i \int \ddqy \cL ]}
		= \frac{\int \DDphi \exp[i \phix - i \phio + i \int \ddqy \cL ]}{\int \DDphi \exp[ i \int \ddqy \cL ]}.
	}
	For a free Klein-Gordon~(i.e., scalar) field, Eq.~(9.39) tells us that the generating functional $\ZJ$ is given by
	\eq{
		\ZJ = \Zo \exp[ -\frac{1}{2} \int \ddqx \ddqy \Jx \DF(x - y) \Jy ],
	}
	where $\Zo = Z[0]$.  Thus, we want to find some $\Jy$ such that
	\eqn{thing1b}{
		\ev{ T e^{i \phix} e^{-i \phio} } = \frac{\ZJ}{\Zo}
	}
	where in general
	\eq{
		\ZJ = \int \DDphi \exp[ i \int \ddqx [ \cL + \Jx \phi(x) ] ]
	}
	by (9.34).  Inspecting Eq.~\refeq{thing1}, we recognize the denominator as $\Zo$ and see that if
	\eq{
		\Jy = \delq(y - x) - \delq(y)
	}
	we have an expression like Eq.~\refeq{thing1b}.  Collecting these findings, we have
	\al{
		\ans{ \ev{ T e^{i \phix} e^{-i \phio} } }&= \frac{\ZJ}{\Zo} \\
		&= \exp[ -\frac{1}{2} \int \ddqy \ddqz \Jy \DF(y - z) \Jz ] \\
		&= \exp[ -\frac{1}{2} \int \ddqy \ddqz \Jy \DF(y - z) [ \delq(z - x) - \delq(z) ] ] \\
		&= \exp[ -\frac{1}{2} \int \ddqy [ \delq(y - x) - \delq(y) ] [ \DF(y - x) - \DF(y) ] ] \\
		&= \exp[ -\frac{1}{2} [ \DF(0) - \DF(x) - \DF(-x) + \DF(0) ] ] \\
		&= \exp[ \DF(x) - \DF(0) ] \\
		&\ans{\; = e^{[ \Dx - \Do ]}, }
	}
	as we wanted to show. \qed
}



\prob{ \label{1b}
	We can use this formula in Euclidean field theory to discuss correlation functions in a theory with spontaneously broken symmetry for $T < \TC$.  Let us consider only the simplest case of a broken $O(2)$ or $U(1)$ symmetry.  We can write the local spin density as a complex variable
	\eq{
		\sx = \sqx + i \swx.
	}
	The global symmetry is the transformation
	\eq{
		\sx \to e^{-i \alp} \sx.
	}
	If we assume that the physics freezes the modulus of $\sx$, we can parameterize
	\eqn{sx}{
		\sx = A e^{i \phix}
	}
	and write an effective Lagrangian for the field $\phix$.  The symmetry of the theory becomes the translation symmetry
	\eqn{symmetry}{
		\phix \to \phix - \alp.
	}
	Show that (for $d > 0$) the most general renormalizable Lagrangian consistent with this symmetry is the free field theory
	\eqn{show1b}{
		\cL = \frac{1}{2} \rho(\vgrad \phi)^2.
	}
	In statistical mechanics, the constant $\rho$ is called the \emph{spin wave modulus}.  A reasonable hypothesis for $\rho$ is that it is finite for $T < \TC$ and tends to 0 as $T \to \TC$ from below.
}

\sol{
	In accordance with the Klein-Gordon Lagrangian in P\&S~(2.6),
	\eqn{KGL}{
		\cL_\text{K-G} = \frac{1}{2} (\pt \phi)^2 - \frac{1}{2} m^2 \phi^2,
	}
	we interpret $(\vgrad \phi)^2$ as $(\pt \phi)^2$.
	
	The Lagrangian cannot have terms of $\order{\phi^n}$ for any $n \neq 0$ since $\phi(x)$ is not invariant under Eq.~\refeq{symmetry}.  Any combination of derivatives of $\phi$ is invariant, however, since $\alp$ is a constant and does not contribute to any derivative.  Thus, only terms like $\pt^n \phi^m$ (where $n$ denotes a power of $\pt$) for $n, m > 0$ and $n \geq m$ are consistent with the symmetry of Eq.~\refeq{symmetry} for $d$ an integer.
	
	Now we must determine which of these terms are renormalizable.  We know that the Lagrangian must have dimension $d$, and that $\phi$ has dimension $(d - 2) / 2$.  Taking a derivative adds a mass dimension.  The theory is renormalizable if the coupling constant $\rho$ has dimension greater than or equal to 0~\cite[p.~322]{Peskin}.  Let $p$ be the dimension of $\rho$.  The dimension of our allowed term is then
	\eq{
		[ \rho \pt^n \phi^m ] = p + n + m \frac{d - 2}{2},
	}
	which we require to be equal to $d$.  Thus we seek solutions to the system of equations
	\al{
		d &= p + n + m \frac{d - 2}{2}, &
		n &\geq m, &
		p &\geq 0.
	}
	Solving with Mathematica, we find that this system has two solutions: $n = m = 2$ and $p = 0$; and $n = m = 1$ and $p = d / 2$.  However, the term $\pt \phi$ for $n = m = 1$ does not contribute to the action because it is a total derivative and does not contribute when the integral over $\cL$ is evaluated:
	\eq{
		\int \dd[d]{x} \pt\phi = \phi \bigg|_{-\infty}^\infty
		= 0.
	}
	Thus the only possibility is $n = m = 2$.  Note that
	\eq{
		\pt^2 \phi^2 = \pt(\pt \phi^2)
		= 2 \pt( \phi \pt \phi)
		= \pt \phi \pt \phi + \phi \pt^2 \phi
		= (\pt \phi)^2,
	}
	since $\phi \pt^2 \phi$ is not invariant under Eq.~\refeq{sx}.  This means that $\rho$ must be dimensionless and that the only allowed terms in the Lagrangian are proportional to $(\pt \phi)^2$, which is consistent with Eq.~\refeq{show1b}. \qed
}



\prob{
	Compute the correlation function $\ev{ \sx \sao }$.  Adjust $A$ to give a physically sensible normalization (assuming that the system has a physical cutoff at the scale of one atomic spacing) and display the dependence of this correlation function on $x$ for $d = 1, 2, 3, 4$.  Explain the significance of your results.
}

\sol{
	Applying Eq.~\refeq{sx},
	\eq{
		\ev{ \sx \sao } = \ev*{ A e^{i \phix} \As e^{-i \phio} }
		= \ev*{ \abs{A}^2 } \ev*{ e^{i \phix} e^{-i \phio} }.
	}
	Now we can apply Eq.~\refeq{show1} to find
	\eqn{thing1c}{
		\ans{ \ev{ \sx \sao } = \abs{A}^2 \exp[ D(x) - D(0) ], }
	}
	where $D(x - y)$ is a Green's function.  Since our Lagrangian is similar to the Klein-Gordon Lagrangian Eq.~\refeq{2.6}, our Green's function is similar to that of the Klein-Gordon operator, which is given by P\&S~(2.56):
	\eq{
		(\pt^2 + m^2) D(x - y) = -i \delq(x - y).
	}
	The Feynman prescription for this Green's function is given by (2.59),
	\eqn{DF}{
		\DF(x - y) = \int \ddqpf \frac{i}{p^2 - m^2 + i \eps} e^{-i p \cdot (x - y)}.
	}
	For the Lagrangian in Eq.~\refeq{show1b}, we set $m = 0$ and insert a factor of $\rho$:
	\eq{
		\rho \pt^2 D(x - y) = -i \deld(x - y),
	}
	so adapting Eq.~\refeq{DF} for this situation yields
	\eqn{DF}{
		\DF(x - y) = \frac{1}{\rho} \int \dddpf \frac{i}{p^2 + i \eps} e^{-i p \cdot (x - y)}.
	}
	We see that $\DF(0)$ diverges, so we absorb it into the constant to make the normalization physically sensible.  We can do this because, as we showed in \ref{1b}, the theory is renormalizable.  Define $A'$ such that
	\eq{
		{A'}^2 = \abs{A}^2 e^{-D(0)}.
	}
	Then Eq.~\refeq{thing1c} can be written
	\eq{
		\ans{ \ev{ \sx \sao } =  {A'}^2 e^{D(x)}. }
	}
	
	To evaluate the divergent integral $D(x)$, we look to the Feynman parameter method we have been using to solve divergent integrals.  Apparently, the Schwinger parametrization is useful in deriving the Feynman parametrization, and it is given by~\cite{Feynman}
	\eq{
		\frac{1}{A} = \intoi \dds e^{-s A}.
	}
	Using this equation, we can write Eq.~\refeq{DF} as
	\eq{
		\DF(x) = \frac{1}{\rho} \int \dddpf \frac{i}{p^2} e^{-i p \cdot x}
		= \frac{i}{\rho} \int \dddpf \intoi \dds e^{-s p^2} e^{-i p \cdot x}.
	}
	Now we can complete the square in the exponential to get a Gaussian integral:
	\al{
		\DF(x) &= \frac{i}{\rho} \int \dddpf \intoi \dds \exp[ -s p^2 - i p \cdot x + \frac{x^2}{4 s} - \frac{x^2}{4 s} ] \\
		&= \frac{i}{\rho} \int \dddpf \intoi \dds \exp[ -s \paren{ p + \frac{i x}{2 s} }^2 - \frac{x^2}{4 s} ] \\
		&= \frac{i}{\rho (2 \pi)^d} \intoi \dds e^{-x^2 / 4 s} \int \dd[d]{u} e^{-s u^2} \\
		&= \frac{i}{\rho (2 \pi)^{d}} \intoi \dds e^{-x^2 / 4 s} \sqrt{ \frac{(2\pi)^d}{(2s)^d} } \\
		&= \frac{i}{\rho (4 \pi)^{d / 2}} \intoi \dds \frac{e^{-x^2 / 4 s}}{s^{d / 2}}
	}
	where we have used~\cite{QFT}
	\eq{
		\int \exp( -\frac{1}{2} x \cdot A \cdot x ) \dd[n]{x} = \sqrt{\frac{(2\pi)^n}{\det A}},
	}
	with $A$ a $d \times d$ diagonal matrix $2s$.  Using Mathematica to integrate with respect to $s$, we find
	\eq{
		\DF(x) = \frac{i}{\rho (4 \pi)^{d / 2}} \frac{2^{d - 2}}{x^{d - 2}} \Gam(d / 2 - 1)
		= \frac{i}{4 \pi^d \rho} \Gam(d / 2 - 1) x^{2 - d}.
	}
	The gamma function diverges as $d \to 2$, so as we have done in previous problems, we expand about $\eps = 2 - d$.  Evaluating the series expansion using Mathematica, we obtain
	\eq{
		\DF(x) = \frac{i}{4 \pi^{1 - \eps} \rho} \Gam(\eps / 2) x^\eps
		\approx \frac{i}{4 \pi \rho} \paren{ \frac{2}{\eps} - \gam + 2 \ln(\pi x) }
		\sim \frac{i}{2 \pi \rho} \ln(x)
		= i \ln(\frac{1}{x^{2 \pi \rho}}).
	}
	We Wick rotate $x \to i x$.  Then the dependence of the correlation function on $x$ for $d = 1, 2, 3, 4$ is
	\ans{\al{
		(d = 1) &\qquad \ev{ \sx \sao } \sim e^{-x / 2 \sqrt{\pi} \rho}, &
		(d = 2) &\qquad \ev{ \sx \sao } \sim x^{2 \pi \rho}, \\
		(d = 3) &\qquad \ev{ \sx \sao } \sim \frac{1}{x}, &
		(d = 4) &\qquad \ev{ \sx \sao } \sim \frac{1}{x^2}.
	}}%
	In $d > 2$ dimensions, the expectation value of the correlation function tends to 0 at large distances $x$.  For $d > 2$, it drops off more quickly as $d$ increases.  The $d \leq 2$ cases depend on $\rho$, which we assume is positive.  The $d = 1$ case drops off with increasing distance, and more quickly with smaller $\rho$.  For $d = 2$, the expectation value of the correlation function increases with increasing distance, and it blows up more quickly with larger $\rho$.
	
	These results are consistent with the Mermin--Wagner theorem, which states that a continuous symmetry cannot be broken in $d \leq 2$ dimensions~\cite{CMW}.  That is, in $d \leq 2$ dimensions, a symmetry-breaking field cannot have a nonzero vacuum expectation value~\cite[p.~460]{Peskin}.  A physical explanation is that each spin has more nearest neighbors in higher dimensions.  Since the spins are inclined to align with their neighbors, there is a higher degree of correlation in higher dimensions at the same distance.  In two dimensions, the correlations are weak enough that they are overpowered by the field fluctuations.
}


\state{Collision frequency and pressure}{
	Consider an ideal relativistic gas in a container.  Given the rate of the collisions of molecules with the wall of the container per unit area per unit time, find the pressure of the gas in the relativistic, non-relativistic, and ultra-relativistic cases, and compare the results.
}

\sol{
	We will consider particles colliding with a wall located on the $yz$ plane.  The number of particles colliding with an area $A$ of this wall in a time $\delt$ is given by
	\eq{
		\ddcNvp = A \vx \,\delt \ddNvp,
	}
	where $\vx$ is velocity in the $x$ direction and $\ddNvp$ is the distribution of the number of particles in momentum space.  This expression indicates that a particle must be a distance of no more than $\vx \,\delt$ from the wall in order to collide with it during the time $\delt$~\cite[p.~77]{Kardar}.
	
	Each particle that collides with the wall transfers $2\px$ of momentum to it.  Only particles moving toward~(rather than away from) the wall can hit it, so we must integrate $\px$ from $-\infty$ to $0$.  However, on average half of the particles have $\px < 0$, meaning that
	\eq{
		\intio \px \ddpx = \frac{1}{2} \intii \px \ddpx.
	}
	The net force exerted by all of the particles is the change in the total momentum, $P$, which we can now write as
	\eq{
		F = \frac{\del P}{\delt}
		= \frac{1}{2 \delt} \int 2\px \ddcNvp
		= A \int \vx \px \ddNvp,
	}
	where the integral is over all of momentum space~\cite[p.~77]{Kardar}.  Then the pressure is simply the force per unit area:
	\eqn{pressure}{
		P = \frac{F}{A} = \int \vx \px \ddNvp.
	}
	
	In the relativistic case, $\ddNvp$ is given by Eq.~\refeq{ddNvp} and
	\eq{
		\vx = \frac{c \px}{\sqrt{m^2 c^2 + \vp^2}},
	}
	so Eq.~\refeq{pressure} becomes
	\al{
		P &= \frac{N}{V} \frac{1}{{4\pi T m^2 c \,K_2(\bet mc^2)}} \int \frac{c \px^2}{\sqrt{m^2 c^2 + \vp^2}} \exp(-\bet c \sqrt{m^2 c^2 + \vp^2}) \ddcp \\
%		&= \frac{N}{V} \frac{1}{{4\pi T m^2 c \,K_2(\bet mc^2)}} \intotp \intopi \intoi \frac{c p^2 \cos^2\phi \sin^2\tht}{\sqrt{m^2 c^2 + p^2}} \exp(-\bet c \sqrt{m^2 c^2 + p^2}) p^2 \sin\tht \ddp \ddtht \ddphi \\
		&= \frac{N}{V} \frac{1}{{4\pi T m^2 c \,K_2(\bet mc^2)}} \intotp \cos^2\phi \ddphi \intopi \sin^3\tht \ddtht \intoi \frac{c p^4}{\sqrt{m^2 c^2 + p^2}} \exp(-\bet c \sqrt{m^2 c^2 + p^2}) \ddp \\
		&= \frac{N}{V} \frac{1}{{4\pi T m^2 c \,K_2(\bet mc^2)}} \frac{4\pi}{3} \intoi \frac{c p^4}{\sqrt{m^2 c^2 + p^2}} \exp(-\bet c \sqrt{m^2 c^2 + p^2}) \ddp.
	}
	Note that $\eps = c \sqrt{m^2 c^2 + p^2}$, and that
	\eq{
		\eps^2 = m^2 c^4 + c^2 p^2
		\qimplies \eps \dde = p c^2 \ddp
		\qimplies
		\ddp = \frac{\eps}{p c^2} \dde.
	}
	Making this substitution, we find
	\eq{
		P = \frac{N}{V} \frac{1}{{4\pi T m^2 c \,K_2(\bet mc^2)}} \frac{4\pi}{3} \intmcsi \frac{c^2 p^4}{\eps} e^{-\bet \eps} \frac{\eps}{p c^2} \dde
		= \frac{N}{V} \frac{1}{{4\pi T m^2 c \,K_2(\bet mc^2)}} \frac{4\pi}{3} \frac{1}{c^3} \intmcsi (\eps^2 - m^2 c^4)^{3/2} e^{-\bet \eps} \dde.
	}
	Using the integral formula~\cite[p.~350]{Integrals}
	\eq{
		\intui (x^2 - u^2)^{\nu - 1} e^{-\mu x} \ddx = \frac{1}{\sqrt{\pi}} \paren{ \frac{2u}{\mu} }^{\nu - 1/2} \Gam(\nu) \,K_{\nu - 1/2}(u\ mu),
	}
	where $u > 0$, $\Re(\mu), \Re(\nu) > 0$, we see that $x \to \eps$, $u \to mc^2$, $\nu \to 5/2$, $\mu \to \bet$.  Noting that $\Gam(5/2) = 3 \sqrt{\pi} / 4$, We find
	\eq{
		P = \frac{N}{V} \frac{1}{{4\pi T m^2 c \,K_2(\bet mc^2)}} \frac{1}{c^3} \frac{4\pi}{3} 4 m^2 c^4 T^2 \frac{3}{4} \,K_2(\bet m c^2)
		= \ans{ \frac{N T}{V}, }
	}
	and so we have recovered the equation of state $P V = N T$ in the relativistic case.
	
	In the non-relativistic case, $\vx = \px / m$ and $\ddNvp$ is given by the Maxwell distribution in Eq.~\refeq{Maxwellp}.  So Eq.~\refeq{pressure} becomes in this case
	\al{
		P &= \frac{N}{V} \frac{1}{(2\pi m T)^{3/2}} \int \frac{\px^2}{2} \exp(-\frac{\vp^2}{2m T}) \ddcp \\
		&= \frac{N}{V} \frac{1}{2 (2\pi m T)^{3/2}} \intotp \intopi \intoi p^2 \cos^2\phi \sin^2\tht \exp(-\frac{p^2}{2m T}) p^2 \sin\tht \ddp \ddtht \ddphi \\
		&= \frac{N}{V} \frac{1}{2 (2\pi m T)^{3/2}} \intotp \cos^2\phi \ddphi \intopi \sin^3\tht \ddtht \intoi p^4 \exp(-\frac{p^2}{2m T}) \ddp \\
		&= \frac{N}{V} \frac{1}{2 (2\pi m T)^{3/2}} \frac{4 \pi}{3} \frac{\Gam(5/2)}{2 (2mT)^{-5/2}}
		= \frac{N}{V} \frac{1}{2 (2\pi m T)^{3/2}} \frac{4 \pi}{3} \frac{3 \sqrt{\pi}}{4} \frac{(2mT)^{5/2}}{2}
		= \ans{ \frac{NV}{T}, }
	}
	where we have used Eq.~\refeq{gaussian}.  So we have once again recovered the equation of state.
	
}



\clearpage
\newcommand{\tht}{\theta}
\newcommand{\thtxx}{\tht_{xx}}
\newcommand{\thtyy}{\tht_{yy}}
\newcommand{\thttt}{\tht_{tt}}
\newcommand{\thtt}{\tht_t}
\newcommand{\thtx}{\tht_x}
\newcommand{\thty}{\tht_y}
\newcommand{\sint}{\sin{\tht}}
\newcommand{\dxdydt}{\dxdy \dd{t}}

\begin{statement}{}
	The nondimensionalized, multidimensional Sine-Gordon equation,
	\beq
		\thtxx + \thtyy - \thttt = \sint
	\eeq
	for $\tht(x, y, t)$, is the Euler-Lagrange equation for the action integral
	\beq
		S[\tht] = \intR \left\{ \frac{1}{2} \left[ \thtt^2 - (\nabla\tht)^2 \right] - \sint \right\} \dxdydt
	\eeq
	with $\nabla\tht = (\pdv*{\tht}{x}, \pdv*{\tht}{y})$.  The functional $S[\tht]$ is invariant under translation of $x$, $y$, and $t$.  Find the associated energy-momentum tensor and energy-momentum vector.
\end{statement}

\begin{solution}
	Expanding out $(\nabla\tht)^2$, the Lagrangian density is
	\beqn \label{lagr3}
		\Ld = \frac{1}{2} (\thtt^2 - \thtx^2 - \thty^2) - \sint.
	\eeqn
	The energy-momentum tensor is defined by
	\beq
		T_{ij} = \pdv{\Ld}{\tht_{x_i}} \pdv{\tht}{x_j} - \Ld \, \delta_{ij},
	\eeq
	where $x_i \in \{ x_0, x_1, x_2 \} = \{ t, x, y \}$.  The diagonal elements of $T$ are then
	\begin{align*}
		T_{00} &= \pdv{\Ld}{\thtt} \pdv{\tht}{t} - \Ld = \thtt^2 - \frac{1}{2} (\thtt^2 - \thtx^2 - \thty^2) + \sint = \frac{1}{2} (\thtt^2 + \thtx^2 + \thty^2) + \sint, \\
		T_{11} &= \pdv{\Ld}{\thtx} \pdv{\tht}{x} - \Ld = -\thtx^2 - \frac{1}{2} (\thtt^2 - \thtx^2 - \thty^2) + \sint = -\frac{1}{2} (\thtt^2 + \thtx^2 - \thty^2) + \sint, \\
		T_{22} &= \pdv{\Ld}{\thty} \pdv{\tht}{y} - \Ld = -\thty^2 - \frac{1}{2} (\thtt^2 - \thtx^2 - \thty^2) + \sint = -\frac{1}{2} (\thtt^2 - \thtx^2 + \thty^2) + \sint,
	\end{align*}
	and the nondiagonal elements are
	\begin{align*}
		T_{01} &= \pdv{\Ld}{\thtt} \pdv{\tht}{x} = \thtt \thtx, &
		T_{02} &= \pdv{\Ld}{\thtt} \pdv{\tht}{y} = \thtt \thty, &
		T_{12} &= \pdv{\Ld}{\thtx} \pdv{\tht}{y} = -\thtx \thty, \\
		T_{10} &= \pdv{\Ld}{\thtx} \pdv{\tht}{t} = -\thtt \thtx, &
		T_{20} &= \pdv{\Ld}{\thty} \pdv{\tht}{t} = -\thtt \thty, &
		T_{21} &= \pdv{\Ld}{\thty} \pdv{\tht}{x} = -\thtx \thty.
	\end{align*}
	In matrix form, we have
	\beq
		T = \mqty[(\thtt^2 + \thtx^2 + \thty^2) / 2 + \sint & \thtt \thtx & \thtt \thty \\
				-\thtt \thtx & -(\thtt^2 + \thtx^2 - \thty^2) / 2 + \sint & -\thtx \thty \\
				-\thtt \thty & -\thtx \thty & -(\thtt^2 - \thtx^2 + \thty^2) / 2 + \sint ].
	\eeq
	The energy-momentum vector is defined by
	\beq
		P_j = \int T_{0j} \dd{x_1} \dd{x_2}.
	\eeq
	Its components are then
	\begin{align*}
		P_0 &= \int \left[ \frac{1}{2} (\thtt^2 + \thtx^2 + \thty^2) + \sint \right] \dxdy, &
		P_1 &= \int \thtt \thtx \dxdy, &
		P_2 &= \int \thtt \thty \dxdy.
	\end{align*}
\vfix
\end{solution}


\clearpage
\state{Boltzmann \texorpdfstring{$H$}{H}-function}{
	The equilibrium distribution function $f(p, q)$ of a non-interacting gas is a Maxwell-Boltzmann distribution.  Show that the entropy of such a system satisfies $S = -\kB H + \const$, where $H = \int f \ln f \dG$ is the Boltzmann $H$-function.
}



\state{BBGKY}{
	Consider for simplicity a 1D system (a system on a circle) of $N$ particles with an arbitrary two-body interaction:
	\eq{
		H = \sumiN \frac{\pii^2}{2m} + \sumi U(\xii) + \sumij V(\xii - \xjj).
	}
	Give a derivation of the first equation of the BBGKY hierarchy at equilibrium for this system, which is a relation between the 1-point and 2-point distribution (correlation) functions.
}


\state{Partition function as a generating functional}{
	Consider the Gibbs distribution of the system described in Problem~5.  For simplicity neglect the kinetic energy. Let $n(x) = \sumi \del(x - \xii)$ be the density, and $\evnx$ its expectation value. Let $C(x, y) = \ev{\del n(x) \,\del n(y)}$, where $\del n(x) = n(x) - \evn$, be the two-point correlation function.
}

\prob{}{
	Show that $\evnx = -T \,\deldvs{\ln Z}{U(x)}$, where $Z[U(x)]$ is the partition function of the Gibbs distribution treated as a functional of the potential $U$.
}

\prob{}{
	Show that
	\eq{
		C(x, y) = T^2 \deldvm{\ln Z}{U(x)}{U(y)}
		= -T \deldv{\evnx}{U(y)}
		= -T \deldv{\evny}{U(x)}.
	}
}




%\makebib

\end{document}