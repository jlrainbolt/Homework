\documentclass[11pt]{article}
\usepackage{homework}

\classname{352}
\homeworknum{2}



\begin{document}

% Environments

\newcommand{\state}[2]{\begin{statement}{#1} #2 \end{statement}}
\newcommand{\prob}[2]{\begin{problem}{#1} #2 \end{problem}}
\newcommand{\subprob}[1]{\begin{subproblem} #1 \end{subproblem}}
\newcommand{\sol}[1]{\begin{solution} #1 \end{solution}}
\newcommand{\fig}[2]{\begin{figure} \centering #2  \label{#1} \end{figure}}

\newcommand{\makebib}{
	\vfill
	\color{black}
	\nocite{*}
	\bibliography{references}{}
	\bibliographystyle{lucas_unsrt}
}
	

% Implication

\newcommand{\qwhere}{\quad \text{where} \quad}
\newcommand{\qimplies}{\quad \implies \quad}
\newcommand{\impliesq}{\implies \quad}



% Brackets

\newcommand{\paren}[1]{\left( #1 \right)}
\newcommand{\brac}[1]{\left[ #1 \right]}
\newcommand{\curly}[1]{\left\{ #1 \right\}}


% Greek

\newcommand{\alp}{\alpha}
\newcommand{\bet}{\beta}
\newcommand{\gam}{\gamma}
\newcommand{\del}{\delta}
\newcommand{\eps}{\epsilon}
\newcommand{\zet}{\zeta}
\newcommand{\tht}{\theta}
\newcommand{\kap}{\kappa}
\newcommand{\lam}{\lambda}
\newcommand{\sig}{\sigma}
\newcommand{\ups}{\upsilon}
\newcommand{\omg}{\omega}

\newcommand{\Gam}{\Gamma}
\newcommand{\Del}{\Delta}
\newcommand{\Tht}{\Theta}
\newcommand{\Lam}{\Lambda}
\newcommand{\Sig}{\Sigma}
\newcommand{\Omg}{\Omega}


% Text

\newcommand{\where}{\text{where }}

% Problem 1

\newcommand{\Hint}{H_\text{int}}
\newcommand{\ddcx}{\dd[3]{x}}
\newcommand{\psib}{\bar{\psi}}

\newcommand{\mh}{m_h}
\newcommand{\mmu}{m_\mu}
\newcommand{\me}{m_e}
\newcommand{\ma}{m_a}

\newcommand{\aexpt}{a_\text{expt.}}
\newcommand{\aQED}{a_\text{QED}}
\renewcommand{\GeV}{\giga\electronvolt}

\newcommand{\gamt}{\gam^5}

\state{Thermodynamics of a relativistic gas}{\hfix}

%
%	1.1
%

\prob{}{
	Find the statistical distribution of a relativistic gas in momentum space, and in energies.  Discuss the relativistic corrections compared to the Maxwell distribution.
}

\sol{
	We will use the Boltzmann distribution for an ideal gas in the classical limit.  The distribution of the density of states in phase space is
	\eq{
		\npq = a \exp(-\frac{\epspq}{T}),
	}
	where $\npq$ is the mean number of molecules of energy $\epspq$ in a phase space volume element $\ddp \ddq$.  Here $a$ is a normalization constant, determined by normalizing to $N / V$ particles per unit volume, where $N$ is the total number of gas molecules and $V$ is the total volume.  The mean number of molecules contained in a single volume element is
	\eq{
		\ddN = \frac{n(p, q)}{(2\pi\hbar)^r} \ddp \ddq,
	}
	where $r$ is the number of translational degrees of freedom~\cite[p.~107--108]{Landau}.  We assume $r = 3$.
	
	The energy of a single relativistic particle is $\eps = c \sqrt{m^2 c^2 + \vp^2}$, where $m$ is its mass, $\vp$ its three-dimensional momentum, and $c$ the speed of light~\cite[p.~110]{Landau}.  This gives us
	\eqn{dNp}{
		\ddNvp = \frac{a}{(2\pi\hbar)^3} \exp(-\frac{c \sqrt{m^2 c^2 + \vp^2}}{T}) \ddcp,
	} 
	where we are ignoring the coordinate-space volume $\ddq$, because it would disappear anyway upon normalization.
	
	Now we must find $a$ by integrating over all of momentum space, which we will carry out  using spherical coordinates with $\ddcp = p^2 \sin\tht \ddp \ddtht \ddphi$.  We find
	\eqn{thingy2}{
		\frac{N}{V} = \iiint \ddN
		= \frac{4\pi a}{(2\pi\hbar)^3} \intoi p^2 \exp(-\frac{c \sqrt{m^2 c^2 + p^2}}{T}) \ddp.
	}
	Let $u = \sqrt{m^2 c^2 + p^2}$.  Then the lower bound of integration for $u$ is $m c$, and
	\eq{
		\dv{u}{p} = \frac{p}{\sqrt{m^2 c^2 + p^2}} 
		 = \frac{\sqrt{u^2 - m^2 c^2}}{u}
		 \qimplies
		 \ddp = \frac{u}{\sqrt{u^2 - m^2 c^2}} \ddu.
	}
	Then we have
	\eqn{thingy}{
		\frac{N}{V} = \frac{4\pi a}{(2\pi\hbar)^3} \intmci u \sqrt{u^2 - m^2 c^2} e^{-c u / T} \ddu.
	}
	Note that~\cite[p.~351]{Integrals}
	\eqn{integral}{
		\int_u^\infty x (x^2 - u^2)^{\nu - 1} e^{-\mu x} \dd{x} = \frac{2^{\nu - 1/2}}{\sqrt{\pi}} \mu^{1/2 - \nu} u^{\nu + 1/2} \,\Gam(\nu) \,K_{\nu + 1/2}(u\mu)
	}
	for $\Re(u \mu) > 0$, where $\Gam(z)$ is the Gamma function and $K_n(z)$ is a modified Bessel function of the second kind~\cite[p.~175]{Pathria}.  Comparing with Eq.~\refeq{thingy}, we have $x \to u$, $u \to mc$, $\nu \to 3/2$, and $\mu \to c/T$.  Note also that $\Gam(3/2) = \sqrt{\pi} / 2$~\cite{Gamma}.  Then, evaluating Eq.~\refeq{thingy},
	\eq{
		\frac{N}{V} = \frac{4\pi a}{(2\pi\hbar)^3} T m^2 c \,K_2(\bet mc^2)
		\qimplies
		a = \frac{N}{V} \frac{(2\pi\hbar)^3}{4\pi} \frac{1}{T m^2 c \,K_2(\bet mc^2)}.
	}
	Substituting into Eq.~\refeq{dNp}, we obtain
	\eqn{1.1ans}{
		\ans{ \ddNvp = \frac{N}{V} \frac{\exp(-\bet c \sqrt{m^2 c^2 + \vp^2})}{4\pi T m^2 c \,K_2(\bet mc^2)} \ddcp }
	}
	as the statistical distribution in momentum space.
	
	To find the distribution in energy space, we will change variables in Eq.~\refeq{dNp} to $\eps = c \sqrt{m^2 c^2 + \vp^2}$.  Noting that
	\eq{
		\dv{p}{\eps} = \frac{c p}{\sqrt{m^2 c^2 + p^2}}
		\qimplies
		\ddp = \frac{\eps}{c^2} \sqrt{\eps^2 / c^2 - m^2 c^2}
		= \frac{\eps}{c^3} \sqrt{\eps^2 - m^2 c^4},
	}
	we have
	\eq{
		\ddNeps = \frac{4\pi b}{(2\pi\hbar)^3} \frac{1}{c^3} \eps \sqrt{\eps^2 - m^2 c^4} e^{-\eps / T} \dde,
	}
	where $b$ is a normalization constant,which we will find by integration:
	\eq{
		\frac{N}{V} = \frac{4\pi b}{(2\pi\hbar)^3} \frac{1}{c^3} \int_{m c^2}^\infty \eps \sqrt{\eps^2 - m^2 c^4} e^{-\eps / T} \dde
	}
	Again comparing to Eq.~\refeq{integral}, we have $x \to \eps$, $u \to mc^2$, $\nu \to 3/2$, and $\mu \to \bet$.  This gives us
	\eq{
		\frac{N}{V} = \frac{4\pi b}{(2\pi\hbar)^3} T m^2 c \,K_{2}(\bet m c^2)
		\qimplies b = a,
	}
	so the statistical distribution in energy space is
	\eqn{1.1ans2}{
		\ans{ \ddNeps = \frac{N}{V} \frac{e^{-\bet \eps} \eps \sqrt{\eps^2 - m^2 c^4}}{T m^2 c^4 \,K_2(\bet mc^2)} \dde. }
	}

	
	The Maxwell distribution in momentum space is~\cite[p.~109]{Landau}
	\eqn{Maxwellp}{
		\ddNvp = \frac{N}{V (2\pi m T)^{3/2}} \exp(-\frac{\px^2 + \py^2 + \pz^2}{2m T}) \ddpxyz.
	}
	From p.~2 of Lecture 4, the Maxwell distribution in energy space is
	\eqn{Maxwelle}{
		\ddNeps = \frac{N}{V} \frac{2}{\sqrt{\pi T^3}} e^{-\eps / T} \sqrt{\eps} \dde.
	}
	
	Both distributions are similar to the relativistic ones in Eqs.~(\ref{1.1ans}--\ref{1.1ans2}).  The Maxwell distributions have the kinetic energy $\eps = \vp^2 / 2m$ in the exponent, whereas Eqs.~(\ref{1.1ans}--\ref{1.1ans2}) have the relativistic energy $\eps = c \sqrt{m^2 c^2 + \vp^2}$.  The factor of $\bet$ in the exponent is the same in both cases.  	However, Eq.~\refeq{1.1ans2} goes as $e^{-\bet \eps} \eps^2$ while Eq.~\refeq{Maxwelle} goes as $e^{-\bet \eps} \sqrt{\eps}$.
	
	The normalization of Eqs.~(\ref{1.1ans}--\ref{1.1ans2}) is different than that of Eqs.~(\ref{Maxwellp}--\ref{Maxwelle}) in order to account for the relativistic energy.  The factor of $1/K_2{\bet mc^2}$ means that the relativistic ``occupation number densities'' fall off much more rapidly with $T$ than the nonrelativistic ones.  This is sensible because the relativistic particles are able to access a much larger range of momenta at high temperatures, which spreads them out over a larger range of energies.
}

%
%	1.2
%

\clearpage
\prob{}{
	Now take the ultra-relativistic limit.  Find the mean energy $\evE$ and the second moment of energy $\evEs$.  Find the free energy and the entropy in the limits of high and low temperature.
}

\sol{
	The ultra-relativistic limit is $T \gg m c^2$~\cite[p.~175]{Pathria}.  Let $u = m c^2 / T$.  Then Eq.~\refeq{1.1ans2} becomes
	\eq{
		\limuo \ddNeps = \limuo \frac{N}{V} \frac{1}{T^2} \frac{e^{-\bet \eps} \eps \sqrt{\eps^2 / T^2 - u^2}}{u^2 \,K_2(u)} \dde
		= \frac{N}{V} \frac{1}{2 T^3} e^{-\bet \eps} \eps^2 \dde,
	}
	where we have used Mathematica to evaluate the limit of the denominator.
	
	The mean energy can be found by $\evE = N \eveps$, where $\eveps$ is the mean energy per molecule:
	\eq{
		\evE = N \eveps
		= V \limuo \int \eps \ddNeps
		= \frac{N}{2 T^3} \intoi \eps^3 e^{-\bet \eps} \dde
		= \frac{N}{2 T^3} 3!\, T^4
		= \ans{ 3N T, }
	}
	where we integrate from $\eps = 0$ since $mc^2 \to 0$ in this limit, and we have used $\intoi x^n e^{-\mu x} \dd{x} = n!\, \mu^{-n - 1}$~\cite[p.~340]{Integrals}.

	The second moment of energy is not an additive quantity, so we cannot simply compute $N \ev{\eps^2}$.  Let $E = \sumiN \epsi$, where $\epsi$ is the energy of a given molecule.  Then
	\eq{
		E^2 = \paren{ \sumiN \epsi } \paren{ \sumjN \epsj }
		= \sumiN \epsi^2 + \sum_{i \neq j} \epsi \epsj,
	}
	and the second moment of energy can be found by
	\eq{
		\evEs = \int \paren{ \sumiN \epsi^2 + \sum_{i \neq j} \epsi \epsj } \prodiN
	}
	
	
%	The second moment of energy \hl{can be found similarly}:
%	\eq{
%		\evEs = N \ev{\eps^2}
%		V \limuo \int \eps^2 \ddNeps
%		= \frac{N}{2 T^3} \intoi \eps^4 e^{-\bet \eps} \dde
%		= \frac{N}{2 T^3} 4!\, T^5
%		= \ans{ 12 N T^2. }
%	}
	
	The Helmholtz free energy is $F = -T \ln Z$, where $Z$ is the partition function~\cite[p.~87]{Landau}.  According to p.~1 of Lecture 4, the single-particle partition function of the Maxwell distribution can be found by
	\eq{
		\ddNvp = \frac{N}{V} \frac{1}{\Zii} e^{-\bet \vp^2 / 2m} \ddcp
		\qimplies
		\Zii = (2\pi m T)^{3/2}.
	}
	Applying this procedure to Eq.~\refeq{1.1ans}, we find
	\eq{
		\Zii = 4\pi T m^2 c \,K_2(\bet mc^2).
	}
	Assuming the gas molecules are indistinguishable, this gives the many-particle partition function
	\eq{
		Z = \frac{1}{N!} \brac{ 4\pi T m^2 c \,K_2(\bet mc^2) }^N.
	}
	
%	Then the free energy is
%	\eq{
%		F = -T \ln Z
%		\approx -T \brac{ N \ln(4\pi T m^2 c \,K_2(\bet mc^2)) - N \ln N + N }
%		= -N T \brac{ \ln(\frac{4\pi}{N} T m^2 c \,K_2(\bet mc^2)) + 1 },
%	}
%	where we have used Stirling's approximation $\ln N! \approx N \ln N - N$.  The entropy is
%	\al{
%		S &= -\paren{ \pdv{F}{T} }_V
%		= N \brac{ \ln(\frac{4\pi}{N} T m^2 c \,K_2(\bet mc^2)) + 1 } + N T \pdv{T} \brac{ \ln(\frac{4\pi}{N} m^2 c) + \ln T + \ln K_2(mc^2 / T) + 1 } \\
%		&= N \brac{ \ln(\frac{4\pi}{N} T m^2 c \,K_2(\bet mc^2)) + 1 } + N T \paren{ \frac{1}{T} + m c^2 \frac{K_1(\bet mc^2) + K_3(\bet mc^2)}{2 T^2} } \\
%		&= N \brac{ \ln(\frac{4\pi}{N} T m^2 c \,K_2(\bet mc^2)) + 1 + N + m c^2 \frac{K_1(\bet mc^2) + K_3(\bet mc^2)}{2 T^2}},
%	}
%	where we have used $\pdv*{K_\nu(z)}{z} = -[ K_{\nu - 1}(z) + K_{\nu + 1}(z) ]/2$~\cite{Derivative}.
%	
%	In the high-temperature limit,
%	\al{
%		\limTi F &= \limTi -N T \brac{ \ln(\frac{4\pi}{N} T m^2 c \,K_2(mc^2 / T)) + 1 }
%		= \limTi -N T \paren{ \ln T + \ln K_2(m c^2 / T) } \\
%		&= \ans{ \infty, } \\[2ex]
%		\limTi S &= \limTi N \brac{ \ln(\frac{4\pi}{N} T m^2 c \,K_2(\bet mc^2)) + 1 + N + m c^2 \frac{K_1(\bet mc^2) + K_3(\bet mc^2)}{2 T^2}} \\
%		&= \limTi N \brac{ \ln T + \ln K_2(mc^2 / T) + m c^2 \frac{K_1(mc^2 / T) + K_3(mc^2 / T)}{2 T^2}},
%	}
%	where we have used Mathematica to evaluate the limits~(graphically when needed).
	
	For the ultra-relativistic case,
	\eq{
		\limuo \Zii = 4\pi \frac{T^3}{c^3} \limuo u^2 \,K_2(u)
		= 8\pi \frac{T^3}{c^3}
		\qimplies
		Z = \frac{1}{N!} \paren{ 8\pi \frac{T^3}{c^3} }^N.
	}
	Then the free energy is
	\eq{
		F = -T \ln Z
		= -T \paren{ N \ln(8\pi \frac{T^3}{c^3}) - \ln N! }
		\approx -N T \paren{ \ln(\frac{8\pi}{N} \frac{T^3}{c^3}) + 1 },  
	}
	where we have used Stirling's approximation $\ln N! \approx N \ln N - N$.  The entropy can be found by $S = -(\pdv*{F}{T})_V$~\cite[p.~47]{Pathria}, which gives us
	\al{
		S &= -\paren{ \pdv{F}{T} }_V
		= \pdv{T} \brac{ N T \paren{ \ln(\frac{8\pi}{N} \frac{T^3}{c^3}) + 1 } }
		= N \paren{ \ln(\frac{8\pi}{N} \frac{T^3}{c^3}) + 1 } + N T \pdv{T} \brac{ \ln(\frac{8\pi}{N c^3}) + 3 \ln T + 1 } \\
		&= N \paren{ \ln(\frac{8\pi}{N} \frac{T^3}{c^3}) + 4 }.
	}	
	In the high-temperature limit,
	\al{
		\limTi F &= \limTi -N T \paren{ \ln(\frac{8\pi}{N} \frac{T^3}{c^3}) + 1 }
		= \limTi -3 N T \ln T
		= \infty, \\
		\limTi S &= \limTi N \paren{ \ln(\frac{8\pi}{N} \frac{T^3}{c^3}) + 4 }
		= \limTi 3 N \ln T
		= \infty.
	}
	In the low-temperature limit,
	\al{
		\limTo F &= \limTo -N T \paren{ \ln(\frac{8\pi}{N} \frac{T^3}{c^3}) + 1 }
		= \limTo -3N T \ln T
		= 0, \\
		\limTo S &= \limTo N \paren{ \ln(\frac{8\pi}{N} \frac{T^3}{c^3}) + 4 }
		= \limTo 3 N \ln T
		= -\infty.
	}
}

%
%	1.3
%

\prob{}{
	In the non-relativistic Maxwell distribution, the different translational degrees of freedom are independent as the kinetic energy is the sum of three independent terms $K = \sum_{i=1}^3 \pii^2 / 2m$.  This is not so in the relativistic case.  For the ultra-relativistic gas compute the quantities
	\al{
		\aij &= \frac{\evpijs}{3 \evpis \evpjs}, &
		\rij &= \frac{\evpijs}{\sqrt{\evpiq \evpjq}},
	}
	in spatial dimensions $d = 2, 3$ (here $i, j$ enumerate spatial dimensions).  Compare them to the non-relativistic case.  Discuss their meaning and dependence on $d$ (at least based on $d = 2,3$).
}



\state{Collision frequency and pressure}{
	Consider an ideal relativistic gas in a container.  Given the rate of the collisions of molecules with the wall of the container per unit area per unit time, find the pressure of the gas in the relativistic, non-relativistic, and ultra-relativistic cases, and compare the results.
}




\state{Boltzmann distribution}{
	Consider an ideal gas consisting of $N$ identical one-dimensional quantum harmonic oscillators with Hamiltonian $H(p, q) = p^2 / 2m + m\omg q^2 / 2$.  Determine the total number of oscillators in states with energies $\eps \geq \epsq = \hbar\omg (\nq + 1/2)$.
}



\state{Boltzmann \texorpdfstring{$H$}{H}-function}{
	The equilibrium distribution function $f(p, q)$ of a non-interacting gas is a Maxwell-Boltzmann distribution.  Show that the entropy of such a system satisfies $S = -\kB H + \const$, where $H = \int f \ln f \dG$ is the Boltzmann $H$-function.
}



\state{BBGKY}{
	Consider for simplicity a 1D system (a system on a circle) of $N$ particles with an arbitrary two-body interaction:
	\eq{
		H = \sumiN \frac{\pii^2}{2m} + \sumi U(\xii) + \sumij V(\xii - \xjj).
	}
	Give a derivation of the first equation of the BBGKY hierarchy at equilibrium for this system, which is a relation between the 1-point and 2-point distribution (correlation) functions.
}


\state{Partition function as a generating functional}{
	Consider the Gibbs distribution of the system described in Problem~5.  For simplicity neglect the kinetic energy. Let $n(x) = \sumi \del(x - \xii)$ be the density, and $\evnx$ its expectation value. Let $C(x, y) = \ev{\del n(x) \,\del n(y)}$, where $\del n(x) = n(x) - \evn$, be the two-point correlation function.
}

\prob{}{
	Show that $\evnx = -T \,\ddvs{\ln Z}{U(x)}$, where $Z[U(x)]$ is the partition function of the Gibbs distribution treated as a functional of the potential $U$.
}

\prob{}{
	Show that
	\eq{
		C(x, y) = T^2 \ddvm{\ln Z}{U(x)}{U(y)}
		= -T \ddv{\evnx}{U(y)}
		= -T \ddv{\evny}{U(x)}.
	}
}




%\makebib

\end{document}