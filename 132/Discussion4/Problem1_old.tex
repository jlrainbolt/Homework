\newcommand{\eps}{\epsilon}
\newcommand{\vx}{\vec{x}}
\newcommand{\phix}{\phi(\vx)}
\newcommand{\vp}{\vec{p}}
\newcommand{\dcx}{\dd[3]{x}}
\newcommand{\dcxp}{\dd[3]{x'}}
\newcommand{\rhox}{\rho(\vx)}
\newcommand{\rhoxp}{\rho(\vx')}
\newcommand{\xh}{\vec{\hat{x}}}
\newcommand{\absx}{\abs{\vx}}
\newcommand{\absxp}{\abs{\vx'}}

\newcommand{\Ylm}{Y_{l m}}
\newcommand{\qlm}{q_{l m}}
\newcommand{\Plm}{P_l^m}
\newcommand{\tht}{\theta}
\newcommand{\cost}{\cos\tht}
\newcommand{\vph}{\varphi}
\newcommand{\tv}{(\tht, \vph)}
\newcommand{\tvp}{(\tht', \vph')}
\newcommand{\Gxxp}{G(\vx, \vx')}
\newcommand{\qplm}{q'_{l m}}

\newcommand{\lap}{\nabla^2}
\newcommand{\evphi}{\ev{\phi}}
\newcommand{\evphix}{\evphi\!(\vx)}
\newcommand{\rhof}{\rho_f}
\newcommand{\fe}{\frac{1}{\eps}}
\newcommand{\tif}{\text{if }}
\newcommand{\Al}{A_l}
\newcommand{\Bl}{B_l}
\newcommand{\Cl}{C_l}

\newcommand{\intoi}{\int_0^\infty}
\newcommand{\intono}{\int_{-1}^{1}}
\newcommand{\intotp}{\int_0^{2\pi}}
\newcommand{\drp}{\dd{r'}}
\newcommand{\dctp}{\dd{(\cost')}}
\newcommand{\dvp}{\dd{\vph'}}

\newcommand{\Ploq}{P_l^0(1)}
\newcommand{\Ylotv}{Y_{l 0}\tv}
\newcommand{\dr}{\dd{r}}
\newcommand{\dct}{\dd{(\cost)}}
\newcommand{\ddv}{\dd{\vph}}

\begin{statement}{}
	Consider a dielectric ball of radius $R$ with dielectric constant $\eps$.  Obtain a multipole expansion for the field, $\phix$, of a point charge $q$ placed at a point $\vx'$ with $\abs{\vx'} = d > R$ (so the charge is outside of the dielectric ball).
	
	Hint: Follow the procedure we used in class to find the multipole expansion of a point charge without the dielectric, but now consider the three regions $r \leq R$, $R \leq r \leq d$, and $r \geq d$.  Obtain the form of the solution in these regions and match suitably.
\end{statement}

\begin{solution}
	In class, we derived the multipole expansion for $\absx \geq R$ when the charge distribution $\rhoxp$ is nonzero only within $\abs{\vx'} \leq R$.  We can find an equivalent expression for the reverse situation (within $\absx \leq R$ when the charge distribution $\rhoxp$ is nonzero only for $\abs{\vx'} \geq R$) using the spherical harmonic expansion of the Green's function $\Gxxp$ in Eq.~(2.78):
	\beq
		\Gxxp = \frac{1}{\abs{\vx - \vx'}}
		= \begin{cases} \sum_{l,m} \dfrac{4\pi}{2l + 1} \dfrac{r^l}{{r'}^{l + 1}} \Ylm^*\tvp \, \Ylm\tv & \tif r < r', \\
		\sum_{l,m} \dfrac{4\pi}{2l + 1} \dfrac{{r'}^l}{r^{l + 1}} \Ylm^*\tvp \, \Ylm\tv & \tif r > r'. \end{cases}
	\eeq
	As in Eq.~(2.79) in the course notes, we integrate and obtain
	\beq
		\phix = \int \Gxxp \, \rhoxp \dcxp
		= \sum_{l, m} \frac{4\pi}{2l + 1} r^l \Ylm\tv \int \frac{\rhoxp}{{r'}^{l+1}} \Ylm^*\tvp \dcxp.
	\eeq
	Combining this with the result of Eq.~(2.79), we have
	\beqn \label{multipole}
		\phix = \begin{cases} \sum_{l, m} \dfrac{4\pi}{2l + 1} r^l \, \qplm \, \Ylm\tv & \tif r < r' \text{ and } \rhoxp(r) = 0, \\
		\sum_{l,m} \dfrac{4\pi}{2l + 1} \dfrac{\qlm}{r^{l+1}} \Ylm\tv & \tif r > r'  \text{ and } \rhoxp(r) = 0, \end{cases}
	\eeqn
	where
	\begin{align*}
		\qlm &\equiv \int \rhoxp \, {r'}^l \, \Ylm^*\tvp \dcxp, &
		\qplm &\equiv \int \frac{\rhoxp}{{r'}^{l+1}} \Ylm^*\tvp \dcxp,
	\end{align*}
	from Eq.~(2.80) and our derivation.  Additionally, the spherical harmonics $\Ylm$ are given by Eq.~(2.58),
	\beq
		\Ylm\tv = \sqrt{\frac{2l + 1}{4\pi}} \sqrt{\frac{(l - m)!}{(l + m)!}} \Plm(\cost) e^{i m \vph},
	\eeq
	and the associated Legendre polynomials $\Plm$ are given by Eq.~(2.59),
	\beq
		\Plm(x) = \frac{(-1)^m}{2^l l!} (1 - x^2)^{m/2} \dv[l + m]{}{x} (x^2 - 1)^l.
	\eeq	
	Poisson's equation inside a dielectric medium is given by Eq.~(3.22),
	\beqn \label{poissdie}
		\lap\evphi = -\frac{4\pi}{\eps} \ev{\rhof},
	\eeqn
	where $\rhof$ is the free charge density.  For this problem, $\rhof = 0$ since the point charge is outside the dielectric.
	
	Without loss of generality, we may choose the location of the point charge to be on the $z$ axis at $z = d$, so $\vx' = (r', 0, 0)$.  We will begin inside the dielectric, where $r \leq R$.  We need a solution to Laplace's equation, which is the first case of \refeq{multipole}, with a factor of $1/\eps$ inserted to account for the dielectric constant:
	\beqn \label{A}
		\evphix = \fe \sum_{l, m} A_{l m} \frac{4\pi}{2l + 1} r^l \, \qplm \, \Ylm\tv
		= \fe \sum_l \Al \frac{4\pi}{2l + 1} r^l \, q'_{l 0} \, \Ylotv \quad \tif r \leq R,
	\eeqn
	where $\Al$ are constants, and $m = 0$ because the system is azimuthally symmetric.  Then
	\begin{align*}
		q'_{l 0} &= \sqrt{\frac{2l + 1}{4\pi}} \sqrt{\frac{l!}{l!}} \int \frac{\rhoxp}{{r'}^{l+1}} P_l^0(\cost' = 1) \dcxp
		= \sqrt{\frac{2l + 1}{4\pi}} \Ploq \intotp \intono \intoi \frac{q \, \delta(d - r')}{{r'}^{l+1}} {r'}^2 \drp \dctp \dvp \\
		&= \sqrt{\frac{2l + 1}{4\pi}} q \Ploq \bigg[ \vph' \bigg]_0^{2\pi} \bigg[ \cost' \bigg]_{-1}^1 \frac{1}{d^{l-1}}
		= \sqrt{4\pi (2l + 1)} q \frac{\Ploq}{d^{l-1}}.
%		&= \sqrt{\frac{2l + 1}{4\pi}} \frac{q}{2^l l!} \intotp \intono \intoi \frac{\delta(d - r')}{{r'}^{l+1}} \left[ \dv[l]{}{(\cost')} (\cos^2\tht' - 1)^l \right]_1 {r'}^2 \drp \dctp \dvp \\
%		&= \sqrt{\frac{2l + 1}{4\pi}} \frac{q}{2^l l!} \left[ \dv[l]{}{(\cost')} (\cos^2\tht' - 1)^l \right]_1 \intotp \dvp \intono \dctp \intoi \frac{\delta(d - r')}{{r'}^{l-1}} \drp \\
%		&= \sqrt{\frac{2l + 1}{4\pi}} \frac{q}{2^l l!} \left[ \dv[l]{}{(\cost')} (\cos^2\tht' - 1)^l \right]_1 \bigg[ \vph \bigg]_0^{2\pi} \bigg[ \cost' \bigg]_{-1}^1 \frac{1}{d^{l-1}} \\
%		&= \frac{\sqrt{4\pi (2l + 1)}}{2^l l!} \frac{q}{d^{l-1}} \left[ \dv[l]{}{(\cost')} (\cos^2\tht' - 1)^l \right]_1.
	\end{align*}
	
	In the region $R \leq d \leq r$, we are in the same regime as the former situation with respect to the position of the charge.  However, we now in free space, where $\ev\phi = \phi$ and we no longer need the factor of $1/\eps$.  Then
	\beqn \label{B}
		\phix = \sum_l \Bl \frac{4\pi}{2l + 1} r^l \, q'_{l 0} \, \Ylotv \quad \tif R \leq r \leq d,
	\eeqn
	where $\Bl$ are constants.
	
	In the region $r > d$, we need to use the second case of \refeq{multipole}.  Once again taking advantage of the azimuthal symmetry, this gives us
	\beqn \label{C}
		\phix = \sum_l \Cl \frac{4\pi}{2l + 1} \frac{q_{l 0}}{r^{l+1}} \Ylotv \quad \tif r \geq d,
	\eeqn
	where $\Cl$ are constants, and
	\begin{align*}
		q_{l 0} &= \sqrt{\frac{2l + 1}{4\pi}} \int \rhoxp \, {r'}^l \, P_l^0(\cost' = 1) \dcxp
		= \sqrt{\frac{2l + 1}{4\pi}} \Ploq \intotp \intono \intoi q \,\delta(d - r') \, {r'}^l {r'}^2 \drp \dctp \dvp \\
		&= \sqrt{4\pi (2l + 1)} q \, \Ploq \, d^{l+2}.
%		&= \sqrt{\frac{2l + 1}{4\pi}} \frac{1}{2^l l!} \left[ \dv[l]{}{(\cost')} (\cos^2\tht' - 1)^l \right]_1 \intotp \dvp \intono \dctp \intoi \delta(d - r') {r'}^{l+2} \drp \\
%		&= \frac{\sqrt{4\pi (2l + 1)}}{2^l l!} d^{l+2} \left[ \dv[l]{}{(\cost')} (\cos^2\tht' - 1)^l \right]_1.
	\end{align*}
	
%	On the boundary $r = d$,
%	\begin{align*}
%		\phix &= \int \Gxxp \, \rhox \dcx
%		= q \intotp \intono \int_d \frac{r^2}{r^2 - 2 d r \cost + d^2} \delta(d - r) \dr \dct \ddv
%		= 2\pi q \intono \frac{d^2}{2d^2 - 2d^2 \cost} \dct \\
%		&= \pi q \intono \frac{1}{1 - \cost} \dct
%		= -\pi q  \bigg[ \ln(1 - \cost) \bigg]_{-1}^{1}
%	\end{align*}
	
	
	Now we must match \refeq{A} and \refeq{B} at $r = R$.  Evaluating \refeq{A}, we have
	\beq
		\evphi\!(r=R) = \fe \sum_l \Al \frac{4\pi}{2l + 1} R^l \, \Ylotv \sqrt{4\pi (2l + 1)} q \frac{\Ploq}{d^{l-1}}
		= \frac{4\pi q}{\eps} \Ploq \sum_l \Al \sqrt{\frac{4\pi}{2l + 1}} \frac{R^l}{d^{l-1}} \Ylotv
%		\evphi\!(r=R) = \frac{4\pi}{\eps} \sum_l \Al  \sqrt{\frac{4\pi}{2l + 1}} \frac{1}{2^l l!} \frac{R^l}{d^{l-1}} \Ylotv \left[ \dv[l]{}{(\cost')} (\cos^2\tht' - 1)^l \right]_1,
	\eeq
	and for \refeq{B}, we have
	\beq
		\phi(r=R) = 4\pi q \Ploq \sum_l \Bl \sqrt{\frac{4\pi}{2l + 1}} \frac{R^l}{d^{l-1}} \Ylotv
%		= 4\pi \sum_l \Bl \sqrt{\frac{4\pi}{2l + 1}} \frac{1}{2^l l!} \frac{R^l}{d^{l-1}} \Ylotv \left[ \dv[l]{}{(\cost')} (\cos^2\tht' - 1)^l \right]_1.
	\eeq
	Equating these gives us $\Al = \eps \Bl$.
	
	We must also match \refeq{B} and \refeq{C} at $r = d$.  Evaluating \refeq{B}, we have
	\beq
		\phi(r=d) = \sum_l \Bl \frac{4\pi}{2l + 1} d^l \, \sqrt{4\pi (2l + 1)} q \frac{\Ploq}{d^{l-1}} \, \Ylotv
		= 4\pi q d \, \Ploq \sum_l \Bl \sqrt{\frac{4\pi}{2l + 1}} \Ylotv
%		= 4\pi d \sum_l \sqrt{\frac{4\pi}{2l + 1}} \frac{1}{2^l l!} \Ylotv \left[ \dv[l]{}{(\cost')} (\cos^2\tht' - 1)^l \right]_1,
	\eeq
	and for \refeq{B}, we have
	\beq
		\phi(r=d) = \sum_l \Cl \frac{4\pi}{2l + 1} \frac{1}{d^{l+1}} \sqrt{4\pi (2l + 1)} q \, \Ploq \, d^{l+2} \Ylotv
		= 4\pi q d \, \Ploq \sum_l \Cl \sqrt{\frac{4\pi}{2l + 1}} \Ylotv
		% = 4\pi d \sum_l \Cl \sqrt{\frac{4\pi}{2l + 1}} \frac{1}{2^l l!} \Ylotv \left[ \dv[l]{}{(\cost')} (\cos^2\tht' - 1)^l \right]_1,
	\eeq
	\hl{which just seems plan wrong :(}
	
\end{solution}