\documentclass[11pt]{article}
\usepackage{geometry}
\usepackage{graphicx}
\usepackage[parfill]{parskip}


\title{Physics 335 project proposal: \\
	Design and construction of a system \\ to test ATLAS TileCal electronics}
\author{\textbf{Student:} Lacey Rainbolt, \texttt{jlrainbolt@uchicago.edu} \\ \textbf{Advisor:} Mark Oreglia, \texttt{m-oreglia@uchicago.edu}}
\date{\today}

\begin{document}
\maketitle

\section*{Project description}

The Tile Calorimeter~(TileCal) is the hadronic calorimeter in the ATLAS experiment measuring proton-proton collisions at CERN's Large Hadron Collider.  The TileCal covers the barrel region of the ATLAS detector and provides energy and direction measurements for physics objects such as jets, taus, hadrons, and missing energy.  The electronics used to record calorimeter data is being redesigned to handle higher rates and ambient radiation at the high luminosity upgrade of the LHC.  Prototype of the new TileCal electronics must be tested in order to assess their functionality and performance under these detector conditions.

The 335 project will consist of designing and building a system to run a prototype of the new electronics.  The system will record and analyze the data stream from the new electronics, as well as exercise the communication paths for configuring the parameters (gains, etc.) on the electronics.  The data will be analyzed to measure noise levels and the linearity of energy response.

\section*{Signatures}


\end{document}
