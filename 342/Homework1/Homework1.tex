\documentclass[11pt]{article}
\usepackage{geometry, titlesec}
\usepackage[parfill]{parskip}
\usepackage[italicdiff]{physics}
\usepackage{amsfonts, amsthm}
\usepackage[cm]{fullpage}
\usepackage{fancyhdr}
\usepackage{enumitem}
\usepackage{xcolor, soul}
\usepackage{kbordermatrix}
\allowdisplaybreaks

\makeatletter
\renewcommand*\env@cases[1][1.2]{%
  \let\@ifnextchar\new@ifnextchar
  \left\lbrace
  \def\arraystretch{#1}%
  \array{@{}l@{\quad}l@{}}%
}
\makeatother

 
\renewcommand{\footrulewidth}{.2pt}
%\setlist[enumerate]{leftmargin=*}
\pagestyle{fancy}
\fancyhf{}
\lhead{\textbf{Physics 342 Homework 1}}
\rhead{Lacey Rainbolt}
\setlength{\headheight}{11pt}
\setlength{\headsep}{11pt}
\setlength{\footskip}{24pt}
\lfoot{\today}
\rfoot{\thepage}

\titleformat{\section}[runin]{\normalfont\large\bfseries}{Problem \thesection.}{1em}{}
\titleformat{\subsection}[runin]{\normalfont\large\bfseries}{\thesubsection}{1em}{}
\titleformat{\subparagraph}[leftmargin]{\normalfont\normalsize\bfseries}{}{0pt}{}

\newcommand{\refeq}[1]{(\ref{#1})}

\newcommand{\beq}{\begin{equation*}}
\newcommand{\eeq}{\end{equation*}}

\newcommand{\beqn}{\begin{equation}}
\newcommand{\eeqn}{\end{equation}}


\renewcommand{\vec}[1]{\mathbf{#1}}
\newcommand{\vfix}{\vspace{-\baselineskip}}


\newenvironment{statement}[1]
{
	\section{#1}
	\color{darkgray}
	\ignorespaces
}
{
%    \smallskip
}

\newenvironment{problem}
{
    \color{darkgray}
    \subsection{}
    \ignorespaces
}


\newenvironment{solution}
{
    \paragraph{Solution.}
    \ignorespaces
}
{
%    \smallskip
}

\newcommand{\Schrodinger}{Schr\"{o}dinger}


\begin{document}

\state{Spin-wave theory~(P\&S 11.1)}{\hfix}

\prob{ \label{1a}
	Prove the following wonderful formula: Let $\phix$ be a free scalar field with propagator $\ev{T \phix \phio} = \Dx$.  Then
	\eqn{show1}{
		\ev{ T e^{i \phix} e^{-i \phio} } = e^{[ \Dx - \Do ]}.
	}
	(The  factor $\Do$ gives a formally divergent adjustment of the overall normalization.)
}

\sol{
	According to P\&S~(9.18),
	\eq{
		\ev*{T \phi(\xq) \phi(\xw)}{\Omg} = \frac{\int \DDphi \phi(\xq) \phi(\xw) \exp[ i \int \ddqx \cL ]}{\int \DDphi \exp[ i \int \ddqx \cL ]}.
	}
	We use this expression to write the left-hand side of Eq.~\refeq{show1}:
	\eqn{thing1}{
		\ev{ T e^{i \phix} e^{-i \phio} } = \frac{\int \DDphi e^{i \phix} e^{-i \phio} \exp[ i \int \ddqy \cL ]}{\int \DDphi \exp[ i \int \ddqy \cL ]}
		= \frac{\int \DDphi \exp[i \phix - i \phio + i \int \ddqy \cL ]}{\int \DDphi \exp[ i \int \ddqy \cL ]}.
	}
	For a free Klein-Gordon~(i.e., scalar) field, Eq.~(9.39) tells us that the generating functional $\ZJ$ is given by
	\eq{
		\ZJ = \Zo \exp[ -\frac{1}{2} \int \ddqx \ddqy \Jx \DF(x - y) \Jy ],
	}
	where $\Zo = Z[0]$.  Thus, we want to find some $\Jy$ such that
	\eqn{thing1b}{
		\ev{ T e^{i \phix} e^{-i \phio} } = \frac{\ZJ}{\Zo}
	}
	where in general
	\eq{
		\ZJ = \int \DDphi \exp[ i \int \ddqx [ \cL + \Jx \phi(x) ] ]
	}
	by (9.34).  Inspecting Eq.~\refeq{thing1}, we recognize the denominator as $\Zo$ and see that if
	\eq{
		\Jy = \delq(y - x) - \delq(y)
	}
	we have an expression like Eq.~\refeq{thing1b}.  Collecting these findings, we have
	\al{
		\ans{ \ev{ T e^{i \phix} e^{-i \phio} } }&= \frac{\ZJ}{\Zo} \\
		&= \exp[ -\frac{1}{2} \int \ddqy \ddqz \Jy \DF(y - z) \Jz ] \\
		&= \exp[ -\frac{1}{2} \int \ddqy \ddqz \Jy \DF(y - z) [ \delq(z - x) - \delq(z) ] ] \\
		&= \exp[ -\frac{1}{2} \int \ddqy [ \delq(y - x) - \delq(y) ] [ \DF(y - x) - \DF(y) ] ] \\
		&= \exp[ -\frac{1}{2} [ \DF(0) - \DF(x) - \DF(-x) + \DF(0) ] ] \\
		&= \exp[ \DF(x) - \DF(0) ] \\
		&\ans{\; = e^{[ \Dx - \Do ]}, }
	}
	as we wanted to show. \qed
}



\prob{ \label{1b}
	We can use this formula in Euclidean field theory to discuss correlation functions in a theory with spontaneously broken symmetry for $T < \TC$.  Let us consider only the simplest case of a broken $O(2)$ or $U(1)$ symmetry.  We can write the local spin density as a complex variable
	\eq{
		\sx = \sqx + i \swx.
	}
	The global symmetry is the transformation
	\eq{
		\sx \to e^{-i \alp} \sx.
	}
	If we assume that the physics freezes the modulus of $\sx$, we can parameterize
	\eqn{sx}{
		\sx = A e^{i \phix}
	}
	and write an effective Lagrangian for the field $\phix$.  The symmetry of the theory becomes the translation symmetry
	\eqn{symmetry}{
		\phix \to \phix - \alp.
	}
	Show that (for $d > 0$) the most general renormalizable Lagrangian consistent with this symmetry is the free field theory
	\eqn{show1b}{
		\cL = \frac{1}{2} \rho(\vgrad \phi)^2.
	}
	In statistical mechanics, the constant $\rho$ is called the \emph{spin wave modulus}.  A reasonable hypothesis for $\rho$ is that it is finite for $T < \TC$ and tends to 0 as $T \to \TC$ from below.
}

\sol{
	In accordance with the Klein-Gordon Lagrangian in P\&S~(2.6),
	\eqn{KGL}{
		\cL_\text{K-G} = \frac{1}{2} (\pt \phi)^2 - \frac{1}{2} m^2 \phi^2,
	}
	we interpret $(\vgrad \phi)^2$ as $(\pt \phi)^2$.
	
	The Lagrangian cannot have terms of $\order{\phi^n}$ for any $n \neq 0$ since $\phi(x)$ is not invariant under Eq.~\refeq{symmetry}.  Any combination of derivatives of $\phi$ is invariant, however, since $\alp$ is a constant and does not contribute to any derivative.  Thus, only terms like $\pt^n \phi^m$ (where $n$ denotes a power of $\pt$) for $n, m > 0$ and $n \geq m$ are consistent with the symmetry of Eq.~\refeq{symmetry} for $d$ an integer.
	
	Now we must determine which of these terms are renormalizable.  We know that the Lagrangian must have dimension $d$, and that $\phi$ has dimension $(d - 2) / 2$.  Taking a derivative adds a mass dimension.  The theory is renormalizable if the coupling constant $\rho$ has dimension greater than or equal to 0~\cite[p.~322]{Peskin}.  Let $p$ be the dimension of $\rho$.  The dimension of our allowed term is then
	\eq{
		[ \rho \pt^n \phi^m ] = p + n + m \frac{d - 2}{2},
	}
	which we require to be equal to $d$.  Thus we seek solutions to the system of equations
	\al{
		d &= p + n + m \frac{d - 2}{2}, &
		n &\geq m, &
		p &\geq 0.
	}
	Solving with Mathematica, we find that this system has two solutions: $n = m = 2$ and $p = 0$; and $n = m = 1$ and $p = d / 2$.  However, the term $\pt \phi$ for $n = m = 1$ does not contribute to the action because it is a total derivative and does not contribute when the integral over $\cL$ is evaluated:
	\eq{
		\int \dd[d]{x} \pt\phi = \phi \bigg|_{-\infty}^\infty
		= 0.
	}
	Thus the only possibility is $n = m = 2$.  Note that
	\eq{
		\pt^2 \phi^2 = \pt(\pt \phi^2)
		= 2 \pt( \phi \pt \phi)
		= \pt \phi \pt \phi + \phi \pt^2 \phi
		= (\pt \phi)^2,
	}
	since $\phi \pt^2 \phi$ is not invariant under Eq.~\refeq{sx}.  This means that $\rho$ must be dimensionless and that the only allowed terms in the Lagrangian are proportional to $(\pt \phi)^2$, which is consistent with Eq.~\refeq{show1b}. \qed
}



\prob{
	Compute the correlation function $\ev{ \sx \sao }$.  Adjust $A$ to give a physically sensible normalization (assuming that the system has a physical cutoff at the scale of one atomic spacing) and display the dependence of this correlation function on $x$ for $d = 1, 2, 3, 4$.  Explain the significance of your results.
}

\sol{
	Applying Eq.~\refeq{sx},
	\eq{
		\ev{ \sx \sao } = \ev*{ A e^{i \phix} \As e^{-i \phio} }
		= \ev*{ \abs{A}^2 } \ev*{ e^{i \phix} e^{-i \phio} }.
	}
	Now we can apply Eq.~\refeq{show1} to find
	\eqn{thing1c}{
		\ans{ \ev{ \sx \sao } = \abs{A}^2 \exp[ D(x) - D(0) ], }
	}
	where $D(x - y)$ is a Green's function.  Since our Lagrangian is similar to the Klein-Gordon Lagrangian Eq.~\refeq{2.6}, our Green's function is similar to that of the Klein-Gordon operator, which is given by P\&S~(2.56):
	\eq{
		(\pt^2 + m^2) D(x - y) = -i \delq(x - y).
	}
	The Feynman prescription for this Green's function is given by (2.59),
	\eqn{DF}{
		\DF(x - y) = \int \ddqpf \frac{i}{p^2 - m^2 + i \eps} e^{-i p \cdot (x - y)}.
	}
	For the Lagrangian in Eq.~\refeq{show1b}, we set $m = 0$ and insert a factor of $\rho$:
	\eq{
		\rho \pt^2 D(x - y) = -i \deld(x - y),
	}
	so adapting Eq.~\refeq{DF} for this situation yields
	\eqn{DF}{
		\DF(x - y) = \frac{1}{\rho} \int \dddpf \frac{i}{p^2 + i \eps} e^{-i p \cdot (x - y)}.
	}
	We see that $\DF(0)$ diverges, so we absorb it into the constant to make the normalization physically sensible.  We can do this because, as we showed in \ref{1b}, the theory is renormalizable.  Define $A'$ such that
	\eq{
		{A'}^2 = \abs{A}^2 e^{-D(0)}.
	}
	Then Eq.~\refeq{thing1c} can be written
	\eq{
		\ans{ \ev{ \sx \sao } =  {A'}^2 e^{D(x)}. }
	}
	
	To evaluate the divergent integral $D(x)$, we look to the Feynman parameter method we have been using to solve divergent integrals.  Apparently, the Schwinger parametrization is useful in deriving the Feynman parametrization, and it is given by~\cite{Feynman}
	\eq{
		\frac{1}{A} = \intoi \dds e^{-s A}.
	}
	Using this equation, we can write Eq.~\refeq{DF} as
	\eq{
		\DF(x) = \frac{1}{\rho} \int \dddpf \frac{i}{p^2} e^{-i p \cdot x}
		= \frac{i}{\rho} \int \dddpf \intoi \dds e^{-s p^2} e^{-i p \cdot x}.
	}
	Now we can complete the square in the exponential to get a Gaussian integral:
	\al{
		\DF(x) &= \frac{i}{\rho} \int \dddpf \intoi \dds \exp[ -s p^2 - i p \cdot x + \frac{x^2}{4 s} - \frac{x^2}{4 s} ] \\
		&= \frac{i}{\rho} \int \dddpf \intoi \dds \exp[ -s \paren{ p + \frac{i x}{2 s} }^2 - \frac{x^2}{4 s} ] \\
		&= \frac{i}{\rho (2 \pi)^d} \intoi \dds e^{-x^2 / 4 s} \int \dd[d]{u} e^{-s u^2} \\
		&= \frac{i}{\rho (2 \pi)^{d}} \intoi \dds e^{-x^2 / 4 s} \sqrt{ \frac{(2\pi)^d}{(2s)^d} } \\
		&= \frac{i}{\rho (4 \pi)^{d / 2}} \intoi \dds \frac{e^{-x^2 / 4 s}}{s^{d / 2}}
	}
	where we have used~\cite{QFT}
	\eq{
		\int \exp( -\frac{1}{2} x \cdot A \cdot x ) \dd[n]{x} = \sqrt{\frac{(2\pi)^n}{\det A}},
	}
	with $A$ a $d \times d$ diagonal matrix $2s$.  Using Mathematica to integrate with respect to $s$, we find
	\eq{
		\DF(x) = \frac{i}{\rho (4 \pi)^{d / 2}} \frac{2^{d - 2}}{x^{d - 2}} \Gam(d / 2 - 1)
		= \frac{i}{4 \pi^d \rho} \Gam(d / 2 - 1) x^{2 - d}.
	}
	The gamma function diverges as $d \to 2$, so as we have done in previous problems, we expand about $\eps = 2 - d$.  Evaluating the series expansion using Mathematica, we obtain
	\eq{
		\DF(x) = \frac{i}{4 \pi^{1 - \eps} \rho} \Gam(\eps / 2) x^\eps
		\approx \frac{i}{4 \pi \rho} \paren{ \frac{2}{\eps} - \gam + 2 \ln(\pi x) }
		\sim \frac{i}{2 \pi \rho} \ln(x)
		= i \ln(\frac{1}{x^{2 \pi \rho}}).
	}
	We Wick rotate $x \to i x$.  Then the dependence of the correlation function on $x$ for $d = 1, 2, 3, 4$ is
	\ans{\al{
		(d = 1) &\qquad \ev{ \sx \sao } \sim e^{-x / 2 \sqrt{\pi} \rho}, &
		(d = 2) &\qquad \ev{ \sx \sao } \sim x^{2 \pi \rho}, \\
		(d = 3) &\qquad \ev{ \sx \sao } \sim \frac{1}{x}, &
		(d = 4) &\qquad \ev{ \sx \sao } \sim \frac{1}{x^2}.
	}}%
	In $d > 2$ dimensions, the expectation value of the correlation function tends to 0 at large distances $x$.  For $d > 2$, it drops off more quickly as $d$ increases.  The $d \leq 2$ cases depend on $\rho$, which we assume is positive.  The $d = 1$ case drops off with increasing distance, and more quickly with smaller $\rho$.  For $d = 2$, the expectation value of the correlation function increases with increasing distance, and it blows up more quickly with larger $\rho$.
	
	These results are consistent with the Mermin--Wagner theorem, which states that a continuous symmetry cannot be broken in $d \leq 2$ dimensions~\cite{CMW}.  That is, in $d \leq 2$ dimensions, a symmetry-breaking field cannot have a nonzero vacuum expectation value~\cite[p.~460]{Peskin}.  A physical explanation is that each spin has more nearest neighbors in higher dimensions.  Since the spins are inclined to align with their neighbors, there is a higher degree of correlation in higher dimensions at the same distance.  In two dimensions, the correlations are weak enough that they are overpowered by the field fluctuations.
}



\newcommand{\Eii}{E^{(1)}}
\newcommand{\vE}{\vec{E}}
\newcommand{\ao}{a_0}
\newcommand{\tht}{\theta}
\newcommand{\vr}{\vec{r}}
\newcommand{\drho}{\dd{\rho}}
\newcommand{\dtht}{\dd{\tht}}
\newcommand{\dphi}{\dd{\phi}}

\begin{statement}{}
	Consider the Stark effect for the $n = 3$ states of hydrogen.  There are initially nine degenerate states $\ket{3, l, m}$ (neglect spin), and an electric field $E$ is turned on in the $z$ direction.
\end{statement}

\begin{problem}
	Construct the $9 \times 9$ matrix representing the perturbed Hamiltonian in this case.  Show your work when deriving the nonzero matrix elements, and provide an explanation as to why the other elements are zero.
\end{problem}

\begin{solution}
	The perturbation operator for the $\vE$ field is given by (5.2.17) in Sakurai:
	\beq
		V = -e Z |\vE|.
	\eeq
	$V$ is a dipole interaction because the hydrogen atom can be thought of as behaving like a dipole when subjected to an external electric field.  Therefore $V$ obeys the dipole selection rule, which is given by (17.2.21) in Shankar:
	\beq
		\mel{n l m}{Z}{n' l' m'} = 0 \qq{unless} \begin{cases} l' = l \pm 1 \\ m' = m \end{cases}.
	\eeq
	The dipole selection rule is a combination of the angular momentum and parity selection rules.  The angular momentum selection rule stipulates that $\mel{n l m}{Z}{n' l' m'} = 0$ unless $l' = l, l \pm 1$ and $m' = m + q$ where $q = 0$ is the magnetic quantum number of the tensor operator $Z$.  The parity selection rule eliminates $l = l'$ because $\mel{n l m}{Z}{n' l' m'} = 0$ unless $l$ and $l'$ have opposite parity.
	
	For the nonzero elements, the hydrogen atom wavefunctions are given by (A.6.3) in Sakurai:
	\beq
		\braket{\vr}{n l m} = \psi_{n l m}(r, \tht, \phi) = R_{n l}(r) Y_l^m(\tht, \phi),
	\eeq
	where
	\beqn \label{R}
		R_{n l}(r) = -\sqrt{\left( \frac{2}{n \ao} \right)^3 \frac{(n - l - 1)!}{2n (n + l)!^3}} e^{-\rho / 2} \rho^l L_{n + l}^{2l + 1}(\rho)
		\qq{where}
		\rho = \frac{2 r}{n \ao}.
	\eeqn
	The associated Laguerre polynomials $L_p^q$ are given by (A.6.4) and (A.6.5),
	\beqn \label{laguerre}
		L_p^q(\rho) = \dv[q]{L_p(\rho)}{\rho}
		\qq{where}
		L_p(\rho) = e^\rho \dv[p]{}{\rho} (\rho^p e^{-\rho}).
	\eeqn
	The spherical harmonics $Y_l^m$ are given by (3.6.37) and (3.6.38),
	\begin{align} \label{harm}
		Y_l^m(\tht, \phi) &= \frac{(-1)^l}{2^l l!} \sqrt{\frac{2l + 1}{4\pi} \frac{(l + m)!}{(l - m)!}} e^{i m \phi} \frac{1}{\sin^m{\tht}} \dv[{l - m}]{}{(\cos\tht)} (\sin\tht)^{2l}, &
		Y_l^{-m}(\tht, \phi) &= (-1)^m {Y_l^m}^*(\tht, \phi)
	\end{align}
	for $m \geq 0$.
	
	The nonzero elements all have $l \in \{0, 1, 2\}$ and $m \in \{-1, 0, 1\}$.  Substituting into \refeq{R}, the relevant $R_{n l}$ are
	\begin{align*}
		R_{3 0}(r) &= -\sqrt{\left( \frac{2}{3 \ao} \right)^3 \frac{(3 - 1)!}{2 (3) 3!^3}} e^{-\rho / 2} L_{3}^{1}(\rho)
		= -\sqrt{\frac{2^3}{3^3 \ao^3} \frac{2}{2^4 3^4}} e^{-\rho / 2} L_{3}^{1}(\rho)
		= -\sqrt{\frac{e^{-\rho}}{3^7 \ao^3}} L_{3}^{1}(\rho), \\
		R_{3 1}(r) &= -\sqrt{\left( \frac{2}{3 \ao} \right)^3 \frac{(3 - 1 - 1)!}{2 (3) (3 + 1)!^3}} e^{-\rho / 2} \rho L_{3 + 1}^{2 + 1}(\rho)
		= -\sqrt{\frac{2^3}{3^3 \ao^3} \frac{1}{2^{10} 3^4}} e^{-\rho / 2} \rho L_{4}^{3}(\rho)
		= -\sqrt{\frac{e^{-\rho}}{2^7 3^7 \ao^3}} \rho L_{4}^{3}(\rho), \\
		R_{3 2}(r) &= -\sqrt{\left( \frac{2}{3 \ao} \right)^3 \frac{(3 - 2 - 1)!}{2 (3) (3 + 2)!^3}} e^{-\rho / 2} \rho^2 L_{3 + 2}^{4 + 1}(\rho)
		= -\sqrt{\frac{2^3}{3^3 \ao^3} \frac{1}{2^{10} 3^4 5^3}} e^{-\rho / 2} \rho^2 L_{5}^{5}(\rho)
		= -\sqrt{\frac{e^{-\rho}}{2^7 3^7 5^3 \ao^3}} \rho^2 L_{5}^{5}(\rho).
	\end{align*}
	From \refeq{laguerre}, the relevant $L_p$ are
	\begin{align*}
		L_3(\rho) &= e^\rho \dv[3]{}{\rho} (\rho^3 e^{-\rho})
		= e^\rho \dv[2]{}{\rho} (3 \rho^2 e^{-\rho} - \rho^3 e^{-\rho})
		= e^\rho \dv{}{\rho} (6 \rho e^{-\rho} - 6 \rho^2 e^{-\rho} + \rho^3 e^{-\rho})
		= 6 - 18 \rho + 9 \rho^2 - \rho^3, \\ \\
		L_4(\rho) &= e^\rho \dv[4]{}{\rho} (\rho^4 e^{-\rho})
		= e^\rho \dv[3]{}{\rho} (4 \rho^3 e^{-\rho} - \rho^4 e^{-\rho})
		= e^\rho \dv[2]{}{\rho} (12 \rho^2 e^{-\rho} - 8 \rho^3 e^{-\rho} + \rho^4 e^{-\rho}) \\
		&= e^\rho \dv{}{\rho} (24 \rho e^{-\rho} - 36 \rho^2 e^{-\rho} + 12 \rho^3 e^{-\rho} - \rho^4 e^{-\rho})
		= 24 - 96 \rho + 72 \rho^2 - 16 \rho^3 + \rho^4, \\ \\
		L_5(\rho) &= e^\rho \dv[5]{}{\rho} (\rho^5 e^{-\rho})
		= e^\rho \dv[4]{}{\rho} (5 \rho^4 e^{-\rho} - \rho^5 e^{-\rho})
		= e^\rho \dv[3]{}{\rho} (20 \rho^3 e^{-\rho} - 10 \rho^4 e^{-\rho} + \rho^5 e^{-\rho}) \\
		&= e^\rho \dv[2]{}{\rho} (60 \rho^2 e^{-\rho} - 60 \rho^3 e^{-\rho} + 15 \rho^4 e^{-\rho} - \rho^5 e^{-\rho}) \\
		&= e^\rho \dv{}{\rho} (120 \rho e^{-\rho} - 240 \rho^2 e^{-\rho} + 120 \rho^3 e^{-\rho} - 20 \rho^4 e^{-\rho} + \rho^5 e^{-\rho})
		= 120 - 600 \rho + 600 \rho^2 - 200 \rho^3 + 25 \rho^4 - \rho^5
	\end{align*}
	and then the relevant $L_p^q$ are
	\begin{align*}
		L_3^1(\rho) &= \dv{L_3(\rho)}{\rho} 
		= -18 + 18 \rho - 3 \rho^2
		= -3 (2^1 3 - 2^1 3 \rho + \rho^2), \\
		L_4^3(\rho) &= \dv[3]{L_4(\rho)}{\rho}
		= -(3!) 16 + \left( \frac{4!}{1!} \right) \rho
		= 2^3 3 (-2^2 + \rho), \\
		L_5^5(\rho) &= \dv[5]{L_5(\rho)}{\rho}
		= 5!
		= 2^3 3^1 5.
	\end{align*}
	\clearpage
	Substituting into \refeq{harm}, the relevant $Y_l^m$ are
	\begin{align*}
		Y_0^0(\tht, \phi) &= \sqrt{\frac{1}{2^2 \pi}}, \\
		Y_1^0(\tht, \phi) &= \sqrt{\frac{3}{2^2 \pi}} \cos\tht, &
		Y_1^{\pm 1}(\tht, \phi) &= \mp \sqrt{\frac{3}{2^3 \pi}} e^{\pm i \phi} \sin\tht, \\
		Y_2^0(\tht, \phi) &= \sqrt{\frac{5}{2^4 \pi}} (3 \cos^2\tht - 1), &
		Y_2^{\pm1}(\tht, \phi) &= \mp \sqrt{\frac{3^1 5}{2^3 \pi}} e^{\pm i \phi} \cos\tht \sin\tht.
	\end{align*}
	
	Note that $Z = r \cos\tht$ in polar coordinates.  In general, the nonzero matrix elements are then
	\begin{align*}
		\mel{3 l m}{V}{3 l' m'} &= -e |\vE| \int_0^{2\pi} \int_0^\pi \int_0^\infty \psi_{3 l m}^*(r, \tht, \phi) r \cos\tht \psi_{3 l' m'}(r, \tht, \phi) r^2 \sin\tht \dd{r} \dd{\tht} \dphi \\
		&= -e |\vE| \left( \frac{3 \ao}{2} \right)^4 \int_0^{2\pi} \int_0^\pi \int_0^\infty \psi_{3 l m}^*(r, \tht, \phi) \psi_{3 l' m'}(r, \tht, \phi) \rho^3 \cos\tht \sin\tht \drho \dtht \dphi \\
		&= -\frac{3^4 \ao^4 e |\vE|}{2^4} \int_0^{2\pi} \int_0^\pi {Y_l^m}^*(\tht, \phi) Y_{l'}^{m'}(\tht, \phi) \cos\tht \sin\tht \dtht \dphi \int_0^\infty R_{3 l}(r) R_{3 l'}(r) \rho^3 \drho.
	\end{align*}
	
	Firstly
	\beq
		V_{3 1 0}^{3 0 0} = \mel{3 1 0}{V}{3 0 0}
		= -\frac{3^4 \ao^4 e |\vE|}{2^4} \int_0^{2\pi} \int_0^\pi {Y_1^0}^*(\tht, \phi) Y_0^0(\tht, \phi) \cos\tht \sin\tht \dtht \dphi \int_0^\infty R_{3 1}(r) R_{3 0}(r) \rho^3 \drho
	\eeq
	where
	\begin{align*}
		\int_0^{2\pi} &\int_0^\pi {Y_1^0}^*(\tht, \phi) Y_0^0(\tht, \phi) \cos\tht \sin\tht \dtht \dphi
		= \int_0^{2\pi} \int_0^\pi \sqrt{\frac{3}{2^2 \pi}} \cos\tht \sqrt{\frac{1}{2^2 \pi}} \cos\tht \sin\tht \dtht \dphi \\
		&= \frac{\sqrt{3}}{2^2 \pi} \int_0^\pi \dphi \int_0^{2\pi} \cos^2\tht \sin\tht \dtht
	\end{align*}
%	\begin{align*}
%		V_{3 2 \pm1}^{3 1 \pm 1} &= \mel{3 2 \pm1}{V}{3 1 \pm1}
%	\end{align*}
	
%	Therefore,
%	\begin{align*}
%		\kbordermatrix{
%			nlm & 3 0 0 & 3 1 -1 & 3 1 0 & 3 1 1 & 3 2 -2 & 3 2 -1 & 3 2 0 & 3 2 1 & 3 2 2 \\
%			3 0 0 & 0 & 0 & & 0 & 0 & 0 & 0 & 0 & 0 \\
%			3 1 -1 & 0 & 0 & 0 & 0 & 0 & & 0 & 0 & 0 \\
%			3 1 0 & & 0 & 0 & 0 & 0 & 0 & & 0 & 0 \\
%			3 1 1 & 0 & 0 & 0 & 0 & 0 & 0 & 0 & & 0 \\
%			3 2 -2 & 0 & 0 & 0 & 0 & 0 & 0 & 0 & 0 & 0 \\
%			3 2 -1 & 0 & & 0 & 0 & 0 & 0 & 0 & 0 & 0 \\
%			3 2 0 & 0 & 0 & & 0 & 0 & 0 & 0 & 0 & 0 \\
%			3 2 1 & 0 & 0 & 0 & & 0 & 0 & 0 & 0 & 0 \\
%			3 2 2 & 0 & 0 & 0 & 0 & 0 & 0 & 0 & 0 & 0 \\
%		}
%	\end{align*}
	
\end{solution}



\begin{problem}
	Determine the first order corrections, $\Eii$, to the energies due to this perturbation, and write down the degeneracies of these energies.
\end{problem}




\newcommand{\kq}{\ket{1}}
\newcommand{\kw}{\ket{2}}
\newcommand{\ke}{\ket{3}}

\newcommand{\vq}{v_1}
\newcommand{\vw}{v_2}
\newcommand{\ve}{v_3}

\begin{statement}{}
	Consider the Hamiltonian $\Ho$ acting on a three-dimensional Hilbert space spanned by the orthonormal basis $\{\kq, \kw, \ke\}$.  $\Ho = \sum_{i = 3}^3 E_i \ketbra{i}$, with energy eigenvalues $\Eq, \Ew, \Ee$.  Assume $\Eq = \Ew = E$.  To $\Ho$, we add a perturbation
	\beq
		V = \vq \ketbra{1}{3} + \vq^* \ketbra{3}{1} + \vw \ketbra{2}{3} + \vw^* \ketbra{3}{2}.
	\eeq
	Here, $\vq$ and $\vw$ are complex constants and small compared to $\Ee$.
\end{statement}

\begin{problem}
	To second order in $V$, write down the explicit form of the effective Hamiltonian acting on the subspace spanned by $\{\kq, \kw\}$.
\end{problem}

\begin{problem}
	By solving the effective Hamiltonian, construct the approximate solution for the eigenvalues and eigenfunctions of $\Ho + V$.  (The eigenkets only need to be constructed within the degenerate subspace.)
\end{problem}


\vfill
I consulted Shankar's \emph{Principles of Quantum Mechanics} in addition to Sakurai's \emph{Modern Quantum Mechanics} while writing up these solutions.

\end{document}