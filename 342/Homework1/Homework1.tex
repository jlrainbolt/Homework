\documentclass[11pt]{article}
\usepackage{geometry, titlesec}
\usepackage[parfill]{parskip}
\usepackage[italicdiff]{physics}
\usepackage{amsfonts, amsthm}
\usepackage[cm]{fullpage}
\usepackage{fancyhdr}
\usepackage{enumitem}
\usepackage{xcolor, soul}
\usepackage{kbordermatrix}
\allowdisplaybreaks

\makeatletter
\renewcommand*\env@cases[1][1.2]{%
  \let\@ifnextchar\new@ifnextchar
  \left\lbrace
  \def\arraystretch{#1}%
  \array{@{}l@{\quad}l@{}}%
}
\makeatother

 
\renewcommand{\footrulewidth}{.2pt}
%\setlist[enumerate]{leftmargin=*}
\pagestyle{fancy}
\fancyhf{}
\lhead{\textbf{Physics 342 Homework 1}}
\rhead{Lacey Rainbolt}
\setlength{\headheight}{11pt}
\setlength{\headsep}{11pt}
\setlength{\footskip}{24pt}
\lfoot{\today}
\rfoot{\thepage}

\titleformat{\section}[runin]{\normalfont\large\bfseries}{Problem \thesection.}{1em}{}
\titleformat{\subsection}[runin]{\normalfont\large\bfseries}{\thesubsection}{1em}{}
\titleformat{\subparagraph}[leftmargin]{\normalfont\normalsize\bfseries}{}{0pt}{}

\newcommand{\refeq}[1]{(\ref{#1})}

\newcommand{\beq}{\begin{equation*}}
\newcommand{\eeq}{\end{equation*}}

\newcommand{\beqn}{\begin{equation}}
\newcommand{\eeqn}{\end{equation}}


\renewcommand{\vec}[1]{\mathbf{#1}}
\newcommand{\vfix}{\vspace{-\baselineskip}}


\newenvironment{statement}[1]
{
	\section{#1}
	\color{darkgray}
	\ignorespaces
}
{
%    \smallskip
}

\newenvironment{problem}
{
    \color{darkgray}
    \subsection{}
    \ignorespaces
}


\newenvironment{solution}
{
    \paragraph{Solution.}
    \ignorespaces
}
{
%    \smallskip
}

\newcommand{\Schrodinger}{Schr\"{o}dinger}


\begin{document}

\newcommand{\Ho}{H_0}
\newcommand{\Sx}{S_x}
\newcommand{\Sy}{S_y}
\newcommand{\Sz}{S_z}

\newcommand{\lam}{\lambda}

\newcommand{\kq}{\ket{1}}
\newcommand{\kw}{\ket{2}}
\newcommand{\ke}{\ket{3}}

\newcommand{\suo}{^{(0)}}

\newcommand{\Eon}{E\suo_n}
\newcommand{\Eoq}{E\suo_1}
\newcommand{\Eow}{E\suo_2}
\newcommand{\Eoe}{E\suo_3}

\newcommand{\Eqn}{E^{(1)}_n}
\newcommand{\Eqq}{E^{(1)}_1}
\newcommand{\Eqw}{E^{(1)}_2}
\newcommand{\Eqe}{E^{(1)}_3}

\newcommand{\kon}{\ket*{n\suo}}
\newcommand{\koq}{\ket*{1\suo}}
\newcommand{\kow}{\ket*{2\suo}}
\newcommand{\koe}{\ket*{3\suo}}

\newcommand{\Deln}{\Delta^{(1)}_n}
\newcommand{\Delq}{\Delta^{(1)}_1}
\newcommand{\Delw}{\Delta^{(1)}_2}
\newcommand{\Dele}{\Delta^{(1)}_3}

\newcommand{\Po}{P_0}
\newcommand{\Vo}{V_0}
\newcommand{\sigx}{\sigma_x}

\begin{statement}{}
	Consider a spin-1 particle.  The unperturbed Hamiltonian is $\Ho = A \Sz^2$, where $A$ is a constant.  Consider the perturbation $V = B(\Sx^2 - \Sy^2)$, where $|A| \gg |B|$.  Note that $S_i$ are the $3 \times 3$ spin matrices.
\end{statement}

\begin{problem}
	Calculate the first-order correction to the energies.
\end{problem}

\begin{solution}
	Firstly, note that
	\begin{align*}
		\Sx &= \frac{\hbar}{\sqrt{2}} \mqty[ 0 & 1 & 0 \\ 1 & 0 & 1 \\ 0 & 1 & 0 ], &
		\Sy &= \frac{\hbar}{\sqrt{2}} \mqty[ 0 & -i & 0 \\ i & 0 & -i \\ 0 & i & 0 ], &
		\Sz &= \hbar \mqty[ 1 & 0 & 0 \\ 0 & 0 & 0 \\ 0 & 0 & -1 ].
	\end{align*}
	Then
	\begin{align*}
		\Ho &= A \hbar^2 \mqty[ 1 & 0 & 0 \\ 0 & 0 & 0 \\ 0 & 0 & 1 ], &
		V &= B \frac{\hbar^2}{2} \left( \mqty[ 1 & 0 & 1 \\ 0 & 2 & 0 \\ 1 & 0 & 1 ] - \mqty[ 1 & 0 & -1 \\ 0 & 2 & 0 \\ -1 & 0 & 1 ] \right)
		= B \hbar^2 \mqty[ 0 & 0 & 1 \\ 0 & 0 & 0 \\ 1 & 0 & 0 ].
	\end{align*}
%	The perturbed Hamiltonian is given by
%	\beq
%		H = \Ho + \lam V = \hbar^2 \mqty[ A & 0 & \lam B \\ 0 & 0 & 0 \\ \lam B & 0 & A ].
%	\eeq
	The eigenvalues of $\Ho$ are $\Eoq = \Eoe = A \hbar^2$ and $\Eow = 0$, so the problem is degenerate.  The eigenkets are the $\Sz$ basis kets:
	\begin{align*}
		\koq &= \kq = \mqty[ 1 \\ 0 \\ 0 ], &
		\kow &= \kw = \mqty[ 0 \\ 1 \\ 0 ], &
		\koe &= \ke = \mqty[ 0 \\ 0 \\ 1 ].
	\end{align*}
	
	We will begin with the correction to $\Eow$, which is nondegenerate.  From (5.1.20) and (5.1.37) in Sakurai, the first-order energy corrections in the unperturbed case are given by
	\beq
		\Deln \equiv \Eqn - \Eon = \mel*{n\suo}{V}{n\suo}.
	\eeq
	This gives us
	\beq
		\Delw = \mel*{2\suo}{V}{2\suo} = \mel{2}{V}{2} = 0.
	\eeq
	
	For $\Eoq$ and $\Eow$, consider the degenerate subspace spanned by $\{\kq, \ke\}$.  Let $\Po$ be a projection onto this subspace, and let
	\beq
		\Vo = \Po V \Po = B \hbar^2 \mqty[ 0 & 1 \\ 1 & 0 ] = B \hbar^2 \sigx,
	\eeq
	where $\sigx$ is the Pauli matrix.  Therefore, we know that $\Vo$ has eigenvalues $v_\pm = \pm B \hbar^2$ and eigenvectors
	\begin{align*}
		\ket{v_+} &= \frac{B \hbar^2}{\sqrt{2}} \mqty[ 1 \\ 1 ] = \frac{B \hbar^2}{\sqrt{2}} (\kq + \ke), &
		\ket{v_-} &= \frac{B \hbar^2}{\sqrt{2}} \mqty[ -1 \\ 1 ] = \frac{B \hbar^2}{\sqrt{2}} (\ke - \kq).
	\end{align*}
	In this basis, $\Vo$ is diagonal.
\end{solution}

\begin{problem}
	Solve the problem exactly, and compare your result to the perturbation theory result.
\end{problem}



\newcommand{\Eii}{E^{(1)}}

\begin{statement}{}
	Consider the Stark effect for the $n = 3$ states of hydrogen.  There are initially nine degenerate states $\ket{3, l, m}$ (neglect spin), and an electric field $E$ is turned on in the $z$ direction.
\end{statement}

\begin{problem}
	Construct the $9 \times 9$ matrix representing the perturbing Hamiltonian in this case.  Show your work when deriving the nonzero matrix elements, and provide an explanation as to why the other elements are zero.
\end{problem}

\begin{problem}
	Determine the first order corrections, $\Eii$, to the energies due to this perturbation, and write down the degeneracies of these energies.
\end{problem}



\newcommand{\Eq}{E_1}
\newcommand{\Ew}{E_2}
\newcommand{\Ee}{E_3}

\newcommand{\vq}{v_1}
\newcommand{\vw}{v_2}
\newcommand{\ve}{v_3}

\begin{statement}{}
	Consider the Hamiltonian $\Ho$ acting on a three-dimensional Hilbert space spanned by the orthonormal basis $\{\kq, \kw, \ke\}$.  $\Ho = \sum_{i = 3}^3 E_i \ketbra{i}$, with energy eigenvalues $\Eq, \Ew, \Ee$.  Assume $\Eq = \Ew = E$.  To $\Ho$, we add a perturbation
	\beq
		V = \vq \ketbra{1}{3} + \vq^* \ketbra{3}{1} + \vw \ketbra{2}{3} + \vw^* \ketbra{3}{2}.
	\eeq
	Here, $\vq$ and $\vw$ are complex constants and small compared to $\Ee$.
\end{statement}

\begin{problem}
	To second order in $V$, write down the explicit form of the effective Hamiltonian acting on the subspace spanned by $\{\kq, \kw\}$.
\end{problem}

\begin{problem}
	By solving the effective Hamiltonian, construct the approximate solution for the eigenvalues and eigenfunctions of $\Ho + V$.  (The eigenkets only need to be constructed within the degenerate subspace.)
\end{problem}


\vfill
While writing up these solutions, I consulted Sakurai's \emph{Modern Quantum Mechanics} and Shankar's \emph{Principles of Quantum Mechanics}.

\end{document}