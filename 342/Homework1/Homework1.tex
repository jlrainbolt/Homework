\documentclass[11pt]{article}
\usepackage{geometry, titlesec}
\usepackage[parfill]{parskip}
\usepackage[italicdiff]{physics}
\usepackage{amsfonts, amsthm}
\usepackage[cm]{fullpage}
\usepackage{fancyhdr}
\usepackage{enumitem}
\usepackage{xcolor, soul}
\usepackage{kbordermatrix}
\allowdisplaybreaks

\makeatletter
\renewcommand*\env@cases[1][1.2]{%
  \let\@ifnextchar\new@ifnextchar
  \left\lbrace
  \def\arraystretch{#1}%
  \array{@{}l@{\quad}l@{}}%
}
\makeatother

 
\renewcommand{\footrulewidth}{.2pt}
%\setlist[enumerate]{leftmargin=*}
\pagestyle{fancy}
\fancyhf{}
\lhead{\textbf{Physics 342 Homework 1}}
\rhead{Lacey Rainbolt}
\setlength{\headheight}{11pt}
\setlength{\headsep}{11pt}
\setlength{\footskip}{24pt}
\lfoot{\today}
\rfoot{\thepage}

\titleformat{\section}[runin]{\normalfont\large\bfseries}{Problem \thesection.}{1em}{}
\titleformat{\subsection}[runin]{\normalfont\large\bfseries}{\thesubsection}{1em}{}
\titleformat{\subparagraph}[leftmargin]{\normalfont\normalsize\bfseries}{}{0pt}{}

\newcommand{\refeq}[1]{(\ref{#1})}

\newcommand{\beq}{\begin{equation*}}
\newcommand{\eeq}{\end{equation*}}

\newcommand{\beqn}{\begin{equation}}
\newcommand{\eeqn}{\end{equation}}


\renewcommand{\vec}[1]{\mathbf{#1}}
\newcommand{\vfix}{\vspace{-\baselineskip}}


\newenvironment{statement}[1]
{
	\section{#1}
	\color{darkgray}
	\ignorespaces
}
{
%    \smallskip
}

\newenvironment{problem}
{
    \color{darkgray}
    \subsection{}
    \ignorespaces
}


\newenvironment{solution}
{
    \paragraph{Solution.}
    \ignorespaces
}
{
%    \smallskip
}

\newcommand{\Schrodinger}{Schr\"{o}dinger}


\begin{document}

\state{(Jackson 9.8)}{\ 
	%\emph{Hint:} The electromagnetic angular momentum density comes from more than the transverse (radiation zone) components of the fields.
}

%
%	Jackson 9.8(a)
%

\prob{}{
	Show that a classical oscillating electric dipole $\vp$ with fields given by
	\aln{ \label{fields1}
		\vH &= \frac{c k^2}{4\pi} (\nh \cross \vp) \frac{e^{i k r}}{r} \paren{ 1 - \frac{1}{i k r} }, &
		\vE &= \frac{1}{4\pi \epso} \curly{ k^2 (\nh \cross \vp) \cross \nh \frac{e^{i k r}}{r} + [ 3 \nh (\nh \vdot \vp) - \vp ] \paren{ \frac{1}{r^3} - \frac{i k}{r^2} } e^{i k r} },
	}
	radiates electromagnetic angular momentum to infinity at the rate
	\eq{
		\dv{\vL}{t} = \frac{k^3}{12 \pi \epso} \Im[ \vp^* \cross \vp ].
	}
	\vfix
}

\sol{
	According to Jackson~(9.20), the time-averaged angular momentum density is
	\eq{
		\vl = \frac{\Re[ \vx \cross (\vE \cross \vHs)}{2 c^2}.
	}
	One of the vector identities on the inside cover of Jackson is $\vaa \cross (\vbb \cross \vcc) = (\vaa \vdot \vcc) \vbb - (\vaa \vdot \vbb) \vcc$, so
	\eqn{l1}{
		\vl = \frac{(\vx \vdot \vHs) \vE - (\vx \vdot \vE) \vHs}{2 c^2}.
	}
	From Eq.~\refeq{fields1}, note that
	\eq{
		\vx \vdot \vHs \propto \vx \vdot (\nh \cross \vps)
		= \vps \vdot (\vx \cross \nh)
		= \vO,
	}
	where we have used the identity $\vaa \vdot (\vbb \cross \vcc) = \vcc \vdot (\vaa \cross \vbb)$ and the fact that $\nh$ points in the $\vx$ direction.  For $\vx \vdot \vE$, note that
	\al{
		\vx \vdot [ (\nh \cross \vp) \cross \nh ] &= -\vx \vdot [ \nh \cross (\nh \cross \vp) ]
		= -\vx \vdot [ (\nh \vdot \vp) \nh - (\nh \vdot \nh) \vp ]
		= -(\nh \vdot \vp) (\vx \vdot \nh) + \vx \vdot \vp \\
		&= -r (\nh \vdot \vp) + \vx \vdot \vp
		= \vx \vdot \vp - \vx \vdot \vp
		= 0, \\[1.5ex]
		\vx \vdot [ 3 \nh (\nh \vdot \vp) - \vp ] &= 3 (\vx \vdot \nh) (\nh \vdot \vp) - \vx \vdot \vp
		= 3r (\nh \vdot \vp) - \vx \vdot \vp
		= 3(\vx \vdot \vp) - \vx \vdot \vp
		= 2(\vx \vdot \vp),
	}
	since $\abs{\vx} = r$ and $\vx = r \,\nh$.  Then
	\eq{
		\vx \vdot \vE = \frac{1}{2\pi \epso} (\vx \vdot \vp) \paren{ \frac{1}{r^3} - \frac{i k}{r^2} } e^{i k r}
		= \frac{1}{2\pi \epso} (\nh \vdot \vp) \paren{ \frac{1}{r^2} - \frac{i k}{r} } e^{i k r}.
	}
	
	With these substitutions, Eq.~\refeq{l1} becomes
	\al{
		\vl &= -\frac{(\vx \vdot \vE) \vHs}{c^2}
		= -\frac{1}{4\pi \epso c^2} (\nh \vdot \vp) \paren{ \frac{1}{r^2} - \frac{i k}{r} } e^{i k r} \frac{c k^2}{4\pi} (\nh \cross \vps) \frac{e^{-i k r}}{r} \paren{ 1 + \frac{1}{i k r} } \\
		&= -\frac{k^2}{16\pi^2 \epso c r} (\nh \vdot \vp) (\nh \cross \vps) \paren{ \frac{1}{r^2} - \frac{i k}{r} } \paren{ 1 - \frac{i}{k r} }
		= -\frac{k^2}{16\pi^2 \epso c} (\nh \vdot \vp) (\nh \cross \vps) \paren{ \frac{1}{r^2} - \frac{i}{k r^3} - \frac{i k}{r} - \frac{1}{r^2} } \\
		&= -\frac{i k^2}{16\pi^2 \epso c r} (\nh \vdot \vp) (\nh \cross \vps) \paren{ \frac{1}{k r^3} + \frac{k}{r^2} }
		= \frac{i k^3}{16\pi^2 \epso c r^2} (\nh \vdot \vp) (\nh \cross \vps) \paren{ \frac{1}{k^2 r^2} + 1 }.
	}
	
	Let $\vL$ be the angular momentum radiated to a distance $R$.  Then
	\eq{
		\vL = \int_R \vl(r) \ddcx
		= \intopi \intotp \intoR \vl(r) \,r^2 \sin\tht \ddr \ddphi \dd\tht,
	}
	and the time derivative is
	\aln{
		\dv{\vL}{t} &= \dv{t}(\intopi \intotp \intoR \vl(r) \,r^2 \sin\tht \ddr \ddphi \dd\tht)
		= \dv{r}{t} \dv{r}(\intopi \intotp \intoR \vl(r) \,r^2 \sin\tht \ddr \ddphi \dd\tht) \notag \\
		&= c \intopi \intotp \vl(r) \,r^2 \sin\tht \ddphi \dd\tht
		= \frac{i k^3}{16\pi^2 \epso} \paren{ \frac{1}{k^2 r^2} + 1 } \intopi \intotp (\nh \vdot \vp) (\nh \cross \vps) \sin\tht \ddphi \dd\tht. \label{dLdt}
	}
	Note that
	\eq{
		[ (\nh \vdot \vp) (\nh \cross \vps) ]_i = \sumje n_j p_j (\nh \cross \vps)_i
		= \sumje \sumke \sumle \epsikl n_j p_j n_k p_l^*,
	}
	so
	\eq{
		\dv{L_i}{t} \propto \sumje \sumke \sumle \epsikl p_j p_l^* \int n_j p_k \ddOmg
		= \sumje \sumke \sumle \epsikl p_j p_l^* \frac{4\pi}{3} \del_{jk}
		= \frac{4\pi}{3} \epsikl p_k p_l^*
		= \frac{4\pi}{3} (\vp \cross \vps)_i,
	}
	where we have used Jackson~(9.47), $\int n_\bet n_\gam \ddOmg = 4\pi \del_{\bet \gam} / 3$.  Making this substitution into Eq.~\refeq{dLdt},
	\eq{
		\dv{\vL}{t} = \frac{i k^3}{6\pi \epso} \paren{ \frac{1}{k^2 r^2} + 1 } (\vp \cross \vps).
	}
	Taking the limit as $r \to \infty$, we find
	\eqn{ans1a}{
		\dv{\vL}{t} = \Re\!\brac{ \frac{i k^3}{12\pi \epso} (\vp \cross \vps) }
		= \Re\!\brac{ -\frac{i k^3}{12\pi \epso} (\vps \cross \vp) }
		= \ans{ \frac{k^3}{12\pi \epso} \Im[ \vps \cross \vp ], }
	}
	as desired. \qed
}

%
%	Jackson 9.8(b)
%

\prob{}{
	What is the ratio of angular momentum radiated to energy radiated?  Interpret.
}

\sol{
	According to Jackson~(9.24), the total power radiated by an oscillating electric dipole $\vp$ is
	\eq{
		P = \dv{E}{t}
		= \frac{c^2 \Zo k^4}{12 \pi} \abs{\vp}^2.
	}
	Then the ratio of angular momentum radiated to energy radiated is
	\eq{
		\frac{\dv*{\vL}{t}}{\dv*{E}{t}} = \frac{k^3}{12\pi \epso} \Im[ \vps \cross \vp ] \frac{12 \pi}{c^2 \Zo k^4 \abs{\vp}^2}
		= \frac{1}{\epso} \Im[ \vps \cross \vp ] \frac{1}{c^2 \Zo k \abs{\vp}^2}
		= \ans{ \frac{\Im[ \vps \cross \vp ]}{\omg \abs{\vp}^2}, }
	}
	where we have used $\Zo = \sqrt{\muo / \epso} = 1 / \sqrt{\epso^2 c^2} = 1 / \epso c$, $c^2 = 1 / (\epso \muo)$, and $\omg = k c$.
	
	In the limit of high frequency, $(\dv*{\vL}{t}) / (\dv*{E}{t}) \to 0$.  In this scenario, the energy radiated dominates over the angular momentum radiated.  Likewise, in the limit of low frequency, $(\dv*{\vL}{t}) / (\dv*{E}{t}) \to \infty$, meaning that angular momentum radiation dominates.  This is sensible because rotational kinetic energy $E \propto \omg^2$, while angular momentum $L \propto \omg$.
}

%
%	Jackson 9.8(c)
%

\prob{}{
	For a charge $e$ rotating in the $xy$ plane at radius $a$ and angular speed $\omg$, show that there is only a $z$ component of radiated angular momentum with magnitude $\dv*{\Lz}{t} = e^2 k^3 a^2 / 6 \pi \epso$.  What about a charge oscillating along the $z$ axis?
}

\sol{
	We know from Homework~5 that the position of a point charge rotating counterclockwise in the $xy$ plane is
	\eq{
		\vx(t) = a \cos(\omg t) \,\vx + a \sin(\omg t) \,\yh.
	}
	\clearpage
	Then the charge distribution is
	\eq{
		\rho(\vx, t) = e \del[ x - a \cos(\omg t) ] \,\del[ y - a \sin(\omg t) ] \,\del(z).
	}
	
	According to Jackson~(4.8), the dipole moment is defined
	\eq{
		\vp = \int \vx' \,\rho(\vx') \ddcxp.
	}
	The components of $\vp$ for the point charge are then
	\al{
		\px &= e \iiint x \,\del[ x - a \cos(\omg t) ] \,\del[ y - a \sin(\omg t) ] \,\del(z) \ddx \ddy \ddz
		= e a \cos(\omg t), \\
		\py &= e \iiint y \,\del[ x - a \cos(\omg t) ] \,\del[ y - a \sin(\omg t) ] \,\del(z) \ddx \ddy \ddz
		= e a \sin(\omg t), \\
		\pz &= e \iiint z \,\del[ x - a \cos(\omg t) ] \,\del[ y - a \sin(\omg t) ] \,\del(z) \ddx \ddy \ddz
		= 0,
	}
	so we can write $\vp = e a \,e^{-i \omg t} (\xh + i\,\yh).$  Substituting into Eq.~\refeq{ans1a},
	\al{
		\dv{\vL}{t} &= \Re\!\brac{ \frac{i k^3}{12\pi \epso} e^2 a^2 e^{-i \omg t} e^{i \omg t} [ (\xh + i\,\yh) \cross (\xh - i\,\yh) ] }
		= \Re\!\brac{ \frac{i e^2 k^3 a^2}{12\pi \epso} (-2i \,\xh \cross \yh) }
		= \Re\!\brac{ \frac{e^2 k^3 a^2}{6\pi \epso} \,\zh } \\
		&= \ans{ \frac{e^2 k^3 a^2}{6\pi \epso} \cos(\omg t) \,\zh, }
	}
	as desired. \qed
	
	A charge oscillating along the $z$ axis with amplitude $a$ has the charge density
	\eq{
		\rho(\vx, t) = e a \,\del(x) \,\del(y) \,\del[ z - \cos(\omg t) ],
	}
	which gives the dipole moment
	\al{
		\px &= e a \iiint x \,\del(x) \,\del(y) \,\del[ z - \cos(\omg t) ] \ddx \ddy \ddz
		= 0, \\
		\py &= e a \iiint y \,\del(x) \,\del(y) \,\del[ z - \cos(\omg t) ] \ddx \ddy \ddz
		= 0, \\
		\pz &= e a \iiint z \,\del(x) \,\del(y) \,\del[ z - \cos(\omg t) ] \ddx \ddy \ddz
		= e a \cos(\omg t).
	}
	In complex notation, $\vp = e a \,e^{-i\omg t} \,\zh$.  Substituting into Eq.~\refeq{ans1a}, we find
	\eq{
		\dv{\vL}{t} = \Re\!\brac{ \frac{i k^3}{12\pi \epso} e^2 a^2 e^{-i \omg t} e^{i \omg t} (\zh \cross \zh) }
		= \ans{ \vO. }
	}
	So we see that a charge undergoing linear motion does not lead to a radiated angular momentum, which is sensible.
}

%
%	Jackson 9.8(d)
%

\prob{}{
	What are the results corresponding to Probs.~{1(a)} and {1(b)} for magnetic dipole radiation?
}

\sol{
	The radiation fields for a magnetic dipole are given by Jackson~(19.35--36),
	\al{
		\vH &= \frac{1}{4\pi} \curly{ k^2 (\nh \cross \vm) \cross \nh \frac{e^{i k r}}{r} + [ 3 \nh (\nh \vdot \vm) - \vm ] \paren{ \frac{1}{r^3} - \frac{i k}{r^2} } e^{i k r} }, &
		\vE &= -\frac{\Zo}{4\pi} k^2 (\nh \cross \vm) \frac{e^{i k r}}{r} \paren{ 1 - \frac{1}{i k r} }.
	}
	\clearpage
	Comparing with Eq.~\refeq{fields1}, we see that $\vH \to -\vE / \Zo$, $\vE \to \Zo \vH$, and $\vp \to \vm / c$ as stated in the book~\cite[p.~413]{Jackson}.  Making these substitutions, the results of Probs.~{1.1(a)} and {(b)} become
	\al{
		\ans{ \dv{\vL}{t}\ }&\ans{= \frac{\muo k^3}{12\pi} \Im[ \vms \cross \vm ], } &
		\ans{ \frac{\dv*{\vL}{t}}{\dv*{E}{t}}\ }&\ans{= \frac{\Im[ \vms \cross \vm ]}{\omg \abs{\vm}^2} }
	}
	where we have used $\mu = 1 / \epso c^2$.
}



\newcommand{\Eii}{E^{(1)}}
\newcommand{\vE}{\vec{E}}
\newcommand{\ao}{a_0}
\newcommand{\tht}{\theta}
\newcommand{\vr}{\vec{r}}
\newcommand{\drho}{\dd{\rho}}
\newcommand{\dcost}{\dd{(\cos\tht)}}
\newcommand{\dphi}{\dd{\phi}}

\begin{statement}{}
	Consider the Stark effect for the $n = 3$ states of hydrogen.  There are initially nine degenerate states $\ket{3, l, m}$ (neglect spin), and an electric field $E$ is turned on in the $z$ direction.
\end{statement}

\begin{problem}
	Construct the $9 \times 9$ matrix representing the perturbed Hamiltonian in this case.  Show your work when deriving the nonzero matrix elements, and provide an explanation as to why the other elements are zero.
\end{problem}

\begin{solution}
	The perturbation operator for the $\vE$ field is given by (5.2.17) in Sakurai:
	\beq
		V = -e Z |\vE|.
	\eeq
	$V$ is a dipole interaction because the hydrogen atom can be thought of as behaving like a dipole when subjected to an external electric field.  Therefore $V$ obeys the dipole selection rule, which is given by (17.2.21) in Shankar:
	\beq
		\mel{n l m}{Z}{n' l' m'} = 0 \qq{unless} \begin{cases} l' = l \pm 1 \\ m' = m \end{cases}.
	\eeq
	The dipole selection rule is a combination of the angular momentum and parity selection rules.  The angular momentum selection rule stipulates that $\mel{n l m}{Z}{n' l' m'} = 0$ unless $l' = l, l \pm 1$ and $m' = m + q$ where $q = 0$ is the magnetic quantum number of the tensor operator $Z$.  The parity selection rule eliminates $l = l'$ because $\mel{n l m}{Z}{n' l' m'} = 0$ unless $l$ and $l'$ have opposite parity.
	
	For the nonzero elements, the hydrogen atom wavefunctions are given by (A.6.3) in Sakurai:
	\beq
		\braket{\vr}{n l m} = \psi_{n l m}(r, \tht, \phi) = R_{n l}(r) Y_l^m(\tht, \phi),
	\eeq
	where
	\beqn \label{R}
		R_{n l}(r) = -\sqrt{\left( \frac{2}{n \ao} \right)^3 \frac{(n - l - 1)!}{2n (n + l)!^3}} e^{-\rho / 2} \rho^l L_{n + l}^{2l + 1}(\rho)
		\qq{where}
		\rho = \frac{2 r}{n \ao}.
	\eeqn
	The associated Laguerre polynomials $L_p^q$ are given by (A.6.4) and (A.6.5),
	\beqn \label{laguerre}
		L_p^q(\rho) = \dv[q]{L_p(\rho)}{\rho}
		\qq{where}
		L_p(\rho) = e^\rho \dv[p]{}{\rho} (\rho^p e^{-\rho}).
	\eeqn
	The spherical harmonics $Y_l^m$ are given by (3.6.37) and (3.6.38),
	\begin{align} \label{harm}
		Y_l^m(\tht, \phi) &= \frac{(-1)^l}{2^l l!} \sqrt{\frac{(2l + 1)}{4\pi} \frac{(l + m)!}{(l - m)!}} e^{i m \phi} \frac{1}{\sin^m{\tht}} \dv[{l - m}]{}{(\cos\tht)} (\sin\tht)^{2l}, &
		Y_l^{-m}(\tht, \phi) &= (-1)^m {Y_l^m}^*(\tht, \phi)
	\end{align}
	for $m \geq 0$.
	
	The nonzero elements all have $l \in \{0, 1, 2\}$ and $m \in \{-1, 0, 1\}$.  Substituting into \refeq{R}, the relevant $R_{n l}$ are
	\begin{align*}
		R_{3 0}(r) &= -\frac{e^{-\rho / 2}}{27 \sqrt{3} \ao^{3/2}} L_3^1(\rho), &
		R_{3 1}(r) &= -\frac{e^{-\rho / 2} \rho}{216 \sqrt{6} \ao^{3/2}} L_4^3(\rho), &
		R_{3 2}(r) &= -\frac{e^{-\rho / 2} \rho^2}{1080 \sqrt{30} \ao^{3/2}} L_5^5(\rho).	\end{align*}
	From \refeq{laguerre}, the relevant $L_p^q$ are
	\begin{align*}
		L_3^1(\rho) &= \frac{24 - 36 \rho + 12 \rho^2 - \rho^3}{6}, \\
		L_4^3(\rho) &= \frac{840 - 840 \rho + 252 \rho^2 - 28 \rho^3 + \rho^4}{24}, \\
		L_5^5(\rho) &= \frac{30240 - 25200 \rho + 7200 \rho^2 - 900 \rho^3 + 50 \rho^4 - \rho^5}{120}.
	\end{align*}
	Substituting into \refeq{harm}, the relevant $Y_l^m$ are
	\begin{align*}
		Y_0^0(\tht, \phi) &= \sqrt{\frac{1}{4 \pi}}, \\
		Y_1^0(\tht, \phi) &= \sqrt{\frac{3}{4 \pi}} \cos\tht, &
		Y_1^{\pm 1}(\tht, \phi) &= \mp \sqrt{\frac{3}{8 \pi}} e^{\pm i \phi} \sin\tht, \\
		Y_2^0(\tht, \phi) &= \sqrt{\frac{5}{16 \pi}} (3 \cos^2\tht - 1), &
		Y_2^{\pm1}(\tht, \phi) &= \mp \sqrt{\frac{15}{8 \pi}} e^{\pm i \phi} \cos\tht \sin\tht.
	\end{align*}
	
	Note that $Z = r \cos\tht$ in polar coordinates.  In general, the nonzero matrix elements are then
	\begin{align*}
		\mel{3 l m}{V}{3 l' m'} &= -e |\vE| \int_0^{2\pi} \int_0^\pi \int_0^\infty \psi_{3 l m}^*(r, \tht, \phi) r \cos\tht \psi_{3 l' m'}(r, \tht, \phi) r^2 \sin\tht \dd{r} \dd{\tht} \dphi \\
		&= -\frac{81 \ao^4 e |\vE|}{16} \int_0^{2\pi} \int_{-1}^1 \int_0^\infty \psi_{3 l m}^*(r, \tht, \phi) \psi_{3 l' m'}(r, \tht, \phi) \rho^3 \cos\tht \drho \dcost \dphi.
	\end{align*}
	For the particular matrix elements, we have
	\begin{align*}
		V_{3 1 0}^{3 0 0} &= \mel{3 1 0}{V}{3 0 0}
		= -\frac{81 \ao^4 e |\vE|}{16} \int_0^{2\pi} \int_{-1}^1 \int_0^\infty R_{31}(r) {Y_1^0}^*(\tht, \phi) R_{3 0} Y_0^0(\tht\phi) \rho^3 \cos\tht \drho \dcost \dphi \\
		&= -\frac{\ao e |\vE|}{4608 \sqrt{6} \pi} \int_0^{2\pi} \int_{-1}^1 \int_0^\infty L_4^3(\rho) L_{3}^{1}(\rho) e^{-\rho} \rho^4 \cos^2\tht \drho \dcost \dphi \\
		&= -\frac{\ao e |\vE|}{2304 \sqrt{6}} \int_{-1}^1 \int_0^\infty L_4^3(\rho) L_{3}^{1}(\rho) e^{-\rho} \rho^4 \cos^2\tht \drho \dcost \\
		&= -\frac{\ao e |\vE|}{3456 \sqrt{6}} \int_{-1}^1 \int_0^\infty L_4^3(\rho) L_{3}^{1}(\rho) e^{-\rho} \rho^4 \drho \\
		&= -\frac{\ao e |\vE|}{497664 \sqrt{6}} \int_0^\infty e^{-\rho} \rho^4 (840 - 840 \rho + 252 \rho^2 - 28 \rho^3 + \rho^4) (24 - 36 \rho + 12 \rho^2 - \rho^3) \drho \\
		&= -\frac{\ao e |\vE|}{497664 \sqrt{6}} \int_0^\infty e^{-\rho} (20160 \rho^4 - 50400 \rho^5 + 46368 \rho^6 - 20664 \rho^7 + 4896 \rho^8 - 634 \rho^9 + 40 \rho^{10} - \rho^{11}) \drho \\
		&= \frac{35}{144 \sqrt{6}} \ao e |\vE| = \mel{3 0 0}{V}{3 1 0},
	\end{align*}
	
	\begin{align*}
		V_{3 2 \pm1}^{3 1 \pm 1} &= \mel{3 2 \pm1}{V}{3 1 \pm1}
		= -\frac{81 \ao^4 e |\vE|}{16} \int_0^{2\pi} \int_{-1}^1 \int_0^\infty R_{3 2}(r) {Y_2^{\pm1}}^*(\tht, \phi) R_{3 1}(r) Y_1^{\pm1}(\tht\phi) \rho^3 \cos\tht \drho \dcost \dphi \\
		&= -\frac{\ao e |\vE|}{737280 \pi} \int_0^{2\pi} \int_{-1}^1 \int_0^\infty L_4^3(\rho) L_5^5(\rho) e^{-\rho} \rho^6 \cos^2\tht \sin^2\tht \drho \dcost \dphi \\
		&= -\frac{\ao e |\vE|}{368640} \int_{-1}^1 \int_0^\infty L_4^3(\rho) L_5^5(\rho) e^{-\rho} \rho^6 \cos^2\tht \sin^2\tht \drho \dcost \\
		&= -\frac{\ao e |\vE|}{5898250} \int_0^\infty L_4^3(\rho) L_5^5(\rho) e^{-\rho} \rho^6 \drho \\
		&= -\frac{\ao e |\vE|}{16986931200} \int_0^\infty e^{-\rho} \rho^6 (-\rho^{15} + 78 \rho^{14} - 2552 \rho^{13} + 45840 \rho^{12} - 496440 \rho^{11} + 3348240 \rho^{10} \\
		&\phantom{mmmmmmmmmmmmmmmmmmmmm} - 14001120 \rho^9 + 34836480 \rho^8 - 46569600 \rho^7 + 25401600 \rho^6) \drho \\
		&= \frac{105}{4096} \ao e |\vE| = \mel{3 1 \pm1}{V}{3 2 \pm1},
	\end{align*}
	
	Therefore,
	\begin{align*}
		\kbordermatrix{
			nlm & 3 0 0 & 3 1 -1 & 3 1 0 & 3 1 1 & 3 2 -2 & 3 2 -1 & 3 2 0 & 3 2 1 & 3 2 2 \\
			3 0 0 & 0 & 0 & & 0 & 0 & 0 & 0 & 0 & 0 \\
			3 1 -1 & 0 & 0 & 0 & 0 & 0 & & 0 & 0 & 0 \\
			3 1 0 & & 0 & 0 & 0 & 0 & 0 & & 0 & 0 \\
			3 1 1 & 0 & 0 & 0 & 0 & 0 & 0 & 0 & & 0 \\
			3 2 -2 & 0 & 0 & 0 & 0 & 0 & 0 & 0 & 0 & 0 \\
			3 2 -1 & 0 & & 0 & 0 & 0 & 0 & 0 & 0 & 0 \\
			3 2 0 & 0 & 0 & & 0 & 0 & 0 & 0 & 0 & 0 \\
			3 2 1 & 0 & 0 & 0 & & 0 & 0 & 0 & 0 & 0 \\
			3 2 2 & 0 & 0 & 0 & 0 & 0 & 0 & 0 & 0 & 0 \\
		}
	\end{align*}
	
\end{solution}



\begin{problem}
	Determine the first order corrections, $\Eii$, to the energies due to this perturbation, and write down the degeneracies of these energies.
\end{problem}




\newcommand{\kq}{\ket{1}}
\newcommand{\kw}{\ket{2}}
\newcommand{\ke}{\ket{3}}

\newcommand{\vq}{v_1}
\newcommand{\vw}{v_2}
\newcommand{\ve}{v_3}

\begin{statement}{}
	Consider the Hamiltonian $\Ho$ acting on a three-dimensional Hilbert space spanned by the orthonormal basis $\{\kq, \kw, \ke\}$.  $\Ho = \sum_{i = 3}^3 E_i \ketbra{i}$, with energy eigenvalues $\Eq, \Ew, \Ee$.  Assume $\Eq = \Ew = E$.  To $\Ho$, we add a perturbation
	\beq
		V = \vq \ketbra{1}{3} + \vq^* \ketbra{3}{1} + \vw \ketbra{2}{3} + \vw^* \ketbra{3}{2}.
	\eeq
	Here, $\vq$ and $\vw$ are complex constants and small compared to $\Ee$.
\end{statement}

\begin{problem}
	To second order in $V$, write down the explicit form of the effective Hamiltonian acting on the subspace spanned by $\{\kq, \kw\}$.
\end{problem}

\begin{problem}
	By solving the effective Hamiltonian, construct the approximate solution for the eigenvalues and eigenfunctions of $\Ho + V$.  (The eigenkets only need to be constructed within the degenerate subspace.)
\end{problem}


\vfill
I consulted Shankar's \emph{Principles of Quantum Mechanics} in addition to Sakurai's \emph{Modern Quantum Mechanics} while writing up these solutions.

\end{document}