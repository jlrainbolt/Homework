 \documentclass[11pt]{article}
\usepackage{geometry, titlesec}
\usepackage[parfill]{parskip}
\usepackage[italicdiff]{physics}
\usepackage{amsfonts, amsthm}
\usepackage[cm]{fullpage}
\usepackage{fancyhdr}
\usepackage{enumitem}
\usepackage{xcolor, soul}
\usepackage{kbordermatrix}
%\allowdisplaybreaks

\makeatletter
\renewcommand*\env@cases[1][1.2]{%
  \let\@ifnextchar\new@ifnextchar
  \left\lbrace
  \def\arraystretch{#1}%
  \array{@{}l@{\quad}l@{}}%
}
\makeatother

 
\renewcommand{\footrulewidth}{.2pt}
%\setlist[enumerate]{leftmargin=*}
\pagestyle{fancy}
\fancyhf{}
\lhead{\textbf{Physics 342 Homework 1}}
\rhead{Lacey Rainbolt}
\setlength{\headheight}{11pt}
\setlength{\headsep}{11pt}
\setlength{\footskip}{24pt}
\lfoot{\today}
\rfoot{\thepage}

\titleformat{\section}[runin]{\normalfont\large\bfseries}{Problem \thesection.}{1em}{}
\titleformat{\subsection}[runin]{\normalfont\large\bfseries}{\thesubsection}{1em}{}
\titleformat{\subparagraph}[leftmargin]{\normalfont\normalsize\bfseries}{}{0pt}{}

\newcommand{\refeq}[1]{(\ref{#1})}

\newcommand{\beq}{\begin{equation*}}
\newcommand{\eeq}{\end{equation*}}

\newcommand{\beqn}{\begin{equation}}
\newcommand{\eeqn}{\end{equation}}


\renewcommand{\vec}[1]{\mathbf{#1}}
\newcommand{\vfix}{\vspace{-\baselineskip}}


\newenvironment{statement}[1]
{
	\section{#1}
	\color{darkgray}
	\ignorespaces
}
{
%    \smallskip
}

\newenvironment{problem}
{
    \color{darkgray}
    \subsection{}
    \ignorespaces
}


\newenvironment{solution}
{
    \paragraph{Solution.}
    \ignorespaces
}
{
%    \smallskip
}

\newcommand{\Schrodinger}{Schr\"{o}dinger}


\begin{document}

\state{(Jackson 9.8)}{\ 
	%\emph{Hint:} The electromagnetic angular momentum density comes from more than the transverse (radiation zone) components of the fields.
}

%
%	Jackson 9.8(a)
%

\prob{}{
	Show that a classical oscillating electric dipole $\vp$ with fields given by
	\aln{ \label{fields1}
		\vH &= \frac{c k^2}{4\pi} (\nh \cross \vp) \frac{e^{i k r}}{r} \paren{ 1 - \frac{1}{i k r} }, &
		\vE &= \frac{1}{4\pi \epso} \curly{ k^2 (\nh \cross \vp) \cross \nh \frac{e^{i k r}}{r} + [ 3 \nh (\nh \vdot \vp) - \vp ] \paren{ \frac{1}{r^3} - \frac{i k}{r^2} } e^{i k r} },
	}
	radiates electromagnetic angular momentum to infinity at the rate
	\eq{
		\dv{\vL}{t} = \frac{k^3}{12 \pi \epso} \Im[ \vp^* \cross \vp ].
	}
	\vfix
}

\sol{
	According to Jackson~(9.20), the time-averaged angular momentum density is
	\eq{
		\vl = \frac{\Re[ \vx \cross (\vE \cross \vHs)}{2 c^2}.
	}
	One of the vector identities on the inside cover of Jackson is $\vaa \cross (\vbb \cross \vcc) = (\vaa \vdot \vcc) \vbb - (\vaa \vdot \vbb) \vcc$, so
	\eqn{l1}{
		\vl = \frac{(\vx \vdot \vHs) \vE - (\vx \vdot \vE) \vHs}{2 c^2}.
	}
	From Eq.~\refeq{fields1}, note that
	\eq{
		\vx \vdot \vHs \propto \vx \vdot (\nh \cross \vps)
		= \vps \vdot (\vx \cross \nh)
		= \vO,
	}
	where we have used the identity $\vaa \vdot (\vbb \cross \vcc) = \vcc \vdot (\vaa \cross \vbb)$ and the fact that $\nh$ points in the $\vx$ direction.  For $\vx \vdot \vE$, note that
	\al{
		\vx \vdot [ (\nh \cross \vp) \cross \nh ] &= -\vx \vdot [ \nh \cross (\nh \cross \vp) ]
		= -\vx \vdot [ (\nh \vdot \vp) \nh - (\nh \vdot \nh) \vp ]
		= -(\nh \vdot \vp) (\vx \vdot \nh) + \vx \vdot \vp \\
		&= -r (\nh \vdot \vp) + \vx \vdot \vp
		= \vx \vdot \vp - \vx \vdot \vp
		= 0, \\[1.5ex]
		\vx \vdot [ 3 \nh (\nh \vdot \vp) - \vp ] &= 3 (\vx \vdot \nh) (\nh \vdot \vp) - \vx \vdot \vp
		= 3r (\nh \vdot \vp) - \vx \vdot \vp
		= 3(\vx \vdot \vp) - \vx \vdot \vp
		= 2(\vx \vdot \vp),
	}
	since $\abs{\vx} = r$ and $\vx = r \,\nh$.  Then
	\eq{
		\vx \vdot \vE = \frac{1}{2\pi \epso} (\vx \vdot \vp) \paren{ \frac{1}{r^3} - \frac{i k}{r^2} } e^{i k r}
		= \frac{1}{2\pi \epso} (\nh \vdot \vp) \paren{ \frac{1}{r^2} - \frac{i k}{r} } e^{i k r}.
	}
	
	With these substitutions, Eq.~\refeq{l1} becomes
	\al{
		\vl &= -\frac{(\vx \vdot \vE) \vHs}{c^2}
		= -\frac{1}{4\pi \epso c^2} (\nh \vdot \vp) \paren{ \frac{1}{r^2} - \frac{i k}{r} } e^{i k r} \frac{c k^2}{4\pi} (\nh \cross \vps) \frac{e^{-i k r}}{r} \paren{ 1 + \frac{1}{i k r} } \\
		&= -\frac{k^2}{16\pi^2 \epso c r} (\nh \vdot \vp) (\nh \cross \vps) \paren{ \frac{1}{r^2} - \frac{i k}{r} } \paren{ 1 - \frac{i}{k r} }
		= -\frac{k^2}{16\pi^2 \epso c} (\nh \vdot \vp) (\nh \cross \vps) \paren{ \frac{1}{r^2} - \frac{i}{k r^3} - \frac{i k}{r} - \frac{1}{r^2} } \\
		&= -\frac{i k^2}{16\pi^2 \epso c r} (\nh \vdot \vp) (\nh \cross \vps) \paren{ \frac{1}{k r^3} + \frac{k}{r^2} }
		= \frac{i k^3}{16\pi^2 \epso c r^2} (\nh \vdot \vp) (\nh \cross \vps) \paren{ \frac{1}{k^2 r^2} + 1 }.
	}
	
	Let $\vL$ be the angular momentum radiated to a distance $R$.  Then
	\eq{
		\vL = \int_R \vl(r) \ddcx
		= \intopi \intotp \intoR \vl(r) \,r^2 \sin\tht \ddr \ddphi \dd\tht,
	}
	and the time derivative is
	\aln{
		\dv{\vL}{t} &= \dv{t}(\intopi \intotp \intoR \vl(r) \,r^2 \sin\tht \ddr \ddphi \dd\tht)
		= \dv{r}{t} \dv{r}(\intopi \intotp \intoR \vl(r) \,r^2 \sin\tht \ddr \ddphi \dd\tht) \notag \\
		&= c \intopi \intotp \vl(r) \,r^2 \sin\tht \ddphi \dd\tht
		= \frac{i k^3}{16\pi^2 \epso} \paren{ \frac{1}{k^2 r^2} + 1 } \intopi \intotp (\nh \vdot \vp) (\nh \cross \vps) \sin\tht \ddphi \dd\tht. \label{dLdt}
	}
	Note that
	\eq{
		[ (\nh \vdot \vp) (\nh \cross \vps) ]_i = \sumje n_j p_j (\nh \cross \vps)_i
		= \sumje \sumke \sumle \epsikl n_j p_j n_k p_l^*,
	}
	so
	\eq{
		\dv{L_i}{t} \propto \sumje \sumke \sumle \epsikl p_j p_l^* \int n_j p_k \ddOmg
		= \sumje \sumke \sumle \epsikl p_j p_l^* \frac{4\pi}{3} \del_{jk}
		= \frac{4\pi}{3} \epsikl p_k p_l^*
		= \frac{4\pi}{3} (\vp \cross \vps)_i,
	}
	where we have used Jackson~(9.47), $\int n_\bet n_\gam \ddOmg = 4\pi \del_{\bet \gam} / 3$.  Making this substitution into Eq.~\refeq{dLdt},
	\eq{
		\dv{\vL}{t} = \frac{i k^3}{6\pi \epso} \paren{ \frac{1}{k^2 r^2} + 1 } (\vp \cross \vps).
	}
	Taking the limit as $r \to \infty$, we find
	\eqn{ans1a}{
		\dv{\vL}{t} = \Re\!\brac{ \frac{i k^3}{12\pi \epso} (\vp \cross \vps) }
		= \Re\!\brac{ -\frac{i k^3}{12\pi \epso} (\vps \cross \vp) }
		= \ans{ \frac{k^3}{12\pi \epso} \Im[ \vps \cross \vp ], }
	}
	as desired. \qed
}

%
%	Jackson 9.8(b)
%

\prob{}{
	What is the ratio of angular momentum radiated to energy radiated?  Interpret.
}

\sol{
	According to Jackson~(9.24), the total power radiated by an oscillating electric dipole $\vp$ is
	\eq{
		P = \dv{E}{t}
		= \frac{c^2 \Zo k^4}{12 \pi} \abs{\vp}^2.
	}
	Then the ratio of angular momentum radiated to energy radiated is
	\eq{
		\frac{\dv*{\vL}{t}}{\dv*{E}{t}} = \frac{k^3}{12\pi \epso} \Im[ \vps \cross \vp ] \frac{12 \pi}{c^2 \Zo k^4 \abs{\vp}^2}
		= \frac{1}{\epso} \Im[ \vps \cross \vp ] \frac{1}{c^2 \Zo k \abs{\vp}^2}
		= \ans{ \frac{\Im[ \vps \cross \vp ]}{\omg \abs{\vp}^2}, }
	}
	where we have used $\Zo = \sqrt{\muo / \epso} = 1 / \sqrt{\epso^2 c^2} = 1 / \epso c$, $c^2 = 1 / (\epso \muo)$, and $\omg = k c$.
	
	In the limit of high frequency, $(\dv*{\vL}{t}) / (\dv*{E}{t}) \to 0$.  In this scenario, the energy radiated dominates over the angular momentum radiated.  Likewise, in the limit of low frequency, $(\dv*{\vL}{t}) / (\dv*{E}{t}) \to \infty$, meaning that angular momentum radiation dominates.  This is sensible because rotational kinetic energy $E \propto \omg^2$, while angular momentum $L \propto \omg$.
}

%
%	Jackson 9.8(c)
%

\prob{}{
	For a charge $e$ rotating in the $xy$ plane at radius $a$ and angular speed $\omg$, show that there is only a $z$ component of radiated angular momentum with magnitude $\dv*{\Lz}{t} = e^2 k^3 a^2 / 6 \pi \epso$.  What about a charge oscillating along the $z$ axis?
}

\sol{
	We know from Homework~5 that the position of a point charge rotating counterclockwise in the $xy$ plane is
	\eq{
		\vx(t) = a \cos(\omg t) \,\vx + a \sin(\omg t) \,\yh.
	}
	\clearpage
	Then the charge distribution is
	\eq{
		\rho(\vx, t) = e \del[ x - a \cos(\omg t) ] \,\del[ y - a \sin(\omg t) ] \,\del(z).
	}
	
	According to Jackson~(4.8), the dipole moment is defined
	\eq{
		\vp = \int \vx' \,\rho(\vx') \ddcxp.
	}
	The components of $\vp$ for the point charge are then
	\al{
		\px &= e \iiint x \,\del[ x - a \cos(\omg t) ] \,\del[ y - a \sin(\omg t) ] \,\del(z) \ddx \ddy \ddz
		= e a \cos(\omg t), \\
		\py &= e \iiint y \,\del[ x - a \cos(\omg t) ] \,\del[ y - a \sin(\omg t) ] \,\del(z) \ddx \ddy \ddz
		= e a \sin(\omg t), \\
		\pz &= e \iiint z \,\del[ x - a \cos(\omg t) ] \,\del[ y - a \sin(\omg t) ] \,\del(z) \ddx \ddy \ddz
		= 0,
	}
	so we can write $\vp = e a \,e^{-i \omg t} (\xh + i\,\yh).$  Substituting into Eq.~\refeq{ans1a},
	\al{
		\dv{\vL}{t} &= \Re\!\brac{ \frac{i k^3}{12\pi \epso} e^2 a^2 e^{-i \omg t} e^{i \omg t} [ (\xh + i\,\yh) \cross (\xh - i\,\yh) ] }
		= \Re\!\brac{ \frac{i e^2 k^3 a^2}{12\pi \epso} (-2i \,\xh \cross \yh) }
		= \Re\!\brac{ \frac{e^2 k^3 a^2}{6\pi \epso} \,\zh } \\
		&= \ans{ \frac{e^2 k^3 a^2}{6\pi \epso} \cos(\omg t) \,\zh, }
	}
	as desired. \qed
	
	A charge oscillating along the $z$ axis with amplitude $a$ has the charge density
	\eq{
		\rho(\vx, t) = e a \,\del(x) \,\del(y) \,\del[ z - \cos(\omg t) ],
	}
	which gives the dipole moment
	\al{
		\px &= e a \iiint x \,\del(x) \,\del(y) \,\del[ z - \cos(\omg t) ] \ddx \ddy \ddz
		= 0, \\
		\py &= e a \iiint y \,\del(x) \,\del(y) \,\del[ z - \cos(\omg t) ] \ddx \ddy \ddz
		= 0, \\
		\pz &= e a \iiint z \,\del(x) \,\del(y) \,\del[ z - \cos(\omg t) ] \ddx \ddy \ddz
		= e a \cos(\omg t).
	}
	In complex notation, $\vp = e a \,e^{-i\omg t} \,\zh$.  Substituting into Eq.~\refeq{ans1a}, we find
	\eq{
		\dv{\vL}{t} = \Re\!\brac{ \frac{i k^3}{12\pi \epso} e^2 a^2 e^{-i \omg t} e^{i \omg t} (\zh \cross \zh) }
		= \ans{ \vO. }
	}
	So we see that a charge undergoing linear motion does not lead to a radiated angular momentum, which is sensible.
}

%
%	Jackson 9.8(d)
%

\prob{}{
	What are the results corresponding to Probs.~{1(a)} and {1(b)} for magnetic dipole radiation?
}

\sol{
	The radiation fields for a magnetic dipole are given by Jackson~(19.35--36),
	\al{
		\vH &= \frac{1}{4\pi} \curly{ k^2 (\nh \cross \vm) \cross \nh \frac{e^{i k r}}{r} + [ 3 \nh (\nh \vdot \vm) - \vm ] \paren{ \frac{1}{r^3} - \frac{i k}{r^2} } e^{i k r} }, &
		\vE &= -\frac{\Zo}{4\pi} k^2 (\nh \cross \vm) \frac{e^{i k r}}{r} \paren{ 1 - \frac{1}{i k r} }.
	}
	\clearpage
	Comparing with Eq.~\refeq{fields1}, we see that $\vH \to -\vE / \Zo$, $\vE \to \Zo \vH$, and $\vp \to \vm / c$ as stated in the book~\cite[p.~413]{Jackson}.  Making these substitutions, the results of Probs.~{1.1(a)} and {(b)} become
	\al{
		\ans{ \dv{\vL}{t}\ }&\ans{= \frac{\muo k^3}{12\pi} \Im[ \vms \cross \vm ], } &
		\ans{ \frac{\dv*{\vL}{t}}{\dv*{E}{t}}\ }&\ans{= \frac{\Im[ \vms \cross \vm ]}{\omg \abs{\vm}^2} }
	}
	where we have used $\mu = 1 / \epso c^2$.
}


\newcommand{\Ha}{H_a}
\newcommand{\Hb}{H_b}
\newcommand{\Hi}{H_i}
\newcommand{\vp}{\vb{p}}
\newcommand{\vpi}{\vp_i}
\newcommand{\alp}{\alpha}
\newcommand{\ri}{r_i}
\newcommand{\ra}{r_a}
\newcommand{\rb}{r_b}
\newcommand{\rab}{r_{a b}}
\newcommand{\vx}{\vb{x}}
\newcommand{\vxi}{\vx_i}
\newcommand{\vxa}{\vx_a}
\newcommand{\vxb}{\vx_b}

\begin{statement}{}
	Consider a system of two electrons, which is described by the Hamiltonian
	\begin{align*}
		H &= \Ha + \Hb + V, &
		\Hi &= \frac{\vpi^2}{2m} - \frac{Z \alp \hbar c}{\ri}, &
		V = \frac{\alp \hbar c}{\rab}.
	\end{align*}
	Here, we label two electrons by $i = a, b$; $\ri = \abs{\vxi}$ and $\rab = \abs{\vxa - \vxb}$ where $\vxi$ is the spatial coordinate for electron $i$; and $Z$ and $\alp$ are constants.  To find an approximate ground state of $H$, let us try a variational wave function
	\beq
		\Psi(\vxa, \vxb) = \frac{A}{4\pi} e^{-B(\ra + \rb)},
	\eeq
	where $A$ is a normalization constant and $B$ is your variational parameter.
\end{statement}


\newcommand{\Hbar}{\bar{H}}
\newcommand{\ot}{\tilde{0}}
\newcommand{\kot}{\ket{\ot}}
\newcommand{\vpa}{\vp_a}
\newcommand{\vpb}{\vp_b}
\newcommand{\vr}{\vb{r}}
\newcommand{\vra}{\vec{r}_a}
\newcommand{\vrb}{\vec{r}_b}
\newcommand{\dcxa}{\dd[3]{\vxa}}
\newcommand{\dcxb}{\dd[3]{\vxb}}
\newcommand{\dcxap}{\dd[3]{\vxa'}}
\newcommand{\dcxbp}{\dd[3]{\vxb'}}
\newcommand{\dra}{\dd{\ra}}
\newcommand{\drb}{\dd{\rb}}
\newcommand{\drap}{\dd{\ra'}}
\newcommand{\drbp}{\dd{\rb'}}
\newcommand{\dr}{\dd{r}}
\newcommand{\tht}{\theta}
\newcommand{\tha}{\tht_a}
\newcommand{\dcta}{\dd{(\cos\tha)}}
\newcommand{\intono}{\int_{-1}^1}
\newcommand{\intab}{\int_{(\ra - \rb)^2}^{(\ra + \rb)^2}}

\begin{problem}
	Compute the variational energy for the given variational parameter $B$.
\end{problem}

\begin{solution}
	The general expression for the variational energy $\Hbar$ is (5.4.1) in Sakurai:
	\beqn \label{Hbar}
		\Hbar = \frac{\ev{H}{\ot}}{\braket{\ot}},
	\eeqn
	where $\kot$ is our trial ket.
	
	For this problem, the numerator of \refeq{Hbar} is
	\begin{align*}
		\ev{H}{\ot} &= \ev{H}{\Psi}
		= \iint \braket{\Psi}{\vxa, \vxb} \mel{\vxa, \vxb}{H}{\vxa', \vxb'} \braket{\vxa', \vxb'}{\Psi} \\
		&= \iint \iint \Psi(\vxa, \vxb) \mel{\vxa, \vxb}{H}{\vxa', \vxb'} \Psi(\vxa', \vxb') \dcxa \dcxb \dcxap \dcxbp,
	\end{align*}
	where
	\beq
		H = \frac{\vpa^2}{2m} + \frac{\vpb^2}{2m} - \frac{Z \alp \hbar c}{\abs{\vxa}} - \frac{Z \alp \hbar c}{\abs{\vxb}} + \frac{\alp \hbar c}{\abs{\vxa - \vxb}},
	\eeq
	so we have five integrals.  For the first,
	\begin{align*}
		\frac{A^2}{32 \pi^2 m} &\iint \iint e^{-B(\ra + \rb)} \mel{\vxa, \vxb}{\vpa^2}{\vxa', \vxb'}^2 e^{-B(\ra' + \rb')} \dcxa \dcxb \dcxap \dcxbp \\
		&= \frac{A^2}{32 \pi^2 m} \iint \iint e^{-B(\ra + \rb)} \bigg( i^2 \hbar^2 \delta(\vxa - \vxa') \delta(\vxb - \vxb') \laplacian_{a'} \bigg) e^{-B(\ra' + \rb')} \dcxa \dcxb \dcxap \dcxbp \\
		&= -\frac{A^2 \hbar^2}{2 m} \iint e^{-B(\ra + \rb)} \left( \pdv[2]{}{\ra} e^{-B(\ra + \rb)} \right) \ra^2 \rb^2 \dra \drb
		= -\frac{A^2 B^2 \hbar^2}{2 m} \intoi \ra^2 e^{-2 B \ra} \dra \intoi \rb^2 e^{-2 B \rb} \drb \\
		&= -\frac{A^2 \hbar^2}{32 B^4 m},
	\end{align*}
	where we have used
	\begin{align*}
		\intoi r^2 e^{-2B r} \dr &= \left[ -\frac{r^2 e^{-2 B r}}{2 B} \right]_0^\infty + \frac{1}{B} \intoi r e^{-2 B r} \dr
		= \frac{1}{B} \left[ -\frac{r e^{-2 B r}}{2 B} \right]_0^\infty + \frac{1}{2 B^2} \intoi e^{-2 B r} \dr
		= \frac{1}{2 B^2} \left[ -\frac{e^{-2 B r}}{2 B} \right]_0^\infty \\
		&= \frac{1}{4 B^3}.
	\end{align*}
	For the second integral, we also have
	\beq
		\frac{A^2}{16 \pi^2} \frac{1}{2 m} \iint \iint e^{-B(\ra + \rb)} \mel{\vxa, \vxb}{\vpb^2}{\vxa', \vxb'}^2 e^{-B(\ra' + \rb')} \dcxa \dcxb \dcxap \dcxbp = -\frac{A^2 \hbar^2}{32 B^4 m}.
	\eeq
	
	For the third integral,
	\begin{align*}
		-\frac{Z \alp \hbar c}{16\pi^2} &\iint \iint e^{-B(\ra + \rb)} \mel{\vxa, \vxb}{\frac{1}{\abs{\vxa}}}{\vxa', \vxb'} e^{-B(\ra' + \rb')} \dcxa \dcxb \dcxap \dcxbp \\
		&= -\frac{Z \alp \hbar c}{16\pi^2}\iint \iint e^{-B(\ra + \rb)} \left( \delta(\vxa - \vxa') \delta(\vxb - \vxb') \frac{1}{\abs{\vxa}} \right) e^{-B(\ra' + \rb')} \dra \drb \drap \drbp \\
		&= -A^2 Z \alp \hbar c \iint \frac{e^{-2B(\ra + \rb)}}{\ra} \ra^2 \rb^2 \dra \drb
		= -A^2 Z \alp \hbar c \intoi \ra e^{-2 B \ra} \dra \intoi \rb^2 e^{-2 B \rb} \drb \\
		&= -\frac{A^2 Z \alp \hbar c}{16 B^5}.
	\end{align*}
	For the fourth integral, we also have
	\beq
		-\frac{Z \alp \hbar c}{16\pi^2} \iint \iint e^{-B(\ra + \rb)} \mel{\vxa, \vxb}{\frac{1}{\abs{\vxb}}}{\vxa', \vxb'} e^{-B(\ra' + \rb')} \dcxa \dcxb \dcxap \dcxbp = -\frac{A^2 Z \alp \hbar c}{16 B^5}.
	\eeq
	
	For the fifth integral, we will orient our coordinate system such that $\vxb$ points in the $z$ direction and stipulate that $\ra > \rb$.  Then
	\beq
		\frac{1}{\abs{\vxa - \vxb}} = \frac{1}{\sqrt{\vxa^2 - 2 \vxa \cdot \vxb + \vxb^2}}
		= \frac{1}{\sqrt{\ra^2 - 2 \ra \rb \cos\tha + \rb^2}},
	\eeq
	and so
	\begin{align}
		\frac{\alp \hbar c}{16\pi^2} &\iint \iint e^{-B(\ra + \rb)} \mel{\vxa, \vxb}{\frac{1}{\abs{\vxa - \vxb}}}{\vxa', \vxb'} e^{-B(\ra' + \rb')} \dcxa \dcxb \dcxap \dcxbp \notag \\
		&= \frac{\alp \hbar c}{16\pi^2} \iint \iint e^{-B(\ra + \rb)}  \left( \delta(\vxa - \vxa') \delta(\vxb - \vxb') \frac{1}{\abs{\vxa - \vxb}} \right) e^{-B(\ra' + \rb')} \dcxa \dcxb \dcxap \dcxbp \notag \\
		&= \frac{\alp \hbar c}{2} \intoi \intono \intoi \frac{e^{-2B(\ra + \rb)}}{\sqrt{\ra^2 - 2 \ra \rb \cos\tha + \rb^2}} \ra^2 \rb^2 \dra \dcta \drb \notag \\
		&= \frac{\alp \hbar c}{2} \intoi \intoi \ra^2 \rb^2 e^{-2B(\ra + \rb)} \intono \frac{\dcta}{\sqrt{\ra^2 - 2 \ra \rb \cos\tha + \rb^2}} \dra \drb. \label{leftoff}
	\end{align}
	For the innermost integral, let $u = \ra^2 - 2 \ra \rb \cos\tha + \rb^2$.  Then
	\beq
		\dcta = -\frac{\du}{2 \ra \rb},
	\eeq
	and we are integrating from $\ra^2 + 2 \ra \rb + \rb^2 = (\ra + \rb)^2$ to $\ra^2 - 2 \ra \rb + \rb^2 = (\ra - \rb)^2$.  So the innermost integral becomes
	\begin{align*}
		\intono \frac{\dcta}{\sqrt{\ra^2 - 2 \ra \rb \cos\tha + \rb^2}} &= \frac{1}{2 \ra \rb} \intab \frac{\du}{\sqrt{u}}
		= \frac{1}{2 \ra \rb} \bigg[ 2 \sqrt{u} \bigg]_{(\ra - \rb)^2}^{(\ra + \rb)^2}
		= \frac{\abs{\ra + \rb} - \abs{\ra - \rb}}{\ra \rb} \\
		&= \frac{\ra + \rb - \ra + \rb}{\ra \rb}
		= \frac{2}{\ra},
	\end{align*}
	where we have used $\ra, \rb > 0$ and our assumption that $\ra > \rb$.  Picking up from \refeq{leftoff}, we now have
	\begin{align*}
		\frac{\alp \hbar c}{16\pi^2} &\iint \iint e^{-B(\ra + \rb)} \mel{\vxa, \vxb}{\frac{1}{\abs{\vxa - \vxb}}}{\vxa', \vxb'} e^{-B(\ra' + \rb')} \dcxa \dcxb \dcxap \dcxbp \\
		&= \alp \hbar c \intoi \ra e^{-2B \ra} \dra \intoi \rb^2 e^{-2B \rb} \drb
		= \frac{\alp \hbar c}{16 B^5}.
	\end{align*}
	
	Putting this all together,
	\beq
		\ev{H}{\ot} = \frac{\alp \hbar c}{16 B^5} - \frac{A^2 B \hbar^2 / m}{16 B^5} - \frac{2 A^2 Z \alp \hbar c}{16 B^5}
		= \frac{1}{16 B^5} \left( (1 - 2 A^2 Z) \alp \hbar c - \frac{A^2 B \hbar^2}{m} \right).
	\eeq
\end{solution}



\begin{problem}
	By minimizing the variational energy, find the optimal value of $B$.
\end{problem}



\newcommand{\px}{p_x}
\newcommand{\py}{p_y}
\newcommand{\vS}{\vb{S}}
\newcommand{\vSq}{\vS_1}
\newcommand{\vSw}{\vS_2}
\newcommand{\Sz}{S^z}
\newcommand{\Szq}{\Sz_1}
\newcommand{\Szw}{\Sz_2}

\begin{statement}{}
	Consider a two-dimensional harmonic oscillator described by the Hamiltonian
	\beq
		\Ho = \frac{\px^2 + \py^2}{2m} + m \omega^2 \frac{x^2 + y^2}{2}.
	\eeq
\end{statement}

\begin{problem}
	How many single-particle states are there for the first excited level?
\end{problem}

\begin{problem}
	Write down the many-body ground state for two electrons (with spin).  What is the eigenvalue of $(\vSq + \vSw)^2$ for this state?  Here $\vS_i$ are the spin operators of the electrons.
\end{problem}

\begin{problem}
	Write down all the first excited many-body states of two electrons (with spin).  Choose them to be eigenstates of the total spin operator, and compute their eigenvalues of $(\vSq + \vSw)^2$ and $\Szq + \Szw$ (where $\Sz_i$ is the $z$ componenet of the spin operator $\vS_i$).
\end{problem}



\vfill
I consulted Sakurai's \emph{Modern Quantum Mechanics}, Shankar's \emph{Principles of Quantum Mechanics}, and Wolfram MathWorld while writing up these solutions.

\end{document}