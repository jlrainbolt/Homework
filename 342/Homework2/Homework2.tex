 \documentclass[11pt]{article}
\usepackage{geometry, titlesec}
\usepackage[parfill]{parskip}
\usepackage[italicdiff]{physics}
\usepackage{amsfonts, amsthm}
\usepackage[cm]{fullpage}
\usepackage{fancyhdr}
\usepackage{enumitem}
\usepackage{xcolor, soul}
\usepackage{kbordermatrix}
%\allowdisplaybreaks

\makeatletter
\renewcommand*\env@cases[1][1.2]{%
  \let\@ifnextchar\new@ifnextchar
  \left\lbrace
  \def\arraystretch{#1}%
  \array{@{}l@{\quad}l@{}}%
}
\makeatother

 
\renewcommand{\footrulewidth}{.2pt}
%\setlist[enumerate]{leftmargin=*}
\pagestyle{fancy}
\fancyhf{}
\lhead{\textbf{Physics 342 Homework 1}}
\rhead{Lacey Rainbolt}
\setlength{\headheight}{11pt}
\setlength{\headsep}{11pt}
\setlength{\footskip}{24pt}
\lfoot{\today}
\rfoot{\thepage}

\titleformat{\section}[runin]{\normalfont\large\bfseries}{Problem \thesection.}{1em}{}
\titleformat{\subsection}[runin]{\normalfont\large\bfseries}{\thesubsection}{1em}{}
\titleformat{\subparagraph}[leftmargin]{\normalfont\normalsize\bfseries}{}{0pt}{}

\newcommand{\refeq}[1]{(\ref{#1})}

\newcommand{\beq}{\begin{equation*}}
\newcommand{\eeq}{\end{equation*}}

\newcommand{\beqn}{\begin{equation}}
\newcommand{\eeqn}{\end{equation}}


\renewcommand{\vec}[1]{\mathbf{#1}}
\newcommand{\vfix}{\vspace{-\baselineskip}}


\newenvironment{statement}[1]
{
	\section{#1}
	\color{darkgray}
	\ignorespaces
}
{
%    \smallskip
}

\newenvironment{problem}
{
    \color{darkgray}
    \subsection{}
    \ignorespaces
}


\newenvironment{solution}
{
    \paragraph{Solution.}
    \ignorespaces
}
{
%    \smallskip
}

\newcommand{\Schrodinger}{Schr\"{o}dinger}


\begin{document}

\state{Spin-wave theory~(P\&S 11.1)}{\hfix}

\prob{ \label{1a}
	Prove the following wonderful formula: Let $\phix$ be a free scalar field with propagator $\ev{T \phix \phio} = \Dx$.  Then
	\eqn{show1}{
		\ev{ T e^{i \phix} e^{-i \phio} } = e^{[ \Dx - \Do ]}.
	}
	(The  factor $\Do$ gives a formally divergent adjustment of the overall normalization.)
}

\sol{
	According to P\&S~(9.18),
	\eq{
		\ev*{T \phi(\xq) \phi(\xw)}{\Omg} = \frac{\int \DDphi \phi(\xq) \phi(\xw) \exp[ i \int \ddqx \cL ]}{\int \DDphi \exp[ i \int \ddqx \cL ]}.
	}
	We use this expression to write the left-hand side of Eq.~\refeq{show1}:
	\eqn{thing1}{
		\ev{ T e^{i \phix} e^{-i \phio} } = \frac{\int \DDphi e^{i \phix} e^{-i \phio} \exp[ i \int \ddqy \cL ]}{\int \DDphi \exp[ i \int \ddqy \cL ]}
		= \frac{\int \DDphi \exp[i \phix - i \phio + i \int \ddqy \cL ]}{\int \DDphi \exp[ i \int \ddqy \cL ]}.
	}
	For a free Klein-Gordon~(i.e., scalar) field, Eq.~(9.39) tells us that the generating functional $\ZJ$ is given by
	\eq{
		\ZJ = \Zo \exp[ -\frac{1}{2} \int \ddqx \ddqy \Jx \DF(x - y) \Jy ],
	}
	where $\Zo = Z[0]$.  Thus, we want to find some $\Jy$ such that
	\eqn{thing1b}{
		\ev{ T e^{i \phix} e^{-i \phio} } = \frac{\ZJ}{\Zo}
	}
	where in general
	\eq{
		\ZJ = \int \DDphi \exp[ i \int \ddqx [ \cL + \Jx \phi(x) ] ]
	}
	by (9.34).  Inspecting Eq.~\refeq{thing1}, we recognize the denominator as $\Zo$ and see that if
	\eq{
		\Jy = \delq(y - x) - \delq(y)
	}
	we have an expression like Eq.~\refeq{thing1b}.  Collecting these findings, we have
	\al{
		\ans{ \ev{ T e^{i \phix} e^{-i \phio} } }&= \frac{\ZJ}{\Zo} \\
		&= \exp[ -\frac{1}{2} \int \ddqy \ddqz \Jy \DF(y - z) \Jz ] \\
		&= \exp[ -\frac{1}{2} \int \ddqy \ddqz \Jy \DF(y - z) [ \delq(z - x) - \delq(z) ] ] \\
		&= \exp[ -\frac{1}{2} \int \ddqy [ \delq(y - x) - \delq(y) ] [ \DF(y - x) - \DF(y) ] ] \\
		&= \exp[ -\frac{1}{2} [ \DF(0) - \DF(x) - \DF(-x) + \DF(0) ] ] \\
		&= \exp[ \DF(x) - \DF(0) ] \\
		&\ans{\; = e^{[ \Dx - \Do ]}, }
	}
	as we wanted to show. \qed
}



\prob{ \label{1b}
	We can use this formula in Euclidean field theory to discuss correlation functions in a theory with spontaneously broken symmetry for $T < \TC$.  Let us consider only the simplest case of a broken $O(2)$ or $U(1)$ symmetry.  We can write the local spin density as a complex variable
	\eq{
		\sx = \sqx + i \swx.
	}
	The global symmetry is the transformation
	\eq{
		\sx \to e^{-i \alp} \sx.
	}
	If we assume that the physics freezes the modulus of $\sx$, we can parameterize
	\eqn{sx}{
		\sx = A e^{i \phix}
	}
	and write an effective Lagrangian for the field $\phix$.  The symmetry of the theory becomes the translation symmetry
	\eqn{symmetry}{
		\phix \to \phix - \alp.
	}
	Show that (for $d > 0$) the most general renormalizable Lagrangian consistent with this symmetry is the free field theory
	\eqn{show1b}{
		\cL = \frac{1}{2} \rho(\vgrad \phi)^2.
	}
	In statistical mechanics, the constant $\rho$ is called the \emph{spin wave modulus}.  A reasonable hypothesis for $\rho$ is that it is finite for $T < \TC$ and tends to 0 as $T \to \TC$ from below.
}

\sol{
	In accordance with the Klein-Gordon Lagrangian in P\&S~(2.6),
	\eqn{KGL}{
		\cL_\text{K-G} = \frac{1}{2} (\pt \phi)^2 - \frac{1}{2} m^2 \phi^2,
	}
	we interpret $(\vgrad \phi)^2$ as $(\pt \phi)^2$.
	
	The Lagrangian cannot have terms of $\order{\phi^n}$ for any $n \neq 0$ since $\phi(x)$ is not invariant under Eq.~\refeq{symmetry}.  Any combination of derivatives of $\phi$ is invariant, however, since $\alp$ is a constant and does not contribute to any derivative.  Thus, only terms like $\pt^n \phi^m$ (where $n$ denotes a power of $\pt$) for $n, m > 0$ and $n \geq m$ are consistent with the symmetry of Eq.~\refeq{symmetry} for $d$ an integer.
	
	Now we must determine which of these terms are renormalizable.  We know that the Lagrangian must have dimension $d$, and that $\phi$ has dimension $(d - 2) / 2$.  Taking a derivative adds a mass dimension.  The theory is renormalizable if the coupling constant $\rho$ has dimension greater than or equal to 0~\cite[p.~322]{Peskin}.  Let $p$ be the dimension of $\rho$.  The dimension of our allowed term is then
	\eq{
		[ \rho \pt^n \phi^m ] = p + n + m \frac{d - 2}{2},
	}
	which we require to be equal to $d$.  Thus we seek solutions to the system of equations
	\al{
		d &= p + n + m \frac{d - 2}{2}, &
		n &\geq m, &
		p &\geq 0.
	}
	Solving with Mathematica, we find that this system has two solutions: $n = m = 2$ and $p = 0$; and $n = m = 1$ and $p = d / 2$.  However, the term $\pt \phi$ for $n = m = 1$ does not contribute to the action because it is a total derivative and does not contribute when the integral over $\cL$ is evaluated:
	\eq{
		\int \dd[d]{x} \pt\phi = \phi \bigg|_{-\infty}^\infty
		= 0.
	}
	Thus the only possibility is $n = m = 2$.  Note that
	\eq{
		\pt^2 \phi^2 = \pt(\pt \phi^2)
		= 2 \pt( \phi \pt \phi)
		= \pt \phi \pt \phi + \phi \pt^2 \phi
		= (\pt \phi)^2,
	}
	since $\phi \pt^2 \phi$ is not invariant under Eq.~\refeq{sx}.  This means that $\rho$ must be dimensionless and that the only allowed terms in the Lagrangian are proportional to $(\pt \phi)^2$, which is consistent with Eq.~\refeq{show1b}. \qed
}



\prob{
	Compute the correlation function $\ev{ \sx \sao }$.  Adjust $A$ to give a physically sensible normalization (assuming that the system has a physical cutoff at the scale of one atomic spacing) and display the dependence of this correlation function on $x$ for $d = 1, 2, 3, 4$.  Explain the significance of your results.
}

\sol{
	Applying Eq.~\refeq{sx},
	\eq{
		\ev{ \sx \sao } = \ev*{ A e^{i \phix} \As e^{-i \phio} }
		= \ev*{ \abs{A}^2 } \ev*{ e^{i \phix} e^{-i \phio} }.
	}
	Now we can apply Eq.~\refeq{show1} to find
	\eqn{thing1c}{
		\ans{ \ev{ \sx \sao } = \abs{A}^2 \exp[ D(x) - D(0) ], }
	}
	where $D(x - y)$ is a Green's function.  Since our Lagrangian is similar to the Klein-Gordon Lagrangian Eq.~\refeq{2.6}, our Green's function is similar to that of the Klein-Gordon operator, which is given by P\&S~(2.56):
	\eq{
		(\pt^2 + m^2) D(x - y) = -i \delq(x - y).
	}
	The Feynman prescription for this Green's function is given by (2.59),
	\eqn{DF}{
		\DF(x - y) = \int \ddqpf \frac{i}{p^2 - m^2 + i \eps} e^{-i p \cdot (x - y)}.
	}
	For the Lagrangian in Eq.~\refeq{show1b}, we set $m = 0$ and insert a factor of $\rho$:
	\eq{
		\rho \pt^2 D(x - y) = -i \deld(x - y),
	}
	so adapting Eq.~\refeq{DF} for this situation yields
	\eqn{DF}{
		\DF(x - y) = \frac{1}{\rho} \int \dddpf \frac{i}{p^2 + i \eps} e^{-i p \cdot (x - y)}.
	}
	We see that $\DF(0)$ diverges, so we absorb it into the constant to make the normalization physically sensible.  We can do this because, as we showed in \ref{1b}, the theory is renormalizable.  Define $A'$ such that
	\eq{
		{A'}^2 = \abs{A}^2 e^{-D(0)}.
	}
	Then Eq.~\refeq{thing1c} can be written
	\eq{
		\ans{ \ev{ \sx \sao } =  {A'}^2 e^{D(x)}. }
	}
	
	To evaluate the divergent integral $D(x)$, we look to the Feynman parameter method we have been using to solve divergent integrals.  Apparently, the Schwinger parametrization is useful in deriving the Feynman parametrization, and it is given by~\cite{Feynman}
	\eq{
		\frac{1}{A} = \intoi \dds e^{-s A}.
	}
	Using this equation, we can write Eq.~\refeq{DF} as
	\eq{
		\DF(x) = \frac{1}{\rho} \int \dddpf \frac{i}{p^2} e^{-i p \cdot x}
		= \frac{i}{\rho} \int \dddpf \intoi \dds e^{-s p^2} e^{-i p \cdot x}.
	}
	Now we can complete the square in the exponential to get a Gaussian integral:
	\al{
		\DF(x) &= \frac{i}{\rho} \int \dddpf \intoi \dds \exp[ -s p^2 - i p \cdot x + \frac{x^2}{4 s} - \frac{x^2}{4 s} ] \\
		&= \frac{i}{\rho} \int \dddpf \intoi \dds \exp[ -s \paren{ p + \frac{i x}{2 s} }^2 - \frac{x^2}{4 s} ] \\
		&= \frac{i}{\rho (2 \pi)^d} \intoi \dds e^{-x^2 / 4 s} \int \dd[d]{u} e^{-s u^2} \\
		&= \frac{i}{\rho (2 \pi)^{d}} \intoi \dds e^{-x^2 / 4 s} \sqrt{ \frac{(2\pi)^d}{(2s)^d} } \\
		&= \frac{i}{\rho (4 \pi)^{d / 2}} \intoi \dds \frac{e^{-x^2 / 4 s}}{s^{d / 2}}
	}
	where we have used~\cite{QFT}
	\eq{
		\int \exp( -\frac{1}{2} x \cdot A \cdot x ) \dd[n]{x} = \sqrt{\frac{(2\pi)^n}{\det A}},
	}
	with $A$ a $d \times d$ diagonal matrix $2s$.  Using Mathematica to integrate with respect to $s$, we find
	\eq{
		\DF(x) = \frac{i}{\rho (4 \pi)^{d / 2}} \frac{2^{d - 2}}{x^{d - 2}} \Gam(d / 2 - 1)
		= \frac{i}{4 \pi^d \rho} \Gam(d / 2 - 1) x^{2 - d}.
	}
	The gamma function diverges as $d \to 2$, so as we have done in previous problems, we expand about $\eps = 2 - d$.  Evaluating the series expansion using Mathematica, we obtain
	\eq{
		\DF(x) = \frac{i}{4 \pi^{1 - \eps} \rho} \Gam(\eps / 2) x^\eps
		\approx \frac{i}{4 \pi \rho} \paren{ \frac{2}{\eps} - \gam + 2 \ln(\pi x) }
		\sim \frac{i}{2 \pi \rho} \ln(x)
		= i \ln(\frac{1}{x^{2 \pi \rho}}).
	}
	We Wick rotate $x \to i x$.  Then the dependence of the correlation function on $x$ for $d = 1, 2, 3, 4$ is
	\ans{\al{
		(d = 1) &\qquad \ev{ \sx \sao } \sim e^{-x / 2 \sqrt{\pi} \rho}, &
		(d = 2) &\qquad \ev{ \sx \sao } \sim x^{2 \pi \rho}, \\
		(d = 3) &\qquad \ev{ \sx \sao } \sim \frac{1}{x}, &
		(d = 4) &\qquad \ev{ \sx \sao } \sim \frac{1}{x^2}.
	}}%
	In $d > 2$ dimensions, the expectation value of the correlation function tends to 0 at large distances $x$.  For $d > 2$, it drops off more quickly as $d$ increases.  The $d \leq 2$ cases depend on $\rho$, which we assume is positive.  The $d = 1$ case drops off with increasing distance, and more quickly with smaller $\rho$.  For $d = 2$, the expectation value of the correlation function increases with increasing distance, and it blows up more quickly with larger $\rho$.
	
	These results are consistent with the Mermin--Wagner theorem, which states that a continuous symmetry cannot be broken in $d \leq 2$ dimensions~\cite{CMW}.  That is, in $d \leq 2$ dimensions, a symmetry-breaking field cannot have a nonzero vacuum expectation value~\cite[p.~460]{Peskin}.  A physical explanation is that each spin has more nearest neighbors in higher dimensions.  Since the spins are inclined to align with their neighbors, there is a higher degree of correlation in higher dimensions at the same distance.  In two dimensions, the correlations are weak enough that they are overpowered by the field fluctuations.
}


\newcommand{\Ha}{H_a}
\newcommand{\Hb}{H_b}
\newcommand{\Hi}{H_i}
\newcommand{\vp}{\vb{p}}
\newcommand{\vpi}{\vp_i}
\newcommand{\alp}{\alpha}
\newcommand{\ri}{r_i}
\newcommand{\ra}{r_a}
\newcommand{\rb}{r_b}
\newcommand{\rab}{r_{a b}}
\newcommand{\vx}{\vb{x}}
\newcommand{\vxi}{\vx_i}
\newcommand{\vxa}{\vx_a}
\newcommand{\vxb}{\vx_b}

\begin{statement}{}
	Consider a system of two electrons, which is described by the Hamiltonian
	\begin{align*}
		H &= \Ha + \Hb + V, &
		\Hi &= \frac{\vpi^2}{2m} - \frac{Z \alp \hbar c}{\ri}, &
		V = \frac{\alp \hbar c}{\rab}.
	\end{align*}
	Here, we label two electrons by $i = a, b$; $\ri = \abs{\vxi}$ and $\rab = \abs{\vxa - \vxb}$ where $\vxi$ is the spatial coordinate for electron $i$; and $Z$ and $\alp$ are constants.  To find an approximate ground state of $H$, let us try a variational wave function
	\beq
		\Psi(\vxa, \vxb) = \frac{A}{4\pi} e^{-B(\ra + \rb)},
	\eeq
	where $A$ is a normalization constant and $B$ is your variational parameter.
\end{statement}


\newcommand{\Hbar}{\bar{H}}
\newcommand{\ot}{\tilde{0}}
\newcommand{\kot}{\ket{\ot}}
\newcommand{\vpa}{\vp_a}
\newcommand{\vpb}{\vp_b}
\newcommand{\vr}{\vb{r}}
\newcommand{\vra}{\vec{r}_a}
\newcommand{\vrb}{\vec{r}_b}
\newcommand{\dcxa}{\dd[3]{\vxa}}
\newcommand{\dcxb}{\dd[3]{\vxb}}
\newcommand{\dcxap}{\dd[3]{\vxa'}}
\newcommand{\dcxbp}{\dd[3]{\vxb'}}
\newcommand{\dra}{\dd{\ra}}
\newcommand{\drb}{\dd{\rb}}
\newcommand{\drap}{\dd{\ra'}}
\newcommand{\drbp}{\dd{\rb'}}
\newcommand{\dr}{\dd{r}}
\newcommand{\tht}{\theta}
\newcommand{\tha}{\tht_a}
\newcommand{\dcta}{\dd{(\cos\tha)}}
\newcommand{\intono}{\int_{-1}^1}
\newcommand{\intab}{\int_{(\ra - \rb)^2}^{(\ra + \rb)^2}}

\begin{problem}
	Compute the variational energy for the given variational parameter $B$.
\end{problem}

\begin{solution}
	The general expression for the variational energy $\Hbar$ is (5.4.1) in Sakurai:
	\beqn \label{Hbar}
		\Hbar = \frac{\ev{H}{\ot}}{\braket{\ot}},
	\eeqn
	where $\kot$ is our trial ket.
	
	For this problem, the numerator of \refeq{Hbar} is
	\begin{align*}
		\ev{H}{\ot} &= \ev{H}{\Psi}
		= \iint \braket{\Psi}{\vxa, \vxb} \mel{\vxa, \vxb}{H}{\vxa', \vxb'} \braket{\vxa', \vxb'}{\Psi} \\
		&= \iint \iint \Psi(\vxa, \vxb) \mel{\vxa, \vxb}{H}{\vxa', \vxb'} \Psi(\vxa', \vxb') \dcxa \dcxb \dcxap \dcxbp,
	\end{align*}
	where
	\beq
		H = \frac{\vpa^2}{2m} + \frac{\vpb^2}{2m} - \frac{Z \alp \hbar c}{\abs{\vxa}} - \frac{Z \alp \hbar c}{\abs{\vxb}} + \frac{\alp \hbar c}{\abs{\vxa - \vxb}},
	\eeq
	so we have five integrals.  For the first,
	\begin{align*}
		\frac{A^2}{32 \pi^2 m} &\iint \iint e^{-B(\ra + \rb)} \mel{\vxa, \vxb}{\vpa^2}{\vxa', \vxb'}^2 e^{-B(\ra' + \rb')} \dcxa \dcxb \dcxap \dcxbp \\
		&= \frac{A^2}{32 \pi^2 m} \iint \iint e^{-B(\ra + \rb)} \bigg( i^2 \hbar^2 \delta(\vxa - \vxa') \delta(\vxb - \vxb') \laplacian_{a'} \bigg) e^{-B(\ra' + \rb')} \dcxa \dcxb \dcxap \dcxbp \\
		&= -\frac{A^2 \hbar^2}{2 m} \iint e^{-B(\ra + \rb)} \left( \pdv[2]{}{\ra} e^{-B(\ra + \rb)} \right) \ra^2 \rb^2 \dra \drb
		= -\frac{A^2 B^2 \hbar^2}{2 m} \intoi \ra^2 e^{-2 B \ra} \dra \intoi \rb^2 e^{-2 B \rb} \drb \\
		&= -\frac{A^2 \hbar^2}{32 B^4 m},
	\end{align*}
	where we have used
	\begin{align*}
		\intoi r^2 e^{-2B r} \dr &= \left[ -\frac{r^2 e^{-2 B r}}{2 B} \right]_0^\infty + \frac{1}{B} \intoi r e^{-2 B r} \dr
		= \frac{1}{B} \left[ -\frac{r e^{-2 B r}}{2 B} \right]_0^\infty + \frac{1}{2 B^2} \intoi e^{-2 B r} \dr
		= \frac{1}{2 B^2} \left[ -\frac{e^{-2 B r}}{2 B} \right]_0^\infty \\
		&= \frac{1}{4 B^3}.
	\end{align*}
	For the second integral, we also have
	\beq
		\frac{A^2}{16 \pi^2} \frac{1}{2 m} \iint \iint e^{-B(\ra + \rb)} \mel{\vxa, \vxb}{\vpb^2}{\vxa', \vxb'}^2 e^{-B(\ra' + \rb')} \dcxa \dcxb \dcxap \dcxbp = -\frac{A^2 \hbar^2}{32 B^4 m}.
	\eeq
	
	For the third integral,
	\begin{align*}
		-\frac{Z \alp \hbar c}{16\pi^2} &\iint \iint e^{-B(\ra + \rb)} \mel{\vxa, \vxb}{\frac{1}{\abs{\vxa}}}{\vxa', \vxb'} e^{-B(\ra' + \rb')} \dcxa \dcxb \dcxap \dcxbp \\
		&= -\frac{Z \alp \hbar c}{16\pi^2}\iint \iint e^{-B(\ra + \rb)} \left( \delta(\vxa - \vxa') \delta(\vxb - \vxb') \frac{1}{\abs{\vxa}} \right) e^{-B(\ra' + \rb')} \dra \drb \drap \drbp \\
		&= -A^2 Z \alp \hbar c \iint \frac{e^{-2B(\ra + \rb)}}{\ra} \ra^2 \rb^2 \dra \drb
		= -A^2 Z \alp \hbar c \intoi \ra e^{-2 B \ra} \dra \intoi \rb^2 e^{-2 B \rb} \drb \\
		&= -\frac{A^2 Z \alp \hbar c}{16 B^5}.
	\end{align*}
	For the fourth integral, we also have
	\beq
		-\frac{Z \alp \hbar c}{16\pi^2} \iint \iint e^{-B(\ra + \rb)} \mel{\vxa, \vxb}{\frac{1}{\abs{\vxb}}}{\vxa', \vxb'} e^{-B(\ra' + \rb')} \dcxa \dcxb \dcxap \dcxbp = -\frac{A^2 Z \alp \hbar c}{16 B^5}.
	\eeq
	
	For the fifth integral, we will orient our coordinate system such that $\vxb$ points in the $z$ direction and stipulate that $\ra > \rb$.  Then
	\beq
		\frac{1}{\abs{\vxa - \vxb}} = \frac{1}{\sqrt{\vxa^2 - 2 \vxa \cdot \vxb + \vxb^2}}
		= \frac{1}{\sqrt{\ra^2 - 2 \ra \rb \cos\tha + \rb^2}},
	\eeq
	and so
	\begin{align}
		\frac{\alp \hbar c}{16\pi^2} &\iint \iint e^{-B(\ra + \rb)} \mel{\vxa, \vxb}{\frac{1}{\abs{\vxa - \vxb}}}{\vxa', \vxb'} e^{-B(\ra' + \rb')} \dcxa \dcxb \dcxap \dcxbp \notag \\
		&= \frac{\alp \hbar c}{16\pi^2} \iint \iint e^{-B(\ra + \rb)}  \left( \delta(\vxa - \vxa') \delta(\vxb - \vxb') \frac{1}{\abs{\vxa - \vxb}} \right) e^{-B(\ra' + \rb')} \dcxa \dcxb \dcxap \dcxbp \notag \\
		&= \frac{\alp \hbar c}{2} \intoi \intono \intoi \frac{e^{-2B(\ra + \rb)}}{\sqrt{\ra^2 - 2 \ra \rb \cos\tha + \rb^2}} \ra^2 \rb^2 \dra \dcta \drb \notag \\
		&= \frac{\alp \hbar c}{2} \intoi \intoi \ra^2 \rb^2 e^{-2B(\ra + \rb)} \intono \frac{\dcta}{\sqrt{\ra^2 - 2 \ra \rb \cos\tha + \rb^2}} \dra \drb. \label{leftoff}
	\end{align}
	For the innermost integral, let $u = \ra^2 - 2 \ra \rb \cos\tha + \rb^2$.  Then
	\beq
		\dcta = -\frac{\du}{2 \ra \rb},
	\eeq
	and we are integrating from $\ra^2 + 2 \ra \rb + \rb^2 = (\ra + \rb)^2$ to $\ra^2 - 2 \ra \rb + \rb^2 = (\ra - \rb)^2$.  So the innermost integral becomes
	\begin{align*}
		\intono \frac{\dcta}{\sqrt{\ra^2 - 2 \ra \rb \cos\tha + \rb^2}} &= \frac{1}{2 \ra \rb} \intab \frac{\du}{\sqrt{u}}
		= \frac{1}{2 \ra \rb} \bigg[ 2 \sqrt{u} \bigg]_{(\ra - \rb)^2}^{(\ra + \rb)^2}
		= \frac{\abs{\ra + \rb} - \abs{\ra - \rb}}{\ra \rb} \\
		&= \frac{\ra + \rb - \ra + \rb}{\ra \rb}
		= \frac{2}{\ra},
	\end{align*}
	where we have used $\ra, \rb > 0$ and our assumption that $\ra > \rb$.  Picking up from \refeq{leftoff}, we now have
	\begin{align*}
		\frac{\alp \hbar c}{16\pi^2} &\iint \iint e^{-B(\ra + \rb)} \mel{\vxa, \vxb}{\frac{1}{\abs{\vxa - \vxb}}}{\vxa', \vxb'} e^{-B(\ra' + \rb')} \dcxa \dcxb \dcxap \dcxbp \\
		&= \alp \hbar c \intoi \ra e^{-2B \ra} \dra \intoi \rb^2 e^{-2B \rb} \drb
		= \frac{\alp \hbar c}{16 B^5}.
	\end{align*}
	
	Putting this all together,
	\beq
		\ev{H}{\ot} = \frac{\alp \hbar c}{16 B^5} - \frac{A^2 B \hbar^2 / m}{16 B^5} - \frac{2 A^2 Z \alp \hbar c}{16 B^5}
		= \frac{1}{16 B^5} \left( (1 - 2 A^2 Z) \alp \hbar c - \frac{A^2 B \hbar^2}{m} \right).
	\eeq
\end{solution}



\begin{problem}
	By minimizing the variational energy, find the optimal value of $B$.
\end{problem}



\newcommand{\px}{p_x}
\newcommand{\py}{p_y}
\newcommand{\vS}{\vb{S}}
\newcommand{\vSq}{\vS_1}
\newcommand{\vSw}{\vS_2}
\newcommand{\Sz}{S^z}
\newcommand{\Szq}{\Sz_1}
\newcommand{\Szw}{\Sz_2}

\begin{statement}{}
	Consider a two-dimensional harmonic oscillator described by the Hamiltonian
	\beq
		\Ho = \frac{\px^2 + \py^2}{2m} + m \omega^2 \frac{x^2 + y^2}{2}.
	\eeq
\end{statement}

\begin{problem}
	How many single-particle states are there for the first excited level?
\end{problem}

\begin{problem}
	Write down the many-body ground state for two electrons (with spin).  What is the eigenvalue of $(\vSq + \vSw)^2$ for this state?  Here $\vS_i$ are the spin operators of the electrons.
\end{problem}

\begin{problem}
	Write down all the first excited many-body states of two electrons (with spin).  Choose them to be eigenstates of the total spin operator, and compute their eigenvalues of $(\vSq + \vSw)^2$ and $\Szq + \Szw$ (where $\Sz_i$ is the $z$ componenet of the spin operator $\vS_i$).
\end{problem}



\vfill
I consulted Sakurai's \emph{Modern Quantum Mechanics}, Shankar's \emph{Principles of Quantum Mechanics}, and Wolfram MathWorld while writing up these solutions.

\end{document}