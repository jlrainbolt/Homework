 \documentclass[11pt]{article}
\usepackage{geometry, titlesec}
\usepackage[parfill]{parskip}
\usepackage[italicdiff]{physics}
\usepackage{amsfonts, amsthm}
\usepackage[cm]{fullpage}
\usepackage{fancyhdr}
\usepackage{enumitem}
\usepackage{xcolor, soul}
\usepackage{kbordermatrix}
%\allowdisplaybreaks

\makeatletter
\renewcommand*\env@cases[1][1.2]{%
  \let\@ifnextchar\new@ifnextchar
  \left\lbrace
  \def\arraystretch{#1}%
  \array{@{}l@{\quad}l@{}}%
}
\makeatother

 
\renewcommand{\footrulewidth}{.2pt}
%\setlist[enumerate]{leftmargin=*}
\pagestyle{fancy}
\fancyhf{}
\lhead{\textbf{Physics 342 Homework 1}}
\rhead{Lacey Rainbolt}
\setlength{\headheight}{11pt}
\setlength{\headsep}{11pt}
\setlength{\footskip}{24pt}
\lfoot{\today}
\rfoot{\thepage}

\titleformat{\section}[runin]{\normalfont\large\bfseries}{Problem \thesection.}{1em}{}
\titleformat{\subsection}[runin]{\normalfont\large\bfseries}{\thesubsection}{1em}{}
\titleformat{\subparagraph}[leftmargin]{\normalfont\normalsize\bfseries}{}{0pt}{}

\newcommand{\refeq}[1]{(\ref{#1})}

\newcommand{\beq}{\begin{equation*}}
\newcommand{\eeq}{\end{equation*}}

\newcommand{\beqn}{\begin{equation}}
\newcommand{\eeqn}{\end{equation}}


\renewcommand{\vec}[1]{\mathbf{#1}}
\newcommand{\vfix}{\vspace{-\baselineskip}}


\newenvironment{statement}[1]
{
	\section{#1}
	\color{darkgray}
	\ignorespaces
}
{
%    \smallskip
}

\newenvironment{problem}
{
    \color{darkgray}
    \subsection{}
    \ignorespaces
}


\newenvironment{solution}
{
    \paragraph{Solution.}
    \ignorespaces
}
{
%    \smallskip
}

\newcommand{\Schrodinger}{Schr\"{o}dinger}


\begin{document}

\newcommand{\psiE}{\psi_E}

\begin{statement}{}
	A particle is initially in the the ground state of an infinite one-dimensional potential box with walls at $x = 0$ and $x = L$.  During the time interval $0 \leq t \leq \infty$, the particle is subject to a perturbation $V(t) = x^2 e^{-t/\tau}$, where $\tau$ is a time constant.  Calculate, to first order in perturbation theory, the probability of finding the particle in its first excited state as a result of this perturbation.
\end{statement}

\begin{solution}
	The wave functions and energy eigenstates for a particle in an infinite one-dimensional box are given by Eq.~(A.2.4) in Sakurai:
	\begin{align*}
		\psiE(x) &= \sqrt{\frac{2}{L}} \sin\left(\frac{n \pi x}{L}\right), &
		E &= \frac{\hbar^2 n^2 \pi^2}{2 m L^2},
	\end{align*}
	where $n = 1, 2, 3, \ldots$
\end{solution}



\newcommand{\Ha}{H_a}
\newcommand{\Hb}{H_b}
\newcommand{\Hi}{H_i}
\newcommand{\vp}{\vb{p}}
\newcommand{\vpi}{\vp_i}
\newcommand{\alp}{\alpha}
\newcommand{\ri}{r_i}
\newcommand{\ra}{r_a}
\newcommand{\rb}{r_b}
\newcommand{\rab}{r_{a b}}
\newcommand{\vx}{\vb{x}}
\newcommand{\vxi}{\vx_i}
\newcommand{\vxa}{\vx_a}
\newcommand{\vxb}{\vx_b}

\begin{statement}{}
	Consider a system of two electrons, which is described by the Hamiltonian
	\begin{align*}
		H &= \Ha + \Hb + V, &
		\Hi &= \frac{\vpi^2}{2m} - \frac{Z \alp \hbar c}{\ri}, &
		V = \frac{\alp \hbar c}{\rab}.
	\end{align*}
	Here, we label two electrons by $i = a, b$; $\ri = \abs{\vxi}$ and $\rab = \abs{\vxa - \vxb}$ where $\vxi$ is the spatial coordinate for electron $i$; and $Z$ and $\alp$ are constants.  To find an approximate ground state of $H$, let us try a variational wave function
	\beq
		\Psi(\vxa, \vxb) = \frac{A}{4\pi} e^{-B(\ra + \rb)},
	\eeq
	where $A$ is a normalization constant and $B$ is your variational parameter.
\end{statement}

\begin{problem}
	Compute the variational energy for the given variational parameter $B$.
\end{problem}

\begin{problem}
	By minimizing the variational energy, find the optimal value of $B$.
\end{problem}



\newcommand{\Ho}{H_0}
\newcommand{\px}{p_x}
\newcommand{\py}{p_y}
\newcommand{\vS}{\vb{S}}
\newcommand{\vSq}{\vS_1}
\newcommand{\vSw}{\vS_2}
\newcommand{\Sz}{S^z}
\newcommand{\Szq}{\Sz_1}
\newcommand{\Szw}{\Sz_2}

\begin{statement}{}
	Consider a two-dimensional harmonic oscillator described by the Hamiltonian
	\beq
		\Ho = \frac{\px^2 + \py^2}{2m} + m \omega^2 \frac{x^2 + y^2}{2}.
	\eeq
\end{statement}

\begin{problem}
	How many single-particle states are there for the first excited level?
\end{problem}

\begin{problem}
	Write down the many-body ground state for two electrons (with spin).  What is the eigenvalue of $(\vSq + \vSw)^2$ for this state?  Here $\vS_i$ are the spin operators of the electrons.
\end{problem}

\begin{problem}
	Write down all the first excited many-body states of two electrons (with spin).  Choose them to be eigenstates of the total spin operator, and compute their eigenvalues of $(\vSq + \vSw)^2$ and $\Szq + \Szw$ (where $\Sz_i$ is the $z$ componenet of the spin operator $\vS_i$).
\end{problem}



\vfill
I consulted Sakurai's \emph{Modern Quantum Mechanics}, Shankar's \emph{Principles of Quantum Mechanics}, and Wolfram MathWorld while writing up these solutions.

\end{document}