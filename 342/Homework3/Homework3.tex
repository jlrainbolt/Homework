 \documentclass[11pt]{article}
\usepackage{geometry, titlesec}
\usepackage[parfill]{parskip}
\usepackage[italicdiff]{physics}
\usepackage{amsfonts, amsthm}
\usepackage[cm]{fullpage}
\usepackage{fancyhdr}
\usepackage{enumitem}
\usepackage{xcolor, soul}
\usepackage{kbordermatrix}
\usepackage{booktabs}
\usepackage{graphicx}
%\allowdisplaybreaks

\makeatletter
\renewcommand*\env@cases[1][1.2]{%
  \let\@ifnextchar\new@ifnextchar
  \left\lbrace
  \def\arraystretch{#1}%
  \array{@{}l@{\quad}l@{}}%
}
\makeatother

 
\renewcommand{\footrulewidth}{.2pt}
%\setlist[enumerate]{leftmargin=*}
\pagestyle{fancy}
\fancyhf{}
\lhead{\textbf{Physics 342 Homework 3}}
\rhead{Lacey Rainbolt}
\setlength{\headheight}{11pt}
\setlength{\headsep}{11pt}
\setlength{\footskip}{24pt}
\lfoot{\today}
\rfoot{\thepage}

\titleformat{\section}[runin]{\normalfont\large\bfseries}{Problem \thesection.}{1em}{}
\titleformat{\subsection}[runin]{\normalfont\large\bfseries}{\thesubsection}{1em}{}
\titleformat{\subparagraph}[leftmargin]{\normalfont\normalsize\bfseries}{}{0pt}{}

\newcommand{\refeq}[1]{(\ref{#1})}

\newcommand{\beq}{\begin{equation*}}
\newcommand{\eeq}{\end{equation*}}

\newcommand{\beqn}{\begin{equation}}
\newcommand{\eeqn}{\end{equation}}


\renewcommand{\vec}[1]{\mathbf{#1}}
\newcommand{\vfix}{\vspace{-\baselineskip}}


\newenvironment{statement}[1]
{
	\section{#1}
	\color{darkgray}
	\ignorespaces
}
{
%    \smallskip
}

\newenvironment{problem}
{
	\color{darkgray}
	\subsection{}
	\ignorespaces
}


\newenvironment{solution}
{
	\paragraph{Solution.}
}
{
	\bigskip
}

\newcommand{\Schrodinger}{Schr\"{o}dinger}
\newcommand{\qimplies}{\quad \implies \quad}


\begin{document}

\state{(Jackson 9.8)}{\ 
	%\emph{Hint:} The electromagnetic angular momentum density comes from more than the transverse (radiation zone) components of the fields.
}

%
%	Jackson 9.8(a)
%

\prob{}{
	Show that a classical oscillating electric dipole $\vp$ with fields given by
	\aln{ \label{fields1}
		\vH &= \frac{c k^2}{4\pi} (\nh \cross \vp) \frac{e^{i k r}}{r} \paren{ 1 - \frac{1}{i k r} }, &
		\vE &= \frac{1}{4\pi \epso} \curly{ k^2 (\nh \cross \vp) \cross \nh \frac{e^{i k r}}{r} + [ 3 \nh (\nh \vdot \vp) - \vp ] \paren{ \frac{1}{r^3} - \frac{i k}{r^2} } e^{i k r} },
	}
	radiates electromagnetic angular momentum to infinity at the rate
	\eq{
		\dv{\vL}{t} = \frac{k^3}{12 \pi \epso} \Im[ \vp^* \cross \vp ].
	}
	\vfix
}

\sol{
	According to Jackson~(9.20), the time-averaged angular momentum density is
	\eq{
		\vl = \frac{\Re[ \vx \cross (\vE \cross \vHs)}{2 c^2}.
	}
	One of the vector identities on the inside cover of Jackson is $\vaa \cross (\vbb \cross \vcc) = (\vaa \vdot \vcc) \vbb - (\vaa \vdot \vbb) \vcc$, so
	\eqn{l1}{
		\vl = \frac{(\vx \vdot \vHs) \vE - (\vx \vdot \vE) \vHs}{2 c^2}.
	}
	From Eq.~\refeq{fields1}, note that
	\eq{
		\vx \vdot \vHs \propto \vx \vdot (\nh \cross \vps)
		= \vps \vdot (\vx \cross \nh)
		= \vO,
	}
	where we have used the identity $\vaa \vdot (\vbb \cross \vcc) = \vcc \vdot (\vaa \cross \vbb)$ and the fact that $\nh$ points in the $\vx$ direction.  For $\vx \vdot \vE$, note that
	\al{
		\vx \vdot [ (\nh \cross \vp) \cross \nh ] &= -\vx \vdot [ \nh \cross (\nh \cross \vp) ]
		= -\vx \vdot [ (\nh \vdot \vp) \nh - (\nh \vdot \nh) \vp ]
		= -(\nh \vdot \vp) (\vx \vdot \nh) + \vx \vdot \vp \\
		&= -r (\nh \vdot \vp) + \vx \vdot \vp
		= \vx \vdot \vp - \vx \vdot \vp
		= 0, \\[1.5ex]
		\vx \vdot [ 3 \nh (\nh \vdot \vp) - \vp ] &= 3 (\vx \vdot \nh) (\nh \vdot \vp) - \vx \vdot \vp
		= 3r (\nh \vdot \vp) - \vx \vdot \vp
		= 3(\vx \vdot \vp) - \vx \vdot \vp
		= 2(\vx \vdot \vp),
	}
	since $\abs{\vx} = r$ and $\vx = r \,\nh$.  Then
	\eq{
		\vx \vdot \vE = \frac{1}{2\pi \epso} (\vx \vdot \vp) \paren{ \frac{1}{r^3} - \frac{i k}{r^2} } e^{i k r}
		= \frac{1}{2\pi \epso} (\nh \vdot \vp) \paren{ \frac{1}{r^2} - \frac{i k}{r} } e^{i k r}.
	}
	
	With these substitutions, Eq.~\refeq{l1} becomes
	\al{
		\vl &= -\frac{(\vx \vdot \vE) \vHs}{c^2}
		= -\frac{1}{4\pi \epso c^2} (\nh \vdot \vp) \paren{ \frac{1}{r^2} - \frac{i k}{r} } e^{i k r} \frac{c k^2}{4\pi} (\nh \cross \vps) \frac{e^{-i k r}}{r} \paren{ 1 + \frac{1}{i k r} } \\
		&= -\frac{k^2}{16\pi^2 \epso c r} (\nh \vdot \vp) (\nh \cross \vps) \paren{ \frac{1}{r^2} - \frac{i k}{r} } \paren{ 1 - \frac{i}{k r} }
		= -\frac{k^2}{16\pi^2 \epso c} (\nh \vdot \vp) (\nh \cross \vps) \paren{ \frac{1}{r^2} - \frac{i}{k r^3} - \frac{i k}{r} - \frac{1}{r^2} } \\
		&= -\frac{i k^2}{16\pi^2 \epso c r} (\nh \vdot \vp) (\nh \cross \vps) \paren{ \frac{1}{k r^3} + \frac{k}{r^2} }
		= \frac{i k^3}{16\pi^2 \epso c r^2} (\nh \vdot \vp) (\nh \cross \vps) \paren{ \frac{1}{k^2 r^2} + 1 }.
	}
	
	Let $\vL$ be the angular momentum radiated to a distance $R$.  Then
	\eq{
		\vL = \int_R \vl(r) \ddcx
		= \intopi \intotp \intoR \vl(r) \,r^2 \sin\tht \ddr \ddphi \dd\tht,
	}
	and the time derivative is
	\aln{
		\dv{\vL}{t} &= \dv{t}(\intopi \intotp \intoR \vl(r) \,r^2 \sin\tht \ddr \ddphi \dd\tht)
		= \dv{r}{t} \dv{r}(\intopi \intotp \intoR \vl(r) \,r^2 \sin\tht \ddr \ddphi \dd\tht) \notag \\
		&= c \intopi \intotp \vl(r) \,r^2 \sin\tht \ddphi \dd\tht
		= \frac{i k^3}{16\pi^2 \epso} \paren{ \frac{1}{k^2 r^2} + 1 } \intopi \intotp (\nh \vdot \vp) (\nh \cross \vps) \sin\tht \ddphi \dd\tht. \label{dLdt}
	}
	Note that
	\eq{
		[ (\nh \vdot \vp) (\nh \cross \vps) ]_i = \sumje n_j p_j (\nh \cross \vps)_i
		= \sumje \sumke \sumle \epsikl n_j p_j n_k p_l^*,
	}
	so
	\eq{
		\dv{L_i}{t} \propto \sumje \sumke \sumle \epsikl p_j p_l^* \int n_j p_k \ddOmg
		= \sumje \sumke \sumle \epsikl p_j p_l^* \frac{4\pi}{3} \del_{jk}
		= \frac{4\pi}{3} \epsikl p_k p_l^*
		= \frac{4\pi}{3} (\vp \cross \vps)_i,
	}
	where we have used Jackson~(9.47), $\int n_\bet n_\gam \ddOmg = 4\pi \del_{\bet \gam} / 3$.  Making this substitution into Eq.~\refeq{dLdt},
	\eq{
		\dv{\vL}{t} = \frac{i k^3}{6\pi \epso} \paren{ \frac{1}{k^2 r^2} + 1 } (\vp \cross \vps).
	}
	Taking the limit as $r \to \infty$, we find
	\eqn{ans1a}{
		\dv{\vL}{t} = \Re\!\brac{ \frac{i k^3}{12\pi \epso} (\vp \cross \vps) }
		= \Re\!\brac{ -\frac{i k^3}{12\pi \epso} (\vps \cross \vp) }
		= \ans{ \frac{k^3}{12\pi \epso} \Im[ \vps \cross \vp ], }
	}
	as desired. \qed
}

%
%	Jackson 9.8(b)
%

\prob{}{
	What is the ratio of angular momentum radiated to energy radiated?  Interpret.
}

\sol{
	According to Jackson~(9.24), the total power radiated by an oscillating electric dipole $\vp$ is
	\eq{
		P = \dv{E}{t}
		= \frac{c^2 \Zo k^4}{12 \pi} \abs{\vp}^2.
	}
	Then the ratio of angular momentum radiated to energy radiated is
	\eq{
		\frac{\dv*{\vL}{t}}{\dv*{E}{t}} = \frac{k^3}{12\pi \epso} \Im[ \vps \cross \vp ] \frac{12 \pi}{c^2 \Zo k^4 \abs{\vp}^2}
		= \frac{1}{\epso} \Im[ \vps \cross \vp ] \frac{1}{c^2 \Zo k \abs{\vp}^2}
		= \ans{ \frac{\Im[ \vps \cross \vp ]}{\omg \abs{\vp}^2}, }
	}
	where we have used $\Zo = \sqrt{\muo / \epso} = 1 / \sqrt{\epso^2 c^2} = 1 / \epso c$, $c^2 = 1 / (\epso \muo)$, and $\omg = k c$.
	
	In the limit of high frequency, $(\dv*{\vL}{t}) / (\dv*{E}{t}) \to 0$.  In this scenario, the energy radiated dominates over the angular momentum radiated.  Likewise, in the limit of low frequency, $(\dv*{\vL}{t}) / (\dv*{E}{t}) \to \infty$, meaning that angular momentum radiation dominates.  This is sensible because rotational kinetic energy $E \propto \omg^2$, while angular momentum $L \propto \omg$.
}

%
%	Jackson 9.8(c)
%

\prob{}{
	For a charge $e$ rotating in the $xy$ plane at radius $a$ and angular speed $\omg$, show that there is only a $z$ component of radiated angular momentum with magnitude $\dv*{\Lz}{t} = e^2 k^3 a^2 / 6 \pi \epso$.  What about a charge oscillating along the $z$ axis?
}

\sol{
	We know from Homework~5 that the position of a point charge rotating counterclockwise in the $xy$ plane is
	\eq{
		\vx(t) = a \cos(\omg t) \,\vx + a \sin(\omg t) \,\yh.
	}
	\clearpage
	Then the charge distribution is
	\eq{
		\rho(\vx, t) = e \del[ x - a \cos(\omg t) ] \,\del[ y - a \sin(\omg t) ] \,\del(z).
	}
	
	According to Jackson~(4.8), the dipole moment is defined
	\eq{
		\vp = \int \vx' \,\rho(\vx') \ddcxp.
	}
	The components of $\vp$ for the point charge are then
	\al{
		\px &= e \iiint x \,\del[ x - a \cos(\omg t) ] \,\del[ y - a \sin(\omg t) ] \,\del(z) \ddx \ddy \ddz
		= e a \cos(\omg t), \\
		\py &= e \iiint y \,\del[ x - a \cos(\omg t) ] \,\del[ y - a \sin(\omg t) ] \,\del(z) \ddx \ddy \ddz
		= e a \sin(\omg t), \\
		\pz &= e \iiint z \,\del[ x - a \cos(\omg t) ] \,\del[ y - a \sin(\omg t) ] \,\del(z) \ddx \ddy \ddz
		= 0,
	}
	so we can write $\vp = e a \,e^{-i \omg t} (\xh + i\,\yh).$  Substituting into Eq.~\refeq{ans1a},
	\al{
		\dv{\vL}{t} &= \Re\!\brac{ \frac{i k^3}{12\pi \epso} e^2 a^2 e^{-i \omg t} e^{i \omg t} [ (\xh + i\,\yh) \cross (\xh - i\,\yh) ] }
		= \Re\!\brac{ \frac{i e^2 k^3 a^2}{12\pi \epso} (-2i \,\xh \cross \yh) }
		= \Re\!\brac{ \frac{e^2 k^3 a^2}{6\pi \epso} \,\zh } \\
		&= \ans{ \frac{e^2 k^3 a^2}{6\pi \epso} \cos(\omg t) \,\zh, }
	}
	as desired. \qed
	
	A charge oscillating along the $z$ axis with amplitude $a$ has the charge density
	\eq{
		\rho(\vx, t) = e a \,\del(x) \,\del(y) \,\del[ z - \cos(\omg t) ],
	}
	which gives the dipole moment
	\al{
		\px &= e a \iiint x \,\del(x) \,\del(y) \,\del[ z - \cos(\omg t) ] \ddx \ddy \ddz
		= 0, \\
		\py &= e a \iiint y \,\del(x) \,\del(y) \,\del[ z - \cos(\omg t) ] \ddx \ddy \ddz
		= 0, \\
		\pz &= e a \iiint z \,\del(x) \,\del(y) \,\del[ z - \cos(\omg t) ] \ddx \ddy \ddz
		= e a \cos(\omg t).
	}
	In complex notation, $\vp = e a \,e^{-i\omg t} \,\zh$.  Substituting into Eq.~\refeq{ans1a}, we find
	\eq{
		\dv{\vL}{t} = \Re\!\brac{ \frac{i k^3}{12\pi \epso} e^2 a^2 e^{-i \omg t} e^{i \omg t} (\zh \cross \zh) }
		= \ans{ \vO. }
	}
	So we see that a charge undergoing linear motion does not lead to a radiated angular momentum, which is sensible.
}

%
%	Jackson 9.8(d)
%

\prob{}{
	What are the results corresponding to Probs.~{1(a)} and {1(b)} for magnetic dipole radiation?
}

\sol{
	The radiation fields for a magnetic dipole are given by Jackson~(19.35--36),
	\al{
		\vH &= \frac{1}{4\pi} \curly{ k^2 (\nh \cross \vm) \cross \nh \frac{e^{i k r}}{r} + [ 3 \nh (\nh \vdot \vm) - \vm ] \paren{ \frac{1}{r^3} - \frac{i k}{r^2} } e^{i k r} }, &
		\vE &= -\frac{\Zo}{4\pi} k^2 (\nh \cross \vm) \frac{e^{i k r}}{r} \paren{ 1 - \frac{1}{i k r} }.
	}
	\clearpage
	Comparing with Eq.~\refeq{fields1}, we see that $\vH \to -\vE / \Zo$, $\vE \to \Zo \vH$, and $\vp \to \vm / c$ as stated in the book~\cite[p.~413]{Jackson}.  Making these substitutions, the results of Probs.~{1.1(a)} and {(b)} become
	\al{
		\ans{ \dv{\vL}{t}\ }&\ans{= \frac{\muo k^3}{12\pi} \Im[ \vms \cross \vm ], } &
		\ans{ \frac{\dv*{\vL}{t}}{\dv*{E}{t}}\ }&\ans{= \frac{\Im[ \vms \cross \vm ]}{\omg \abs{\vm}^2} }
	}
	where we have used $\mu = 1 / \epso c^2$.
}



\newcommand{\Vo}{V_0}
\newcommand{\vr}{\vb{r}}
\newcommand{\Rl}{R_l}
\newcommand{\Ylm}{Y_{l m}}
\newcommand{\Rlr}{\Rl(r)}
\newcommand{\vep}{\varepsilon}
%\newcommand{\tht}{\theta}
\newcommand{\del}{\delta}
\newcommand{\dell}{\del_l}
\newcommand{\delo}{\del_0}
\newcommand{\Al}{A_l}
\newcommand{\absr}{\abs{r}}

\newcommand{\tsc}{\text{sc}}
\newcommand{\asc}{a_\tsc}
\newcommand{\kap}{\kappa}
\newcommand{\avep}{\abs{\vep}}

\newcommand{\betal}{\beta_l}
\newcommand{\psiE}{\psi_E}
\newcommand{\uE}{u_E}
\newcommand{\El}{E_l}
\newcommand{\eff}{\text{eff}}
\newcommand{\Veff}{V_\eff}

\newcommand{\jl}{j_l}
\newcommand{\jo}{j_0}
\newcommand{\nl}{n_l}
\newcommand{\hlq}{h_l^{(1)}}
\newcommand{\hoq}{h_0^{(1)}}
\newcommand{\Ro}{R_0}
\newcommand{\constant}{\text{constant }}


\clearpage
\begin{statement}{}
	Consider a quantum particle with mass $m$ moving in the presence of the square well potential
	\beq
		V(r) = \begin{cases}
			-\Vo & r \leq a, \\
			0 & r > a.
		\end{cases}
	\eeq
	\vfix
\end{statement}

\begin{problem}
	Writing the wave function in polar coordinates as $\psi(\vr) = \Rlr \, \Ylm(\tht, \phi)$, write down the {\Schrodinger} equation obeyed by $\Rl$.
\end{problem}

\begin{solution}
	From (A.5.1) in Sakurai, the full {\Schrodinger} equation is
	\beq
		-\frac{\hbar^2}{2m} \left[ \frac{1}{r^2} \pdv{r} \left( r^2 \pdv{\psiE}{r} \right) + \frac{1}{r^2 \sin\tht} \pdv{\tht} \left( \sin\tht \pdv{\psiE}{\tht} \right) + \frac{1}{r^2 \sin^2\tht} \pdv[2]{\psiE}{\phi} \right] + V(r) \,\psiE = E \psiE,
	\eeq
	where the angular part of $\psiE$ satisfies (A.5.4),
	\beq
		-\left[ \frac{1}{\sin\tht} \pdv{}{\tht} \left( \sin\tht \pdv{}{\tht} \right) + \frac{1}{\sin^2\tht} \pdv[2]{}{\phi} \right] \Ylm = l (l + 1) \Ylm.
	\eeq
	Then the equivalent one-dimensional {\Schrodinger} equation is the equation immediately following (A.5.8),
	\beqn \label{schrod}
		-\frac{\hbar^2}{2m} \dv[2]{\uE}{r} + \left[ V(r) + \frac{l (l + 1) \hbar^2}{2 m r^2} \right] \uE = E \uE,
	\eeqn
	where $\uE(r) = r R(r)$.  In terms of $\Rl$,
	\beq
		-\frac{\hbar^2}{2m} \dv[2]{r} (r \Rl) + \left[ V(r) + \frac{l (l + 1) \hbar^2}{2 m r^2} \right] r \Rl = E r \Rl.
	\eeq
	or
	\beq
		\frac{\hbar^2}{2 m} \left[ -\frac{1}{r^2} \pdv{r} \left( r^2 \pdv{r} \right) + V(r) + \frac{l (l + 1)}{r^2} \right] \Rl(r) = \El \, \Rl(r).
	\eeq
	From (7.7.1), the effective potential at low energies for the $l$th partial wave is
	\beq
		\Veff = V(r) + \frac{\hbar^2}{2m} \frac{l (l + 1)}{r^2},
	\eeq
	so the {\Schrodinger} equation can be rewritten as
	\beq
		\left[ -\frac{\hbar^2}{2 m} \frac{1}{r^2} \pdv{}{r} \left( r^2 \pdv{}{r} \right) + \Veff\, \right] \Rl(r) = \El \, \Rl(r).
	\eeq
	\vfix
\end{solution}

\begin{problem}
	When $\Vo$ is a certain value, there is one bound state for the $s$ wave ($l = 0$).  The bound state energy $\vep$ is small ($0 < \avep \ll \Vo$).  Obtain the range of the depth of the well $\Vo$ (? $\leq \Vo <$ ?).  Also, calculate for the bound state the probability for the particle to exist outside of the well.
\end{problem}

\begin{solution}
	Inside the well, $\Rl$ are given by (A.5.16),
	\beq
		\Rl(r) = \constant \jl(\alp r),
	\eeq
	where $\alp$ is defined in Eq.~(A.5.17),
	\beq
		\alp = \sqrt{\frac{2m (\Vo - \abs{E})}{\hbar^2}}, \quad r < a,
	\eeq
	and the spherical Bessel functions $\jl$ is given by (A.5.12),
	\begin{align*}
		\jl(\rho) = \sqrt{\frac{\pi}{2\rho}} J_{l + 1/2}(\rho).
	\end{align*}
	For the $s$ wave, the relevant Bessel function is given by (A.5.12),
	\beq
		\jo(\rho) = \frac{\sin\rho}{\rho}.
	\eeq
%	Then we have
%	\beqn \label{inside}
%		\Ro(r) \propto \sqrt{\frac{\hbar^2}{2 m (\Vo - \avep)}} \frac{1}{r} \sin(\sqrt{\frac{2 m \Vo}{\hbar^2}} r), \quad r < a,
%	\eeqn
%	for the solution inside the well.
	But for $l - 0$, $\Veff$ reduces to $V(r)$, so \refeq{schrod} reduces to the one-dimensional problem for $\uE$,
	\beq
		-\frac{\hbar^2}{2m} \dv[2]{\uE}{r} + V(r) \uE = E \uE.
	\eeq
	The bound-state solutions are given by (A.2.6),
	\beq
		\uE \sim \begin{cases}
			e^{-\kap r} & \text{for } r > a, \\
			\cos kr \quad \text{(even parity)} & \text{for } r < a, \\
			\sin kr \quad \text{(odd parity)} & \text{for } r > a,
		\end{cases}
	\eeq
	where $k$ and $\kap$ are defined by (A.2.7),
	\begin{align*}
		k &= \sqrt{\frac{2m (\Vo - \abs{E})}{\hbar^2}}, &
		\kap &= \sqrt{\frac{2m \abs{E}}{\hbar^2}}.
	\end{align*}
	So we see that $\alp = k$, and thus we are interested in the odd-parity solutions to the one-dimensional problem.
	
	For the one-dimensional problem, the allowed values of bound-state energy
	\beq
		E = -\frac{\hbar^2 \kap^2}{2m}
	\eeq
	can be found by solving (A.2.8),
	\begin{align*}
		k a \tan k a &= \kap a \quad\text{(even parity)}, &
		k a \cot k a &= -\kap a \quad\text{(odd parity)},
	\end{align*}
	where $\kap$ and $k$ are related by (A.2.9),
	\beq
		\frac{2 m \Vo a^2}{\hbar^2} = (k^2 + \kap^2) a^2.
	\eeq
	We are interested in the odd parity solutions, so we want to solve
	\beqn \label{ka}
		k a \cot k a = -\kap a.
	\eeqn
	For the right side, we can write
	\beqn \label{kapa}
		\kap a = -\sqrt{\frac{2 m \Vo a^2}{\hbar^2} - k^2 a^2} \equiv -\sqrt{z^2 - (ka)^2},
	\eeqn
	where we have defined $z$.
	
	Now we can solve the equation graphically.  Note that $ka$ and $z$ are both positive definite. This means the odd parity equation in \refeq{ka} has its first $ka$ axis intercept at $ka = \pi/2$, where the slope is negative.  Note also that $\kap a$ given by \refeq{kapa} is an equation for one quarter of an ellipse in quadrant IV, so it is not defined above the $ka$ axis.  Therefore it is not possible for the two graphs to intersect for $z < \pi / 2$.  For $z > 3\pi / 2$, the plots intersect twice, meaning there is more than one bound state.  In Fig.~\ref{plot}, this is illustrated with $\kap a$ for $z = n \pi / 2$ with $n = 1, 2, 3, \ldots$.
	
	\begin{figure} \centering
		\includegraphics[trim=2cm 0 0 0,clip,width=0.75\textwidth]{plot}
		\caption{Plot demonstrating single bound state solutions to \refeq{ka} in the range $\pi/2 < z < 3\pi/2$, where $z$ is defined in \refeq{kapa}.}
		\label{plot}
	\end{figure}
	
	Finally, we have the restriction
	\beq
		\frac{\pi}{2} < \sqrt{\frac{2 m \Vo a^2}{\hbar^2}} < \frac{3\pi}{2}
		\qimplies
		\frac{\pi^2 \hbar^2}{8 m a^2} < \Vo < \frac{9\pi^2 \hbar^2}{8 m a^2}.
	\eeq
	\vfix
\end{solution}



\begin{problem}
	Consider the scattering problem by the well.  For each $l$, for large enough $r$, when $\Rlr$ is given by $\Rlr \sim \Al \sin(kr - l\pi / 2 + \dell) / r$, $\dell$ is called the scattering phase shift.  For the value of $\Vo$ within the range you obtained in the above problem, when the energy of the incident wave is is $E = 9 \Vo / 16$, calculate $\tan\delo$~(where $\delo$ is the scattering phase shift for the $s$ wave).
\end{problem}

\begin{solution}
%	From (7.6.35) in Sakurai,
%	\beq
%		\tan\dell = \frac{k R \jl'(kR) - \betal \jl(kR)}{k R \nl'(kR) - \betal \nl(kR)},
%	\eeq
%	where (7.6.34) defines $\betal$,
%	\beq
%		\betal \equiv \left[ \frac{r}{\Al} \dv{\Al}{r} \right]_{r=R},
%	\eeq
%	and
%	\beq
%		\jl'(kR) = \left. \dv{\jl}{(kr)} \right|_{kr = kR}
%	\eeq
\end{solution}

\begin{problem}
	Now consider the $S$ matrix, $S \equiv \exp(2i \delo) = \exp(i \delo) / \exp(-i \delo)$.  Compare the condition on $s$ wave bound state energies and the zero of the denominator of $S$.  Explain their relation.
\end{problem}




\newcommand{\gam}{\gamma}
\newcommand{\chio}{\chi_0}
\newcommand{\chior}{\chio(r)}
\newcommand{\Gam}{\Gamma}

\begin{statement}{}
	Consider a three dimensional potential
	\beq
		V(\absr) = \frac{\hbar^2 \gam}{2m} \del(\absr - a).
	\eeq
	The $s$ wave {\Schrodinger} equation is given by
	\beq
		-\frac{\hbar^2}{2m} \dv[2]{\chior}{r} + \frac{\hbar^2 \gam}{2m} \del(r - a) \, \chior = E \,\chior.
	\eeq
	The $s$ wave function must be regular (zero) at $r = 0$.  At $r = a$, it is continuous, but its derivative can jump.
\end{statement}

\begin{problem}
	Calculate the $s$ wave scattering phase shift $\do(k)$, where $k$ is related to $E$ as $E = \hbar^2 k^2 / 2m$.
\end{problem}

\begin{problem}
	When $\gam \gg k$, $1/a$ and when $\sin ka$ is not small, discuss the behavior of the scattering phase shift.
\end{problem}

\begin{problem}
	Obtain the condition to have resonant states and calculate the energy of the resonant states.
\end{problem}

\begin{problem}
	Calculate the width $\Gam$ of the resonance.  Discuss its behavior when $\gam$ is big.
\end{problem}

\begin{problem}
	When the velocity of the incident wave is small, obtain the scattering cross section.
\end{problem}



\vfill
I consulted Sakurai's \emph{Modern Quantum Mechanics}, Shankar's \emph{Principles of Quantum Mechanics}, and the Wikipedia article on a particle in a spherically symmetric potential while writing up these solutions.

\end{document}