 \documentclass[11pt]{article}
\usepackage{geometry, titlesec}
\usepackage[parfill]{parskip}
\usepackage[italicdiff]{physics}
\usepackage{amsfonts, amsthm}
\usepackage[cm]{fullpage}
\usepackage{fancyhdr}
\usepackage{enumitem}
\usepackage{xcolor, soul}
\usepackage{kbordermatrix}
%\allowdisplaybreaks

\makeatletter
\renewcommand*\env@cases[1][1.2]{%
  \let\@ifnextchar\new@ifnextchar
  \left\lbrace
  \def\arraystretch{#1}%
  \array{@{}l@{\quad}l@{}}%
}
\makeatother

 
\renewcommand{\footrulewidth}{.2pt}
%\setlist[enumerate]{leftmargin=*}
\pagestyle{fancy}
\fancyhf{}
\lhead{\textbf{Physics 342 Homework 3}}
\rhead{Lacey Rainbolt}
\setlength{\headheight}{11pt}
\setlength{\headsep}{11pt}
\setlength{\footskip}{24pt}
\lfoot{\today}
\rfoot{\thepage}

\titleformat{\section}[runin]{\normalfont\large\bfseries}{Problem \thesection.}{1em}{}
\titleformat{\subsection}[runin]{\normalfont\large\bfseries}{\thesubsection}{1em}{}
\titleformat{\subparagraph}[leftmargin]{\normalfont\normalsize\bfseries}{}{0pt}{}

\newcommand{\refeq}[1]{(\ref{#1})}

\newcommand{\beq}{\begin{equation*}}
\newcommand{\eeq}{\end{equation*}}

\newcommand{\beqn}{\begin{equation}}
\newcommand{\eeqn}{\end{equation}}


\renewcommand{\vec}[1]{\mathbf{#1}}
\newcommand{\vfix}{\vspace{-\baselineskip}}


\newenvironment{statement}[1]
{
	\section{#1}
	\color{darkgray}
	\ignorespaces
}
{
%    \smallskip
}

\newenvironment{problem}
{
	\color{darkgray}
	\subsection{}
	\ignorespaces
}


\newenvironment{solution}
{
	\paragraph{Solution.}
}
{
	\bigskip
}

\newcommand{\Schrodinger}{Schr\"{o}dinger}


\begin{document}


\newcommand{\vE}{\vb{E}}
\newcommand{\vS}{\vb{S}}
\newcommand{\vSA}{\vS^A}
\newcommand{\vSB}{\vS^B}
\newcommand{\Max}{\text{Max}}
\newcommand{\kpsi}{\ket{\psi}}
\newcommand{\up}{\uparrow}
\newcommand{\dn}{\downarrow}
\newcommand{\kupz}{\ket{\up_z}}
\newcommand{\kdnz}{\ket{\dn_z}}
\newcommand{\Sx}{S_x}
\newcommand{\Sz}{S_z}
\newcommand{\SxA}{\Sx^A}
\newcommand{\SzA}{\Sz^A}
\newcommand{\SxB}{\Sx^B}
\newcommand{\SzB}{\Sz^B}
\newcommand{\dhf}{\dfrac{1}{2}}

\begin{statement}{}
	Consider the following probabalistic game: There are four doors ($Q, R, S, T$).  Behind each door is a device which displays $\pm 1$ randomly according to the probability $P(Q=\pm1, R=\pm1, S=\pm1, T=\pm1)$.  Alice and Bob are on the same team.  Alice has to choose either $Q$ and $R$, and then Bob has to choose either $S$ and $T$.  When the numbers match, they get $+1$ point; when the numbers do not match, they get $-1$ point.  However, when they open $Q$ and $T$, it's an exception.  When the numbers (do not) match, they get $-1$~($+1$).
\end{statement}

\begin{problem} \label{1.1}
	Let's assume Alice and Bob open the doors completely randomly.  When all numbers are $+1$ with probability 1, what is the expectation value of the point they get?
\end{problem}

\begin{solution}
	Let $\vE$ be the expectation value of the number of points.  In this case, the numbers behind the two doors will always match.  So
	\beq
		\vE = \frac{QS + RS + RT - QT}{4}
		= \frac{1 + 1 + 1 - 1}{4}
		= \frac{1}{2}.
	\eeq
\end{solution}
\vfix



\begin{problem}
	As it turns out, irrespective of how hard you fine tune the probability $P(Q=\pm1, R=\pm1, S=\pm1, T=\pm1)$, the expectation value of the point Alice and Bob get cannot exceed a certain value $\Max$:
	\beq
		\frac{\vE(QS) + \vE(RS) + \vE(RT) - \vE(QT)}{4} \leq \Max.
	\eeq
	Here, $\vE(QS)$, etc. is the expectation value of the point when Alice opens $Q$ and Bob opens $S$.  This is a Bell inequality.  Determine $\Max$.
	
	\emph{Hint:} For a given realization of the numbers $Q = \pm1$, $R = \pm1$, $S = \pm1$, $T = \pm1$, which occurs with probability $P(Q, R, S, T)$, note that $QS + RS + RT - QT = (Q+R)S + (R-Q)S$, where one of $\{(R+Q), (R-Q)\}$ is 2 and the other 0.
\end{problem}

\begin{solution}
	The possiblities are listed in the following table:
	\beq
		\begin{array}{l | c c c c c c c c c c c c c c c c}
			Q & +1 & +1 & +1 & +1 & -1 & -1 & +1 & +1 & +1 & -1 & -1 & -1 & -1 & -1 & +1 & -1 \\
			R & +1 & +1 & +1 & -1 & +1 & +1 & -1 & +1 & -1 & +1 & -1 & -1 & -1 & +1 & -1 & -1 \\
			S & +1 & +1 & -1 & +1 & +1 & +1 & +1 & -1 & -1 & -1 & +1 & -1 & +1 & -1 & -1 & -1 \\
			T & +1 & -1 & +1 & +1 & +1 & -1 & -1 & -1 & +1 & +1 & +1 & +1 & -1 & -1 & -1 & -1 \\ \hline \\[-1.5ex]
			\vE & \dhf & \dhf & -\dhf & -\dhf & \dhf & -\dhf & \dhf & -\dhf & -\dhf & \dhf & -\dhf & \dhf & -\dhf & -\dhf & \dhf & \dhf
		\end{array}
	\eeq
	Clearly,
	\beq
		\Max = \frac{1}{2}.
	\eeq
\end{solution}
\vfix



\begin{problem}
	Frustrated by the upper bound set by the Bell inequality, Bob decides to cheat.  He now changes the value of $T$ after Alice chooses $Q$ or $R$.  Assume $Q,R,S$ are set to be $+1$ with probability 1.  To make the expectation value of the point they get equal to $+1$, what values should Bob set after Alice chooses $Q$ or $R$?
\end{problem}

\begin{solution}
	If Alice chooses $R$, Bob should set $T = 1$.  If Alice chooses $Q$, Bob should set $T = -1$.  This way,
	\beq
		\frac{\vE(QS) + \vE(RS) + \vE(RT) - \vE(QT)}{4}
		= \frac{1 + 1 + 1 + 1}{4}
		= 1.
	\eeq
\end{solution}
\vfix



\begin{problem}
	Now consider a quantum mechanical version of the game.  There are quantum states of two spin-$1/2$ degrees of freedom shared by Alice and Bob.  Alice can measure the $z$ component or $x$ components of the first spin $\vSA$.  (This corresponds to $Q = \pm1$ or $R = \pm1$.)  Bob can measure the $-(z+x)$ component or the $(z-x)$ component of the first spin $\vSB$.  (This corresponds to $S = \pm1$ or $T = \pm1$.)
	
	More specifically, Alice and Bob share the quantum state
	\beq
		\kpsi = \frac{\kupz \otimes \kdnz - \kdnz \otimes \kupz}{\sqrt{2}}.
	\eeq
	The operators to be measured are
	\begin{align*}
		Q &= \SzA, &
		R &= \SxA, &
		S &= -\frac{\SzB + \SxB}{\sqrt{2}}, &
		T &= \frac{\SxB - \SzB}{\sqrt{2}}.
	\end{align*}
	Let us consider the case when Alice measures $Q$ and Bob measures $T$.  Calculate the probability $P(Q, T)$ for Alice and Bob getting the measurement outcomes $(Q, T) = (\pm1, \pm1)$.
\end{problem}



\begin{problem}
	Similarly, consider the case when Alice measures $R$ and Bob measures $T$.  Calculate the probability $P(R, T)$ for Alice and Bob getting the measurement outcomes $(R, T) = (\pm1, \pm1)$.
\end{problem}



\begin{problem}
	Compute the expectation values $\vE(QT)$, $\vE(RT)$, $\vE(QS)$, and $\vE(RS)$.  Compute
	\beq
		\frac{\vE(QS) + \vE(RS) + \vE(RT) - \vE(QT)}{4}.
	\eeq
\end{problem}




\vfill
I consulted Sakurai's \emph{Modern Quantum Mechanics}, Shankar's \emph{Principles of Quantum Mechanics}, and Wolfram MathWorld while writing up these solutions.

\end{document}