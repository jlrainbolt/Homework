\documentclass[11pt]{article}
\usepackage{homework}

\classname{444}
\homeworknum{2}



\begin{document}

% Environments

\newcommand{\state}[2]{\begin{statement}{#1} #2 \end{statement}}
\newcommand{\prob}[2]{\begin{problem}{#1} #2 \end{problem}}
\newcommand{\subprob}[1]{\begin{subproblem} #1 \end{subproblem}}
\newcommand{\sol}[1]{\begin{solution} #1 \end{solution}}
\newcommand{\fig}[2]{\begin{figure} \centering #2  \label{#1} \end{figure}}

\newcommand{\makebib}{
	\vfill
	\color{black}
	\bibliography{references}{}
	\bibliographystyle{lucas_unsrt}
}
	

% Implication

\newcommand{\qwhere}{\quad \text{where} \quad}
\newcommand{\qimplies}{\quad \implies \quad}
\newcommand{\impliesq}{\implies \quad}



% Brackets

\newcommand{\paren}[1]{\left( #1 \right)}
\newcommand{\brac}[1]{\left[ #1 \right]}


% Greek

\newcommand{\alp}{\alpha}
\newcommand{\bet}{\beta}
\newcommand{\gam}{\gamma}
\newcommand{\del}{\delta}
\newcommand{\eps}{\epsilon}
\newcommand{\zet}{\zeta}
\newcommand{\tht}{\theta}
\newcommand{\kap}{\kappa}
\newcommand{\lam}{\lambda}
\newcommand{\sig}{\sigma}
\newcommand{\ups}{\upsilon}
\newcommand{\omg}{\omega}

\newcommand{\Gam}{\Gamma}
\newcommand{\Del}{\Delta}
\newcommand{\Tht}{\Theta}
\newcommand{\Lam}{\Lambda}
\newcommand{\Sig}{\Sigma}
\newcommand{\Omg}{\Omega}
% Problem 1

\newcommand{\Psii}{\Psi^i}
\newcommand{\Psiix}{\Psii(x)}

\newcommand{\Pii}{\Pi^i}

\newcommand{\Phii}{\Phi^i}
\newcommand{\Phiix}{\Phii(x)}
\newcommand{\PhiN}{\Phi^N}
\newcommand{\PhiNx}{\PhiN(x)}
\newcommand{\Phiq}{\Phi^1}
\newcommand{\Phiw}{\Phi^2}

\newcommand{\ddcx}{\dd[3]{x}}

\newcommand{\delij}{\del^{i j}}
\newcommand{\delkl}{\del^{k l}}
\newcommand{\delil}{\del^{i l}}
\newcommand{\deljk}{\del^{j k}}
\newcommand{\delik}{\del^{i k}}
\newcommand{\deljl}{\del^{j l}}

\newcommand{\DF}{D_F}

\newcommand{\sigx}{\sig(x)}

\newcommand{\pii}{\pi^i}
\newcommand{\pij}{\pi^j}
\newcommand{\pik}{\pi^k}
\newcommand{\pil}{\pi^l}
\newcommand{\piix}{\pi(x)}

\newcommand{\pq}{p_1}
\newcommand{\pw}{p_2}
\newcommand{\pe}{p_3}
\newcommand{\pr}{p_4}

\newcommand{\vp}{\vb{p}}
\newcommand{\vpsi}{\vp_i}

\newcommand{\mpi}{m_\pi}

\state{(Jackson 9.8)}{\ 
	%\emph{Hint:} The electromagnetic angular momentum density comes from more than the transverse (radiation zone) components of the fields.
}

%
%	Jackson 9.8(a)
%

\prob{}{
	Show that a classical oscillating electric dipole $\vp$ with fields given by
	\aln{ \label{fields1}
		\vH &= \frac{c k^2}{4\pi} (\nh \cross \vp) \frac{e^{i k r}}{r} \paren{ 1 - \frac{1}{i k r} }, &
		\vE &= \frac{1}{4\pi \epso} \curly{ k^2 (\nh \cross \vp) \cross \nh \frac{e^{i k r}}{r} + [ 3 \nh (\nh \vdot \vp) - \vp ] \paren{ \frac{1}{r^3} - \frac{i k}{r^2} } e^{i k r} },
	}
	radiates electromagnetic angular momentum to infinity at the rate
	\eq{
		\dv{\vL}{t} = \frac{k^3}{12 \pi \epso} \Im[ \vp^* \cross \vp ].
	}
	\vfix
}

\sol{
	According to Jackson~(9.20), the time-averaged angular momentum density is
	\eq{
		\vl = \frac{\Re[ \vx \cross (\vE \cross \vHs)}{2 c^2}.
	}
	One of the vector identities on the inside cover of Jackson is $\vaa \cross (\vbb \cross \vcc) = (\vaa \vdot \vcc) \vbb - (\vaa \vdot \vbb) \vcc$, so
	\eqn{l1}{
		\vl = \frac{(\vx \vdot \vHs) \vE - (\vx \vdot \vE) \vHs}{2 c^2}.
	}
	From Eq.~\refeq{fields1}, note that
	\eq{
		\vx \vdot \vHs \propto \vx \vdot (\nh \cross \vps)
		= \vps \vdot (\vx \cross \nh)
		= \vO,
	}
	where we have used the identity $\vaa \vdot (\vbb \cross \vcc) = \vcc \vdot (\vaa \cross \vbb)$ and the fact that $\nh$ points in the $\vx$ direction.  For $\vx \vdot \vE$, note that
	\al{
		\vx \vdot [ (\nh \cross \vp) \cross \nh ] &= -\vx \vdot [ \nh \cross (\nh \cross \vp) ]
		= -\vx \vdot [ (\nh \vdot \vp) \nh - (\nh \vdot \nh) \vp ]
		= -(\nh \vdot \vp) (\vx \vdot \nh) + \vx \vdot \vp \\
		&= -r (\nh \vdot \vp) + \vx \vdot \vp
		= \vx \vdot \vp - \vx \vdot \vp
		= 0, \\[1.5ex]
		\vx \vdot [ 3 \nh (\nh \vdot \vp) - \vp ] &= 3 (\vx \vdot \nh) (\nh \vdot \vp) - \vx \vdot \vp
		= 3r (\nh \vdot \vp) - \vx \vdot \vp
		= 3(\vx \vdot \vp) - \vx \vdot \vp
		= 2(\vx \vdot \vp),
	}
	since $\abs{\vx} = r$ and $\vx = r \,\nh$.  Then
	\eq{
		\vx \vdot \vE = \frac{1}{2\pi \epso} (\vx \vdot \vp) \paren{ \frac{1}{r^3} - \frac{i k}{r^2} } e^{i k r}
		= \frac{1}{2\pi \epso} (\nh \vdot \vp) \paren{ \frac{1}{r^2} - \frac{i k}{r} } e^{i k r}.
	}
	
	With these substitutions, Eq.~\refeq{l1} becomes
	\al{
		\vl &= -\frac{(\vx \vdot \vE) \vHs}{c^2}
		= -\frac{1}{4\pi \epso c^2} (\nh \vdot \vp) \paren{ \frac{1}{r^2} - \frac{i k}{r} } e^{i k r} \frac{c k^2}{4\pi} (\nh \cross \vps) \frac{e^{-i k r}}{r} \paren{ 1 + \frac{1}{i k r} } \\
		&= -\frac{k^2}{16\pi^2 \epso c r} (\nh \vdot \vp) (\nh \cross \vps) \paren{ \frac{1}{r^2} - \frac{i k}{r} } \paren{ 1 - \frac{i}{k r} }
		= -\frac{k^2}{16\pi^2 \epso c} (\nh \vdot \vp) (\nh \cross \vps) \paren{ \frac{1}{r^2} - \frac{i}{k r^3} - \frac{i k}{r} - \frac{1}{r^2} } \\
		&= -\frac{i k^2}{16\pi^2 \epso c r} (\nh \vdot \vp) (\nh \cross \vps) \paren{ \frac{1}{k r^3} + \frac{k}{r^2} }
		= \frac{i k^3}{16\pi^2 \epso c r^2} (\nh \vdot \vp) (\nh \cross \vps) \paren{ \frac{1}{k^2 r^2} + 1 }.
	}
	
	Let $\vL$ be the angular momentum radiated to a distance $R$.  Then
	\eq{
		\vL = \int_R \vl(r) \ddcx
		= \intopi \intotp \intoR \vl(r) \,r^2 \sin\tht \ddr \ddphi \dd\tht,
	}
	and the time derivative is
	\aln{
		\dv{\vL}{t} &= \dv{t}(\intopi \intotp \intoR \vl(r) \,r^2 \sin\tht \ddr \ddphi \dd\tht)
		= \dv{r}{t} \dv{r}(\intopi \intotp \intoR \vl(r) \,r^2 \sin\tht \ddr \ddphi \dd\tht) \notag \\
		&= c \intopi \intotp \vl(r) \,r^2 \sin\tht \ddphi \dd\tht
		= \frac{i k^3}{16\pi^2 \epso} \paren{ \frac{1}{k^2 r^2} + 1 } \intopi \intotp (\nh \vdot \vp) (\nh \cross \vps) \sin\tht \ddphi \dd\tht. \label{dLdt}
	}
	Note that
	\eq{
		[ (\nh \vdot \vp) (\nh \cross \vps) ]_i = \sumje n_j p_j (\nh \cross \vps)_i
		= \sumje \sumke \sumle \epsikl n_j p_j n_k p_l^*,
	}
	so
	\eq{
		\dv{L_i}{t} \propto \sumje \sumke \sumle \epsikl p_j p_l^* \int n_j p_k \ddOmg
		= \sumje \sumke \sumle \epsikl p_j p_l^* \frac{4\pi}{3} \del_{jk}
		= \frac{4\pi}{3} \epsikl p_k p_l^*
		= \frac{4\pi}{3} (\vp \cross \vps)_i,
	}
	where we have used Jackson~(9.47), $\int n_\bet n_\gam \ddOmg = 4\pi \del_{\bet \gam} / 3$.  Making this substitution into Eq.~\refeq{dLdt},
	\eq{
		\dv{\vL}{t} = \frac{i k^3}{6\pi \epso} \paren{ \frac{1}{k^2 r^2} + 1 } (\vp \cross \vps).
	}
	Taking the limit as $r \to \infty$, we find
	\eqn{ans1a}{
		\dv{\vL}{t} = \Re\!\brac{ \frac{i k^3}{12\pi \epso} (\vp \cross \vps) }
		= \Re\!\brac{ -\frac{i k^3}{12\pi \epso} (\vps \cross \vp) }
		= \ans{ \frac{k^3}{12\pi \epso} \Im[ \vps \cross \vp ], }
	}
	as desired. \qed
}

%
%	Jackson 9.8(b)
%

\prob{}{
	What is the ratio of angular momentum radiated to energy radiated?  Interpret.
}

\sol{
	According to Jackson~(9.24), the total power radiated by an oscillating electric dipole $\vp$ is
	\eq{
		P = \dv{E}{t}
		= \frac{c^2 \Zo k^4}{12 \pi} \abs{\vp}^2.
	}
	Then the ratio of angular momentum radiated to energy radiated is
	\eq{
		\frac{\dv*{\vL}{t}}{\dv*{E}{t}} = \frac{k^3}{12\pi \epso} \Im[ \vps \cross \vp ] \frac{12 \pi}{c^2 \Zo k^4 \abs{\vp}^2}
		= \frac{1}{\epso} \Im[ \vps \cross \vp ] \frac{1}{c^2 \Zo k \abs{\vp}^2}
		= \ans{ \frac{\Im[ \vps \cross \vp ]}{\omg \abs{\vp}^2}, }
	}
	where we have used $\Zo = \sqrt{\muo / \epso} = 1 / \sqrt{\epso^2 c^2} = 1 / \epso c$, $c^2 = 1 / (\epso \muo)$, and $\omg = k c$.
	
	In the limit of high frequency, $(\dv*{\vL}{t}) / (\dv*{E}{t}) \to 0$.  In this scenario, the energy radiated dominates over the angular momentum radiated.  Likewise, in the limit of low frequency, $(\dv*{\vL}{t}) / (\dv*{E}{t}) \to \infty$, meaning that angular momentum radiation dominates.  This is sensible because rotational kinetic energy $E \propto \omg^2$, while angular momentum $L \propto \omg$.
}

%
%	Jackson 9.8(c)
%

\prob{}{
	For a charge $e$ rotating in the $xy$ plane at radius $a$ and angular speed $\omg$, show that there is only a $z$ component of radiated angular momentum with magnitude $\dv*{\Lz}{t} = e^2 k^3 a^2 / 6 \pi \epso$.  What about a charge oscillating along the $z$ axis?
}

\sol{
	We know from Homework~5 that the position of a point charge rotating counterclockwise in the $xy$ plane is
	\eq{
		\vx(t) = a \cos(\omg t) \,\vx + a \sin(\omg t) \,\yh.
	}
	\clearpage
	Then the charge distribution is
	\eq{
		\rho(\vx, t) = e \del[ x - a \cos(\omg t) ] \,\del[ y - a \sin(\omg t) ] \,\del(z).
	}
	
	According to Jackson~(4.8), the dipole moment is defined
	\eq{
		\vp = \int \vx' \,\rho(\vx') \ddcxp.
	}
	The components of $\vp$ for the point charge are then
	\al{
		\px &= e \iiint x \,\del[ x - a \cos(\omg t) ] \,\del[ y - a \sin(\omg t) ] \,\del(z) \ddx \ddy \ddz
		= e a \cos(\omg t), \\
		\py &= e \iiint y \,\del[ x - a \cos(\omg t) ] \,\del[ y - a \sin(\omg t) ] \,\del(z) \ddx \ddy \ddz
		= e a \sin(\omg t), \\
		\pz &= e \iiint z \,\del[ x - a \cos(\omg t) ] \,\del[ y - a \sin(\omg t) ] \,\del(z) \ddx \ddy \ddz
		= 0,
	}
	so we can write $\vp = e a \,e^{-i \omg t} (\xh + i\,\yh).$  Substituting into Eq.~\refeq{ans1a},
	\al{
		\dv{\vL}{t} &= \Re\!\brac{ \frac{i k^3}{12\pi \epso} e^2 a^2 e^{-i \omg t} e^{i \omg t} [ (\xh + i\,\yh) \cross (\xh - i\,\yh) ] }
		= \Re\!\brac{ \frac{i e^2 k^3 a^2}{12\pi \epso} (-2i \,\xh \cross \yh) }
		= \Re\!\brac{ \frac{e^2 k^3 a^2}{6\pi \epso} \,\zh } \\
		&= \ans{ \frac{e^2 k^3 a^2}{6\pi \epso} \cos(\omg t) \,\zh, }
	}
	as desired. \qed
	
	A charge oscillating along the $z$ axis with amplitude $a$ has the charge density
	\eq{
		\rho(\vx, t) = e a \,\del(x) \,\del(y) \,\del[ z - \cos(\omg t) ],
	}
	which gives the dipole moment
	\al{
		\px &= e a \iiint x \,\del(x) \,\del(y) \,\del[ z - \cos(\omg t) ] \ddx \ddy \ddz
		= 0, \\
		\py &= e a \iiint y \,\del(x) \,\del(y) \,\del[ z - \cos(\omg t) ] \ddx \ddy \ddz
		= 0, \\
		\pz &= e a \iiint z \,\del(x) \,\del(y) \,\del[ z - \cos(\omg t) ] \ddx \ddy \ddz
		= e a \cos(\omg t).
	}
	In complex notation, $\vp = e a \,e^{-i\omg t} \,\zh$.  Substituting into Eq.~\refeq{ans1a}, we find
	\eq{
		\dv{\vL}{t} = \Re\!\brac{ \frac{i k^3}{12\pi \epso} e^2 a^2 e^{-i \omg t} e^{i \omg t} (\zh \cross \zh) }
		= \ans{ \vO. }
	}
	So we see that a charge undergoing linear motion does not lead to a radiated angular momentum, which is sensible.
}

%
%	Jackson 9.8(d)
%

\prob{}{
	What are the results corresponding to Probs.~{1(a)} and {1(b)} for magnetic dipole radiation?
}

\sol{
	The radiation fields for a magnetic dipole are given by Jackson~(19.35--36),
	\al{
		\vH &= \frac{1}{4\pi} \curly{ k^2 (\nh \cross \vm) \cross \nh \frac{e^{i k r}}{r} + [ 3 \nh (\nh \vdot \vm) - \vm ] \paren{ \frac{1}{r^3} - \frac{i k}{r^2} } e^{i k r} }, &
		\vE &= -\frac{\Zo}{4\pi} k^2 (\nh \cross \vm) \frac{e^{i k r}}{r} \paren{ 1 - \frac{1}{i k r} }.
	}
	\clearpage
	Comparing with Eq.~\refeq{fields1}, we see that $\vH \to -\vE / \Zo$, $\vE \to \Zo \vH$, and $\vp \to \vm / c$ as stated in the book~\cite[p.~413]{Jackson}.  Making these substitutions, the results of Probs.~{1.1(a)} and {(b)} become
	\al{
		\ans{ \dv{\vL}{t}\ }&\ans{= \frac{\muo k^3}{12\pi} \Im[ \vms \cross \vm ], } &
		\ans{ \frac{\dv*{\vL}{t}}{\dv*{E}{t}}\ }&\ans{= \frac{\Im[ \vms \cross \vm ]}{\omg \abs{\vm}^2} }
	}
	where we have used $\mu = 1 / \epso c^2$.
}






\state{Quantum statistical mechanics (Peskin \& Schroeder 9.2)}{\hfix}

\prob{
	Evaluate the quantum statistical partition function
	\eqn{given2a}{
		Z = \Tr(e^{-\bet H})
	}
	(where $\bet = 1 / k T$) using the strategy of Section~9.1 for evaluating the matrix elements of $e^{-i H t}$ in terms of functional integrals.  Show that one again finds a functional integral, over functions defined on a domain that is of length $\bet$ and periodically connected in the time direction.  Note that the Euclidean form of the Lagrangian appears in the weight.
}

\sol{
	The trace is independent of representation~\cite[pp.~38--39]{Sakurai}, so we may write it in the position basis, denoted by $q$.  In this basis, Eq.~\refeq{given2a} becomes
	\eqn{thing2a}{
		Z = \int \dddq \ev{e^{-\bet H}}{q}.
	}
	Following the steps of p.~280 of P\&S, we discretize the temperature interval into $N$ slices of width $\eps = \bet / N$.  Then we can write
	\eq{
		e^{-\bet H} = \prodkN e^{-\eps H}.
	}
	We insert a complete set of states between each of the factors of $e^{-\eps H}$, in the form
	\eq{
		1 = \int \dddqsk \ketbra{\qsk}.
	}
	So Eq.~\refeq{thing2a} can be written as
	\eq{
		Z = \int \dddq \bra{q} \paren{ \prodkNq \int \dddqsk e^{-\eps H} \ketbra{\qsk} } e^{-\eps H} \ket{q}
	}
	Taking the limit $\eps \to 0$ as in P\&S~(9.9), this becomes
	\aln{
		Z &= \int \dddq \bra{q} \paren{ \prodkNq \int \dddqsk (1 - \eps H) \ketbra{\qsk} } (1 - \eps H) \ket{q} \notag \\
		&= \int \dddq \paren{ \prodkqNq \int \dddqsk } \mel*{q}{(1 - \eps H)}{\qsNq} \cdots \mel*{\qsw}{(1 - \eps H)}{\qsq} \mel*{\qsq}{(1 - \eps H)}{q} \label{thing2a2}
	}
	Assuming $H$ can be written as $H(q, p) = f(q) + g(p)$, its matrix element can be written using P\&S~(9.10):
	\eq{
		\mel*{\qskq}{H(q, p)}{\qsk} = \int \dddpskf H(\qsk, \psk) \exp[ i \psk \cdot (\qskq - \qsk) ].
	}
	We can again use the $\eps \to 0$ limit to write $1 - \eps H$ and $e^{-\eps H}$~\cite[p.~281]{Peskin}.  Then our matrix elements in Eq.~\refeq{thing2a2} can be written as
	\eq{
		\mel*{\qskq}{e^{-\eps H}}{\qsk} = \int \dddpskf \exp[ i \psk \cdot (\qskq - \qsk) - \eps H(\qsk, \psk) ].
	}
	Now Eq.~\refeq{thing2a2} can be written in a form similar to P\&S~(9.11),
	\al{
		Z &= \int \dddq \paren{ \prodkqNq \int \dddqsk } \paren{ \prodkN \int \dddpskf } \exp[ i \psNq \cdot (q - \qsNq) - \eps H(\qsNq, \psNq) ] \\[-1ex]
		&\hspace{20em} \phantom{=\ } \times \exp\curly{ \sumkqNw \Big[ i \psk \cdot (\qskq - \qsk) - \eps H(\qsk, \psk) \Big] } \\[1ex]
		&\hspace{25em} \phantom{=\ } \times \exp[ i \psN \cdot (\qsq - q) - \eps H(q, \psN) ].
	}
	Let $q = \qsq = \qsNpq$ and $\eps' = i \eps$.  Then we can write $Z$ in a form even more similar to (9.11),
	\eqn{thing2a3}{
		Z = \paren{ \prodkN \int \dddqsk \int \dddpskf } \exp\curly{ i \sumkN \Big[ \psk \cdot (\qskq - \qsk) + \eps' H(\qsk, \psk) \Big] }.
	}
	Assuming the Hamiltonian takes the form
	\eq{
		H = \frac{p^2}{2m} + V(q),
	}
	we can integrate over the momenta by completing the square as on p.~282 of P\&S.  Adapting their expression, we find
	\eq{
		\int \dddpskf \exp[ i \paren{ \psk \cdot (\qskq - \qsk) + \eps' \frac{\psk^2}{2 m} } ] = \paren{ \frac{i m}{2 \pi \eps'} }^{d / 2} \exp[ -\frac{i m}{2 \eps'} (\qskq - \qsk)^2 ].
	}
	Feeding this into Eq.~\refeq{thing2a3} yields
	\eq{
		Z  = \paren{ \frac{i m}{2 \pi \eps'} }^{N d / 2} \paren{ \prodkN \int \dddqsk \int \dddpskf } \exp\curly{ -i \sumkN \brac{ \frac{m}{2 \eps'} (\qskq - \qsk)^2 - \eps' H(\qsk, \psk) } },
	}
	which resembles (9.13).  Since (9.13) is the discretized form of (9.3), we adapt (9.3) and Eq.~\refeq{disc} as we take $\eps' \to 0$ to write
	\eqn{ans2a}{
		\ans{ Z = \int \cD \qT \, s e^{-\SE[\qT]},
		\qwhere
		\SE = \intobet \ddT \paren{ \frac{m}{2} \qd^2 + V(q) }
		\equiv \intobet \ddT \LE. }
	}
	Here we have defined $\LE$, the Euclidean form of the Lagrangian.  Thus we have shown that $Z$ is a functional integral over functions, with $p$ and $q$ defined on an interval of length $\bet$ that is periodically connected; that is, $q(0) = q(\bet)$ and $p(0) = p(\bet)$.  The periodicity was imposed when we set $\qsq = \qsNpq$. \qed
}



\prob{
	Evaluate this integral for a simple harmonic oscillator,
	\eqn{lagr}{
		\LE = \frac{1}{2} \xd^2 + \frac{1}{2} \omg^2 x^2,
	}
	by introducing a Fourier decomposition of $\xt$:
	\eqn{fourier}{
		\xt = \sumn \xn \frac{1}{\sqrt{\bet}} e^{2 \pi i n t / \bet}.
	}
	The dependence of the result on $\bet$ is a bit subtle to obtain explicitly, since the measure for the integral over $\xt$ depends on $\bet$ in any discretization.  However, the dependence on $\omg$ should be unambiguous.  Show that, up to a (possibly divergent and $\bet$-dependent) constant, the integral reproduces exactly the familiar expression for the quantum partition function of an oscillator.  [You may find the identity
	\eq{
		\sinh z = z \prodnqi \paren{ 1 + \frac{z^2}{(n \pi)^2} }
	}
	useful.]
}

\sol{
	From Eq.~\refeq{fourier}, note that
	\eq{
		\xd = \dv{t}( \sumn \xn \frac{1}{\sqrt{\bet}} e^{2 \pi i n t / \bet} )
		= \sumn \xn \frac{2 \pi i n}{\bet^{3 / 2}} e^{2 \pi i n t / \bet}.
	}
	Feeding this result into Eq.~\refeq{lagr} yields
	\al{
		\LE& = \frac{1}{2} \paren{ \sumn \xn \frac{2 \pi i n}{\bet^{3 / 2}} e^{2 \pi i n t / \bet} } \paren{ \summ \xmm \frac{2 \pi i m}{\bet^{3 / 2}} e^{2 \pi i m t / \bet} } + \frac{\omg^2}{2} \paren{ \sumn \xn \frac{1}{\sqrt{\bet}} e^{2 \pi i n t / \bet} } \paren{ \summ \xmm \frac{1}{\sqrt{\bet}} e^{2 \pi i m t / \bet} } \\
		&= \frac{1}{2} \sumn \summ \paren{ -\frac{(2 \pi)^2}{\bet^3} n m \xn \xmm e^{2 \pi i (n + m) t / \bet} + \frac{\omg^2}{\bet} e^{2 \pi i (n + m) t / \bet} } \\
		&= \frac{1}{2 \bet} \sumn \summ \xn \xmm e^{2 \pi i (n + m) t / \bet} \paren{ -\frac{(2 \pi)^2}{\bet^2} n m + \omg^2 }.
	}
	Now from Eq.~\refeq{ans2a}, we have
	\aln{
		\SE &= \frac{1}{2 \bet} \intobet \ddt \sumn \summ \xn \xmm e^{2 \pi i (n + m) t / \bet} \paren{ -\frac{(2 \pi)^2}{\bet^2} n m + \omg^2 } \notag \\
		&= \frac{1}{2} \frac{1}{2 \pi} \intotp \ddphi \sumn \summ \xn \xmm e^{i (n + m) \phi} \paren{ -\frac{(2 \pi)^2}{\bet^2} n m + \omg^2 } \notag \\
		&= \frac{1}{2} \sumn \summ \xn \xmm \delnmm \paren{ -\frac{(2 \pi)^2}{\bet^2} n m + \omg^2 } \label{thing2b}
	}
	where we have defined $\phi = 2 \pi t / \bet$ and applied~\cite{Kronecker}
	\eq{
		\delxn = \frac{1}{2 \pi} \intotp e^{i (x - n) \phi} \ddphi.
	}
	Picking up from Eq.~\refeq{thing2b},
	\eq{
		\SE = \frac{1}{2} \sumnii \xn \xmn \paren{ \frac{(2 \pi)^2}{\bet^2} n^2 + \omg^2 }
		= \sumnoi \abs{\xn}^2 \brac{ \paren{ \frac{2 n \pi}{\bet} }^2 + \omg^2 }
		= \xo^2 \omg^2 + \sumnqi \abs{\xn}^2 \brac{ \paren{ \frac{2 n \pi}{\bet} }^2 + \omg^2 },
	}
	where we have used the fact that since $x$ is real, Eq.~\refeq{fourier} implies $\xmn = \xn^*$~\cite[p.~285]{Peskin}.  This also means $\xo$ is real.  Referring once more to Eq.~\refeq{ans2a}, we have the integral
	\eq{
		Z = \int \cD \xt \exp\curly{ -\abs{\xo}^2 \omg^2 - \sumnqi \abs{\xn}^2 \brac{ \paren{ \frac{2 n \pi}{\bet} }^2 + \omg^2 } }.
	}
	We can rewrite the integral as~\cite[p.~285]{Peskin}
	\eq{
		\cD \xt = \prodno \ddRexn \ddImxn.
	}
	This yields
	\aln{
		Z &= \int \ddxo e^{-\omg^2 \xo^2} \paren{ \prodno \int \ddRexn \int \ddImxn } \exp\curly{ -\sumnqi \brac{ (\Rexn)^2 + (\Imxn)^2 } \brac{ \paren{ \frac{2 n \pi}{\bet} }^2 + \omg^2 } } \notag \\[2ex]
		%
		&= \int \ddxo e^{-\omg^2 \xo^2} \paren{ \prodno \int \ddRexn \exp\curly{ -\sumnqi (\Rexn)^2 \brac{ \paren{ \frac{2 n \pi}{\bet} }^2 + \omg^2 } }  } \notag \\
		&\hspace{10em} \phantom{=\ } \times \paren{ \prodno \int \ddImxn \exp\curly{ -\sumnqi (\Imxn)^2 \brac{ \paren{ \frac{2 n \pi}{\bet} }^2 + \omg^2 } } } \notag \\[2ex]
		%
		&= \frac{\sqrt{2 \pi}}{\omg} \paren{ \prodno \sqrt{ \frac{\pi}{(2 n \pi / \bet)^2 + \omg^2}} } \paren{ \prodno \sqrt{ \frac{\pi}{(2 n \pi / \bet)^2 + \omg^2}} }. \label{thing2b2}
	}
	where we have used~\cite{QFT}
	\eq{
		\intnii e^{-a x^2 / 2} = \sqrt{ \frac{2 \pi}{a} }.
	}
	Picking up from Eq.~\refeq{thing2b2}, we have
	\al{
		Z &= \frac{\sqrt{2 \pi}}{\omg} \prodno \frac{\pi}{(2 n \pi / \bet)^2 + \omg^2} \\
		&= \frac{\sqrt{2 \pi}}{\omg} \prodno \paren{ \frac{\bet}{2 n \pi} }^2 \frac{\pi}{1 + (\omg \bet / 2 n \pi)^2} \\
		&= \frac{\bet \sqrt{2 \pi}}{2} \paren{ \prodno \frac{\bet^2}{4 \pi n^2} } \frac{1}{\omg \bet / 2} \paren{ \prodno \frac{1}{1 + (\omg \bet / 2)^2 / (n \pi)^2} } \\
		&= \ans{ \frac{\bet \sqrt{2 \pi}}{2} \paren{ \prodno \frac{\bet^2}{4 \pi n^2} } \frac{1}{\sinh(\omg \bet / 2)}. }
	}
	We note that $\prodno \bet^2 / 4 \pi n^2$ is a divergent, beta-dependent constant.  Otherwise, this is what we would expect for the quantum partition function of a harmonic oscillator. \qed
}



\prob{
	Generalize this construction to field theory.  Show that the quantum statistical partition function for a free scalar field can be written in terms of a functional integral.  The value of this integral is given formally by
	\eq{
		[ \det(-\pt^2 + m^2) ]^{-1 / 2},
	}
	where the operator acts on functions on Euclidean space that are periodic in the time direction with periodicity $\bet$.  As before, the $\bet$ dependence of this expression is difficult to compute directly.  However, the dependence on $m^2$ is unambiguous.  Show that the determinant indeed reproduces the partition function for relativistic scalar particles.
}

\makebib

\end{document}
