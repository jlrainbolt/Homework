\documentclass[11pt]{article}
\usepackage{homework}

\classname{444}
\homeworknum{2}



\begin{document}

% Environments

\newcommand{\state}[2]{\begin{statement}{#1} #2 \end{statement}}
\newcommand{\prob}[2]{\begin{problem}{#1} #2 \end{problem}}
\newcommand{\subprob}[1]{\begin{subproblem} #1 \end{subproblem}}
\newcommand{\sol}[1]{\begin{solution} #1 \end{solution}}
\newcommand{\fig}[2]{\begin{figure} \centering #2  \label{#1} \end{figure}}

\newcommand{\makebib}{
	\vfill
	\color{black}
	\nocite{*}
	\bibliography{references}{}
	\bibliographystyle{lucas_unsrt}
}
	

% Implication

\newcommand{\qwhere}{\quad \text{where} \quad}
\newcommand{\qimplies}{\quad \implies \quad}
\newcommand{\impliesq}{\implies \quad}



% Brackets

\newcommand{\paren}[1]{\left( #1 \right)}
\newcommand{\brac}[1]{\left[ #1 \right]}
\newcommand{\curly}[1]{\left\{ #1 \right\}}


% Greek

\newcommand{\alp}{\alpha}
\newcommand{\bet}{\beta}
\newcommand{\gam}{\gamma}
\newcommand{\del}{\delta}
\newcommand{\eps}{\epsilon}
\newcommand{\zet}{\zeta}
\newcommand{\tht}{\theta}
\newcommand{\kap}{\kappa}
\newcommand{\lam}{\lambda}
\newcommand{\sig}{\sigma}
\newcommand{\ups}{\upsilon}
\newcommand{\omg}{\omega}

\newcommand{\Gam}{\Gamma}
\newcommand{\Del}{\Delta}
\newcommand{\Tht}{\Theta}
\newcommand{\Lam}{\Lambda}
\newcommand{\Sig}{\Sigma}
\newcommand{\Omg}{\Omega}


% Text

\newcommand{\where}{\text{where }}

% Problem 1

\newcommand{\Hint}{H_\text{int}}
\newcommand{\ddcx}{\dd[3]{x}}
\newcommand{\psib}{\bar{\psi}}

\newcommand{\mh}{m_h}
\newcommand{\mmu}{m_\mu}
\newcommand{\me}{m_e}
\newcommand{\ma}{m_a}

\newcommand{\aexpt}{a_\text{expt.}}
\newcommand{\aQED}{a_\text{QED}}
\renewcommand{\GeV}{\giga\electronvolt}

\newcommand{\gamt}{\gam^5}



\state{}{
	Consider the path integral for a single point particle, with the action
	\eqn{given1}{
		S = \intoq \ddt \brac{ \psmt \xdmt + \frac{\Nt}{2} [ \pst - m^2 - i \eps ] }.
	}
	This represents the quantization of the coordinates and momenta of the particle, subject to the mass shell constraint $p^2 = m^2$ (together with the $i \eps$ prescription) imposed by the Lagrange multiplier $N$.  This action admits the reparametrization symmetry $\del x = \alp p$, $\del p = 0$, $\del N = -\ptst \alp$ where $\alpt$ is any function.  This symmetry allows us to fix the gauge condition $\Nt = T$; the constant $T$ must still be integrated over, however.
}

\prob{
	Path integrate over $\xt$, subject to the boundary conditions $\xm(0) = \xm$, $\xm(1) = \ym$, yielding a delta function $\delpd$ along the path.  Solve this constraint (find the set of functions that solve it) and path integrate over those $\ptt$ to find the quantum mechanical propagation amplitude
	\eqn{show1a}{
		\bkyx = \DF(x - y)
		= \intoi \ddT (2 \pi i T)^{-d / 2} \exp[ -\frac{i}{2} \paren{ (m^2 - i \eps) T + \frac{(x - y)^2}{T} } ],
	}
	where $d$ is the number of spacetime dimensions.
}

\sol{


%	The amplitude for a particle to propagate from $\xa$ to $\xb$ in time $T$ may be found by P\&S~(9.3),
%	\eq{
%		\mel*{\xb}{e^{-i H T / \hbar}}{\xa} = U(\xa, \xb; T)
%		= \int \cD \xt e^{i S[\xt] / \hbar},
%	}
%	where the path integral is defined by (9.4),
%	\eq{
%		\int \cD \xt \equiv \frac{1}{\Ceps} \prodk \intnii \ddxkf,
%	}
%	with $\Ceps$ a constant.  The sum runs over $N$ time slices of duration $\eps = T / N$, and the act of dividing the path up into these pieces is called discretization.  The action for the discretized path is~\cite[p.~277]{Peskin}
%	\eq{
%		S = \intoT \ddt \paren{ \frac{m}{2} \xd^2 - V(x) } \to \sumk \brac{ \frac{m}{2} \frac{(\xkq - \xk)^2}{\eps} - \eps V\paren{ \frac{\xkq + \xk}{2} } }.
%	}
%	
%	In order to perform a path integral over $\xt$, we write Eq.~\refeq{given1} as
%	\eq{
%		S = \intoq \ddt \brac{ \psmt \xdmt + \frac{\Nt}{2} [ \pst - m^2 - i \eps ] }
%	}
%	
%	We discretize the action in Eq.~\refeq{given1} as
%	\eq{
%		S = \sumk \Sk
%	}
	
	
	
	
%	sub $T$ for $\Nt$
%	
%	First we need to discretize the action~\cite[p.~277]{Peskin}
%	\eq{
%		S = \sumk \Sk
%		= \sumk \intoq \ddt \brac{ \psmt \frac{\xmkq - \xmk}{\eps} + \eps \frac{\Tk}{2} [ \pk^2(t) - m^2 - i \eps ] }.
%	}
%	
%	\hl{(9.4)}
%	\eq{
%		\cP \equiv \frac{1}{\Ceps \Deps \Feps} \prodk \int \ddTkf \int \dddpktf \int \dddxktf e^{i \Sk / \hbar}
%	}
%	
%	
%	or p.~278
%	
%	\eq{
%		U(\xq, \xN; 1) = \intnii \dddxptf \exp[ \frac{i}{\hbar} \paren{ {\psm}_{N-1} (\xN^\mu - {x'}^\mu + \eps \frac{T_{N-1}}{2} [ p_{N-1}^2 - m^2 - i \eps ] } ] U(\xq, x'; 1 - \eps)
%	}
}


\clearpage
\prob{
	Use this integral representation to show that $\DF$ satisfies
	\eqn{show1b}{
		(\pt^2 + m^2) \DF = i \deld(x - y).
	}
}

\sol{
	Feeding Eq.~\refeq{show1a} into the left-hand side of Eq.~\refeq{show1b}, we have
	\eq{
		(\pt^2 + m^2) \DF = m^2 \DF + \intoi \ddT (2 \pi i T)^{-d / 2} \pt^2 \exp[ -\frac{i}{2} \paren{ (m^2 - i \eps) T + \frac{(x - y)^2}{T} } ],
	}
	where
	\al{
		\pt^2 &\exp[ -\frac{i}{2} \paren{ (m^2 - i \eps) T + \frac{(x - y)^2}{T} } ] = \pt\curly{ -\frac{i}{2} \pt\paren{ \frac{(x - y)^2}{T} } \exp[ -\frac{i}{2} \paren{ (m^2 - i \eps) T + \frac{(x - y)^2}{T} } ] } \\[2ex]
		&= -\frac{i}{2} \pt^2\paren{ \frac{(x - y)^2}{T} } \exp[ -\frac{i}{2} \paren{ (m^2 - i \eps) T + \frac{(x - y)^2}{T} } ] \\
		&\hspace{10em} \phantom{=\ } - \frac{i}{2} \pt\paren{ \frac{(x - y)^2}{T} } \pt\curly{ \exp[ -\frac{i}{2} \paren{ (m^2 - i \eps) T + \frac{(x - y)^2}{T} } ] } \\[2ex]
		&= -\frac{i}{2} \pt^2\paren{ \frac{(x - y)^2}{T} } \exp[ -\frac{i}{2} \paren{ (m^2 - i \eps) T + \frac{(x - y)^2}{T} } ] \\
		&\hspace{10em} \phantom{=\ } - \frac{i}{2} \pt\paren{ \frac{(x - y)^2}{T} } \curly{ -\frac{i}{2} \pt\paren{ \frac{(x - y)^2}{T} } \exp[ -\frac{i}{2} \paren{ (m^2 - i \eps) T + \frac{(x - y)^2}{T} } ] } \\[2ex]
		&= -\frac{i}{2} \pt^2\paren{ \frac{(x - y)^2}{T} } \exp[ -\frac{i}{2} \paren{ (m^2 - i \eps) T + \frac{(x - y)^2}{T} } ] \\
		&\hspace{10em} \phantom{=\ } - \frac{1}{4} \brac{ \pt\paren{ \frac{(x - y)^2}{T} } }^2 \exp[ -\frac{i}{2} \paren{ (m^2 - i \eps) T + \frac{(x - y)^2}{T} } ] \\[2ex]
		&= -\brac{ \frac{i}{2 T} \pt^2(x - y)^2 + \frac{1}{4 T^2} \brac{ \pt(x - y)^2 }^2 } \exp[ -\frac{i}{2} \paren{ (m^2 - i \eps) T + \frac{(x - y)^2}{T} } ].
	}
	
	Note that
	\eq{
		\pt(x - y)^2 %= \ptsn (\xm - \ym)^2
		= \ptsn (\xsm \xm - 2 \xsm \ym + \ysm \ym),
	}
	so
	\eq{
		\brac{ \pt(x - y)^2 }^2 = \ptsn (\xsm \xm - 2 \xsm \ym + \ysm \ym) \ptn (\xsm \xm - 2 \xsm \ym + \ysm \ym).
	}
	Note also that
	\eq{
		\pt^2(x - y)^2 = \ptsn \ptn (\xsm \xm - 2 \xsm \ym + \ysm \ym)
	}
%	and that
%	\al{
%		 \pdv[2]{t}(x^2 - 2 x y + y^2) = \pdv[2]{t} \dv{x}{t} \dv{x^2}{x} - 2 x \pdv[2]{y}{t} - 2 y \pdv[2]{x}{t} 
%	}
}



\clearpage
\prob{
	Evaluate the $T$ integral in terms of Bessel functions.
}

\sol{
	The integral in Eq.~\refeq{show1a} has the form~\cite[p.~368]{}
	\eq{
		\intoi x^{\nu - 1} e^{-\bet / x - \gam x} \dd{x} = 2 \paren{ \frac{\bet}{\gam} }^{\nu / 2} K_\nu(2 \sqrt{\bet \gam}),
	}
	where $K_\nu$ is the modified Bessel function of the second kind.  To evaluate Eq.~\refeq{show1a}m we note that
	\al{
		x &\to T, &
		\nu &\to 1 - \frac{d}{2}, &
		\bet &\to \frac{i}{2} (x - y)^2, &
		\gam &\to \frac{i}{2} (m^2 - i \eps).
	}
	So we have
	\al{
		\bkyx &= (2 \pi i)^{-d / 2} \intoi \ddT T^{d / 2} \exp[ -\frac{i}{2} \paren{ (m^2 - i \eps) T + \frac{(x - y)^2}{T} } ] \\
		&= 2 (2 \pi i)^{-d / 2} \paren{ \frac{(x - y)^2}{m^2 - i \eps} }^{1 / 2 - d / 4} K_{1 - d / 2}\paren{ 2 \sqrt{-\frac{(x - y)^2 (m^2 - i \eps)}{4} } } \\
		&= \ans{ 2^{1 - d / 2} (\pi i)^{-d / 2} \paren{ \frac{(x - y)^2}{m^2 - i \eps} }^{1 / 2 - d / 4} K_{1 - d / 2}\paren{ i \sqrt{(x - y)^2 (m^2 - i \eps)} }. }
	}
	\vfix
}






\clearpage
\state{Quantum statistical mechanics (Peskin \& Schroeder 9.2)}{\hfix}

\prob{
	Evaluate the quantum statistical partition function
	\eq{
		Z = \Tr(e^{-\bet H})
	}
	(where $\bet = 1 / k T$) using the strategy of Section~9.1 for evaluating the matrix elements of $e^{-i H t}$ in terms of functional integrals.  Show that one again finds a functional integral, over functions defined on a domain that is of length $\bet$ and periodically connected in the time direction.  Note that the Euclidean form of the Lagrangian appears in the weight.
}



\prob{
	Evaluate this integral for a simple harmonic oscillator,
	\eq{
		\LE = \frac{1}{2} \xd^2 + \frac{1}{2} \omg^2 x^2,
	}
	by introducing a Fourier decomposition of $\xt$:
	\eq{
		\xt = \sumn \xn \frac{1}{\sqrt{\bet}} e^{2 \pi i n t / \bet}.
	}
	The dependence of the result on $\bet$ is a bit subtle to obtain explicitly, since the measure for the integral over $\xt$ depends on $\bet$ in any discretization.  However, the dependence on $\omg$ should be unambiguous.  Show that, up to a (possibly divergent and $\bet$-dependent) constant, the integral reproduces exactly the familiar expression for the quantum partition function of an oscillator.  [You may find the identity
	\eq{
		\sinh z = z \prodnqi \paren{ 1 + \frac{z^2}{(n \pi)^2} }
	}
	useful.]
}



\prob{
	Generalize this construction to field theory.  Show that the quantum statistical partition function for a free scalar field can be written in terms of a functional integral.  The value of this integral is given formally by
	\eq{
		[ \det(-\pt^2 + m^2) ]^{-1 / 2},
	}
	where the operator acts on functions on Euclidean space that are periodic in the time direction with periodicity $\bet$.  As before, the $\bet$ dependence of this expression is difficult to compute directly.  However, the dependence on $m^2$ is unambiguous.  Show that the determinant indeed reproduces the partition function for relativistic scalar particles.
}

%\makebib

\end{document}
