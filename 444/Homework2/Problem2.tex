\state{Quantum statistical mechanics (Peskin \& Schroeder 9.2)}{\hfix}

\prob{
	Evaluate the quantum statistical partition function
	\eqn{given2a}{
		Z = \Tr(e^{-\bet H})
	}
	(where $\bet = 1 / k T$) using the strategy of Section~9.1 for evaluating the matrix elements of $e^{-i H t}$ in terms of functional integrals.  Show that one again finds a functional integral, over functions defined on a domain that is of length $\bet$ and periodically connected in the time direction.  Note that the Euclidean form of the Lagrangian appears in the weight.
}

\sol{
	The trace is independent of representation~\cite[pp.~38--39]{Sakurai}, so we may write it in the position basis, denoted by $q$.  In this basis, Eq.~\refeq{given2a} becomes
	\eqn{thing2a}{
		Z = \int \dddq \ev{e^{-\bet H}}{q}.
	}
	Following the steps of p.~280 of P\&S, we discretize the temperature interval into $N$ slices of width $\eps = \bet / N$.  Then we can write
	\eq{
		e^{-\bet H} = \prodkN e^{-\eps H}.
	}
	We insert a complete set of states between each of the factors of $e^{-\eps H}$, in the form
	\eq{
		1 = \int \dddqsk \ketbra{\qsk}.
	}
	So Eq.~\refeq{thing2a} can be written as
	\eq{
		Z = \int \dddq \bra{q} \paren{ \prodkNq \int \dddqsk e^{-\eps H} \ketbra{\qsk} } e^{-\eps H} \ket{q}
	}
	Taking the limit $\eps \to 0$ as in P\&S~(9.9), this becomes
	\aln{
		Z &= \int \dddq \bra{q} \paren{ \prodkNq \int \dddqsk (1 - \eps H) \ketbra{\qsk} } (1 - \eps H) \ket{q} \notag \\
		&= \int \dddq \paren{ \prodkqNq \int \dddqsk } \mel*{q}{(1 - \eps H)}{\qsNq} \cdots \mel*{\qsw}{(1 - \eps H)}{\qsq} \mel*{\qsq}{(1 - \eps H)}{q} \label{thing2a2}
	}
	Assuming $H$ can be written as $H(q, p) = f(q) + g(p)$, its matrix element can be written using P\&S~(9.10):
	\eq{
		\mel*{\qskq}{H(q, p)}{\qsk} = \int \dddpskf H(\qsk, \psk) \exp[ i \psk \cdot (\qskq - \qsk) ].
	}
	We can again use the $\eps \to 0$ limit to write $1 - \eps H$ and $e^{-\eps H}$~\cite[p.~281]{Peskin}.  Then our matrix elements in Eq.~\refeq{thing2a2} can be written as
	\eq{
		\mel*{\qskq}{e^{-\eps H}}{\qsk} = \int \dddpskf \exp[ i \psk \cdot (\qskq - \qsk) - \eps H(\qsk, \psk) ].
	}
	Now Eq.~\refeq{thing2a2} can be written in a form similar to P\&S~(9.11),
	\al{
		Z &= \int \dddq \paren{ \prodkqNq \int \dddqsk } \paren{ \prodkN \int \dddpskf } \exp[ i \psNq \cdot (q - \qsNq) - \eps H(\qsNq, \psNq) ] \\[-1ex]
		&\hspace{20em} \phantom{=\ } \times \exp\curly{ \sumkqNw \Big[ i \psk \cdot (\qskq - \qsk) - \eps H(\qsk, \psk) \Big] } \\[1ex]
		&\hspace{25em} \phantom{=\ } \times \exp[ i \psN \cdot (\qsq - q) - \eps H(q, \psN) ].
	}
	Let $q = \qsq = \qsNpq$ and $\eps' = i \eps$.  Then we can write $Z$ in a form even more similar to (9.11),
	\eqn{thing2a3}{
		Z = \paren{ \prodkN \int \dddqsk \int \dddpskf } \exp\curly{ i \sumkN \Big[ \psk \cdot (\qskq - \qsk) + \eps' H(\qsk, \psk) \Big] }.
	}
	Assuming the Hamiltonian takes the form
	\eq{
		H = \frac{p^2}{2m} + V(q),
	}
	we can integrate over the momenta by completing the square as on p.~282 of P\&S.  Adapting their expression, we find
	\eq{
		\int \dddpskf \exp[ i \paren{ \psk \cdot (\qskq - \qsk) + \eps' \frac{\psk^2}{2 m} } ] = \paren{ \frac{i m}{2 \pi \eps'} }^{d / 2} \exp[ -\frac{i m}{2 \eps'} (\qskq - \qsk)^2 ].
	}
	Feeding this into Eq.~\refeq{thing2a3} yields
	\eq{
		Z  = \paren{ \frac{i m}{2 \pi \eps'} }^{N d / 2} \paren{ \prodkN \int \dddqsk \int \dddpskf } \exp\curly{ -i \sumkN \brac{ \frac{m}{2 \eps'} (\qskq - \qsk)^2 - \eps' H(\qsk, \psk) } },
	}
	which resembles (9.13).  Since (9.13) is the discretized form of (9.3), we adapt (9.3) and Eq.~\refeq{disc} as we take $\eps' \to 0$ to write
	\eqn{ans2a}{
		\ans{ Z = \int \cD \qT \, e^{-\SE[\qT]},
		\qwhere
		\SE = \intobet \ddT \paren{ \frac{m}{2} \qd^2 + V(q) }
		\equiv \intobet \ddT \LE. }
	}
	Here we have defined $\LE$, the Euclidean form of the Lagrangian.  Thus we have shown that $Z$ is a functional integral over functions, with $p$ and $q$ defined on an interval of length $\bet$ that is periodically connected; that is, $q(0) = q(\bet)$ and $p(0) = p(\bet)$.  The periodicity was imposed when we set $\qsq = \qsNpq$. \qed
}



\prob{ \label{2b}
	Evaluate this integral for a simple harmonic oscillator,
	\eqn{lagr}{
		\LE = \frac{1}{2} \xd^2 + \frac{1}{2} \omg^2 x^2,
	}
	by introducing a Fourier decomposition of $\xt$:
	\eqn{fourier}{
		\xt = \sumn \xn \frac{1}{\sqrt{\bet}} e^{2 \pi i n t / \bet}.
	}
	The dependence of the result on $\bet$ is a bit subtle to obtain explicitly, since the measure for the integral over $\xt$ depends on $\bet$ in any discretization.  However, the dependence on $\omg$ should be unambiguous.  Show that, up to a (possibly divergent and $\bet$-dependent) constant, the integral reproduces exactly the familiar expression for the quantum partition function of an oscillator.  [You may find the identity
	\eq{
		\sinh z = z \prodnqi \paren{ 1 + \frac{z^2}{(n \pi)^2} }
	}
	useful.]
}

\sol{
	From Eq.~\refeq{fourier}, note that
	\eq{
		\xd = \dv{t}( \sumn \xn \frac{1}{\sqrt{\bet}} e^{2 \pi i n t / \bet} )
		= \sumn \xn \frac{2 \pi i n}{\bet^{3 / 2}} e^{2 \pi i n t / \bet}.
	}
	Feeding this result into Eq.~\refeq{lagr} yields
	\al{
		\LE& = \frac{1}{2} \paren{ \sumn \xn \frac{2 \pi i n}{\bet^{3 / 2}} e^{2 \pi i n t / \bet} } \paren{ \summ \xmm \frac{2 \pi i m}{\bet^{3 / 2}} e^{2 \pi i m t / \bet} } + \frac{\omg^2}{2} \paren{ \sumn \xn \frac{1}{\sqrt{\bet}} e^{2 \pi i n t / \bet} } \paren{ \summ \xmm \frac{1}{\sqrt{\bet}} e^{2 \pi i m t / \bet} } \\
		&= \frac{1}{2} \sumn \summ \paren{ -\frac{(2 \pi)^2}{\bet^3} n m \xn \xmm e^{2 \pi i (n + m) t / \bet} + \frac{\omg^2}{\bet} e^{2 \pi i (n + m) t / \bet} } \\
		&= \frac{1}{2 \bet} \sumn \summ \xn \xmm e^{2 \pi i (n + m) t / \bet} \paren{ -\frac{(2 \pi)^2}{\bet^2} n m + \omg^2 }.
	}
	Now from Eq.~\refeq{ans2a}, we have
	\aln{
		\SE &= \frac{1}{2 \bet} \intobet \ddt \sumn \summ \xn \xmm e^{2 \pi i (n + m) t / \bet} \paren{ -\frac{(2 \pi)^2}{\bet^2} n m + \omg^2 } \notag \\
		&= \frac{1}{2} \frac{1}{2 \pi} \intotp \ddphi \sumn \summ \xn \xmm e^{i (n + m) \phi} \paren{ -\frac{(2 \pi)^2}{\bet^2} n m + \omg^2 } \notag \\
		&= \frac{1}{2} \sumn \summ \xn \xmm \delnmm \paren{ -\frac{(2 \pi)^2}{\bet^2} n m + \omg^2 } \label{thing2b}
	}
	where we have defined $\phi = 2 \pi t / \bet$ and applied~\cite{Kronecker}
	\eqn{kronecker}{
		\delxn = \frac{1}{2 \pi} \intotp e^{i (x - n) \phi} \ddphi.
	}
	Picking up from Eq.~\refeq{thing2b},
	\eq{
		\SE = \frac{1}{2} \sumnii \xn \xmn \paren{ \frac{(2 \pi)^2}{\bet^2} n^2 + \omg^2 }
		= \sumnoi \abs{\xn}^2 \brac{ \paren{ \frac{2 n \pi}{\bet} }^2 + \omg^2 }
		= \xo^2 \omg^2 + \sumnqi \abs{\xn}^2 \brac{ \paren{ \frac{2 n \pi}{\bet} }^2 + \omg^2 },
	}
	where we have used the fact that since $x$ is real, Eq.~\refeq{fourier} implies $\xmn = \xn^*$~\cite[p.~285]{Peskin}.  This also means $\xo$ is real.  Referring once more to Eq.~\refeq{ans2a}, we have the integral
	\eq{
		Z = \int \cD \xt \exp\curly{ -\abs{\xo}^2 \omg^2 - \sumnqi \abs{\xn}^2 \brac{ \paren{ \frac{2 n \pi}{\bet} }^2 + \omg^2 } }.
	}
	We can rewrite the integral as~\cite[p.~285]{Peskin}
	\eq{
		\cD \xt = \prodno \ddRexn \ddImxn.
	}
	This yields
	\aln{
		Z &= \int \ddxo e^{-\omg^2 \xo^2} \paren{ \prodno \int \ddRexn \int \ddImxn } \exp\curly{ -\sumnqi \brac{ (\Rexn)^2 + (\Imxn)^2 } \brac{ \paren{ \frac{2 n \pi}{\bet} }^2 + \omg^2 } } \notag \\[2ex]
		%
		&= \int \ddxo e^{-\omg^2 \xo^2} \paren{ \prodno \int \ddRexn \exp\curly{ -\sumnqi (\Rexn)^2 \brac{ \paren{ \frac{2 n \pi}{\bet} }^2 + \omg^2 } } } \notag \\
		&\hspace{10em} \phantom{=\ } \times \paren{ \prodno \int \ddImxn \exp\curly{ -\sumnqi (\Imxn)^2 \brac{ \paren{ \frac{2 n \pi}{\bet} }^2 + \omg^2 } } } \notag \\[2ex]
		%
		&= \frac{\sqrt{2 \pi}}{\omg} \paren{ \prodno \sqrt{ \frac{\pi}{(2 n \pi / \bet)^2 + \omg^2}} } \paren{ \prodno \sqrt{ \frac{\pi}{(2 n \pi / \bet)^2 + \omg^2}} }. \label{thing2b2}
	}
	where we have used~\cite{QFT}
	\eq{
		\intnii e^{-a x^2 / 2} = \sqrt{ \frac{2 \pi}{a} }.
	}
	Picking up from Eq.~\refeq{thing2b2}, we have
	\al{
		Z &= \frac{\sqrt{2 \pi}}{\omg} \prodno \frac{\pi}{(2 n \pi / \bet)^2 + \omg^2} \\
		&= \frac{\sqrt{2 \pi}}{\omg} \prodno \paren{ \frac{\bet}{2 n \pi} }^2 \frac{\pi}{1 + (\omg \bet / 2 n \pi)^2} \\
		&= \frac{\bet \sqrt{2 \pi}}{2} \paren{ \prodno \frac{\bet^2}{4 \pi n^2} } \frac{1}{\omg \bet / 2} \paren{ \prodno \frac{1}{1 + (\omg \bet / 2)^2 / (n \pi)^2} } \\
		&= \ans{ \frac{\bet \sqrt{2 \pi}}{2} \paren{ \prodno \frac{\bet^2}{4 \pi n^2} } \frac{1}{\sinh(\omg \bet / 2)}. }
	}
	We note that $\prodno \bet^2 / 4 \pi n^2$ is a divergent, beta-dependent constant.  Otherwise, this is what we would expect for the quantum partition function of a harmonic oscillator. \qed
}



\prob{
	Generalize this construction to field theory.  Show that the quantum statistical partition function for a free scalar field can be written in terms of a functional integral.  The value of this integral is given formally by
	\eqn{det}{
		[ \det(-\pt^2 + m^2) ]^{-1 / 2},
	}
	where the operator acts on functions on Euclidean space that are periodic in the time direction with periodicity $\bet$.  As before, the $\bet$ dependence of this expression is difficult to compute directly.  However, the dependence on $m^2$ is unambiguous.  Show that the determinant indeed reproduces the partition function for relativistic scalar particles.
}

\sol{
	Peskin \& Schroeder~(9.47) gives an expression for the quantum statistical partition function of a field:
	\eq{
		Z[J] = \int \cD\phi \exp[ -\int \ddqxE (\cLE - J \phi) ].
	}
	To write the Euclidean action for a scalar field, we adapt P\&S~(9.46):
	\eqn{action}{
		\SE = \int \ddqxE \paren{ \frac{1}{2} (\ptEsm \phi)^2 + \frac{1}{2} m^2 \phi^2 }.
	}
	We can introduce the Fourier decomposition by adapting (9.21) and Eq.~\refeq{fourier}:
	\eqn{fourier2}{
		\phix = \frac{1}{V} \sumn e^{-i \ksn \cdot x} \phi(\ksn)
		= \frac{1}{\sqrt{\bet L^3}} \sumn \sumvksn e^{i (2 \pi n t / \bet - \vksn \vdot \vx)} \phivksn
		= \frac{1}{\sqrt{\bet L^3}} \sumn e^{2 \pi i n t / \bet} \sumvksn e^{-i \vksn \vdot \vx} \phivksn,
	}
	where $V = L^4$ is the 4-volume of our space and $k_n^\mu = 2 \pi n^\mu / L$.  Using this expression, note that
	\al{
		(\pt \phi)^2 &= \frac{1}{\bet L^3} \ptsn \paren{ \sumn e^{2 \pi i n t / \bet} \sumvksn e^{-i \vksn \vdot \vx} \phivksn } \ptn \paren{ \summ e^{2 \pi i m t / \bet} \sumvksm e^{-i \vksm \vdot \vx} \phivksm } \\
		&= -\frac{1}{\bet L^3} \sumn \summ \sumvksn \sumvksm \brac{ \paren{ \frac{2 \pi}{\bet} }^2 n m + \vksn \vdot \vksm } e^{2 \pi i (n + m) t / \bet} e^{-i (\vksn + \vksm) \vdot \vx} \phivksn \phivksm.
	}
	Then, substituting Eq.~\refeq{fourier2} into Eq.~\refeq{action} gives us
	\al{
		\SE &= -\frac{1}{2 \bet L^3} \int \ddqx \sumn \summ \sumvksn \sumvksm \brac{ \paren{ \frac{2 \pi}{\bet} }^2 n m + \vksn \vdot \vksm - m^2 } e^{2 \pi i (n + m) t / \bet} e^{-i (\vksn + \vksm) \vdot \vx} \phivksn \phivksm \\
		&= -\frac{1}{2} \sumn \summ \sumvksn \sumvksm \brac{ \paren{ \frac{2 \pi}{\bet} }^2 n m + \vksn \vdot \vksm - m^2 } \delnmm \delkmk \phivksn \phivksm \\
		&= \frac{1}{2} \sumn \sumvksn \brac{ \paren{ \frac{2 \pi n}{\bet} }^2 + \ksn^2 + m^2 } \abs{\phivksn}^2 \\
		&= \sumvk \curly{ (\kso^2 + m^2) \phivkso^2 + \sumnqi \brac{ \paren{ \frac{2 \pi n}{\bet} }^2 + \ksn^2 + m^2 } \abs{\phivksn}^2 },
	}
	where we have once again used Eq.~\refeq{kronecker} and $\phi^*_{\vksn} = \phi_{-\vksn}$~\cite[p.~285]{Peskin}.  Performing the integral as in \ref{2b}, the partition function is
	\aln{
		Z &= \prodvk \int \ddphio e^{-(\kso^2 + m^2) \phivkso^2} \paren{ \prodno \int \ddRephin \exp\curly{ -\sumnqi \sumvksn (\Rephin)^2 \brac{ \paren{ \frac{2 n \pi}{\bet} }^2 + \ksn^2 + m^2 } } } \notag \\
		&\hspace{10em} \phantom{=\ } \times \paren{ \prodno \int \ddImphin \exp\curly{ -\sumnqi \prodvksn (\Imphin)^2 \brac{ \paren{ \frac{2 n \pi}{\bet} }^2 + \ksn^2 + m^2 } } } \notag \\[2ex]
		%
		&= \ans{ \frac{\bet \sqrt{2 \pi}}{2} \paren{ \prodno \frac{\bet^2}{4 \pi n^2} } \paren{ \prodvk \frac{1}{\sinh[ (k^2 + m^2) \bet / 2]} }. } \label{ans2c}
	}
	where we have referred to our steps in \ref{2b}.
	
	For the determinant representation, we use
	\eq{
		\ln(Z) = \ln[ \det(-\ptE^2 + m^2) ]
		= \Tr[ \ln(-\ptE^2 + m^2) ]
		= \int \ddqx \intoi \frac{\ddT}{T} \ev*{e^{-T (-\ptE^2 + m^2)}}{x}.
	}
	Unfortunately, however, I do not have time to complete this part of the problem.
}