\documentclass[11pt]{article}
\usepackage{homework}

\classname{444}
\homeworknum{4}



\begin{document}

% Environments

\newcommand{\state}[2]{\begin{statement}{#1} #2 \end{statement}}
\newcommand{\prob}[2]{\begin{problem}{#1} #2 \end{problem}}
\newcommand{\subprob}[1]{\begin{subproblem} #1 \end{subproblem}}
\newcommand{\sol}[1]{\begin{solution} #1 \end{solution}}
\newcommand{\fig}[2]{\begin{figure} \centering #2  \label{#1} \end{figure}}

\newcommand{\makebib}{
	\vfill
	\color{black}
	\bibliography{references}{}
	\bibliographystyle{lucas_unsrt}
}
	

% Implication

\newcommand{\qwhere}{\quad \text{where} \quad}
\newcommand{\qimplies}{\quad \implies \quad}
\newcommand{\impliesq}{\implies \quad}



% Brackets

\newcommand{\paren}[1]{\left( #1 \right)}
\newcommand{\brac}[1]{\left[ #1 \right]}


% Greek

\newcommand{\alp}{\alpha}
\newcommand{\bet}{\beta}
\newcommand{\gam}{\gamma}
\newcommand{\del}{\delta}
\newcommand{\eps}{\epsilon}
\newcommand{\zet}{\zeta}
\newcommand{\tht}{\theta}
\newcommand{\kap}{\kappa}
\newcommand{\lam}{\lambda}
\newcommand{\sig}{\sigma}
\newcommand{\ups}{\upsilon}
\newcommand{\omg}{\omega}

\newcommand{\Gam}{\Gamma}
\newcommand{\Del}{\Delta}
\newcommand{\Tht}{\Theta}
\newcommand{\Lam}{\Lambda}
\newcommand{\Sig}{\Sigma}
\newcommand{\Omg}{\Omega}
% Problem 1

\newcommand{\Psii}{\Psi^i}
\newcommand{\Psiix}{\Psii(x)}

\newcommand{\Pii}{\Pi^i}

\newcommand{\Phii}{\Phi^i}
\newcommand{\Phiix}{\Phii(x)}
\newcommand{\PhiN}{\Phi^N}
\newcommand{\PhiNx}{\PhiN(x)}
\newcommand{\Phiq}{\Phi^1}
\newcommand{\Phiw}{\Phi^2}

\newcommand{\ddcx}{\dd[3]{x}}

\newcommand{\delij}{\del^{i j}}
\newcommand{\delkl}{\del^{k l}}
\newcommand{\delil}{\del^{i l}}
\newcommand{\deljk}{\del^{j k}}
\newcommand{\delik}{\del^{i k}}
\newcommand{\deljl}{\del^{j l}}

\newcommand{\DF}{D_F}

\newcommand{\sigx}{\sig(x)}

\newcommand{\pii}{\pi^i}
\newcommand{\pij}{\pi^j}
\newcommand{\pik}{\pi^k}
\newcommand{\pil}{\pi^l}
\newcommand{\piix}{\pi(x)}

\newcommand{\pq}{p_1}
\newcommand{\pw}{p_2}
\newcommand{\pe}{p_3}
\newcommand{\pr}{p_4}

\newcommand{\vp}{\vb{p}}
\newcommand{\vpsi}{\vp_i}

\newcommand{\mpi}{m_\pi}

\state{(Jackson 9.8)}{\ 
	%\emph{Hint:} The electromagnetic angular momentum density comes from more than the transverse (radiation zone) components of the fields.
}

%
%	Jackson 9.8(a)
%

\prob{}{
	Show that a classical oscillating electric dipole $\vp$ with fields given by
	\aln{ \label{fields1}
		\vH &= \frac{c k^2}{4\pi} (\nh \cross \vp) \frac{e^{i k r}}{r} \paren{ 1 - \frac{1}{i k r} }, &
		\vE &= \frac{1}{4\pi \epso} \curly{ k^2 (\nh \cross \vp) \cross \nh \frac{e^{i k r}}{r} + [ 3 \nh (\nh \vdot \vp) - \vp ] \paren{ \frac{1}{r^3} - \frac{i k}{r^2} } e^{i k r} },
	}
	radiates electromagnetic angular momentum to infinity at the rate
	\eq{
		\dv{\vL}{t} = \frac{k^3}{12 \pi \epso} \Im[ \vp^* \cross \vp ].
	}
	\vfix
}

\sol{
	According to Jackson~(9.20), the time-averaged angular momentum density is
	\eq{
		\vl = \frac{\Re[ \vx \cross (\vE \cross \vHs)}{2 c^2}.
	}
	One of the vector identities on the inside cover of Jackson is $\vaa \cross (\vbb \cross \vcc) = (\vaa \vdot \vcc) \vbb - (\vaa \vdot \vbb) \vcc$, so
	\eqn{l1}{
		\vl = \frac{(\vx \vdot \vHs) \vE - (\vx \vdot \vE) \vHs}{2 c^2}.
	}
	From Eq.~\refeq{fields1}, note that
	\eq{
		\vx \vdot \vHs \propto \vx \vdot (\nh \cross \vps)
		= \vps \vdot (\vx \cross \nh)
		= \vO,
	}
	where we have used the identity $\vaa \vdot (\vbb \cross \vcc) = \vcc \vdot (\vaa \cross \vbb)$ and the fact that $\nh$ points in the $\vx$ direction.  For $\vx \vdot \vE$, note that
	\al{
		\vx \vdot [ (\nh \cross \vp) \cross \nh ] &= -\vx \vdot [ \nh \cross (\nh \cross \vp) ]
		= -\vx \vdot [ (\nh \vdot \vp) \nh - (\nh \vdot \nh) \vp ]
		= -(\nh \vdot \vp) (\vx \vdot \nh) + \vx \vdot \vp \\
		&= -r (\nh \vdot \vp) + \vx \vdot \vp
		= \vx \vdot \vp - \vx \vdot \vp
		= 0, \\[1.5ex]
		\vx \vdot [ 3 \nh (\nh \vdot \vp) - \vp ] &= 3 (\vx \vdot \nh) (\nh \vdot \vp) - \vx \vdot \vp
		= 3r (\nh \vdot \vp) - \vx \vdot \vp
		= 3(\vx \vdot \vp) - \vx \vdot \vp
		= 2(\vx \vdot \vp),
	}
	since $\abs{\vx} = r$ and $\vx = r \,\nh$.  Then
	\eq{
		\vx \vdot \vE = \frac{1}{2\pi \epso} (\vx \vdot \vp) \paren{ \frac{1}{r^3} - \frac{i k}{r^2} } e^{i k r}
		= \frac{1}{2\pi \epso} (\nh \vdot \vp) \paren{ \frac{1}{r^2} - \frac{i k}{r} } e^{i k r}.
	}
	
	With these substitutions, Eq.~\refeq{l1} becomes
	\al{
		\vl &= -\frac{(\vx \vdot \vE) \vHs}{c^2}
		= -\frac{1}{4\pi \epso c^2} (\nh \vdot \vp) \paren{ \frac{1}{r^2} - \frac{i k}{r} } e^{i k r} \frac{c k^2}{4\pi} (\nh \cross \vps) \frac{e^{-i k r}}{r} \paren{ 1 + \frac{1}{i k r} } \\
		&= -\frac{k^2}{16\pi^2 \epso c r} (\nh \vdot \vp) (\nh \cross \vps) \paren{ \frac{1}{r^2} - \frac{i k}{r} } \paren{ 1 - \frac{i}{k r} }
		= -\frac{k^2}{16\pi^2 \epso c} (\nh \vdot \vp) (\nh \cross \vps) \paren{ \frac{1}{r^2} - \frac{i}{k r^3} - \frac{i k}{r} - \frac{1}{r^2} } \\
		&= -\frac{i k^2}{16\pi^2 \epso c r} (\nh \vdot \vp) (\nh \cross \vps) \paren{ \frac{1}{k r^3} + \frac{k}{r^2} }
		= \frac{i k^3}{16\pi^2 \epso c r^2} (\nh \vdot \vp) (\nh \cross \vps) \paren{ \frac{1}{k^2 r^2} + 1 }.
	}
	
	Let $\vL$ be the angular momentum radiated to a distance $R$.  Then
	\eq{
		\vL = \int_R \vl(r) \ddcx
		= \intopi \intotp \intoR \vl(r) \,r^2 \sin\tht \ddr \ddphi \dd\tht,
	}
	and the time derivative is
	\aln{
		\dv{\vL}{t} &= \dv{t}(\intopi \intotp \intoR \vl(r) \,r^2 \sin\tht \ddr \ddphi \dd\tht)
		= \dv{r}{t} \dv{r}(\intopi \intotp \intoR \vl(r) \,r^2 \sin\tht \ddr \ddphi \dd\tht) \notag \\
		&= c \intopi \intotp \vl(r) \,r^2 \sin\tht \ddphi \dd\tht
		= \frac{i k^3}{16\pi^2 \epso} \paren{ \frac{1}{k^2 r^2} + 1 } \intopi \intotp (\nh \vdot \vp) (\nh \cross \vps) \sin\tht \ddphi \dd\tht. \label{dLdt}
	}
	Note that
	\eq{
		[ (\nh \vdot \vp) (\nh \cross \vps) ]_i = \sumje n_j p_j (\nh \cross \vps)_i
		= \sumje \sumke \sumle \epsikl n_j p_j n_k p_l^*,
	}
	so
	\eq{
		\dv{L_i}{t} \propto \sumje \sumke \sumle \epsikl p_j p_l^* \int n_j p_k \ddOmg
		= \sumje \sumke \sumle \epsikl p_j p_l^* \frac{4\pi}{3} \del_{jk}
		= \frac{4\pi}{3} \epsikl p_k p_l^*
		= \frac{4\pi}{3} (\vp \cross \vps)_i,
	}
	where we have used Jackson~(9.47), $\int n_\bet n_\gam \ddOmg = 4\pi \del_{\bet \gam} / 3$.  Making this substitution into Eq.~\refeq{dLdt},
	\eq{
		\dv{\vL}{t} = \frac{i k^3}{6\pi \epso} \paren{ \frac{1}{k^2 r^2} + 1 } (\vp \cross \vps).
	}
	Taking the limit as $r \to \infty$, we find
	\eqn{ans1a}{
		\dv{\vL}{t} = \Re\!\brac{ \frac{i k^3}{12\pi \epso} (\vp \cross \vps) }
		= \Re\!\brac{ -\frac{i k^3}{12\pi \epso} (\vps \cross \vp) }
		= \ans{ \frac{k^3}{12\pi \epso} \Im[ \vps \cross \vp ], }
	}
	as desired. \qed
}

%
%	Jackson 9.8(b)
%

\prob{}{
	What is the ratio of angular momentum radiated to energy radiated?  Interpret.
}

\sol{
	According to Jackson~(9.24), the total power radiated by an oscillating electric dipole $\vp$ is
	\eq{
		P = \dv{E}{t}
		= \frac{c^2 \Zo k^4}{12 \pi} \abs{\vp}^2.
	}
	Then the ratio of angular momentum radiated to energy radiated is
	\eq{
		\frac{\dv*{\vL}{t}}{\dv*{E}{t}} = \frac{k^3}{12\pi \epso} \Im[ \vps \cross \vp ] \frac{12 \pi}{c^2 \Zo k^4 \abs{\vp}^2}
		= \frac{1}{\epso} \Im[ \vps \cross \vp ] \frac{1}{c^2 \Zo k \abs{\vp}^2}
		= \ans{ \frac{\Im[ \vps \cross \vp ]}{\omg \abs{\vp}^2}, }
	}
	where we have used $\Zo = \sqrt{\muo / \epso} = 1 / \sqrt{\epso^2 c^2} = 1 / \epso c$, $c^2 = 1 / (\epso \muo)$, and $\omg = k c$.
	
	In the limit of high frequency, $(\dv*{\vL}{t}) / (\dv*{E}{t}) \to 0$.  In this scenario, the energy radiated dominates over the angular momentum radiated.  Likewise, in the limit of low frequency, $(\dv*{\vL}{t}) / (\dv*{E}{t}) \to \infty$, meaning that angular momentum radiation dominates.  This is sensible because rotational kinetic energy $E \propto \omg^2$, while angular momentum $L \propto \omg$.
}

%
%	Jackson 9.8(c)
%

\prob{}{
	For a charge $e$ rotating in the $xy$ plane at radius $a$ and angular speed $\omg$, show that there is only a $z$ component of radiated angular momentum with magnitude $\dv*{\Lz}{t} = e^2 k^3 a^2 / 6 \pi \epso$.  What about a charge oscillating along the $z$ axis?
}

\sol{
	We know from Homework~5 that the position of a point charge rotating counterclockwise in the $xy$ plane is
	\eq{
		\vx(t) = a \cos(\omg t) \,\vx + a \sin(\omg t) \,\yh.
	}
	\clearpage
	Then the charge distribution is
	\eq{
		\rho(\vx, t) = e \del[ x - a \cos(\omg t) ] \,\del[ y - a \sin(\omg t) ] \,\del(z).
	}
	
	According to Jackson~(4.8), the dipole moment is defined
	\eq{
		\vp = \int \vx' \,\rho(\vx') \ddcxp.
	}
	The components of $\vp$ for the point charge are then
	\al{
		\px &= e \iiint x \,\del[ x - a \cos(\omg t) ] \,\del[ y - a \sin(\omg t) ] \,\del(z) \ddx \ddy \ddz
		= e a \cos(\omg t), \\
		\py &= e \iiint y \,\del[ x - a \cos(\omg t) ] \,\del[ y - a \sin(\omg t) ] \,\del(z) \ddx \ddy \ddz
		= e a \sin(\omg t), \\
		\pz &= e \iiint z \,\del[ x - a \cos(\omg t) ] \,\del[ y - a \sin(\omg t) ] \,\del(z) \ddx \ddy \ddz
		= 0,
	}
	so we can write $\vp = e a \,e^{-i \omg t} (\xh + i\,\yh).$  Substituting into Eq.~\refeq{ans1a},
	\al{
		\dv{\vL}{t} &= \Re\!\brac{ \frac{i k^3}{12\pi \epso} e^2 a^2 e^{-i \omg t} e^{i \omg t} [ (\xh + i\,\yh) \cross (\xh - i\,\yh) ] }
		= \Re\!\brac{ \frac{i e^2 k^3 a^2}{12\pi \epso} (-2i \,\xh \cross \yh) }
		= \Re\!\brac{ \frac{e^2 k^3 a^2}{6\pi \epso} \,\zh } \\
		&= \ans{ \frac{e^2 k^3 a^2}{6\pi \epso} \cos(\omg t) \,\zh, }
	}
	as desired. \qed
	
	A charge oscillating along the $z$ axis with amplitude $a$ has the charge density
	\eq{
		\rho(\vx, t) = e a \,\del(x) \,\del(y) \,\del[ z - \cos(\omg t) ],
	}
	which gives the dipole moment
	\al{
		\px &= e a \iiint x \,\del(x) \,\del(y) \,\del[ z - \cos(\omg t) ] \ddx \ddy \ddz
		= 0, \\
		\py &= e a \iiint y \,\del(x) \,\del(y) \,\del[ z - \cos(\omg t) ] \ddx \ddy \ddz
		= 0, \\
		\pz &= e a \iiint z \,\del(x) \,\del(y) \,\del[ z - \cos(\omg t) ] \ddx \ddy \ddz
		= e a \cos(\omg t).
	}
	In complex notation, $\vp = e a \,e^{-i\omg t} \,\zh$.  Substituting into Eq.~\refeq{ans1a}, we find
	\eq{
		\dv{\vL}{t} = \Re\!\brac{ \frac{i k^3}{12\pi \epso} e^2 a^2 e^{-i \omg t} e^{i \omg t} (\zh \cross \zh) }
		= \ans{ \vO. }
	}
	So we see that a charge undergoing linear motion does not lead to a radiated angular momentum, which is sensible.
}

%
%	Jackson 9.8(d)
%

\prob{}{
	What are the results corresponding to Probs.~{1(a)} and {1(b)} for magnetic dipole radiation?
}

\sol{
	The radiation fields for a magnetic dipole are given by Jackson~(19.35--36),
	\al{
		\vH &= \frac{1}{4\pi} \curly{ k^2 (\nh \cross \vm) \cross \nh \frac{e^{i k r}}{r} + [ 3 \nh (\nh \vdot \vm) - \vm ] \paren{ \frac{1}{r^3} - \frac{i k}{r^2} } e^{i k r} }, &
		\vE &= -\frac{\Zo}{4\pi} k^2 (\nh \cross \vm) \frac{e^{i k r}}{r} \paren{ 1 - \frac{1}{i k r} }.
	}
	\clearpage
	Comparing with Eq.~\refeq{fields1}, we see that $\vH \to -\vE / \Zo$, $\vE \to \Zo \vH$, and $\vp \to \vm / c$ as stated in the book~\cite[p.~413]{Jackson}.  Making these substitutions, the results of Probs.~{1.1(a)} and {(b)} become
	\al{
		\ans{ \dv{\vL}{t}\ }&\ans{= \frac{\muo k^3}{12\pi} \Im[ \vms \cross \vm ], } &
		\ans{ \frac{\dv*{\vL}{t}}{\dv*{E}{t}}\ }&\ans{= \frac{\Im[ \vms \cross \vm ]}{\omg \abs{\vm}^2} }
	}
	where we have used $\mu = 1 / \epso c^2$.
}

\clearpage
\state{The Gross-Neveu model~(P\&S 11.3)}{
	The Gross-Neveu model is a model in two spacetime dimensions of fermions with a discrete chiral symmetry:
	\eqn{given2}{
		\cL = \psibsi i \ptsl \psisi + \frac{1}{2} g^2 (\psibsi \psisi)^2
	}
	with $i = 1, \ldots, N$.  The kinetic term of two-dimensional fermions is built from matrices $\gamm$ that satisfy the two-dimensional Dirac algebra.  These matrices can be $2 \times 2$:
	\al{
		\gamo &= \sigw, &
		\gamq &= i \sigq,
	}
	where $\sigi$ are Pauli sigma matrices.  Define
	\eq{
		\gamt = \gamo \gamq = \sige;
	}
	this matrix anticommutes with the $\gamm$.
}

\prob{ \label{2a}
	Show that this theory is invariant with respect to
	\eqn{trans}{
		\psisi \to \gamt \psisi,
	}
	and that this symmetry forbids the appearance of a fermion mass.
}

\sol{
	Under this transformation, the Lagrangian Eq.~\refeq{given2} is unchanged:
	\al{
		\cL &= i \psidsi \gamo \gamm \ptsm \psisi + \frac{1}{2} g^2 (\psidsi \gamo \psisi)^2 \\
		&\to i \psidsi \gamt \gamo \gamm \ptsm \gamt \psisi + \frac{1}{2} g^2 (\psidsi \gamt \gamo \gamt \psisi)^2 \\
		&= -i \psidsi \gamt \gamo \gamt \gamm \ptsm \psisi + \frac{1}{2} g^2 (-\psidsi \gamo \psisi)^2 \\
		&= i \psidsi \gamo \gamm \ptsm \psisi + \frac{1}{2} g^2 (\psidsi \gamo \psisi)^2 \\
		&= \psibsi i \ptsl \psisi + \frac{1}{2} g^2 (\psibsi \psisi)^2.
	}
	Here we have used $\ptsl = \gamm \ptsm$~\cite[p.~49]{Peskin}, $\psib = \psid \gamo$ by P\&S~(3.32), and (3.69)--(3.71):
	\al{
		(\gamt)^\dag &= \gamt, &
		(\gamt)^2 &= 1, &
		\{ \gamt, \gamm \} &= 0,
	}
	which also hold for the 2-dimensional Dirac algebra.  Thus we have shown that the theory is invariant with respect to the transformation in Eq.~\refeq{trans} because it does not change the Lagrangian. \qed
	
	If the theory had a mass, the first term would take the form of the Dirac Lagrangian in (3.34):
	\eqn{Dirac}{
		\cLDirac = \psibsi (i \ptsl - m) \psisi.
	}
	However, this term is not invariant under Eq.~\refeq{trans}:
	\al{
		\cLDirac &= i \psidsi \gamo \gamm \ptsm \psisi - m \psidsi \gamo \psisi \\
		&\to i \psidsi \gamt \gamo \gamm \ptsm \gamt \psisi - m \psidsi \gamt \gamo \gamt \psisi \\
		&= i \psidsi \gamo \gamm \ptsm \psisi + m \psidsi \gamo \psisi \\
		&= \psibsi (i \ptsl + m) \psisi.
	}
	Since the sign of the mass term changes under the transformation, a nonzero fermion mass $m$ is forbidden. \qed
}



\prob{
	Show that this theory is renormalizable in 2 dimensions (at the level of dimensional analysis).
}

\sol{
	We need to find the dimension of the coupling constant $g$.  As in \ref{1b}, the Lagrangian must have dimension $d = 2$.  The $\gamm$ are dimensionless, and $\pt$ adds one mass dimension.  We can find the dimension of $\psisi$ by requiring that the dimension of the first term of Eq.~\refeq{given2} is 2.  Let $[ \psisi ] = n$.  Then
	\eq{
		2 = [ \psibsi \ptsl \psisi ]
		= n + 1 + n
		\qimplies
		n = \frac{1}{2}.
	}
	We may now use this result in the second term to find the dimension of $g$.  Let $[ g ] = m$.  Then
	\eq{
		2 = [ g^2 (\psibsi \psisi)^2 ]
		= 2m + 2 (n + n)
		= 2 (m + 1)
		\qimplies
		m = 0,
	}
	meaning $g$ is dimensionless.  Therefore the theory is indeed renormalizable~\cite[p.~322]{Peskin}. \qed
}



\prob{
	Show that the functional integral for this theory can be represented in the following form:
	\eqn{show2c}{
		\int \DDpsi e^{i \int \ddsx \cL} = \int \DDpsi \DDsig \exp[ i \int \ddsx \curly{ \psibsi i \ptsl \psisi - \sig \psibsi \psisi - \frac{1}{2 g^2} \sig^2 } ],
	}
	where $\sigx$ (not to be confused with a Pauli matrix) is a new scalar field with no kinetic energy terms.
}

\sol{
	Completing the square in the exponent of the right-hand side of Eq.~\refeq{show2c} yields
	\eq{
		\int \DDpsi \DDsig \exp[ i \int \ddsx \curly{ -\frac{1}{2 g^2} (\sig + g^2 \psibsi \psisi)^2 + \frac{g^2}{2} (\psibsi \psisi)^2 + \psibsi i \ptsl \psisi } ].
	}
	Pulling out the integral over $\sig$, note that
	\eqn{integral}{
		\int \DDsig \exp[ -i \int \ddsx \frac{1}{2 g^2} (\sig + g^2 \psibsi \psisi)^2 ] \propto \frac{1}{\sqrt{2 \det(g^2)}} = \const.
	}
	This is obtained from P\&S~(9.24),
	\eqn{Gaussian}{
		\paren{ \prodk \int \ddxisk } \exp[ -\xisi \Bsij \xisj ] = \const \times [ \det B ]^{-1/2},
	}
	where in Eq.~\refeq{integral} we obtain an expression like this if we shift our variable of integration.  We obtain
	\al{
		\ans{ \int \DDpsi \DDsig \exp[ i \int \ddsx \curly{ \psibsi i \ptsl \psisi - \sig \psibsi \psisi - \frac{1}{2 g^2} \sig^2 } ] }
		&= \const \times \int \DDpsi \exp[ i \int \ddsx \curly{ \frac{g^2}{2} (\psibsi \psisi)^2 + \psibsi i \ptsl \psisi } ] \\
		&\ans{\; = \int \DDpsi e^{i \int \ddsx \cL}, }
	}
	where $\cL$ is given by Eq.~\refeq{given2}.  The overall constant can be disregarded, so we have proven Eq.~\refeq{show2c}. \qed
}



\prob{ \label{2d}
	Compute the leading correction to the effective potential for $\sig$ by integrating over the fermion fields $\psisi$.  You will encounter the determinant of a Dirac operator; to evaluate this determinant, diagonalize the operator by first going to Fourier components and then diagonalizing the $2 \times 2$ Pauli matrix associated with each Fourier mode.  (Alternatively, you might just take the determinant of this $2 \times 2$ matrix.)   This 1-loop contribution requires a renormalization proportional to $\sigw$ (that is, a renormalization of $g^2$).  Renormalize by minimal subtraction.
}

\sol{
 	The right-hand side of Eq.~\refeq{show2c} can be written
 	\eqn{thing2d}{
 		\int \DDsig \exp[ -i \int \ddsx \frac{1}{2 g^2} \sig^2 ] \int \DDpsi \exp[ i \int \ddsx \psibsi (i \ptsl - \sig) \psisi ].
 	}
 	The integral of the exponential argument is the Dirac Lagrangian Eq.~\refeq{Dirac} with $m \to \sig$.  A similar integral is given by P\&S~(9.76):
 	\eq{
 		\int \DDpsi \DDpsib \exp[ i \int \ddsx \psib (i \Dsl - m) \psi ] = \det(i \Dsl - m).
 	}
 	For the integral over $\psi$ in Eq.~\refeq{thing2d}, we have $\Dsl \to \ptsl$.  The result is then
 	\eq{
 		\int \DDpsi \DDpsib \exp[ i \int \ddsx \psibsi (i \ptsl - \sig) \psisi ] = [ \det(i \ptsl - \sig) ]^N.
 	}
 	Here we get a power of $N$ because we are integrating over $\psibsi, \psisi$ for $i = 1, \ldots, N$.  Now we apply (9.77),
 	\eq{
 		\det B = \exp[ \tr(\log B) ].
 	}
 	Now we follow a similar procedure as in (11.71),
 	\eqn{11.71}{
 		\tr[ \ln(\pt^2 + m^2) ] = (V T) \int \ddqkf \ln(-k^2 + m^2),
 	}
 	where $(V T)$ is the four-dimensional volume of the functional integral.  We can find the eigenvalues of the Dirac operator in two dimensions using a Fourier series~\cite[p.~20]{Peskin}:
 	\eq{
 		\psi(x, t) = \int \ddcpf e^{i p x} \psi(p, t)
 		\qimplies
 		\ptsl \psi(x, t) = \int \ddcpf \ptsl e^{i p x} \psi(p, t)
 		= \int \ddcpf i \psl e^{i p x} \psi(p, t).
 	}
 	This means the eigenvalues are $i \psl$.  Adapting Eq.~\refeq{11.71}, then, we have
 	\eq{
 		\tr[ \ln(i \ptsl - \sig) ] = (V T) \int \ddskf \ln(-\ksl - m)
 		= (V T) \int \ddskf \ln(-k^2 + \sig^2).
 	}
 	Here we have used the result
 	\al{
 		\det(-\gamm \ksm - \sig) &= \det( -\gamo \kso + \gamq \ksq - \sig ) \\
 		&= \det( -\sigw \kso + \sigq \ksq - \sig ) \\
 		&= \det\paren{ -\mqty[ 0 & -i \\ i & 0 ] \kso + \mqty[ 0 & 1 \\ 1 & 0 ] \ksq - \sig } \\
 		&= \det \mqty[ -\sig & i \kso + \ksq \\ -i \kso + \ksq & -\sig ] \\
 		&= \sig^2 - \kso^2 - \ksq^2 \\
 		&= \sig^2 - k^2.
 	}
 	Now we may apply (11.72),
 	\eq{
 		\int \dddkf \ln(-k^2 + m^2) = -i \frac{\Gam(-d / 2)}{(4\pi)^{d / 2}} \frac{1}{(m^2)^{-d / 2}}.
 	}
 	Since $\Gam(-1)$ diverges, we must renormalize.  Letting $d = 2 - \eps$ and expanding about $\eps = 0$~(using Mathematica), we find
 	\eqn{M1}{
 		\int \ddskf \ln(-k^2 + \sig^2) = -i \frac{\Gam(\eps / 2 - 1)}{(4\pi)^{1 - \eps / 2}} \sig^{2 - \eps}
 		\approx \frac{i \sig^2}{4 \pi} \paren{ \frac{2}{\eps} + 1 - \gam + \ln(4\pi) - 2 \ln(\sig) }.
 	}
 	Now we renormalize by minimal subtraction, which involves replacing the $1 / \eps$ poles by some mass parameter $M$ that makes the final equation dimensionally correct~\cite[pp.~376--377]{Peskin}.  This gives us
 	\eqn{M2}{
 		\int \ddskf \ln(-k^2 + \sig^2) \to \frac{i \sig^2}{4 \pi} \brac{ 1 - \gam + \ln(4\pi) - 2 \ln(\frac{\sig}{M}) }.
 	}
  	Applying our work to Eq.~\refeq{show2c}, we have
 	\al{
 		\int \DDpsi e^{i \int \ddsx \cL} &= \int \DDsig \exp[ -i \int \ddsx \frac{1}{2 g^2} \sig^2 ] [ \det(i \ptsl - \sig) ]^N \\
 		&= \int \DDsig \exp[ (VT) \paren{ \frac{-i}{2 g^2} \sig^2 + N \int \ddskf \ln(-k^2 + \sig^2) } ] \\
 		&= \int \DDsig \exp[ -i (VT) \sig^2 \curly{ \frac{1}{2 g^2} - \frac{N}{4\pi} \brac{ 1 - \gam + \ln(4\pi) - 2 \ln(\frac{\sig}{M}) } } ].
 	}
 	The effective potential is defined in P\&S~(11.50),
 	\eq{
 		\Gam[\phicl] = -(V T) \Veff(\phi).
	}
	Here $\phicl$ is a classical field defined by the vacuum expectation value of a quantum field in the presence of a nonzero source $J(x)$~\cite[p.~348]{Peskin}:
	\eqn{vev}{
		\phicl(x) = \ev{\phi(x)}{\Omg}_J
		\equiv v.
	}
	$\Gam[\phicl]$ is given by (11.63),
 	\eq{
 		\Gam[\phicl] = \int \ddqx \cLq[\phicl] + \frac{i}{2} \log \det\paren{ -\fdv[2]{\cLq}{\phi}{\phi} } - i \cdot (\text{connected diagrams}) + \int \ddqx \del\cL[\phicl].
 	}
 	We need only concern ourselves with the first two terms since the connected diagrams have at least two loops, and the last term contains the counterterms that are used for renormalization~\cite[p.~372]{Peskin}.  So the leading correction to the effective potential is
 	\eqn{Veff}{
 		\ans{ \Veff = \sig^2 \curly{ \frac{1}{2 g^2} + \frac{N}{4\pi} \brac{ \ln(\frac{\sig^2}{M^2}) + \gam - \ln(4\pi) - 1 } }, }
 	}
 	where we have referred to (11.79).
}



\prob{ \label{2e}
	Ignoring two-loop and higher-order contributions, minimize this potential.  Show that the $\sig$ field acquires a vacuum expectation value which breaks the symmetry of \ref{2a}.  Convince yourself that this result does not depend on the particular renormalization condition chosen.
}

\sol{
	We minimize the potential by imposing (11.51),
	\eq{
		\pdv{\phicl} \Veff(\phicl) = 0,
	}
	where minimizing $\phicl$ is the vacuum expectation value, $v$, by Eq.~\refeq{vev}.
	
	For the problem at hand, we want to find $\sig$ to minimize the potential Eq.~\refeq{Veff}.  We impose
	\eq{
		0 = \pdv{\Veff}{\sig}
		= \sig \curly{ \frac{1}{g^2} + \frac{N}{2 \pi} \brac{ \gamma - \ln(4\pi) + \ln(\frac{\sig^2}{M^2}) } },
	}
	where we have evaluated the derivative using Mathematica.  Solving for $\sig$, we find the vacuum expectation value:
	\eqn{thing2e}{
		\ln(4\pi) - \gam - \frac{2\pi}{N g^2} = \ln(\frac{\sig^2}{M^2})
		\qimplies
		\ans{ \sig = \pm 2 \sqrt{\pi} M \exp( -\frac{\gam}{2} - \frac{\pi}{N g^2} ) = \pm v. }
	}
	This vacuum expectation value breaks the symmetry of \ref{2a} because it acts as a nonzero mass for the fields $\psisi$.  As we saw in \ref{2d}, the form of the Lagrangian with $\sig$ looks like the Dirac Lagrangian with mass $\sig = \pm v$, and as we showed in \ref{2a}, the theory is not invariant under Eq.~\refeq{trans} when such a mass is included.  Therefore, this vacuum expectation value breaks the symmetry. \qed
	
	This result does not depend on the particular renormalization condition chosen because any condition we choose would give us a nonzero vacuum expectation value, which is all we need for the symmetry to be broken.  Say we had chosen instead the modified minimal subtraction scheme, with Eq.~\refeq{M2} replaced by
 	\eqn{M2}{
 		\int \ddskf \ln(-k^2 + \sig^2) \to \frac{i \sig^2}{4 \pi} \brac{ 1 - \ln(\frac{\sig^2}{{M'}^2}) },
 	}
 	where $M'$ replaces the divergence and the terms involving $\gam$ and $\ln(4\pi)$, as in the examples in (11.77) and (11.78).  Then minimizing the potential gives us
 	\eq{
 		0 = \frac{1}{g^2} + \frac{N}{2\pi} \ln(\frac{\sig^2}{{M'}^2})
 		\qimplies
 		\sig = \pm M' e^{-\pi / N g^2} = \pm v.
 	}
 	Comparing this to Eq.~\refeq{thing2e}, all we have done is absorb part of the exponential into $M'$; that is, $M' = 2 \sqrt{\pi} M e^{-\gam / 2}$.  We see that no matter how we handle the divergence, it is impossible for the vacuum expectation value to be zero.  So the symmetry will be broken regardless of which normalization condition we choose.
	
%	For the problem at hand, we want to find $\sig^2$ to minimize the potential Eq.~\refeq{Veff}.  We impose
%	\al{
%		0 &= \pdv{\Veff}{\sig^2} \\
%		&= \frac{1}{2 g^2} + \frac{N}{4\pi} \brac{ \ln(\frac{\sig^2}{M^2}) - 1 } + \sig^2 \pdv{\sig^2}\curly{ \frac{1}{2 g^2} + \frac{N}{4\pi} \brac{ \ln(\frac{\sig^2}{M^2}) - 1 } } \\
%		&= \frac{1}{2 g^2} + \frac{N}{4\pi} \brac{ \ln(\frac{\sig^2}{M^2}) - 1 } + \frac{N}{4\pi} \\
%		&= \frac{1}{2 g^2} + \frac{N}{4\pi} \ln(\frac{\sig^2}{M^2}).
%	}
%	Solving for $\sig$, we find the vacuum expectation value:
%	\eq{
%		-\frac{2\pi}{N g^2} = \ln(\frac{\sig^2}{M^2})
%		\qimplies
%		\ans{ \sig = \pm M e^{-\pi / N g^2} = \pm v. }
%	}
%	This vacuum expectation value breaks the symmetry of \ref{2a} because it acts as a nonzero mass for the fields $\psisi$.  As we saw in \ref{2d}, the form of the Lagrangian with $\sig$ looks like the Dirac Lagrangian with mass $\sig = \pm v$, and as we showed in \ref{2a}, the theory is not invariant under Eq.~\refeq{trans} when such a mass is included.  Therefore, this vacuum expectation value breaks the symmetry. \qed
%	
%	This result does not depend on the particular renormalization condition chosen because all we have done is absorb the divergence into some parameter $M$, which parameterizes a sequence of possible renormalization conditions~\cite[p.~377]{Peskin}.  From Eqs.~\refeq{M1} and \refeq{M2},
%	\eq{
%		\frac{2}{\eps} + 1 - \gam + \ln(4\pi) - 2 \ln(\sig) =  1 - \ln(\sig^2) + \ln(M^2)
%%		\qimplies
%%		2 \ln(M) = \frac{2}{\eps} - \gam + \ln(4\pi)
%		\qimplies M = \exp[ \frac{1}{\eps} - \frac{\gam}{2} + \frac{\ln(4\pi)}{2} ].
%	}
%	No matter how we handle the divergent term, it is impossible for $M$ to be zero.  So the symmetry will be broken regardless of which normalization condition we choose.
}



\prob{
	Note that the effective potential derived in \ref{2e} depends on $g$ and $N$ according to the form
	\eq{
		\Veff(\sigcl) = N \cdot f(g^2 N).
	}
	(The overall factor of $N$ is expected in a theory with $N$ fields.)  Construct a few of the higher-order contributions to the effective potential and show that they contain additional factors of $N^{-1}$ which suppress them if we take the limit $N \to \infty$, ($g^2 N$) fixed.  In this limit, the result of \ref{2e} is unambiguous.
}


\makebib

\end{document}
