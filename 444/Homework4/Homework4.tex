\documentclass[11pt]{article}
\usepackage{homework}

\classname{444}
\homeworknum{4}



\begin{document}

% Environments

\newcommand{\state}[2]{\begin{statement}{#1} #2 \end{statement}}
\newcommand{\prob}[2]{\begin{problem}{#1} #2 \end{problem}}
\newcommand{\subprob}[1]{\begin{subproblem} #1 \end{subproblem}}
\newcommand{\sol}[1]{\begin{solution} #1 \end{solution}}
\newcommand{\fig}[2]{\begin{figure} \centering #2  \label{#1} \end{figure}}

\newcommand{\makebib}{
	\vfill
	\color{black}
	\bibliography{references}{}
	\bibliographystyle{lucas_unsrt}
}
	

% Implication

\newcommand{\qwhere}{\quad \text{where} \quad}
\newcommand{\qimplies}{\quad \implies \quad}
\newcommand{\impliesq}{\implies \quad}



% Brackets

\newcommand{\paren}[1]{\left( #1 \right)}
\newcommand{\brac}[1]{\left[ #1 \right]}


% Greek

\newcommand{\alp}{\alpha}
\newcommand{\bet}{\beta}
\newcommand{\gam}{\gamma}
\newcommand{\del}{\delta}
\newcommand{\eps}{\epsilon}
\newcommand{\zet}{\zeta}
\newcommand{\tht}{\theta}
\newcommand{\kap}{\kappa}
\newcommand{\lam}{\lambda}
\newcommand{\sig}{\sigma}
\newcommand{\ups}{\upsilon}
\newcommand{\omg}{\omega}

\newcommand{\Gam}{\Gamma}
\newcommand{\Del}{\Delta}
\newcommand{\Tht}{\Theta}
\newcommand{\Lam}{\Lambda}
\newcommand{\Sig}{\Sigma}
\newcommand{\Omg}{\Omega}
% Problem 1

\newcommand{\Psii}{\Psi^i}
\newcommand{\Psiix}{\Psii(x)}

\newcommand{\Pii}{\Pi^i}

\newcommand{\Phii}{\Phi^i}
\newcommand{\Phiix}{\Phii(x)}
\newcommand{\PhiN}{\Phi^N}
\newcommand{\PhiNx}{\PhiN(x)}
\newcommand{\Phiq}{\Phi^1}
\newcommand{\Phiw}{\Phi^2}

\newcommand{\ddcx}{\dd[3]{x}}

\newcommand{\delij}{\del^{i j}}
\newcommand{\delkl}{\del^{k l}}
\newcommand{\delil}{\del^{i l}}
\newcommand{\deljk}{\del^{j k}}
\newcommand{\delik}{\del^{i k}}
\newcommand{\deljl}{\del^{j l}}

\newcommand{\DF}{D_F}

\newcommand{\sigx}{\sig(x)}

\newcommand{\pii}{\pi^i}
\newcommand{\pij}{\pi^j}
\newcommand{\pik}{\pi^k}
\newcommand{\pil}{\pi^l}
\newcommand{\piix}{\pi(x)}

\newcommand{\pq}{p_1}
\newcommand{\pw}{p_2}
\newcommand{\pe}{p_3}
\newcommand{\pr}{p_4}

\newcommand{\vp}{\vb{p}}
\newcommand{\vpsi}{\vp_i}

\newcommand{\mpi}{m_\pi}



\state{Spin-wave theory~(P\&S 11.1)}{\hfix}

\prob{ \label{1a}
	Prove the following wonderful formula: Let $\phix$ be a free scalar field with propagator $\ev{T \phix \phio} = \Dx$.  Then
	\eqn{show1}{
		\ev{ T e^{i \phix} e^{-i \phio} } = e^{[ \Dx - \Do ]}.
	}
	(The  factor $\Do$ gives a formally divergent adjustment of the overall normalization.)
}

\sol{
	According to P\&S~(9.18),
	\eq{
		\ev*{T \phi(\xq) \phi(\xw)}{\Omg} = \frac{\int \DDphi \phi(\xq) \phi(\xw) \exp[ i \int \ddqx \cL ]}{\int \DDphi \exp[ i \int \ddqx \cL ]}.
	}
	We use this expression to write the left-hand side of Eq.~\refeq{show1}:
	\eqn{thing1}{
		\ev{ T e^{i \phix} e^{-i \phio} } = \frac{\int \DDphi e^{i \phix} e^{-i \phio} \exp[ i \int \ddqy \cL ]}{\int \DDphi \exp[ i \int \ddqy \cL ]}
		= \frac{\int \DDphi \exp[i \phix - i \phio + i \int \ddqy \cL ]}{\int \DDphi \exp[ i \int \ddqy \cL ]}.
	}
	For a free Klein-Gordon~(i.e., scalar) field, Eq.~(9.39) tells us that the generating functional $\ZJ$ is given by
	\eq{
		\ZJ = \Zo \exp[ -\frac{1}{2} \int \ddqx \ddqy \Jx \DF(x - y) \Jy ],
	}
	where $\Zo = Z[0]$.  Thus, we want to find some $\Jy$ such that
	\eqn{thing1b}{
		\ev{ T e^{i \phix} e^{-i \phio} } = \frac{\ZJ}{\Zo}
	}
	where in general
	\eq{
		\ZJ = \int \DDphi \exp[ i \int \ddqx [ \cL + \Jx \phi(x) ] ]
	}
	by (9.34).  Inspecting Eq.~\refeq{thing1}, we recognize the denominator as $\Zo$ and see that if
	\eq{
		\Jy = \delq(y - x) - \delq(y)
	}
	we have an expression like Eq.~\refeq{thing1b}.  Collecting these findings, we have
	\al{
		\ans{ \ev{ T e^{i \phix} e^{-i \phio} } }&= \frac{\ZJ}{\Zo} \\
		&= \exp[ -\frac{1}{2} \int \ddqy \ddqz \Jy \DF(y - z) \Jz ] \\
		&= \exp[ -\frac{1}{2} \int \ddqy \ddqz \Jy \DF(y - z) [ \delq(z - x) - \delq(z) ] ] \\
		&= \exp[ -\frac{1}{2} \int \ddqy [ \delq(y - x) - \delq(y) ] [ \DF(y - x) - \DF(y) ] ] \\
		&= \exp[ -\frac{1}{2} [ \DF(0) - \DF(x) - \DF(-x) + \DF(0) ] ] \\
		&= \exp[ \DF(x) - \DF(0) ] \\
		&\ans{\; = e^{[ \Dx - \Do ]}, }
	}
	as we wanted to show. \qed
}



\prob{ \label{1b}
	We can use this formula in Euclidean field theory to discuss correlation functions in a theory with spontaneously broken symmetry for $T < \TC$.  Let us consider only the simplest case of a broken $O(2)$ or $U(1)$ symmetry.  We can write the local spin density as a complex variable
	\eq{
		\sx = \sqx + i \swx.
	}
	The global symmetry is the transformation
	\eq{
		\sx \to e^{-i \alp} \sx.
	}
	If we assume that the physics freezes the modulus of $\sx$, we can parameterize
	\eqn{sx}{
		\sx = A e^{i \phix}
	}
	and write an effective Lagrangian for the field $\phix$.  The symmetry of the theory becomes the translation symmetry
	\eqn{symmetry}{
		\phix \to \phix - \alp.
	}
	Show that (for $d > 0$) the most general renormalizable Lagrangian consistent with this symmetry is the free field theory
	\eqn{show1b}{
		\cL = \frac{1}{2} \rho(\vgrad \phi)^2.
	}
	In statistical mechanics, the constant $\rho$ is called the \emph{spin wave modulus}.  A reasonable hypothesis for $\rho$ is that it is finite for $T < \TC$ and tends to 0 as $T \to \TC$ from below.
}

\sol{
	In accordance with the Klein-Gordon Lagrangian in P\&S~(2.6),
	\eqn{KGL}{
		\cL_\text{K-G} = \frac{1}{2} (\pt \phi)^2 - \frac{1}{2} m^2 \phi^2,
	}
	we interpret $(\vgrad \phi)^2$ as $(\pt \phi)^2$.
	
	The Lagrangian cannot have terms of $\order{\phi^n}$ for any $n \neq 0$ since $\phi(x)$ is not invariant under Eq.~\refeq{symmetry}.  Any combination of derivatives of $\phi$ is invariant, however, since $\alp$ is a constant and does not contribute to any derivative.  Thus, only terms like $\pt^n \phi^m$ (where $n$ denotes a power of $\pt$) for $n, m > 0$ and $n \geq m$ are consistent with the symmetry of Eq.~\refeq{symmetry} for $d$ an integer.
	
	Now we must determine which of these terms are renormalizable.  We know that the Lagrangian must have dimension $d$, and that $\phi$ has dimension $(d - 2) / 2$.  Taking a derivative adds a mass dimension.  The theory is renormalizable if the coupling constant $\rho$ has dimension greater than or equal to 0~\cite[p.~322]{Peskin}.  Let $p$ be the dimension of $\rho$.  The dimension of our allowed term is then
	\eq{
		[ \rho \pt^n \phi^m ] = p + n + m \frac{d - 2}{2},
	}
	which we require to be equal to $d$.  Thus we seek solutions to the system of equations
	\al{
		d &= p + n + m \frac{d - 2}{2}, &
		n &\geq m, &
		p &\geq 0.
	}
	Solving with Mathematica, we find that this system has two solutions: $n = m = 2$ and $p = 0$; and $n = m = 1$ and $p = d / 2$.  However, the term $\pt \phi$ for $n = m = 1$ does not contribute to the action because it is a total derivative and does not contribute when the integral over $\cL$ is evaluated:
	\eq{
		\int \dd[d]{x} \pt\phi = \phi \bigg|_{-\infty}^\infty
		= 0.
	}
	Thus the only possibility is $n = m = 2$.  Note that
	\eq{
		\pt^2 \phi^2 = \pt(\pt \phi^2)
		= 2 \pt( \phi \pt \phi)
		= \pt \phi \pt \phi + \phi \pt^2 \phi
		= (\pt \phi)^2,
	}
	since $\phi \pt^2 \phi$ is not invariant under Eq.~\refeq{sx}.  This means that $\rho$ must be dimensionless and that the only allowed terms in the Lagrangian are proportional to $(\pt \phi)^2$, which is consistent with Eq.~\refeq{show1b}. \qed
}



\prob{
	Compute the correlation function $\ev{ \sx \sao }$.  Adjust $A$ to give a physically sensible normalization (assuming that the system has a physical cutoff at the scale of one atomic spacing) and display the dependence of this correlation function on $x$ for $d = 1, 2, 3, 4$.  Explain the significance of your results.
}

\sol{
	Applying Eq.~\refeq{sx},
	\eq{
		\ev{ \sx \sao } = \ev*{ A e^{i \phix} \As e^{-i \phio} }
		= \ev*{ \abs{A}^2 } \ev*{ e^{i \phix} e^{-i \phio} }.
	}
	Now we can apply Eq.~\refeq{show1} to find
	\eqn{thing1c}{
		\ans{ \ev{ \sx \sao } = \abs{A}^2 \exp[ D(x) - D(0) ], }
	}
	where $D(x - y)$ is a Green's function.  Since our Lagrangian is similar to the Klein-Gordon Lagrangian Eq.~\refeq{2.6}, our Green's function is similar to that of the Klein-Gordon operator, which is given by P\&S~(2.56):
	\eq{
		(\pt^2 + m^2) D(x - y) = -i \delq(x - y).
	}
	The Feynman prescription for this Green's function is given by (2.59),
	\eqn{DF}{
		\DF(x - y) = \int \ddqpf \frac{i}{p^2 - m^2 + i \eps} e^{-i p \cdot (x - y)}.
	}
	For the Lagrangian in Eq.~\refeq{show1b}, we set $m = 0$ and insert a factor of $\rho$:
	\eq{
		\rho \pt^2 D(x - y) = -i \deld(x - y),
	}
	so adapting Eq.~\refeq{DF} for this situation yields
	\eqn{DF}{
		\DF(x - y) = \frac{1}{\rho} \int \dddpf \frac{i}{p^2 + i \eps} e^{-i p \cdot (x - y)}.
	}
	We see that $\DF(0)$ diverges, so we absorb it into the constant to make the normalization physically sensible.  We can do this because, as we showed in \ref{1b}, the theory is renormalizable.  Define $A'$ such that
	\eq{
		{A'}^2 = \abs{A}^2 e^{-D(0)}.
	}
	Then Eq.~\refeq{thing1c} can be written
	\eq{
		\ans{ \ev{ \sx \sao } =  {A'}^2 e^{D(x)}. }
	}
	
	To evaluate the divergent integral $D(x)$, we look to the Feynman parameter method we have been using to solve divergent integrals.  Apparently, the Schwinger parametrization is useful in deriving the Feynman parametrization, and it is given by~\cite{Feynman}
	\eq{
		\frac{1}{A} = \intoi \dds e^{-s A}.
	}
	Using this equation, we can write Eq.~\refeq{DF} as
	\eq{
		\DF(x) = \frac{1}{\rho} \int \dddpf \frac{i}{p^2} e^{-i p \cdot x}
		= \frac{i}{\rho} \int \dddpf \intoi \dds e^{-s p^2} e^{-i p \cdot x}.
	}
	Now we can complete the square in the exponential to get a Gaussian integral:
	\al{
		\DF(x) &= \frac{i}{\rho} \int \dddpf \intoi \dds \exp[ -s p^2 - i p \cdot x + \frac{x^2}{4 s} - \frac{x^2}{4 s} ] \\
		&= \frac{i}{\rho} \int \dddpf \intoi \dds \exp[ -s \paren{ p + \frac{i x}{2 s} }^2 - \frac{x^2}{4 s} ] \\
		&= \frac{i}{\rho (2 \pi)^d} \intoi \dds e^{-x^2 / 4 s} \int \dd[d]{u} e^{-s u^2} \\
		&= \frac{i}{\rho (2 \pi)^{d}} \intoi \dds e^{-x^2 / 4 s} \sqrt{ \frac{(2\pi)^d}{(2s)^d} } \\
		&= \frac{i}{\rho (4 \pi)^{d / 2}} \intoi \dds \frac{e^{-x^2 / 4 s}}{s^{d / 2}}
	}
	where we have used~\cite{QFT}
	\eq{
		\int \exp( -\frac{1}{2} x \cdot A \cdot x ) \dd[n]{x} = \sqrt{\frac{(2\pi)^n}{\det A}},
	}
	with $A$ a $d \times d$ diagonal matrix $2s$.  Using Mathematica to integrate with respect to $s$, we find
	\eq{
		\DF(x) = \frac{i}{\rho (4 \pi)^{d / 2}} \frac{2^{d - 2}}{x^{d - 2}} \Gam(d / 2 - 1)
		= \frac{i}{4 \pi^d \rho} \Gam(d / 2 - 1) x^{2 - d}.
	}
	The gamma function diverges as $d \to 2$, so as we have done in previous problems, we expand about $\eps = 2 - d$.  Evaluating the series expansion using Mathematica, we obtain
	\eq{
		\DF(x) = \frac{i}{4 \pi^{1 - \eps} \rho} \Gam(\eps / 2) x^\eps
		\approx \frac{i}{4 \pi \rho} \paren{ \frac{2}{\eps} - \gam + 2 \ln(\pi x) }
		\sim \frac{i}{2 \pi \rho} \ln(x)
		= i \ln(\frac{1}{x^{2 \pi \rho}}).
	}
	We Wick rotate $x \to i x$.  Then the dependence of the correlation function on $x$ for $d = 1, 2, 3, 4$ is
	\ans{\al{
		(d = 1) &\qquad \ev{ \sx \sao } \sim e^{-x}, &
		(d = 2) &\qquad \ev{ \sx \sao } \sim x^{2 \pi \rho}, \\
		(d = 3) &\qquad \ev{ \sx \sao } \sim \frac{1}{x}, &
		(d = 4) &\qquad \ev{ \sx \sao } \sim \frac{1}{x^2}.
	}}%
	\hl{how on earth to explain these??}
%
%
%	We begin with the case $d = 1$.  We Wick rotate $p \to i p$, and so we need to evaluate
%	\eq{
%		\intnii \ddpf \paren{-i \frac{e^{i p x}}{p^2} \equiv \intnii \ddp \fp },
%	}
%	where we have defined $\fp$.  We choose a contour $\sC$ in the positive half of the imaginary plane that encloses $p = 0$.  The residue at the second-order pole $p = 0$ can be evaluated using
%	\eq{
%		\Res_{z = c} f(z) = \frac{1}{(n - 1)!} \lim_{z \to c} \dv[n-1]{z}[ (z - c)^n f(z) ],
%	}
%	where $f(z)$ is the integrand and $n$ is the order of the pole at $z = c$~\cite{Residue}.  We have
%	\eq{
%		\Res_{p = 0} f(p) = \lim_{p \to 0} \dv{p}[ -i p^2 \frac{e^{i p x}}{2 \pi p^2} ]
%		= -i \lim_{k \to 0} \dv{e^{i p x}}{2 \pi k}
%		= -i \lim_{k \to 0} i x \frac{e^{i p x}}{2\pi}
%		= \frac{x}{2\pi}.
%	}
%	Then, by the residue theorem~\cite{ResidueTheorem},
%	\eq{
%		\intnii \ddpf \fp = 2 \pi i \Res_{k = 0}
%		= i x.
%	}
%	So the correlation function for $d = 1$ is
%	\eq{
%		\ans{ \ev{ \sx \sao } =  {A'}^2 e^{-x / \rho}. }
%	}
%	
%	For the case $d = 2$, we have
%	\eq{
%		\intnii \ddspf \paren{ -i \frac{e^{i p \cdot x}}{p^2} } \equiv \intnii \ddsp \fp.
%	}
%	To rewrite the dot product, we transform to polar coordinates:
%	\eq{
%		\intnii \ddsp \fp = -\frac{i}{(2\pi)^2} \intoi \ddp \intotp \ddtht \frac{e^{i p x \cos\tht}}{p}
%		= \frac{1}{2\pi i} \intoi \ddp \frac{I_0(i p x)}{p}
%	}
%	where we have evaluated the integral using Mathematica, and $I_n(x)$ is the modified Bessel function of the first kind.  Now we have a simple pole at $p = 0$, and we can find the residue using~\cite{Residue}
%	\eq{
%		\Res_{z = c} f(z) = \lim_{z \to c} (z - c) f(z).
%	}
%	For our problem,
%	\eq{
%		\Res_{p = 0} = \lim_{p \to 0} \frac{I_0(i p x)}{2\pi i}
%		= \frac{1}{2 \pi i}
%	}
%
%	 in order to write the dot product as $p x \cos\tht$; however, it does not seem like this will work.  We would need to introduce a factor of $\sin\tht$ in order to integrate with respect to $\tht$, but in two dimensions this factor would not be canceled by the Jacobian as it would be in three dimensions.
%		
%	For the case $d = 3$, we have
%	\eq{
%		\intnii \ddcpf \paren{ -i \frac{e^{i p \cdot x}}{p^2} } \equiv \intnii \ddsp \fp
%	}
%
%@misc{			Residue,
%	author		= "{Wikipedia contributors}",
%	title			= "Residue (complex analysis)",
%	howpublished= "From Wikipedia, the Free Encyclopedia",
%	url			= "https://en.wikipedia.org/wiki/Residue_(complex_analysis)"
%}
%
%@misc{			ResidueTheorem,
%	author		= "{Wikipedia contributors}",
%	title			= "Residue theorem",
%	howpublished= "From Wikipedia, the Free Encyclopedia",
%	url			= "https://en.wikipedia.org/wiki/Residue_theorem"
%}
}






\clearpage
\state{The Gross-Neveu model~(P\&S 11.3)}{
	The Gross-Neveu model is a model in two spacetime dimensions of fermions with a discrete chiral symmetry:
	\eqn{given2}{
		\cL = \psibsi i \ptsl \psisi + \frac{1}{2} g^2 (\psibsi \psisi)^2
	}
	with $i = 1, \ldots, N$.  The kinetic term of two-dimensional fermions is built from matrices $\gamm$ that satisfy the two-dimensional Dirac algebra.  These matrices can be $2 \times 2$:
	\al{
		\gamo &= \sigw, &
		\gamq &= i \sigq,
	}
	where $\sigi$ are Pauli sigma matrices.  Define
	\eq{
		\gamt = \gamo \gamq = \sige;
	}
	this matrix anticommutes with the $\gamm$.
}

\prob{ \label{2a}
	Show that this theory is invariant with respect to
	\eqn{trans}{
		\psisi \to \gamt \psisi,
	}
	and that this symmetry forbids the appearance of a fermion mass.
}

\sol{
	Under this transformation, the Lagrangian Eq.~\refeq{given2} is unchanged:
	\al{
		\cL &= i \psidsi \gamo \gamm \ptsm \psisi + \frac{1}{2} g^2 (\psidsi \gamo \psisi)^2 \\
		&\to i \psidsi \gamt \gamo \gamm \ptsm \gamt \psisi + \frac{1}{2} g^2 (\psidsi \gamt \gamo \gamt \psisi)^2 \\
		&= -i \psidsi \gamt \gamo \gamt \gamm \ptsm \psisi + \frac{1}{2} g^2 (-\psidsi \gamo \psisi)^2 \\
		&= i \psidsi \gamo \gamm \ptsm \psisi + \frac{1}{2} g^2 (\psidsi \gamo \psisi)^2 \\
		&= \psibsi i \ptsl \psisi + \frac{1}{2} g^2 (\psibsi \psisi)^2.
	}
	Here we have used $\ptsl = \gamm \ptsm$~\cite[p.~49]{Peskin}, $\psib = \psid \gamo$ by P\&S~(3.32), and (3.69)--(3.71):
	\al{
		(\gamt)^\dag &= \gamt, &
		(\gamt)^2 &= 1, &
		\{ \gamt, \gamm \} &= 0,
	}
	which also hold for the 2-dimensional Dirac algebra.  Thus we have shown that the theory is invariant with respect to the transformation in Eq.~\refeq{trans} because it does not change the Lagrangian. \qed
	
	If the theory had a mass, the first term would take the form of the Dirac Lagrangian in (3.34):
	\eqn{Dirac}{
		\cLDirac = \psibsi (i \ptsl - m) \psisi.
	}
	However, this term is not invariant under Eq.~\refeq{trans}:
	\al{
		\cLDirac &= i \psidsi \gamo \gamm \ptsm \psisi - m \psidsi \gamo \psisi \\
		&\to i \psidsi \gamt \gamo \gamm \ptsm \gamt \psisi - m \psidsi \gamt \gamo \gamt \psisi \\
		&= i \psidsi \gamo \gamm \ptsm \psisi + m \psidsi \gamo \psisi \\
		&= \psibsi (i \ptsl + m) \psisi.
	}
	Since the sign of the mass term changes under the transformation, a nonzero fermion mass $m$ is forbidden. \qed
}



\prob{
	Show that this theory is renormalizable in 2 dimensions (at the level of dimensional analysis).
}

\sol{
	We need to find the dimension of the coupling constant $g$.  As in \ref{1b}, the Lagrangian must have dimension $d = 2$.  The $\gamm$ are dimensionless, and $\pt$ adds one mass dimension.  We can find the dimension of $\psisi$ by requiring that the dimension of the first term of Eq.~\refeq{given2} is 2.  Let $[ \psisi ] = n$.  Then
	\eq{
		2 = [ \psibsi \ptsl \psisi ]
		= n + 1 + n
		\qimplies
		n = \frac{1}{2}.
	}
	We may now use this result in the second term to find the dimension of $g$.  Let $[ g ] = m$.  Then
	\eq{
		2 = [ g^2 (\psibsi \psisi)^2 ]
		= 2m + 2 (n + n)
		= 2 (m + 1)
		\qimplies
		m = 0,
	}
	meaning $g$ is dimensionless.  Therefore the theory is indeed renormalizable~\cite[p.~322]{Peskin}. \qed
}



\prob{
	Show that the functional integral for this theory can be represented in the following form:
	\eqn{show2c}{
		\int \DDpsi e^{i \int \ddsx \cL} = \int \DDpsi \DDsig \exp[ i \int \ddsx \curly{ \psibsi i \ptsl \psisi - \sig \psibsi \psisi - \frac{1}{2 g^2} \sig^2 } ],
	}
	where $\sigx$ (not to be confused with a Pauli matrix) is a new scalar field with no kinetic energy terms.
}

\sol{
	Completing the square in the exponent of the right-hand side of Eq.~\refeq{show2c} yields
	\eq{
		\int \DDpsi \DDsig \exp[ i \int \ddsx \curly{ -\frac{1}{2 g^2} (\sig + g^2 \psibsi \psisi)^2 + \frac{g^2}{2} (\psibsi \psisi)^2 + \psibsi i \ptsl \psisi } ].
	}
	Pulling out the integral over $\sig$, note that
	\eqn{integral}{
		\int \DDsig \exp[ -i \int \ddsx \frac{1}{2 g^2} (\sig + g^2 \psibsi \psisi)^2 ] \propto \frac{1}{\sqrt{2 \det(g^2)}} = \const.
	}
	This is obtained from P\&S~(9.24),
	\eqn{Gaussian}{
		\paren{ \prodk \int \ddxisk } \exp[ -\xisi \Bsij \xisj ] = \const \times [ \det B ]^{-1/2},
	}
	where in Eq.~\refeq{integral} we obtain an expression like this if we shift our variable of integration.  We obtain
	\al{
		\ans{ \int \DDpsi \DDsig \exp[ i \int \ddsx \curly{ \psibsi i \ptsl \psisi - \sig \psibsi \psisi - \frac{1}{2 g^2} \sig^2 } ] }
		&= \const \times \int \DDpsi \exp[ i \int \ddsx \curly{ \frac{g^2}{2} (\psibsi \psisi)^2 + \psibsi i \ptsl \psisi } ] \\
		&\ans{\; = \int \DDpsi e^{i \int \ddsx \cL}, }
	}
	where $\cL$ is given by Eq.~\refeq{given2}.  The overall constant can be disregarded, so we have proven Eq.~\refeq{show2c}. \qed
}



\prob{ \label{2d}
	Compute the leading correction to the effective potential for $\sig$ by integrating over the fermion fields $\psisi$.  You will encounter the determinant of a Dirac operator; to evaluate this determinant, diagonalize the operator by first going to Fourier components and then diagonalizing the $2 \times 2$ Pauli matrix associated with each Fourier mode.  (Alternatively, you might just take the determinant of this $2 \times 2$ matrix.)   This 1-loop contribution requires a renormalization proportional to $\sigw$ (that is, a renormalization of $g^2$).  Renormalize by minimal subtraction.
}

\sol{
 	The right-hand side of Eq.~\refeq{show2c} can be written
 	\eqn{thing2d}{
 		\int \DDsig \exp[ -i \int \ddsx \frac{1}{2 g^2} \sig^2 ] \int \DDpsi \exp[ i \int \ddsx \psibsi (i \ptsl - \sig) \psisi ].
 	}
 	The integral of the exponential argument is the Dirac Lagrangian Eq.~\refeq{Dirac} with $m \to \sig$.  A similar integral is given by P\&S~(9.76):
 	\eq{
 		\int \DDpsi \DDpsib \exp[ i \int \ddsx \psib (i \Dsl - m) \psi ] = \det(i \Dsl - m).
 	}
 	For the integral over $\psi$ in Eq.~\refeq{thing2d}, we have $\Dsl \to \ptsl$.  The result is then
 	\eq{
 		\int \DDpsi \DDpsib \exp[ i \int \ddsx \psibsi (i \ptsl - \sig) \psisi ] = [ \det(i \ptsl - \sig) ]^N.
 	}
 	Here we get a power of $N$ because we are integrating over $\psibsi, \psisi$ for $i = 1, \ldots, N$.  Now we apply (9.77),
 	\eq{
 		\det B = \exp[ \tr(\log B) ].
 	}
 	Now we follow a similar procedure as in (11.71),
 	\eq{
 		\tr[ \ln(\pt^2 + m^2) ] = (V T) \int \ddqkf \ln(-k^2 + m^2),
 	}
 	where $(V T)$ is the four-dimensional volume of the functional integral.  Adapting this for the Dirac operator in two dimensions, we have
 	\eq{
 		\tr[ \ln(i \ptsl - \sig) ] = (V T) \int \ddskf \ln(-k^2 + \sig^2).
 	}
 	Here we have used the result
 	\al{
 		\det(i \gamm \ptsm - \sig) &= \det( i \gamo \ptso + i \gamq \ptsq - \sig ) \\
 		&= \det( i \sigw \ptso - \sigq \ptsq - \sig ) \\
 		&= \det\paren{ i \mqty[ 0 & -i \\ i & 0 ] \ptso - \mqty[ 0 & 1 \\ 1 & 0 ] \ptsq - \sig } \\
 		&= \det \mqty[ -\sig & \ptso - \ptsq \\ -(\ptso + \ptsq) & -\sig ] \\
 		&= \sig^2 + \ptso^2 + \ptsq^2 \\
 		&= \ptm \ptsm + \sig^2.
 	}
 	Now we may apply (11.72),
 	\eq{
 		\int \ddqkf \ln(-k^2 + m^2) = -i \frac{\Gam(-d / 2)}{(4\pi)^{d / 2}} \frac{1}{(m^2)^{-d / 2}}.
 	}
 	Letting $d = 2 - \eps$ and expanding about $\eps = 0$~(using Mathematica), we find
 	\eqn{M1}{
 		\int \ddskf \ln(-k^2 + \sig^2) = -i \frac{\Gam(\eps / 2 - 1)}{(4\pi)^{1 - \eps / 2}} \sig^{2 - \eps}
 		\approx \frac{i \sig^2}{4 \pi} \paren{ \frac{2}{\eps} + 1 - \gam + \ln(4\pi) - 2 \ln(\sig) }.
 	}
 	Now we renormalize by minimal subtraction.  Following the examples in (11.77) and (11.78), we make the replacement
 	\eqn{M2}{
 		\int \ddskf \ln(-k^2 + \sig^2) \to \frac{i \sig^2}{4 \pi} \brac{ 1 - \ln(\frac{\sig^2}{M^2}) },
 	}
 	where $M$ is an arbitrary mass parameter.
 	
 	Applying our work to Eq.~\refeq{show2c}, we have
 	\al{
 		\int \DDpsi e^{i \int \ddsx \cL} &= \int \DDsig \exp[ -i \int \ddsx \frac{1}{2 g^2} \sig^2 ] [ \det(i \ptsl - \sig) ]^N \\
 		&= \int \DDsig \exp[ \int \ddsx \paren{ \frac{-i}{2 g^2} \sig^2 + N \int \ddskf \ln(-k^2 + \sig^2) } ] \\
 		&= \int \DDsig \exp[ -i \int \ddsx \sig^2 \curly{ \frac{1}{2 g^2} + \frac{N}{4\pi} \brac{ \ln(\frac{\sig^2}{M^2}) - 1 } } ].
 	}
 	Then the leading correction to the effective potential is
 	\eqn{Veff}{
 		\ans{ \Veff = \sig^2 \curly{ \frac{1}{2 g^2} + \frac{N}{4\pi} \brac{ \ln(\frac{\sig^2}{M^2}) - 1 } }, }
 	}
 	where we have referred to (11.79).
}



\prob{ \label{2e}
	Ignoring two-loop and higher-order contributions, minimize this potential.  Show that the $\sig$ field acquires a vacuum expectation value which breaks the symmetry of \ref{2a}.  Convince yourself that this result does not depend on the particular renormalization condition chosen.
}

\sol{
	We minimize the potential by imposing
	\eq{
		\left. \pdv{\phi^a} V \right|_{\phi^a(x) = \phiso^a} = 0,
	}
	where $\phi^a_0$ is a constant field that minimizes $V$~\cite[p.~351]{Peskin}.  The minimizing $\phiso$ is called the vacuum expectation value, $v$.
	
	For the problem at hand, we want to find $\sig^2$ to minimize the potential Eq.~\refeq{Veff}.  We impose
	\al{
		0 &= \pdv{\Veff}{\sig^2} \\
		&= \frac{1}{2 g^2} + \frac{N}{4\pi} \brac{ \ln(\frac{\sig^2}{M^2}) - 1 } + \sig^2 \pdv{\sig^2}\curly{ \frac{1}{2 g^2} + \frac{N}{4\pi} \brac{ \ln(\frac{\sig^2}{M^2}) - 1 } } \\
		&= \frac{1}{2 g^2} + \frac{N}{4\pi} \brac{ \ln(\frac{\sig^2}{M^2}) - 1 } + \frac{N}{4\pi} \\
		&= \frac{1}{2 g^2} + \frac{N}{4\pi} \ln(\frac{\sig^2}{M^2}).
	}
	Solving for $\sig$, we find the vacuum expectation value:
	\eq{
		-\frac{2\pi}{N g^2} = \ln(\frac{\sig^2}{M^2})
		\qimplies
		\ans{ \sig = \pm M e^{-\pi / N g^2} = \pm v. }
	}
	This vacuum expectation value breaks the symmetry of \ref{2a} because it acts as a nonzero mass for the fields $\psisi$.  As we saw in \ref{2d}, the form of the Lagrangian with $\sig$ looks like the Dirac Lagrangian with mass $\sig = \pm v$, and as we showed in \ref{2a}, the theory is not invariant under Eq.~\refeq{trans} when such a mass is included.  Therefore, this vacuum expectation value breaks the symmetry. \qed
	
	This result does not depend on the particular renormalization condition chosen because all we have done is absorb the divergence into some parameter $M$, which parameterizes a sequence of possible renormalization conditions~\cite[p.~377]{Peskin}.  From Eqs.~\refeq{M1} and \refeq{M2},
	\eq{
		\frac{2}{\eps} + 1 - \gam + \ln(4\pi) - 2 \ln(\sig) =  1 - \ln(\sig^2) + \ln(M^2)
%		\qimplies
%		2 \ln(M) = \frac{2}{\eps} - \gam + \ln(4\pi)
		\qimplies M = \exp[ \frac{1}{\eps} - \frac{\gam}{2} + \frac{\ln(4\pi)}{2} ].
	}
	No matter how we handle the divergent term, it is impossible for $M$ to be zero.  So the symmetry will be broken regardless of which normalization condition we choose.
}



\prob{
	Note that the effective potential derived in \ref{2e} depends on $g$ and $N$ according to the form
	\eq{
		\Veff(\sigcl) = N \cdot f(g^2 N).
	}
	(The overall factor of $N$ is expected in a theory with $N$ fields.)  Construct a few of the higher-order contributions to the effective potential and show that they contain additional factors of $N^{-1}$ which suppress them if we take the limit $N \to \infty$, ($g^2 N$) fixed.  In this limit, the result of \ref{2e} is unambiguous.
}


\makebib

\end{document}
