\documentclass[11pt]{article}
\usepackage{homework}

\classname{444}
\homeworknum{4}



\begin{document}

% Environments

\newcommand{\state}[2]{\begin{statement}{#1} #2 \end{statement}}
\newcommand{\prob}[2]{\begin{problem}{#1} #2 \end{problem}}
\newcommand{\subprob}[1]{\begin{subproblem} #1 \end{subproblem}}
\newcommand{\sol}[1]{\begin{solution} #1 \end{solution}}
\newcommand{\fig}[2]{\begin{figure} \centering #2  \label{#1} \end{figure}}

\newcommand{\makebib}{
	\vfill
	\color{black}
	\bibliography{references}{}
	\bibliographystyle{lucas_unsrt}
}
	

% Implication

\newcommand{\qwhere}{\quad \text{where} \quad}
\newcommand{\qimplies}{\quad \implies \quad}
\newcommand{\impliesq}{\implies \quad}



% Brackets

\newcommand{\paren}[1]{\left( #1 \right)}
\newcommand{\brac}[1]{\left[ #1 \right]}


% Greek

\newcommand{\alp}{\alpha}
\newcommand{\bet}{\beta}
\newcommand{\gam}{\gamma}
\newcommand{\del}{\delta}
\newcommand{\eps}{\epsilon}
\newcommand{\zet}{\zeta}
\newcommand{\tht}{\theta}
\newcommand{\kap}{\kappa}
\newcommand{\lam}{\lambda}
\newcommand{\sig}{\sigma}
\newcommand{\ups}{\upsilon}
\newcommand{\omg}{\omega}

\newcommand{\Gam}{\Gamma}
\newcommand{\Del}{\Delta}
\newcommand{\Tht}{\Theta}
\newcommand{\Lam}{\Lambda}
\newcommand{\Sig}{\Sigma}
\newcommand{\Omg}{\Omega}
% Problem 1

\newcommand{\Psii}{\Psi^i}
\newcommand{\Psiix}{\Psii(x)}

\newcommand{\Pii}{\Pi^i}

\newcommand{\Phii}{\Phi^i}
\newcommand{\Phiix}{\Phii(x)}
\newcommand{\PhiN}{\Phi^N}
\newcommand{\PhiNx}{\PhiN(x)}
\newcommand{\Phiq}{\Phi^1}
\newcommand{\Phiw}{\Phi^2}

\newcommand{\ddcx}{\dd[3]{x}}

\newcommand{\delij}{\del^{i j}}
\newcommand{\delkl}{\del^{k l}}
\newcommand{\delil}{\del^{i l}}
\newcommand{\deljk}{\del^{j k}}
\newcommand{\delik}{\del^{i k}}
\newcommand{\deljl}{\del^{j l}}

\newcommand{\DF}{D_F}

\newcommand{\sigx}{\sig(x)}

\newcommand{\pii}{\pi^i}
\newcommand{\pij}{\pi^j}
\newcommand{\pik}{\pi^k}
\newcommand{\pil}{\pi^l}
\newcommand{\piix}{\pi(x)}

\newcommand{\pq}{p_1}
\newcommand{\pw}{p_2}
\newcommand{\pe}{p_3}
\newcommand{\pr}{p_4}

\newcommand{\vp}{\vb{p}}
\newcommand{\vpsi}{\vp_i}

\newcommand{\mpi}{m_\pi}



\state{Spin-wave theory~(P\&S 11.1)}{\hfix}

\prob{
	Prove the following wonderful formula: Let $\phix$ be a free scalar field with propagator $\ev{T \phix \phio} = \Dx$.  Then
	\eq{
		\ev{ T e^{i \phix} e^{-i \phio} } = e^{[ \Dx - \Do ]}.
	}
	(The  factor $\Do$ gives a formally divergent adjustment of the overall normalization.)
}



\prob{
	We can use this formula in Euclidean field theory to discuss correlation functions in a theory with spontaneously broken symmetry for $T < \TC$.  Let us consider only the simplest case of a broken $\O(2)$ or $U(1)$ symmetry.  We can write the local spin density as a complex variable
	\eq{
		\sx = \sqx + i \swx.
	}
	The global symmetry is the transformation
	\eq{
		\sx \to e^{-i \alp} \sx.
	}
	If we assume that the physics freezes the modulus of $\sx$, we can parameterize
	\eq{
		\sx = A e^{i \phix}
	}
	and write an effective Lagrangian for the field $\phix$.  The symmetry of the theory becomes the translation symmetry
	\eq{
		\phix \to \phix - \alp.
	}
	Show that (for $d > 0$) the most general renormalizable Lagrangian consistent with this symmetry is the free field theory
	\eq{
		\cL = \frac{1}{2} \rho(\vgrad \phi)^2.
	}
	In statistical mechanics, the constant $\rho$ is called the \emph{spin wave modulus}.  A reasonable hypothesis for $\rho$ os that it is finite for $T < \TC$ and tends to 0 as $T \to \TC$ from below.
}



\prob{
	Compute the correlation function $\ev{ \sx \sao }$.  Adjust $A$ to give a physically sensible normalization (assuming that the system has a physical cutoff at the scale of one atomic spacing) and display the dependence of this correlation function on $x$ for $d = 1, 2, 3, 4$.  Explain the significance of your results.
}







\state{The Gross-Neveu model~(P\&S 11.3)}{
	The Gross-Neveu model is a model in two spacetime dimensions of fermions with a discrete chiral symmetry:
	\eq{
		\cL = \psibsi i \ptsl \psisi + \frac{1}{2} g^2 (\psibsi \psisi)^2
	}
	with $i = 1, \ldots, N$.  The kinetic term of two-dimensional fermions is built from matrices $\gamm$ that satisfy the two-dimensional Dirac algebra.  These matrices can be $2 \times 2$:
	\al{
		\gamo &= \sigw, &
		\gamq &= i \sigq,
	}
	where $\sigi$ are Pauli sigma matrices.  Define
	\eq{
		\gamt = \gamo \gamq = \sige;
	}
	this matrix anticommutes with the $\gamm$.
}

\prob{ \label{2a}
	Show that this theory is invariant with respect to
	\eq{
		\psisi \to \gamt \psisi,
	}
	and that this symmetry forbids the appearance of a fermion mass.
}



\prob{
	Show that this theory is renormalizable in 2 dimensions (at the level of dimensional analysis).
}



\prob{
	Show that the functional integral for this theory can be represented in the following form:
	\eq{
		\int \DDpsi e^{i \int \ddsx \cL} = \int \DDpsi \DDsig \exp[ i \int \ddsx \curly{ \psibsi i \ptsl \psisi - \sig \psibsi \psisi - \frac{1}{2 g^2} \sig^2 } ],
	}
	where $\sigx$ (not to be confused with a Pauli matrix) is a new scalar field with no kinetic energy terms.
}



\prob{
	Compute the leading correction to the effective potential for $\sig$ by integrating over the fermion fields $\psisi$.  You will encounter the determinant of a Dirac operator; to evaluate this determinant, diagonalize the operator by first going to Fourier components and then diagonalizing the $2 \times 2$ Pauli matrix associated with each Fourier mode.  (Alternatively, you might just take the determinant of this $2 \times 2$ matrix.)   This 1-loop contribution requires a renormalization proportional to $\sigw$ (that is, a renormalization of $g^2$).  Renormalize by minimal subtraction.
}



\prob{ \label{2e}
	Ignoring two-loop and higher-order contributions, minimize this potential.  Show that the $\sig$ field acquires a vacuum expectation value which breaks the symmetry of \ref{2a}.  Convince yourself that this result does not depend on the particular renormalization condition chosen.
}



\prob{
	Note that the effective potential derived in \ref{2e} depends on $g$ and $N$ according to the form
	\eq{
		\Veff(\sigcl) = N \cdot f(g^2 N).
	}
	(The overall factor of $N$ is expected in a theory with $N$ fields.)  Construct a few of the higher-order contributions to the effective potential and show that they contain additional factors of $N^{-1}$ which suppress them if we take the limit $N \to \infty$, ($g^2 N$) fixed.  In this limit, the result of \ref{2e} is unambiguous.
}


\makebib

\end{document}
