\documentclass[11pt]{article}
\usepackage{homework}

\classname{444}
\homeworknum{3}



\begin{document}

% Environments

\newcommand{\state}[2]{\begin{statement}{#1} #2 \end{statement}}
\newcommand{\prob}[2]{\begin{problem}{#1} #2 \end{problem}}
\newcommand{\subprob}[1]{\begin{subproblem} #1 \end{subproblem}}
\newcommand{\sol}[1]{\begin{solution} #1 \end{solution}}
\newcommand{\fig}[2]{\begin{figure} \centering #2  \label{#1} \end{figure}}

\newcommand{\makebib}{
	\vfill
	\color{black}
	\bibliography{references}{}
	\bibliographystyle{lucas_unsrt}
}
	

% Implication

\newcommand{\qwhere}{\quad \text{where} \quad}
\newcommand{\qimplies}{\quad \implies \quad}
\newcommand{\impliesq}{\implies \quad}



% Brackets

\newcommand{\paren}[1]{\left( #1 \right)}
\newcommand{\brac}[1]{\left[ #1 \right]}


% Greek

\newcommand{\alp}{\alpha}
\newcommand{\bet}{\beta}
\newcommand{\gam}{\gamma}
\newcommand{\del}{\delta}
\newcommand{\eps}{\epsilon}
\newcommand{\zet}{\zeta}
\newcommand{\tht}{\theta}
\newcommand{\kap}{\kappa}
\newcommand{\lam}{\lambda}
\newcommand{\sig}{\sigma}
\newcommand{\ups}{\upsilon}
\newcommand{\omg}{\omega}

\newcommand{\Gam}{\Gamma}
\newcommand{\Del}{\Delta}
\newcommand{\Tht}{\Theta}
\newcommand{\Lam}{\Lambda}
\newcommand{\Sig}{\Sigma}
\newcommand{\Omg}{\Omega}
% Problem 1

\newcommand{\Psii}{\Psi^i}
\newcommand{\Psiix}{\Psii(x)}

\newcommand{\Pii}{\Pi^i}

\newcommand{\Phii}{\Phi^i}
\newcommand{\Phiix}{\Phii(x)}
\newcommand{\PhiN}{\Phi^N}
\newcommand{\PhiNx}{\PhiN(x)}
\newcommand{\Phiq}{\Phi^1}
\newcommand{\Phiw}{\Phi^2}

\newcommand{\ddcx}{\dd[3]{x}}

\newcommand{\delij}{\del^{i j}}
\newcommand{\delkl}{\del^{k l}}
\newcommand{\delil}{\del^{i l}}
\newcommand{\deljk}{\del^{j k}}
\newcommand{\delik}{\del^{i k}}
\newcommand{\deljl}{\del^{j l}}

\newcommand{\DF}{D_F}

\newcommand{\sigx}{\sig(x)}

\newcommand{\pii}{\pi^i}
\newcommand{\pij}{\pi^j}
\newcommand{\pik}{\pi^k}
\newcommand{\pil}{\pi^l}
\newcommand{\piix}{\pi(x)}

\newcommand{\pq}{p_1}
\newcommand{\pw}{p_2}
\newcommand{\pe}{p_3}
\newcommand{\pr}{p_4}

\newcommand{\vp}{\vb{p}}
\newcommand{\vpsi}{\vp_i}

\newcommand{\mpi}{m_\pi}


\state{Renormalization of Yukawa theory (P\&S 10.2)}{
	Consider the pseudoscalar Yukawa Lagrangian,
	\eqn{given1}{
		\cL = \frac{1}{2} (\ptsm \phi)^2 - \frac{1}{2} m^2 \phi^2 + \psib (i \ptsl - M) \psi - i g \psib \gamt \psi \phi,
	}
	where $\phi$ is a real scalar field and $\psi$ is a Dirac fermion.  Notice that this Lagrangian is invariant under the parity transformation $\psi(t, \vx) \to \gamo \psi(zt, -\vx)$, $\phi(t, \vx) \to -\phi(t, -\vx)$, in which the field $\phi$ carries odd parity.
}

\prob{
	Determine the superficially divergent amplitudes and work out the Feynman rules for renormalized perturbation theory for this Lagrangian.  Include all necessary counterterm vertices.  Show that the theory contains a superficially divergent $4 \phi$ amplitude.  This means that the theory cannot be renormalized unless one includes a scalar self-interaction,
	\eq{
		\del\cL = \frac{\lam}{4!} \phi^4,
	}
	and a counterterm of the same form.  It is of course possible to set the renormalized value of this coupling to zero, but that is not a natural choice, since the counterterm will still be nonzero.  Are any further interactions required?
}

\sol{
	The Feynman rules for a pseudoscalar Yukawa theory are~\cite[pp.~24--25]{FR}
	\hl{copy over the diagrams}
	
%	\al{
%		\centergraphics{diag/vertex} &= -g \gamt, &
%		\centergraphics{diag/propagator} &= \frac{i}{q^2 - m^2 + i \eps}, &
%		add the fermion one.
%	}
	These Feynman rules are similar enough to those for QED; that is, the powers of $k$ are the same, each propagator has a momentum integral, each vertex has a delta function, and each vertex involves one $\phi$ line and two fermion lines~\cite[p.~316]{Peskin}.  So we can adapt P\&S~(10.4) for the superficial degree of divergence:
	\eq{
		D = 4 - \Nphi - \frac{3}{2} \Nf,
	}
	where $\Nphi$ is the number of external $\phi$ lines and $\Nf$ is the number of external fermion lines.
	
	This means the superficially divergent amplitudes are a subset of those appearing in Fig.~10.2 of P\&S, with the photon lines replaced by pseudoscalar lines:
	
	\hl{draw the diagrams from p.~318}
	
	We ignore (a) since it is irrelevant to scattering processes~\cite[pp.~317--318]{Peskin}.  Amplitudes~(b) and (d) vanish because the theory is invariant under the parity transformation, which means all amplitudes with zero fermion legs and an odd number of external $\phi$ legs vanish~\cite[pp.~318, 323--324]{Peskin}.  So the superficially divergent amplitudes are
	
	\hl{draw the remaining ones but in blue}
	
	To work out the renormalized theory, we rescale the field as in P\&S~(10.15):
	\eq{
		\phi = \Zq^{1/2} \phir.
	}
	The rescaling for the fermion is~\cite[p.~330]{Peskin}
	\eq{
		\psi = \Zw^{1/2} \psir.
	}
	Feeding these into Eq.~\refeq{given1}, we obtain the renormalized Lagrangian~\cite[p.~324]{Peskin}
	\eq{
		\cL = \frac{1}{2} \Zq (\ptsm \phi)^2 - \frac{1}{2} \Zq m^2 \phi^2 + \Zw \psib (i \ptsl - M) \psi - i \Zq^{1/2} \Zw \go \psib \gamt \psi \phi,
	}
	where $\mo$ and $\Mo$ are the bare masses, and $\go$ is the bare coupling constant.  Define~\cite[pp.~324, 331]{Peskin}
	\al{
		\delZq &= \Zq - 1, &
		\delZw &= \Zw - 1, &
		\delm &= \mo^2 \Zq - m^2, &
		\delM &= \Mo \Zw - M, &
		\delg &= (\go / g) \Zq^{1/2} \Zw - 1.
	}
}



\prob{
	Compute the divergent part (the pole as $d \to 4$) of each counterterm, to the one-loop order of perturbation theory, implementing a sufficient set of renormalization conditions.  You need not worry about finite parts of the counterterms.  Since the divergent parts must have a fixed dependence on the external momenta, you can simplify this calculation by choosing the momenta in the simplest possible form.
}


\makebib

\end{document}
