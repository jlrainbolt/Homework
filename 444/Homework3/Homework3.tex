\documentclass[11pt]{article}
\usepackage{homework}

\classname{444}
\homeworknum{3}



\begin{document}

% Environments

\newcommand{\state}[2]{\begin{statement}{#1} #2 \end{statement}}
\newcommand{\prob}[2]{\begin{problem}{#1} #2 \end{problem}}
\newcommand{\subprob}[1]{\begin{subproblem} #1 \end{subproblem}}
\newcommand{\sol}[1]{\begin{solution} #1 \end{solution}}
\newcommand{\fig}[2]{\begin{figure} \centering #2  \label{#1} \end{figure}}

\newcommand{\makebib}{
	\vfill
	\color{black}
	\bibliography{references}{}
	\bibliographystyle{lucas_unsrt}
}
	

% Implication

\newcommand{\qwhere}{\quad \text{where} \quad}
\newcommand{\qimplies}{\quad \implies \quad}
\newcommand{\impliesq}{\implies \quad}



% Brackets

\newcommand{\paren}[1]{\left( #1 \right)}
\newcommand{\brac}[1]{\left[ #1 \right]}


% Greek

\newcommand{\alp}{\alpha}
\newcommand{\bet}{\beta}
\newcommand{\gam}{\gamma}
\newcommand{\del}{\delta}
\newcommand{\eps}{\epsilon}
\newcommand{\zet}{\zeta}
\newcommand{\tht}{\theta}
\newcommand{\kap}{\kappa}
\newcommand{\lam}{\lambda}
\newcommand{\sig}{\sigma}
\newcommand{\ups}{\upsilon}
\newcommand{\omg}{\omega}

\newcommand{\Gam}{\Gamma}
\newcommand{\Del}{\Delta}
\newcommand{\Tht}{\Theta}
\newcommand{\Lam}{\Lambda}
\newcommand{\Sig}{\Sigma}
\newcommand{\Omg}{\Omega}
% Problem 1

\newcommand{\Psii}{\Psi^i}
\newcommand{\Psiix}{\Psii(x)}

\newcommand{\Pii}{\Pi^i}

\newcommand{\Phii}{\Phi^i}
\newcommand{\Phiix}{\Phii(x)}
\newcommand{\PhiN}{\Phi^N}
\newcommand{\PhiNx}{\PhiN(x)}
\newcommand{\Phiq}{\Phi^1}
\newcommand{\Phiw}{\Phi^2}

\newcommand{\ddcx}{\dd[3]{x}}

\newcommand{\delij}{\del^{i j}}
\newcommand{\delkl}{\del^{k l}}
\newcommand{\delil}{\del^{i l}}
\newcommand{\deljk}{\del^{j k}}
\newcommand{\delik}{\del^{i k}}
\newcommand{\deljl}{\del^{j l}}

\newcommand{\DF}{D_F}

\newcommand{\sigx}{\sig(x)}

\newcommand{\pii}{\pi^i}
\newcommand{\pij}{\pi^j}
\newcommand{\pik}{\pi^k}
\newcommand{\pil}{\pi^l}
\newcommand{\piix}{\pi(x)}

\newcommand{\pq}{p_1}
\newcommand{\pw}{p_2}
\newcommand{\pe}{p_3}
\newcommand{\pr}{p_4}

\newcommand{\vp}{\vb{p}}
\newcommand{\vpsi}{\vp_i}

\newcommand{\mpi}{m_\pi}


\state{Renormalization of Yukawa theory (P\&S 10.2)}{
	Consider the pseudoscalar Yukawa Lagrangian,
	\eqn{given1}{
		\cL = \frac{1}{2} (\ptsm \phi)^2 - \frac{1}{2} m^2 \phi^2 + \psib (i \ptsl - M) \psi - i g \psib \gamt \psi \phi,
	}
	where $\phi$ is a real scalar field and $\psi$ is a Dirac fermion.  Notice that this Lagrangian is invariant under the parity transformation $\psi(t, \vx) \to \gamo \psi(zt, -\vx)$, $\phi(t, \vx) \to -\phi(t, -\vx)$, in which the field $\phi$ carries odd parity.
}

\prob{ \label{1a}
	Determine the superficially divergent amplitudes and work out the Feynman rules for renormalized perturbation theory for this Lagrangian.  Include all necessary counterterm vertices.  Show that the theory contains a superficially divergent $4 \phi$ amplitude.  This means that the theory cannot be renormalized unless one includes a scalar self-interaction,
	\eqn{given1a}{
		\del\cL = \frac{\lam}{4!} \phi^4,
	}
	and a counterterm of the same form.  It is of course possible to set the renormalized value of this coupling to zero, but that is not a natural choice, since the counterterm will still be nonzero.  Are any further interactions required?
}

\sol{
	We write Eq.~\refeq{given1} explicitly in terms of the bare masses $\mo, \Mo$ and the bare coupling constant $\go$:
	\eqn{newlagr}{
		\cL = \frac{1}{2} (\ptsm \phi)^2 - \frac{1}{2} \mo^2 \phi^2 + \psib (i \ptsl - \Mo) \psi - i \go \psib \gamt \psi \phi,
	}
	The Feynman rules for a pseudoscalar Yukawa theory are~\cite[pp.~24--25]{FR} \\[-5ex]
	\al{
		\centergraphics{diag/scalar_prop} &= \frac{i}{q^2 - \mo^2 + i \eps} &
		\centergraphics{diag/fermion_prop} &= \frac{i (\psl + \Mo)}{p^2 - \Mo^2 + i \eps} \\[-7ex]
	}
	\eq{
		\centergraphics{diag/scalar_fermion_vertex} = \go \gamt
	}
	These Feynman rules are similar enough to those for QED; that is, the powers of $k$ are the same, each propagator has a momentum integral, each vertex has a delta function, and each vertex involves one $\phi$ line and two fermion lines~\cite[p.~316]{Peskin}.  So we can adapt P\&S~(10.4) for the superficial degree of divergence:
	\eq{
		D = 4 - \Nphi - \frac{3}{2} \Nf,
	}
	where $\Nphi$ is the number of external $\phi$ lines and $\Nf$ is the number of external fermion lines.
	
	This means the superficially divergent amplitudes are a subset of those appearing in Fig.~10.2 of P\&S, with the photon lines replaced by pseudoscalar lines: \\[-4ex]
	\al{
		\text{(a)} &\: \centergraphics{diag/amp_a} \quad D = 4 &
		\text{(b)} &\: \centergraphics{diag/amp_b} \quad D = 3 &
		\text{(c)} &\: \centergraphics{diag/amp_c} \quad D = 2 \\
		\text{(d)} &\: \centergraphics{diag/amp_d} \quad D = 1 &
		\text{(e)} &\: \centergraphics{diag/amp_e} \quad D = 0 &
		\text{(f)} &\: \centergraphics{diag/amp_f} \quad D = 1 \\
		& &
		\text{(g)} &\: \centergraphics{diag/amp_g} \quad D = 0 &
	}
	We ignore (a) since it is irrelevant to scattering processes~\cite[pp.~317--318]{Peskin}.  Amplitudes~(b) and (d) vanish because the theory is invariant under the parity transformation, which means all amplitudes with zero external fermion legs and an odd number of external $\phi$ legs vanish~\cite[pp.~318, 323--324]{Peskin}.  So the superficially divergent amplitudes are \\[-3ex]
	\aln{
		\ans{ \text{(c)} } &\: \centergraphics{diag/blue_amp_c} \quad \ans{ D = 2 } &
		\ans{ \text{(e)} } &\: \centergraphics{diag/blue_amp_e} \quad \ans{ D = 0 } &
		\ans{ \text{(f)} } &\: \centergraphics{diag/blue_amp_f} \quad \ans{ D = 1 } \notag \\
		& &
		\ans{ \text{(g)} } &\: \centergraphics{diag/blue_amp_g} \quad \ans{ D = 0 } & \label{amps}
	}%
	\ans{Note that amplitude~(e) is a $4\phi$ amplitude.}  Since it is superficially divergent, according to the problem statement  we must introduce the scalar self-interaction given by Eq.~\refeq{given1a}.  We subtract this term as in the $\phi^4$ theory~\cite[p.~324]{Peskin}.  The Feynman rule for this vertex is~\cite[p.~325]{Peskin}
	\eq{
		\centergraphics{diag/scalar_vertex} = -i \lamo.
	}
	With the addition of this new term, our Lagrangian in Eq.~\refeq{newlagr} becomes
	\eqn{newlagr2}{
		\cL = \frac{1}{2} (\ptsm \phi)^2 - \frac{1}{2} \mo^2 \phi^2 + \psib (i \ptsl - \Mo) \psi - i \go \psib \gamt \psi \phi - \frac{\lamo}{4!} \phi^4,
	}
	where $\lamo$ is the bare coupling constant for the scalar self-interaction.  To work out the renormalized theory, we rescale the field as in P\&S~(10.15):
	\eq{
		\phi = \Zq^{1/2} \phir.
	}
	The rescaling for the fermion is~\cite[p.~330]{Peskin}
	\eq{
		\psi = \Zw^{1/2} \psir.
	}
	Feeding these into Eq.~\refeq{newlagr2}, we obtain the renormalized Lagrangian~\cite[p.~324]{Peskin}
	\eqn{lagr2}{
		\cL = \frac{1}{2} \Zq (\ptsm \phi)^2 - \frac{1}{2} \Zq \mo^2 \phi^2 + \Zw \psib (i \ptsl - \Mo) \psi - i \Zq^{1/2} \Zw \go \psib \gamt \psi \phi - \frac{\lamo}{4!} \Zq^2 \phi^4.
	}
	Define~\cite[pp.~324, 331]{Peskin}
	\al{
		\delZq &= \Zq - 1, &
		\delZw &= \Zw - 1, &
		\delm &= \mo^2 \Zq - m^2, \\
		\delM &= \Mo \Zw - M, &
		\delg &= (\go / g) \Zq^{1/2} \Zw - 1, &
		\dellam &= \lamo \Zq^2 - \lam
	}
	Then Eq.~\refeq{lagr2} becomes
	\al{
		\cL &= \frac{1}{2} (1 + \delZq) (\ptsm \phi)^2 - \frac{1}{2} (m^2 + \delm) \phi^2 + \psib [ i (\delZw + 1) \ptsl - (M + \delM) ] \psi - i g (1 + \delg) \psib \gamt \psi \phi + \frac{\lam + \dellam}{4!} \phi^4 \\[2ex]
		&= \frac{1}{2} (\ptsm \phi)^2 - \frac{1}{2} m^2 \phi^2 + \psib (i \ptsl - M) \psi - i g \psib \gamt \psi \phi - \frac{\lam}{4!} \phi^4 \notag \\
		&\hspace{5em} \phantom{=\ } + \frac{1}{2} \delZq (\ptsm \phi)^2 - \frac{1}{2} \delm \phi^2 + \psib (i \delZw \ptsl - \delM) \psi - i g \delg \psib \gamt \psi \phi - \frac{\dellam}{4!} \phi^4.
	}
	Here the first first five terms look like Eq.~\refeq{newlagr2}, but written in terms of the physical masses and couplings.  The last five terms are the counterterms~\cite[p.~325]{Peskin}.
	
	The Feynman rules for the renormalized theory are~\cite[p.~325]{Peskin} \\[-5ex]
	\ans{\al{
		\centergraphics{diag/blue_scalar_prop} &= \frac{i}{q^2 - m^2 + i \eps} &
		\centergraphics{diag/ren_scalar_prop} &= i (p^2 \delZq - \delm) \\
		\centergraphics{diag/blue_fermion_prop} &= \frac{i (\psl + M)}{p^2 - M^2 + i \eps} &
		\centergraphics{diag/ren_fermion_prop} &= i (\psl \delZw - \delM) \\
		\centergraphics{diag/blue_scalar_fermion_vertex} &= g \gamt &
		\centergraphics{diag/ren_scalar_fermion_vertex} &= g \delg \gamt \\
		\centergraphics{diag/blue_scalar_vertex} &= -i \lam &
		\centergraphics{diag/ren_scalar_vertex} &= -i \dellam
	}}%
	No further interactions are required because once we have added the $\phi^4$ term, the Lagrangian in Eq.~\refeq{lagr2} contains terms that reflect all of the amplitudes in Eq.~\refeq{amps}.
}



\prob{
	Compute the divergent part (the pole as $d \to 4$) of each counterterm, to the one-loop order of perturbation theory, implementing a sufficient set of renormalization conditions.  You need not worry about finite parts of the counterterms.  Since the divergent parts must have a fixed dependence on the external momenta, you can simplify this calculation by choosing the momenta in the simplest possible form.
}

\sol{
	To compute the divergent part of the fermion propagator counterterm to one-loop order, we include the fermion self-energy similar to P\&S~(7.15):
	\vspace{-.5cm}
	\eq{
		\centergraphics{diag/ren_fermion_prop_black}
		\quad + \quad
		\centergraphics{diag/self_energy}
	}
	\vspace{-1.25cm}
	
	The fermion-self energy here looks similar to that in QED, so we may adapt P\&S~(7.16) for that term.  Using our Feynman rules from \ref{1a}, we have
	\aln{
		-i M^2(p^2) &= i (\psl \delZw - \delM) + g^2 \int \dddkf \gamt \frac{i (\ksl + M)}{p^2 - M^2 + i \eps} \gamt \frac{i}{(p - k)^2 - m^2 + i \eps} \notag \\
		&= i (\psl \delZw - \delM) + g^2 \int \dddkf \frac{\ksl - M}{(p^2 - M^2 + i \eps) [ (p - k)^2 - m^2 + i \eps ]} \label{thing1b2},
	}
	where we have used P\&S~(3.70), $(\gamt)^2 = 1$, and (3.71), $\{ \gamt, \gamm \} = 0$, which implies $\gamt \gamm \gamt = -\gamm$.  Following the procedure on pp.~217--218, we introduce the Feynman parameter $x$ to combine the denominators:
	\eqn{FP}{
		\frac{1}{k^2 - \mo^2 + i \eps} \frac{1}{(p - k)^2 - \mu^2 + i \eps} = \intoq \ddx \frac{1}{[ k^2 - 2 x k \cdot p + x p^2 - x \mu^2 - (1 - x) \mo^2 + i \eps ]^2}.
	}
	Let $\ell = k - x p$ and $\Del = -x(1 - x) p^2 + x m^2 + (1 - x) M^2$.  Then Eq.~\refeq{thing1b2} can be written
	\eqn{thing1b2b}{
		-i M^2(p^2) = i (\psl \delZw - \delM) + g^2 \intoq \ddx \int \dddlf \frac{x \psl - M}{(\ell^2 - \Del + i \eps)^2}.
	}
	To evaluate the integral, we can write it in terms of the Euclidean 4-momentum defined by~\cite[p.~193]{Peskin}
	\aln{ \label{ellE}
		\ello &\equiv i \ellEo, &
		\vell &= \vellE.
	}
	Then we can write
	\eqn{int1}{
		\int \dddlf \frac{1}{(\ell^2 - \Del)^2} = \frac{i}{(-1)^2} \frac{1}{(2\pi)^d} \int \dddlE \frac{1}{(\ellE^2 + \Del)^2}
		= i \int \dddlEf \frac{1}{(\ellE^2 + \Del)^2}.
	}
	Then we can apply (7.84), which takes the limit as $d \to 4$:
	\eqn{int2}{
		\int \dddlEf \frac{1}{(\ellE^2 + \Del)^2} \to \frac{1}{(4\pi)^2} \paren{ \frac{2}{\eps} - \gam + \ln(\frac{4\pi}{\Del}) }
		\to \frac{1}{8 \pi^2 \eps},
	}
	where $\eps = 4 - d$~\cite[p.~250]{Peskin}, and we have omitted the finite parts.  Making these substitutions into Eq.~\refeq{Sigw2}, we find
	\al{
		-i M^2(p^2) &= i (\psl \delZw - \delM) + \frac{i g^2}{8 \pi^2 \eps} \intoq \ddx (x \psl - M) \\
		&= i (\psl \delZw - \delM) + \frac{i g^2}{8 \pi^2 \eps} \brac{ \frac{x^2}{2} \psl - M x }_0^1 \\
		&= i (\psl \delZw - \delM) + \frac{i g^2}{8 \pi^2 \eps} \paren{ \frac{\psl}{2} - M } \\
		&= i \psl \paren{ \delZw + \frac{g^2}{16 \pi^2 \eps} } - i \paren{ \delM + \frac{g^2}{8 \pi^2 \eps} M }.
	}
	This implies that
	\aln{ \label{rules1}
		\delZq &= -\frac{g^2}{16 \pi^2 \eps}, &
		\delM &= -\frac{g^2}{8 \pi^2 \eps} M
	}
	are the conditions to eliminate the divergence.
	
	
	
	For the scalar-fermion vertex, we can adapt some of our work from Prob.~2(a) of Homework~1.  With the one-loop diagram similar to the one on p.~189 of P\&S, we have
	\eq{
		\centergraphics{diag/ren_scalar_fermion_vertex_black}
		\quad + \quad
		\centergraphics{diag/ren_scalar_fermion_loop}
	}
	We adapt Peskin \& Schroeder (6.38) using the pseudoscalar field Feynman rules to write~\cite[p.~123]{Peskin}
	\aln{
		\ubpp \del\Gamm(p, p') \up &= \ubpp g \delg \gamt \up + i g^3 \int \dddkf \ubpp \frac{\gamt (\ksl' + M) \gamt (\ksl + M) \gamt}{[ (k - p)^2 - m^2 + i \eps ] ({k'}^2 - M^2 + i \eps) (k^2 - M^2 + i \eps)} \up \notag \\
		&= \ubpp g \delg \gamt \up + i g^3 \gamt \int \dddkf \ubpp \frac{(\ksl' + M) (\ksl - M)}{[ (k - p)^2 - m^2 + i \eps ] ({k'}^2 - M^2 + i \eps) (k^2 - M^2 + i \eps)} \up, \label{thing1bc}
	}
	where we have once more used $(\gamt)^2 = 1$ and $\{ \gamt, \gamm \} = 0$.  We use Peskin \& Schroeder~(6.41) to write 
	\eqn{denom}{
		\frac{1}{[ (k - p)^2 - m^2 + i \eps ] ({k'}^2 - M^2 + i \eps) (k^2 - M^2 + i \eps)} = \intoq \ddx \ddy \ddz \del(x + y + z - 1) \frac{2}{D^3},
	}
	where~\cite[pp.~190--191]{Peskin}
	\aln{
		D &= k^2 + 2 k (q y - p z) + z (p^2 - m^2) - (1 - z) M^2 + i \eps \notag \\
		&= k^2 - 2 k p z + z (p^2 - m^2) - (1 - z) M^2 + i \eps \notag \\
		&= \ell^2 - \Del + i \eps. \label{D}
	}
	Here we have used $x + y + z = 1$ and set $q = 0$ (so $k' = k$) as in Prob.~2(a) of Homework~1.  We have defined $\ell \equiv k - z p$~\cite[p.~191]{Peskin}, and $\Del \equiv (1 - z)^2 M^2 + z m^2$.  For the numerator of Eq.~\refeq{thing1bc}, we use $\ell \equiv k - z p$~\cite[p.~191]{Peskin}, and define
	\aln{
		N &\equiv \ubpp (\lsl + z \psl + M) (\lsl + z \psl - M) \up \notag \\
		&= \ubpp (\lsl \lsl + z \lsl \psl - M \lsl + z \psl \lsl + z^2 \psl \psl - z M \psl + M \lsl + z M \psl - M^2) \up. \label{N1}
	}
	To simplify $N$ we apply~(7.87),
	\eq{
		\int \dddlf \frac{\ellm \elln}{D^3} = \int \dddlf \frac{\gmn \ell^2}{d D^3},
	}
	the fact that we may drop terms linear in $\ell$ by (6.45),
	\eq{
		\int \dddlf \frac{\ellm}{D^3} = 0,
	}
	as well as~\cite[pp.~191--192]{Peskin}
	\al{
		\psl \up &= M \up, &
		\ubpp \psl' &= \ubpp M,
	}
	The Eq.~\refeq{N1} becomes
	\eq{
		N = \ubpp (\ell^2 + z^2 M^2 - z M^2 + z M^2 - M^2) \up
		= \ubpp [ \ell^2 + (z^2 - 1) M^2 ] \up.
	}	
	With this and Eqs.~\refeq{denom} and \refeq{D}, we can write Eq.~\refeq{thing1bc} in the form
	\eqn{thing1bc2}{
		\del\Gamm(p, p') = g \delg \gamt + 2 i g^3 \gamt \intoq \ddz \dddlf \frac{\ell^2 + (z^2 - 1) M^2}{(\ell^2 - \Del + i \eps)^3}.
	}
	To solve the integrals over $\ell$, we use some work from Prob.~1(b) of Homework~1.  We substituted $\eps = 4 - d$ into P\&S~(7.85) and (7.86) and found
	\al{
		\int \dddlf \frac{1}{(\ell^2 - \Del)^3} &= -\frac{i}{(4\pi)^{2 - \eps / 2}} \frac{\Gam(1 + \eps / 2)}{\Gam(3)} \paren{ \frac{1}{\Del} }^{1 + \eps / 2}, \\
		\int \dddlf \frac{\ell^2}{(\ell^2 - \Del)^3} &= \frac{i}{(4\pi)^{2 - \eps / 2}} \frac{4 - \eps}{2} \frac{\Gam(\eps / 2)}{\Gam(3)} \paren{ \frac{1}{\Del} }^{\eps / 2}.
	}
	Then for Eq.~\refeq{thing1bc2}, we have
	\al{
		\del\Gamm(p, p') &= g \delg \gamt - \frac{2}{(4\pi)^{2 - \eps / 2}} g^3 \gamt \intoq \ddz (1 - z) \brac{ \frac{4 - \eps}{2} \frac{\Gam(\eps / 2)}{\Gam(3)} \paren{ \frac{1}{\Del} }^{\eps / 2} - (z^2 - 1) M^2 \frac{\Gam(1 + \eps / 2)}{\Gam(3)} \paren{ \frac{1}{\Del} }^{1 + \eps / 2} } \\
		&= g \delg \gamt - \frac{g^3}{(4\pi)^2} \gamt \intoq \ddz (1 - z) \paren{ \frac{4\pi}{\Del} }^{\eps / 2} \brac{ \frac{4 - \eps}{2} \Gam(\eps / 2) + \frac{(1 - z^2) M^2}{\Del} \Gam(1 + \eps / 2) } \\
		&\to g \delg \gamt - \frac{g^3}{(4\pi)^2} \frac{4}{\eps} \gamt \intoq \ddz (1 - z) \\
%		&= g \delg \gamt + \frac{i g^3}{8 \pi^2} \frac{1}{\eps} \gamt, \\
		&= g \gamt \paren{ \delg - \frac{g^2}{8 \pi^2} \frac{1}{\eps} } \gamt,
	}
	where we have taken the $\eps \to 0$ limit using Mathematica and retained only divergent terms.   Then in order to eliminate the divergence, we need
	\eqn{rules2}{
		\delg = \frac{g^2}{8 \pi^2} \frac{1}{\eps}.
	}
	
	
	
	For the divergent part of the pseudoscalar propagator, we must include the pseudoscalar self-energy.  Since our Feynman rules include two different vertices involving the pseudoscalar, we can draw two different self-energy diagrams:
	\vspace{-.25cm}
	\eqn{diag2}{
		\centergraphics{diag/ren_scalar_prop_black}
		\quad + \quad
		\centergraphics{diag/scalar_fermion_loop}
		\quad + \quad
		\centergraphics{diag/scalar_loop}
	}
	Adapting P\&S~(10.32) for the second diagram, we have
	\eq{
		-g^2 \int \dddkf \tr[ \frac{i \gamt (\ksl + \psl + M)}{(k + p)^2 - M^2 + i \eps} \frac{i \gamt (\ksl + M)}{k^2 - M^2 + i \eps} ]
		= -g^2 \int \dddkf \tr[ -\frac{(\ksl + \psl - M) (\ksl + M)}{[ (k + p)^2 - M^2 + i \eps ] [ k^2 - M^2 + i \eps ]} ].
	}
	We introduce the Feynman parameter $x$~\cite[pp.~217, 327]{Peskin}:
	\eqn{FP}{
		\frac{1}{(k + p)^2 - M^2 + i \eps} \frac{1}{k^2 - M^2 + i \eps} = \intoq \ddx \frac{1}{[ k^2 + 2 x k \cdot p + x p^2 - M^2 + i \eps ]^2}
		\equiv \intoq \ddx \frac{1}{D^2}.
	}
	Let $\ell = k + x p$ and $\Del = M^2 - x (1 - x) p^2$~\cite[pp.~327, 329]{Peskin}.  Then $D = \ell^2 - \Del + i \eps$.  For the numerator, note that
	\eq{
		(\ksl + \psl - M) (\ksl + M) = \ksl \ksl + M \ksl + \psl \ksl + M \psl - M \ksl - M^2,
	}
	so
	\aln{
		N &\equiv \tr[ -(\ksl + \psl - M) (\ksl + M) ] = 4 [ k \cdot (p + k) - M^2 ] \notag \\
		&= 4 [ (\ell - x p) \cdot (p + \ell - x p) - M^2 ] \notag \\
		&= 4 [ \ell^2 + (1 - x) \ell \cdot p - x p \cdot \ell + x (1 - x) p^2 - M^2 ] \notag \\
		&= 4 [ \ell^2 + x (1 - x) p^2 - M^2 ] \label{N2}
	}
	since
	\aln{ \label{traces}
		\tr(\vq) &= 0, &
		\tr(\text{any odd number of $\gam$'s}) &= 0, &
		\tr(\gamm \gamn) &= 4 \gmn
	}
	by (A.27).  This is similar to the expression obtained in P\&S~(10.32).  Now we can evaluate the integral using (A.45) with $n = 2$,
	\eq{
		\int \dddlf \frac{\ell^2}{(\ell^2 - \Del)^2} = \frac{(-1) i}{(4\pi)^{d / 2}} \frac{d}{2} \frac{\Gam(1 - d / 2)}{\Gam(2)} \paren{ \frac{1}{\Del} }^{1 - d / 2}
		= -\frac{i}{(4\pi)^{2 - \eps / 2}} \frac{4 - \eps}{2} \Gam(\eps / 2 - 1) \paren{ \frac{1}{\Del} }^{\eps / 2 - 1}
		\to -\frac{\Del}{4 \pi^2 \eps},
	}
	as well as Eqs.~\refeq{int1} and \refeq{int2}.  Utilizing our work in Eqs.~\refeq{FP} and \refeq{N2}, then, yields
	\aln{
		-g^2 \int \dddkf \tr[ -\frac{(\ksl + \psl - M) (\ksl + M)}{[ (k + p)^2 - M^2 + i \eps ] [ k^2 - M^2 + i \eps ]} ]
		&= -4 g^2 \intoq \ddx \int \dddlf \frac{\ell^2 - x (1 - x) p^2 - M^2}{(\ell^2 - \Del)^2} \notag \\
		&\to -4 i g^2 \intoq \ddx \brac{ \frac{\Del}{4 \pi^2 \eps} - \frac{x (1 - x) p^2 + M^2}{8 \pi^2 \eps} } \notag \\
		&= -\frac{i g^2}{\pi^2 \eps} \intoq \ddx \brac{ M^2 - x (1 - x) p^2 - \frac{1}{2} x (1 - x) p^2 - \frac{M^2}{2} } \notag \\
		&= -\frac{i g^2}{2 \pi^2 \eps} \intoq \ddx \brac{ M^2 - 3 x (1 - x) p^2 } \notag \\
		&= \frac{i g^2}{2 \pi^2 \eps} \paren{ \frac{p^2}{2} - M^2 } \label{term2}
	}
	for the second diagram.

	For the third diagram in Eq.~\refeq{diag2}, we may adapt (10.29):
	\eq{
		-i \frac{\lam}{2} \int \dddkf \frac{i}{k^2 - m^2 + i \eps} = \frac{\lam}{2} \int \dddkf \frac{1}{k^2 - m^2 + i \eps}.
	}
	Let $\ell = k$ and $\Del = m^2$.  Then we can use P\&S~(A.44) with $n = 1$ and substitute $\eps = 4 - d$:
	\eq{
		\int \dddlf \frac{1}{\ell^2 - \Del} = -\frac{i}{(4\pi)^{d / 2}} \frac{\Gam(1 - d / 2)}{\Gam(1)} \paren{ \frac{1}{\Del} }^{1 - d / 2}
		= -\frac{i}{(4\pi)^{2 - \eps / 2}} \Gam(\eps / 2 - 1) \paren{ \frac{1}{\Del} }^{\eps / 2 - 1}.
	}
	This gives us
	\eqn{term3}{
		\frac{\lam}{2} \int \dddlf \frac{1}{\ell^2 - \Del} \to \frac{i \lam}{16 \pi^2 \eps} m^2
	}
	for the third diagram.
	
	Feeding Eqs.~\refeq{term2} and \refeq{term3} into the sum of all three diagrams in Eq.~\refeq{diag2}, we have
	\al{
		-i M^2(p^2) &= i (p^2 \delZw - \delm) - g^2 \int \dddkf \tr[ -\frac{(\ksl + \psl - M) (\ksl + M)}{[ (k + p)^2 - M^2 + i \eps ] [ k^2 - M^2 + i \eps ]} ] + \frac{\lam}{2} \int \dddkf \frac{1}{k^2 - m^2 + i \eps} \\
		&= i (p^2 \delZw - \delm) + \frac{i g^2}{2 \pi^2 \eps} \paren{ \frac{p^2}{2} - M^2 } + \frac{i \lam}{16 \pi^2 \eps} m^2 \\
		&= i \curly{ p^2 \brac{ \delZw + \frac{g^2}{4 \pi^2 \eps} } - \brac{ \delm + \paren{ \frac{g^2 M^2}{2 \pi^2 \eps} - \frac{\lam m^2}{16 \pi^2 \eps} } } },
	}
	which implies
	\aln{ \label{rules3}
		\delZw &= -\frac{g^2}{4 \pi^2 \eps}, &
		\delm &= \frac{\lam m^2}{16 \pi^2 \eps} - \frac{g^2 M^2}{2 \pi^2 \eps}
	}
	is needed.
	
	
	
	For the $4\phi$ vertex, we need to consider all of the one-loop diagrams of $\phi^4$ theory~\cite[p.~326]{Peskin}, as well as a diagram with a fermion loop~\cite[p.~121]{Peskin}:
	\eqn{diag3}{
		\centergraphics{diag/ren_scalar_vertex_black}
		\quad + \quad
		\centergraphics{diag/scalar_s_loop}
		\quad + \quad
		\centergraphics{diag/scalar_t_loop}
		\quad + \quad
		\centergraphics{diag/scalar_u_loop}
		\quad + \quad
		\centergraphics{diag/scalars_fermion_loop}
	}
	The second diagram can be computed in an identical manner to the corresponding scalar diagram in P\&S~(10.20) and (10.23):
	\eqn{diaggg}{
		\frac{(-i \lam)^2}{2} \int \dddkf \frac{i}{k^2 - m^2 + i \eps} \frac{i}{(k + p)^2 - m^2 + i \eps}
		= \frac{\lam^2}{2} \int \dddkf \frac{1}{(k^2 - m^2 + i \eps) [ (k + p)^2 - m^2 + i \eps ]}.
	}
	We introduce the Feynman parameter $x$ as in (10.23):
	\eq{
		\frac{1}{k^2 - m^2 + i \eps} \frac{1}{(k + p)^2 - m^2 + i \eps} = \intoq \ddx \frac{1}{[ k^2 + 2 x k \cdot p + x p^2 - m^2 ]^2}
		\equiv \intoq \frac{1}{D^2}.
	}
	We note that this $D$ is identical to the one in Eq.~\refeq{FP} with $M \to m$, so we may reuse our work from that calculation.  Let $\Del = m^2 - x (1 - x) p^2$, so $D = \ell^2 - \Del + i \eps$.  Once more applying Eqs.~\refeq{int1} and \refeq{int2}, we have
	\eqn{thing4}{
		\frac{\lam^2}{2} \int \dddkf \frac{1}{(k^2 - m^2 + i \eps) [ (k + p)^2 - m^2 + i \eps ]} \to \frac{\lam^2}{2} \frac{i}{8 \pi^2 \eps}
		= \frac{i \lam^2}{16 \pi^2 \eps}
	}
	as $\eps \to 0$.
	
	We note that the third and fourth diagrams in Eq.~\refeq{diag3} also take the form of Eq.~\refeq{diaggg} since $p^2 = s$, where $s$ is a Mandelstam variable.  We can evaluate the third and fourth diagrams by replacing $p^2 = s$ by $t$ and by $u$, respectively~\cite[pp.~156, 326]{Peskin}.  However, we see in Eq.~\refeq{thing4} that the divergent term does not depend on the Mandelstam variable, so we conclude that the sum of these three diagrams is
	\eqn{3diags}{
		\frac{3 i \lam^2 m^2}{16 \pi^2 \eps}.
	}
	
	For the final (fermion loop) diagram, we need to evaluate the trace of the four vertices~\cite[p.~120]{Peskin}.  We label the momenta as on p.~326 of P\&S:
	\eq{
		\centergraphics{diag/scalars_fermion_loop_labels}
	}
	We have
	{\footnotesize \aln{
		-g^4 &\int \dddkf \tr\brac{ \frac{\gamt (\ksl + M)}{k^2 - M^2 + i \eps} \frac{\gamt (\ksl + \pw + M)}{(k + \pw)^2 - M^2 + i \eps} \frac{\gamt (\ksl + \pw + \pr + M)}{(k + \pw + \pr)^2 - M^2 + i \eps} \frac{\gamt (\ksl + \pslw + \psle + \pslr + M)}{(k + \pw + \pe + \pslr)^2 - M^2 + i \eps} } \notag \\
		&= -g^4 ]int \dddkf \tr\brac{ \frac{(\ksl - M) (\ksl + \pslw + M) (\ksl + \pslw + \pslr - M) (\ksl + \pslw + \psle + \pslr + M)}{(k^2 - M^2 + i \eps) [ (k + \pw)^2 - M^2 + i \eps ] [ (k + \pslw + \pslr)^2 - M^2 + i \eps ] [ (k + \pslw + \psle + \pslr)^2 - M^2 + i \eps ]} }. \label{ugh}
	}}%
	Now we introduce the Feynman parameters, using Peskin \& Schroeder~(6.41) with $n = 4$:
	{\footnotesize \al{
		&\frac{1}{(k^2 - M^2 + i \eps) [ (k + \pw)^2 - M^2 + i \eps ] [ (k + \pslw + \pslr)^2 - M^2 + i \eps ] [ (k + \pslw + \psle + \pslr)^2 - M^2 + i \eps ]} \\
		&\hspace{1em} = \intoq \ddw \ddx \ddy \ddz \frac{3! \, \del(1 - w - x - y - z)}{\{ w (k^2 - M^2 + i \eps) + x [ (k + \pw)^2 - M^2 + i \eps ] + y [ (k + \pslw + \pslr)^2 - M^2 + i \eps ] + z [ (k + \pslw + \psle + \pslr)^2 - M^2 + i \eps ] \}^4 } \\
		&\hspace{1em} \equiv 3!\intoq \ddw \ddx \ddy \ddz \frac{\del(1 - w - x - y - z)}{(\ell^2 - \Del + i \eps)^4},
	}}%
	where we have defined $\ell$ and $\Del$, although not explicitly, in hopes that the divergent terms are independent of $\Del$.  We do know that $\ell = k + (\text{other terms})$, so we see that the numerator in Eq.~\refeq{ugh} has terms of $\order{\ell^4}$, $\order{\ell^2}$, and $\order{\ell^0}$.  The terms of odd powers of $\ell$ vanish when we take the trace by (A.27), which tells us that
	\eqn{traces2}{
		\tr(\gamm \gamn \gamr \gams) = 4 (\gmn \grs - \gmr \gns + \gms \gnr)
	}
	along with the identities in Eq.~\refeq{traces}.  From P\&S~(A.44) and (A.45),
	\al{
		\int \dddlf \frac{1}{(\ell^2 - \Del)^4} &= \frac{(-1)^4 i}{(4\pi)^{d / 2}} \frac{\Gam(4 - d / 2)}{\Gam(4)} \paren{ \frac{1}{\Del} }^{4 - d / 2}
		= \frac{i}{(4\pi)^{2 - \eps / 2}} \frac{\Gam(2 + \eps / 2)}{6} \paren{ \frac{1}{\Del} }^{2 + \eps / 2}
		\to 0, \\
		%
		\int \dddlf \frac{\ell^2}{(\ell^2 - \Del)^4} &= \frac{(-1)^3 i}{(4\pi)^2} \frac{d}{2} \frac{\Gam(4 - d / 2 - 1)}{\Gam(4)} \paren{ \frac{1}{\Del} }^{4 - d / 2 - 1}
		= -\frac{i}{(4\pi)^{2 - \eps / 2}} \frac{4 - \eps}{2} \frac{\Gam(1 + \eps / 2)}{6} \paren{ \frac{1}{\Del} }^{1 + \eps / 2}
		\to 0,
	}
	where in the final step we have taken the $\eps \to 0$ limit and kept only terms of $\order{\eps^{-1}}$.  So these terms do not contribute to the divergence.  Finally, from (A.47),
	\al{
		\int \dddlf \frac{(\ell^2)^2}{(\ell^2 - \Del)^4} &= \frac{(-1)^4 i}{(4\pi)^2} \frac{d (d + 2)}{4} \frac{\Gam(4 - d / 2 - 2)}{\Gam(4)} \paren{ \frac{1}{\Del} }^{4 - d / 2 - 2} \\
		&= \frac{i}{(4\pi)^{2 - \eps / 2}} \frac{(4 - \eps) (6 - \eps)}{4} \frac{\Gam(\eps / 2)}{6} \paren{ \frac{1}{\Del} }^{\eps / 2} \\
		&\to \frac{i}{8 \pi^2 \eps}.
	}
	We note that the $\ell^4$ term in the numerator of Eq.~\refeq{ugh} picks up a factor of $4$ when we take the trace by Eq.~\refeq{traces2}.  So for the fourth diagram in Eq.~\refeq{diag3}, we have
	\eqn{blah}{
		-g^4 \int \dddkf \tr\bigg[ \cdots \bigg] = -(3!) (4) \intoq \ddw \ddx \ddy \ddz \del(1 - w - x - y - z) \frac{i g^4}{8 \pi^2 \eps}
		= \frac{i g^4}{2 \pi^2 \eps},
	}
	where we have used Mathematica to evaluate the integral.
	
	Summing Eqs.~\refeq{3diags} and \refeq{blah} with the Feynman rule for the appropriate counterterm, the sum of the diagrams in Eq.~\refeq{diag3} is
	\eq{
		-i M^2(p^2) = \frac{3 i \lam^2 m^2}{16 \pi^2 \eps} - \frac{i g^4}{2 \pi^2 \eps} - i \dellam
		= i \brac{ \paren{ \frac{3 \lam^2 m^2}{16 \pi^2 \eps} - \frac{g^4}{2 \pi^2 \eps} } - \dellam },
	}
	which implies that we need
	\eqn{rules4}{
		\dellam = \frac{3 \lam^2 m^2}{16 \pi^2 \eps} - \frac{g^4}{2 \pi^2 \eps}
	}
	in order to eliminate the divergence.
	
	
	
	Summing up our results in Eqs.~\refeq{rules1}, \refeq{rules2}, \refeq{rules3}, and \refeq{rules4}, the renormalization conditions are
	\ans{\al{
		\delZq &= -\frac{g^2}{16 \pi^2 \eps}, &
		\delM &= -\frac{g^2}{8 \pi^2 \eps} M, &
		\delg &= \frac{g^2}{8 \pi^2} \frac{1}{\eps}, \\
		\delZw &= -\frac{g^2}{4 \pi^2 \eps}, &
		\delm &= \frac{\lam m^2}{16 \pi^2 \eps} - \frac{g^2 M^2}{2 \pi^2 \eps}, &
		\dellam &= \frac{3 \lam^2 m^2}{16 \pi^2 \eps} - \frac{g^4}{2 \pi^2 \eps}.
	}}%
}


\makebib

\end{document}
