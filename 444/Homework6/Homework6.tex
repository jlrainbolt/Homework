\documentclass[11pt]{article}
\usepackage{homework}

\classname{444}
\homeworknum{6}



\begin{document}

% Environments

\newcommand{\state}[2]{\begin{statement}{#1} #2 \end{statement}}
\newcommand{\prob}[2]{\begin{problem}{#1} #2 \end{problem}}
\newcommand{\subprob}[1]{\begin{subproblem} #1 \end{subproblem}}
\newcommand{\sol}[1]{\begin{solution} #1 \end{solution}}
\newcommand{\fig}[2]{\begin{figure} \centering #2  \label{#1} \end{figure}}

\newcommand{\makebib}{
	\vfill
	\color{black}
	\nocite{*}
	\bibliography{references}{}
	\bibliographystyle{lucas_unsrt}
}
	

% Implication

\newcommand{\qwhere}{\quad \text{where} \quad}
\newcommand{\qimplies}{\quad \implies \quad}
\newcommand{\impliesq}{\implies \quad}



% Brackets

\newcommand{\paren}[1]{\left( #1 \right)}
\newcommand{\brac}[1]{\left[ #1 \right]}
\newcommand{\curly}[1]{\left\{ #1 \right\}}


% Greek

\newcommand{\alp}{\alpha}
\newcommand{\bet}{\beta}
\newcommand{\gam}{\gamma}
\newcommand{\del}{\delta}
\newcommand{\eps}{\epsilon}
\newcommand{\zet}{\zeta}
\newcommand{\tht}{\theta}
\newcommand{\kap}{\kappa}
\newcommand{\lam}{\lambda}
\newcommand{\sig}{\sigma}
\newcommand{\ups}{\upsilon}
\newcommand{\omg}{\omega}

\newcommand{\Gam}{\Gamma}
\newcommand{\Del}{\Delta}
\newcommand{\Tht}{\Theta}
\newcommand{\Lam}{\Lambda}
\newcommand{\Sig}{\Sigma}
\newcommand{\Omg}{\Omega}


% Text

\newcommand{\where}{\text{where }}

% Problem 1

\newcommand{\Hint}{H_\text{int}}
\newcommand{\ddcx}{\dd[3]{x}}
\newcommand{\psib}{\bar{\psi}}

\newcommand{\mh}{m_h}
\newcommand{\mmu}{m_\mu}
\newcommand{\me}{m_e}
\newcommand{\ma}{m_a}

\newcommand{\aexpt}{a_\text{expt.}}
\newcommand{\aQED}{a_\text{QED}}
\renewcommand{\GeV}{\giga\electronvolt}

\newcommand{\gamt}{\gam^5}



\state{The $\boldsymbol{\CPN}$ model~(P\&S~13.3)}{
	The nonlinear sigma model discussed in the text can be thought of as a quantum theory of fields that are coordinates on the unit sphere.  A slightly more complicated space of high symmetry is complex projective space, $\CPN$.  This space can be defined as the space of $(N + 1)$-dimensional complex vectors $(\zq, \ldots, \zNq)$ subject to the condition
	\eq{
		\sumj \abs{\zj}^2 = 1,
	}
	with points related by an overall phase rotation identified, that is,
	\eq{
		(e^{i \alp} \zq, \ldots, e^{i \alp} \zNq,) \; \text{identified with} \; (\zq, \ldots, \zNq).
	}
	In this problem, we study that two-dimensional quantum field theory whose fields are coordinates on this space.
}

\prob{
	One way to represent a theory of coordinates on $\CPN$ is to write a Lagrangian depending on fields $\zj(x)$, subject to the constraint, which also has the total symmetry
	\eqn{symmetry}{
		\zj(x) \to e^{i \alp(x)} \zj(x),
	}
	independently at each point $x$.  Show that the following Lagrangian has this symmetry:
	\eqn{lagr}{
		\cL = \frac{1}{g^2} \brac{ \abs{\ptsm \zj}^2 - \abs*{\zjs \ptsm \zj}^2 }.
	}
	To prove the invariance, you will need to use the constraint on the $\zj$, and its consequence
	\eqn{consequence}{
		\zjs \ptsm \zj = -(\ptsm \zjs) \zj.
	}
	Show that the nonlinear sigma model for the case $N = 3$ can be converted to the $\CPN$ model for the case $N = 1$ by the substitution
	\eqn{substitution}{
		\nii = \zs \sigi z,
	}
	where $\sigi$ are the Pauli sigma matrices.
}

\sol{
	The original Lagrangian can be written
		\eq{
		\cL = \frac{1}{g^2} \brac{ (\ptsm \zj) (\ptsm \zjs) - \zjs \zk (\ptsm \zj) (\ptsm \zks) }.
	}
	For the transformation,
	\al{
		\cL &\to \frac{1}{g^2} \brac{ \abs{\ptsm(e^{i \alp} \zj)}^2 - \abs{(e^{i \alp} \zj)^* \ptsm(e^{i \alp} \zj)}^2 } \\[1ex]
		&= \frac{1}{g^2} \brac{ \abs{\zj \ptsm e^{i \alp} + e^{i \alp} \ptsm \zj}^2 - \abs{e^{-i \alp} \zjs (\zj \ptsm e^{i \alp} + e^{i \alp} \ptsm \zj)}^2 } \\[1ex]
		&= \frac{1}{g^2} \brac{ \abs{\zj \ptsm e^{i \alp} + e^{i \alp} \ptsm \zj}^2 - \zjs \zk (\zj \ptsm e^{i \alp} + e^{i \alp} \ptsm \zj) (\zks \ptsm e^{-i \alp} + e^{-i \alp} \ptsm \zks) } \\[1ex]
		&= \frac{1}{g^2} \brac{ \abs{\zj \ptsm e^{i \alp} + e^{i \alp} \ptsm \zj}^2 - (\abs{\zj}^2 \ptsm e^{i \alp} + \zjs e^{i \alp} \ptsm \zj) (\abs{\zk}^2 \ptsm e^{-i \alp} - \zk e^{-i \alp} \ptsm \zks) } \\[1ex]
		%
		&= \frac{1}{g^2} \left[ (\ptsm e^{i \alp}) (\ptsm e^{-i \alp}) + e^{-i \alp} \zj (\ptsm e^{i \alp}) (\ptsm \zjs) + e^{i \alp} \zjs (\ptsm e^{-i \alp}) (\ptsm \zj) + (\ptsm \zj) (\ptsm \zjs) \right. \\
		&\hspace{5em} \phantom{=\ } - (\ptsm e^{i \alp}) (\ptsm e^{-i \alp}) - e^{-i \alp} \zk (\ptsm e^{i \alp}) (\ptsm \zks) - e^{i \alp} \zjs (\ptsm e^{-i \alp}) (\ptsm \zj) - \zjs \zk (\ptsm \zj) (\ptsm \zks) \big] \\[1ex]
		%
		&= \frac{1}{g^2} \brac{ (\ptsm \zj) (\ptsm \zjs) - \zjs \zk (\ptsm \zj) (\ptsm \zks) },
	}
	where we have used $\abs{\zj}^2 = 1$.  So the Lagrangian has the symmetry Eq.~\refeq{symmetry}.
	
	The Lagrangian for the nonlinear sigma model is given by P\&S~(13.67),
	\eqn{nonlinear}{
		\cL = \frac{1}{2 g^2} \abs{ \ptsm \vn }^2,
	}
	where $\vn(x)$ is an $N$-component vector field constrained to satisfy P\&S~(13.66),
	\eq{
		\sumiN \abs{ \nii(x) }^2 = 1.
	}
%	For $N = 3$, the Lagrangian is
%	\eq{
%		\cL = \frac{1}{2 g^2} \brac{ (\ptsm \nq)^2 + (\ptsm \nw)^2 + (\ptsm \nee)^2 }.
%	}
	Making the substitution Eq.~\refeq{substitution} in Eq.~\refeq{nonlinear},
	\al{
		\cL &\to \frac{1}{2 g^2} \abs{ \ptsm (\zs \sigi z) }^2 \\
		&= \frac{1}{2 g^2} \abs{ (\ptsm \zs) \sigi z + \zs \sigi (\ptsm z) }^2 \\
		&= \frac{1}{2 g^2} \Big[ (\ptsm \zs) \sigi z + \zs \sigi (\ptsm z) \Big] \brac{ (\ptsm \zs) \sigi z + \zs \sigi (\ptsm z) } \\
		&= \frac{1}{2 g^2} \brac{ (\ptsm \zs) \sigi z (\ptsm \zs) \sigi z + (\ptsm \zs) \sigi z \zs \sigi (\ptsm z) + \zs \sigi (\ptsm z) (\ptsm \zs) \sigi z + \zs \sigi (\ptsm z) \zs \sigi (\ptsm z ) }
	}
	where we have used ${\sig^\dagger}^i = \sigi$.
}


\clearpage
\prob{
	To write the Lagrangian in a simpler form, introduce a scalar Lagrange multiplier $\lam$ which implements the constraint and also a vector Lagrange multiplier $\Asm$ to express the local symmetry.  More specifically, show that the Lagrangian of the $\CPN$ model is obtained from the Lagrangian
	\eqn{showb}{
		\cL = \frac{1}{g^2} \paren{ \abs{\Dsm \zj}^2 - \lam (\abs{\zj}^2 - 1) },
	}
	where $\Dsm = (\ptsm + i \Asm)$, by functionally integrating over the fields $\lam$ and $\Asm$.
}

\sol{
	The path integral for Eq.~\refeq{showb} is
	\eq{
		\int \DDA \DDlam \DDsz \, \exp[ \frac{i}{g^2} \int \ddsx \paren{ \abs{\Dsm \zj}^2 - \lam (\abs{\zj}^2 - 1) } ].
	}
	Since $\lam$ is a Lagrange multiplier, we can compute this integral easily:
	\eq{
		\int \DDA \DDlam \DDsz \, \exp[ \frac{i}{g^2} \int \ddsx \paren{ \abs{\Dsm \zj}^2 - \lam (\abs{\zj}^2 - 1) } ]
		= \int \DDA \DDsz \, \dels(\abs{\zj}^2 - 1) \exp[ \frac{i}{g^2} \int \ddsx \abs{\Dsm \zj}^2 ],
	}
	where we have applied~\cite{Integrals}
	\eq{
		\int \frac{\dd[d]{k}}{(2\pi)^d} e^{i k (x - y)} = \del^{(d)}(x - y)
	}
	and ignored the overall constant.  Now we have
	\aln{
		\int \DDA \DDlam \DDsz \, &\exp[ \frac{i}{g^2} \int \ddsx \paren{ \abs{\Dsm \zj}^2 - \lam (\abs{\zj}^2 - 1) } ] \notag \\
		&= \int \DDA \DDsz \, \dels(\abs{\zj}^2 - 1) \exp[ \frac{i}{g^2} \int \ddsx (\ptsm + i \Asm) \zj (\ptm - i \Ams) \zjs ] \notag \\
		&= \int \DDA \DDsz \, \dels(\abs{\zj}^2 - 1) \exp[ \frac{i}{g^2} \int \ddsx \curly{ \abs{\ptsm \zj}^2 + 2 i (\ptm \zjs) \zj \Asm + \abs{A}^2 } ], \label{thingb}
	}
	where we have imposed $\abs{\zj}^2 = 1$.  We apply~\cite{Integrals}
	\al{
		\intnii \ddx \exp( -\frac{1}{2} a x^2 + J x ) &= \exp( \frac{J^2}{2 a} ) \intnii \ddx \exp[ -\frac{1}{a} \paren{ x - \frac{J}{a} }^2 ] \\
		&= \exp( \frac{J^2}{2 a} ) \intnii \dd{w} \exp( -\frac{1}{2} a w^2 ) \\
		&= \sqrt{ \frac{2\pi}{a} } \exp( \frac{J^2}{2 a} ).
	}
	In Eq.~\refeq{thingb}, $a = -2 i / g^2$ and $J = 2 (\ptm \zjs) \zj / g^2$.  Then
	\eq{
		\frac{J^2}{2 a} = -\frac{4 \abs*{(\ptm \zjs) \zj}^2}{g^4} \frac{g^2}{2 i}
		= \frac{i}{g^2} \abs{\zj \ptm \zjs}^2
		= \frac{i}{g^2} \abs{\zjs \ptm \zj}^2,
	}
	where we have used Eq.~\refeq{consequence}.  Finally, Eq.~\refeq{thingb} gives us
	\ans{\al{
		\int \DDA \DDlam \DDsz \, &\exp[ \frac{i}{g^2} \int \ddsx \paren{ \abs{\Dsm \zj}^2 - \lam (\abs{\zj}^2 - 1) } ] \\
		&= \DDsz \, \dels(\abs{\zj}^2 - 1) \exp[ \frac{i}{g^2} \int \ddsx \curly{ \abs{\ptsm \zj}^2 - \abs{\zjs \ptm \zj}^2 } ],
	}}%
	where we have again ignored the overall constant.  Thus we have shown that the Lagrangian Eq.~\refeq{lagr} can be obtained from the Lagrangian Eq.~\refeq{showb} by functionally integrating over the Lagrange multipliers. \qed
}


\clearpage
\prob{
	We can solve the $\CPN$ model in the limit $N \to \infty$ by integrating over the fields $\zj$.  Show that this integral leads to the expression
	\eq{
		Z = \int \DDA \DDlam \exp( -N \tr \ln(-D^2 - \lam) + \frac{i}{g^2} \int \ddsx \lam ),
	}
	where we have kept only the leading terms for $N \to \infty$, $g^2 N$ fixed.  Using methods similar to those we used for the nonlinear sigma model, examine the conditions for minimizing the exponent with respect to $\lam$ and $\Asm$.  Show that these conditions have a solution at $\Asm = 0$ and $\lam = m^2 > 0$.  Show that, if $g^2$ is renormalized at the scale $M$, $m$ can be written as
	\eq{
		m = M \exp( -\frac{2\pi}{g^2 N} ).
	}
}


% 13.115 


\prob{
	Now expand the exponent about $\Asm = 0$.  Show that the first nontrivial term in this expansion is proportional to the vacuum polarization of massive scalar fields.  Evaluate this expression using dimensional regularization, and show that it yields a standard kinetic energy term for $\Asm$.  Thus the strange nonlinear field theory that we started with is finally transformed into a theory of $N + 1$ massive scalar fields interacting with a massless photon.
}


\makebib

\end{document}
