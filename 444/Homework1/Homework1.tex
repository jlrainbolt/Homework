\documentclass[11pt]{article}
\usepackage{homework}

\classname{444}
\homeworknum{1}



\begin{document}

% Environments

\newcommand{\state}[2]{\begin{statement}{#1} #2 \end{statement}}
\newcommand{\prob}[2]{\begin{problem}{#1} #2 \end{problem}}
\newcommand{\subprob}[1]{\begin{subproblem} #1 \end{subproblem}}
\newcommand{\sol}[1]{\begin{solution} #1 \end{solution}}
\newcommand{\fig}[2]{\begin{figure} \centering #2  \label{#1} \end{figure}}

\newcommand{\makebib}{
	\vfill
	\color{black}
	\bibliography{references}{}
	\bibliographystyle{lucas_unsrt}
}
	

% Implication

\newcommand{\qwhere}{\quad \text{where} \quad}
\newcommand{\qimplies}{\quad \implies \quad}
\newcommand{\impliesq}{\implies \quad}



% Brackets

\newcommand{\paren}[1]{\left( #1 \right)}
\newcommand{\brac}[1]{\left[ #1 \right]}


% Greek

\newcommand{\alp}{\alpha}
\newcommand{\bet}{\beta}
\newcommand{\gam}{\gamma}
\newcommand{\del}{\delta}
\newcommand{\eps}{\epsilon}
\newcommand{\zet}{\zeta}
\newcommand{\tht}{\theta}
\newcommand{\kap}{\kappa}
\newcommand{\lam}{\lambda}
\newcommand{\sig}{\sigma}
\newcommand{\ups}{\upsilon}
\newcommand{\omg}{\omega}

\newcommand{\Gam}{\Gamma}
\newcommand{\Del}{\Delta}
\newcommand{\Tht}{\Theta}
\newcommand{\Lam}{\Lambda}
\newcommand{\Sig}{\Sigma}
\newcommand{\Omg}{\Omega}
% Problem 1

\newcommand{\Psii}{\Psi^i}
\newcommand{\Psiix}{\Psii(x)}

\newcommand{\Pii}{\Pi^i}

\newcommand{\Phii}{\Phi^i}
\newcommand{\Phiix}{\Phii(x)}
\newcommand{\PhiN}{\Phi^N}
\newcommand{\PhiNx}{\PhiN(x)}
\newcommand{\Phiq}{\Phi^1}
\newcommand{\Phiw}{\Phi^2}

\newcommand{\ddcx}{\dd[3]{x}}

\newcommand{\delij}{\del^{i j}}
\newcommand{\delkl}{\del^{k l}}
\newcommand{\delil}{\del^{i l}}
\newcommand{\deljk}{\del^{j k}}
\newcommand{\delik}{\del^{i k}}
\newcommand{\deljl}{\del^{j l}}

\newcommand{\DF}{D_F}

\newcommand{\sigx}{\sig(x)}

\newcommand{\pii}{\pi^i}
\newcommand{\pij}{\pi^j}
\newcommand{\pik}{\pi^k}
\newcommand{\pil}{\pi^l}
\newcommand{\piix}{\pi(x)}

\newcommand{\pq}{p_1}
\newcommand{\pw}{p_2}
\newcommand{\pe}{p_3}
\newcommand{\pr}{p_4}

\newcommand{\vp}{\vb{p}}
\newcommand{\vpsi}{\vp_i}

\newcommand{\mpi}{m_\pi}



\state{Alternative regulators in QED (Peskin \& Schroeder 7.2)}{
	In Section.~7.5, we saw that the Ward identity can be violated by an improperly chosen regulator.  Let us check the validity of the identity $\Zq = \Zw$, to order $\alp$, for several choices of the regulator.  We have already verified that the relation holds for Paul-Villars regularization.
}

\prob{
	Recompute $\del\Zq$ and $\del\Zw$, defining the integrals~(6.49) and (6.50) by simply placing an upper limit $\Lam$ on the integration over $\ellE$.  Show that, with this definition, $\del\Zq \neq \del\Zw$.
}

\sol{
	From (7.47) in Peskin \& Schroeder,
	\eq{
		\Gammqo = \frac{1}{\Zq} \gamm,
	}
	we can find an expression for $\del\Zq$, similar to how (7.31) is obtained from (7.26).  Roughly,
	\eqn{thingy}{
		\frac{1}{\Zq + \del\Zq} \gamm \approx \Zq (1 - \del\Zq) \gamm
		= \Gammqo + \del\Gammqo
		\qimplies
		\del\Gammqo = -\del\Zq \gamm.
	}
	According to (6.33),
	\eq{
		\Gammppp = \gamm \Fq(q^2) + \frac{i \sigmn \qsn}{2 m} \Fw(q^2).
	}
	We note that $\Gamm = \gamm$, $\Fq = 1$, and $\Fw = 0$ to lowest order~\cite[pp.~185--186]{Peskin}.  Then we can write
	\eqn{Gam}{
		\del\Gammqo = \gamm \del\Fqo + \frac{i \sigmn \qsn}{2 m} \del\Fwo.
	}
	Using this equation and the identity $\gamm \gamsm = 4$~\cite{Gamma}, Eq.~\refeq{thingy} can be written
	\eqn{Z1}{
		\del \Zq = -\frac{1}{4} \gamsm \del\Gammqo
		= -\del\Fqo - \gamsm \frac{i \sigmn \qsn}{8 m} \del\Fwo.
	}
	
	In order to find $\del\Gamm$ we use (6.47):
	\aln{
		\ubpp \del\Gammppp \up &= 2 i e^2 \int \ddqlf \intoq \ddx \ddy \ddz \del(x + y + z - 1) \frac{2}{D^3} \notag \\
		&\hspace{5em} \phantom{=\ } \times \ubpp \left\{ \gamm \brac{ -\frac{\ell^2}{2} + (1 - x) (1 - y) q^2 + (1 - 4z + z^2) m^2 } \right. \label{thing1} \\
		&\hspace{15em} \phantom{=\ } \left. + \frac{i \sigmn \qsn}{2 m} [ 2 m^2 z (1 - z) ] \right\} \up, \notag
	}
	where $\Del \equiv -x y q^2 + (1 - z)^2 m^2$ by (6.44), $\ell \equiv k + y q - z p$, and $D = \ell^2 - \Del + i \eps$~\cite[p.~191]{Peskin}.  The momenta $k$ and $p$ are assigned in the Feynman diagram on p.~189 of Peskin \& Schroeder, and $x, y$ are Feynman parameters~\cite[p.~190]{Peskin}.

	The forms of the two integrals we need to compute are given by Peskin \& Schroeder~(6.49) and (6.50).  Equation~(6.49) is
	\eqn{6.49}{
		\int \ddqlf \frac{1}{(\ell^2 - \Delta)^m} = \frac{i (-1)^m}{(4\pi)^2} \frac{1}{(m - 1) (m - 2)} \frac{1}{\Delta^{m - 2}}.
	}
	Here $m = 3$ because we have $D^{-3}$ in Eq.~\refeq{thing1}.  We can evaluate the left-hand side using the Euclidian 4-momentum defined in (6.48),
	\al{
		\ello &\equiv \ellEo, &
		\vell &= \vellE.
	}
	Following the steps on p.~193, we have
	\eq{
		\int \ddqlf \frac{1}{(\ell^2 - \Delta)^3} = \frac{i}{(-1)^3} \frac{1}{(2\pi)^4} \int \frac{\ddqlE}{(\ellE^2 + \Del)^3}
		= \frac{i (-1)^3}{(2\pi)^4} \int \ddOmgq \intoLam \ddlE \frac{\ellE^3}{(\ellE^2 + \Delta)^3},
	}
	where we have replaced the upper (infinite) bound of integration by a finite number $\Lam$.  Evaluating this integral using Mathematica and using $\int \ddOmgq = 2 \pi^2$~\cite[p.~193]{Peskin}, we find
	\aln{
		\int \ddqlf \frac{1}{(\ell^2 - \Del)^3} &= \frac{i (-1)^3}{(2\pi)^4} (2 \pi^2) \frac{\Lam^4}{4 \Del (\Del + \Lam^2)^2} \notag \\
		&= -\frac{i}{32 \pi^2} \frac{\Lam^4}{\Del (\Del + \Lam^2)^2} \notag \\
		&\approx -\frac{i}{32 \pi^2} \frac{1}{\Del}
		\equiv \alp, \label{ans1}
	}
	where we have taken the limit $\Lam \gg \Del$~\cite[p.~218]{Peskin} and defined $\alp$.  Equation~(6.50) in Peskin \& Schroeder is
	\eq{
		\int \ddqlf \frac{\ell^2}{(\ell^2 - \Del)^m} = \frac{i (-1)^{m - 1}}{(4\pi)^2} \frac{2}{(m - 1) (m - 2) (m - 3)} \frac{1}{\Del^{m - 3}}.
	}
	Following similar steps as for Eq.~\refeq{thing1}, the left-hand side is
	\aln{
		\int \ddqlf \frac{\ell^2}{(\ell^2 - \Del)^3} &= \frac{i}{(-1)^3} \frac{1}{(2\pi)^4} \int \ddqlE \frac{\ellE^2}{(\ellE^2 + \Del)^3} \notag \\
		&= \frac{i (-1)^3}{(2\pi)^4} \int \ddOmgq \intoLam \ddlE \frac{\ellE^5}{(\ellE^2 + \Del)^3} \notag \\
		&= \frac{i (-1)^3}{(2\pi)^4} (2 \pi^2) \frac{1}{4} \brac{ \frac{\Del (3 \Del + 4 \Lam^2)}{(\Del + \Lam^2)^2} + 2 \ln(\frac{\Del + \Lam^2}{\Del}) - 3 } \notag \\
		&= -\frac{i}{32 \pi^2} \brac{ \frac{\Del (3 \Del + 4 \Lam^2)}{(\Del + \Lam^2)^2} + 2 \ln(\frac{\Del +\Lam^2}{\Del}) - 3 } \notag \\
		&\approx -\frac{i}{16 \pi^2} \ln(\frac{\Lam^2}{\Del})
		\equiv \bet, \label{ans2}
	}
	where we have defined $\bet$ and ignored terms of $\order{\Lam^{-2}}$~\cite[p.~218]{Peskin}.  We also ignore constant terms since they do not diverge~\cite[p.~196]{Peskin}.
	
	We now set $q^2 = 0$, and define $\Delo = (1 - z)^2 m^2$.  Then $\Del \to \Delo$ in our expression and $\alp \to \alpo, \bet \to \beto$ (which are functions of $\Delo$).  Feeding in Eqs.~\refeq{ans1} and \refeq{ans2}, Eq.~\refeq{thing1} can be written
	\eq{
		\ubpp \del\Gammqo \up = 2 i e^2 \intoq \ddx \ddy \ddz \del(x + y + z - 1) \ubpp \int \curly{ \gamm \brac{ -\beto + 2 (1 - 4z + z^2) m^2 \alpo } } \up.
	}
	Then
	\al{
		\del\Fqo &= 2 i e^2 \intoq \ddx \ddy \ddz \del(x + y + z - 1) \brac{ -\beto + 2 (1 - 4z + z^2) m^2 \alpo } \\
		&= 2 i e^2 \intoq \ddz (1 - z) \brac{ -\beto + 2 m^2 (1 - 4z + z^2) \alpo }, \\[2ex]
		%
		\del\Fwo &= 8 i e^2 \intoq \ddx \ddy \ddz \del(x + y + z - 1) m^2 z (1 - z) \alpo \\
		&= 8 i e^2 \intoq \ddz m^2 z (1 - z)^2 \alpo.
	}
	We ignore $\del\Fwo$ since it is not affected by the divergence~\cite[p.~196]{Peskin}.  In order to avoid issues coming from the divergence in $\del\Fqo$, we add a $z \mu^2$ term to $\Delo$~\cite[p.~195]{Peskin}.  So, feeding these results into Eq.~\refeq{Z1}, we obtain
	\eqn{Z1ans}{
		\del\Zq = -2 i e^2 \intoq \ddz (1 - z) \brac{ -\beto + 2 (1 - 4z + z^2) m^2 \alpo },
	}
	where 
	\aln{ \label{Z1stuff}
		\alpo &= -\frac{i}{32 \pi^2} \frac{1}{\Delo}, &
		\beto &=  -\frac{i}{16 \pi^2} \ln(\frac{\Lam^2}{\Delo}), &
		\Delo &= (1 - z)^2 m^2 + z \mu^2.
	}
	
	For $\del\Zw$, we can use the first part of Peskin \& Schroeder (7.31),
	\eqn{Z2}{
		\del\Zw = \left. \dv{\Sigw}{\psl} \right|_{\psl = m},
	}
	where $\Sigw$ is given by (7.17),
	\eqn{Sigw}{
		-i \Sigwp = -e^2 \intoq \ddx \int \ddqlf \frac{-2 x \psl + 4 \mo}{(\ell^2 - \Del + i \eps)^2},
	}
	where $\Del = -x (1 - x) p^2 + x \mu^2 + (1 - x) \mo^2$.  We may once again follow the steps on p.~193 to evaluate the integral, now with $m = 2$.  Changing the upper bound of integration to $\Lam$ once more, we have
	\al{
		\int \ddqlf \frac{1}{(\ell^2 - \Delta)^2} &= \frac{i}{(-1)^2} \frac{1}{(2\pi)^4} \int \ddqlE \frac{1}{(\ellE^2 + \Del)^2}\\
		&= \frac{i (-1)^2}{(2\pi)^4} \int \ddOmgq \intoLam \ddlE \frac{\ellE^3}{(\ellE^2 + \Del)^2} \\
		&= \frac{i}{(2\pi)^4} (2 \pi^2) \frac{1}{2} \brac{ \frac{\Del}{\Del + \Lam^2} + \ln(\frac{\Del + \Lam^2}{\Del}) - 1 } \\
		&= \frac{i}{16 \pi^2} \brac{ \frac{\Del}{\Del + \Lam^2} + \ln(\frac{\Del + \Lam^2}{\Del}) - 1 } \\
		&\approx \frac{i}{16 \pi^2} \ln(\frac{\Lam^2}{\Del}),
	}
	where we have evaluated the integral using Mathematica, taken the large $\Lam$ limit, and dropped the irrelevant constant.  Substituting back into Eq.~\refeq{Sigw}, we find
	\eq{
		\Sigwp = -i e^2 \intoq \ddx (-2 x \psl + 4 \mo) \frac{i}{16 \pi^2} \ln(\frac{\Lam^2}{\Del}).
	}
	Note that
	\aln{
		\dv{\Sigw}{\psl} &= \frac{e^2}{16 \pi^2} \dv{\psl} \brac{ \intoq \ddx (-2 x \psl + 4 \mo) \ln(\frac{\Lam^2}{\Del}) } \notag \\
		&= \frac{e^2}{16 \pi^2} \intoq \ddx \brac{ \ln(\frac{\Lam^2}{\Del}) \dv{\psl}(-2 x \psl + 4 \mo) + (-2 x \psl + 4 \mo) \dv{\psl} \ln(\frac{\Lam^2}{\Del}) } \notag \\
		&= \frac{e^2}{16 \pi^2} \intoq \ddx \brac{ \ln(\frac{\Lam^2}{\Del}) \dv{\psl}(-2 x \psl + 4 \mo) + (-2 x \psl + 4 \mo) \dv{\Del} \ln(\frac{\Lam^2}{\Del}) \dv{\Del}{\psl} }. \label{der}
	}
	Using $p^2 = \psl^2$~\cite[p.~220]{Peskin}, note that
	\eq{
		\dv{\Del}{\psl} = \dv{\psl}[ -x (1 - x) \psl^2 + x \mu^2 + (1 - x) \mo^2 ]
		= -2x (1 - x) \psl.
	}
	Also,
	\al{
		\dv{\psl}(-2 x \psl + 4 \mo) &= -2x, &
		\dv{\Del}\brac{ \ln(\frac{\Lam^2}{\Del}) } &= \dv{\Del}\brac{ \ln(\Lam^2) - \ln(\Del) }
		= -\frac{1}{\Del}.
	}
	Making these substitutions in Eq.~\refeq{der},
	\eq{
		\dv{\Sigw}{\psl} = \frac{e^2}{16 \pi^2} \intoq \ddx \brac{ -2x \ln(\frac{\Lam^2}{\Del}) - \frac{(2 x \psl - 4 \mo) [ 2x (1 - x) \psl ]}{\Del} }.
	}
	We now define
	\eqn{Z2stuff}{
		\Delm \equiv -x (1 - x) m^2 + x \mu^2 + (1 - x) \mo^2
		\approx (1 - x)^2 m^2 + x \mu^2,
	}
	since $m \approx \mo$.  Then Eq.~\refeq{Z2} becomes
	\eqn{Z2ans}{
		\del\Zw = \frac{e^2}{16 \pi^2} \intoq \ddx \brac{ -2x \ln(\frac{\Lam^2}{\Del}) - \frac{(2 x m + 4 \mo) [ 2x (1 - x) m ]}{\Delm} }.
	}
	
	Now we write out $\del\Zq$ and $\del\Zw$ fully, feeding Eqs.~\refeq{Z1stuff} and \refeq{Z2stuff} into Eqs.~\refeq{Z1ans} and \refeq{Z2ans}, respectively.  We also rename $x \to z$ in $\del\Zw$:
	\al{
		\del\Zq &= -2 i e^2 \intoq \ddz (1 - z) \brac{ -\frac{i}{16 \pi^2} \ln(\frac{\Lam^2}{\Del}) + 2 (1 - 4z + z^2) m^2 \paren{ -\frac{i}{32 \pi^2} \frac{1}{\Delo} } } \\
		&= \frac{e^2}{8 \pi^2} \intoq \ddz (1 - z) \brac{ \ln(\frac{\Lam^2}{(1 - z)^2 m^2 + z \mu^2}) - \frac{m^2 (1 - 4z + z^2)}{(1 - z)^2 m^2 + z \mu^2} }, \\[2ex]
		%
		\del\Zw &= -\frac{e^2}{8 \pi^2} \intoq \ddz \brac{ z \ln(\frac{\Lam^2}{(1 - z)^2 m^2 + z \mu^2}) + \frac{2 z m^2 (1 - z) (2 + z)}{(1 - z)^2 m^2 + z \mu^2} }.
	}
	Clearly $\del\Zq \neq \del\Zw$, as we wanted to show. \qed
}



%\prob{
%	Recompute $\del\Zq$ and $\del\Zw$, defining the integrals~(6.49) and (6.50) by dimensional regularization.  You may take the Dirac matrices to be $4 \times 4$ as usual, but note that, in $d$ dimensions,
%	\eq{
%		\gsmn \gamm \gamn = d.
%	}
%	Show that, with this definition, $\del\Zq = \del\Zw$.
%}
%
%
%
%
%
%
%\state{(Peskin \& Schroeder 7.3)}{
%	Consider a theory of elementary fermions that couple both to QED and to a Yukawa field $\phi$:
%	\eq{
%		\Hint = \int \ddcx \frac{\lam}{\sqrt{2}} \phi \psib \psi + \int \ddcx e \Asm \psib \gamm \psi.
%	}
%}
%
%\prob{
%	Verify that the contribution to $\Zq$ from the vertex diagram with a virtual $\phi$ equals the contribution to $\Zq$ from the diagram with a virtual $\phi$.  Use dimensional regularization.  Is the Ward identity generally true in this theory?
%}
%
%
%
%\prob{
%	Now consider the renormalization of the $\phi \psib \psi$ vertex.  Show that the rescaling of this vertex at $q^2 = 0$ is \emph{not} canceled by the correction to $\Zw$.  (It suffices to compute the ultraviolet-divergent parts of the diagrams.)  In this theory, the vertex and field-strength rescaling give additional shifts of the observable coupling constant relative to its bare value.
%}


\makebib

\end{document}
